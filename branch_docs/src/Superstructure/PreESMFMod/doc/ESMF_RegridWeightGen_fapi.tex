%                **** IMPORTANT NOTICE *****
% This LaTeX file has been automatically produced by ProTeX v. 1.1
% Any changes made to this file will likely be lost next time
% this file is regenerated from its source. Send questions 
% to Arlindo da Silva, dasilva@gsfc.nasa.gov
 
\setlength{\oldparskip}{\parskip}
\setlength{\parskip}{1.5ex}
\setlength{\oldparindent}{\parindent}
\setlength{\parindent}{0pt}
\setlength{\oldbaselineskip}{\baselineskip}
\setlength{\baselineskip}{11pt}
 
%--------------------- SHORT-HAND MACROS ----------------------
\def\bv{\begin{verbatim}}
\def\ev{\end{verbatim}}
\def\be{\begin{equation}}
\def\ee{\end{equation}}
\def\bea{\begin{eqnarray}}
\def\eea{\end{eqnarray}}
\def\bi{\begin{itemize}}
\def\ei{\end{itemize}}
\def\bn{\begin{enumerate}}
\def\en{\end{enumerate}}
\def\bd{\begin{description}}
\def\ed{\end{description}}
\def\({\left (}
\def\){\right )}
\def\[{\left [}
\def\]{\right ]}
\def\<{\left  \langle}
\def\>{\right \rangle}
\def\cI{{\cal I}}
\def\diag{\mathop{\rm diag}}
\def\tr{\mathop{\rm tr}}
%-------------------------------------------------------------

\markboth{Left}{Source File: ESMF\_RegridWeightGen.F90,  Date: Tue May  5 21:00:17 MDT 2020
}

 
%/////////////////////////////////////////////////////////////
\subsubsection [ESMF\_RegridWeightGen] {ESMF\_RegridWeightGen - Generate regrid weight file from grid files}


   \label{api:esmf_regridweightgenfile}
\bigskip{\sf INTERFACE:}
\begin{verbatim}   ! Private name; call using ESMF_RegridWeightGen()
   subroutine ESMF_RegridWeightGenFile(srcFile, dstFile, &
     weightFile, rhFile, regridmethod, polemethod, regridPoleNPnts, lineType, normType, &
     extrapMethod, extrapNumSrcPnts, extrapDistExponent, extrapNumLevels, &
     unmappedaction, ignoreDegenerate, srcFileType, dstFileType, &
     srcRegionalFlag, dstRegionalFlag, srcMeshname, dstMeshname,  &
     srcMissingvalueFlag, srcMissingvalueVar, &
     dstMissingvalueFlag, dstMissingvalueVar, &
     useSrcCoordFlag, srcCoordinateVars, &
     useDstCoordFlag, dstCoordinateVars, &
     useSrcCornerFlag, useDstCornerFlag, &
     useUserAreaFlag, largefileFlag, &
     netcdf4fileFlag, weightOnlyFlag, &
     tileFilePath, &
     verboseFlag, rc)
 \end{verbatim}{\em ARGUMENTS:}
\begin{verbatim} 
   character(len=*),             intent(in)            :: srcFile
   character(len=*),             intent(in)            :: dstFile
 -- The following arguments require argument keyword syntax (e.g. rc=rc). --
   character(len=*),             intent(in),  optional :: weightFile
   character(len=*),             intent(in),  optional :: rhFile
   type(ESMF_RegridMethod_Flag), intent(in),  optional :: regridmethod
   type(ESMF_PoleMethod_Flag),   intent(in),  optional :: polemethod
   integer,                      intent(in),  optional :: regridPoleNPnts
   type(ESMF_LineType_Flag),     intent(in),  optional :: lineType
   type(ESMF_NormType_Flag),     intent(in),  optional :: normType
   type(ESMF_ExtrapMethod_Flag),   intent(in),    optional :: extrapMethod
   integer,                        intent(in),    optional :: extrapNumSrcPnts
   real,                           intent(in),    optional :: extrapDistExponent
   integer,                      intent(in), optional :: extrapNumLevels
   type(ESMF_UnmappedAction_Flag),intent(in), optional :: unmappedaction
   logical,                      intent(in),  optional :: ignoreDegenerate
   type(ESMF_FileFormat_Flag),   intent(in),  optional :: srcFileType
   type(ESMF_FileFormat_Flag),   intent(in),  optional :: dstFileType
   logical,                      intent(in),  optional :: srcRegionalFlag
   logical,                      intent(in),  optional :: dstRegionalFlag
   character(len=*),             intent(in),  optional :: srcMeshname
   character(len=*),             intent(in),  optional :: dstMeshname
   logical,                      intent(in),  optional :: srcMissingValueFlag
   character(len=*),             intent(in),  optional :: srcMissingValueVar
   logical,                      intent(in),  optional :: dstMissingValueFlag
   character(len=*),             intent(in),  optional :: dstMissingValueVar
   logical,                      intent(in),  optional :: useSrcCoordFlag
   character(len=*),             intent(in),  optional :: srcCoordinateVars(:)
   logical,                      intent(in),  optional :: useDstCoordFlag
   character(len=*),             intent(in),  optional :: dstCoordinateVars(:)
   logical,                      intent(in),  optional :: useSrcCornerFlag
   logical,                      intent(in),  optional :: useDstCornerFlag
   logical,                      intent(in),  optional :: useUserAreaFlag
   logical,                      intent(in),  optional :: largefileFlag
   logical,                      intent(in),  optional :: netcdf4fileFlag
   logical,                      intent(in),  optional :: weightOnlyFlag
   logical,                      intent(in),  optional :: verboseFlag
   character(len=*),             intent(in),  optional :: tileFilePath
   integer,                      intent(out), optional :: rc
 \end{verbatim}
{\sf DESCRIPTION:\\ }


   This subroutine provides the same function as the {\tt ESMF\_RegridWeightGen} application
   described in Section~\ref{sec:ESMF_RegridWeightGen}.  It takes two grid files in NetCDF format and writes out an
   interpolation weight file also in NetCDF format.  The interpolation weights can be generated with the
   bilinear~(\ref{sec:interpolation:bilinear}), higher-order patch~(\ref{sec:interpolation:patch}),
   or first order conservative~(\ref{sec:interpolation:conserve}) methods.  The grid files can be in
   one of the following four formats:
   \begin{itemize}
   \item The SCRIP format~(\ref{sec:fileformat:scrip})
   \item The native ESMF format for an unstructured grid~(\ref{sec:fileformat:esmf})
   \item The CF Convention Single Tile File format~(\ref{sec:fileformat:gridspec})
   \item The proposed CF Unstructured grid (UGRID) format~(\ref{sec:fileformat:ugrid})
   \item The GRIDSPEC Mosaic File format~(\ref{sec:fileformat:mosaic})
   \end{itemize}
   \smallskip
   The weight file is created in SCRIP format~(\ref{sec:weightfileformat}).
   The optional arguments allow users to specify various options to control the regrid operation,
   such as which pole option to use,
   whether to use user-specified area in the conservative regridding, or whether ESMF should generate masks using a given
   variable's missing value.  There are also optional arguments specific to a certain type of the grid file.
   All the optional arguments are similar to the command line arguments for the {\tt ESMF\_RegridWeightGen}
   application~(\ref{sec:regridusage}). The acceptable values and the default value for the optional arguments
   are listed below.
  
   The arguments are:
     \begin{description}
     \item [srcFile]
       The source grid file name.
     \item [dstFile]
       The destination grid file name.
     \item [weightFile]
       The interpolation weight file name.
     \item [{[rhFile]}]
       The RouteHandle file name.
     \item [{[regridmethod]}]
       The type of interpolation. Please see Section~\ref{opt:regridmethod}
       for a list of valid options. If not specified, defaults to
       {\tt ESMF\_REGRIDMETHOD\_BILINEAR}.
     \item [{[polemethod]}]
       A flag to indicate which type of artificial pole
       to construct on the source Grid for regridding. Please see
       Section~\ref{const:polemethod} for a list of valid options.
       The default value varies depending on the regridding method and the grid type and format.
     \item [{[regridPoleNPnts]}]
       If {\tt polemethod} is set to {\tt ESMF\_POLEMETHOD\_NPNTAVG}, this argument is required to
       specify how many points should be averaged over at the pole.
     \item [{[lineType]}]
             This argument controls the path of the line which connects two points on a sphere surface. This in
             turn controls the path along which distances are calculated and the shape of the edges that make
             up a cell. Both of these quantities can influence how interpolation weights are calculated.
             As would be expected, this argument is only applicable when {\tt srcField} and {\tt dstField} are
             built on grids which lie on the surface of a sphere. Section~\ref{opt:lineType} shows a
             list of valid options for this argument. If not specified, the default depends on the
             regrid method. Section~\ref{opt:lineType} has the defaults by line type. Figure~\ref{line_type_support} shows
             which line types are supported for each regrid method as well as showing the default line type by regrid method.
       \item [{[normType]}]
             This argument controls the type of normalization used when generating conservative weights. This option
             only applies to weights generated with {\tt regridmethod=ESMF\_REGRIDMETHOD\_CONSERVE}. Please see
             Section~\ref{opt:normType} for a
             list of valid options. If not specified {\tt normType} defaults to {\tt ESMF\_NORMTYPE\_DSTAREA}.
       \item [{[extrapMethod]}]
             The type of extrapolation. Please see Section~\ref{opt:extrapmethod}
             for a list of valid options. If not specified, defaults to
             {\tt ESMF\_EXTRAPMETHOD\_NONE}.
       \item [{[extrapNumSrcPnts]}]
             The number of source points to use for the extrapolation methods that use more than one source point
             (e.g. {\tt ESMF\_EXTRAPMETHOD\_NEAREST\_IDAVG}). If not specified, defaults to 8.
       \item [{[extrapDistExponent]}]
             The exponent to raise the distance to when calculating weights for
             the {\tt ESMF\_EXTRAPMETHOD\_NEAREST\_IDAVG} extrapolation method. A higher value reduces the influence
             of more distant points. If not specified, defaults to 2.0.
       \item [{[unmappedaction]}]
             Specifies what should happen if there are destination points that
             can't be mapped to a source cell. Please see Section~\ref{const:unmappedaction} for a
             list of valid options. If not specified, {\tt unmappedaction} defaults to {\tt ESMF\_UNMAPPEDACTION\_ERROR}.
       \item [{[ignoreDegenerate]}]
             Ignore degenerate cells when checking the input Grids or Meshes for errors. If this is set to true, then the
             regridding proceeds, but degenerate cells will be skipped. If set to false, a degenerate cell produces an error.
             If not specified, {\tt ignoreDegenerate} defaults to false.
     \item [{[srcFileType]}]
       The file format of the source grid. Please see
       Section~\ref{const:fileformatflag} for a list of valid options. 
        If not specifed, the program will determine the file format automatically.
     \item [{[dstFileType]}]
       The file format of the destination grid.  Please see Section~\ref{const:fileformatflag} for a list of valid options.
        If not specifed, the program will determine the file format automatically.
     \item [{[srcRegionalFlag]}]
       If .TRUE., the source grid is a regional grid, otherwise,
       it is a global grid.  The default value is .FALSE.
     \item [{[dstRegionalFlag]}]
       If .TRUE., the destination grid is a regional grid, otherwise,
       it is a global grid.  The default value is .FALSE.
     \item [{[srcMeshname]}]
       If the source file is in UGRID format, this argument is required
       to define the dummy variable name in the grid file that contains the
       mesh topology info.
     \item [{[dstMeshname]}]
       If the destination file is in UGRID format, this argument is required
       to define the dummy variable name in the grid file that contains the
       mesh topology info.
     \item [{[srcMissingValueFlag]}]
       If .TRUE., the source grid mask will be constructed using the missing
       values of the variable defined in {\tt srcMissingValueVar}. This flag is
       only used for the grid defined in  the GRIDSPEC or the UGRID file formats.
       The default value is .FALSE..
     \item [{[srcMissingValueVar]}]
       If {\tt srcMissingValueFlag} is .TRUE., the argument is required to define
       the variable name whose missing values will be used to construct the grid
       mask.  It is only used for the grid defined in  the GRIDSPEC or the UGRID
       file formats.
     \item [{[dstMissingValueFlag]}]
       If .TRUE., the destination grid mask will be constructed using the missing
       values of the variable defined in {\tt dstMissingValueVar}. This flag is
       only used for the grid defined in  the GRIDSPEC or the UGRID file formats.
       The default value is .FALSE..
     \item [{[dstMissingValueVar]}]
       If {\tt dstMissingValueFlag} is .TRUE., the argument is required to define
       the variable name whose missing values will be used to construct the grid
       mask.  It is only used for the grid defined in  the GRIDSPEC or the UGRID
       file formats.
     \item [{[useSrcCoordFlag]}]
       If .TRUE., the coordinate variables defined in {\tt srcCoordinateVars} will
       be used as the longitude and latitude variables for the source grid.
       This flag is only used for the GRIDSPEC file format.  The default is .FALSE.
     \item [{[srcCoordinateVars]}]
       If {\tt useSrcCoordFlag} is .TRUE., this argument defines the longitude and
  !     latitude variables in the source grid file to be used for the regrid.
       This argument is only used when the grid file is in GRIDSPEC format.
       {\tt srcCoordinateVars} should be a array of 2 elements.
     \item [{[useDstCoordFlag]}]
       If .TRUE., the coordinate variables defined in {\tt dstCoordinateVars} will
       be used as the longitude and latitude variables for the destination grid.
       This flag is only used for the GRIDSPEC file format.  The default is .FALSE.
     \item [{[dstCoordinateVars]}]
       If {\tt useDstCoordFlag} is .TRUE., this argument defines the longitude and
       latitude variables in the destination grid file to be used for the regrid.
       This argument is only used when the grid file is in GRIDSPEC format.
       {\tt dstCoordinateVars} should be a array of 2 elements.
     \item [{[useSrcCornerFlag]}]
       If {\tt useSrcCornerFlag} is .TRUE., the corner coordinates of the source file
       will be used for regridding. Otherwise, the center coordinates will be us ed.
       The default is .FALSE. The corner stagger is not supported for the SCRIP formatted input
       grid or multi-tile GRIDSPEC MOSAIC input grid.
     \item [{[useDstCornerFlag]}]
       If {\tt useDstCornerFlag} is .TRUE., the corner coordinates of the destination file
       will be used for regridding. Otherwise, the center coordinates will be used.
       The default is .FALSE. The corner stagger is not supported for the SCRIP formatted input
       grid or multi-tile GRIDSPEC MOSAIC input grid.
     \item [{[useUserAreaFlag]}]
       If .TRUE., the element area values defined in the grid files are used.
       Only the SCRIP and ESMF format grid files have user specified areas. This flag
       is only used for conservative regridding. The default is .FALSE..
     \item [{[largefileFlag]}]
       If .TRUE., the output weight file is in NetCDF 64bit offset format.
       The default is .FALSE..
     \item [{[netcdf4fileFlag]}]
       If .TRUE., the output weight file is in NetCDF4 file format.
       The default is .FALSE..
     \item [{[weightOnlyFlag]}]
       If .TRUE., the output weight file only contains factorList and factorIndexList.
       The default is .FALSE..
     \item [{[verboseFlag]}]
       If .TRUE., it will print summary information about the regrid parameters,
       default to .FALSE..
     \item[{[tileFilePath]}]
       Optional argument to define the path where the tile files reside. If it
       is given, it overwrites the path defined in {\tt gridlocation} variable
       in the mosaic file.
     \item [{[rc]}]
       Return code; equals {\tt ESMF\_SUCCESS} if there are no errors.
     \end{description} 
%/////////////////////////////////////////////////////////////
 
\mbox{}\hrulefill\ 
 
\subsubsection [ESMF\_RegridWeightGen] {ESMF\_RegridWeightGen - Generate regrid routeHandle and an optional weight file from grid files with user-specified distribution}


   \label{api:esmf_regridweightgenDG}
\bigskip{\sf INTERFACE:}
\begin{verbatim}   ! Private name; call using ESMF_RegridWeightGen()
   subroutine ESMF_RegridWeightGenDG(srcFile, dstFile, regridRouteHandle, &
     srcElementDistgrid, dstElementDistgrid, &
     srcNodalDistgrid, dstNodalDistgrid, &
     weightFile, regridmethod, lineType, normType, &
     extrapMethod, extrapNumSrcPnts, extrapDistExponent, extrapNumLevels,&
     unmappedaction, ignoreDegenerate, useUserAreaFlag, &
     largefileFlag, netcdf4fileFlag, &
     weightOnlyFlag, verboseFlag, rc)
 \end{verbatim}{\em ARGUMENTS:}
\begin{verbatim} 
   character(len=*),             intent(in)            :: srcFile
   character(len=*),             intent(in)            :: dstFile
   type(ESMF_RouteHandle),       intent(out)           :: regridRouteHandle
 -- The following arguments require argument keyword syntax (e.g. rc=rc). --
   type(ESMF_DistGrid),          intent(in),  optional :: srcElementDistgrid
   type(ESMF_DistGrid),          intent(in),  optional :: dstElementDistgrid
   character(len=*),             intent(in),  optional :: weightFile
   type(ESMF_DistGrid),          intent(in),  optional :: srcNodalDistgrid
   type(ESMF_DistGrid),          intent(in),  optional :: dstNodalDistgrid
   type(ESMF_RegridMethod_Flag), intent(in),  optional :: regridmethod
   type(ESMF_LineType_Flag),     intent(in),  optional :: lineType
   type(ESMF_NormType_Flag),     intent(in),  optional :: normType
   type(ESMF_ExtrapMethod_Flag),   intent(in),    optional :: extrapMethod
   integer,                        intent(in),    optional :: extrapNumSrcPnts
   real,                           intent(in),    optional :: extrapDistExponent
   integer,                      intent(in),  optional :: extrapNumLevels
   type(ESMF_UnmappedAction_Flag),intent(in), optional :: unmappedaction
   logical,                      intent(in),  optional :: ignoreDegenerate
   logical,                      intent(in),  optional :: useUserAreaFlag
   logical,                      intent(in),  optional :: largefileFlag
   logical,                      intent(in),  optional :: netcdf4fileFlag
   logical,                      intent(in),  optional :: weightOnlyFlag
   logical,                      intent(in),  optional :: verboseFlag
   integer,                      intent(out), optional :: rc
 \end{verbatim}
{\sf DESCRIPTION:\\ }


   This subroutine does online regridding weight generation from files with user specified distribution.
   The main differences between this API and the one in \ref{api:esmf_regridweightgenfile} are listed below:
   \begin{itemize}
   \item The input grids are always represented as {\tt ESMF\_Mesh} whether they are logically rectangular or unstructured.
   \item The input grids will be decomposed using a user-specified distribution instead of a fixed decomposition in the
   other subroutine if {\tt srcElementDistgrid} and {\tt dstElementDistgrid} are specified.
   \item The source and destination grid files have to be in the SCRIP grid file format.
   \item This subroutine has one additional required argument {\tt regridRouteHandle} and four additional optional
   arguments: {\tt srcElementDistgrid}, {\tt dstElementDistgrid}, {\tt srcNodelDistgrid} and {\tt dstNodalDistgrid}.
   These four arguments are of type {\tt ESMF\_DistGrid}, they are used to define the distribution of the source
   and destination grid elements and nodes. The output {\tt regridRouteHandle} allows users to regrid the field
   values later in the application.
   \item The {\tt weightFile} argument is optional. When it is given, a weightfile will be generated as well.
   \end{itemize}
   \smallskip
  
   The arguments are:
     \begin{description}
     \item [srcFile]
       The source grid file name in SCRIP grid file format
     \item [dstFile]
       The destination grid file name in SCRIP grid file format
     \item [regridRouteHandle]
       The regrid RouteHandle returned by {\tt ESMF\_FieldRegridStore()}
     \item [srcElementDistgrid]
       An optional distGrid that specifies the distribution of the source grid's elements. If not
       specified, a system-defined block decomposition is used.
     \item [dstElementDistgrid]
       An optional distGrid that specifies the distribution of the destination grid's elements. If
       not specified, a system-defined block decomposition is used.
     \item [weightFile]
       The interpolation weight file name. If present, an output weight file will be generated.
     \item [srcNodalDistgrid]
       An optional distGrid that specifies the distribution of the source grid's nodes
     \item [dstNodalDistgrid]
       An optional distGrid that specifies the distribution of the destination grid's nodes
     \item [{[regridmethod]}]
       The type of interpolation. Please see Section~\ref{opt:regridmethod}
       for a list of valid options. If not specified, defaults to
       {\tt ESMF\_REGRIDMETHOD\_BILINEAR}.
     \item [{[lineType]}]
             This argument controls the path of the line which connects two points on a sphere surface. This in
             turn controls the path along which distances are calculated and the shape of the edges that make
             up a cell. Both of these quantities can influence how interpolation weights are calculated.
             As would be expected, this argument is only applicable when {\tt srcField} and {\tt dstField} are
             built on grids which lie on the surface of a sphere. Section~\ref{opt:lineType} shows a
             list of valid options for this argument. If not specified, the default depends on the
             regrid method. Section~\ref{opt:lineType} has the defaults by line type. Figure~\ref{line_type_support} shows
             which line types are supported for each regrid method as well as showing the default line type by regrid method.
       \item [{[normType]}]
             This argument controls the type of normalization used when generating conservative weights. This option
             only applies to weights generated with {\tt regridmethod=ESMF\_REGRIDMETHOD\_CONSERVE}. Please see
             Section~\ref{opt:normType} for a
             list of valid options. If not specified {\tt normType} defaults to {\tt ESMF\_NORMTYPE\_DSTAREA}.
       \item [{[extrapMethod]}]
             The type of extrapolation. Please see Section~\ref{opt:extrapmethod}
             for a list of valid options. If not specified, defaults to
             {\tt ESMF\_EXTRAPMETHOD\_NONE}.
       \item [{[extrapNumSrcPnts]}]
             The number of source points to use for the extrapolation methods that use more than one source point
             (e.g. {\tt ESMF\_EXTRAPMETHOD\_NEAREST\_IDAVG}). If not specified, defaults to 8..
       \item [{[extrapDistExponent]}]
             The exponent to raise the distance to when calculating weights for
             the {\tt ESMF\_EXTRAPMETHOD\_NEAREST\_IDAVG} extrapolation method. A higher value reduces the influence
             of more distant points. If not specified, defaults to 2.0.
       \item [{[unmappedaction]}]
             Specifies what should happen if there are destination points that
             can't be mapped to a source cell. Please see Section~\ref{const:unmappedaction} for a
             list of valid options. If not specified, {\tt unmappedaction} defaults to {\tt ESMF\_UNMAPPEDACTION\_ERROR}.
       \item [{[ignoreDegenerate]}]
             Ignore degenerate cells when checking the input Grids or Meshes for errors. If this is set to true, then the
             regridding proceeds, but degenerate cells will be skipped. If set to false, a degenerate cell produces an error.
             If not specified, {\tt ignoreDegenerate} defaults to false.
     \item [{[useUserAreaFlag]}]
       If .TRUE., the element area values defined in the grid files are used.
       Only the SCRIP and ESMF format grid files have user specified areas. This flag
       is only used for conservative regridding. The default is .FALSE.
     \item [{[largefileFlag]}]
       If .TRUE., the output weight file is in NetCDF 64bit offset format.
       The default is .FALSE.
     \item [{[netcdf4fileFlag]}]
       If .TRUE., the output weight file is in NetCDF4 file format.
       The default is .FALSE.
     \item [{[weightOnlyFlag]}]
       If .TRUE., the output weight file only contains factorList and factorIndexList.
       The default is .FALSE.
     \item [{[verboseFlag]}]
       If .TRUE., it will print summary information about the regrid parameters,
       default to .FALSE.
     \item [{[rc]}]
       Return code; equals {\tt ESMF\_SUCCESS} if there are no errors.
     \end{description}
%...............................................................
\setlength{\parskip}{\oldparskip}
\setlength{\parindent}{\oldparindent}
\setlength{\baselineskip}{\oldbaselineskip}
