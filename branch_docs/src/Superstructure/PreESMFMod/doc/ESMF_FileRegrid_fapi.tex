%                **** IMPORTANT NOTICE *****
% This LaTeX file has been automatically produced by ProTeX v. 1.1
% Any changes made to this file will likely be lost next time
% this file is regenerated from its source. Send questions 
% to Arlindo da Silva, dasilva@gsfc.nasa.gov
 
\setlength{\oldparskip}{\parskip}
\setlength{\parskip}{1.5ex}
\setlength{\oldparindent}{\parindent}
\setlength{\parindent}{0pt}
\setlength{\oldbaselineskip}{\baselineskip}
\setlength{\baselineskip}{11pt}
 
%--------------------- SHORT-HAND MACROS ----------------------
\def\bv{\begin{verbatim}}
\def\ev{\end{verbatim}}
\def\be{\begin{equation}}
\def\ee{\end{equation}}
\def\bea{\begin{eqnarray}}
\def\eea{\end{eqnarray}}
\def\bi{\begin{itemize}}
\def\ei{\end{itemize}}
\def\bn{\begin{enumerate}}
\def\en{\end{enumerate}}
\def\bd{\begin{description}}
\def\ed{\end{description}}
\def\({\left (}
\def\){\right )}
\def\[{\left [}
\def\]{\right ]}
\def\<{\left  \langle}
\def\>{\right \rangle}
\def\cI{{\cal I}}
\def\diag{\mathop{\rm diag}}
\def\tr{\mathop{\rm tr}}
%-------------------------------------------------------------

\markboth{Left}{Source File: ESMF\_FileRegrid.F90,  Date: Tue May  5 21:00:18 MDT 2020
}

 
%/////////////////////////////////////////////////////////////
\subsubsection [ESMF\_FileRegrid] {ESMF\_FileRegrid - Regrid variables defined in the grid files}


   \label{api:esmf_fileregrid}
\bigskip{\sf INTERFACE:}
\begin{verbatim}   subroutine ESMF_FileRegrid(srcFile, dstFile, srcVarName, dstVarName, &
     dstLoc, srcDataFile, dstDataFile, tileFilePath, &
     dstCoordVars, regridmethod, polemethod, regridPoleNPnts, &
     unmappedaction, ignoreDegenerate, srcRegionalFlag, dstRegionalFlag, &
     verboseFlag, rc)
 \end{verbatim}{\em ARGUMENTS:}
\begin{verbatim} 
   character(len=*),             intent(in)            :: srcFile
   character(len=*),             intent(in)            :: dstFile
   character(len=*),             intent(in)            :: srcVarName
   character(len=*),             intent(in)            :: dstVarName
 -- The following arguments require argument keyword syntax (e.g. rc=rc). --
   character(len=*),             intent(in),  optional :: dstLoc
   character(len=*),             intent(in),  optional :: srcDataFile     
   character(len=*),             intent(in),  optional :: dstDataFile     
   character(len=*),             intent(in),  optional :: tileFilePath
   character(len=*),             intent(in),  optional :: dstCoordVars
   type(ESMF_RegridMethod_Flag), intent(in),  optional :: regridmethod
   type(ESMF_PoleMethod_Flag),   intent(in),  optional :: polemethod
   integer,                      intent(in),  optional :: regridPoleNPnts
   type(ESMF_UnmappedAction_Flag),intent(in), optional :: unmappedaction
   logical,                      intent(in),  optional :: ignoreDegenerate
   logical,                      intent(in),  optional :: srcRegionalFlag
   logical,                      intent(in),  optional :: dstRegionalFlag
   logical,                      intent(in),  optional :: verboseFlag
   integer,                      intent(out), optional :: rc
 \end{verbatim}
{\sf DESCRIPTION:\\ }


   This subroutine provides the same function as the {\tt ESMF\_Regrid} application
   described in Section~\ref{sec:ESMF_Regrid}.  It takes two grid files in NetCDF format and interpolate
   the variable defined in the source grid file to the destination variable using one of the ESMF supported
   regrid methods -- bilinear~(\ref{sec:interpolation:bilinear}), higher-order patch~(\ref{sec:interpolation:patch}),
   first order conservative~(\ref{sec:interpolation:conserve}) or nearest neighbor methods.
   The grid files can be in one of the following two formats:
   \begin{itemize}
   \item The GRIDSPEC Tile grid file following the CF metadata convention~(\ref{sec:fileformat:gridspec}) for logically rectangular grids
   \item The proposed CF Unstructured grid (UGRID) format~(\ref{sec:fileformat:ugrid}) for  unstructured grids.
   \end{itemize}
   \smallskip
   The optional arguments allow users to specify various options to control the regrid operation,
   such as which pole option to use, or whether to use user-specified area in the conservative regridding.
   The acceptable values and the default value for the optional arguments are listed below.
  
   The arguments are:
     \begin{description}
     \item [srcFile]
       The source grid file name.
     \item [dstFile]
       The destination grid file name.
     \item [srcVarName]
       The source variable names to be regridded. If more than one, separate them by comma.
     \item [dstVarName]
       The destination variable names to be regridded to. If more than one, separate them by comma.
     \item [{[dstLoc]}]
       The destination variable's location, either 'node' or 'face'.  This
       argument is only used when the destination grid file is UGRID, the regridding method is
       non-conservative and the destination variable does not exist in the destination grid file.
       If not specified, default is 'face'.
     \item [{[srcDataFile]}]
       The input data file prefix if the srcFile is in GRIDSPEC MOSAIC
       fileformat.  The tilename and the file extension (.nc) will be added to
       the prefix.  The tilename is defined in the MOSAIC file using variable "gridtiles".
     \item [{[dstDataFile]}]
       The output data file prefix if the dstFile is in GRIDSPEC MOSAIC
       fileformat.  The tilename and the file extension (.nc) will be added to
       the prefix.  The tilename is defined in the MOSAIC file using variable "gridtiles".
     \item [{[tileFilePath]}]
       The alternative file path for the tile files and mosaic data files when either srcFile or
       dstFile is a GRIDSPEC MOSAIC grid.  The path can be either relative or absolute.  If it is
       relative, it is relative to the working directory.  When specified, the gridlocation variable
       defined in the Mosaic file will be ignored.
     \item [{[dstCoordVars]}]
       The destination coordinate variable names if the dstVarName does not exist in the dstFile
     \item [{[regridmethod]}]
       The type of interpolation. Please see Section~\ref{opt:regridmethod}
       for a list of valid options. If not specified, defaults to
       {\tt ESMF\_REGRIDMETHOD\_BILINEAR}.
     \item [{[polemethod]}]
       A flag to indicate which type of artificial pole
       to construct on the source Grid for regridding. Please see
       Section~\ref{const:polemethod} for a list of valid options.
       The default value varies depending on the regridding method and the grid type and format.
     \item [{[regridPoleNPnts]}]
       If {\tt polemethod} is set to {\tt ESMF\_POLEMETHOD\_NPNTAVG}, this argument is required to
       specify how many points should be averaged over at the pole.
     \item [{[unmappedaction]}]
             Specifies what should happen if there are destination points that
             can't be mapped to a source cell. Please see Section~\ref{const:unmappedaction} for a
             list of valid options. If not specified, {\tt unmappedaction} defaults to {\tt ESMF\_UNMAPPEDACTION\_ERROR}.
     \item [{[ignoreDegenerate]}]
             Ignore degenerate cells when checking the input Grids or Meshes for errors. If this is set to true, then the
             regridding proceeds, but degenerate cells will be skipped. If set to false, a degenerate cell produces an error.
             If not specified, {\tt ignoreDegenerate} defaults to false.
     \item [{[srcRegionalFlag]}]
       If .TRUE., the source grid is a regional grid, otherwise,
       it is a global grid.  The default value is .FALSE.
     \item [{[dstRegionalFlag]}]
       If .TRUE., the destination grid is a regional grid, otherwise,
       it is a global grid.  The default value is .FALSE.
     \item [{[verboseFlag]}]
       If .TRUE., it will print summary information about the regrid parameters,
       default to .FALSE.
     \item [{[rc]}]
       Return code; equals {\tt ESMF\_SUCCESS} if there are no errors.
     \end{description}
%...............................................................
\setlength{\parskip}{\oldparskip}
\setlength{\parindent}{\oldparindent}
\setlength{\baselineskip}{\oldbaselineskip}
