%                **** IMPORTANT NOTICE *****
% This LaTeX file has been automatically produced by ProTeX v. 1.1
% Any changes made to this file will likely be lost next time
% this file is regenerated from its source. Send questions 
% to Arlindo da Silva, dasilva@gsfc.nasa.gov
 
\setlength{\oldparskip}{\parskip}
\setlength{\parskip}{1.5ex}
\setlength{\oldparindent}{\parindent}
\setlength{\parindent}{0pt}
\setlength{\oldbaselineskip}{\baselineskip}
\setlength{\baselineskip}{11pt}
 
%--------------------- SHORT-HAND MACROS ----------------------
\def\bv{\begin{verbatim}}
\def\ev{\end{verbatim}}
\def\be{\begin{equation}}
\def\ee{\end{equation}}
\def\bea{\begin{eqnarray}}
\def\eea{\end{eqnarray}}
\def\bi{\begin{itemize}}
\def\ei{\end{itemize}}
\def\bn{\begin{enumerate}}
\def\en{\end{enumerate}}
\def\bd{\begin{description}}
\def\ed{\end{description}}
\def\({\left (}
\def\){\right )}
\def\[{\left [}
\def\]{\right ]}
\def\<{\left  \langle}
\def\>{\right \rangle}
\def\cI{{\cal I}}
\def\diag{\mathop{\rm diag}}
\def\tr{\mathop{\rm tr}}
%-------------------------------------------------------------

\markboth{Left}{Source File: ESMF\_Attribute.F90,  Date: Tue May  5 21:00:12 MDT 2020
}

 
%/////////////////////////////////////////////////////////////
 
%/////////////////////////////////////////////////////////////
 
\mbox{}\hrulefill\ 
 
\subsubsection [ESMF\_AttributeAdd] {ESMF\_AttributeAdd - Add an ESMF standard Attribute package}


  
\bigskip{\sf INTERFACE:}
\begin{verbatim}   ! Private name; call using ESMF_AttributeAdd()
   subroutine ESMF_AttAddPackStd(<object>, convention, purpose, attpack, rc)\end{verbatim}{\em ARGUMENTS:}
\begin{verbatim}   <object>, see below for supported values
   character (len = *), intent(in) :: convention
   character (len = *), intent(in) :: purpose
   type(ESMF_AttPack), intent(inout), optional :: attpack
   integer, intent(out), optional :: rc\end{verbatim}
{\sf DESCRIPTION:\\ }


   Add an ESMF standard Attribute package. See Section~\ref{sec:AttPacks}
   for a description of Attribute packages.
  
   Supported values for <object> are:
   \begin{description}
   \item type(ESMF\_Array), intent(inout) :: array
   \item type(ESMF\_CplComp), intent(inout) :: comp
   \item type(ESMF\_GridComp), intent(inout) :: comp
   \item type(ESMF\_SciComp), intent(inout) :: comp
   \item type(ESMF\_Field), intent(inout) :: field
   \item type(ESMF\_Grid), intent(inout) :: grid
   \item type(ESMF\_State), intent(inout) :: state
   \end{description}
  
   The arguments are:
   \begin{description}
   \item [<object>]
   An {\tt ESMF} object.
   \item [convention]
   The convention of the new Attribute package.
   \item [purpose]
   The purpose of the new Attribute package.
   \item [{[attpack]}]
   An optional handle to the Attribute package that is to be created.
   \item [{[rc]}]
   Return code; equals {\tt ESMF\_SUCCESS} if there are no errors.
   \end{description}
  
   
%/////////////////////////////////////////////////////////////
 
\mbox{}\hrulefill\ 
 
\subsubsection [ESMF\_AttributeAdd] {ESMF\_AttributeAdd - Add an ESMF standard Attribute package containing nested standard Attribute packages}


  
\bigskip{\sf INTERFACE:}
\begin{verbatim}   ! Private name; call using ESMF_AttributeAdd()
   subroutine ESMF_AttAddPackStdN(<object>, convention, purpose, &
   nestConvention, nestPurpose, nestAttPackInstanceCountList, &
   nestAttPackInstanceNameList, nestCount, &
   nestAttPackInstanceNameCount, attpack, rc)\end{verbatim}{\em ARGUMENTS:}
\begin{verbatim}   <object>, see below for supported values
   character (len = *), intent(in) :: convention
   character (len = *), intent(in) :: purpose
   character (len = *), intent(in) :: nestConvention(:)
   character (len = *), intent(in) :: nestPurpose(:)
   integer, intent(in) :: nestAttPackInstanceCountList(:)
   character (len = *), intent(out) :: nestAttPackInstanceNameList(:)
   integer, intent(in), optional :: nestCount
   integer, intent(out), optional :: nestAttPackInstanceNameCount
   type(ESMF_AttPack), intent(inout), optional :: attpack
   integer, intent(out), optional :: rc\end{verbatim}
{\sf DESCRIPTION:\\ }


   Add an ESMF standard Attribute package which contains a user-specified
   number of nested standard Attribute packages. ESMF generates and returns
   default instance names for the nested Attribute packages. These names
   can be used later to distinguish among multiple nested Attribute
   packages of the same type in calls to {\tt ESMF\_AttributeGet()},
   {\tt ESMF\_AttributeSet()}, and {\tt ESMF\_AttributeRemove()}.
   See Section~\ref{sec:AttPacks} for a description of Attribute packages.
  
   Supported values for <object> are:
   \begin{description}
   \item type(ESMF\_CplComp), intent(inout) :: comp
   \item type(ESMF\_GridComp), intent(inout) :: comp
   \item type(ESMF\_SciComp), intent(inout) :: comp
   \end{description}
  
   The arguments are:
   \begin{description}
   \item [<object>]
   An {\tt ESMF} object.
   \item [convention]
   The convention of the new Attribute package.
   \item [purpose]
   The purpose of the new Attribute package.
   \item [nestConvention]
   The convention(s) of the standard Attribute package(s) around
   which to nest the new Attribute package.
   \item [nestPurpose]
   The purpose(s) of the standard Attribute package(s) around
   which to nest the new Attribute package.
   \item [nestAttPackInstanceCountList]
   The desired number of nested Attribute package instances for each
   nested (nestConvention, nestPurpose) package type. Note: if only one
   of each nested package type is desired, then the
   {\tt ESMF\_AttributeAdd()} overloaded method
   {\tt ESMF\_AttAddPackStd()} should be used.
   \item [nestAttPackInstanceNameList]
   The name(s) of the nested Attribute package instances, generated
   by ESMF, used to distinguish between multiple instances of the
   same convention and purpose.
   \item [{[nestCount]}]
   The count of the number of nested Attribute package types to add to
   the new Attribute package.
   \item [{[nestAttPackInstanceNameCount]}]
   The number of nested Attribute package instance names.
   \item [{[attpack]}]
   An optional handle to the Attribute package that is to be created.
   \item [{[rc]}]
   Return code; equals {\tt ESMF\_SUCCESS} if there are no errors.
   \end{description}
  
   
%/////////////////////////////////////////////////////////////
 
\mbox{}\hrulefill\ 
 
\subsubsection [ESMF\_AttributeAdd] {ESMF\_AttributeAdd - Add a custom Attribute package or modify an existing Attribute package}


  
\bigskip{\sf INTERFACE:}
\begin{verbatim}   ! Private name; call using ESMF_AttributeAdd()
   subroutine ESMF_AttAddPackCst(<object>, convention, purpose, &
   attrList, count, redundant, attpack, rc)\end{verbatim}{\em ARGUMENTS:}
\begin{verbatim}   <object>, see below for supported values
   character (len = *), intent(in) :: convention
   character (len = *), intent(in) :: purpose
   character (len = *), intent(in) :: attrList(:)
   integer, intent(in), optional :: count
   logical, intent(in), optional :: redundant
   type(ESMF_AttPack), intent(inout), optional :: attpack
   integer, intent(out), optional :: rc\end{verbatim}
{\sf DESCRIPTION:\\ }


   Add a custom Attribute package to <object>, or add
   Attributes to an existing Attribute package. The {\tt redundant} flag can
   be set to {\tt .true.} to create redundant Attribute packages. Otherwise,
   Attributes will be added to an existing package. The {\tt attpack} will be
   used instead of {\tt convention} and {\tt purpose} if both are present.
   See Section~\ref{sec:AttPacks} for a description of Attribute packages.
  
   Supported values for <object> are:
   \begin{description}
   \item type(ESMF\_Array), intent(inout) :: array
   \item type(ESMF\_ArrayBundle), intent(inout) :: arraybundle
   \item type(ESMF\_CplComp), intent(inout) :: comp
   \item type(ESMF\_GridComp), intent(inout) :: comp
   \item type(ESMF\_SciComp), intent(inout) :: comp
   \item type(ESMF\_DistGrid), intent(inout) :: distgrid
   \item type(ESMF\_Field), intent(inout) :: field
   \item type(ESMF\_FieldBundle), intent(inout) :: fieldbundle
   \item type(ESMF\_Grid), intent(inout) :: grid
   \item type(ESMF\_State), intent(inout) :: state
   \end{description}
  
   The arguments are:
   \begin{description}
   \item [<object>]
   An {\tt ESMF} object.
   \item [convention]
   The convention of the Attribute package.
   \item [purpose]
   The purpose of the Attribute package.
   \item [attrList]
   The list of Attribute names to specify the custom Attribute package.
   \item [{[count]}]
   The number of Attributes to add to the custom Attribute package.
   \item [{[redundant]}]
   A flag to determine whether or not to create redundant Attribute
   packages. If an Attribute package already exists with the specified
   {\tt convention} and {\tt purpose} and {\tt redundant} is set to
   {\tt .true.} then a redundant Attribute package will be created.
   The default value is {\tt .false.}.
   \item [{[attpack]}]
   The handle to the Attribute package that was created.
   This can also be used as an input parameter to indicate the
   Attribute package to which additional Attributes should be added.
   \item [{[rc]}]
   Return code; equals {\tt ESMF\_SUCCESS} if there are no errors.
   \end{description}
  
   
%/////////////////////////////////////////////////////////////
 
\mbox{}\hrulefill\ 
 
\subsubsection [ESMF\_AttributeAdd] {ESMF\_AttributeAdd - Add a custom Attribute package with nested Attribute packages}


  
\bigskip{\sf INTERFACE:}
\begin{verbatim}   ! Private name; call using ESMF_AttributeAdd()
   subroutine ESMF_AttAddPackCstN(<object>, convention, purpose, &
   attrList, count, nestConvention, nestPurpose, nestCount, attpack, rc)\end{verbatim}{\em ARGUMENTS:}
\begin{verbatim}   <object>, see below for supported values
   character (len = *), intent(in) :: convention
   character (len = *), intent(in) :: purpose
   character (len = *), intent(in), optional :: attrList(:)
   integer, intent(in), optional :: count
   character (len = *), intent(in) :: nestConvention(:)
   character (len = *), intent(in) :: nestPurpose(:)
   integer, intent(in), optional :: nestCount
   type(ESMF_AttPack), intent(inout), optional :: attpack
   integer, intent(out), optional :: rc\end{verbatim}
{\sf DESCRIPTION:\\ }


   Add a custom Attribute package, with one or more nested Attribute
   packages, to <object>. Allows for building full multiple-child Attribute
   hierarchies (multi-child trees).
   See Section~\ref{sec:AttPacks} for a description of Attribute packages.
  
   Supported values for <object> are:
   \begin{description}
   \item type(ESMF\_Array), intent(inout) :: array
   \item type(ESMF\_ArrayBundle), intent(inout) :: arraybundle
   \item type(ESMF\_CplComp), intent(inout) :: comp
   \item type(ESMF\_GridComp), intent(inout) :: comp
   \item type(ESMF\_SciComp), intent(inout) :: comp
   \item type(ESMF\_DistGrid), intent(inout) :: distgrid
   \item type(ESMF\_Field), intent(inout) :: field
   \item type(ESMF\_FieldBundle), intent(inout) :: fieldbundle
   \item type(ESMF\_Grid), intent(inout) :: grid
   \item type(ESMF\_State), intent(inout) :: state
   \end{description}
  
   The arguments are:
   \begin{description}
   \item [<object>]
   An {\tt ESMF} object.
   \item [convention]
   The convention of the Attribute package.
   \item [purpose]
   The purpose of the Attribute package.
   \item [{[attrList]}]
   The list of Attribute names to specify the custom Attribute package.
   \item [{[count]}]
   The number of Attributes to add to the custom Attribute package.
   \item [nestConvention]
   The convention(s) of the Attribute package(s) around which to nest
   the new Attribute package.
   \item [nestPurpose]
   The purpose(s) of the Attribute package(s) around which to nest the
   new Attribute package.
   \item [{[nestCount]}]
   The number of nested Attribute packages to add to the custom
   Attribute package.
   \item [{[attpack]}]
   An optional handle to the Attribute package that is to be created.
   \item [{[rc]}]
   Return code; equals {\tt ESMF\_SUCCESS} if there are no errors.
   \end{description}
  
   
%/////////////////////////////////////////////////////////////
 
\mbox{}\hrulefill\ 
 
\subsubsection [ESMF\_AttributeAdd] {ESMF\_AttributeAdd - Add a custom Attribute package with a single nested Attribute package}


  
\bigskip{\sf INTERFACE:}
\begin{verbatim}   ! Private name; call using ESMF_AttributeAdd()
   subroutine ESMF_AttAddPackCstN1(<object>, convention, purpose, &
   attrList, count, nestConvention, nestPurpose, attpack, rc)\end{verbatim}{\em ARGUMENTS:}
\begin{verbatim}   <object>, see below for supported values
   character (len = *), intent(in) :: convention
   character (len = *), intent(in) :: purpose
   character (len = *), intent(in), optional :: attrList(:)
   integer, intent(in), optional :: count
   character (len = *), intent(in) :: nestConvention
   character (len = *), intent(in) :: nestPurpose
   type(ESMF_AttPack), intent(inout), optional :: attpack
   integer, intent(out), optional :: rc\end{verbatim}
{\sf DESCRIPTION:\\ }


   Add a custom Attribute package, with a single nested Attribute
   package, to <object>. Allows for building single-child Attribute
   hierarchies (single-child trees).
   See Section~\ref{sec:AttPacks} for a description of Attribute packages.
  
   Supported values for <object> are:
   \begin{description}
   \item type(ESMF\_Array), intent(inout) :: array
   \item type(ESMF\_ArrayBundle), intent(inout) :: arraybundle
   \item type(ESMF\_CplComp), intent(inout) :: comp
   \item type(ESMF\_GridComp), intent(inout) :: comp
   \item type(ESMF\_SciComp), intent(inout) :: comp
   \item type(ESMF\_DistGrid), intent(inout) :: distgrid
   \item type(ESMF\_Field), intent(inout) :: field
   \item type(ESMF\_FieldBundle), intent(inout) :: fieldbundle
   \item type(ESMF\_Grid), intent(inout) :: grid
   \item type(ESMF\_State), intent(inout) :: state
   \end{description}
  
   The arguments are:
   \begin{description}
   \item [<object>]
   An {\tt ESMF} object.
   \item [convention]
   The convention of the Attribute package.
   \item [purpose]
   The purpose of the Attribute package.
   \item [{[attrList]}]
   The list of Attribute names to specify the custom Attribute package.
   \item [{[count]}]
   The number of Attributes to add to the custom Attribute package.
   \item [nestConvention]
   The convention of the Attribute package around which to nest
   the new Attribute package.
   \item [nestPurpose]
   The purpose of the Attribute package around which to nest the
   new Attribute package.
   \item [{[attpack]}]
   An optional handle to the Attribute package that is to be created.
   \item [{[rc]}]
   Return code; equals {\tt ESMF\_SUCCESS} if there are no errors.
   \end{description}
  
   
%/////////////////////////////////////////////////////////////
 
\mbox{}\hrulefill\ 
 
\subsubsection [ESMF\_AttributeCopy] {ESMF\_AttributeCopy - Copy an Attribute hierarchy}


  
\bigskip{\sf INTERFACE:}
\begin{verbatim}   ! Private name; call using ESMF_AttributeCopy()
   subroutine ESMF_AttributeCopy(<object1>, <object2>, attcopy, rc)\end{verbatim}{\em ARGUMENTS:}
\begin{verbatim}   <object1>, see below for supported values
   <object2>, see below for supported values
   type(ESMF_AttCopy_Flag),intent(in) optional :: attcopy
   integer, intent(out), optional :: rc\end{verbatim}
{\sf DESCRIPTION:\\ }


   Copy an Attribute hierarchy from <object1> to <object2>. The
   default behavior is to ignore (instead of replace) values on
   pre-existing Attributes.
  
   Supported values for <object1> are:
   \begin{description}
   \item type(ESMF\_CplComp), intent(in) :: comp1
   \item type(ESMF\_GridComp), intent(in) :: comp1
   \item type(ESMF\_SciComp), intent(in) :: comp1
   \item type(ESMF\_Field), intent(inout) :: field1
   \item type(ESMF\_FieldBundle), intent(inout) :: fieldbundle1
   \item type(ESMF\_Grid), intent(inout) :: grid1
   \item type(ESMF\_State), intent(in) :: state
   \end{description}
  
   Supported values for <object2> are:
   \begin{description}
   \item type(ESMF\_CplComp), intent(inout) :: comp2
   \item type(ESMF\_GridComp), intent(inout) :: comp2
   \item type(ESMF\_SciComp), intent(inout) :: comp2
   \item type(ESMF\_Field), intent(inout) :: field2
   \item type(ESMF\_FieldBundle), intent(inout) :: fieldbundle2
   \item type(ESMF\_Grid), intent(inout) :: grid2
   \item type(ESMF\_State), intent(inout) :: state
   \end{description}
  
   NOTE: Copies between different ESMF object types are not possible at this time.
  
   The arguments are:
   \begin{description}
   \item [<object1>]
   An {\tt Attribute}-bearing ESMF object.
   \item [<object2>]
   An {\tt Attribute}-bearing ESMF object.
   \item [{[attcopy]}]
   A flag to determine if the copy is to be by reference, value,
   or hybrid. This flag is documented in section \ref{const:attcopy}.
   The default is to copy by value.
   \item [{[rc]}]
   Return code; equals {\tt ESMF\_SUCCESS} if there are no errors.
   \end{description}
   
%/////////////////////////////////////////////////////////////
 
\mbox{}\hrulefill\ 
 
\subsubsection [ESMF\_AttributeGet] {ESMF\_AttributeGet - Get an Attribute from an ESMF\_AttPack}


  
\bigskip{\sf INTERFACE:}
\begin{verbatim}   subroutine ESMF_AttributeGet(<object>, name, attpack, <value> &
   <defaultvalue>, attnestflag, isPresent, rc)\end{verbatim}{\em ARGUMENTS:}
\begin{verbatim}   <object>, see below for supported values
   character (len = *), intent(in) :: name
   type(ESMF_AttPack), intent(inout) :: attpack
   <value>, see below for supported values
 -- The following arguments require argument keyword syntax (e.g. rc=rc). --
   <defaultvalue>, see below for supported values
   type(ESMF_AttNest_Flag),intent(in), optional :: attnestflag
   logical, intent(out), optional :: isPresent
   integer, intent(out), optional :: rc\end{verbatim}
{\sf DESCRIPTION:\\ }


   Return an Attribute {\tt value} from the <object>, or from an Attribute
   package on the <object>, specified by {\tt attpack}. Internal information can also
   be retrieved from Grid objects by prepending 'ESMF:' to the name of the
   piece of information that is requested. See
   Section~\ref{sec:InternalInfo} for more information
   on which pieces of Grid data can be retrieved through this interface.
   A {\tt defaultvalue} argument
   may be given if a return code is not desired when the Attribute is not
   found. See Section~\ref{sec:AttPacks} for a description of Attribute
   packages.
  
   Supported values for <object> are:
   \begin{description}
   \item type(ESMF\_Array), intent(in) :: array
   \item type(ESMF\_ArrayBundle), intent(in) :: arraybundle
   \item type(ESMF\_CplComp), intent(in) :: comp
   \item type(ESMF\_GridComp), intent(in) :: comp
   \item type(ESMF\_SciComp), intent(in) :: comp
   \item type(ESMF\_DistGrid), intent(in) :: distgrid
   \item type(ESMF\_Field), intent(in) :: field
   \item type(ESMF\_FieldBundle), intent(in) :: fieldbundle
   \item type(ESMF\_Grid), intent(in) :: grid
   \item type(ESMF\_State), intent(in) :: state
   \end{description}
  
   Supported values for <value> are:
   \begin{description}
   \item integer(ESMF\_KIND\_I4), intent(out) :: value
   \item integer(ESMF\_KIND\_I8), intent(out) :: value
   \item real (ESMF\_KIND\_R4), intent(out) :: value
   \item real (ESMF\_KIND\_R8), intent(out) :: value
   \item logical, intent(out) :: value
   \item character (len = *), intent(out) :: value
   \end{description}
  
   Supported values for <defaultvalue> are:
   \begin{description}
   \item integer(ESMF\_KIND\_I4), intent(in), optional :: defaultvalue
   \item integer(ESMF\_KIND\_I8), intent(in), optional :: defaultvalue
   \item real (ESMF\_KIND\_R4), intent(in), optional :: defaultvalue
   \item real (ESMF\_KIND\_R8), intent(in), optional :: defaultvalue
   \item logical, intent(in), optional :: defaultvalue
   \item character (len = *), intent(in), optional :: defaultvalue
   \end{description}
  
   The arguments are:
   \begin{description}
   \item [<object>]
   An {\tt ESMF} object.
   \item [name]
   The name of the Attribute to retrieve.
   \item [attpack]
   A handle to the Attribute package.
   \item [<value>]
   The value of the named Attribute.
   \item [{[<defaultvalue>]}]
   The default value of the named Attribute.
   \item [{[attnestflag]}]
   A flag to determine whether to descend the
   Attribute hierarchy when looking for this Attribute, the default
   is {\tt ESMF\_ATTNEST\_ON}. This flag is documented in section
   \ref{const:attnest}.
   \item [{[isPresent]}]
   A logical flag to tell if this Attribute is present or not.
   \item [{[rc]}]
   Return code; equals {\tt ESMF\_SUCCESS} if there are no errors.
   \end{description}
  
   
%/////////////////////////////////////////////////////////////
 
\mbox{}\hrulefill\ 
 
\subsubsection [ESMF\_AttributeGet] {ESMF\_AttributeGet - Get an Attribute pointing to internal class information from an ESMF\_AttPack}


  
\bigskip{\sf INTERFACE:}
\begin{verbatim}   subroutine ESMF_AttributeGet(<object>, name, attpack, <value>, &
   <defaultvalue>, inputList, attnestflag, isPresent, rc)\end{verbatim}{\em ARGUMENTS:}
\begin{verbatim}   <object>, see below for supported values
   character (len = *), intent(in) :: name
   type(ESMF_AttPack), intent(inout) :: attpack
   <value>, see below for supported values
 -- The following arguments require argument keyword syntax (e.g. rc=rc). --
   <defaultvalue>, see below for supported values
   character (len = *), intent(in), optional :: inputList(:)
   type(ESMF_AttNest_Flag),intent(in), optional :: attnestflag
   logical, intent(out), optional :: isPresent
   integer, intent(out), optional :: rc\end{verbatim}
{\sf DESCRIPTION:\\ }


   Return an Attribute {\tt value} from the <object>, or from an Attribute
   package on the <object>, specified by {\tt attpack}. Internal class information can
   be retrieved by prepending 'ESMF:' to the name of the
   piece of information that is requested. See
   Section~\ref{sec:InternalInfo} for more information
   on this capability.
   A {\tt defaultvalue} argument
   may be given if a return code is not desired when the Attribute is not
   found. See Section~\ref{sec:AttPacks} for a description of Attribute
   packages.
  
   Supported values for <object> are:
   \begin{description}
   \item type(ESMF\_Grid), intent(in) :: grid
   \end{description}
  
   Supported values for <value> are:
   \begin{description}
   \item integer(ESMF\_KIND\_I4), intent(out) :: value
   \item character (len = *), intent(out) :: value
   \end{description}
  
   Supported values for <defaultvalue> are:
   \begin{description}
   \item integer(ESMF\_KIND\_I4), intent(in), optional :: defaultvalue
   \item character (len = *), intent(in), optional :: defaultvalue
   \end{description}
  
   The arguments are:
   \begin{description}
   \item [<object>]
   An {\tt ESMF} object.
   \item [name]
   The name of the Attribute to retrieve.
   \item [attpack]
   A handle to the Attribute package.
   \item [<value argument>]
   The value of the named Attribute.
   \item [{[<defaultvalue argument>]}]
   The default value of the named Attribute.
   \item [{[inputList]}]
   A list of the input parameters required to retrieve internal info.
   \item [{[attnestflag]}]
   A flag to determine whether to descend the
   Attribute hierarchy when looking for this Attribute, the default
   is {\tt ESMF\_ATTNEST\_ON}. This flag is documented in section
   \ref{const:attnest}.
   \item [{[isPresent]}]
   A logical flag to tell if this Attribute is present or not.
   \item [{[rc]}]
   Return code; equals {\tt ESMF\_SUCCESS} if there are no errors.
   \end{description}
  
   
%/////////////////////////////////////////////////////////////
 
\mbox{}\hrulefill\ 
 
\subsubsection [ESMF\_AttributeGet] {ESMF\_AttributeGet - Get an Attribute from an ESMF\_AttPack}


  
\bigskip{\sf INTERFACE:}
\begin{verbatim}   subroutine ESMF_AttributeGet(<object>, name, attpack, <valueList>, &
   <defaultvalueList>, attnestflag, itemCount, &
   isPresent, rc)\end{verbatim}{\em ARGUMENTS:}
\begin{verbatim}   <object>, see below for supported values
   character (len = *), intent(in) :: name
   type(ESMF_AttPack), intent(inout) :: attpack
   <valueList>, see below for supported values
 -- The following arguments require argument keyword syntax (e.g. rc=rc). --
   <defaultvalueList>, see below for supported values
   type(ESMF_AttNest_Flag),intent(in), optional :: attnestflag
   integer, intent(out), optional :: itemCount
   logical, intent(out), optional :: isPresent
   integer, intent(out), optional :: rc\end{verbatim}
{\sf DESCRIPTION:\\ }


   Return an Attribute {\tt valueList} from the <object>, or from an
   Attribute package on the <object>, specified by {\tt attpack}. Internal
   information can also be retrieved from Grid objects by prepending 'ESMF:'
   to the name of the piece of information that is requested. See
   Section~\ref{sec:InternalInfo} for more information
   on which pieces of Grid data can be retrieved through this interface.
   A {\tt defaultvalueList} list argument may be given if
   a return code is not desired when the Attribute is not found.
   See Section~\ref{sec:AttPacks} for a description of Attribute packages.
  
   Supported values for <object> are:
   \begin{description}
   \item type(ESMF\_Array), intent(in) :: array
   \item type(ESMF\_ArrayBundle), intent(in) :: arraybundle
   \item type(ESMF\_CplComp), intent(in) :: comp
   \item type(ESMF\_GridComp), intent(in) :: comp
   \item type(ESMF\_SciComp), intent(in) :: comp
   \item type(ESMF\_DistGrid), intent(in) :: distgrid
   \item type(ESMF\_Field), intent(in) :: field
   \item type(ESMF\_FieldBundle), intent(in) :: fieldbundle
   \item type(ESMF\_Grid), intent(in) :: grid
   \item type(ESMF\_State), intent(in) :: state
   \end{description}
  
   Supported values for <valueList> are:
   \begin{description}
   \item integer(ESMF\_KIND\_I4), intent(out) :: valueList(:)
   \item integer(ESMF\_KIND\_I8), intent(out) :: valueList(:)
   \item real (ESMF\_KIND\_R4), intent(out) :: valueList(:)
   \item real (ESMF\_KIND\_R8), intent(out) :: valueList(:)
   \item logical, intent(out) :: valueList(:)
   \item character (len = *), intent(out) :: valueList(:)
   \end{description}
  
   Supported values for <defaultvalueList> are:
   \begin{description}
   \item integer(ESMF\_KIND\_I4), intent(in), optional :: defaultvalueList(:)
   \item integer(ESMF\_KIND\_I8), intent(in), optional :: defaultvalueList(:)
   \item real (ESMF\_KIND\_R4), intent(in), optional :: defaultvalueList(:)
   \item real (ESMF\_KIND\_R8), intent(in), optional :: defaultvalueList(:)
   \item logical, intent(in), optional :: defaultvalueList(:)
   \item character (len = *), intent(in), optional :: defaultvalueList(:)
   \end{description}
  
   The arguments are:
   \begin{description}
   \item [<object>]
   An {\tt ESMF} object.
   \item [name]
   The name of the Attribute to retrieve.
   \item [attpack]
   A handle to the Attribute package.
   \item [<valueList>]
   The valueList of the named Attribute.
   \item [{[<defaultvalueList>]}]
   The default value list of the named Attribute.
   \item [{[attnestflag]}]
   A flag to determine whether to descend the
   Attribute hierarchy when looking for this Attribute, the default
   is {\tt ESMF\_ATTNEST\_ON}. This flag is documented in section
   \ref{const:attnest}.
   \item [{[itemCount]}]
   The number of items in a multi-valued Attribute.
   \item [{[isPresent]}]
   A logical flag to tell if this Attribute is present or not.
   \item [{[rc]}]
   Return code; equals {\tt ESMF\_SUCCESS} if there are no errors.
   \end{description}
  
   
%/////////////////////////////////////////////////////////////
 
\mbox{}\hrulefill\ 
 
\subsubsection [ESMF\_AttributeGet] {ESMF\_AttributeGet - Get an Attribute pointing to internal class information from an ESMF\_AttPack}


  
\bigskip{\sf INTERFACE:}
\begin{verbatim}   subroutine ESMF_AttributeGet(<object>, name, attpack, <valueList>, &
   <defaultvalueList>, inputList, attnestflag, &
   itemCount, isPresent, rc)\end{verbatim}{\em ARGUMENTS:}
\begin{verbatim}   <object>, see below for supported values
   character (len = *), intent(in) :: name
   type(ESMF_AttPack), intent(inout) :: attpack
   <valueList>, see below for supported values
 -- The following arguments require argument keyword syntax (e.g. rc=rc). --
   <defaultvalueList>, see below for supported values
   character (len = *), intent(in), optional :: inputList(:)
   type(ESMF_AttNest_Flag),intent(in), optional :: attnestflag
   integer, intent(out), optional :: itemCount
   logical, intent(out), optional :: isPresent
   integer, intent(out), optional :: rc\end{verbatim}
{\sf DESCRIPTION:\\ }


   Return an Attribute {\tt valueList} from the <object>, or from an
   Attribute package on the <object>, specified by {\tt attpack}. Internal class
   information can be retrieved by prepending 'ESMF:'
   to the name of the piece of information that is requested. See
   Section~\ref{sec:InternalInfo} for more information
   on this capability.
   A {\tt defaultvalueList} list argument may be given if
   a return code is not desired when the Attribute is not found.
   See Section~\ref{sec:AttPacks} for a description of Attribute packages.
  
   Supported values for <object> are:
   \begin{description}
   \item type(ESMF\_Grid), intent(in) :: grid
   \end{description}
  
   Supported values for <valueList> are:
   \begin{description}
   \item integer(ESMF\_KIND\_I4), intent(out) :: valueList(:)
   \item real (ESMF\_KIND\_R8), intent(out) :: valueList(:)
   \item logical, intent(out) :: valueList(:)
   \end{description}
  
   Supported values for <defaultvalueList> are:
   \begin{description}
   \item integer(ESMF\_KIND\_I4), intent(in), optional :: defaultvalueList(:)
   \item real (ESMF\_KIND\_R8), intent(in), optional :: defaultvalueList(:)
   \item logical, intent(in), optional :: defaultvalueList(:)
   \end{description}
  
   The arguments are:
   \begin{description}
   \item [<object>]
   An {\tt ESMF} object.
   \item [name]
   The name of the Attribute to retrieve.
   \item [attpack]
   A handle to the Attribute package.
   \item [<valueList>]
   The valueList of the named Attribute.
   \item [{[<defaultvalueList>]}]
   The default value list of the named Attribute.
   \item [{[inputList]}]
   A list of the input parameters required to retrieve internal info.
   \item [{[attnestflag]}]
   A flag to determine whether to descend the
   Attribute hierarchy when looking for this Attribute, the default
   is {\tt ESMF\_ATTNEST\_ON}. This flag is documented in section
   \ref{const:attnest}.
   \item [{[itemCount]}]
   The number of items in a multi-valued Attribute.
   \item [{[isPresent]}]
   A logical flag to tell if this Attribute is present or not.
   \item [{[rc]}]
   Return code; equals {\tt ESMF\_SUCCESS} if there are no errors.
   \end{description}
  
   
%/////////////////////////////////////////////////////////////
 
\mbox{}\hrulefill\ 
 
\subsubsection [ESMF\_AttributeGet] {ESMF\_AttributeGet - Get an Attribute}


  
\bigskip{\sf INTERFACE:}
\begin{verbatim}   subroutine ESMF_AttributeGet(<object>, name, <value>, <defaultvalue>, &
   convention, purpose, attPackInstanceName, attnestflag, isPresent, rc)\end{verbatim}{\em ARGUMENTS:}
\begin{verbatim}   <object>, see below for supported values
   character (len = *), intent(in) :: name
   <value>, see below for supported values
   <defaultvalue>, see below for supported values
   character (len = *), intent(in), optional :: convention
   character (len = *), intent(in), optional :: purpose
   character (len = *), intent(in), optional :: attPackInstanceName
   type(ESMF_AttNest_Flag),intent(in), optional :: attnestflag
   logical, intent(out), optional :: isPresent
   integer, intent(out), optional :: rc\end{verbatim}
{\sf DESCRIPTION:\\ }


   Return an Attribute {\tt value} from the <object>, or from an Attribute
   package on the <object>, specified by {\tt convention},
   {\tt purpose}, and {\tt attPackInstanceName}. Internal information can also
   be retrieved from Grid objects by prepending 'ESMF:' to the name of the
   piece of information that is requested. See
   Section~\ref{sec:InternalInfo} for more information
   on which pieces of Grid data can be retrieved through this interface.
   A {\tt defaultvalue} argument
   may be given if a return code is not desired when the Attribute is not
   found. See Section~\ref{sec:AttPacks} for a description of Attribute
   packages.
  
   Supported values for <object> are:
   \begin{description}
   \item type(ESMF\_Array), intent(in) :: array
   \item type(ESMF\_ArrayBundle), intent(in) :: arraybundle
   \item type(ESMF\_CplComp), intent(in) :: comp
   \item type(ESMF\_GridComp), intent(in) :: comp
   \item type(ESMF\_SciComp), intent(in) :: comp
   \item type(ESMF\_DistGrid), intent(in) :: distgrid
   \item type(ESMF\_Field), intent(in) :: field
   \item type(ESMF\_FieldBundle), intent(in) :: fieldbundle
   \item type(ESMF\_Grid), intent(in) :: grid
   \item type(ESMF\_State), intent(in) :: state
   \end{description}
  
   Supported values for <value> are:
   \begin{description}
   \item integer(ESMF\_KIND\_I4), intent(out) :: value
   \item integer(ESMF\_KIND\_I8), intent(out) :: value
   \item real (ESMF\_KIND\_R4), intent(out) :: value
   \item real (ESMF\_KIND\_R8), intent(out) :: value
   \item logical, intent(out) :: value
   \item character (len = *), intent(out) :: value
   \end{description}
  
   Supported values for <defaultvalue> are:
   \begin{description}
   \item integer(ESMF\_KIND\_I4), intent(in), optional :: defaultvalue
   \item integer(ESMF\_KIND\_I8), intent(in), optional :: defaultvalue
   \item real (ESMF\_KIND\_R4), intent(in), optional :: defaultvalue
   \item real (ESMF\_KIND\_R8), intent(in), optional :: defaultvalue
   \item logical, intent(in), optional :: defaultvalue
   \item character (len = *), intent(in), optional :: defaultvalue
   \end{description}
  
   The arguments are:
   \begin{description}
   \item [<object>]
   An {\tt ESMF} object.
   \item [name]
   The name of the Attribute to retrieve.
   \item [<value>]
   The value of the named Attribute.
   \item [{[<defaultvalue>]}]
   The default value of the named Attribute.
   \item [{[convention]}]
   The convention of the Attribute package.
   \item [{[purpose]}]
   The purpose of the Attribute package.
   \item [{[attPackInstanceName]}]
   The name of an Attribute package instance, specifying which one
   of multiple Attribute package instances of the same convention
   and purpose, within a nest. If not specified, defaults to the
   first instance.
   \item [{[attnestflag]}]
   A flag to determine whether to descend the
   Attribute hierarchy when looking for this Attribute, the default
   is {\tt ESMF\_ATTNEST\_ON}. This flag is documented in section
   \ref{const:attnest}.
   \item [{[isPresent]}]
   A logical flag to tell if this Attribute is present or not.
   \item [{[rc]}]
   Return code; equals {\tt ESMF\_SUCCESS} if there are no errors.
   \end{description}
  
   
%/////////////////////////////////////////////////////////////
 
\mbox{}\hrulefill\ 
 
\subsubsection [ESMF\_AttributeGet] {ESMF\_AttributeGet - Get an Attribute pointing to internal class information}


  
\bigskip{\sf INTERFACE:}
\begin{verbatim}   subroutine ESMF_AttributeGet(<object>, name, <value>, <defaultvalue>, &
   inputList, convention, purpose, attPackInstanceName, attnestflag, &
   isPresent, rc)\end{verbatim}{\em ARGUMENTS:}
\begin{verbatim}   <object>, see below for supported values
   character (len = *), intent(in) :: name
   <value>, see below for supported values
   <defaultvalue>, see below for supported values
   character (len = *), intent(in), optional :: inputList(:)
   character (len = *), intent(in), optional :: convention
   character (len = *), intent(in), optional :: purpose
   character (len = *), intent(in), optional :: attPackInstanceName
   type(ESMF_AttNest_Flag),intent(in), optional :: attnestflag
   logical, intent(out), optional :: isPresent
   integer, intent(out), optional :: rc\end{verbatim}
{\sf DESCRIPTION:\\ }


   Return an Attribute {\tt value} from the <object>, or from an Attribute
   package on the <object>, specified by {\tt convention},
   {\tt purpose}, and {\tt attPackInstanceName}. Internal class information can
   be retrieved by prepending 'ESMF:' to the name of the
   piece of information that is requested. See
   Section~\ref{sec:InternalInfo} for more information
   on this capability.
   A {\tt defaultvalue} argument
   may be given if a return code is not desired when the Attribute is not
   found. See Section~\ref{sec:AttPacks} for a description of Attribute
   packages.
  
   Supported values for <object> are:
   \begin{description}
   \item type(ESMF\_Grid), intent(in) :: grid
   \end{description}
  
   Supported values for <value> are:
   \begin{description}
   \item integer(ESMF\_KIND\_I4), intent(out) :: value
   \item character (len = *), intent(out) :: value
   \end{description}
  
   Supported values for <defaultvalue> are:
   \begin{description}
   \item integer(ESMF\_KIND\_I4), intent(in), optional :: defaultvalue
   \item character (len = *), intent(in), optional :: defaultvalue
   \end{description}
  
   The arguments are:
   \begin{description}
   \item [<object>]
   An {\tt ESMF} object.
   \item [name]
   The name of the Attribute to retrieve.
   \item [<value argument>]
   The value of the named Attribute.
   \item [{[<defaultvalue argument>]}]
   The default value of the named Attribute.
   \item [{[inputList]}]
   A list of the input parameters required to retrieve internal info.
   \item [{[convention]}]
   The convention of the Attribute package.
   \item [{[purpose]}]
   The purpose of the Attribute package.
   \item [{[attPackInstanceName]}]
   The name of an Attribute package instance, specifying which one
   of multiple Attribute package instances of the same convention
   and purpose, within a nest. If not specified, defaults to the
   first instance.
   \item [{[attnestflag]}]
   A flag to determine whether to descend the
   Attribute hierarchy when looking for this Attribute, the default
   is {\tt ESMF\_ATTNEST\_ON}. This flag is documented in section
   \ref{const:attnest}.
   \item [{[isPresent]}]
   A logical flag to tell if this Attribute is present or not.
   \item [{[rc]}]
   Return code; equals {\tt ESMF\_SUCCESS} if there are no errors.
   \end{description}
  
   
%/////////////////////////////////////////////////////////////
 
\mbox{}\hrulefill\ 
 
\subsubsection [ESMF\_AttributeGet] {ESMF\_AttributeGet - Get an Attribute}


  
\bigskip{\sf INTERFACE:}
\begin{verbatim}   subroutine ESMF_AttributeGet(<object>, name, <valueList>, &
   <defaultvalueList>, convention, purpose, attPackInstanceName, &
   attnestflag, itemCount, isPresent, rc)\end{verbatim}{\em ARGUMENTS:}
\begin{verbatim}   <object>, see below for supported values
   character (len = *), intent(in) :: name
   <valueList>, see below for supported values
   <defaultvalueList>, see below for supported values
   character (len = *), intent(in), optional :: convention
   character (len = *), intent(in), optional :: purpose
   character (len = *), intent(in), optional :: attPackInstanceName
   type(ESMF_AttNest_Flag),intent(in), optional :: attnestflag
   integer, intent(out), optional :: itemCount
   logical, intent(out), optional :: isPresent
   integer, intent(out), optional :: rc\end{verbatim}
{\sf DESCRIPTION:\\ }


   Return an Attribute {\tt valueList} from the <object>, or from an
   Attribute package on the <object>, specified by {\tt convention},
   {\tt purpose}, and {\tt attPackInstanceName}. Internal
   information can also be retrieved from Grid objects by prepending 'ESMF:'
   to the name of the piece of information that is requested. See
   Section~\ref{sec:InternalInfo} for more information
   on which pieces of Grid data can be retrieved through this interface.
   A {\tt defaultvalueList} list argument may be given if
   a return code is not desired when the Attribute is not found.
   See Section~\ref{sec:AttPacks} for a description of Attribute packages.
  
   Supported values for <object> are:
   \begin{description}
   \item type(ESMF\_Array), intent(in) :: array
   \item type(ESMF\_ArrayBundle), intent(in) :: arraybundle
   \item type(ESMF\_CplComp), intent(in) :: comp
   \item type(ESMF\_GridComp), intent(in) :: comp
   \item type(ESMF\_SciComp), intent(in) :: comp
   \item type(ESMF\_DistGrid), intent(in) :: distgrid
   \item type(ESMF\_Field), intent(in) :: field
   \item type(ESMF\_FieldBundle), intent(in) :: fieldbundle
   \item type(ESMF\_Grid), intent(in) :: grid
   \item type(ESMF\_State), intent(in) :: state
   \end{description}
  
   Supported values for <valueList> are:
   \begin{description}
   \item integer(ESMF\_KIND\_I4), intent(out) :: valueList(:)
   \item integer(ESMF\_KIND\_I8), intent(out) :: valueList(:)
   \item real (ESMF\_KIND\_R4), intent(out) :: valueList(:)
   \item real (ESMF\_KIND\_R8), intent(out) :: valueList(:)
   \item logical, intent(out) :: valueList(:)
   \item character (len = *), intent(out) :: valueList(:)
   \end{description}
  
   Supported values for <defaultvalueList> are:
   \begin{description}
   \item integer(ESMF\_KIND\_I4), intent(in), optional :: defaultvalueList(:)
   \item integer(ESMF\_KIND\_I8), intent(in), optional :: defaultvalueList(:)
   \item real (ESMF\_KIND\_R4), intent(in), optional :: defaultvalueList(:)
   \item real (ESMF\_KIND\_R8), intent(in), optional :: defaultvalueList(:)
   \item logical, intent(in), optional :: defaultvalueList(:)
   \item character (len = *), intent(in), optional :: defaultvalueList(:)
   \end{description}
  
   The arguments are:
   \begin{description}
   \item [<object>]
   An {\tt ESMF} object.
   \item [name]
   The name of the Attribute to retrieve.
   \item [<valueList>]
   The valueList of the named Attribute.
   \item [{[<defaultvalueList>]}]
   The default value list of the named Attribute.
   \item [{[convention]}]
   The convention of the Attribute package.
   \item [{[purpose]}]
   The purpose of the Attribute package.
   \item [{[attPackInstanceName]}]
   The name of an Attribute package instance, specifying which one
   of multiple Attribute package instances of the same convention
   and purpose, within a nest. If not specified, defaults to the
   first instance.
   \item [{[attnestflag]}]
   A flag to determine whether to descend the
   Attribute hierarchy when looking for this Attribute, the default
   is {\tt ESMF\_ATTNEST\_ON}. This flag is documented in section
   \ref{const:attnest}.
   \item [{[itemCount]}]
   The number of items in a multi-valued Attribute.
   \item [{[isPresent]}]
   A logical flag to tell if this Attribute is present or not.
   \item [{[rc]}]
   Return code; equals {\tt ESMF\_SUCCESS} if there are no errors.
   \end{description}
  
   
%/////////////////////////////////////////////////////////////
 
\mbox{}\hrulefill\ 
 
\subsubsection [ESMF\_AttributeGet] {ESMF\_AttributeGet - Get an Attribute pointing to internal class information}


  
\bigskip{\sf INTERFACE:}
\begin{verbatim}   subroutine ESMF_AttributeGet(<object>, name, <valueList>, &
   <defaultvalueList>, inputList, convention, purpose, attPackInstanceName, &
   attnestflag, itemCount, isPresent, rc)\end{verbatim}{\em ARGUMENTS:}
\begin{verbatim}   <object>, see below for supported values
   character (len = *), intent(in) :: name
   <valueList>, see below for supported values
   <defaultvalueList>, see below for supported values
   character (len = *), intent(in), optional :: inputList(:)
   character (len = *), intent(in), optional :: convention
   character (len = *), intent(in), optional :: purpose
   character (len = *), intent(in), optional :: attPackInstanceName
   type(ESMF_AttNest_Flag),intent(in), optional :: attnestflag
   integer, intent(out), optional :: itemCount
   logical, intent(out), optional :: isPresent
   integer, intent(out), optional :: rc\end{verbatim}
{\sf DESCRIPTION:\\ }


   Return an Attribute {\tt valueList} from the <object>, or from an
   Attribute package on the <object>, specified by {\tt convention},
   {\tt purpose}, and {\tt attPackInstanceName}. Internal class
   information can be retrieved by prepending 'ESMF:'
   to the name of the piece of information that is requested. See
   Section~\ref{sec:InternalInfo} for more information
   on this capability.
   A {\tt defaultvalueList} list argument may be given if
   a return code is not desired when the Attribute is not found.
   See Section~\ref{sec:AttPacks} for a description of Attribute packages.
  
   Supported values for <object> are:
   \begin{description}
   \item type(ESMF\_Grid), intent(in) :: grid
   \end{description}
  
   Supported values for <valueList> are:
   \begin{description}
   \item integer(ESMF\_KIND\_I4), intent(out) :: valueList(:)
   \item real (ESMF\_KIND\_R8), intent(out) :: valueList(:)
   \item logical, intent(out) :: valueList(:)
   \end{description}
  
   Supported values for <defaultvalueList> are:
   \begin{description}
   \item integer(ESMF\_KIND\_I4), intent(in), optional :: defaultvalueList(:)
   \item real (ESMF\_KIND\_R8), intent(in), optional :: defaultvalueList(:)
   \item logical, intent(in), optional :: defaultvalueList(:)
   \end{description}
  
   The arguments are:
   \begin{description}
   \item [<object>]
   An {\tt ESMF} object.
   \item [name]
   The name of the Attribute to retrieve.
   \item [<valueList>]
   The valueList of the named Attribute.
   \item [{[<defaultvalueList>]}]
   The default value list of the named Attribute.
   \item [{[inputList]}]
   A list of the input parameters required to retrieve internal info.
   \item [{[convention]}]
   The convention of the Attribute package.
   \item [{[purpose]}]
   The purpose of the Attribute package.
   \item [{[attPackInstanceName]}]
   The name of an Attribute package instance, specifying which one
   of multiple Attribute package instances of the same convention
   and purpose, within a nest. If not specified, defaults to the
   first instance.
   \item [{[attnestflag]}]
   A flag to determine whether to descend the
   Attribute hierarchy when looking for this Attribute, the default
   is {\tt ESMF\_ATTNEST\_ON}. This flag is documented in section
   \ref{const:attnest}.
   \item [{[itemCount]}]
   The number of items in a multi-valued Attribute.
   \item [{[isPresent]}]
   A logical flag to tell if this Attribute is present or not.
   \item [{[rc]}]
   Return code; equals {\tt ESMF\_SUCCESS} if there are no errors.
   \end{description}
  
   
%/////////////////////////////////////////////////////////////
 
\mbox{}\hrulefill\ 
 
\subsubsection [ESMF\_AttributeGet] {ESMF\_AttributeGet - Get the Attribute count from an ESMF\_AttPack}


  
\bigskip{\sf INTERFACE:}
\begin{verbatim}   ! Private name; call using ESMF_AttributeGet()
   subroutine ESMF_AttributeGetCount(<object>, attpack, count, &
   attcountflag, attnestflag, rc)\end{verbatim}{\em ARGUMENTS:}
\begin{verbatim}   <object>, see below for supported values
   type(ESMF_AttPack), intent(inout) :: attpack
   integer, intent(out) :: count
   type(ESMF_AttGetCountFlag), intent(in), optional :: attcountflag
   type(ESMF_AttNest_Flag), intent(in), optional :: attnestflag
   integer, intent(out), optional :: rc\end{verbatim}
{\sf DESCRIPTION:\\ }


   Return the Attribute count for <object>.
  
   Supported values for <object> are:
   \begin{description}
   \item type(ESMF\_Array), intent(in) :: array
   \item type(ESMF\_ArrayBundle), intent(in) :: arraybundle
   \item type(ESMF\_CplComp), intent(in) :: comp
   \item type(ESMF\_GridComp), intent(in) :: comp
   \item type(ESMF\_SciComp), intent(in) :: comp
   \item type(ESMF\_DistGrid), intent(in) :: distgrid
   \item type(ESMF\_Field), intent(in) :: field
   \item type(ESMF\_FieldBundle), intent(in) :: fieldbundle
   \item type(ESMF\_Grid), intent(in) :: grid
   \item type(ESMF\_State), intent(in) :: state
   \end{description}
  
   The arguments are:
   \begin{description}
   \item [<object>]
   An {\tt ESMF} object.
   \item [attpack]
   A handle to the Attribute package.
   \item [count]
   The number of all existing Attributes of the type designated in the
   {\it attcountflag}, not just Attribute that have been set.
   \item [{[attcountflag]}]
   The flag to specify which attribute count to return, the
   default is ESMF\_ATTGETCOUNT\_ATTRIBUTE. This flag is documented
   in section \ref{const:attgetcount}.
   \item [{[attnestflag]}]
   A flag to determine whether to descend the
   Attribute hierarchy when looking for this Attribute, the default
   is {\tt ESMF\_ATTNEST\_ON}. This flag is documented in section
   \ref{const:attnest}.
   \item [{[rc]}]
   Return code; equals {\tt ESMF\_SUCCESS} if there are no errors.
   \end{description}
  
  EOP!------------------------------------------------------------------------------ 
%/////////////////////////////////////////////////////////////
 
\mbox{}\hrulefill\ 
 
\subsubsection [ESMF\_AttributeGet] {ESMF\_AttributeGet - Get the Attribute count}


  
\bigskip{\sf INTERFACE:}
\begin{verbatim}   ! Private name; call using ESMF_AttributeGet()
   subroutine ESMF_AttributeGetCount(<object>, count, &
   convention, purpose, attPackInstanceName, &
   attcountflag, attnestflag, rc)\end{verbatim}{\em ARGUMENTS:}
\begin{verbatim}   <object>, see below for supported values
   integer, intent(out) :: count
   character (len=*), intent(in), optional :: convention
   character (len=*), intent(in), optional :: purpose
   character (len=*), intent(in), optional :: attPackInstanceName
   type(ESMF_AttGetCountFlag), intent(in), optional :: attcountflag
   type(ESMF_AttNest_Flag), intent(in), optional :: attnestflag
   integer, intent(out), optional :: rc\end{verbatim}
{\sf DESCRIPTION:\\ }


   Return the Attribute count for <object>.
  
   Supported values for <object> are:
   \begin{description}
   \item type(ESMF\_Array), intent(in) :: array
   \item type(ESMF\_ArrayBundle), intent(in) :: arraybundle
   \item type(ESMF\_CplComp), intent(in) :: comp
   \item type(ESMF\_GridComp), intent(in) :: comp
   \item type(ESMF\_SciComp), intent(in) :: comp
   \item type(ESMF\_DistGrid), intent(in) :: distgrid
   \item type(ESMF\_Field), intent(in) :: field
   \item type(ESMF\_FieldBundle), intent(in) :: fieldbundle
   \item type(ESMF\_Grid), intent(in) :: grid
   \item type(ESMF\_State), intent(in) :: state
   \end{description}
  
   The arguments are:
   \begin{description}
   \item [<object>]
   An {\tt ESMF} object.
   \item [count]
   The number of all existing Attributes of the type designated in the
   {\it attcountflag}, not just Attribute that have been set.
   \item [{[convention]}]
   The convention of the Attribute package.
   \item [{[purpose]}]
   The purpose of the Attribute package.
   \item [{[attPackInstanceName]}]
   The name of an Attribute package instance, specifying which one
   of multiple Attribute package instances of the same convention
   and purpose, within a nest. If not specified, defaults to the
   first instance.
   \item [{[attcountflag]}]
   The flag to specify which attribute count to return, the
   default is ESMF\_ATTGETCOUNT\_ATTRIBUTE. This flag is documented
   in section \ref{const:attgetcount}.
   \item [{[attnestflag]}]
   A flag to determine whether to descend the
   Attribute hierarchy when looking for this Attribute, the default
   is {\tt ESMF\_ATTNEST\_ON}. This flag is documented in section
   \ref{const:attnest}.
   \item [{[rc]}]
   Return code; equals {\tt ESMF\_SUCCESS} if there are no errors.
   \end{description}
   
%/////////////////////////////////////////////////////////////
 
\mbox{}\hrulefill\ 
 
\subsubsection [ESMF\_AttributeGet] {ESMF\_AttributeGet - Get Attribute info by name from an ESMF\_AttPack}


  
\bigskip{\sf INTERFACE:}
\begin{verbatim}   ! Private name; call using ESMF_AttributeGet()
   subroutine ESMF_AttributeGetInfoByNamAP(<object>, name, attpack, &
   attnestflag, typekind, itemCount, isPresent, rc)\end{verbatim}{\em ARGUMENTS:}
\begin{verbatim}   <object>, see below for supported values
   character (len = *), intent(in) :: name
   type(ESMF_AttPack), intent(inout) :: attpack
 -- The following arguments require argument keyword syntax (e.g. rc=rc). --
   type(ESMF_AttNest_Flag), intent(in), optional :: attnestflag
   type(ESMF_TypeKind_Flag), intent(out), optional :: typekind
   integer, intent(out), optional :: itemCount
   logical, intent(out), optional :: isPresent
   integer, intent(out), optional :: rc\end{verbatim}
{\sf DESCRIPTION:\\ }


   Return information associated with an Attribute in an Attribute package,
   including {\tt typekind} and {\tt itemCount}.
  
   Supported values for <object> are:
   \begin{description}
   \item type(ESMF\_Array), intent(in) :: array
   \item type(ESMF\_ArrayBundle), intent(in) :: arraybundle
   \item type(ESMF\_CplComp), intent(in) :: comp
   \item type(ESMF\_GridComp), intent(in) :: comp
   \item type(ESMF\_SciComp), intent(in) :: comp
   \item type(ESMF\_DistGrid), intent(in) :: distgrid
   \item type(ESMF\_Field), intent(in) :: field
   \item type(ESMF\_FieldBundle), intent(in) :: fieldbundle
   \item type(ESMF\_Grid), intent(in) :: grid
   \item type(ESMF\_State), intent(in) :: state
   \end{description}
  
   The arguments are:
   \begin{description}
   \item [<object>]
   An {\tt ESMF} object.
   \item [name]
   The name of the Attribute to query.
   \item [attpack]
   A handle to the Attribute package.
   \item [{[attnestflag]}]
   A flag to determine whether to descend the
   Attribute hierarchy when looking for this Attribute, the default
   is {\tt ESMF\_ATTNEST\_ON}. This flag is documented in section
   \ref{const:attnest}.
   \item [{[typekind]}]
   The typekind of the Attribute. This flag is documented in section
   \ref{const:typekind}.
   \item [{[itemCount]}]
   The number of items in this Attribute.
   \item [{[isPresent]}]
   A logical flag to tell if this Attribute is present or not.
   \item [{[rc]}]
   Return code; equals {\tt ESMF\_SUCCESS} if there are no errors.
   \end{description}
  
   
%/////////////////////////////////////////////////////////////
 
\mbox{}\hrulefill\ 
 
\subsubsection [ESMF\_AttributeGet] {ESMF\_AttributeGet - Get Attribute info by name}


  
\bigskip{\sf INTERFACE:}
\begin{verbatim}   ! Private name; call using ESMF_AttributeGet()
   subroutine ESMF_AttributeGetInfoByNam(<object>, name, &
   convention, purpose, attPackInstanceName, &
   attnestflag, typekind, itemCount, isPresent, rc)\end{verbatim}{\em ARGUMENTS:}
\begin{verbatim}   <object>, see below for supported values
   character (len = *), intent(in) :: name
 -- The following arguments require argument keyword syntax (e.g. rc=rc). --
   character (len=*), intent(in), optional :: convention
   character (len=*), intent(in), optional :: purpose
   character (len=*), intent(in), optional :: attPackInstanceName
   type(ESMF_AttNest_Flag), intent(in), optional :: attnestflag
   type(ESMF_TypeKind_Flag), intent(out), optional :: typekind
   integer, intent(out), optional :: itemCount
   logical, intent(out), optional :: isPresent
   integer, intent(out), optional :: rc\end{verbatim}
{\sf DESCRIPTION:\\ }


   Return information associated with the named Attribute,
   including {\tt typekind} and {\tt itemCount}.
  
   Supported values for <object> are:
   \begin{description}
   \item type(ESMF\_Array), intent(in) :: array
   \item type(ESMF\_ArrayBundle), intent(in) :: arraybundle
   \item type(ESMF\_CplComp), intent(in) :: comp
   \item type(ESMF\_GridComp), intent(in) :: comp
   \item type(ESMF\_SciComp), intent(in) :: comp
   \item type(ESMF\_DistGrid), intent(in) :: distgrid
   \item type(ESMF\_Field), intent(in) :: field
   \item type(ESMF\_FieldBundle), intent(in) :: fieldbundle
   \item type(ESMF\_Grid), intent(in) :: grid
   \item type(ESMF\_State), intent(in) :: state
   \end{description}
  
   The arguments are:
   \begin{description}
   \item [<object>]
   An {\tt ESMF} object.
   \item [name]
   The name of the Attribute to query.
   \item [{[convention]}]
   The convention of the Attribute package.
   \item [{[purpose]}]
   The purpose of the Attribute package.
   \item [{[attPackInstanceName]}]
   The name of an Attribute package instance, specifying which one
   of multiple Attribute package instances of the same convention
   and purpose, within a nest. If not specified, defaults to the
   first instance.
   \item [{[attnestflag]}]
   A flag to determine whether to descend the
   Attribute hierarchy when looking for this Attribute, the default
   is {\tt ESMF\_ATTNEST\_ON}. This flag is documented in section
   \ref{const:attnest}.
   \item [{[typekind]}]
   The typekind of the Attribute. This flag is documented in section
   \ref{const:typekind}.
   \item [{[itemCount]}]
   The number of items in this Attribute.
   \item [{[isPresent]}]
   A logical flag to tell if this Attribute is present or not.
   \item [{[rc]}]
   Return code; equals {\tt ESMF\_SUCCESS} if there are no errors.
   \end{description}
  
   
%/////////////////////////////////////////////////////////////
 
\mbox{}\hrulefill\ 
 
\subsubsection [ESMF\_AttributeGet] {ESMF\_AttributeGet - Get Attribute info by index number from an ESMF\_AttPack}


  
\bigskip{\sf INTERFACE:}
\begin{verbatim}   ! Private name; call using ESMF_AttributeGet()
   subroutine ESMF_AttributeGetInfoByNum(<object>, attributeIndex, &
   name, attpack, attnestflag, typekind, itemcount, isPresent, rc)\end{verbatim}{\em ARGUMENTS:}
\begin{verbatim}   <object>, see below for supported values
   integer, intent(in) :: attributeIndex
   character (len = *), intent(out) :: name
   type(ESMF_AttPack), intent(inout) :: attpack
   type(ESMF_AttNest_Flag), intent(in), optional :: attnestflag
   type(ESMF_TypeKind_Flag), intent(out), optional :: typekind
   integer, intent(out), optional :: itemCount
   logical, intent(out), optional :: isPresent
   integer, intent(out), optional :: rc\end{verbatim}
{\sf DESCRIPTION:\\ }


   Returns information associated with the indexed Attribute,
   including {\tt name}, {\tt typekind} and {\tt itemCount}. Keep in
   mind that these indices start from 1, as expected in a Fortran API.
  
   Supported values for <object> are:
   \begin{description}
   \item type(ESMF\_Array), intent(in) :: array
   \item type(ESMF\_ArrayBundle), intent(in) :: arraybundle
   \item type(ESMF\_CplComp), intent(in) :: comp
   \item type(ESMF\_GridComp), intent(in) :: comp
   \item type(ESMF\_SciComp), intent(in) :: comp
   \item type(ESMF\_DistGrid), intent(in) :: distgrid
   \item type(ESMF\_Field), intent(in) :: field
   \item type(ESMF\_FieldBundle), intent(in) :: fieldbundle
   \item type(ESMF\_Grid), intent(in) :: grid
   \item type(ESMF\_State), intent(in) :: state
   \end{description}
  
   The arguments are:
   \begin{description}
   \item [<object>]
   An {\tt ESMF} object.
   \item [attributeIndex]
   The index number of the Attribute to query.
   \item [name]
   The name of the Attribute.
   \item [attpack]
   A handle to the Attribute package.
   \item [{[attnestflag]}]
   A flag to determine whether to descend the
   Attribute hierarchy when looking for this Attribute, the default
   is {\tt ESMF\_ATTNEST\_ON}. This flag is documented in section
   \ref{const:attnest}.
   \item [{[typekind]}]
   The typekind of the Attribute. This flag is documented in section
   \ref{const:typekind}.
   \item [{[itemCount]}]
   The number of items in this Attribute.
   \item [{[isPresent]}]
   A logical flag to tell if this Attribute is present or not.
   \item [{[rc]}]
   Return code; equals {\tt ESMF\_SUCCESS} if there are no errors.
   \end{description}
  
   
%/////////////////////////////////////////////////////////////
 
\mbox{}\hrulefill\ 
 
\subsubsection [ESMF\_AttributeGet] {ESMF\_AttributeGet - Get Attribute info by index number}


  
\bigskip{\sf INTERFACE:}
\begin{verbatim}   ! Private name; call using ESMF_AttributeGet()
   subroutine ESMF_AttributeGetInfoByNum(<object>, attributeIndex, &
   name, convention, purpose, attPackInstanceName, attnestflag, &
   typekind, itemcount, isPresent, rc)\end{verbatim}{\em ARGUMENTS:}
\begin{verbatim}   <object>, see below for supported values
   integer, intent(in) :: attributeIndex
   character (len = *), intent(out) :: name
   character (len = *), intent(in), optional :: convention
   character (len = *), intent(in), optional :: purpose
   character (len = *), intent(in), optional :: attPackInstanceName
   type(ESMF_AttNest_Flag), intent(in), optional :: attnestflag
   type(ESMF_TypeKind_Flag), intent(out), optional :: typekind
   integer, intent(out), optional :: itemCount
   logical, intent(out), optional :: isPresent
   integer, intent(out), optional :: rc\end{verbatim}
{\sf DESCRIPTION:\\ }


   Returns information associated with the indexed Attribute,
   including {\tt name}, {\tt typekind} and {\tt itemCount}. Keep in
   mind that these indices start from 1, as expected in a Fortran API.
  
   Supported values for <object> are:
   \begin{description}
   \item type(ESMF\_Array), intent(in) :: array
   \item type(ESMF\_ArrayBundle), intent(in) :: arraybundle
   \item type(ESMF\_CplComp), intent(in) :: comp
   \item type(ESMF\_GridComp), intent(in) :: comp
   \item type(ESMF\_SciComp), intent(in) :: comp
   \item type(ESMF\_DistGrid), intent(in) :: distgrid
   \item type(ESMF\_Field), intent(in) :: field
   \item type(ESMF\_FieldBundle), intent(in) :: fieldbundle
   \item type(ESMF\_Grid), intent(in) :: grid
   \item type(ESMF\_State), intent(in) :: state
   \end{description}
  
   The arguments are:
   \begin{description}
   \item [<object>]
   An {\tt ESMF} object.
   \item [attributeIndex]
   The index number of the Attribute to query.
   \item [name]
   The name of the Attribute.
   \item [{[convention]}]
   The convention of the Attribute package.
   \item [{[purpose]}]
   The purpose of the Attribute package.
   \item [{[attPackInstanceName]}]
   The name of an Attribute package instance, specifying which one
   of multiple Attribute package instances of the same convention
   and purpose, within a nest. If not specified, defaults to the
   first instance.
   \item [{[attnestflag]}]
   A flag to determine whether to descend the
   Attribute hierarchy when looking for this Attribute, the default
   is {\tt ESMF\_ATTNEST\_ON}. This flag is documented in section
   \ref{const:attnest}.
   \item [{[typekind]}]
   The typekind of the Attribute. This flag is documented in section
   \ref{const:typekind}.
   \item [{[itemCount]}]
   The number of items in this Attribute.
   \item [{[isPresent]}]
   A logical flag to tell if this Attribute is present or not.
   \item [{[rc]}]
   Return code; equals {\tt ESMF\_SUCCESS} if there are no errors.
   \end{description}
  
   
%/////////////////////////////////////////////////////////////
 
\mbox{}\hrulefill\ 
 
\subsubsection [ESMF\_AttributeGet] {ESMF\_AttributeGet - Get Attribute package instance names from an ESMF\_AttPack}


  
\bigskip{\sf INTERFACE:}
\begin{verbatim}   ! Private name; call using ESMF_AttributeGet()
   subroutine ESMF_AttributeGetAPinstNamesAP(<object>, attpack, &
   attPackInstanceNameList, attPackInstanceNameCount, &
   attnestflag, rc)\end{verbatim}{\em ARGUMENTS:}
\begin{verbatim}   <object>, see below for supported values
   type(ESMF_AttPack), intent(inout) :: attpack
   character (len = *), intent(out) :: attPackInstanceNameList(:)
   integer, intent(out) :: attPackInstanceNameCount
 -- The following arguments require argument keyword syntax (e.g. rc=rc). --
   type(ESMF_AttNest_Flag), intent(in), optional :: attnestflag
   integer, intent(out), optional :: rc\end{verbatim}
{\sf DESCRIPTION:\\ }


   Get the Attribute package instance names of the ESMF\_AttPack.
   Also get the number of such names.
   See Section~\ref{sec:AttPacks} for a description of Attribute packages.
  
   Supported values for <object> are:
   \begin{description}
   \item type(ESMF\_CplComp), intent(in) :: comp
   \item type(ESMF\_GridComp), intent(in) :: comp
   \item type(ESMF\_SciComp), intent(in) :: comp
   \end{description}
  
   The arguments are:
   \begin{description}
   \item [<object>]
   An {\tt ESMF} object.
   \item [attpack]
   A handle to the Attribute package.
   \item [attPackInstanceNameList]
   The name(s) of the Attribute package instances of the given
   convention and purpose.
   \item [attPackInstanceNameCount]
   The number of Attribute package instance names.
   \item [{[attnestflag]}]
   A flag to determine whether to descend the
   Attribute hierarchy when searching for this Attribute package,
   the default is {\tt ESMF\_ATTNEST\_ON}. This flag is documented
   in section \ref{const:attnest}.
   \item [{[rc]}]
   Return code; equals {\tt ESMF\_SUCCESS} if there are no errors.
   \end{description}
  
   
%/////////////////////////////////////////////////////////////
 
\mbox{}\hrulefill\ 
 
\subsubsection [ESMF\_AttributeGet] {ESMF\_AttributeGet - Get Attribute package instance names}


  
\bigskip{\sf INTERFACE:}
\begin{verbatim}   ! Private name; call using ESMF_AttributeGet()
   subroutine ESMF_AttributeGetAPinstNames(<object>, convention, purpose, &
   attPackInstanceNameList, attPackInstanceNameCount, attnestflag, rc)\end{verbatim}{\em ARGUMENTS:}
\begin{verbatim}   <object>, see below for supported values
   character (len = *), intent(in), :: convention
   character (len = *), intent(in), :: purpose
   character (len = *), intent(out) :: attPackInstanceNameList(:)
   integer, intent(out) :: attPackInstanceNameCount
   type(ESMF_AttNest_Flag), intent(in), optional :: attnestflag
   integer, intent(out), optional :: rc\end{verbatim}
{\sf DESCRIPTION:\\ }


   Get the Attribute package instance names of the specified convention
   and purpose. Also get the number of such names.
   See Section~\ref{sec:AttPacks} for a description of Attribute packages
   and their conventions, purposes, and object types.
  
   Supported values for <object> are:
   \begin{description}
   \item type(ESMF\_CplComp), intent(in) :: comp
   \item type(ESMF\_GridComp), intent(in) :: comp
   \item type(ESMF\_SciComp), intent(in) :: comp
   \end{description}
  
   The arguments are:
   \begin{description}
   \item [<object>]
   An {\tt ESMF} object.
   \item [convention]
   The convention of the Attribute package instances.
   \item [purpose]
   The purpose of the Attribute package instances.
   \item [attPackInstanceNameList]
   The name(s) of the Attribute package instances of the given
   convention and purpose.
   \item [attPackInstanceNameCount]
   The number of Attribute package instance names.
   \item [{[attnestflag]}]
   A flag to determine whether to descend the
   Attribute hierarchy when searching for this Attribute package,
   the default is {\tt ESMF\_ATTNEST\_ON}. This flag is documented
   in section \ref{const:attnest}.
   \item [{[rc]}]
   Return code; equals {\tt ESMF\_SUCCESS} if there are no errors.
   \end{description}
  
   
%/////////////////////////////////////////////////////////////
 
\mbox{}\hrulefill\ 
 
\subsubsection [ESMF\_AttributeGetAttPack] {ESMF\_AttributeGetAttPack - Get an ESMF Attribute package object and/or query for presence}


  
\bigskip{\sf INTERFACE:}
\begin{verbatim}   ! Private name; call using ESMF_AttributeGetAttPack()
   subroutine ESMF_AttGetAttPack(<object>, convention, purpose, &
   attPackInstanceName, attpack, attnestflag, isPresent, rc)\end{verbatim}{\em ARGUMENTS:}
\begin{verbatim}   <object>, see below for supported values
   character (len = *), intent(in) :: convention
   character (len = *), intent(in) :: purpose
 -- The following arguments require argument keyword syntax (e.g. rc=rc). --
   character (len = *), intent(in), optional :: attPackInstanceName
   type(ESMF_AttPack), intent(inout), optional :: attpack
   type(ESMF_AttNest_Flag), intent(in), optional :: attnestflag
   logical, intent(out), optional :: isPresent
   integer, intent(out), optional :: rc\end{verbatim}
{\sf DESCRIPTION:\\ }


   Get an ESMF Attribute package object. If there are redundant Attribute
   packages on this object then the {\it most recently created} one will be
   retrieved.
   See Section~\ref{sec:AttPacks} for a description of Attribute packages.
  
   Supported values for <object> are:
   \begin{description}
   \item type(ESMF\_Array), intent(inout) :: array
   \item type(ESMF\_ArrayBundle), intent(inout) :: arraybundle
   \item type(ESMF\_CplComp), intent(inout) :: comp
   \item type(ESMF\_GridComp), intent(inout) :: comp
   \item type(ESMF\_SciComp), intent(inout) :: comp
   \item type(ESMF\_DistGrid), intent(inout) :: distgrid
   \item type(ESMF\_Field), intent(inout) :: field
   \item type(ESMF\_FieldBundle), intent(inout) :: fieldbundle
   \item type(ESMF\_Grid), intent(inout) :: grid
   \item type(ESMF\_State), intent(inout) :: state
   \end{description}
  
   The arguments are:
   \begin{description}
   \item [<object>]
   An {\tt ESMF} object.
   \item [convention]
   The convention of the Attribute package.
   \item [purpose]
   The purpose of the Attribute package.
   \item [{[attPackInstanceName]}]
   The name of an Attribute package instance, specifying which one
   of multiple Attribute package instances of the same convention
   and purpose, within a nest. If not specified, defaults to the
   first instance.
   \item [{[attpack]}]
   A handle to the Attribute package.
   \item [{[attnestflag]}]
   A flag to determine whether to descend the
   Attribute hierarchy when searching for this Attribute package, the
   default is {\tt ESMF\_ATTNEST\_ON}. This flag is documented in
   section \ref{const:attnest}.
   \item [{[isPresent]}]
   A logical flag to tell if this Attribute package is present or not.
   \item [{[rc]}]
   Return code; equals {\tt ESMF\_SUCCESS} if there are no errors.
   \end{description}
  
   
%/////////////////////////////////////////////////////////////
 
\mbox{}\hrulefill\ 
 
\subsubsection [ESMF\_AttributeLink] {ESMF\_AttributeLink - Link a Component Attribute hierarchy to that of a Component or State}


  
\bigskip{\sf INTERFACE:}
\begin{verbatim}   ! Private name; call using ESMF_AttributeLink()
   subroutine ESMF_CompAttLink(<object1>, <object2>, rc)\end{verbatim}{\em ARGUMENTS:}
\begin{verbatim}   <object1>, see below for supported values
   <object2>, see below for supported values
   integer, intent(out), optional :: rc\end{verbatim}
{\sf DESCRIPTION:\\ }


   Attach a {\tt CplComp}, {\tt GridComp}, or {\tt SciComp} Attribute
   hierarchy to the
   hierarchy of a {\tt CplComp}, {\tt GridComp}, {\tt SciComp}, or
   {\tt State}.
  
   Supported values for the <object1> are:
   \begin{description}
   \item type(ESMF\_CplComp), intent(inout) :: comp1
   \item type(ESMF\_GridComp), intent(inout) :: comp1
   \item type(ESMF\_SCiComp), intent(inout) :: comp1
   \end{description}
  
   Supported values for the <object2> are:
   \begin{description}
   \item type(ESMF\_CplComp), intent(in) :: comp2
   \item type(ESMF\_GridComp), intent(in) :: comp2
   \item type(ESMF\_SciComp), intent(in) :: comp2
   \item type(ESMF\_State), intent(in) :: state
   \end{description}
  
   The arguments are:
   \begin{description}
   \item [<object1>]
   The \textit{parent} object in the Attribute hierarchy link.
   \item [<object2>]
   The \textit{child} object in the Attribute hierarchy link.
   \item [{[rc]}]
   Return code; equals {\tt ESMF\_SUCCESS} if there are no errors.
   \end{description}
  
   
%/////////////////////////////////////////////////////////////
 
\mbox{}\hrulefill\ 
 
\subsubsection [ESMF\_AttributeLink] {ESMF\_AttributeLink - Link a State Attribute hierarchy with the hierarchy of an Array, ArrayBundle, Field, FieldBundle, or State}


  
\bigskip{\sf INTERFACE:}
\begin{verbatim}   ! Private name; call using ESMF_AttributeLink()
   subroutine ESMF_StateAttLink(state, <object>, rc)\end{verbatim}{\em ARGUMENTS:}
\begin{verbatim}   type(ESMF_State), intent(inout) :: state
   <object>, see below for supported values
   integer, intent(out), optional :: rc\end{verbatim}
{\sf DESCRIPTION:\\ }


   Attach a {\tt State} Attribute hierarchy to the hierarchy of
   a {\tt Fieldbundle}, {\tt Field}, or another {\tt State}.
  
   Supported values for the <object> are:
   \begin{description}
   \item type(ESMF\_Array), intent(in) :: array
   \item type(ESMF\_ArrayBundle), intent(in) :: arraybundle
   \item type(ESMF\_Field), intent(in) :: field
   \item type(ESMF\_FieldBundle), intent(in) :: fieldbundle
   \item type(ESMF\_State), intent(in) :: state
   \end{description}
  
   The arguments are:
   \begin{description}
   \item [state]
   An {\tt ESMF\_State} object.
   \item [<object>]
   The object with which to link hierarchies.
   \item [{[rc]}]
   Return code; equals {\tt ESMF\_SUCCESS} if there are no errors.
   \end{description}
  
   
%/////////////////////////////////////////////////////////////
 
\mbox{}\hrulefill\ 
 
\subsubsection [ESMF\_AttributeLink] {ESMF\_AttributeLink - Link a FieldBundle and Field Attribute hierarchy}


  
\bigskip{\sf INTERFACE:}
\begin{verbatim}   ! Private name; call using ESMF_AttributeLink()
   subroutine ESMF_FieldBundleAttLink(fieldbundle, field, rc)\end{verbatim}{\em ARGUMENTS:}
\begin{verbatim}   type(ESMF_FieldBundle), intent(inout) :: fieldbundle
   type(ESMF_Field), intent(in) :: field
   integer, intent(out), optional :: rc\end{verbatim}
{\sf DESCRIPTION:\\ }


   Attach a {\tt FieldBundle} Attribute hierarchy to the hierarchy of
   a {\tt Field}.
  
   The arguments are:
   \begin{description}
   \item [fieldbundle]
   An {\tt ESMF\_FieldBundle} object.
   \item [field]
   An {\tt ESMF\_Field} object.
   \item [{[rc]}]
   Return code; equals {\tt ESMF\_SUCCESS} if there are no errors.
   \end{description}
  
   
%/////////////////////////////////////////////////////////////
 
\mbox{}\hrulefill\ 
 
\subsubsection [ESMF\_AttributeLink] {ESMF\_AttributeLink - Link a Field and Grid Attribute hierarchy}


  
\bigskip{\sf INTERFACE:}
\begin{verbatim}   ! Private name; call using ESMF_AttributeLink()
   subroutine ESMF_FieldAttLink(field, grid, rc)\end{verbatim}{\em ARGUMENTS:}
\begin{verbatim}   type(ESMF_Field), intent(inout) :: field
   type(ESMF_Grid), intent(in) :: grid
   integer, intent(out), optional :: rc\end{verbatim}
{\sf DESCRIPTION:\\ }


   Attach a {\tt Field} Attribute hierarchy to the hierarchy of
   a {\tt Grid}.
  
   The arguments are:
   \begin{description}
   \item [field]
   An {\tt ESMF\_Field} object.
   \item [grid]
   An {\tt ESMF\_Grid} object.
   \item [{[rc]}]
   Return code; equals {\tt ESMF\_SUCCESS} if there are no errors.
   \end{description}
  
   
%/////////////////////////////////////////////////////////////
 
\mbox{}\hrulefill\ 
 
\subsubsection [ESMF\_AttributeLink] {ESMF\_AttributeLink - Link an ArrayBundle and Array Attribute hierarchy}


  
\bigskip{\sf INTERFACE:}
\begin{verbatim}   ! Private name; call using ESMF_AttributeLink()
   subroutine ESMF_ArrayBundleAttLink(arraybundle, array, rc)\end{verbatim}{\em ARGUMENTS:}
\begin{verbatim}   type(ESMF_ArrayBundle), intent(inout) :: arraybundle
   type(ESMF_Array), intent(in) :: array
   integer, intent(out), optional :: rc\end{verbatim}
{\sf DESCRIPTION:\\ }


   Attach an {\tt ArrayBundle} Attribute hierarchy to the hierarchy of
   an {\tt Array}.
  
   The arguments are:
   \begin{description}
   \item [arraybundle]
   An {\tt ESMF\_ArrayBundle} object.
   \item [array]
   An {\tt ESMF\_Array} object.
   \item [{[rc]}]
   Return code; equals {\tt ESMF\_SUCCESS} if there are no errors.
   \end{description}
  
   
%/////////////////////////////////////////////////////////////
 
\mbox{}\hrulefill\ 
 
\subsubsection [ESMF\_AttributeLinkRemove] {ESMF\_AttributeLinkRemove - Unlink a Component Attribute hierarchy from that of a Component or State}


  
\bigskip{\sf INTERFACE:}
\begin{verbatim}   ! Private name; call using ESMF_AttributeLinkRemove()
   subroutine ESMF_CompAttLinkRemove(<object1>, <object2>, rc)\end{verbatim}{\em ARGUMENTS:}
\begin{verbatim}   <object1>, see below for supported values
   <object2>, see below for supported values
   integer, intent(out), optional :: rc\end{verbatim}
{\sf DESCRIPTION:\\ }


   Unattach a {\tt CplComp}, {\tt GridComp}, or {\tt SciComp} Attribute
   hierarchy from the hierarchy of a {\tt CplComp}, {\tt GridComp},
   {\tt SciComp}, or {\tt State}.
  
   Supported values for the <object1> are:
   \begin{description}
   \item type(ESMF\_CplComp), intent(inout) :: comp1
   \item type(ESMF\_GridComp), intent(inout) :: comp1
   \item type(ESMF\_SciComp), intent(inout) :: comp1
   \end{description}
  
   Supported values for the <object2> are:
   \begin{description}
   \item type(ESMF\_CplComp), intent(in) :: comp2
   \item type(ESMF\_GridComp), intent(in) :: comp2
   \item type(ESMF\_SciComp), intent(in) :: comp2
   \item type(ESMF\_State), intent(in) :: state
   \end{description}
  
   The arguments are:
   \begin{description}
   \item [<object1>]
   The \textit{parent} object in the Attribute hierarchy link.
   \item [<object2>]
   The \textit{child} object in the Attribute hierarchy link.
   \item [{[rc]}]
   Return code; equals {\tt ESMF\_SUCCESS} if there are no errors.
   \end{description}
  
   
%/////////////////////////////////////////////////////////////
 
\mbox{}\hrulefill\ 
 
\subsubsection [ESMF\_AttributeLinkRemove] {ESMF\_AttributeLinkRemove - Unlink a State Attribute hierarchy from the hierarchy of an Array, ArrayBundle, Field, FieldBundle, or State}


  
\bigskip{\sf INTERFACE:}
\begin{verbatim}   ! Private name; call using ESMF_AttributeLinkRemove()
   subroutine ESMF_StateAttLinkRemove(state, <object>, rc)\end{verbatim}{\em ARGUMENTS:}
\begin{verbatim}   type(ESMF_State), intent(inout) :: state
   <object>, see below for supported values
   integer, intent(out), optional :: rc\end{verbatim}
{\sf DESCRIPTION:\\ }


   Unattach a {\tt State} Attribute hierarchy from the hierarchy of
   a {\tt Fieldbundle}, {\tt Field}, or another {\tt State}.
  
   Supported values for the <object> are:
   \begin{description}
   \item type(ESMF\_Array), intent(in) :: array
   \item type(ESMF\_ArrayBundle), intent(in) :: arraybundle
   \item type(ESMF\_Field), intent(in) :: field
   \item type(ESMF\_FieldBundle), intent(in) :: fieldbundle
   \item type(ESMF\_State), intent(in) :: state
   \end{description}
  
   The arguments are:
   \begin{description}
   \item [state]
   An {\tt ESMF\_State} object.
   \item [<object>]
   The object with which to unlink hierarchies.
   \item [{[rc]}]
   Return code; equals {\tt ESMF\_SUCCESS} if there are no errors.
   \end{description}
  
   
%/////////////////////////////////////////////////////////////
 
\mbox{}\hrulefill\ 
 
\subsubsection [ESMF\_AttributeLinkRemove] {ESMF\_AttributeLinkRemove - Unlink a FieldBundle and Field Attribute hierarchy}


  
\bigskip{\sf INTERFACE:}
\begin{verbatim}   ! Private name; call using ESMF_AttributeLinkRemove()
   subroutine ESMF_FieldBundleAttLinkRemove(fieldbundle, field, rc)\end{verbatim}{\em ARGUMENTS:}
\begin{verbatim}   type(ESMF_FieldBundle), intent(inout) :: fieldbundle
   type(ESMF_Field), intent(in) :: field
   integer, intent(out), optional :: rc\end{verbatim}
{\sf DESCRIPTION:\\ }


   Unattach a {\tt FieldBundle} Attribute hierarchy from the hierarchy of
   a {\tt Field}.
  
   The arguments are:
   \begin{description}
   \item [fieldbundle]
   An {\tt ESMF\_FieldBundle} object.
   \item [field]
   An {\tt ESMF\_Field} object.
   \item [{[rc]}]
   Return code; equals {\tt ESMF\_SUCCESS} if there are no errors.
   \end{description}
  
   
%/////////////////////////////////////////////////////////////
 
\mbox{}\hrulefill\ 
 
\subsubsection [ESMF\_AttributeLinkRemove] {ESMF\_AttributeLinkRemove - Unlink a Field and Grid Attribute hierarchy}


  
\bigskip{\sf INTERFACE:}
\begin{verbatim}   ! Private name; call using ESMF_AttributeLinkRemove()
   subroutine ESMF_FieldAttLinkRemove(field, grid, rc)\end{verbatim}{\em ARGUMENTS:}
\begin{verbatim}   type(ESMF_Field), intent(inout) :: field
   type(ESMF_Grid), intent(in) :: grid
   integer, intent(out), optional :: rc\end{verbatim}
{\sf DESCRIPTION:\\ }


   Unattach a {\tt Field} Attribute hierarchy from the hierarchy of
   a {\tt Grid}.
  
   The arguments are:
   \begin{description}
   \item [field]
   An {\tt ESMF\_Field} object.
   \item [grid]
   An {\tt ESMF\_Grid} object.
   \item [{[rc]}]
   Return code; equals {\tt ESMF\_SUCCESS} if there are no errors.
   \end{description}
  
   
%/////////////////////////////////////////////////////////////
 
\mbox{}\hrulefill\ 
 
\subsubsection [ESMF\_AttributeLinkRemove] {ESMF\_AttributeLinkRemove - Unlink an ArrayBundle and Array Attribute hierarchy}


  
\bigskip{\sf INTERFACE:}
\begin{verbatim}   ! Private name; call using ESMF_AttributeLinkRemove()
   subroutine ESMF_ArrayBundleAttLinkRemove(arraybundle, array, rc)\end{verbatim}{\em ARGUMENTS:}
\begin{verbatim}   type(ESMF_ArrayBundle), intent(inout) :: arraybundle
   type(ESMF_Array), intent(in) :: array
   integer, intent(out), optional :: rc\end{verbatim}
{\sf DESCRIPTION:\\ }


   Unattach an {\tt ArrayBundle} Attribute hierarchy from the hierarchy of
   an {\tt Array}.
  
   The arguments are:
   \begin{description}
   \item [arraybundle]
   An {\tt ESMF\_ArrayBundle} object.
   \item [array]
   An {\tt ESMF\_Array} object.
   \item [{[rc]}]
   Return code; equals {\tt ESMF\_SUCCESS} if there are no errors.
   \end{description}
  
   
%/////////////////////////////////////////////////////////////
 
\mbox{}\hrulefill\ 
 
\subsubsection [ESMF\_AttributeRead] {ESMF\_AttributeRead - Read Attributes from an XML file}


   \label{api:AttributeRead}
  
\bigskip{\sf INTERFACE:}
\begin{verbatim}   subroutine ESMF_AttributeRead(<object>, fileName, schemaFileName, rc)\end{verbatim}{\em ARGUMENTS:}
\begin{verbatim}   <object>, see below for supported values
   character (len = *), intent(in), optional :: fileName
   character (len = *), intent(in), optional :: schemaFileName
   integer, intent(out), optional :: rc\end{verbatim}
{\sf DESCRIPTION:\\ }


   Read Attributes for <object> from fileName, whose format is XML.
   schemaFileName format is XSD. If present, the schemaFileName is used to
   validate the contents of fileName. schemaFileName must be specified for
   a fileName containing custom, user-defined Attributes.
  
   Requires the third-party Xerces C++ XML Parser library to be installed,
   v3.1.0 or newer. For more details, see the "ESMF Users Guide",
   "Building and Installing the ESMF, Third Party Libraries, Xerces" and
   the website
   \newline
   "http://xerces.apache.org/xerces-c". Also please see the
   section on Attribute I/O,~\ref{io:attributeio}.
  
   Supported values for <object> are:
   \begin{description}
   \item type(ESMF\_Array), intent(inout) :: array ! not yet implemented
   \item type(ESMF\_ArrayBundle), intent(inout) :: arrayBundle ! not yet implemented
   \item type(ESMF\_CplComp), intent(inout) :: cplComp
   \item type(ESMF\_GridComp), intent(inout) :: gridComp
   \item type(ESMF\_SciComp), intent(inout) :: gridComp
   \item type(ESMF\_Field), intent(inout) :: field
   \item type(ESMF\_FieldBundle), intent(inout) :: fieldbundle ! not yet implemented
   \item type(ESMF\_Grid), intent(inout) :: grid
   \item type(ESMF\_DistGrid), intent(inout) :: distGrid ! not yet implemented
   \end{description}
  
   The arguments are:
   \begin{description}
   \item [<object>]
   The {\tt ESMF} object onto which the read Attributes will be placed.
   \item [{[fileName]}]
   The name of the XML file to read.
   \item [{[schemaFileName]}]
   The name of the XSD file to validate the contents of fileName.
   \item [{[rc]}]
   Return code; equals {\tt ESMF\_SUCCESS} if there are no errors.
   \end{description}
   
%/////////////////////////////////////////////////////////////
 
\mbox{}\hrulefill\ 
 
\subsubsection [ESMF\_AttributeRemove] {ESMF\_AttributeRemove - Remove an Attribute or Attribute package using an ESMF\_AttPack}


  
\bigskip{\sf INTERFACE:}
\begin{verbatim}   subroutine ESMF_AttributeRemove(<object>, name, &
   attpack, rc)\end{verbatim}{\em ARGUMENTS:}
\begin{verbatim}   <object>, see below for supported values
 -- The following arguments require argument keyword syntax (e.g. rc=rc). --
   character (len = *), intent(in), optional :: name
   type(ESMF_AttPack), intent(inout) :: attpack
   integer, intent(out), optional :: rc\end{verbatim}
{\sf DESCRIPTION:\\ }


   Remove an Attribute, or Attribute package on <object>.
   See Section~\ref{sec:AttPacks} for a description of Attribute packages
   and their conventions, purposes, and object types.
  
   Supported values for <object> are:
   \begin{description}
   \item type(ESMF\_Array), intent(inout) :: array
   \item type(ESMF\_ArrayBundle), intent(inout) :: arraybundle
   \item type(ESMF\_CplComp), intent(inout) :: comp
   \item type(ESMF\_GridComp), intent(inout) :: comp
   \item type(ESMF\_SciComp), intent(inout) :: comp
   \item type(ESMF\_DistGrid), intent(inout) :: distgrid
   \item type(ESMF\_Field), intent(inout) :: field
   \item type(ESMF\_FieldBundle), intent(inout) :: fieldbundle
   \item type(ESMF\_Grid), intent(inout) :: grid
   \item type(ESMF\_State), intent(inout) :: state
   \end{description}
  
   The arguments are:
   \begin{description}
   \item [<object>]
   An {\tt ESMF} object.
   \item [{[name]}]
   The name of the Attribute to remove.
   \item [attpack]
   A handle to the Attribute package.
   \item [{[rc]}]
   Return code; equals {\tt ESMF\_SUCCESS} if there are no errors.
   \end{description}
  
   NOTE: An entire Attribute package can be removed by specifying
   {\tt attpack} only, without {\tt name}. By specifying
   {\tt attpack} an Attribute will be removed
   from the corresponding Attribute package, if it exists. An
   Attribute can be removed directly from <object> by specifying
   {\tt name}, without {\tt attpack}.
  
   
%/////////////////////////////////////////////////////////////
 
\mbox{}\hrulefill\ 
 
\subsubsection [ESMF\_AttributeRemove] {ESMF\_AttributeRemove - Remove an Attribute or Attribute package}


  
\bigskip{\sf INTERFACE:}
\begin{verbatim}   subroutine ESMF_AttributeRemove(<object>, name, convention, purpose, &
   attPackInstanceName, rc)\end{verbatim}{\em ARGUMENTS:}
\begin{verbatim}   <object>, see below for supported values
   character (len = *), intent(in), optional :: name
   character (len = *), intent(in), optional :: convention
   character (len = *), intent(in), optional :: purpose
   character (len = *), intent(in), optional :: attPackInstanceName
   integer, intent(out), optional :: rc\end{verbatim}
{\sf DESCRIPTION:\\ }


   Remove an Attribute, or Attribute package on <object>.
   See Section~\ref{sec:AttPacks} for a description of Attribute packages
   and their conventions, purposes, and object types.
  
   Supported values for <object> are:
   \begin{description}
   \item type(ESMF\_Array), intent(inout) :: array
   \item type(ESMF\_ArrayBundle), intent(inout) :: arraybundle
   \item type(ESMF\_CplComp), intent(inout) :: comp
   \item type(ESMF\_GridComp), intent(inout) :: comp
   \item type(ESMF\_SciComp), intent(inout) :: comp
   \item type(ESMF\_DistGrid), intent(inout) :: distgrid
   \item type(ESMF\_Field), intent(inout) :: field
   \item type(ESMF\_FieldBundle), intent(inout) :: fieldbundle
   \item type(ESMF\_Grid), intent(inout) :: grid
   \item type(ESMF\_State), intent(inout) :: state
   \end{description}
  
   The arguments are:
   \begin{description}
   \item [<object>]
   An {\tt ESMF} object.
   \item [{[name]}]
   The name of the Attribute to remove.
   \item [{[convention]}]
   The convention of the Attribute package.
   \item [{[purpose]}]
   The purpose of the Attribute package.
   \item [{[attPackInstanceName]}]
   The name of an Attribute package instance, specifying which one
   of multiple Attribute package instances of the same convention
   and purpose, within a nest. If not specified, defaults to the
   first instance.
   \item [{[rc]}]
   Return code; equals {\tt ESMF\_SUCCESS} if there are no errors.
   \end{description}
  
   NOTE: An entire Attribute package can be removed by specifying
   {\tt convention}, {\tt purpose}, and {\tt attPackInstanceName}
   only, without {\tt name}. An
   Attribute can be removed directly from <object> by specifying
   {\tt name}, without {\tt convention}, {\tt purpose}, and
   {\tt attPackInstanceName}.
  
   
%/////////////////////////////////////////////////////////////
 
\mbox{}\hrulefill\ 
 
\subsubsection [ESMF\_AttributeSet] {ESMF\_AttributeSet - Set an Attribute in an ESMF\_AttPack}


  
\bigskip{\sf INTERFACE:}
\begin{verbatim}   subroutine ESMF_AttributeSet(<object>, name, <value>, attpack, &
   rc)\end{verbatim}{\em ARGUMENTS:}
\begin{verbatim}   <object>, see below for supported values
   character (len = *), intent(in) :: name
   <value>, see below for supported values
   type(ESMF_AttPack), intent(inout) :: attpack
 -- The following arguments require argument keyword syntax (e.g. rc=rc). --
   integer, intent(out), optional :: rc\end{verbatim}
{\sf DESCRIPTION:\\ }


   Attach an Attribute to <object>, or set an Attribute in an
   Attribute package. The Attribute has a {\tt name} and {\tt value},
   and, if in an Attribute package, a {\tt attpack}.
   See Section~\ref{sec:AttPacks} for a description of Attribute packages.
  
   Supported values for <object> are:
   \begin{description}
   \item type(ESMF\_Array), intent(inout) :: array
   \item type(ESMF\_ArrayBundle), intent(inout) :: arraybundle
   \item type(ESMF\_CplComp), intent(inout) :: comp
   \item type(ESMF\_GridComp), intent(inout) :: comp
   \item type(ESMF\_SciComp), intent(inout) :: comp
   \item type(ESMF\_DistGrid), intent(inout) :: distgrid
   \item type(ESMF\_Field), intent(inout) :: field
   \item type(ESMF\_FieldBundle), intent(inout) :: fieldbundle
   \item type(ESMF\_Grid), intent(inout) :: grid
   \item type(ESMF\_State), intent(inout) :: state
   \end{description}
  
   Supported values for the <value> are:
   \begin{description}
   \item integer(ESMF\_KIND\_I4), intent(in) :: value
   \item integer(ESMF\_KIND\_I8), intent(in) :: value
   \item real (ESMF\_KIND\_R4), intent(in) :: value
   \item real (ESMF\_KIND\_R8), intent(in) :: value
   \item logical, intent(in) :: value
   \item character (len = *), intent(in) :: value
   \end{description}
  
   The arguments are:
   \begin{description}
   \item [<object>]
   An {\tt ESMF} object.
   \item [name]
   The name of the Attribute to set.
   \item [<value argument>]
   The value of the Attribute to set.
   \item [attpack]
   A handle to the Attribute package.
   \item [{[rc]}]
   Return code; equals {\tt ESMF\_SUCCESS} if there are no errors.
   \end{description}
  
   
%/////////////////////////////////////////////////////////////
 
\mbox{}\hrulefill\ 
 
\subsubsection [ESMF\_AttributeSet] {ESMF\_AttributeSet - Set an Attribute to point to internal class information in an ESMF\_AttPack}


  
\bigskip{\sf INTERFACE:}
\begin{verbatim}   subroutine ESMF_AttributeSet(<object>, name, <value>, attpack, &
   inputList, rc)\end{verbatim}{\em ARGUMENTS:}
\begin{verbatim}   <object>, see below for supported values
   character (len = *), intent(in) :: name
   <value>, see below for supported values
   type(ESMF_AttPack), intent(inout) :: attpack
 -- The following arguments require argument keyword syntax (e.g. rc=rc). --
   character (len = *), intent(in), optional :: inputList(:)
   integer, intent(out), optional :: rc\end{verbatim}
{\sf DESCRIPTION:\\ }


   Attach an Attribute to <object>, or set an Attribute in an
   Attribute package. The Attribute has a {\tt name} and {\tt value},
   and, if in an Attribute package, a {\tt attpack}.
   See Section~\ref{sec:AttPacks} for a description of Attribute packages.
   The Attribute can
   also be set to be a pointer to internal class information. See Section
   \ref{sec:InternalInfo} for a description of this capability.
  
   Supported values for <object> are:
   \begin{description}
   \item type(ESMF\_Grid), intent(inout) :: grid
   \end{description}
  
   Supported values for the <value> are:
   \begin{description}
   \item character (len = *), intent(in), :: value
   \end{description}
  
   The arguments are:
   \begin{description}
   \item [<object>]
   An {\tt ESMF} object.
   \item [name]
   The name of the Attribute to set.
   \item [<value argument>]
   The value of the Attribute to set.
   \item [attpack]
   A handle to the Attribute package.
   \item [{[inputList]}]
   A list of the input parameters required to set internal info.
   \item [{[rc]}]
   Return code; equals {\tt ESMF\_SUCCESS} if there are no errors.
   \end{description}
  
   
%/////////////////////////////////////////////////////////////
 
\mbox{}\hrulefill\ 
 
\subsubsection [ESMF\_AttributeSet] {ESMF\_AttributeSet - Set an Attribute in an ESMF\_AttPack}


  
\bigskip{\sf INTERFACE:}
\begin{verbatim}   subroutine ESMF_AttributeSet(<object>, name, <valueList>, attpack, &
   itemCount, rc)\end{verbatim}{\em ARGUMENTS:}
\begin{verbatim}   <object>, see below for supported values
   character (len = *), intent(in) :: name
   <valueList>, see below for supported values
   type(ESMF_AttPack), intent(in) :: attpack
 -- The following arguments require argument keyword syntax (e.g. rc=rc). --
   integer, intent(in), optional :: itemCount
   integer, intent(out), optional :: rc\end{verbatim}
{\sf DESCRIPTION:\\ }


   Attach an Attribute to <object>, or set an Attribute in an
   Attribute package. The Attribute has a {\tt name} and a
   {\tt valueList}, with an {\tt itemCount}, and, if in an Attribute
   package, a {\tt attpack}. See Section~\ref{sec:AttPacks} for a
   description of Attribute packages.
  
   Supported values for <object> are:
   \begin{description}
   \item type(ESMF\_Array), intent(inout) :: array
   \item type(ESMF\_ArrayBundle), intent(inout) :: arraybundle
   \item type(ESMF\_CplComp), intent(inout) :: comp
   \item type(ESMF\_GridComp), intent(inout) :: comp
   \item type(ESMF\_SciComp), intent(inout) :: comp
   \item type(ESMF\_DistGrid), intent(inout) :: distgrid
   \item type(ESMF\_Field), intent(inout) :: field
   \item type(ESMF\_FieldBundle), intent(inout) :: fieldbundle
   \item type(ESMF\_Grid), intent(inout) :: grid
   \item type(ESMF\_State), intent(inout) :: state
   \end{description}
  
   Supported values for the <valueList> are:
   \begin{description}
   \item integer(ESMF\_KIND\_I4), intent(in) :: valueList(:)
   \item integer(ESMF\_KIND\_I8), intent(in) :: valueList(:)
   \item real (ESMF\_KIND\_R4), intent(in) :: valueList(:)
   \item real (ESMF\_KIND\_R8), intent(in) :: valueList(:)
   \item logical, intent(in) :: valueList(:)
   \item character (len = *), intent(in) :: valueList(:)
   \end{description}
  
   The arguments are:
   \begin{description}
   \item [<object>]
   An {\tt ESMF} object.
   \item [name]
   The name of the Attribute to set.
   \item [<valueList argument>]
   The valueList of the Attribute to set.
   \item [attpack]
   A handle to the Attribute package.
   \item [{[itemCount]}]
   The number of items in a multi-valued Attribute.
   \item [{[rc]}]
   Return code; equals {\tt ESMF\_SUCCESS} if there are no errors.
   \end{description}
  
   
%/////////////////////////////////////////////////////////////
 
\mbox{}\hrulefill\ 
 
\subsubsection [ESMF\_AttributeSet] {ESMF\_AttributeSet - Set an Attribute}


  
\bigskip{\sf INTERFACE:}
\begin{verbatim}   subroutine ESMF_AttributeSet(<object>, name, <value>, &
   convention, purpose, attPackInstanceName, rc)\end{verbatim}{\em ARGUMENTS:}
\begin{verbatim}   <object>, see below for supported values
   character (len = *), intent(in) :: name
   <value>, see below for supported values
   character (len = *), intent(in), optional :: convention
   character (len = *), intent(in), optional :: purpose
   character (len = *), intent(in), optional :: attPackInstanceName
   integer, intent(out), optional :: rc\end{verbatim}
{\sf DESCRIPTION:\\ }


   Attach an Attribute to <object>, or set an Attribute in an
   Attribute package. The Attribute has a {\tt name} and {\tt value},
   and, if in an Attribute package, {\tt convention}, {\tt purpose}, and
   {\tt attPackInstanceName}.
   See Section~\ref{sec:AttPacks} for a description of Attribute packages.
  
   Supported values for <object> are:
   \begin{description}
   \item type(ESMF\_Array), intent(inout) :: array
   \item type(ESMF\_ArrayBundle), intent(inout) :: arraybundle
   \item type(ESMF\_CplComp), intent(inout) :: comp
   \item type(ESMF\_GridComp), intent(inout) :: comp
   \item type(ESMF\_SciComp), intent(inout) :: comp
   \item type(ESMF\_DistGrid), intent(inout) :: distgrid
   \item type(ESMF\_Field), intent(inout) :: field
   \item type(ESMF\_FieldBundle), intent(inout) :: fieldbundle
   \item type(ESMF\_Grid), intent(inout) :: grid
   \item type(ESMF\_State), intent(inout) :: state
   \end{description}
  
   Supported values for the <value> are:
   \begin{description}
   \item integer(ESMF\_KIND\_I4), intent(in) :: value
   \item integer(ESMF\_KIND\_I8), intent(in) :: value
   \item real (ESMF\_KIND\_R4), intent(in) :: value
   \item real (ESMF\_KIND\_R8), intent(in) :: value
   \item logical, intent(in) :: value
   \item character (len = *), intent(in) :: value
   \end{description}
  
   The arguments are:
   \begin{description}
   \item [<object>]
   An {\tt ESMF} object.
   \item [name]
   The name of the Attribute to set.
   \item [<value argument>]
   The value of the Attribute to set.
   \item [{[convention]}]
   The convention of the Attribute package.
   \item [{[purpose]}]
   The purpose of the Attribute package.
   \item [{[attPackInstanceName]}]
   The name of an Attribute package instance, specifying which one
   of multiple Attribute package instances of the same convention
   and purpose, within a nest. If not specified, defaults to the
   first instance. (Not implemented yet)
   \item [{[rc]}]
   Return code; equals {\tt ESMF\_SUCCESS} if there are no errors.
   \end{description}
  
   
%/////////////////////////////////////////////////////////////
 
\mbox{}\hrulefill\ 
 
\subsubsection [ESMF\_AttributeSet] {ESMF\_AttributeSet - Set an Attribute to point to internal class information}


  
\bigskip{\sf INTERFACE:}
\begin{verbatim}   subroutine ESMF_AttributeSet(<object>, name, <value>, inputList,
   convention, purpose, attPackInstanceName, rc)\end{verbatim}{\em ARGUMENTS:}
\begin{verbatim}   <object>, see below for supported values
   character (len = *), intent(in) :: name
   <value>, see below for supported values
   character (len = *), intent(in), optional :: inputList(:)
   character (len = *), intent(in), optional :: convention
   character (len = *), intent(in), optional :: purpose
   character (len = *), intent(in), optional :: attPackInstanceName
   integer, intent(out), optional :: rc\end{verbatim}
{\sf DESCRIPTION:\\ }


   Attach an Attribute to <object>, or set an Attribute in an
   Attribute package. The Attribute has a {\tt name} and {\tt value},
   and, if in an Attribute package, {\tt convention}, {\tt purpose}, and
   {\tt attPackInstanceName}.
   See Section~\ref{sec:AttPacks} for a description of Attribute packages.
   The Attribute can
   also be set to be a pointer to internal class information. See Section
   \ref{sec:InternalInfo} for a description of this capability.
  
   Supported values for <object> are:
   \begin{description}
   \item type(ESMF\_Grid), intent(inout) :: grid
   \end{description}
  
   Supported values for the <value> are:
   \begin{description}
   \item character (len = *), intent(in), :: value
   \end{description}
  
   The arguments are:
   \begin{description}
   \item [<object>]
   An {\tt ESMF} object.
   \item [name]
   The name of the Attribute to set.
   \item [<value argument>]
   The value of the Attribute to set.
   \item [{[inputList]}]
   A list of the input parameters required to set internal info.
   \item [{[convention]}]
   The convention of the Attribute package.
   \item [{[purpose]}]
   The purpose of the Attribute package.
   \item [{[attPackInstanceName]}]
   The name of an Attribute package instance, specifying which one
   of multiple Attribute package instances of the same convention
   and purpose, within a nest. If not specified, defaults to the
   first instance. (Not implemented yet)
   \item [{[rc]}]
   Return code; equals {\tt ESMF\_SUCCESS} if there are no errors.
   \end{description}
  
   
%/////////////////////////////////////////////////////////////
 
\mbox{}\hrulefill\ 
 
\subsubsection [ESMF\_AttributeSet] {ESMF\_AttributeSet - Set an Attribute}


  
\bigskip{\sf INTERFACE:}
\begin{verbatim}   subroutine ESMF_AttributeSet(<object>, name, <valueList>, &
   convention, purpose, attPackInstanceName, itemCount, rc)\end{verbatim}{\em ARGUMENTS:}
\begin{verbatim}   <object>, see below for supported values
   character (len = *), intent(in) :: name
   <valueList>, see below for supported values
   character (len = *), intent(in), optional :: convention
   character (len = *), intent(in), optional :: purpose
   character (len = *), intent(in), optional :: attPackInstanceName
   integer, intent(in), optional :: itemCount
   integer, intent(out), optional :: rc\end{verbatim}
{\sf DESCRIPTION:\\ }


   Attach an Attribute to <object>, or set an Attribute in an
   Attribute package. The Attribute has a {\tt name} and a
   {\tt valueList}, with an {\tt itemCount}, and, if in an Attribute
   package, {\tt convention}, {\tt purpose}, and {\tt attPackInstanceName}.
   See Section~\ref{sec:AttPacks} for a
   description of Attribute packages.
  
   Supported values for <object> are:
   \begin{description}
   \item type(ESMF\_Array), intent(inout) :: array
   \item type(ESMF\_ArrayBundle), intent(inout) :: arraybundle
   \item type(ESMF\_CplComp), intent(inout) :: comp
   \item type(ESMF\_GridComp), intent(inout) :: comp
   \item type(ESMF\_SciComp), intent(inout) :: comp
   \item type(ESMF\_DistGrid), intent(inout) :: distgrid
   \item type(ESMF\_Field), intent(inout) :: field
   \item type(ESMF\_FieldBundle), intent(inout) :: fieldbundle
   \item type(ESMF\_Grid), intent(inout) :: grid
   \item type(ESMF\_State), intent(inout) :: state
   \end{description}
  
   Supported values for the <valueList> are:
   \begin{description}
   \item integer(ESMF\_KIND\_I4), intent(in) :: valueList(:)
   \item integer(ESMF\_KIND\_I8), intent(in) :: valueList(:)
   \item real (ESMF\_KIND\_R4), intent(in) :: valueList(:)
   \item real (ESMF\_KIND\_R8), intent(in) :: valueList(:)
   \item logical, intent(in) :: valueList(:)
   \item character (len = *), intent(in) :: valueList(:)
   \end{description}
  
   The arguments are:
   \begin{description}
   \item [<object>]
   An {\tt ESMF} object.
   \item [name]
   The name of the Attribute to set.
   \item [<valueList argument>]
   The valueList of the Attribute to set.
   \item [{[convention]}]
   The convention of the Attribute package.
   \item [{[purpose]}]
   The purpose of the Attribute package.
   \item [{[attPackInstanceName]}]
   The name of an Attribute package instance, specifying which one
   of multiple Attribute package instances of the same convention
   and purpose, within a nest. If not specified, defaults to the
   first instance. (Not implemented yet)
   \item [{[itemCount]}]
   The number of items in a multi-valued Attribute.
   \item [{[rc]}]
   Return code; equals {\tt ESMF\_SUCCESS} if there are no errors.
   \end{description}
  
   
%/////////////////////////////////////////////////////////////
 
\mbox{}\hrulefill\ 
 
\subsubsection [ESMF\_AttributeUpdate] {ESMF\_AttributeUpdate - Update an Attribute hierarchy}


  
\bigskip{\sf INTERFACE:}
\begin{verbatim}   subroutine ESMF_AttributeUpdate(<object>, vm, rootList, reconcile, rc)\end{verbatim}{\em ARGUMENTS:}
\begin{verbatim}   <object>, see below for supported values
   type(ESMF_VM), intent(in) :: vm
   integer, intent(in) :: rootList(:)
   logical, intent(in), optional :: reconcile
   integer, intent(out), optional :: rc\end{verbatim}
{\sf DESCRIPTION:\\ }


   Update an Attribute hierarchy during runtime. The information from
   the PETs in the {\tt rootList} is transferred to the PETs that are not
   in the {\tt rootList}. Care should be taken to ensure that the
   information contained in the Attributes on the PETs in the {\tt rootList}
   is consistent.
   If changes have been made to the underlying object hierarchy then either
   {\tt ESMF\_StateReconcile()} or the {\tt reconcile} flag must be used to
   resolve them. The same applies if changes are made to both PETs and in
   the {\tt rootList} and PETs outside of the {\tt rootList}, or if the same
   changes are made in a different order.
  
   Supported values for <object> are:
   \begin{description}
   \item type(ESMF\_Array), intent(inout) :: array
   \item type(ESMF\_ArrayBundle), intent(inout) :: arraybundle
   \item type(ESMF\_CplComp), intent(inout) :: comp
   \item type(ESMF\_GridComp), intent(inout) :: comp
   \item type(ESMF\_SciComp), intent(inout) :: comp
   \item type(ESMF\_Field), intent(inout) :: field
   \item type(ESMF\_FieldBundle), intent(inout) :: fieldbundle
   \item type(ESMF\_State), intent(inout) :: state
   \end{description}
  
   The arguments are:
   \begin{description}
   \item [<object>]
   An {\tt ESMF} object.
   \item [vm]
   The virtual machine over which this Attribute hierarchy
   should be updated.
   \item [rootList]
   The list of PETs that are to be used as the source of the update.
   \item [{[reconcile]}] 
   A logical flag used to indicate whether to use reconcile behavior 
   or normal update behavior. If {\tt reconcile} is set to 
   {\tt .true.} then the values of the root PETs will be sent to 
   the nonroot PETs without exception. Otherwise, an algorithm that 
   is optimized to use minimal memory will be used to update only 
   the modified parts of the Attribute hierarchy on the nonroot 
   PETs. The default value is {\tt .false.}. 
   \item [{[rc]}]
   Return code; equals {\tt ESMF\_SUCCESS} if there are no errors.
   \end{description}
  
   
%/////////////////////////////////////////////////////////////
 
\mbox{}\hrulefill\ 
 
\subsubsection [ESMF\_AttributeWrite] {ESMF\_AttributeWrite - Write an Attribute package}


   \label{api:AttributeWrite}
  
\bigskip{\sf INTERFACE:}
\begin{verbatim}   subroutine ESMF_AttributeWrite(<object>, convention, purpose, &
   attwriteflag, rc)\end{verbatim}{\em ARGUMENTS:}
\begin{verbatim}   <object>, see below for supported values
   character (len = *), intent(in), optional :: convention
   character (len = *), intent(in), optional :: purpose
   type(ESMF_AttWriteFlag), intent(in), optional :: attwriteflag
   integer, intent(out), optional :: rc\end{verbatim}
{\sf DESCRIPTION:\\ }


   Write the Attribute package for <object>. The Attribute package defines
   the convention, purpose, and object type of the associated Attributes. Either
   tab-delimited or xml format is achieved by using {\tt attwriteflag}.
   Currently, only ESMF/ESG/CF Field Attribute packages can be written in
   tab-delimited format. See Section~\ref{sec:AttPacks} for a description
   of Attribute packages and their conventions, purposes, and object types.
  
   This call is collective across the current VM.
  
   Writing Attribute XML files is performed with the standard C++ output
   file stream facility.
  
   Note: For an object type of {\tt ESMF\_GridComp}, convention='WaterML',
   purpose='TimeSeries', and
   \newline
   attwriteflag=ESMF\_ATTWRITE\_XML, an XML file
   conforming to a hydrologic standard called WaterML will be written. See
   the following for more information:
  
   \begin{description}
   \item{"http://his.cuahsi.org/wofws.html"}
   \item{"http://www.earthsystemcurator.org/projects/waterml.shtml"}
   \end{description}
  
   An ESMF Use Test Case is available which showcases an example of how
   to write a WaterML file; please see
  
   \begin{description}
   \item{"http://esmf.cvs.sourceforge.net/viewvc/esmf/use\_test\_cases/ESMF\_WaterML"}
   \item{"http://esmf.cvs.sourceforge.net/viewvc/esmf/use\_test\_cases/README"}
   \end{description}
  
   Supported values for <object> are:
   \begin{description}
   \item type(ESMF\_Array), intent(in) :: array
   \item type(ESMF\_ArrayBundle), intent(in) :: arraybundle
   \item type(ESMF\_CplComp), intent(in) :: comp
   \item type(ESMF\_GridComp), intent(in) :: comp
   \item type(ESMF\_SciComp), intent(in) :: comp
   \item type(ESMF\_DistGrid), intent(in) :: distgrid
   \item type(ESMF\_Field), intent(in) :: field
   \item type(ESMF\_FieldBundle), intent(in) :: fieldbundle
   \item type(ESMF\_Grid), intent(in) :: grid
   \item type(ESMF\_State), intent(in) :: state
   \end{description}
  
   The arguments are:
   \begin{description}
   \item [<object>]
   An {\tt ESMF} object.
   \item [{[convention]}]
   The convention of the Attribute package.
   \item [{[purpose]}]
   The purpose of the Attribute package.
   \item [{[attwriteflag]}]
   The flag to specify which format is desired for the write, the
   default is ESMF\_ATTWRITE\_TAB. This flag is documented in
   section \ref{const:attwrite}.
   \item [{[rc]}]
   Return code; equals {\tt ESMF\_SUCCESS} if there are no errors.
   \end{description}
  
%...............................................................
\setlength{\parskip}{\oldparskip}
\setlength{\parindent}{\oldparindent}
\setlength{\baselineskip}{\oldbaselineskip}
