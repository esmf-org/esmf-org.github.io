%                **** IMPORTANT NOTICE *****
% This LaTeX file has been automatically produced by ProTeX v. 1.1
% Any changes made to this file will likely be lost next time
% this file is regenerated from its source. Send questions 
% to Arlindo da Silva, dasilva@gsfc.nasa.gov
 
\setlength{\oldparskip}{\parskip}
\setlength{\parskip}{1.5ex}
\setlength{\oldparindent}{\parindent}
\setlength{\parindent}{0pt}
\setlength{\oldbaselineskip}{\baselineskip}
\setlength{\baselineskip}{11pt}
 
%--------------------- SHORT-HAND MACROS ----------------------
\def\bv{\begin{verbatim}}
\def\ev{\end{verbatim}}
\def\be{\begin{equation}}
\def\ee{\end{equation}}
\def\bea{\begin{eqnarray}}
\def\eea{\end{eqnarray}}
\def\bi{\begin{itemize}}
\def\ei{\end{itemize}}
\def\bn{\begin{enumerate}}
\def\en{\end{enumerate}}
\def\bd{\begin{description}}
\def\ed{\end{description}}
\def\({\left (}
\def\){\right )}
\def\[{\left [}
\def\]{\right ]}
\def\<{\left  \langle}
\def\>{\right \rangle}
\def\cI{{\cal I}}
\def\diag{\mathop{\rm diag}}
\def\tr{\mathop{\rm tr}}
%-------------------------------------------------------------

\markboth{Left}{Source File: ESMF\_AttributeCIMEx.F90,  Date: Tue May  5 21:00:14 MDT 2020
}

 
%/////////////////////////////////////////////////////////////

   \subsubsection{CIM Attribute packages}
   \label{sec:attribute:usage:cimAttPack}
  
  \begin{sloppypar}
   This example illustrates the use of the Metafor CIM Attribute packages,
   supplied by ESMF, to create an Attribute hierarchy on an ESMF object tree.
   Gridded, coupler and science Components are used together with a State
   and a realistic Field
   to create a simple ESMF object tree.  CIM Attributes packages are created
   on the Components and Field, and then the individual Attributes within the
   packages are populated with values.  Finally, all the Attributes are written
   to a CIM-formatted XML file.  For a more comprehensive example, see the
   ESMF\_AttributeCIM system test.
  \end{sloppypar} 
%/////////////////////////////////////////////////////////////

 \begin{verbatim}
      ! Use ESMF framework module
      use ESMF
      use ESMF_TestMod
      implicit none

      ! Local variables
      integer                 :: rc, finalrc, petCount, localPet, result
          type(ESMF_AttPack)      :: attpack
      type(ESMF_VM)           :: vm
      type(ESMF_Field)        :: ozone
      type(ESMF_State)        :: exportState
      type(ESMF_CplComp)      :: cplcomp
      type(ESMF_GridComp)     :: gridcomp
      type(ESMF_SciComp)      :: scicomp
      character(ESMF_MAXSTR)  :: convCIM, purpComp, purpProp, purpSci
      character(ESMF_MAXSTR)  :: purpField, purpPlatform
      character(ESMF_MAXSTR)  :: convISO, purpRP, purpCitation
      character(ESMF_MAXSTR), dimension(2)  :: compPropAtt
      character(ESMF_MAXSTR), dimension(2)  :: rad_sciPropAtt
      character(ESMF_MAXSTR)  :: testname
      character(ESMF_MAXSTR)  :: failMsg
 
\end{verbatim}
 
%/////////////////////////////////////////////////////////////

 \begin{verbatim}


      ! initialize ESMF
      finalrc = ESMF_SUCCESS
      call ESMF_Initialize(vm=vm, defaultlogfilename="AttributeCIMEx.Log", &
        logkindflag=ESMF_LOGKIND_MULTI, rc=rc)
      if (rc /= ESMF_SUCCESS) call ESMF_Finalize(endflag=ESMF_END_ABORT)

      ! get the vm
      call ESMF_VMGet(vm, petCount=petCount, localPet=localPet, rc=rc)
      if (rc /= ESMF_SUCCESS) call ESMF_Finalize(endflag=ESMF_END_ABORT)
 
\end{verbatim}
 
%/////////////////////////////////////////////////////////////

  \begin{sloppypar}
      Create the ESMF objects that will hold the CIM Attributes.
      These objects include all three Component types (coupler, gridded,
      and science Components) as well as a State, and a Field.
      In this example we are constructing empty Fields without an
      underlying Grid.
  \end{sloppypar} 
%/////////////////////////////////////////////////////////////

 \begin{verbatim}
      ! Create top-level Coupler Component
      cplcomp = ESMF_CplCompCreate(name="coupler_component", &
        petList=(/0/), rc=rc)

 
\end{verbatim}
 
%/////////////////////////////////////////////////////////////

 \begin{verbatim}

      ! Create Gridded Component as a child of the Coupler Component
      gridcomp = ESMF_GridCompCreate(name="gridded_component", &
        petList=(/0/), rc=rc)

      call ESMF_AttributeLink(cplcomp, gridcomp, rc=rc)

 
\end{verbatim}
 
%/////////////////////////////////////////////////////////////

 \begin{verbatim}

      ! Create Science Component as a child of the Gridded Component
      scicomp = ESMF_SciCompCreate(name="science_component", rc=rc)

      call ESMF_AttributeLink(gridcomp, scicomp, rc=rc)

 
\end{verbatim}
 
%/////////////////////////////////////////////////////////////

 \begin{verbatim}

      ! Create State
      exportState = ESMF_StateCreate(name="exportState",  &
        stateintent=ESMF_STATEINTENT_EXPORT, rc=rc)
 
\end{verbatim}
 
%/////////////////////////////////////////////////////////////

 \begin{verbatim}


      ! Create Field
      ozone = ESMF_FieldEmptyCreate(name='ozone', rc=rc)
 
\end{verbatim}
 
%/////////////////////////////////////////////////////////////

 \begin{verbatim}
      convCIM = 'CIM 1.5'
      purpComp = 'ModelComp'
      purpProp = 'CompProp'
      purpSci = 'SciProp'
      purpField = 'Inputs'
      purpPlatform = 'Platform'

      convISO = 'ISO 19115'
      purpRP = 'RespParty'
      purpCitation = 'Citation'
 
\end{verbatim}
 
%/////////////////////////////////////////////////////////////

  \begin{sloppypar}
      Add CIM Component package and Attributes to the Coupler Component.
  \end{sloppypar} 
%/////////////////////////////////////////////////////////////

 \begin{verbatim}
      call ESMF_AttributeAdd(cplcomp,  &
                             convention=convCIM, purpose=purpComp, rc=rc)

      call ESMF_AttributeSet(cplcomp, "ShortName", "Driver", &
                             convention=convCIM, purpose=purpComp, rc=rc)
      call ESMF_AttributeSet(cplcomp, "LongName", &
                             "Model Driver", &
                             convention=convCIM, purpose=purpComp, rc=rc)
      call ESMF_AttributeSet(cplcomp, "ModelType", &
                             "climate", &
                             convention=convCIM, purpose=purpComp, rc=rc)
 
\end{verbatim}
 
%/////////////////////////////////////////////////////////////

 \begin{verbatim}


      ! Simulation run attributes
      call ESMF_AttributeSet(cplcomp, 'SimulationShortName', &
                                      'SMS.f09_g16.X.hector', &
        convention=convCIM, purpose=purpComp, rc=rc)

      call ESMF_AttributeSet(cplcomp, 'SimulationLongName', &
        'EarthSys - Earth System Modeling Framework Earth System Model 1.0', &
        convention=convCIM, purpose=purpComp, rc=rc)

      call ESMF_AttributeSet(cplcomp, 'SimulationRationale', &
  'EarthSys-ESMF simulation run in respect to CMIP5 core experiment 1.1 ()', &
        convention=convCIM, purpose=purpComp, rc=rc)

      call ESMF_AttributeSet(cplcomp, 'SimulationStartDate', &
                                       '1960-01-01T00:00:00Z', &
        convention=convCIM, purpose=purpComp, rc=rc)

      call ESMF_AttributeSet(cplcomp, 'SimulationDuration', 'P10Y', &
        convention=convCIM, purpose=purpComp, rc=rc)

      call ESMF_AttributeSet(cplcomp, &
         'SimulationNumberOfProcessingElements', '16', &
          convention=convCIM, purpose=purpComp, rc=rc)
 
\end{verbatim}
 
%/////////////////////////////////////////////////////////////

 \begin{verbatim}

      call ESMF_AttributeGetAttPack(cplcomp, convCIM, purpPlatform, &
        attpack=attpack, rc=rc)

      call ESMF_AttributeSet(cplcomp, 'MachineName', 'HECToR', &
        convention=convCIM, purpose=purpPlatform, rc=rc)
 
\end{verbatim}
 
%/////////////////////////////////////////////////////////////

  \begin{sloppypar}
      Now add CIM Attribute packages and Attributes to the Gridded Component
      and Field.  Also, add a CIM Component Properties package, to contain
      two custom attributes.
  \end{sloppypar} 
%/////////////////////////////////////////////////////////////

 \begin{verbatim}
      ! Add CIM Attribute package to the gridded Component
      call ESMF_AttributeAdd(gridcomp, convention=convCIM, &
        purpose=purpComp, rc=rc)
 
\end{verbatim}
 
%/////////////////////////////////////////////////////////////

 \begin{verbatim}


      ! Specify the gridded Component to have a Component Properties
      ! package with two custom attributes, with user-specified names
      compPropAtt(1) = 'SimulationType'
      compPropAtt(2) = 'SimulationURL'
      call ESMF_AttributeAdd(gridcomp, convention=convCIM, purpose=purpProp, &
        attrList=compPropAtt, rc=rc)
 
\end{verbatim}
 
%/////////////////////////////////////////////////////////////

 \begin{verbatim}


      ! Add CIM Attribute package to the Field
      call ESMF_AttributeAdd(ozone, convention=convCIM, purpose=purpField, &
        rc=rc)
 
\end{verbatim}
 
%/////////////////////////////////////////////////////////////

  \begin{sloppypar}
       The standard Attribute package supplied by ESMF for a CIM Component
       contains several Attributes, grouped into sub-packages.  These
       Attributes conform to the CIM convention as defined by Metafor and
       their values are set individually.
  \end{sloppypar} 
%/////////////////////////////////////////////////////////////

 \begin{verbatim}
      !
      ! Top-level model component attributes, set on gridded component
      !
      call ESMF_AttributeSet(gridcomp, 'ShortName', 'EarthSys_Atmos', &
        convention=convCIM, purpose=purpComp, rc=rc)
 
\end{verbatim}
 
%/////////////////////////////////////////////////////////////

 \begin{verbatim}

      call ESMF_AttributeSet(gridcomp, 'LongName', &
        'Earth System High Resolution Global Atmosphere Model', &
        convention=convCIM, purpose=purpComp, rc=rc)
 
\end{verbatim}
 
%/////////////////////////////////////////////////////////////

 \begin{verbatim}

      call ESMF_AttributeSet(gridcomp, 'Description', &
        'EarthSys brings together expertise from the global ' // &
        'community in a concerted effort to develop coupled ' // &
        'climate models with increased horizontal resolutions.  ' // &
        'Increasing the horizontal resolution of coupled climate ' // &
        'models will allow us to capture climate processes and ' // &
        'weather systems in much greater detail.', &
        convention=convCIM, purpose=purpComp, rc=rc)
 
\end{verbatim}
 
%/////////////////////////////////////////////////////////////

 \begin{verbatim}

      call ESMF_AttributeSet(gridcomp, 'Version', '2.0', &
        convention=convCIM, purpose=purpComp, rc=rc)
 
\end{verbatim}
 
%/////////////////////////////////////////////////////////////

 \begin{verbatim}

      call ESMF_AttributeSet(gridcomp, 'ReleaseDate', '2009-01-01T00:00:00Z', &
        convention=convCIM, purpose=purpComp, rc=rc)
 
\end{verbatim}
 
%/////////////////////////////////////////////////////////////

 \begin{verbatim}

      call ESMF_AttributeSet(gridcomp, 'ModelType', 'aerosol', &
        convention=convCIM, purpose=purpComp, rc=rc)
 
\end{verbatim}
 
%/////////////////////////////////////////////////////////////

 \begin{verbatim}

      call ESMF_AttributeSet(gridcomp, 'URL', &
        'www.earthsys.org', convention=convCIM, purpose=purpComp, rc=rc)
 
\end{verbatim}
 
%/////////////////////////////////////////////////////////////

 \begin{verbatim}

      call ESMF_AttributeSet(gridcomp, 'MetadataVersion', '1.1', &
        convention=convCIM, purpose=purpComp, rc=rc)
 
\end{verbatim}
 
%/////////////////////////////////////////////////////////////

 \begin{verbatim}


      ! Document genealogy
      call ESMF_AttributeSet(gridcomp, 'PreviousVersion', &
                                       'EarthSys1 Atmosphere', &
        convention=convCIM, purpose=purpComp, rc=rc)
 
\end{verbatim}
 
%/////////////////////////////////////////////////////////////

 \begin{verbatim}

      call ESMF_AttributeSet(gridcomp, 'PreviousVersionDescription', &
       'Horizontal resolution increased to 1.20 x 0.80 degrees; ' // &
       'Timestep reduced from 30 minutes to 15 minutes.', &
        convention=convCIM, purpose=purpComp, rc=rc)
 
\end{verbatim}
 
%/////////////////////////////////////////////////////////////

 \begin{verbatim}

      call ESMF_AttributeGetAttPack(gridcomp, convCIM, purpPlatform, &
        attpack=attpack, rc=rc)

      ! Platform description attributes
      call ESMF_AttributeSet(gridcomp, 'CompilerName', 'Pathscale', &
        convention=convCIM, purpose=purpPlatform, rc=rc)
 
\end{verbatim}
 
%/////////////////////////////////////////////////////////////

 \begin{verbatim}

      call ESMF_AttributeSet(gridcomp, 'CompilerVersion', '3.0', &
        convention=convCIM, purpose=purpPlatform, rc=rc)
 
\end{verbatim}
 
%/////////////////////////////////////////////////////////////

 \begin{verbatim}

      call ESMF_AttributeSet(gridcomp, 'MachineName', 'HECToR', &
        convention=convCIM, purpose=purpPlatform, rc=rc)
 
\end{verbatim}
 
%/////////////////////////////////////////////////////////////

 \begin{verbatim}

      call ESMF_AttributeSet(gridcomp, 'MachineDescription', &
        'HECToR (Phase 2a) is currently an integrated system known ' // &
        'as Rainier, which includes a scalar MPP XT4 system, a vector ' // &
        'system known as BlackWidow, and storage systems.', &
        convention=convCIM, purpose=purpPlatform, rc=rc)
 
\end{verbatim}
 
%/////////////////////////////////////////////////////////////

 \begin{verbatim}

      call ESMF_AttributeSet(gridcomp, 'MachineSystem', 'Parallel', &
        convention=convCIM, purpose=purpPlatform, rc=rc)
      call ESMF_AttributeSet(gridcomp, 'MachineOperatingSystem', 'Unicos', &
        convention=convCIM, purpose=purpPlatform, rc=rc)
      call ESMF_AttributeSet(gridcomp, 'MachineVendor', 'Cray Inc', &
        convention=convCIM, purpose=purpPlatform, rc=rc)
      call ESMF_AttributeSet(gridcomp, 'MachineInterconnectType', &
                                       'Cray Interconnect', &
        convention=convCIM, purpose=purpPlatform, rc=rc)
 
\end{verbatim}
 
%/////////////////////////////////////////////////////////////

 \begin{verbatim}

      call ESMF_AttributeSet(gridcomp, 'MachineMaximumProcessors', '22656', &
        convention=convCIM, purpose=purpPlatform, rc=rc)
 
\end{verbatim}
 
%/////////////////////////////////////////////////////////////

 \begin{verbatim}

      call ESMF_AttributeSet(gridcomp, 'MachineCoresPerProcessor', '4', &
        convention=convCIM, purpose=purpPlatform, rc=rc)
 
\end{verbatim}
 
%/////////////////////////////////////////////////////////////

 \begin{verbatim}

      call ESMF_AttributeSet(gridcomp, 'MachineProcessorType', 'AMD X86_64', &
        convention=convCIM, purpose=purpPlatform, rc=rc)
 
\end{verbatim}
 
%/////////////////////////////////////////////////////////////

 \begin{verbatim}

      call ESMF_AttributeGetAttPack(gridcomp, convCIM, purpProp, &
         attpack=attpack, rc=rc)

      ! Component Properties: custom attributes
      call ESMF_AttributeSet(gridcomp, 'SimulationType', 'branch', &
        convention=convCIM, purpose=purpProp, rc=rc)
 
\end{verbatim}
 
%/////////////////////////////////////////////////////////////

 \begin{verbatim}

      call ESMF_AttributeSet(gridcomp, 'SimulationURL', &
                                       'http://earthsys.org/simulations', &
        convention=convCIM, purpose=purpProp, rc=rc)

 
\end{verbatim}
 
%/////////////////////////////////////////////////////////////

  \begin{sloppypar}
      Set the attribute values of the Responsible Party sub-package, created
      above for the gridded Component in the ESMF\_AttributeAdd(gridcomp, ...)
      call.
  \end{sloppypar} 
%/////////////////////////////////////////////////////////////

 \begin{verbatim}
      call ESMF_AttributeGetAttPack(gridcomp, convISO, purpRP, &
        attpack=attpack, rc=rc)

      ! Responsible party attributes (for Principal Investigator)
      call ESMF_AttributeSet(gridcomp, 'Name', 'John Doe', &
        convention=convISO, purpose=purpRP, rc=rc)
      call ESMF_AttributeSet(gridcomp, 'Abbreviation', 'JD', &
        convention=convISO, purpose=purpRP, rc=rc)
      call ESMF_AttributeSet(gridcomp, 'PhysicalAddress', &
          'Department of Meteorology, University of ABC', &
        convention=convISO, purpose=purpRP, rc=rc)
      call ESMF_AttributeSet(gridcomp, 'EmailAddress', &
                                       'john.doe@earthsys.org', &
        convention=convISO, purpose=purpRP, rc=rc)
      call ESMF_AttributeSet(gridcomp, 'ResponsiblePartyRole', 'PI', &
        convention=convISO, purpose=purpRP, rc=rc)
 
\end{verbatim}
 
%/////////////////////////////////////////////////////////////

 \begin{verbatim}

      call ESMF_AttributeSet(gridcomp, 'URL', 'www.earthsys.org', &
        convention=convISO, purpose=purpRP, rc=rc)

 
\end{verbatim}
 
%/////////////////////////////////////////////////////////////

  \begin{sloppypar}
      Set the attribute values of the Citation sub-package, created above
      for the gridded Component in the ESMF\_AttributeAdd(gridcomp, ...) call.
  \end{sloppypar} 
%/////////////////////////////////////////////////////////////

 \begin{verbatim}
      call ESMF_AttributeGetAttPack(gridcomp, convISO, purpCitation, &
        attpack=attpack, rc=rc)

      ! Citation attributes
      call ESMF_AttributeSet(gridcomp, 'ShortTitle', 'Doe_2009', &
        convention=convISO, purpose=purpCitation, rc=rc)
      call ESMF_AttributeSet(gridcomp, 'LongTitle', &
       'Doe, J.A.; Norton, A.B.; ' // &
       'Clark, G.H.; Davies, I.J.. 2009 EarthSys: ' // &
       'The Earth System High Resolution Global Atmosphere Model - Model ' // &
       'description and basic evaluation. Journal of Climate, 15 (2). ' // &
       '1261-1296.', &
        convention=convISO, purpose=purpCitation, rc=rc)
 
\end{verbatim}
 
%/////////////////////////////////////////////////////////////

 \begin{verbatim}

      call ESMF_AttributeSet(gridcomp, 'Date', '2010-03-15', &
        convention=convISO, purpose=purpCitation, rc=rc)
 
\end{verbatim}
 
%/////////////////////////////////////////////////////////////

 \begin{verbatim}

      call ESMF_AttributeSet(gridcomp, 'PresentationForm', 'Online Refereed', &
        convention=convISO, purpose=purpCitation, rc=rc)
 
\end{verbatim}
 
%/////////////////////////////////////////////////////////////

 \begin{verbatim}

      call ESMF_AttributeSet(gridcomp, 'DOI', 'doi:17.1035/2009JCLI4508.1', &
        convention=convISO, purpose=purpCitation, rc=rc)
 
\end{verbatim}
 
%/////////////////////////////////////////////////////////////

 \begin{verbatim}

      call ESMF_AttributeSet(gridcomp, 'URL', &
                             'http://www.earthsys.org/publications', &
        convention=convISO, purpose=purpCitation, rc=rc)

 
\end{verbatim}
 
%/////////////////////////////////////////////////////////////

  \begin{sloppypar}
       Add Component attributes to the Science Component and then add
       scientific properties to it.
  \end{sloppypar} 
%/////////////////////////////////////////////////////////////

 \begin{verbatim}
      call ESMF_AttributeAdd(scicomp,  &
                             convention=convCIM, purpose=purpComp, rc=rc)

      call ESMF_AttributeSet(scicomp, "ShortName", "AtmosRadiation", &
                             convention=convCIM, purpose=purpComp, rc=rc)
      call ESMF_AttributeSet(scicomp, "LongName", &
                             "Atmosphere Radiation", &
                             convention=convCIM, purpose=purpComp, rc=rc)
      call ESMF_AttributeSet(scicomp, "ModelType", &
                             "radiation", &
                             convention=convCIM, purpose=purpComp, rc=rc)
 
\end{verbatim}
 
%/////////////////////////////////////////////////////////////

 \begin{verbatim}
      rad_sciPropAtt(1) = 'LongwaveSchemeType'
      rad_sciPropAtt(2) = 'LongwaveSchemeMethod'

      call ESMF_AttributeAdd(scicomp,  &
                             convention=convCIM, purpose=purpSci, &
                             attrList=rad_sciPropAtt, rc=rc)

      call ESMF_AttributeSet(scicomp, &
                             'LongwaveSchemeType', &
                             'wide-band model', &
                             convention=convCIM, purpose=purpSci, rc=rc)
      call ESMF_AttributeSet(scicomp, &
                             'LongwaveSchemeMethod', &
                             'two-stream', &
                             convention=convCIM, purpose=purpSci, rc=rc)
 
\end{verbatim}
 
%/////////////////////////////////////////////////////////////

  \begin{sloppypar}
       The standard Attribute package currently supplied by ESMF for
       CIM Fields contains a standard CF-Extended package nested within it.
  \end{sloppypar} 
%/////////////////////////////////////////////////////////////

 \begin{verbatim}
      call ESMF_AttributeGetAttPack(ozone, convCIM, purpField, &
        attpack=attpack, rc=rc)

      ! ozone CF-Extended Attributes
      call ESMF_AttributeSet(ozone, 'ShortName', 'Global_O3_mon', &
       convention=convCIM, purpose=purpField, rc=rc)
      call ESMF_AttributeSet(ozone, 'StandardName', 'ozone', &
       convention=convCIM, purpose=purpField, rc=rc)
      call ESMF_AttributeSet(ozone, 'LongName', 'ozone', &
       convention=convCIM, purpose=purpField, rc=rc)
      call ESMF_AttributeSet(ozone, 'Units', 'unknown', &
       convention=convCIM, purpose=purpField, rc=rc)
 
\end{verbatim}
 
%/////////////////////////////////////////////////////////////

 \begin{verbatim}


      ! ozone CIM Attributes
      call ESMF_AttributeSet(ozone, 'CouplingPurpose', 'Boundary', &
       convention=convCIM, purpose=purpField, rc=rc)
 
\end{verbatim}
 
%/////////////////////////////////////////////////////////////

 \begin{verbatim}

      call ESMF_AttributeSet(ozone, 'CouplingSource', 'EarthSys_Atmos', &
       convention=convCIM, purpose=purpField, rc=rc)
 
\end{verbatim}
 
%/////////////////////////////////////////////////////////////

 \begin{verbatim}

      call ESMF_AttributeSet(ozone, 'CouplingTarget', &
       'EarthSys_AtmosDynCore', convention=convCIM, &
        purpose=purpField, rc=rc)
 
\end{verbatim}
 
%/////////////////////////////////////////////////////////////

 \begin{verbatim}

      call ESMF_AttributeSet(ozone, 'Description', &
                                    'Global Ozone concentration ' // &
                                    'monitoring in the atmosphere.', &
       convention=convCIM, purpose=purpField, rc=rc)
 
\end{verbatim}
 
%/////////////////////////////////////////////////////////////

 \begin{verbatim}

      call ESMF_AttributeSet(ozone, 'SpatialRegriddingMethod', &
                                    'Conservative-First-Order', &
       convention=convCIM, purpose=purpField, rc=rc)
 
\end{verbatim}
 
%/////////////////////////////////////////////////////////////

 \begin{verbatim}

      call ESMF_AttributeSet(ozone, 'SpatialRegriddingDimension', '3D', &
       convention=convCIM, purpose=purpField, rc=rc)
 
\end{verbatim}
 
%/////////////////////////////////////////////////////////////

 \begin{verbatim}

      call ESMF_AttributeSet(ozone, 'Frequency', '15 Minutes', &
       convention=convCIM, purpose=purpField, rc=rc)
 
\end{verbatim}
 
%/////////////////////////////////////////////////////////////

 \begin{verbatim}

      call ESMF_AttributeSet(ozone, 'TimeTransformationType', &
                                    'TimeInterpolation', &
       convention=convCIM, purpose=purpField, rc=rc)

 
\end{verbatim}
 
%/////////////////////////////////////////////////////////////

  \begin{sloppypar}
       Adding the Field to the State will automatically link the
       Attribute hierarchies from the State to the Field
  \end{sloppypar} 
%/////////////////////////////////////////////////////////////

 \begin{verbatim}
      ! Add the Field directly to the State
      call ESMF_StateAdd(exportState, fieldList=(/ozone/), rc=rc)

 
\end{verbatim}
 
%/////////////////////////////////////////////////////////////

  \begin{sloppypar}
       The Attribute link between a Component and a State must be set manually.
  \end{sloppypar} 
%/////////////////////////////////////////////////////////////

 \begin{verbatim}
      ! Link the State to the gridded Component
      call ESMF_AttributeLink(gridcomp, exportState, rc=rc)

 
\end{verbatim}
 
%/////////////////////////////////////////////////////////////

  \begin{sloppypar}
       Write the entire CIM Attribute hierarchy, beginning at the gridded
       Component (the top), to an XML file formatted to conform to CIM
       specifications.  The CIM output tree structure differs from the
       internal Attribute hierarchy in that it has all the attributes of
       the fields within its top-level <modelComponent> record.  The filename
       used, gridded\_component.xml, is derived from the name of the gridded
       Component, given as an input argument in the ESMF\_GridCompCreate()
       call above.  The file is written to the examples execution directory.
  \end{sloppypar} 
%/////////////////////////////////////////////////////////////

 \begin{verbatim}
      call ESMF_AttributeWrite(cplcomp, convCIM, purpComp, &
        attwriteflag=ESMF_ATTWRITE_XML,rc=rc)
 
\end{verbatim}
 
 
%/////////////////////////////////////////////////////////////

 \begin{verbatim}

      call ESMF_StateDestroy(exportState, rc=rc)
 
\end{verbatim}
 
%/////////////////////////////////////////////////////////////

 \begin{verbatim}

      call ESMF_SciCompDestroy(scicomp, rc=rc)
      call ESMF_GridCompDestroy(gridcomp, rc=rc)
      call ESMF_CplCompDestroy(cplcomp, rc=rc)
 
\end{verbatim}
 
%/////////////////////////////////////////////////////////////

 \begin{verbatim}

      call ESMF_Finalize(rc=rc)
 
\end{verbatim}

%...............................................................
\setlength{\parskip}{\oldparskip}
\setlength{\parindent}{\oldparindent}
\setlength{\baselineskip}{\oldbaselineskip}
