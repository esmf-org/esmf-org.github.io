%                **** IMPORTANT NOTICE *****
% This LaTeX file has been automatically produced by ProTeX v. 1.1
% Any changes made to this file will likely be lost next time
% this file is regenerated from its source. Send questions 
% to Arlindo da Silva, dasilva@gsfc.nasa.gov
 
\setlength{\oldparskip}{\parskip}
\setlength{\parskip}{1.5ex}
\setlength{\oldparindent}{\parindent}
\setlength{\parindent}{0pt}
\setlength{\oldbaselineskip}{\baselineskip}
\setlength{\baselineskip}{11pt}
 
%--------------------- SHORT-HAND MACROS ----------------------
\def\bv{\begin{verbatim}}
\def\ev{\end{verbatim}}
\def\be{\begin{equation}}
\def\ee{\end{equation}}
\def\bea{\begin{eqnarray}}
\def\eea{\end{eqnarray}}
\def\bi{\begin{itemize}}
\def\ei{\end{itemize}}
\def\bn{\begin{enumerate}}
\def\en{\end{enumerate}}
\def\bd{\begin{description}}
\def\ed{\end{description}}
\def\({\left (}
\def\){\right )}
\def\[{\left [}
\def\]{\right ]}
\def\<{\left  \langle}
\def\>{\right \rangle}
\def\cI{{\cal I}}
\def\diag{\mathop{\rm diag}}
\def\tr{\mathop{\rm tr}}
%-------------------------------------------------------------

\markboth{Left}{Source File: ESMC\_Init.h,  Date: Tue May  5 21:00:19 MDT 2020
}

 
%/////////////////////////////////////////////////////////////
\subsubsection [ESMC\_Initialize] {ESMC\_Initialize - Initialize ESMF}


  
\bigskip{\sf INTERFACE:}
\begin{verbatim}   int ESMC_Initialize(
     int *rc,        // return code
     ...);           // optional arguments (see below)
 \end{verbatim}{\em RETURN VALUE:}
\begin{verbatim}    Return code; equals ESMF_SUCCESS if there are no errors.\end{verbatim}
{\sf DESCRIPTION:\\ }


    Initialize the ESMF.  This method must be called before
    any other ESMF methods are used.  The method contains a
    barrier before returning, ensuring that all processes
    made it successfully through initialization.
  
    Typically {\tt ESMC\_Initialize()} will call {\tt MPI\_Init()} 
    internally unless MPI has been initialized by the user code before
    initializing the framework. If the MPI initialization is left to
    {\tt ESMC\_Initialize()} it inherits all of the MPI implementation 
    dependent limitations of what may or may not be done before 
    {\tt MPI\_Init()}. For instance, it is unsafe for some MPI implementations,
    such as MPICH, to do I/O before the MPI environment is initialized. Please
    consult the documentation of your MPI implementation for details.
  
    Optional arguments are recognised.  To indicate the end of the optional
    argument list, {\tt ESMC\_ArgLast} must be used.  A minimal call to
    {\tt ESMC\_Initialize()} would be:
   \begin{verbatim}
      ESMC_Initialize (NULL, ESMC_ArgLast);\end{verbatim}
    The optional arguments are specified using the {\tt ESMC\_InitArg} macros.
    For example, to turn off logging so that no log files would be created, the
    {\tt ESMC\_Initialize()} call would be coded as:
   \begin{verbatim}
      ESMC_Initialize (&rc,
        ESMC_InitArgLogKindFlag(ESMC_LOGKIND_NONE),
        ESMC_ArgLast);\end{verbatim}
    Before exiting the application the user must call {\tt ESMC\_Finalize()}
    to release resources and clean up the ESMF gracefully.
  
    The arguments are:
    \begin{description}
    \item [{[rc]}]
      Return code; equals {\tt ESMF\_SUCCESS} if there are no errors.
      {\tt NULL} may be passed when the return code is not desired.
    \item [{[ESMC\_InitArgDefaultCalKind(ARG)]}]
      Macro specifying the default calendar kind for the entire
      application.  Valid values for {\tt ARG} are documented in section
      \ref{const:calkindflag_c}.
      If not specified, defaults to {\tt ESMC\_CALKIND\_NOCALENDAR}.
    \item [{[ESMC\_InitArgDefaultConfigFilename(ARG)]}]
      Macro specifying the name of the default configuration file for the
      Config class.  If not specified, no default file is used.
    \item [{[ESMC\_InitArgLogFilename(ARG)]}]
      Macro specifying the name used as part of the default log file name for
      the default log.  If not specified, defaults to {\tt ESMF\_LogFile}.
    \item [{[ESMC\_InitArgLogKindFlag(ARG)]}]
      Macro specifying the default Log kind to be used by ESMF Log Manager.
      Valid values for {\tt ARG} are  documented in section
      \ref{const:clogkindflag}.
      If not specified, defaults to {\tt ESMC\_LOGKIND\_MULTI}.
    \item [ESMC\_ArgLast]
      Macro indicating the end of the optional argument list.  This must be
      provided even when there are no optional arguments.
    \end{description} 
%/////////////////////////////////////////////////////////////
 
\mbox{}\hrulefill\ 
 
\subsubsection [ESMC\_Finalize] {ESMC\_Finalize - Finalize the ESMF Framework}


  
\bigskip{\sf INTERFACE:}
\begin{verbatim}   int ESMC_Finalize(void);
 \end{verbatim}{\em RETURN VALUE:}
\begin{verbatim}    Return code; equals ESMF_SUCCESS if there are no errors.\end{verbatim}
{\sf DESCRIPTION:\\ }


   This must be called once on each PET before the application exits to
   allow ESMF to flush buffers, close open connections, and release
   internal resources cleanly.
%...............................................................
\setlength{\parskip}{\oldparskip}
\setlength{\parindent}{\oldparindent}
\setlength{\baselineskip}{\oldbaselineskip}
