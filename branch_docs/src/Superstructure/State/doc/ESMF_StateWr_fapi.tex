%                **** IMPORTANT NOTICE *****
% This LaTeX file has been automatically produced by ProTeX v. 1.1
% Any changes made to this file will likely be lost next time
% this file is regenerated from its source. Send questions 
% to Arlindo da Silva, dasilva@gsfc.nasa.gov
 
\setlength{\oldparskip}{\parskip}
\setlength{\parskip}{1.5ex}
\setlength{\oldparindent}{\parindent}
\setlength{\parindent}{0pt}
\setlength{\oldbaselineskip}{\baselineskip}
\setlength{\baselineskip}{11pt}
 
%--------------------- SHORT-HAND MACROS ----------------------
\def\bv{\begin{verbatim}}
\def\ev{\end{verbatim}}
\def\be{\begin{equation}}
\def\ee{\end{equation}}
\def\bea{\begin{eqnarray}}
\def\eea{\end{eqnarray}}
\def\bi{\begin{itemize}}
\def\ei{\end{itemize}}
\def\bn{\begin{enumerate}}
\def\en{\end{enumerate}}
\def\bd{\begin{description}}
\def\ed{\end{description}}
\def\({\left (}
\def\){\right )}
\def\[{\left [}
\def\]{\right ]}
\def\<{\left  \langle}
\def\>{\right \rangle}
\def\cI{{\cal I}}
\def\diag{\mathop{\rm diag}}
\def\tr{\mathop{\rm tr}}
%-------------------------------------------------------------

\markboth{Left}{Source File: ESMF\_StateWr.F90,  Date: Tue May  5 21:00:08 MDT 2020
}

 
%/////////////////////////////////////////////////////////////
\subsubsection [ESMF\_StateWrite] {ESMF\_StateWrite -- Write items from a State to file}


  
\bigskip{\sf INTERFACE:}
\begin{verbatim}       subroutine ESMF_StateWrite(state, fileName, rc)\end{verbatim}{\em ARGUMENTS:}
\begin{verbatim}       type(ESMF_State),  intent(in)            :: state 
       character (len=*), intent(in)            :: fileName
       integer,           intent(out), optional :: rc \end{verbatim}
{\sf DESCRIPTION:\\ }


       Currently limited to write out all Arrays of a State object to a
       netCDF file.  Future releases will enable more item types of a State to
       be written to files of various formats.
  
       Writing is currently limited to PET 0; future versions of ESMF will allow
       parallel writing, as well as parallel reading.
  
       See Section~\ref{example:StateRdWr} for an example.
  
       Note that the third party NetCDF library must be installed.  For more
       details, see the "ESMF Users Guide",
       "Building and Installing the ESMF, Third Party Libraries, NetCDF" and
       the website http://www.unidata.ucar.edu/software/netcdf.
  
       The arguments are:
       \begin{description}
       \item[state]
         The {\tt ESMF\_State} from which to write items.  Currently limited to
         Arrays.
       \item[fileName]
         File to be written.  
       \item[{[rc]}]
         Return code; equals {\tt ESMF\_SUCCESS} if there are no errors.
         Equals {\tt ESMF\_RC\_LIB\_NOT\_PRESENT} if the NetCDF library is
         not present.
       \end{description}
  
%...............................................................
\setlength{\parskip}{\oldparskip}
\setlength{\parindent}{\oldparindent}
\setlength{\baselineskip}{\oldbaselineskip}
