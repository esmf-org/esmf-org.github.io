%                **** IMPORTANT NOTICE *****
% This LaTeX file has been automatically produced by ProTeX v. 1.1
% Any changes made to this file will likely be lost next time
% this file is regenerated from its source. Send questions 
% to Arlindo da Silva, dasilva@gsfc.nasa.gov
 
\setlength{\oldparskip}{\parskip}
\setlength{\parskip}{1.5ex}
\setlength{\oldparindent}{\parindent}
\setlength{\parindent}{0pt}
\setlength{\oldbaselineskip}{\baselineskip}
\setlength{\baselineskip}{11pt}
 
%--------------------- SHORT-HAND MACROS ----------------------
\def\bv{\begin{verbatim}}
\def\ev{\end{verbatim}}
\def\be{\begin{equation}}
\def\ee{\end{equation}}
\def\bea{\begin{eqnarray}}
\def\eea{\end{eqnarray}}
\def\bi{\begin{itemize}}
\def\ei{\end{itemize}}
\def\bn{\begin{enumerate}}
\def\en{\end{enumerate}}
\def\bd{\begin{description}}
\def\ed{\end{description}}
\def\({\left (}
\def\){\right )}
\def\[{\left [}
\def\]{\right ]}
\def\<{\left  \langle}
\def\>{\right \rangle}
\def\cI{{\cal I}}
\def\diag{\mathop{\rm diag}}
\def\tr{\mathop{\rm tr}}
%-------------------------------------------------------------

\markboth{Left}{Source File: ESMF\_StateReadWriteEx.F90,  Date: Tue May  5 21:00:09 MDT 2020
}

 
%/////////////////////////////////////////////////////////////

  \subsubsection{Read Arrays from a NetCDF file and add to a State}
   \label{example:StateRdWr}
   This program shows an example of reading and writing Arrays from a State
   from/to a NetCDF file. 
%/////////////////////////////////////////////////////////////

 \begin{verbatim}
    ! ESMF Framework module
    use ESMF
    use ESMF_TestMod
    implicit none

    ! Local variables
    type(ESMF_State) :: state
    type(ESMF_Array) :: latArray, lonArray, timeArray, humidArray, &
                        tempArray, pArray, rhArray
    type(ESMF_VM) :: vm
    integer :: localPet, rc
 
\end{verbatim}
 
%/////////////////////////////////////////////////////////////

    The following line of code will read all Array data contained in a NetCDF
    file, place them in {\tt ESMF\_Arrays} and add them to an {\tt ESMF\_State}.
    Only PET 0 reads the file; the States in the other PETs remain empty.
    Currently, the data is not decomposed or distributed; each PET
    has only 1 DE and only PET 0 contains data after reading the file.
    Future versions of ESMF will support data decomposition and distribution
    upon reading a file.
  
    Note that the third party NetCDF library must be installed.  For more
    details, see the "ESMF Users Guide", 
    "Building and Installing the ESMF, Third Party Libraries, NetCDF" and
    the website http://www.unidata.ucar.edu/software/netcdf. 
%/////////////////////////////////////////////////////////////

 \begin{verbatim}
    ! Read the NetCDF data file into Array objects in the State on PET 0
    call ESMF_StateRead(state, "io_netcdf_testdata.nc", rc=rc)

    ! If the NetCDF library is not present (on PET 0), cleanup and exit 
    if (rc == ESMF_RC_LIB_NOT_PRESENT) then
      call ESMF_StateDestroy(state, rc=rc)
      goto 10
    endif
 
\end{verbatim}
 
%/////////////////////////////////////////////////////////////

    Only reading data into {\tt ESMF\_Arrays} is supported at this time;
    {\tt ESMF\_ArrayBundles}, {\tt ESMF\_Fields}, and {\tt ESMF\_FieldBundles}
    will be supported in future releases of ESMF. 
%/////////////////////////////////////////////////////////////

  \subsubsection{Print Array data from a State}
  
    To see that the State now contains the same data as in the file, the
    following shows how to print out what Arrays are contained within the
    State and to print the data contained within each Array.  The NetCDF utility
    "ncdump" can be used to view the contents of the NetCDF file.
    In this example, only PET 0 will contain data. 
%/////////////////////////////////////////////////////////////

 \begin{verbatim}
    if (localPet == 0) then
      ! Print the names and attributes of Array objects contained in the State
      call ESMF_StatePrint(state, rc=rc)

      ! Get each Array by name from the State
      call ESMF_StateGet(state, "lat",  latArray,   rc=rc)
      call ESMF_StateGet(state, "lon",  lonArray,   rc=rc)
      call ESMF_StateGet(state, "time", timeArray,  rc=rc)
      call ESMF_StateGet(state, "Q",    humidArray, rc=rc)
      call ESMF_StateGet(state, "TEMP", tempArray,  rc=rc)
      call ESMF_StateGet(state, "p",    pArray,     rc=rc)
      call ESMF_StateGet(state, "rh",   rhArray,    rc=rc)

      ! Print out the Array data
      call ESMF_ArrayPrint(latArray,   rc=rc)
      call ESMF_ArrayPrint(lonArray,   rc=rc)
      call ESMF_ArrayPrint(timeArray,  rc=rc)
      call ESMF_ArrayPrint(humidArray, rc=rc)
      call ESMF_ArrayPrint(tempArray,  rc=rc)
      call ESMF_ArrayPrint(pArray,     rc=rc)
      call ESMF_ArrayPrint(rhArray,    rc=rc)
    endif
 
\end{verbatim}
 
%/////////////////////////////////////////////////////////////

    Note that the Arrays "lat", "lon", and "time" hold spatial and temporal
    coordinate data for the dimensions latitude, longitude and time,
    respectively.  These will be used in future releases of ESMF to create
    {\tt ESMF\_Grids}. 
%/////////////////////////////////////////////////////////////

  \subsubsection{Write Array data within a State to a NetCDF file}
  
    All the Array data within the State on PET 0 can be written out to a NetCDF
    file as follows: 
%/////////////////////////////////////////////////////////////

 \begin{verbatim}
    ! Write Arrays within the State on PET 0 to a NetCDF file
    call ESMF_StateWrite(state, "io_netcdf_testdata_out.nc", rc=rc)
 
\end{verbatim}
 
%/////////////////////////////////////////////////////////////

    Currently writing is limited to PET 0; future versions of ESMF will allow
    parallel writing, as well as parallel reading.
%...............................................................
\setlength{\parskip}{\oldparskip}
\setlength{\parindent}{\oldparindent}
\setlength{\baselineskip}{\oldbaselineskip}
