%                **** IMPORTANT NOTICE *****
% This LaTeX file has been automatically produced by ProTeX v. 1.1
% Any changes made to this file will likely be lost next time
% this file is regenerated from its source. Send questions 
% to Arlindo da Silva, dasilva@gsfc.nasa.gov
 
\setlength{\oldparskip}{\parskip}
\setlength{\parskip}{1.5ex}
\setlength{\oldparindent}{\parindent}
\setlength{\parindent}{0pt}
\setlength{\oldbaselineskip}{\baselineskip}
\setlength{\baselineskip}{11pt}
 
%--------------------- SHORT-HAND MACROS ----------------------
\def\bv{\begin{verbatim}}
\def\ev{\end{verbatim}}
\def\be{\begin{equation}}
\def\ee{\end{equation}}
\def\bea{\begin{eqnarray}}
\def\eea{\end{eqnarray}}
\def\bi{\begin{itemize}}
\def\ei{\end{itemize}}
\def\bn{\begin{enumerate}}
\def\en{\end{enumerate}}
\def\bd{\begin{description}}
\def\ed{\end{description}}
\def\({\left (}
\def\){\right )}
\def\[{\left [}
\def\]{\right ]}
\def\<{\left  \langle}
\def\>{\right \rangle}
\def\cI{{\cal I}}
\def\diag{\mathop{\rm diag}}
\def\tr{\mathop{\rm tr}}
%-------------------------------------------------------------

\markboth{Left}{Source File: ESMF\_StateRemRep.F90,  Date: Tue May  5 21:00:08 MDT 2020
}

 
%/////////////////////////////////////////////////////////////
\subsubsection [ESMF\_StateRemove] {ESMF\_StateRemove - Remove an item from a State - (DEPRECATED METHOD)}


  
\bigskip{\sf INTERFACE:}
\begin{verbatim}   ! Private name; call using ESMF_StateRemove ()
   subroutine ESMF_StateRemoveOneItem (state, itemName, &
       relaxedFlag, rc)\end{verbatim}{\em ARGUMENTS:}
\begin{verbatim}     type(ESMF_State), intent(inout) :: state
     character(*), intent(in) :: itemName
 -- The following arguments require argument keyword syntax (e.g. rc=rc). --
     logical, intent(in), optional :: relaxedFlag
     integer, intent(out), optional :: rc\end{verbatim}
{\sf STATUS:}
   \begin{itemize}
   \item\apiStatusCompatibleVersion{5.2.0r}
   \item\apiDeprecatedMethodWithReplacement{5.3.1}{ESMF\_StateRemove}{esmfstateremovelist}
   Rationale: The list version is consistent with other ESMF container
   operations which use lists.
   \end{itemize}
  
{\sf DESCRIPTION:\\ }


   Remove an existing reference to an item from a {\tt State}.
  
   The arguments are:
   \begin{description}
   \item[state]
   The {\tt ESMF\_State} within which {\tt itemName} will be removed.
   \item[itemName]
   The name of the item to be removed. This is a reference only.
   The item itself is unchanged.
  
   If the {\tt state} contains nested {\tt ESMF\_State}s,
   the {\tt itemName} argument may specify a fully qualified name
   to remove the desired item with a single call. This is performed
   using the "/" character to separate the names of the intermediate
   State names leading to the desired item. (E.g.,
   {\tt itemName="state1/state12/item"}.
  
   Since an item could potentially be referenced by multiple containers,
   it remains the responsibility of the user to manage its
   destruction when it is no longer in use.
   \item[{[relaxedflag]}]
   A setting of {\tt .true.} indicates a relaxed definition of "remove",
   where it is {\em not} an error if {\tt itemName} is not present in the
   {\tt state}. For {\tt .false.} this is treated
   as an error condition. The default setting is {\tt .false.}.
   \item[{[rc]}]
   Return code; equals {\tt ESMF\_SUCCESS} if there are no errors.
   \end{description} 
%/////////////////////////////////////////////////////////////
 
\mbox{}\hrulefill\ 
 
\subsubsection [ESMF\_StateRemove] {ESMF\_StateRemove - Remove a list of items from a State}


   \label{esmfstateremovelist}
  
\bigskip{\sf INTERFACE:}
\begin{verbatim}   ! Private name; call using ESMF_StateRemove ()
   subroutine ESMF_StateRemoveList (state, itemNameList, relaxedFlag, rc)\end{verbatim}{\em ARGUMENTS:}
\begin{verbatim}     type(ESMF_State), intent(inout) :: state
     character(*), intent(in) :: itemNameList(:)
 -- The following arguments require argument keyword syntax (e.g. rc=rc). --
     logical, intent(in), optional :: relaxedFlag
     integer, intent(out), optional :: rc\end{verbatim}
{\sf STATUS:}
   \begin{itemize}
   \item\apiStatusCompatibleVersion{5.3.1}
   \end{itemize}
  
{\sf DESCRIPTION:\\ }


   Remove existing references to items from a {\tt State}.
  
   The arguments are:
   \begin{description}
   \item[state]
   The {\tt ESMF\_State} within which {\tt itemName} will be removed.
   \item[itemNameList]
   The name of the items to be removed. This is a reference only.
   The items themselves are unchanged.
  
   If the {\tt state} contains nested {\tt ESMF\_State}s,
   the {\tt itemName} arguments may specify fully qualified names
   to remove the desired items with a single call. This is performed
   using the "/" character to separate the names of the intermediate
   State names leading to the desired items. (E.g.,
   {\tt itemName="state1/state12/item"}.
  
   Since items could potentially be referenced by multiple containers,
   it remains the responsibility of the user to manage their
   destruction when they are no longer in use.
   \item[{[relaxedflag]}]
   A setting of {\tt .true.} indicates a relaxed definition of "remove",
   where it is {\em not} an error if an item in the {\tt itemNameList}
   is not present in the {\tt state}. For {\tt .false.} this is treated
   as an error condition. The default setting is {\tt .false.}.
   \item[{[rc]}]
   Return code; equals {\tt ESMF\_SUCCESS} if there are no errors.
   \end{description} 
%/////////////////////////////////////////////////////////////
 
\mbox{}\hrulefill\ 
 
\subsubsection [ESMF\_StateReplace] {ESMF\_StateReplace - Replace a list of items within a State}


  
\bigskip{\sf INTERFACE:}
\begin{verbatim}   subroutine ESMF_StateReplace(state, <itemList>, relaxedflag, rc)\end{verbatim}{\em ARGUMENTS:}
\begin{verbatim}   type(ESMF_State), intent(inout) :: state
   <itemList>, see below for supported values
 -- The following arguments require argument keyword syntax (e.g. rc=rc). --
   logical, intent(in), optional :: relaxedflag
   integer, intent(out), optional :: rc\end{verbatim}
{\sf STATUS:}
   \begin{itemize}
   \item\apiStatusCompatibleVersion{5.2.0r}
   \end{itemize}
  
{\sf DESCRIPTION:\\ }


   Replace a list of items with a {\tt ESMF\_State}. If an item in
   <itemList> does not match any items already present in {\tt state}, an
   error is returned.
  
   Supported values for <itemList> are:
   \begin{description}
   \item type(ESMF\_Array), intent(in) :: arrayList(:)
   \item type(ESMF\_ArrayBundle), intent(in) :: arraybundleList(:)
   \item type(ESMF\_Field), intent(in) :: fieldList(:)
   \item type(ESMF\_FieldBundle), intent(in) :: fieldbundleList(:)
   \item type(ESMF\_RouteHandle), intent(in) :: routehandleList(:)
   \item type(ESMF\_State), intent(in) :: nestedStateList(:)
   \end{description}
  
   The arguments are:
   \begin{description}
   \item[state]
   An {\tt ESMF\_State} within which the <itemList> items will be replaced.
   \item[<itemList>]
   The list of items to be replaced.
   This is a reference only; when
   the {\tt ESMF\_State} is destroyed the <itemList> contained in it will
   not be destroyed. Also, the items in the <itemList> cannot be safely
   destroyed before the {\tt ESMF\_State} is destroyed.
   Since <itemList> items can be added to multiple containers, it remains
   the responsibility of the user to manage their
   destruction when they are no longer in use.
   \item[{[relaxedflag]}]
   A setting of {\tt .true.} indicates a relaxed definition of "replace",
   where it is {\em not} an error if {\tt <itemList>} contains items
   with names that are not found in {\tt state}. The {\tt State}
   is left unchanged for these items. For {\tt .false.} this is treated
   as an error condition. The default setting is {\tt .false.}.
   \item[{[rc]}]
   Return code; equals {\tt ESMF\_SUCCESS} if there are no errors.
   \end{description}
%...............................................................
\setlength{\parskip}{\oldparskip}
\setlength{\parindent}{\oldparindent}
\setlength{\baselineskip}{\oldbaselineskip}
