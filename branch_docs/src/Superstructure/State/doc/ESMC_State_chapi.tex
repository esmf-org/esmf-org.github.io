%                **** IMPORTANT NOTICE *****
% This LaTeX file has been automatically produced by ProTeX v. 1.1
% Any changes made to this file will likely be lost next time
% this file is regenerated from its source. Send questions 
% to Arlindo da Silva, dasilva@gsfc.nasa.gov
 
\setlength{\oldparskip}{\parskip}
\setlength{\parskip}{1.5ex}
\setlength{\oldparindent}{\parindent}
\setlength{\parindent}{0pt}
\setlength{\oldbaselineskip}{\baselineskip}
\setlength{\baselineskip}{11pt}
 
%--------------------- SHORT-HAND MACROS ----------------------
\def\bv{\begin{verbatim}}
\def\ev{\end{verbatim}}
\def\be{\begin{equation}}
\def\ee{\end{equation}}
\def\bea{\begin{eqnarray}}
\def\eea{\end{eqnarray}}
\def\bi{\begin{itemize}}
\def\ei{\end{itemize}}
\def\bn{\begin{enumerate}}
\def\en{\end{enumerate}}
\def\bd{\begin{description}}
\def\ed{\end{description}}
\def\({\left (}
\def\){\right )}
\def\[{\left [}
\def\]{\right ]}
\def\<{\left  \langle}
\def\>{\right \rangle}
\def\cI{{\cal I}}
\def\diag{\mathop{\rm diag}}
\def\tr{\mathop{\rm tr}}
%-------------------------------------------------------------

\markboth{Left}{Source File: ESMC\_State.h,  Date: Tue May  5 21:00:08 MDT 2020
}

 
%/////////////////////////////////////////////////////////////
\subsubsection [ESMC\_StateAddArray] {ESMC\_StateAddArray - Add an Array object to a State}


  
\bigskip{\sf INTERFACE:}
\begin{verbatim} int ESMC_StateAddArray(
   ESMC_State state,   // in
   ESMC_Array array    // in
 );\end{verbatim}{\em RETURN VALUE:}
\begin{verbatim}    Return code; equals ESMF_SUCCESS if there are no errors.\end{verbatim}
{\sf DESCRIPTION:\\ }


  
    Add an Array object to a {\tt ESMC\_State} object.
  
    The arguments are:
    \begin{description}
    \item[state]
      The State object.
    \item[array]
      The Array object to be included within the State.
    \end{description}
   
%/////////////////////////////////////////////////////////////
 
\mbox{}\hrulefill\ 
 
\subsubsection [ESMC\_StateAddField] {ESMC\_StateAddField - Add a Field object to a State}


  
\bigskip{\sf INTERFACE:}
\begin{verbatim} int ESMC_StateAddField(
   ESMC_State state,   // in
   ESMC_Field field    // in
 );\end{verbatim}{\em RETURN VALUE:}
\begin{verbatim}    Return code; equals ESMF_SUCCESS if there are no errors.\end{verbatim}
{\sf DESCRIPTION:\\ }


  
    Add an Array object to a {\tt ESMC\_State} object.
  
    The arguments are:
    \begin{description}
    \item[state]
      The State object.
    \item[array]
      The Array object to be included within the State.
    \end{description}
   
%/////////////////////////////////////////////////////////////
 
\mbox{}\hrulefill\ 
 
\subsubsection [ESMC\_StateCreate] {ESMC\_StateCreate - Create an Array}


  
\bigskip{\sf INTERFACE:}
\begin{verbatim} ESMC_State ESMC_StateCreate(
   const char *name,  // in
   int *rc            // out
 );\end{verbatim}{\em RETURN VALUE:}
\begin{verbatim}    Newly created ESMC_State object.\end{verbatim}
{\sf DESCRIPTION:\\ }


  
    Create an {\tt ESMC\_State} object.
  
    The arguments are:
    \begin{description}
    \item[{[name]}]
      The name for the State object. If not specified, i.e. NULL,
      a default unique name will be generated: "StateNNN" where NNN
      is a unique sequence number from 001 to 999.
    \item[rc]
      Return code; equals {\tt ESMF\_SUCCESS} if there are no errors.
    \end{description}
   
%/////////////////////////////////////////////////////////////
 
\mbox{}\hrulefill\ 
 
\subsubsection [ESMC\_StateDestroy] {ESMC\_StateDestroy - Destroy a State}


  
\bigskip{\sf INTERFACE:}
\begin{verbatim} int ESMC_StateDestroy(
   ESMC_State *state    // in
 );\end{verbatim}{\em RETURN VALUE:}
\begin{verbatim}    Return code; equals ESMF_SUCCESS if there are no errors.\end{verbatim}
{\sf DESCRIPTION:\\ }


  
    Destroy a {\tt ESMC\_State} object.
  
    The arguments are:
    \begin{description}
    \item[state]
      The State to be destroyed.
    \end{description}
   
%/////////////////////////////////////////////////////////////
 
\mbox{}\hrulefill\ 
 
\subsubsection [ESMC\_StateGetArray] {ESMC\_StateGetArray - Obtains an Array object from a State}


  
\bigskip{\sf INTERFACE:}
\begin{verbatim} int ESMC_StateGetArray(
   ESMC_State state,    // in
   const char *name,    // in
   ESMC_Array *array    // out
 );\end{verbatim}{\em RETURN VALUE:}
\begin{verbatim}    Return code; equals ESMF_SUCCESS if there are no errors.\end{verbatim}
{\sf DESCRIPTION:\\ }


  
    Obtain a pointer to an {\tt ESMC\_Array} object contained within
    a State.
  
    The arguments are:
    \begin{description}
    \item[state]
      The State object.
    \item[name]
      The name of the desired Array object.
    \item[array]
      A pointer to the Array object.
    \end{description}
   
%/////////////////////////////////////////////////////////////
 
\mbox{}\hrulefill\ 
 
\subsubsection [ESMC\_StateGetField] {ESMC\_StateGetField - Obtains a Field object from a State}


  
\bigskip{\sf INTERFACE:}
\begin{verbatim} int ESMC_StateGetField(
   ESMC_State state,    // in
   const char *name,    // in
   ESMC_Field *field    // out
 );\end{verbatim}{\em RETURN VALUE:}
\begin{verbatim}    Return code; equals ESMF_SUCCESS if there are no errors.\end{verbatim}
{\sf DESCRIPTION:\\ }


  
    Obtain a pointer to a {\tt ESMC\_Field} object contained within
    a State.
  
    The arguments are:
    \begin{description}
    \item[state]
      The State object.
    \item[name]
      The name of the desired Field object.
    \item[array]
      A pointer to the Field object.
    \end{description}
   
%/////////////////////////////////////////////////////////////
 
\mbox{}\hrulefill\ 
 
\subsubsection [ESMC\_StatePrint] {ESMC\_StatePrint - Print the contents of a State}


  
\bigskip{\sf INTERFACE:}
\begin{verbatim} int ESMC_StatePrint(
   ESMC_State state    // in
 );\end{verbatim}{\em RETURN VALUE:}
\begin{verbatim}    Return code; equals ESMF_SUCCESS if there are no errors.\end{verbatim}
{\sf DESCRIPTION:\\ }


  
    Prints the contents of a {\tt ESMC\_State} object.
  
    The arguments are:
    \begin{description}
    \item[state]
      The State to be printed.
    \end{description}
  
%...............................................................
\setlength{\parskip}{\oldparskip}
\setlength{\parindent}{\oldparindent}
\setlength{\baselineskip}{\oldbaselineskip}
