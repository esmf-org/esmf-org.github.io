%                **** IMPORTANT NOTICE *****
% This LaTeX file has been automatically produced by ProTeX v. 1.1
% Any changes made to this file will likely be lost next time
% this file is regenerated from its source. Send questions 
% to Arlindo da Silva, dasilva@gsfc.nasa.gov
 
\setlength{\oldparskip}{\parskip}
\setlength{\parskip}{1.5ex}
\setlength{\oldparindent}{\parindent}
\setlength{\parindent}{0pt}
\setlength{\oldbaselineskip}{\baselineskip}
\setlength{\baselineskip}{11pt}
 
%--------------------- SHORT-HAND MACROS ----------------------
\def\bv{\begin{verbatim}}
\def\ev{\end{verbatim}}
\def\be{\begin{equation}}
\def\ee{\end{equation}}
\def\bea{\begin{eqnarray}}
\def\eea{\end{eqnarray}}
\def\bi{\begin{itemize}}
\def\ei{\end{itemize}}
\def\bn{\begin{enumerate}}
\def\en{\end{enumerate}}
\def\bd{\begin{description}}
\def\ed{\end{description}}
\def\({\left (}
\def\){\right )}
\def\[{\left [}
\def\]{\right ]}
\def\<{\left  \langle}
\def\>{\right \rangle}
\def\cI{{\cal I}}
\def\diag{\mathop{\rm diag}}
\def\tr{\mathop{\rm tr}}
%-------------------------------------------------------------

\markboth{Left}{Source File: ESMF\_StateGet.F90,  Date: Tue May  5 21:00:08 MDT 2020
}

 
%/////////////////////////////////////////////////////////////
\subsubsection [ESMF\_StateGetDataPointer] {ESMF\_StateGetDataPointer - Retrieve Fortran pointer directly from a State }


   
\bigskip{\sf INTERFACE:}
\begin{verbatim}   ! Private name; call using ESMF_StateGetDataPointer() 
   subroutine ESMF_StateGetDataPointer<rank><type><kind>(state, itemName, 
   dataPointer, datacopyflag, nestedStateName, rc) 
   \end{verbatim}{\em ARGUMENTS:}
\begin{verbatim}   type(ESMF_State), intent(in) :: state 
   character(len=*), intent(in) :: itemName 
   <type> (ESMF_KIND_<kind>), dimension(<rank>), pointer :: dataPointer 
   type(ESMF_DataCopy_Flag), intent(in), optional :: datacopyflag 
   character(len=*), intent(in), optional :: nestedStateName 
   integer, intent(out), optional :: rc 
   
   \end{verbatim}
{\sf STATUS:}
   \begin{itemize} 
   \item\apiStatusCompatibleVersion{5.2.0r} 
   \end{itemize} 
   
{\sf DESCRIPTION:\\ }

 
   Retrieves data from a state, returning a direct Fortran pointer to 
   the data array. 
   Valid type/kind/rank combinations supported by the 
   framework are: ranks 1 to 7, type real of kind *4 or *8, 
   and type integer of kind *1, *2, *4, or *8. 
   
   The arguments are: 
   \begin{description} 
   \item[state] 
   The {\tt ESMF\_State} to query. 
   \item[itemName] 
   The name of the FieldBundle, Field, or Array to return data from. 
   \item[dataPointer] 
   An unassociated Fortran pointer of the proper Type, Kind, and Rank as the data 
   in the State. When this call returns successfully, the pointer will now reference 
   the data in the State. This is either a reference or a copy, depending on the 
   setting of the following argument. The default is to return a reference. 
   \item[{[datacopyflag]}] 
   Defaults to {\tt ESMF\_DATACOPY\_REFERENCE}. If set to {\tt ESMF\_DATACOPY\_VALUE}, a separate 
   copy of the data will be made and the pointer will point at the copy. 
   \item[{[nestedStateName]}] 
   Optional. If multiple states are present, a specific state name must be given. 
   \item[{[fieldName]}] 
   Optional. If {\tt itemName} refers to a fieldbundle then the name of the field 
   in the fieldbundle must also be given. 
   \item[{[rc]}] 
   Return code; equals {\tt ESMF\_SUCCESS} if there are no errors. 
   \end{description} 
   
%...............................................................
\setlength{\parskip}{\oldparskip}
\setlength{\parindent}{\oldparindent}
\setlength{\baselineskip}{\oldbaselineskip}
