%                **** IMPORTANT NOTICE *****
% This LaTeX file has been automatically produced by ProTeX v. 1.1
% Any changes made to this file will likely be lost next time
% this file is regenerated from its source. Send questions 
% to Arlindo da Silva, dasilva@gsfc.nasa.gov
 
\setlength{\oldparskip}{\parskip}
\setlength{\parskip}{1.5ex}
\setlength{\oldparindent}{\parindent}
\setlength{\parindent}{0pt}
\setlength{\oldbaselineskip}{\baselineskip}
\setlength{\baselineskip}{11pt}
 
%--------------------- SHORT-HAND MACROS ----------------------
\def\bv{\begin{verbatim}}
\def\ev{\end{verbatim}}
\def\be{\begin{equation}}
\def\ee{\end{equation}}
\def\bea{\begin{eqnarray}}
\def\eea{\end{eqnarray}}
\def\bi{\begin{itemize}}
\def\ei{\end{itemize}}
\def\bn{\begin{enumerate}}
\def\en{\end{enumerate}}
\def\bd{\begin{description}}
\def\ed{\end{description}}
\def\({\left (}
\def\){\right )}
\def\[{\left [}
\def\]{\right ]}
\def\<{\left  \langle}
\def\>{\right \rangle}
\def\cI{{\cal I}}
\def\diag{\mathop{\rm diag}}
\def\tr{\mathop{\rm tr}}
%-------------------------------------------------------------

\markboth{Left}{Source File: ESMF\_StateEx.F90,  Date: Tue May  5 21:00:09 MDT 2020
}

 
%/////////////////////////////////////////////////////////////

  \subsubsection{Add items to a State}
     
    Creation of an empty {\tt ESMF\_State}, and adding an {\tt ESMF\_FieldBundle}
    to it.  Note that the {\tt ESMF\_FieldBundle} does not get destroyed when
    the {\tt ESMF\_State} is destroyed; the {\tt ESMF\_State} only contains
    a reference to the objects it contains.  It also does not make a copy;
    the original objects can be updated and code accessing them by using
    the {\tt ESMF\_State} will see the updated version. 
%/////////////////////////////////////////////////////////////

 \begin{verbatim}
    statename = "Ocean"
    state2 = ESMF_StateCreate(name=statename,  &
                              stateintent=ESMF_STATEINTENT_EXPORT, rc=rc)  
 
\end{verbatim}
 
%/////////////////////////////////////////////////////////////

 \begin{verbatim}
    bundlename = "Temperature"
    bundle1 = ESMF_FieldBundleCreate(name=bundlename, rc=rc)
    print *, "FieldBundle Create returned", rc
 
\end{verbatim}
 
%/////////////////////////////////////////////////////////////

 \begin{verbatim}
    call ESMF_StateAdd(state2, (/bundle1/), rc=rc)
    print *, "StateAdd returned", rc
 
\end{verbatim}
 
%/////////////////////////////////////////////////////////////

 \begin{verbatim}
    call ESMF_StateDestroy(state2, rc=rc)
 
\end{verbatim}
 
%/////////////////////////////////////////////////////////////

 \begin{verbatim}
    call ESMF_FieldBundleDestroy(bundle1, rc=rc)
 
\end{verbatim}
 
%/////////////////////////////////////////////////////////////

  \subsubsection{Add placeholders to a State}
     
   If a component could potentially produce a large number of optional
   items, one strategy is to add the names only of those objects to the
   {\tt ESMF\_State}.  Other components can call framework routines to
   set the {\tt ESMF\_NEEDED} flag to indicate they require that data.
   The original component can query this flag and then produce only the
   data that is required by another component. 
%/////////////////////////////////////////////////////////////

 \begin{verbatim}
    statename = "Ocean"
    state3 = ESMF_StateCreate(name=statename,  &
                              stateintent=ESMF_STATEINTENT_EXPORT, rc=rc)  
 
\end{verbatim}
 
%/////////////////////////////////////////////////////////////

 \begin{verbatim}
    dataname = "Downward wind:needed"
    call ESMF_AttributeSet (state3, dataname, .false., rc=rc)
 
\end{verbatim}
 
%/////////////////////////////////////////////////////////////

 \begin{verbatim}
    dataname = "Humidity:needed"
    call ESMF_AttributeSet (state3, dataname, .false., rc=rc)
 
\end{verbatim}
 
%/////////////////////////////////////////////////////////////

  \subsubsection{Mark an item {\tt NEEDED}}
     
   How to set the {\tt NEEDED} state of an item. 
%/////////////////////////////////////////////////////////////

 \begin{verbatim}
    dataname = "Downward wind:needed"
    call ESMF_AttributeSet (state3, name=dataname, value=.true., rc=rc)
 
\end{verbatim}
 
%/////////////////////////////////////////////////////////////

  \subsubsection{Create a {\tt NEEDED} item}
     
   Query an item for the {\tt NEEDED} status, and creating an item on demand.
   Similar flags exist for "Ready", "Valid", and "Required for Restart",
   to mark each data item as ready, having been validated, or needed if the
   application is to be checkpointed and restarted.  The flags are supported
   to help coordinate the data exchange between components. 
%/////////////////////////////////////////////////////////////

 \begin{verbatim}
    dataname = "Downward wind:needed"
    call ESMF_AttributeGet (state3, dataname, valueList=neededFlag, rc=rc)
 
\end{verbatim}
 
%/////////////////////////////////////////////////////////////

 \begin{verbatim}
    if (rc == ESMF_SUCCESS .and. neededFlag(1)) then
        bundlename = dataname
        bundle2 = ESMF_FieldBundleCreate(name=bundlename, rc=rc)
 
\end{verbatim}
 
%/////////////////////////////////////////////////////////////

 \begin{verbatim}
        call ESMF_StateAdd(state3, (/bundle2/), rc=rc)
 
\end{verbatim}
 
%/////////////////////////////////////////////////////////////

 \begin{verbatim}
    else
        print *, "Data not marked as needed", trim(dataname)
    endif
 
\end{verbatim}

%...............................................................
\setlength{\parskip}{\oldparskip}
\setlength{\parindent}{\oldparindent}
\setlength{\baselineskip}{\oldbaselineskip}
