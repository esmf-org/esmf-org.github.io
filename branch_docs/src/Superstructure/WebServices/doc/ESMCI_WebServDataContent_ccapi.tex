%                **** IMPORTANT NOTICE *****
% This LaTeX file has been automatically produced by ProTeX v. 1.1
% Any changes made to this file will likely be lost next time
% this file is regenerated from its source. Send questions 
% to Arlindo da Silva, dasilva@gsfc.nasa.gov
 
\setlength{\oldparskip}{\parskip}
\setlength{\parskip}{1.5ex}
\setlength{\oldparindent}{\parindent}
\setlength{\parindent}{0pt}
\setlength{\oldbaselineskip}{\baselineskip}
\setlength{\baselineskip}{11pt}
 
%--------------------- SHORT-HAND MACROS ----------------------
\def\bv{\begin{verbatim}}
\def\ev{\end{verbatim}}
\def\be{\begin{equation}}
\def\ee{\end{equation}}
\def\bea{\begin{eqnarray}}
\def\eea{\end{eqnarray}}
\def\bi{\begin{itemize}}
\def\ei{\end{itemize}}
\def\bn{\begin{enumerate}}
\def\en{\end{enumerate}}
\def\bd{\begin{description}}
\def\ed{\end{description}}
\def\({\left (}
\def\){\right )}
\def\[{\left [}
\def\]{\right ]}
\def\<{\left  \langle}
\def\>{\right \rangle}
\def\cI{{\cal I}}
\def\diag{\mathop{\rm diag}}
\def\tr{\mathop{\rm tr}}
%-------------------------------------------------------------

\markboth{Left}{Source File: ESMCI\_WebServDataContent.C,  Date: Tue May  5 21:00:15 MDT 2020
}

 
%/////////////////////////////////////////////////////////////
\subsubsection{ESMCI\_WebServDataContent::ESMCI\_WebServDataContent() (Source File: ESMCI\_WebServDataContent.C)}


  
\bigskip{\sf INTERFACE:}
\begin{verbatim} ESMCI_WebServDataContent::ESMCI_WebServDataContent(\end{verbatim}{\em ARGUMENTS:}
\begin{verbatim}   int           numLatValues,
   int           numLonValues
   )\end{verbatim}
{\sf DESCRIPTION:\\ }


      Creates and sets up a container for a set of grid-based data variables
      for the specified timestamp.
   
%/////////////////////////////////////////////////////////////
 
\mbox{}\hrulefill\
 
\subsubsection{ESMCI\_WebServDataContent::~ESMCI\_WebServDataContent() (Source File: ESMCI\_WebServDataContent.C)}


  
\bigskip{\sf INTERFACE:}
\begin{verbatim} ESMCI_WebServDataContent::~ESMCI_WebServDataContent(\end{verbatim}{\em ARGUMENTS:}
\begin{verbatim}   )\end{verbatim}
{\sf DESCRIPTION:\\ }


      Cleans up memory allocated for CAM output file.
   
%/////////////////////////////////////////////////////////////
 
\mbox{}\hrulefill\
 
\subsubsection{ESMCI\_WebServDataContent::setTimeStamp() (Source File: ESMCI\_WebServDataContent.C)}


  
\bigskip{\sf INTERFACE:}
\begin{verbatim} void  ESMCI_WebServDataContent::setTimeStamp(\end{verbatim}{\em RETURN VALUE:}
\begin{verbatim} \end{verbatim}{\em ARGUMENTS:}
\begin{verbatim}   double  timestamp             // the timestamp for the data values
   )\end{verbatim}
{\sf DESCRIPTION:\\ }


      Sets the timestamp for the data values.
   
%/////////////////////////////////////////////////////////////
 
\mbox{}\hrulefill\
 
\subsubsection{ESMCI\_WebServDataContent::addDataValues() (Source File: ESMCI\_WebServDataContent.C)}


  
\bigskip{\sf INTERFACE:}
\begin{verbatim} void  ESMCI_WebServDataContent::addDataValues(\end{verbatim}{\em RETURN VALUE:}
\begin{verbatim} \end{verbatim}{\em ARGUMENTS:}
\begin{verbatim}   string   varName,             // the variable name
   double*  dataValues   // array of doubles containing the data values
   )\end{verbatim}
{\sf DESCRIPTION:\\ }


      Sets the timestamp for the data values.
   
%/////////////////////////////////////////////////////////////
 
\mbox{}\hrulefill\
 
\subsubsection{ESMCI\_WebServDataContent::getDataValues() (Source File: ESMCI\_WebServDataContent.C)}


  
\bigskip{\sf INTERFACE:}
\begin{verbatim} double*  ESMCI_WebServDataContent::getDataValues(\end{verbatim}{\em RETURN VALUE:}
\begin{verbatim}     double  data value for the specified variable, time, lat and lon\end{verbatim}{\em ARGUMENTS:}
\begin{verbatim}   string  varName               // variable name of data value to lookup
   )\end{verbatim}
{\sf DESCRIPTION:\\ }


      Looks up the data value for the specified variable at the specified
      time, latitude and longitude.
   
%/////////////////////////////////////////////////////////////
 
\mbox{}\hrulefill\
 
\subsubsection{ESMCI\_WebServDataContent::getDataValue() (Source File: ESMCI\_WebServDataContent.C)}


  
\bigskip{\sf INTERFACE:}
\begin{verbatim} double  ESMCI_WebServDataContent::getDataValue(\end{verbatim}{\em RETURN VALUE:}
\begin{verbatim}     double  data value for the specified variable, time, lat and lon\end{verbatim}{\em ARGUMENTS:}
\begin{verbatim}   string  varName,              // variable name of data value to lookup
   int     latValueIdx,          // latitude value name of data value to lookup
   int     lonValueIdx           // longitude value name of data value to lookup
   )\end{verbatim}
{\sf DESCRIPTION:\\ }


      Looks up the data value for the specified variable at the specified
      time, latitude and longitude.
   
%/////////////////////////////////////////////////////////////
 
\mbox{}\hrulefill\
 
\subsubsection{ESMCI\_WebServDataContent::print() (Source File: ESMCI\_WebServDataContent.C)}


  
\bigskip{\sf INTERFACE:}
\begin{verbatim} void  ESMCI_WebServDataContent::print(\end{verbatim}{\em RETURN VALUE:}
\begin{verbatim} \end{verbatim}{\em ARGUMENTS:}
\begin{verbatim}   )\end{verbatim}
{\sf DESCRIPTION:\\ }


      Prints out the data values (including time, lat and lon) as they were
      read from the output file.
  
%...............................................................
\setlength{\parskip}{\oldparskip}
\setlength{\parindent}{\oldparindent}
\setlength{\baselineskip}{\oldbaselineskip}
