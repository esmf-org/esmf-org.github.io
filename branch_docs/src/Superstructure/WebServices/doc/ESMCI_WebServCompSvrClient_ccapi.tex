%                **** IMPORTANT NOTICE *****
% This LaTeX file has been automatically produced by ProTeX v. 1.1
% Any changes made to this file will likely be lost next time
% this file is regenerated from its source. Send questions 
% to Arlindo da Silva, dasilva@gsfc.nasa.gov
 
\setlength{\oldparskip}{\parskip}
\setlength{\parskip}{1.5ex}
\setlength{\oldparindent}{\parindent}
\setlength{\parindent}{0pt}
\setlength{\oldbaselineskip}{\baselineskip}
\setlength{\baselineskip}{11pt}
 
%--------------------- SHORT-HAND MACROS ----------------------
\def\bv{\begin{verbatim}}
\def\ev{\end{verbatim}}
\def\be{\begin{equation}}
\def\ee{\end{equation}}
\def\bea{\begin{eqnarray}}
\def\eea{\end{eqnarray}}
\def\bi{\begin{itemize}}
\def\ei{\end{itemize}}
\def\bn{\begin{enumerate}}
\def\en{\end{enumerate}}
\def\bd{\begin{description}}
\def\ed{\end{description}}
\def\({\left (}
\def\){\right )}
\def\[{\left [}
\def\]{\right ]}
\def\<{\left  \langle}
\def\>{\right \rangle}
\def\cI{{\cal I}}
\def\diag{\mathop{\rm diag}}
\def\tr{\mathop{\rm tr}}
%-------------------------------------------------------------

\markboth{Left}{Source File: ESMCI\_WebServCompSvrClient.C,  Date: Tue May  5 21:00:16 MDT 2020
}

 
%/////////////////////////////////////////////////////////////
\subsubsection{ESMCI\_WebServCompSvrClient::ESMCI\_WebServCompSvrClient() (Source File: ESMCI\_WebServCompSvrClient.C)}


  
\bigskip{\sf INTERFACE:}
\begin{verbatim} ESMCI_WebServCompSvrClient::ESMCI_WebServCompSvrClient(\end{verbatim}{\em ARGUMENTS:}
\begin{verbatim}   const char*  host,            // (in) the name of the host machine running the
                                 // component service
   int          port,            // (in) the port number of the component service
                                 // to which this client will connect
   int          clientId // (in) the id of the client on the Process Controller
   ) : ESMCI_WebServNetEsmfClient(host, port)\end{verbatim}
{\sf DESCRIPTION:\\ }


      Initialize the ESMF Component client with the name of the host and port
      where the component service is running.
   
%/////////////////////////////////////////////////////////////
 
\mbox{}\hrulefill\
 
\subsubsection{ESMCI\_WebServCompSvrClient::~ESMCI\_WebServCompSvrClient() (Source File: ESMCI\_WebServCompSvrClient.C)}


  
\bigskip{\sf INTERFACE:}
\begin{verbatim} ESMCI_WebServCompSvrClient::~ESMCI_WebServCompSvrClient(\end{verbatim}{\em ARGUMENTS:}
\begin{verbatim}   )\end{verbatim}
{\sf DESCRIPTION:\\ }


      Cleans up the ESMF Component client by disconnecting from the service.
   
%/////////////////////////////////////////////////////////////
 
\mbox{}\hrulefill\
 
\subsubsection{ESMCI\_WebServCompSvrClient::setClientId() (Source File: ESMCI\_WebServCompSvrClient.C)}


  
\bigskip{\sf INTERFACE:}
\begin{verbatim} void  ESMCI_WebServCompSvrClient::setClientId(\end{verbatim}{\em RETURN VALUE:}
\begin{verbatim} \end{verbatim}{\em ARGUMENTS:}
\begin{verbatim}   int  clientId         // (in) the unique id of the client on the Process Controller
   )\end{verbatim}
{\sf DESCRIPTION:\\ }


      Sets the id of the client on the Process Controller.
   
%/////////////////////////////////////////////////////////////
 
\mbox{}\hrulefill\
 
\subsubsection{ESMCI\_WebServCompSvrClient::init() (Source File: ESMCI\_WebServCompSvrClient.C)}


  
\bigskip{\sf INTERFACE:}
\begin{verbatim} int  ESMCI_WebServCompSvrClient::init(\end{verbatim}{\em RETURN VALUE:}
\begin{verbatim}     int  the current state of the component service;
          ESMF_FAILURE if an error occurs\end{verbatim}{\em ARGUMENTS:}
\begin{verbatim}   )\end{verbatim}
{\sf DESCRIPTION:\\ }


      Connects to the component server, makes a request to initialize the
      component, retrieve the server status, and then disconnect from
      the server.
   
%/////////////////////////////////////////////////////////////
 
\mbox{}\hrulefill\
 
\subsubsection{ESMCI\_WebServCompSvrClient::run() (Source File: ESMCI\_WebServCompSvrClient.C)}


  
\bigskip{\sf INTERFACE:}
\begin{verbatim} int  ESMCI_WebServCompSvrClient::run(\end{verbatim}{\em RETURN VALUE:}
\begin{verbatim}     int  the current state of the component service;
          ESMF_FAILURE if an error occurs\end{verbatim}{\em ARGUMENTS:}
\begin{verbatim}   )\end{verbatim}
{\sf DESCRIPTION:\\ }


      Connects to the component server, makes a request to run the
      component, retrieve the server status, and then disconnect from
      the server.
   
%/////////////////////////////////////////////////////////////
 
\mbox{}\hrulefill\
 
\subsubsection{ESMCI\_WebServCompSvrClient::timestep() (Source File: ESMCI\_WebServCompSvrClient.C)}


  
\bigskip{\sf INTERFACE:}
\begin{verbatim} int  ESMCI_WebServCompSvrClient::timestep(\end{verbatim}{\em RETURN VALUE:}
\begin{verbatim}     int  the current state of the component service;
          ESMF_FAILURE if an error occurs\end{verbatim}{\em ARGUMENTS:}
\begin{verbatim}   int  numTimesteps             // The number of timesteps to run
   )\end{verbatim}
{\sf DESCRIPTION:\\ }


      Connects to the component server, makes a request to run the
      component for the specified number of timesteps, retrieve the server
      status, and then disconnect from the server.
   
%/////////////////////////////////////////////////////////////
 
\mbox{}\hrulefill\
 
\subsubsection{ESMCI\_WebServCompSvrClient::final() (Source File: ESMCI\_WebServCompSvrClient.C)}


  
\bigskip{\sf INTERFACE:}
\begin{verbatim} int  ESMCI_WebServCompSvrClient::final(\end{verbatim}{\em RETURN VALUE:}
\begin{verbatim}     int  the current state of the component service;
          ESMF_FAILURE if an error occurs\end{verbatim}{\em ARGUMENTS:}
\begin{verbatim}   )\end{verbatim}
{\sf DESCRIPTION:\\ }


      Connects to the component server, makes a request to finalize the
      component, retrieve the server status, and then disconnect from
      the server.
   
%/////////////////////////////////////////////////////////////
 
\mbox{}\hrulefill\
 
\subsubsection{ESMCI\_WebServCompSvrClient::state() (Source File: ESMCI\_WebServCompSvrClient.C)}


  
\bigskip{\sf INTERFACE:}
\begin{verbatim} int  ESMCI_WebServCompSvrClient::state(\end{verbatim}{\em RETURN VALUE:}
\begin{verbatim}     int  the current state of the component service;
          ESMF_FAILURE if an error occurs\end{verbatim}{\em ARGUMENTS:}
\begin{verbatim}   )\end{verbatim}
{\sf DESCRIPTION:\\ }


      Connects to the component server, makes a request to get the current
      service state, retrieve the server status, and then disconnect from
      the server.
   
%/////////////////////////////////////////////////////////////
 
\mbox{}\hrulefill\
 
\subsubsection{ESMCI\_WebServCompSvrClient::files() (Source File: ESMCI\_WebServCompSvrClient.C)}


  
\bigskip{\sf INTERFACE:}
\begin{verbatim} vector<string>  ESMCI_WebServCompSvrClient::files(\end{verbatim}{\em RETURN VALUE:}
\begin{verbatim}     vector<string>  a list of filenames that contain the export data\end{verbatim}{\em ARGUMENTS:}
\begin{verbatim}   )\end{verbatim}
{\sf DESCRIPTION:\\ }


      Connects to the component server, makes a request to get the export
      filenames, retrieve the filenames and the component status, and then
      disconnect from the server.
   
%/////////////////////////////////////////////////////////////
 
\mbox{}\hrulefill\
 
\subsubsection{ESMCI\_WebServCompSvrClient::dataDesc() (Source File: ESMCI\_WebServCompSvrClient.C)}


  
\bigskip{\sf INTERFACE:}
\begin{verbatim} ESMCI_WebServDataDesc*  ESMCI_WebServCompSvrClient::dataDesc(\end{verbatim}{\em RETURN VALUE:}
\begin{verbatim}     ESMCI_WebServDataDesc*  pointer to the data descriptor object\end{verbatim}{\em ARGUMENTS:}
\begin{verbatim}   )\end{verbatim}
{\sf DESCRIPTION:\\ }


      Connects to the component server, makes a request to get the description
      of the output data, retrieves the description and the component status,
      and then disconnects from the server.
   
%/////////////////////////////////////////////////////////////
 
\mbox{}\hrulefill\
 
\subsubsection{ESMCI\_WebServCompSvrClient::outputData() (Source File: ESMCI\_WebServCompSvrClient.C)}


  
\bigskip{\sf INTERFACE:}
\begin{verbatim} ESMCI_WebServDataContent*  ESMCI_WebServCompSvrClient::outputData(\end{verbatim}{\em RETURN VALUE:}
\begin{verbatim}     ESMCI_WebServDataContent*  pointer to the data content object\end{verbatim}{\em ARGUMENTS:}
\begin{verbatim}   double      timestamp, // (in) timestamp for which the output data is returned
   int*        retNumVars,
   string**    retVarNames,
   int*        retNumLats,
   int*        retNumLons
   )\end{verbatim}
{\sf DESCRIPTION:\\ }


      Connects to the component server, makes a request to get the output data
      for a specified timestamp, retrieves the output data and the component
      status, and then disconnects from the server.
   
%/////////////////////////////////////////////////////////////
 
\mbox{}\hrulefill\
 
\subsubsection{ESMCI\_WebServCompSvrClient::end() (Source File: ESMCI\_WebServCompSvrClient.C)}


  
\bigskip{\sf INTERFACE:}
\begin{verbatim} int  ESMCI_WebServCompSvrClient::end(\end{verbatim}{\em RETURN VALUE:}
\begin{verbatim}     int  the current state of the component service;
          ESMF_FAILURE if an error occurs\end{verbatim}{\em ARGUMENTS:}
\begin{verbatim}   )\end{verbatim}
{\sf DESCRIPTION:\\ }


      Connects to the component server, makes a request to end the client
      session on the component server, retrieve the server status, and then
      disconnect from the server.
   
%/////////////////////////////////////////////////////////////
 
\mbox{}\hrulefill\
 
\subsubsection{ESMCI\_WebServCompSvrClient::killServer() (Source File: ESMCI\_WebServCompSvrClient.C)}


  
\bigskip{\sf INTERFACE:}
\begin{verbatim} int  ESMCI_WebServCompSvrClient::killServer(\end{verbatim}{\em RETURN VALUE:}
\begin{verbatim}     int  ESMF_SUCCESS if the request is successfully sent;
          ESMF_FAILURE if an error occurs\end{verbatim}{\em ARGUMENTS:}
\begin{verbatim}   )\end{verbatim}
{\sf DESCRIPTION:\\ }


      Connects to the component server, makes a request to kill the component
      server process.  Nothing is expected in return.
  
%...............................................................
\setlength{\parskip}{\oldparskip}
\setlength{\parindent}{\oldparindent}
\setlength{\baselineskip}{\oldbaselineskip}
