%                **** IMPORTANT NOTICE *****
% This LaTeX file has been automatically produced by ProTeX v. 1.1
% Any changes made to this file will likely be lost next time
% this file is regenerated from its source. Send questions 
% to Arlindo da Silva, dasilva@gsfc.nasa.gov
 
\setlength{\oldparskip}{\parskip}
\setlength{\parskip}{1.5ex}
\setlength{\oldparindent}{\parindent}
\setlength{\parindent}{0pt}
\setlength{\oldbaselineskip}{\baselineskip}
\setlength{\baselineskip}{11pt}
 
%--------------------- SHORT-HAND MACROS ----------------------
\def\bv{\begin{verbatim}}
\def\ev{\end{verbatim}}
\def\be{\begin{equation}}
\def\ee{\end{equation}}
\def\bea{\begin{eqnarray}}
\def\eea{\end{eqnarray}}
\def\bi{\begin{itemize}}
\def\ei{\end{itemize}}
\def\bn{\begin{enumerate}}
\def\en{\end{enumerate}}
\def\bd{\begin{description}}
\def\ed{\end{description}}
\def\({\left (}
\def\){\right )}
\def\[{\left [}
\def\]{\right ]}
\def\<{\left  \langle}
\def\>{\right \rangle}
\def\cI{{\cal I}}
\def\diag{\mathop{\rm diag}}
\def\tr{\mathop{\rm tr}}
%-------------------------------------------------------------

\markboth{Left}{Source File: ESMCI\_WebServ\_F.C,  Date: Tue May  5 21:00:15 MDT 2020
}

 
%/////////////////////////////////////////////////////////////
\subsubsection{c\_esmc\_componentsvcloop() (Source File: ESMCI\_WebServ\_F.C)}


  
\bigskip{\sf INTERFACE:}
\begin{verbatim} void FTN_X(c_esmc_componentsvcloop)(\end{verbatim}{\em RETURN VALUE:}
\begin{verbatim} \end{verbatim}{\em ARGUMENTS:}
\begin{verbatim}   char*                   clientId,             // (in) the client identifier
   char*                   registrarHost,
   ESMCI::GridComp*        comp,                                 // (in) the grid component
   ESMCI::State*           importState,          // (in) the component import state
   ESMCI::State*           exportState,          // (in) the component export state
   ESMCI::Clock*           clock,                                // (in) the component clock
   ESMC_BlockingFlag*      blockingFlag,         // (in) the blocking flag
   int*                    phase,                // (in) the phase
   int*                    portNum,                      // (in) the service port number
   int*                    rc,                         // (in) the return code
   ESMCI_FortranStrLenArg  clientIdLen,          // (in) the length of the client id
   ESMCI_FortranStrLenArg  registrarHostLen
   )\end{verbatim}
{\sf DESCRIPTION:\\ }


      Creates a component service on the specified port and calls the
      loop method to listen for client requests.
   
%/////////////////////////////////////////////////////////////
 
\mbox{}\hrulefill\
 
\subsubsection{c\_esmc\_cplcomponentsvcloop() (Source File: ESMCI\_WebServ\_F.C)}


  
\bigskip{\sf INTERFACE:}
\begin{verbatim} void FTN_X(c_esmc_cplcomponentsvcloop)(\end{verbatim}{\em RETURN VALUE:}
\begin{verbatim} \end{verbatim}{\em ARGUMENTS:}
\begin{verbatim}   char*                   clientId,             // (in) the client identifier
   char*                   registrarHost,
   ESMCI::CplComp*         comp,                                 // (in) the grid component
   ESMCI::State*           importState,          // (in) the component import state
   ESMCI::State*           exportState,          // (in) the component export state
   ESMCI::Clock*           clock,                                // (in) the component clock
   ESMC_BlockingFlag*      blockingFlag,         // (in) the blocking flag
   int*                    phase,                // (in) the phase
   int*                    portNum,                      // (in) the service port number
   int*                    rc,                         // (in) the return code
   ESMCI_FortranStrLenArg  clientIdLen,          // (in) the length of the client id
   ESMCI_FortranStrLenArg  registrarHostLen
   )\end{verbatim}
{\sf DESCRIPTION:\\ }


      Creates a component service on the specified port and calls the
      loop method to listen for client requests.
   
%/////////////////////////////////////////////////////////////
 
\mbox{}\hrulefill\
 
\subsubsection{c\_esmc\_registercomponent() (Source File: ESMCI\_WebServ\_F.C)}


  
\bigskip{\sf INTERFACE:}
\begin{verbatim} void FTN_X(c_esmc_registercomponent)(\end{verbatim}{\em RETURN VALUE:}
\begin{verbatim} \end{verbatim}{\em ARGUMENTS:}
\begin{verbatim}   char*                   compName,             // (in) the grid component name
   char*                   compDesc,             // (in) the grid component description
   char*                   clientId,     // (in) the client identifier
   char*                   registrarHost,
   int*                    portNum,     // (in) the service port number
   int*                    rc,          // (in) the return code
   ESMCI_FortranStrLenArg  compNameLen,  // (in) the length of the component name
   ESMCI_FortranStrLenArg  compDescLen,  // (in) the length of the comp desc
   ESMCI_FortranStrLenArg  clientIdLen,  // (in) the length of the client id
   ESMCI_FortranStrLenArg  registrarHostLen
   )\end{verbatim}
{\sf DESCRIPTION:\\ }


      Parses the input parameters and uses that information to register the
      component with the Registrar.
   
%/////////////////////////////////////////////////////////////
 
\mbox{}\hrulefill\
 
\subsubsection{c\_esmc\_unregistercomponent() (Source File: ESMCI\_WebServ\_F.C)}


  
\bigskip{\sf INTERFACE:}
\begin{verbatim} void FTN_X(c_esmc_unregistercomponent)(\end{verbatim}{\em RETURN VALUE:}
\begin{verbatim} \end{verbatim}{\em ARGUMENTS:}
\begin{verbatim}   char*                   clientId,             // (in) the client identifier
   char*                   registrarHost,
   int*                    rc,          // (in) the return code
   ESMCI_FortranStrLenArg  clientIdLen,  // (in) the length of the clientId
   ESMCI_FortranStrLenArg  registrarHostLen
   )\end{verbatim}
{\sf DESCRIPTION:\\ }


      Parses the input parameters and uses that information to register the
      component with the Registrar.
   
%/////////////////////////////////////////////////////////////
 
\mbox{}\hrulefill\
 
\subsubsection{c\_esmc\_getportnum() (Source File: ESMCI\_WebServ\_F.C)}


  
\bigskip{\sf INTERFACE:}
\begin{verbatim} void FTN_X(c_esmc_getportnum)(\end{verbatim}{\em RETURN VALUE:}
\begin{verbatim} \end{verbatim}{\em ARGUMENTS:}
\begin{verbatim}   int*               portNum,                   // (out) the service port number
   int*               rc                               // (out) the return code
   )\end{verbatim}
{\sf DESCRIPTION:\\ }


      Determines a suitable, available port number for the service.
   
%/////////////////////////////////////////////////////////////
 
\mbox{}\hrulefill\
 
\subsubsection{c\_esmc\_addoutputfilename() (Source File: ESMCI\_WebServ\_F.C)}


  
\bigskip{\sf INTERFACE:}
\begin{verbatim} void FTN_X(c_esmc_addoutputfilename)(\end{verbatim}{\em RETURN VALUE:}
\begin{verbatim} \end{verbatim}{\em ARGUMENTS:}
\begin{verbatim}   char*                   filename,             // (in) the output filename
   int*                    rc,                      // (out) the return code
   ESMCI_FortranStrLenArg  filenameLen   // (in) the length of the filename
   )\end{verbatim}
{\sf DESCRIPTION:\\ }


      Adds a filename to the list of output filenames.
   
%/////////////////////////////////////////////////////////////
 
\mbox{}\hrulefill\
 
\subsubsection{c\_esmc\_addoutputdata() (Source File: ESMCI\_WebServ\_F.C)}


  
\bigskip{\sf INTERFACE:}
\begin{verbatim} void FTN_X(c_esmc_addoutputdata)(\end{verbatim}{\em RETURN VALUE:}
\begin{verbatim} \end{verbatim}{\em ARGUMENTS:}
\begin{verbatim}   double*                 timestamp,    // (in)
   char*                   varName,              // (in)
   double**                dataValues,   // (in)
   int*                    rc,                      // (out) the return code
   ESMCI_FortranStrLenArg  varNameLen    // (in) the length of the var name
   )\end{verbatim}
{\sf DESCRIPTION:\\ }


      Adds output data to the current output data structure.
  
%...............................................................
\setlength{\parskip}{\oldparskip}
\setlength{\parindent}{\oldparindent}
\setlength{\baselineskip}{\oldbaselineskip}
