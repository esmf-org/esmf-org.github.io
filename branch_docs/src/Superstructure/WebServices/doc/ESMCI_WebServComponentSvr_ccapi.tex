%                **** IMPORTANT NOTICE *****
% This LaTeX file has been automatically produced by ProTeX v. 1.1
% Any changes made to this file will likely be lost next time
% this file is regenerated from its source. Send questions 
% to Arlindo da Silva, dasilva@gsfc.nasa.gov
 
\setlength{\oldparskip}{\parskip}
\setlength{\parskip}{1.5ex}
\setlength{\oldparindent}{\parindent}
\setlength{\parindent}{0pt}
\setlength{\oldbaselineskip}{\baselineskip}
\setlength{\baselineskip}{11pt}
 
%--------------------- SHORT-HAND MACROS ----------------------
\def\bv{\begin{verbatim}}
\def\ev{\end{verbatim}}
\def\be{\begin{equation}}
\def\ee{\end{equation}}
\def\bea{\begin{eqnarray}}
\def\eea{\end{eqnarray}}
\def\bi{\begin{itemize}}
\def\ei{\end{itemize}}
\def\bn{\begin{enumerate}}
\def\en{\end{enumerate}}
\def\bd{\begin{description}}
\def\ed{\end{description}}
\def\({\left (}
\def\){\right )}
\def\[{\left [}
\def\]{\right ]}
\def\<{\left  \langle}
\def\>{\right \rangle}
\def\cI{{\cal I}}
\def\diag{\mathop{\rm diag}}
\def\tr{\mathop{\rm tr}}
%-------------------------------------------------------------

\markboth{Left}{Source File: ESMCI\_WebServComponentSvr.C,  Date: Tue May  5 21:00:15 MDT 2020
}

 
%/////////////////////////////////////////////////////////////
\subsubsection{ESMCI\_WebServComponentSvr::ESMCI\_WebServComponentSvr() (Source File: ESMCI\_WebServComponentSvr.C)}


  
\bigskip{\sf INTERFACE:}
\begin{verbatim} ESMCI_WebServComponentSvr::ESMCI_WebServComponentSvr(\end{verbatim}{\em ARGUMENTS:}
\begin{verbatim}   int  port,       // (in) the port number on which to setup the socket service
                    // to listen for requests
   int  clientId,   // (in) the id of the client for whom this service is
                    // being run
   string  registrarHost  // (in) name of host one which registrar is running
   )\end{verbatim}
{\sf DESCRIPTION:\\ }


      Initialize the ESMF Component service with the default values as well
      as the specified port number.
   
%/////////////////////////////////////////////////////////////
 
\mbox{}\hrulefill\
 
\subsubsection{ESMCI\_WebServComponentSvr::~ESMCI\_WebServComponentSvr() (Source File: ESMCI\_WebServComponentSvr.C)}


  
\bigskip{\sf INTERFACE:}
\begin{verbatim} ESMCI_WebServComponentSvr::~ESMCI_WebServComponentSvr(\end{verbatim}{\em ARGUMENTS:}
\begin{verbatim}   )\end{verbatim}
{\sf DESCRIPTION:\\ }


      Cleanup the component service.  For now, all this involves is making
      sure the socket is disconnected.
   
%/////////////////////////////////////////////////////////////
 
\mbox{}\hrulefill\
 
\subsubsection{ESMCI\_WebServComponentSvr::setPort() (Source File: ESMCI\_WebServComponentSvr.C)}


  
\bigskip{\sf INTERFACE:}
\begin{verbatim} void  ESMCI_WebServComponentSvr::setPort(\end{verbatim}{\em RETURN VALUE:}
\begin{verbatim} \end{verbatim}{\em ARGUMENTS:}
\begin{verbatim}   int  port    // (in) number of the port on which component service listens
                // for requests
   )\end{verbatim}
{\sf DESCRIPTION:\\ }


      Sets the number of the port on which the component service listens
      for requests.
   
%/////////////////////////////////////////////////////////////
 
\mbox{}\hrulefill\
 
\subsubsection{ESMCI\_WebServComponentSvr::setOutputDesc() (Source File: ESMCI\_WebServComponentSvr.C)}


  
\bigskip{\sf INTERFACE:}
\begin{verbatim} void  ESMCI_WebServComponentSvr::setOutputDesc(\end{verbatim}{\em RETURN VALUE:}
\begin{verbatim} \end{verbatim}{\em ARGUMENTS:}
\begin{verbatim}   ESMCI_WebServDataDesc*  desc  // (in) description of output data
   )\end{verbatim}
{\sf DESCRIPTION:\\ }


      Allocates the output data structure and sets up its description.
   
%/////////////////////////////////////////////////////////////
 
\mbox{}\hrulefill\
 
\subsubsection{ESMCI\_WebServComponentSvr::addOutputFilename() (Source File: ESMCI\_WebServComponentSvr.C)}


  
\bigskip{\sf INTERFACE:}
\begin{verbatim} void  ESMCI_WebServComponentSvr::addOutputFilename(\end{verbatim}{\em RETURN VALUE:}
\begin{verbatim} \end{verbatim}{\em ARGUMENTS:}
\begin{verbatim}   string  filename    // the name of the output filename to add to the list
   )\end{verbatim}
{\sf DESCRIPTION:\\ }


      Adds the specified filename to the list of output filenames.
   
%/////////////////////////////////////////////////////////////
 
\mbox{}\hrulefill\
 
\subsubsection{ESMCI\_WebServComponentSvr::addOutputData() (Source File: ESMCI\_WebServComponentSvr.C)}


  
\bigskip{\sf INTERFACE:}
\begin{verbatim} void  ESMCI_WebServComponentSvr::addOutputData(\end{verbatim}{\em RETURN VALUE:}
\begin{verbatim} \end{verbatim}{\em ARGUMENTS:}
\begin{verbatim}   TODO: Change structure to support output data format
   string  filename    // the name of the output filename to add to the list
   )\end{verbatim}
{\sf DESCRIPTION:\\ }


      Adds the specified data information to the list of output data.
   
%/////////////////////////////////////////////////////////////
 
\mbox{}\hrulefill\
 
\subsubsection{ESMCI\_WebServComponentSvr::requestLoop() (Source File: ESMCI\_WebServComponentSvr.C)}


  
\bigskip{\sf INTERFACE:}
\begin{verbatim} int  ESMCI_WebServComponentSvr::requestLoop(\end{verbatim}{\em RETURN VALUE:}
\begin{verbatim}      {\tt ESMF\_SUCCESS} or error code on failure.\end{verbatim}{\em ARGUMENTS:}
\begin{verbatim}   ESMCI::GridComp*   comp,          // (in) the grid component
   ESMCI::State*      importState,   // (in) import state
   ESMCI::State*      exportState,   // (in) export state
   ESMCI::Clock*      clock,         // (in) clock
   int                phase,         // (in) phase
   ESMC_BlockingFlag  blockingFlag   // (in) blocking flag
   )\end{verbatim}
{\sf DESCRIPTION:\\ }


      Sets up a socket service for a grid component server to handle client
      requests.  The input parameters are all saved for later use when the
      client makes requests of the server to initialize, run, and finalize.
   
%/////////////////////////////////////////////////////////////
 
\mbox{}\hrulefill\
 
\subsubsection{ESMCI\_WebServComponentSvr::cplCompRequestLoop() (Source File: ESMCI\_WebServComponentSvr.C)}


  
\bigskip{\sf INTERFACE:}
\begin{verbatim} int  ESMCI_WebServComponentSvr::cplCompRequestLoop(\end{verbatim}{\em RETURN VALUE:}
\begin{verbatim}      {\tt ESMF\_SUCCESS} or error code on failure.\end{verbatim}{\em ARGUMENTS:}
\begin{verbatim}   ESMCI::CplComp*    comp,          // (in) the coupler component
   ESMCI::State*      importState,   // (in) import state
   ESMCI::State*      exportState,   // (in) export state
   ESMCI::Clock*      clock,         // (in) clock
   int                phase,         // (in) phase
   ESMC_BlockingFlag  blockingFlag   // (in) blocking flag
   )\end{verbatim}
{\sf DESCRIPTION:\\ }


      Sets up a socket service for a grid component server to handle client
      requests.  The input parameters are all saved for later use when the
      client makes requests of the server to initialize, run, and finalize.
   
%/////////////////////////////////////////////////////////////
 
\mbox{}\hrulefill\
 
\subsubsection{ESMCI\_WebServComponentSvr::getNextRequest() (Source File: ESMCI\_WebServComponentSvr.C)}


  
\bigskip{\sf INTERFACE:}
\begin{verbatim} int  ESMCI_WebServComponentSvr::getNextRequest(\end{verbatim}{\em RETURN VALUE:}
\begin{verbatim}      int  id of the client request (defined in ESMCI_WebServNetEsmf.h);
           ESMF_FAILURE if error\end{verbatim}{\em ARGUMENTS:}
\begin{verbatim}   )\end{verbatim}
{\sf DESCRIPTION:\\ }


      Listens on a server socket for client requests, and as the requests
      arrive, reads the request id from the socket and returns it.
   
%/////////////////////////////////////////////////////////////
 
\mbox{}\hrulefill\
 
\subsubsection{ESMCI\_WebServComponentSvr::serviceRequest() (Source File: ESMCI\_WebServComponentSvr.C)}


  
\bigskip{\sf INTERFACE:}
\begin{verbatim} int  ESMCI_WebServComponentSvr::serviceRequest(\end{verbatim}{\em RETURN VALUE:}
\begin{verbatim}      int  id of the client request (the same value that's passed in)\end{verbatim}{\em ARGUMENTS:}
\begin{verbatim}   int  request    // id of the client request
   )\end{verbatim}
{\sf DESCRIPTION:\\ }


      Calls the appropriate process method based on the client request id.
   
%/////////////////////////////////////////////////////////////
 
\mbox{}\hrulefill\
 
\subsubsection{ESMCI\_WebServComponentSvr::setStatus() (Source File: ESMCI\_WebServComponentSvr.C)}


  
\bigskip{\sf INTERFACE:}
\begin{verbatim} void  ESMCI_WebServComponentSvr::setStatus(\end{verbatim}{\em RETURN VALUE:}
\begin{verbatim} \end{verbatim}{\em ARGUMENTS:}
\begin{verbatim}   int  status      // new status value
   )\end{verbatim}
{\sf DESCRIPTION:\\ }


      Sets the current status... has to lock the status mutex before setting
      it and unlock the mutex after setting it.
   
%/////////////////////////////////////////////////////////////
 
\mbox{}\hrulefill\
 
\subsubsection{ESMCI\_WebServComponentSvr::processInit() (Source File: ESMCI\_WebServComponentSvr.C)}


  
\bigskip{\sf INTERFACE:}
\begin{verbatim} int  ESMCI_WebServComponentSvr::processInit(\end{verbatim}{\em RETURN VALUE:}
\begin{verbatim}      {\tt ESMF\_SUCCESS} or error code on failure.\end{verbatim}{\em ARGUMENTS:}
\begin{verbatim}   )\end{verbatim}
{\sf DESCRIPTION:\\ }


      Processes the request to initialize the component.  This method reads the
      client id from the socket and uses it to validate the client information.
      It then reads the names of input files (if any) from the socket.  It
      then creates a new thread which is responsible for calling the component
      initialization routine and writing the component status to the socket
      to complete the transaction.
  
      (KDS: The whole import file stuff was not used for the CCSM/CAM project,
            so I removed all of the file processing code (it was commented out
            anyways), but left a placeholder if it needs to be added back in.
   
%/////////////////////////////////////////////////////////////
 
\mbox{}\hrulefill\
 
\subsubsection{ESMCI\_WebServComponentSvr::processRun() (Source File: ESMCI\_WebServComponentSvr.C)}


  
\bigskip{\sf INTERFACE:}
\begin{verbatim} int  ESMCI_WebServComponentSvr::processRun(\end{verbatim}{\em RETURN VALUE:}
\begin{verbatim}      {\tt ESMF\_SUCCESS} or error code on failure.\end{verbatim}{\em ARGUMENTS:}
\begin{verbatim}   )\end{verbatim}
{\sf DESCRIPTION:\\ }


      Processes the request to run the component.  This method reads the
      client id from the socket and uses it to validate the client information.
      It then creates a new thread which is responsible for calling the
      component run routine and writing the component status to the socket
      to complete the transaction.
   
%/////////////////////////////////////////////////////////////
 
\mbox{}\hrulefill\
 
\subsubsection{ESMCI\_WebServComponentSvr::processTimestep() (Source File: ESMCI\_WebServComponentSvr.C)}


  
\bigskip{\sf INTERFACE:}
\begin{verbatim} int  ESMCI_WebServComponentSvr::processTimestep(\end{verbatim}{\em RETURN VALUE:}
\begin{verbatim}      {\tt ESMF\_SUCCESS} or error code on failure.\end{verbatim}{\em ARGUMENTS:}
\begin{verbatim}   )\end{verbatim}
{\sf DESCRIPTION:\\ }


      Processes the request to run the component for the specified number of
      timesteps.  This method reads the client id from the socket and uses it
      to validate the client information.  It then gets the value for the
      number of timesteps to run from the client.
      It creates a new thread which is responsible for calling the
      component run routine and writing the component status to the socket
      to complete the transaction.
   
%/////////////////////////////////////////////////////////////
 
\mbox{}\hrulefill\
 
\subsubsection{ESMCI\_WebServComponentSvr::processFinal() (Source File: ESMCI\_WebServComponentSvr.C)}


  
\bigskip{\sf INTERFACE:}
\begin{verbatim} int  ESMCI_WebServComponentSvr::processFinal(\end{verbatim}{\em RETURN VALUE:}
\begin{verbatim}      {\tt ESMF\_SUCCESS} or error code on failure.\end{verbatim}{\em ARGUMENTS:}
\begin{verbatim}   )\end{verbatim}
{\sf DESCRIPTION:\\ }


      Processes the request to finalize the component.  This method reads the
      client id from the socket and uses it to validate the client information.
      It then creates a new thread which is responsible for calling the
      component finalize routine and writing the component status to the
      socket to complete the transaction.
   
%/////////////////////////////////////////////////////////////
 
\mbox{}\hrulefill\
 
\subsubsection{ESMCI\_WebServComponentSvr::processState() (Source File: ESMCI\_WebServComponentSvr.C)}


  
\bigskip{\sf INTERFACE:}
\begin{verbatim} int  ESMCI_WebServComponentSvr::processState(\end{verbatim}{\em RETURN VALUE:}
\begin{verbatim}      {\tt ESMF\_SUCCESS} or error code on failure.\end{verbatim}{\em ARGUMENTS:}
\begin{verbatim}   )\end{verbatim}
{\sf DESCRIPTION:\\ }


      Processes the request to retrieve the component state.  This method
      reads the client id from the socket (the client id is actually not used
      right now). The component state is then written to the socket to
      complete the transaction.
   
%/////////////////////////////////////////////////////////////
 
\mbox{}\hrulefill\
 
\subsubsection{ESMCI\_WebServComponentSvr::processFiles() (Source File: ESMCI\_WebServComponentSvr.C)}


  
\bigskip{\sf INTERFACE:}
\begin{verbatim} int  ESMCI_WebServComponentSvr::processFiles(\end{verbatim}{\em RETURN VALUE:}
\begin{verbatim}      {\tt ESMF\_SUCCESS} or error code on failure.\end{verbatim}{\em ARGUMENTS:}
\begin{verbatim}   )\end{verbatim}
{\sf DESCRIPTION:\\ }


      Processes the request to retrieve the export filenames.  This method
      reads the client id from the socket and uses it to validate the client
      information. Next, the list of export files is written out to the
      socket.  And finally, the component status is written to the socket
      to complete the transaction.
   
%/////////////////////////////////////////////////////////////
 
\mbox{}\hrulefill\
 
\subsubsection{ESMCI\_WebServComponentSvr::processGetDataDesc() (Source File: ESMCI\_WebServComponentSvr.C)}


  
\bigskip{\sf INTERFACE:}
\begin{verbatim} int  ESMCI_WebServComponentSvr::processGetDataDesc(\end{verbatim}{\em RETURN VALUE:}
\begin{verbatim}      {\tt ESMF\_SUCCESS} or error code on failure.\end{verbatim}{\em ARGUMENTS:}
\begin{verbatim}   )\end{verbatim}
{\sf DESCRIPTION:\\ }


      Processes the request to retrieve the export data.  This method
      reads the client id from the socket and uses it to lookup the client
      information.  It then reads the data parameters (variable name, time,
      lat and lon) from the socket and uses that information to read the
      data from the socket.  The data and the component status are then
      written back to the socket to complete the transaction.
  
      (KDS: This design is very specific to CCSM/CAM and is hardcoded for
            that prototype.  This needs to be redesigned to be more generic.)
      (KDS: Also, getting one value for a specific time/lat/lon is really
            inefficient and not practical.  There needs to be a way to handle
            more data values at a time.)
   
%/////////////////////////////////////////////////////////////
 
\mbox{}\hrulefill\
 
\subsubsection{ESMCI\_WebServComponentSvr::processGetData() (Source File: ESMCI\_WebServComponentSvr.C)}


  
\bigskip{\sf INTERFACE:}
\begin{verbatim} int  ESMCI_WebServComponentSvr::processGetData(\end{verbatim}{\em RETURN VALUE:}
\begin{verbatim}      {\tt ESMF\_SUCCESS} or error code on failure.\end{verbatim}{\em ARGUMENTS:}
\begin{verbatim}   )\end{verbatim}
{\sf DESCRIPTION:\\ }


      Processes the request to retrieve the export data.  This method
      reads the client id from the socket and uses it to lookup the client
      information.  It then reads the data parameters (variable name, time,
      lat and lon) from the socket and uses that information to read the
      data from the socket.  The data and the component status are then
      written back to the socket to complete the transaction.
  
      (KDS: This design is very specific to CCSM/CAM and is hardcoded for
            that prototype.  This needs to be redesigned to be more generic.)
      (KDS: Also, getting one value for a specific time/lat/lon is really
            inefficient and not practical.  There needs to be a way to handle
            more data values at a time.)
   
%/////////////////////////////////////////////////////////////
 
\mbox{}\hrulefill\
 
\subsubsection{ESMCI\_WebServComponentSvr::processEnd() (Source File: ESMCI\_WebServComponentSvr.C)}


  
\bigskip{\sf INTERFACE:}
\begin{verbatim} int  ESMCI_WebServComponentSvr::processEnd(\end{verbatim}{\em RETURN VALUE:}
\begin{verbatim}      {\tt ESMF\_SUCCESS} or error code on failure.\end{verbatim}{\em ARGUMENTS:}
\begin{verbatim}   )\end{verbatim}
{\sf DESCRIPTION:\\ }


      Processes the request to end a client session.  This method reads the
      client id from the socket and uses it to validate the client.
      The component status is updated and written to the socket to complete
      the transaction.
      KDS: I think this method is not necessary any more, since the process
           will kill the server when it's completed.
   
%/////////////////////////////////////////////////////////////
 
\mbox{}\hrulefill\
 
\subsubsection{ESMCI\_WebServComponentSvr::runInit() (Source File: ESMCI\_WebServComponentSvr.C)}


  
\bigskip{\sf INTERFACE:}
\begin{verbatim} void  ESMCI_WebServComponentSvr::runInit(\end{verbatim}{\em RETURN VALUE:}
\begin{verbatim} \end{verbatim}{\em ARGUMENTS:}
\begin{verbatim}   )\end{verbatim}
{\sf DESCRIPTION:\\ }


      Makes the call to the grid component initialization routine.
   
%/////////////////////////////////////////////////////////////
 
\mbox{}\hrulefill\
 
\subsubsection{ESMCI\_WebServComponentSvr::runRun() (Source File: ESMCI\_WebServComponentSvr.C)}


  
\bigskip{\sf INTERFACE:}
\begin{verbatim} void  ESMCI_WebServComponentSvr::runRun(\end{verbatim}{\em RETURN VALUE:}
\begin{verbatim} \end{verbatim}{\em ARGUMENTS:}
\begin{verbatim}   )\end{verbatim}
{\sf DESCRIPTION:\\ }


      Makes the call to the grid component run routine.
   
%/////////////////////////////////////////////////////////////
 
\mbox{}\hrulefill\
 
\subsubsection{ESMCI\_WebServComponentSvr::runTimeStep() (Source File: ESMCI\_WebServComponentSvr.C)}


  
\bigskip{\sf INTERFACE:}
\begin{verbatim} void  ESMCI_WebServComponentSvr::runTimeStep(\end{verbatim}{\em RETURN VALUE:}
\begin{verbatim} \end{verbatim}{\em ARGUMENTS:}
\begin{verbatim}   )\end{verbatim}
{\sf DESCRIPTION:\\ }


      Makes the call to the grid component timestep routine.
   
%/////////////////////////////////////////////////////////////
 
\mbox{}\hrulefill\
 
\subsubsection{ESMCI\_WebServComponentSvr::runFinal() (Source File: ESMCI\_WebServComponentSvr.C)}


  
\bigskip{\sf INTERFACE:}
\begin{verbatim} void  ESMCI_WebServComponentSvr::runFinal(\end{verbatim}{\em RETURN VALUE:}
\begin{verbatim} \end{verbatim}{\em ARGUMENTS:}
\begin{verbatim}   )\end{verbatim}
{\sf DESCRIPTION:\\ }


      Makes the call to the grid component finalization routine.
   
%/////////////////////////////////////////////////////////////
 
\mbox{}\hrulefill\
 
\subsubsection{initThreadStartup() (Source File: ESMCI\_WebServComponentSvr.C)}


  
\bigskip{\sf INTERFACE:}
\begin{verbatim} void*  initThreadStartup(\end{verbatim}{\em RETURN VALUE:}
\begin{verbatim} \end{verbatim}{\em ARGUMENTS:}
\begin{verbatim}   void*  tgtObject      // the component service object
   )\end{verbatim}
{\sf DESCRIPTION:\\ }


      Function called to run the initialization method for a component service.
   
%/////////////////////////////////////////////////////////////
 
\mbox{}\hrulefill\
 
\subsubsection{runThreadStartup() (Source File: ESMCI\_WebServComponentSvr.C)}


  
\bigskip{\sf INTERFACE:}
\begin{verbatim} void*  runThreadStartup(\end{verbatim}{\em RETURN VALUE:}
\begin{verbatim} \end{verbatim}{\em ARGUMENTS:}
\begin{verbatim}   void*  tgtObject      // the component service object
   )\end{verbatim}
{\sf DESCRIPTION:\\ }


      Function called to run the run method for a component service.
   
%/////////////////////////////////////////////////////////////
 
\mbox{}\hrulefill\
 
\subsubsection{timeStepThreadStartup() (Source File: ESMCI\_WebServComponentSvr.C)}


  
\bigskip{\sf INTERFACE:}
\begin{verbatim} void*  timeStepThreadStartup(\end{verbatim}{\em RETURN VALUE:}
\begin{verbatim} \end{verbatim}{\em ARGUMENTS:}
\begin{verbatim}   void*  tgtObject      // the component service object
   )\end{verbatim}
{\sf DESCRIPTION:\\ }


      Function called to run the timestep method for a component service.
   
%/////////////////////////////////////////////////////////////
 
\mbox{}\hrulefill\
 
\subsubsection{finalThreadStartup() (Source File: ESMCI\_WebServComponentSvr.C)}


  
\bigskip{\sf INTERFACE:}
\begin{verbatim} void*  finalThreadStartup(\end{verbatim}{\em RETURN VALUE:}
\begin{verbatim} \end{verbatim}{\em ARGUMENTS:}
\begin{verbatim}   void*  tgtObject      // the component service object
   )\end{verbatim}
{\sf DESCRIPTION:\\ }


      Function called to run the finalization method for a component service.
  
%...............................................................
\setlength{\parskip}{\oldparskip}
\setlength{\parindent}{\oldparindent}
\setlength{\baselineskip}{\oldbaselineskip}
