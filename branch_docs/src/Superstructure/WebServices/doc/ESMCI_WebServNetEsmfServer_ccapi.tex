%                **** IMPORTANT NOTICE *****
% This LaTeX file has been automatically produced by ProTeX v. 1.1
% Any changes made to this file will likely be lost next time
% this file is regenerated from its source. Send questions 
% to Arlindo da Silva, dasilva@gsfc.nasa.gov
 
\setlength{\oldparskip}{\parskip}
\setlength{\parskip}{1.5ex}
\setlength{\oldparindent}{\parindent}
\setlength{\parindent}{0pt}
\setlength{\oldbaselineskip}{\baselineskip}
\setlength{\baselineskip}{11pt}
 
%--------------------- SHORT-HAND MACROS ----------------------
\def\bv{\begin{verbatim}}
\def\ev{\end{verbatim}}
\def\be{\begin{equation}}
\def\ee{\end{equation}}
\def\bea{\begin{eqnarray}}
\def\eea{\end{eqnarray}}
\def\bi{\begin{itemize}}
\def\ei{\end{itemize}}
\def\bn{\begin{enumerate}}
\def\en{\end{enumerate}}
\def\bd{\begin{description}}
\def\ed{\end{description}}
\def\({\left (}
\def\){\right )}
\def\[{\left [}
\def\]{\right ]}
\def\<{\left  \langle}
\def\>{\right \rangle}
\def\cI{{\cal I}}
\def\diag{\mathop{\rm diag}}
\def\tr{\mathop{\rm tr}}
%-------------------------------------------------------------

\markboth{Left}{Source File: ESMCI\_WebServNetEsmfServer.C,  Date: Tue May  5 21:00:15 MDT 2020
}

 
%/////////////////////////////////////////////////////////////
\subsubsection{ESMCI\_WebServNetEsmfServer::ESMCI\_WebServNetEsmfServer() (Source File: ESMCI\_WebServNetEsmfServer.C)}


  
\bigskip{\sf INTERFACE:}
\begin{verbatim} ESMCI_WebServNetEsmfServer::ESMCI_WebServNetEsmfServer(\end{verbatim}{\em ARGUMENTS:}
\begin{verbatim}   int  port             // (in) the port number on which to setup the socket service
                                         // to listen for requests
   )\end{verbatim}
{\sf DESCRIPTION:\\ }


      Initialize the ESMF Component service with the default values as well
      as the specified port number.
   
%/////////////////////////////////////////////////////////////
 
\mbox{}\hrulefill\
 
\subsubsection{ESMCI\_WebServNetEsmfServer::~ESMCI\_WebServNetEsmfServer() (Source File: ESMCI\_WebServNetEsmfServer.C)}


  
\bigskip{\sf INTERFACE:}
\begin{verbatim} ESMCI_WebServNetEsmfServer::~ESMCI_WebServNetEsmfServer(\end{verbatim}{\em ARGUMENTS:}
\begin{verbatim}   )\end{verbatim}
{\sf DESCRIPTION:\\ }


      Cleanup the component service.  For now, all this involves is making
      sure the socket is disconnected.
   
%/////////////////////////////////////////////////////////////
 
\mbox{}\hrulefill\
 
\subsubsection{ESMCI\_WebServNetEsmfServer::setPort() (Source File: ESMCI\_WebServNetEsmfServer.C)}


  
\bigskip{\sf INTERFACE:}
\begin{verbatim} void  ESMCI_WebServNetEsmfServer::setPort(\end{verbatim}{\em RETURN VALUE:}
\begin{verbatim} \end{verbatim}{\em ARGUMENTS:}
\begin{verbatim}   int  port             // (in) number of the port on which component service listens
                                         // for requests
   )\end{verbatim}
{\sf DESCRIPTION:\\ }


      Sets the number of the port on which the component service listens
      for requests.
   
%/////////////////////////////////////////////////////////////
 
\mbox{}\hrulefill\
 
\subsubsection{ESMCI\_WebServNetEsmfServer::requestLoop() (Source File: ESMCI\_WebServNetEsmfServer.C)}


  
\bigskip{\sf INTERFACE:}
\begin{verbatim} int  ESMCI_WebServNetEsmfServer::requestLoop(\end{verbatim}{\em RETURN VALUE:}
\begin{verbatim}      {\tt ESMF\_SUCCESS} or error code on failure.\end{verbatim}{\em ARGUMENTS:}
\begin{verbatim}   ESMCI::GridComp*   comp,                              // (in) the grid component
   ESMCI::State*      importState,       // (in) import state
   ESMCI::State*      exportState,       // (in) export state
   ESMCI::Clock*      clock,                     // (in) clock
   int                phase,                     // (in) phase
   ESMC_BlockingFlag  blockingFlag       // (in) blocking flag
   )\end{verbatim}
{\sf DESCRIPTION:\\ }


      Sets up a socket service for a grid component server to handle client
      requests.  The input parameters are all saved for later use when the
      client makes requests of the server to initialize, run, and finalize.
   
%/////////////////////////////////////////////////////////////
 
\mbox{}\hrulefill\
 
\subsubsection{ESMCI\_WebServNetEsmfServer::requestLoop() (Source File: ESMCI\_WebServNetEsmfServer.C)}


  
\bigskip{\sf INTERFACE:}
\begin{verbatim} int  ESMCI_WebServNetEsmfServer::requestLoop(\end{verbatim}{\em RETURN VALUE:}
\begin{verbatim}      {\tt ESMF\_SUCCESS} or error code on failure.\end{verbatim}{\em ARGUMENTS:}
\begin{verbatim}   ESMCI::CplComp*    comp,                              // (in) the coupler component
   ESMCI::State*      importState,       // (in) import state
   ESMCI::State*      exportState,       // (in) export state
   ESMCI::Clock*      clock,                     // (in) clock
   int                phase,                     // (in) phase
   ESMC_BlockingFlag  blockingFlag       // (in) blocking flag
   )\end{verbatim}
{\sf DESCRIPTION:\\ }


      Sets up a socket service for a coupler component server to handle client
      requests.  The input parameters are all saved for later use when the
      client makes requests of the server to initialize, run, and finalize.
   
%/////////////////////////////////////////////////////////////
 
\mbox{}\hrulefill\
 
\subsubsection{ESMCI\_WebServNetEsmfServer::getNextRequest() (Source File: ESMCI\_WebServNetEsmfServer.C)}


  
\bigskip{\sf INTERFACE:}
\begin{verbatim} int  ESMCI_WebServNetEsmfServer::getNextRequest(\end{verbatim}{\em RETURN VALUE:}
\begin{verbatim}      int  id of the client request (defined in ESMCI_WebServNetEsmf.h);
           ESMF_FAILURE if error\end{verbatim}{\em ARGUMENTS:}
\begin{verbatim}   )\end{verbatim}
{\sf DESCRIPTION:\\ }


      Listens on a server socket for client requests, and as the requests
      arrive, reads the request id from the socket and returns it.
   
%/////////////////////////////////////////////////////////////
 
\mbox{}\hrulefill\
 
\subsubsection{ESMCI\_WebServNetEsmfServer::serviceRequest() (Source File: ESMCI\_WebServNetEsmfServer.C)}


  
\bigskip{\sf INTERFACE:}
\begin{verbatim} int  ESMCI_WebServNetEsmfServer::serviceRequest(\end{verbatim}{\em RETURN VALUE:}
\begin{verbatim}      int  id of the client request (the same value that's passed in)\end{verbatim}{\em ARGUMENTS:}
\begin{verbatim}   int  request          // id of the client request
   )\end{verbatim}
{\sf DESCRIPTION:\\ }


      Calls the appropriate process method based on the client request id.
   
%/////////////////////////////////////////////////////////////
 
\mbox{}\hrulefill\
 
\subsubsection{ESMCI\_WebServNetEsmfServer::getRequestId() (Source File: ESMCI\_WebServNetEsmfServer.C)}


  
\bigskip{\sf INTERFACE:}
\begin{verbatim} int  ESMCI_WebServNetEsmfServer::getRequestId(\end{verbatim}{\em RETURN VALUE:}
\begin{verbatim}      int  id of the request based on the specified string; ESMF_FAILURE
           if the id cannot be found\end{verbatim}{\em ARGUMENTS:}
\begin{verbatim}   const char  request[]         // request string for which the id is to be returned
   )\end{verbatim}
{\sf DESCRIPTION:\\ }


      Looks up a request id based on a specified string value.
   
%/////////////////////////////////////////////////////////////
 
\mbox{}\hrulefill\
 
\subsubsection{ESMCI\_WebServNetEsmfServer::getRequestFromId() (Source File: ESMCI\_WebServNetEsmfServer.C)}


  
\bigskip{\sf INTERFACE:}
\begin{verbatim} char*  ESMCI_WebServNetEsmfServer::getRequestFromId(\end{verbatim}{\em RETURN VALUE:}
\begin{verbatim}      char*  string value for the specified request id; the string, "UNKN",
             if the value cannot be found\end{verbatim}{\em ARGUMENTS:}
\begin{verbatim}   int  id               // request id for which the string value is to be returned
   )\end{verbatim}
{\sf DESCRIPTION:\\ }


      Looks up a request string value based on a specified request id.
   
%/////////////////////////////////////////////////////////////
 
\mbox{}\hrulefill\
 
\subsubsection{ESMCI\_WebServNetEsmfServer::processNew() (Source File: ESMCI\_WebServNetEsmfServer.C)}


  
\bigskip{\sf INTERFACE:}
\begin{verbatim} int  ESMCI_WebServNetEsmfServer::processNew(\end{verbatim}{\em RETURN VALUE:}
\begin{verbatim}      {\tt ESMF\_SUCCESS} or error code on failure.\end{verbatim}{\em ARGUMENTS:}
\begin{verbatim}   )\end{verbatim}
{\sf DESCRIPTION:\\ }


      Processes the request for a new client session.  This method reads the
      client name from the socket, generates a new client id, creates a new
      client info object and adds it to the list of clients, and then writes
      the new client id to the socket to complete the transaction.
   
%/////////////////////////////////////////////////////////////
 
\mbox{}\hrulefill\
 
\subsubsection{ESMCI\_WebServNetEsmfServer::processInit() (Source File: ESMCI\_WebServNetEsmfServer.C)}


  
\bigskip{\sf INTERFACE:}
\begin{verbatim} int  ESMCI_WebServNetEsmfServer::processInit(\end{verbatim}{\em RETURN VALUE:}
\begin{verbatim}      {\tt ESMF\_SUCCESS} or error code on failure.\end{verbatim}{\em ARGUMENTS:}
\begin{verbatim}   )\end{verbatim}
{\sf DESCRIPTION:\\ }


      Processes the request to initialize the component.  This method reads the
      client id from the socket and uses it to lookup the client information.
      It then reads the names of input files (if any) from the socket and
      imports the input file contents into an ESMF import state object.
      Next, the component initialization routine is called, and finally, the
      component status is written to the socket to complete the transaction.
  
      (KDS: The whole import file stuff is still "iffy"... it works, but the
            file has to be locally accessible... and I currently only support
            one input file.)
   
%/////////////////////////////////////////////////////////////
 
\mbox{}\hrulefill\
 
\subsubsection{ESMCI\_WebServNetEsmfServer::processRun() (Source File: ESMCI\_WebServNetEsmfServer.C)}


  
\bigskip{\sf INTERFACE:}
\begin{verbatim} int  ESMCI_WebServNetEsmfServer::processRun(\end{verbatim}{\em RETURN VALUE:}
\begin{verbatim}      {\tt ESMF\_SUCCESS} or error code on failure.\end{verbatim}{\em ARGUMENTS:}
\begin{verbatim}   )\end{verbatim}
{\sf DESCRIPTION:\\ }


      Processes the request to run the component.  This method reads the
      client id from the socket and uses it to lookup the client information.
      Next, the component run routine is called, and finally, the
      component status is written to the socket to complete the transaction.
   
%/////////////////////////////////////////////////////////////
 
\mbox{}\hrulefill\
 
\subsubsection{ESMCI\_WebServNetEsmfServer::processFinal() (Source File: ESMCI\_WebServNetEsmfServer.C)}


  
\bigskip{\sf INTERFACE:}
\begin{verbatim} int  ESMCI_WebServNetEsmfServer::processFinal(\end{verbatim}{\em RETURN VALUE:}
\begin{verbatim}      {\tt ESMF\_SUCCESS} or error code on failure.\end{verbatim}{\em ARGUMENTS:}
\begin{verbatim}   )\end{verbatim}
{\sf DESCRIPTION:\\ }


      Processes the request to finalize the component.  This method reads the
      client id from the socket and uses it to lookup the client information.
      Next, the component finalize routine is called, and then the export
      state is written to the socket.  Finally, the component status is
      written to the socket to complete the transaction.
   
%/////////////////////////////////////////////////////////////
 
\mbox{}\hrulefill\
 
\subsubsection{ESMCI\_WebServNetEsmfServer::processState() (Source File: ESMCI\_WebServNetEsmfServer.C)}


  
\bigskip{\sf INTERFACE:}
\begin{verbatim} int  ESMCI_WebServNetEsmfServer::processState(\end{verbatim}{\em RETURN VALUE:}
\begin{verbatim}      {\tt ESMF\_SUCCESS} or error code on failure.\end{verbatim}{\em ARGUMENTS:}
\begin{verbatim}   )\end{verbatim}
{\sf DESCRIPTION:\\ }


      Processes the request to retrieve the component state.  This method
      reads the client id from the socket and uses it to lookup the client
      information. The component state is retrieved from the client
      information and is written to the socket to complete the transaction.
   
%/////////////////////////////////////////////////////////////
 
\mbox{}\hrulefill\
 
\subsubsection{ESMCI\_WebServNetEsmfServer::processFiles() (Source File: ESMCI\_WebServNetEsmfServer.C)}


  
\bigskip{\sf INTERFACE:}
\begin{verbatim} int  ESMCI_WebServNetEsmfServer::processFiles(\end{verbatim}{\em RETURN VALUE:}
\begin{verbatim}      {\tt ESMF\_SUCCESS} or error code on failure.\end{verbatim}{\em ARGUMENTS:}
\begin{verbatim}   )\end{verbatim}
{\sf DESCRIPTION:\\ }


      Processes the request to retrieve the export filenames.  This method
      reads the client id from the socket and uses it to lookup the client
      information. Next, list of export files is retrieved from the client
      information and is written out to the socket.  Finally, the component
      status is written to the socket to complete the transaction.
   
%/////////////////////////////////////////////////////////////
 
\mbox{}\hrulefill\
 
\subsubsection{ESMCI\_WebServNetEsmfServer::processEnd() (Source File: ESMCI\_WebServNetEsmfServer.C)}


  
\bigskip{\sf INTERFACE:}
\begin{verbatim} int  ESMCI_WebServNetEsmfServer::processEnd(\end{verbatim}{\em RETURN VALUE:}
\begin{verbatim}      {\tt ESMF\_SUCCESS} or error code on failure.\end{verbatim}{\em ARGUMENTS:}
\begin{verbatim}   )\end{verbatim}
{\sf DESCRIPTION:\\ }


      Processes the request to end a client session.  This method reads the
      client id from the socket and uses it to lookup the client information.
      The client information is deleted from the list of clients, and finally,
      the component status is written to the socket to complete the
      transaction.
   
%/////////////////////////////////////////////////////////////
 
\mbox{}\hrulefill\
 
\subsubsection{ESMCI\_WebServNetEsmfServer::processPing() (Source File: ESMCI\_WebServNetEsmfServer.C)}


  
\bigskip{\sf INTERFACE:}
\begin{verbatim} int  ESMCI_WebServNetEsmfServer::processPing(\end{verbatim}{\em RETURN VALUE:}
\begin{verbatim}      {\tt ESMF\_SUCCESS} or error code on failure.\end{verbatim}{\em ARGUMENTS:}
\begin{verbatim}   )\end{verbatim}
{\sf DESCRIPTION:\\ }


      Processes the request to ping the service.  Doesn't actually do anything.
   
%/////////////////////////////////////////////////////////////
 
\mbox{}\hrulefill\
 
\subsubsection{ESMCI\_WebServNetEsmfServer::getNextClientId() (Source File: ESMCI\_WebServNetEsmfServer.C)}


  
\bigskip{\sf INTERFACE:}
\begin{verbatim} int  ESMCI_WebServNetEsmfServer::getNextClientId(\end{verbatim}{\em RETURN VALUE:}
\begin{verbatim}      int  the next available client identifier\end{verbatim}{\em ARGUMENTS:}
\begin{verbatim}   )\end{verbatim}
{\sf DESCRIPTION:\\ }


      Increments the next client identifier value by one and returns the
      new value.
   
%/////////////////////////////////////////////////////////////
 
\mbox{}\hrulefill\
 
\subsubsection{ESMCI\_WebServNetEsmfServer::copyFile() (Source File: ESMCI\_WebServNetEsmfServer.C)}


  
\bigskip{\sf INTERFACE:}
\begin{verbatim} void  ESMCI_WebServNetEsmfServer::copyFile(\end{verbatim}{\em RETURN VALUE:}
\begin{verbatim} \end{verbatim}{\em ARGUMENTS:}
\begin{verbatim}   const char*  srcFilename,     // the name of the file to copy
   const char*  destFilename     // the name of the destination file
   )\end{verbatim}
{\sf DESCRIPTION:\\ }


      Copies the source file to the destination file.
  
%...............................................................
\setlength{\parskip}{\oldparskip}
\setlength{\parindent}{\oldparindent}
\setlength{\baselineskip}{\oldbaselineskip}
