%                **** IMPORTANT NOTICE *****
% This LaTeX file has been automatically produced by ProTeX v. 1.1
% Any changes made to this file will likely be lost next time
% this file is regenerated from its source. Send questions 
% to Arlindo da Silva, dasilva@gsfc.nasa.gov
 
\setlength{\oldparskip}{\parskip}
\setlength{\parskip}{1.5ex}
\setlength{\oldparindent}{\parindent}
\setlength{\parindent}{0pt}
\setlength{\oldbaselineskip}{\baselineskip}
\setlength{\baselineskip}{11pt}
 
%--------------------- SHORT-HAND MACROS ----------------------
\def\bv{\begin{verbatim}}
\def\ev{\end{verbatim}}
\def\be{\begin{equation}}
\def\ee{\end{equation}}
\def\bea{\begin{eqnarray}}
\def\eea{\end{eqnarray}}
\def\bi{\begin{itemize}}
\def\ei{\end{itemize}}
\def\bn{\begin{enumerate}}
\def\en{\end{enumerate}}
\def\bd{\begin{description}}
\def\ed{\end{description}}
\def\({\left (}
\def\){\right )}
\def\[{\left [}
\def\]{\right ]}
\def\<{\left  \langle}
\def\>{\right \rangle}
\def\cI{{\cal I}}
\def\diag{\mathop{\rm diag}}
\def\tr{\mathop{\rm tr}}
%-------------------------------------------------------------

\markboth{Left}{Source File: ESMCI\_WebServDataDesc.C,  Date: Tue May  5 21:00:15 MDT 2020
}

 
%/////////////////////////////////////////////////////////////
\subsubsection{ESMCI\_WebServDataDesc::ESMCI\_WebServDataDesc() (Source File: ESMCI\_WebServDataDesc.C)}


  
\bigskip{\sf INTERFACE:}
\begin{verbatim} ESMCI_WebServDataDesc::ESMCI_WebServDataDesc(\end{verbatim}{\em ARGUMENTS:}
\begin{verbatim}   int           numVars,                        // the number of variables in the set of data
   string*       varNames,       // the names of the variables contained in the set
   int           numLatValues,   // the number of latitudes for the data
   double*       latValues,      // the latitude values for the data
   int           numLonValues,   // the number of longitudes for the data
   double*       lonValues       // the longitude values for the data
   )\end{verbatim}
{\sf DESCRIPTION:\\ }


      Creates and sets up a container for a set of grid-based data variables
      for the specified timestamp.
   
%/////////////////////////////////////////////////////////////
 
\mbox{}\hrulefill\
 
\subsubsection{ESMCI\_WebServDataDesc::~ESMCI\_WebServDataDesc() (Source File: ESMCI\_WebServDataDesc.C)}


  
\bigskip{\sf INTERFACE:}
\begin{verbatim} ESMCI_WebServDataDesc::~ESMCI_WebServDataDesc(\end{verbatim}{\em ARGUMENTS:}
\begin{verbatim}   )\end{verbatim}
{\sf DESCRIPTION:\\ }


      Cleans up memory allocated for CAM output file.
   
%/////////////////////////////////////////////////////////////
 
\mbox{}\hrulefill\
 
\subsubsection{ESMCI\_WebServDataDesc::getVarIndex() (Source File: ESMCI\_WebServDataDesc.C)}


  
\bigskip{\sf INTERFACE:}
\begin{verbatim} int  ESMCI_WebServDataDesc::getVarIndex(\end{verbatim}{\em RETURN VALUE:}
\begin{verbatim}     int  index of specified time value in array of time values\end{verbatim}{\em ARGUMENTS:}
\begin{verbatim}   string  varName       // the variable name to lookup in the array of names
   )\end{verbatim}
{\sf DESCRIPTION:\\ }


      Looks up the index of the specified variable name in the array of
      variable names.
   
%/////////////////////////////////////////////////////////////
 
\mbox{}\hrulefill\
 
\subsubsection{ESMCI\_WebServDataDesc::getLatIndex() (Source File: ESMCI\_WebServDataDesc.C)}


  
\bigskip{\sf INTERFACE:}
\begin{verbatim} int  ESMCI_WebServDataDesc::getLatIndex(\end{verbatim}{\em RETURN VALUE:}
\begin{verbatim}     int  index of specified lat value in array of lat values; -1 if there's
          a problem.\end{verbatim}{\em ARGUMENTS:}
\begin{verbatim}   double  latValue      // the lat value to lookup in the array of lat values
   )\end{verbatim}
{\sf DESCRIPTION:\\ }


      Looks up the index of the specified lat value in the array of latitude
      values read from the CAM output file.
   
%/////////////////////////////////////////////////////////////
 
\mbox{}\hrulefill\
 
\subsubsection{ESMCI\_WebServDataDesc::getLonIndex() (Source File: ESMCI\_WebServDataDesc.C)}


  
\bigskip{\sf INTERFACE:}
\begin{verbatim} int  ESMCI_WebServDataDesc::getLonIndex(\end{verbatim}{\em RETURN VALUE:}
\begin{verbatim}     int  index of specified lon value in array of lon values; -1 if there's
          a problem.\end{verbatim}{\em ARGUMENTS:}
\begin{verbatim}   double  lonValue      // the lon value to lookup in the array of lon values
   )\end{verbatim}
{\sf DESCRIPTION:\\ }


      Looks up the index of the specified lon value in the array of longitude
      values read from the CAM output file.
   
%/////////////////////////////////////////////////////////////
 
\mbox{}\hrulefill\
 
\subsubsection{ESMCI\_WebServDataDesc::print() (Source File: ESMCI\_WebServDataDesc.C)}


  
\bigskip{\sf INTERFACE:}
\begin{verbatim} void  ESMCI_WebServDataDesc::print(\end{verbatim}{\em RETURN VALUE:}
\begin{verbatim} \end{verbatim}{\em ARGUMENTS:}
\begin{verbatim}   )\end{verbatim}
{\sf DESCRIPTION:\\ }


      Prints out the data values (including time, lat and lon) as they were
      read from the output file.
  
%...............................................................
\setlength{\parskip}{\oldparskip}
\setlength{\parindent}{\oldparindent}
\setlength{\baselineskip}{\oldbaselineskip}
