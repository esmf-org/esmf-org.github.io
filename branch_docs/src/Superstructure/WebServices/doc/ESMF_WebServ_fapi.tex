%                **** IMPORTANT NOTICE *****
% This LaTeX file has been automatically produced by ProTeX v. 1.1
% Any changes made to this file will likely be lost next time
% this file is regenerated from its source. Send questions 
% to Arlindo da Silva, dasilva@gsfc.nasa.gov
 
\setlength{\oldparskip}{\parskip}
\setlength{\parskip}{1.5ex}
\setlength{\oldparindent}{\parindent}
\setlength{\parindent}{0pt}
\setlength{\oldbaselineskip}{\baselineskip}
\setlength{\baselineskip}{11pt}
 
%--------------------- SHORT-HAND MACROS ----------------------
\def\bv{\begin{verbatim}}
\def\ev{\end{verbatim}}
\def\be{\begin{equation}}
\def\ee{\end{equation}}
\def\bea{\begin{eqnarray}}
\def\eea{\end{eqnarray}}
\def\bi{\begin{itemize}}
\def\ei{\end{itemize}}
\def\bn{\begin{enumerate}}
\def\en{\end{enumerate}}
\def\bd{\begin{description}}
\def\ed{\end{description}}
\def\({\left (}
\def\){\right )}
\def\[{\left [}
\def\]{\right ]}
\def\<{\left  \langle}
\def\>{\right \rangle}
\def\cI{{\cal I}}
\def\diag{\mathop{\rm diag}}
\def\tr{\mathop{\rm tr}}
%-------------------------------------------------------------

\markboth{Left}{Source File: ESMF\_WebServ.F90,  Date: Tue May  5 21:00:16 MDT 2020
}

 
%/////////////////////////////////////////////////////////////

  
%/////////////////////////////////////////////////////////////
 
\mbox{}\hrulefill\ 
 
\subsubsection [ESMF\_WebServicesLoop] {ESMF\_WebServicesLoop }


  
\bigskip{\sf INTERFACE:}
\begin{verbatim}   subroutine ESMF_WebServicesLoop(comp, portNum, clientId, registrarHost, rc)
 \end{verbatim}{\em ARGUMENTS:}
\begin{verbatim}     type(ESMF_GridComp)                         :: comp
     integer,            intent(inout), optional :: portNum
     character(len=*),   intent(in),    optional, target :: clientId
     character(len=*),   intent(in),    optional, target :: registrarHost
     integer,            intent(out),   optional :: rc\end{verbatim}
{\sf DESCRIPTION:\\ }


     Encapsulates all of the functionality necessary to setup a component as
     a component service.  On the root PET, it registers the
     component service and then enters into a loop that waits for requests on 
     a socket.  The loop continues until an "exit" request is received, at 
     which point it exits the loop and unregisters the service.  On
     any PET other than the root PET, it sets up a process block that waits
     for instructions from the root PET.  Instructions will come as requests
     are received from the socket.
  
   The arguments are:
   \begin{description}
   \item[{[comp]}]
     {\tt ESMF\_CplComp} object that represents the Grid Component for which
     routine is run.
   \item[{[portNum]}]
     Number of the port on which the component service is listening.
   \item[{[clientId]}]
     Identifier of the client responsible for this component service.  If a
     Process Controller application manages this component service, then the
     clientId is provided to the component service application in the command
     line.  Otherwise, the clientId is not necessary.
   \item[{[registrarHost]}]
     Name of the host on which the Registrar is running.  Needed so the
     component service can notify the Registrar when it is ready to receive
     requests from clients.
   \item[{[rc]}]
     Return code; equals {\tt ESMF\_SUCCESS} if there are no errors.
   \end{description}
   
%/////////////////////////////////////////////////////////////
 
\mbox{}\hrulefill\ 
 
\subsubsection [ESMF\_WebServicesCplCompLoop] {ESMF\_WebServicesCplCompLoop }


  
\bigskip{\sf INTERFACE:}
\begin{verbatim}   subroutine ESMF_WebServicesCplCompLoop(comp, portNum, clientId, registrarHost, rc)
 
 \end{verbatim}{\em ARGUMENTS:}
\begin{verbatim}     type(ESMF_CplComp)                         :: comp
     integer,           intent(inout), optional :: portNum
     character(len=*),  intent(in),    optional, target :: clientId
     character(len=*),  intent(in),    optional, target :: registrarHost
     integer,           intent(out),   optional :: rc\end{verbatim}
{\sf DESCRIPTION:\\ }


     Encapsulates all of the functionality necessary to setup a component as
     a component service.  On the root PET, it registers the
     component service and then enters into a loop that waits for requests on 
     a socket.  The loop continues until an "exit" request is received, at 
     which point it exits the loop and unregisters the service.  On
     any PET other than the root PET, it sets up a process block that waits
     for instructions from the root PET.  Instructions will come as requests
     are received from the socket.
  
   The arguments are:
   \begin{description}
   \item[{[comp]}]
     {\tt ESMF\_CplComp} object that represents the Grid Component for which
     routine is run.
   \item[{[portNum]}]
     Number of the port on which the component service is listening.
   \item[{[clientId]}]
     Identifier of the client responsible for this component service.  If a
     Process Controller application manages this component service, then the
     clientId is provided to the component service application in the command
     line.  Otherwise, the clientId is not necessary.
   \item[{[registrarHost]}]
     Name of the host on which the Registrar is running.  Needed so the
     component service can notify the Registrar when it is ready to receive
     requests from clients.
   \item[{[rc]}]
     Return code; equals {\tt ESMF\_SUCCESS} if there are no errors.
   \end{description}
  
%...............................................................
\setlength{\parskip}{\oldparskip}
\setlength{\parindent}{\oldparindent}
\setlength{\baselineskip}{\oldbaselineskip}
