%                **** IMPORTANT NOTICE *****
% This LaTeX file has been automatically produced by ProTeX v. 1.1
% Any changes made to this file will likely be lost next time
% this file is regenerated from its source. Send questions 
% to Arlindo da Silva, dasilva@gsfc.nasa.gov
 
\setlength{\oldparskip}{\parskip}
\setlength{\parskip}{1.5ex}
\setlength{\oldparindent}{\parindent}
\setlength{\parindent}{0pt}
\setlength{\oldbaselineskip}{\baselineskip}
\setlength{\baselineskip}{11pt}
 
%--------------------- SHORT-HAND MACROS ----------------------
\def\bv{\begin{verbatim}}
\def\ev{\end{verbatim}}
\def\be{\begin{equation}}
\def\ee{\end{equation}}
\def\bea{\begin{eqnarray}}
\def\eea{\end{eqnarray}}
\def\bi{\begin{itemize}}
\def\ei{\end{itemize}}
\def\bn{\begin{enumerate}}
\def\en{\end{enumerate}}
\def\bd{\begin{description}}
\def\ed{\end{description}}
\def\({\left (}
\def\){\right )}
\def\[{\left [}
\def\]{\right ]}
\def\<{\left  \langle}
\def\>{\right \rangle}
\def\cI{{\cal I}}
\def\diag{\mathop{\rm diag}}
\def\tr{\mathop{\rm tr}}
%-------------------------------------------------------------

\markboth{Left}{Source File: ESMCI\_WebServForkClient.C,  Date: Tue May  5 21:00:15 MDT 2020
}

 
%/////////////////////////////////////////////////////////////
\subsubsection{ESMCI\_WebServForkClient::ESMCI\_WebServForkClient() (Source File: ESMCI\_WebServForkClient.C)}


  
\bigskip{\sf INTERFACE:}
\begin{verbatim} ESMCI_WebServForkClient::ESMCI_WebServForkClient(\end{verbatim}{\em ARGUMENTS:}
\begin{verbatim}   string                hostName,
   string                scriptDir,
   string                scriptName
   )\end{verbatim}
{\sf DESCRIPTION:\\ }


      Instantiates a Component Service Manager that uses the system fork
      method for starting a component service.
   
%/////////////////////////////////////////////////////////////
 
\mbox{}\hrulefill\
 
\subsubsection{ESMCI\_WebServForkClient::~ESMCI\_WebServForkClient() (Source File: ESMCI\_WebServForkClient.C)}


  
\bigskip{\sf INTERFACE:}
\begin{verbatim} ESMCI_WebServForkClient::~ESMCI_WebServForkClient(\end{verbatim}{\em ARGUMENTS:}
\begin{verbatim}   )\end{verbatim}
{\sf DESCRIPTION:\\ }


      Cleans up memory allocated for CAM output file.
   
%/////////////////////////////////////////////////////////////
 
\mbox{}\hrulefill\
 
\subsubsection{ESMCI\_WebServForkClient::submitJob() (Source File: ESMCI\_WebServForkClient.C)}


  
\bigskip{\sf INTERFACE:}
\begin{verbatim} string  ESMCI_WebServForkClient::submitJob(\end{verbatim}{\em ARGUMENTS:}
\begin{verbatim}   int     clientId,
   string  registrarHost,
   int     portNum
   )\end{verbatim}
{\sf DESCRIPTION:\\ }


      Submits the job specified by the argument to the job scheduler.
   
%/////////////////////////////////////////////////////////////
 
\mbox{}\hrulefill\
 
\subsubsection{ESMCI\_WebServForkClient::cancelJob() (Source File: ESMCI\_WebServForkClient.C)}


  
\bigskip{\sf INTERFACE:}
\begin{verbatim} int  ESMCI_WebServForkClient::cancelJob(\end{verbatim}{\em ARGUMENTS:}
\begin{verbatim}   string  jobId
   )\end{verbatim}
{\sf DESCRIPTION:\\ }


      Cancels the job specified by the job id.
   
%/////////////////////////////////////////////////////////////
 
\mbox{}\hrulefill\
 
\subsubsection{ESMCI\_WebServForkClient::jobStatus() (Source File: ESMCI\_WebServForkClient.C)}


  
\bigskip{\sf INTERFACE:}
\begin{verbatim} int  ESMCI_WebServForkClient::jobStatus(\end{verbatim}{\em ARGUMENTS:}
\begin{verbatim}   string  jobId
   )\end{verbatim}
{\sf DESCRIPTION:\\ }


      Returns the status of the job specified by the job id.
   
%/////////////////////////////////////////////////////////////
 
\mbox{}\hrulefill\
 
\subsubsection{ESMCI\_WebServForkClient::extractPid() (Source File: ESMCI\_WebServForkClient.C)}


  
\bigskip{\sf INTERFACE:}
\begin{verbatim} int  ESMCI_WebServForkClient::extractPid(\end{verbatim}{\em ARGUMENTS:}
\begin{verbatim}   string  jobId
   )\end{verbatim}
{\sf DESCRIPTION:\\ }


      Returns the process id for the specified job id.
  
%...............................................................
\setlength{\parskip}{\oldparskip}
\setlength{\parindent}{\oldparindent}
\setlength{\baselineskip}{\oldbaselineskip}
