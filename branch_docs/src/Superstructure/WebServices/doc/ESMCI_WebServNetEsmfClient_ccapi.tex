%                **** IMPORTANT NOTICE *****
% This LaTeX file has been automatically produced by ProTeX v. 1.1
% Any changes made to this file will likely be lost next time
% this file is regenerated from its source. Send questions 
% to Arlindo da Silva, dasilva@gsfc.nasa.gov
 
\setlength{\oldparskip}{\parskip}
\setlength{\parskip}{1.5ex}
\setlength{\oldparindent}{\parindent}
\setlength{\parindent}{0pt}
\setlength{\oldbaselineskip}{\baselineskip}
\setlength{\baselineskip}{11pt}
 
%--------------------- SHORT-HAND MACROS ----------------------
\def\bv{\begin{verbatim}}
\def\ev{\end{verbatim}}
\def\be{\begin{equation}}
\def\ee{\end{equation}}
\def\bea{\begin{eqnarray}}
\def\eea{\end{eqnarray}}
\def\bi{\begin{itemize}}
\def\ei{\end{itemize}}
\def\bn{\begin{enumerate}}
\def\en{\end{enumerate}}
\def\bd{\begin{description}}
\def\ed{\end{description}}
\def\({\left (}
\def\){\right )}
\def\[{\left [}
\def\]{\right ]}
\def\<{\left  \langle}
\def\>{\right \rangle}
\def\cI{{\cal I}}
\def\diag{\mathop{\rm diag}}
\def\tr{\mathop{\rm tr}}
%-------------------------------------------------------------

\markboth{Left}{Source File: ESMCI\_WebServNetEsmfClient.C,  Date: Tue May  5 21:00:15 MDT 2020
}

 
%/////////////////////////////////////////////////////////////
\subsubsection{ESMCI\_WebServNetEsmfClient::ESMCI\_WebServNetEsmfClient() (Source File: ESMCI\_WebServNetEsmfClient.C)}


  
\bigskip{\sf INTERFACE:}
\begin{verbatim} ESMCI_WebServNetEsmfClient::ESMCI_WebServNetEsmfClient(\end{verbatim}{\em ARGUMENTS:}
\begin{verbatim}   const char*  host,   // (in) the name of the host machine running the
                        // component service
   int          port    // (in) the port number of the component service
                        // to which this client will connect
   )\end{verbatim}
{\sf DESCRIPTION:\\ }


      Initialize the ESMF Component client with the name of the host and port
      where the component service is running.
   
%/////////////////////////////////////////////////////////////
 
\mbox{}\hrulefill\
 
\subsubsection{ESMCI\_WebServNetEsmfClient::~ESMCI\_WebServNetEsmfClient() (Source File: ESMCI\_WebServNetEsmfClient.C)}


  
\bigskip{\sf INTERFACE:}
\begin{verbatim} ESMCI_WebServNetEsmfClient::~ESMCI_WebServNetEsmfClient(\end{verbatim}{\em ARGUMENTS:}
\begin{verbatim}   )\end{verbatim}
{\sf DESCRIPTION:\\ }


      Cleans up the ESMF Component client by disconnecting from the service.
   
%/////////////////////////////////////////////////////////////
 
\mbox{}\hrulefill\
 
\subsubsection{ESMCI\_WebServNetEsmfClient::setHost() (Source File: ESMCI\_WebServNetEsmfClient.C)}


  
\bigskip{\sf INTERFACE:}
\begin{verbatim} void  ESMCI_WebServNetEsmfClient::setHost(\end{verbatim}{\em RETURN VALUE:}
\begin{verbatim} \end{verbatim}{\em ARGUMENTS:}
\begin{verbatim}   const char*  host    // (in) the name of the host machine running the
                        // component service
   )\end{verbatim}
{\sf DESCRIPTION:\\ }


      Sets the name of the component service host machine.
   
%/////////////////////////////////////////////////////////////
 
\mbox{}\hrulefill\
 
\subsubsection{ESMCI\_WebServNetEsmfClient::setPort() (Source File: ESMCI\_WebServNetEsmfClient.C)}


  
\bigskip{\sf INTERFACE:}
\begin{verbatim} void  ESMCI_WebServNetEsmfClient::setPort(\end{verbatim}{\em RETURN VALUE:}
\begin{verbatim} \end{verbatim}{\em ARGUMENTS:}
\begin{verbatim}   int          port    // (in) the port number of the component service
                        // to which this client will connect
   )\end{verbatim}
{\sf DESCRIPTION:\\ }


      Sets the component service port number.
   
%/////////////////////////////////////////////////////////////
 
\mbox{}\hrulefill\
 
\subsubsection{ESMCI\_WebServNetEsmfClient::sendRequest() (Source File: ESMCI\_WebServNetEsmfClient.C)}


  
\bigskip{\sf INTERFACE:}
\begin{verbatim} int  ESMCI_WebServNetEsmfClient::sendRequest(\end{verbatim}{\em RETURN VALUE:}
\begin{verbatim}      int  number of bytes written to the socket (in addition to the request
           msg); ESMF_FAILURE if there is an error.\end{verbatim}{\em ARGUMENTS:}
\begin{verbatim}   int    request,               // (in) the request identifier
   int    length,                // (in) the length of the data to send
   void*  data                   // (in) the buffer containing the data to send
   )\end{verbatim}
{\sf DESCRIPTION:\\ }


      Sends a request to the component service.  First, it sends the request
      identifier.  Then, it sends any ancillary data.
   
%/////////////////////////////////////////////////////////////
 
\mbox{}\hrulefill\
 
\subsubsection{ESMCI\_WebServNetEsmfClient::sendData() (Source File: ESMCI\_WebServNetEsmfClient.C)}


  
\bigskip{\sf INTERFACE:}
\begin{verbatim} int  ESMCI_WebServNetEsmfClient::sendData(\end{verbatim}{\em RETURN VALUE:}
\begin{verbatim}      int  number of bytes written to the socket; ESMF_FAILURE if there is
           an error.\end{verbatim}{\em ARGUMENTS:}
\begin{verbatim}   int    length,                // (in) the length of the data to send
   void*  data                   // (in) the buffer containing the data to send
   )\end{verbatim}
{\sf DESCRIPTION:\\ }


      Sends a packet of data to the service.
   
%/////////////////////////////////////////////////////////////
 
\mbox{}\hrulefill\
 
\subsubsection{ESMCI\_WebServNetEsmfClient::sendString() (Source File: ESMCI\_WebServNetEsmfClient.C)}


  
\bigskip{\sf INTERFACE:}
\begin{verbatim} int  ESMCI_WebServNetEsmfClient::sendString(\end{verbatim}{\em RETURN VALUE:}
\begin{verbatim}      int  number of bytes written to the socket; ESMF_FAILURE if there is
           an error.\end{verbatim}{\em ARGUMENTS:}
\begin{verbatim}   const char*  data             // (in) the string containing the data to send
   )\end{verbatim}
{\sf DESCRIPTION:\\ }


      Sends a string to the service.
   
%/////////////////////////////////////////////////////////////
 
\mbox{}\hrulefill\
 
\subsubsection{ESMCI\_WebServNetEsmfClient::getResponse() (Source File: ESMCI\_WebServNetEsmfClient.C)}


  
\bigskip{\sf INTERFACE:}
\begin{verbatim} int  ESMCI_WebServNetEsmfClient::getResponse(\end{verbatim}{\em RETURN VALUE:}
\begin{verbatim}      int  number of bytes read from the socket; ESMF_FAILURE if there is
           and error.\end{verbatim}{\em ARGUMENTS:}
\begin{verbatim}   int    request,               // (in) the request identifier (ignored)
   int&   length,                // (out) the length of the data placed in the buffer
   void*  data                   // (out) the buffer containing the received data
   )\end{verbatim}
{\sf DESCRIPTION:\\ }


      Reads a request response from the component service.
   
%/////////////////////////////////////////////////////////////
 
\mbox{}\hrulefill\
 
\subsubsection{ESMCI\_WebServNetEsmfClient::connect() (Source File: ESMCI\_WebServNetEsmfClient.C)}


  
\bigskip{\sf INTERFACE:}
\begin{verbatim} int  ESMCI_WebServNetEsmfClient::connect(\end{verbatim}{\em RETURN VALUE:}
\begin{verbatim}     int  socket file descriptor if successful, ESMF_FAILURE otherwise.\end{verbatim}{\em ARGUMENTS:}
\begin{verbatim}   )\end{verbatim}
{\sf DESCRIPTION:\\ }


      Connects to the component service identified by the host name and port
      number data members.
   
%/////////////////////////////////////////////////////////////
 
\mbox{}\hrulefill\
 
\subsubsection{ESMCI\_WebServNetEsmfClient::disconnect() (Source File: ESMCI\_WebServNetEsmfClient.C)}


  
\bigskip{\sf INTERFACE:}
\begin{verbatim} void  ESMCI_WebServNetEsmfClient::disconnect(\end{verbatim}{\em RETURN VALUE:}
\begin{verbatim} \end{verbatim}{\em ARGUMENTS:}
\begin{verbatim}   )\end{verbatim}
{\sf DESCRIPTION:\\ }


      Disconnects from the component service.
   
%/////////////////////////////////////////////////////////////
 
\mbox{}\hrulefill\
 
\subsubsection{ESMCI\_WebServNetEsmfClient::getStateStr() (Source File: ESMCI\_WebServNetEsmfClient.C)}


  
\bigskip{\sf INTERFACE:}
\begin{verbatim} char*  ESMCI_WebServNetEsmfClient::getStateStr(\end{verbatim}{\em RETURN VALUE:}
\begin{verbatim}      char*  string value for the specified request id\end{verbatim}{\em ARGUMENTS:}
\begin{verbatim}   int  state      // request id for which the string value is to be returned
   )\end{verbatim}
{\sf DESCRIPTION:\\ }


      Looks up a state string value based on a specified state value.
   
%/////////////////////////////////////////////////////////////
 
\mbox{}\hrulefill\
 
\subsubsection{ESMCI\_WebServNetEsmfClient::getStateValue() (Source File: ESMCI\_WebServNetEsmfClient.C)}


  
\bigskip{\sf INTERFACE:}
\begin{verbatim} int  ESMCI_WebServNetEsmfClient::getStateValue(\end{verbatim}{\em RETURN VALUE:}
\begin{verbatim}      int  net esmf state based on the specified string; ESMF_FAILURE
           if the id cannot be found\end{verbatim}{\em ARGUMENTS:}
\begin{verbatim}   const char*  stateStr // request string for which the id is to be returned
   )\end{verbatim}
{\sf DESCRIPTION:\\ }


      Looks up a state value id based on a specified string value.
  
%...............................................................
\setlength{\parskip}{\oldparskip}
\setlength{\parindent}{\oldparindent}
\setlength{\baselineskip}{\oldbaselineskip}
