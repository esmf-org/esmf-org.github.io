%                **** IMPORTANT NOTICE *****
% This LaTeX file has been automatically produced by ProTeX v. 1.1
% Any changes made to this file will likely be lost next time
% this file is regenerated from its source. Send questions 
% to Arlindo da Silva, dasilva@gsfc.nasa.gov
 
\setlength{\oldparskip}{\parskip}
\setlength{\parskip}{1.5ex}
\setlength{\oldparindent}{\parindent}
\setlength{\parindent}{0pt}
\setlength{\oldbaselineskip}{\baselineskip}
\setlength{\baselineskip}{11pt}
 
%--------------------- SHORT-HAND MACROS ----------------------
\def\bv{\begin{verbatim}}
\def\ev{\end{verbatim}}
\def\be{\begin{equation}}
\def\ee{\end{equation}}
\def\bea{\begin{eqnarray}}
\def\eea{\end{eqnarray}}
\def\bi{\begin{itemize}}
\def\ei{\end{itemize}}
\def\bn{\begin{enumerate}}
\def\en{\end{enumerate}}
\def\bd{\begin{description}}
\def\ed{\end{description}}
\def\({\left (}
\def\){\right )}
\def\[{\left [}
\def\]{\right ]}
\def\<{\left  \langle}
\def\>{\right \rangle}
\def\cI{{\cal I}}
\def\diag{\mathop{\rm diag}}
\def\tr{\mathop{\rm tr}}
%-------------------------------------------------------------

\markboth{Left}{Source File: ESMCI\_WebServSocketUtils.C,  Date: Tue May  5 21:00:15 MDT 2020
}

 
%/////////////////////////////////////////////////////////////
\subsubsection{ESMCI\_WebServGetRequestFromId() (Source File: ESMCI\_WebServSocketUtils.C)}


  
\bigskip{\sf INTERFACE:}
\begin{verbatim} char*  ESMCI_WebServGetRequestFromId(\end{verbatim}{\em RETURN VALUE:}
\begin{verbatim}      char*  string value for the specified request id\end{verbatim}{\em ARGUMENTS:}
\begin{verbatim}   int  id      // request id for which the string value is to be returned
   )\end{verbatim}
{\sf DESCRIPTION:\\ }


      Looks up a request string value based on a specified request id.
   
%/////////////////////////////////////////////////////////////
 
\mbox{}\hrulefill\
 
\subsubsection{ESMCI\_WebServGetRequestId() (Source File: ESMCI\_WebServSocketUtils.C)}


  
\bigskip{\sf INTERFACE:}
\begin{verbatim} int  ESMCI_WebServGetRequestId(\end{verbatim}{\em RETURN VALUE:}
\begin{verbatim}      int  id of the request based on the specified string; ESMF_FAILURE
           if the id cannot be found\end{verbatim}{\em ARGUMENTS:}
\begin{verbatim}   const char  request[] // request string for which the id is to be returned
   )\end{verbatim}
{\sf DESCRIPTION:\\ }


      Looks up a request id based on a specified string value.
   
%/////////////////////////////////////////////////////////////
 
\mbox{}\hrulefill\
 
\subsubsection{ESMCI\_WebServNotify() (Source File: ESMCI\_WebServSocketUtils.C)}


  
\bigskip{\sf INTERFACE:}
\begin{verbatim} void  ESMCI_WebServNotify(\end{verbatim}{\em RETURN VALUE:}
\begin{verbatim} \end{verbatim}{\em ARGUMENTS:}
\begin{verbatim}   const char  msg[],                                    // message to print to stderr
   WebServSeverity    severity = WebServPRINT,   // level of severity
   const char  proc[] = NULL             // method/function/procedure name
   )\end{verbatim}
{\sf DESCRIPTION:\\ }


      Formats and prints the specified message to stderr.  If the severity
      is set to "FATAL", then this function will also exit the application.
   
%/////////////////////////////////////////////////////////////
 
\mbox{}\hrulefill\
 
\subsubsection{ESMCI\_WebServSend() (Source File: ESMCI\_WebServSocketUtils.C)}


  
\bigskip{\sf INTERFACE:}
\begin{verbatim} int  ESMCI_WebServSend(\end{verbatim}{\em RETURN VALUE:}
\begin{verbatim}     int  the total number of bytes written to the socket\end{verbatim}{\em ARGUMENTS:}
\begin{verbatim}   int    fd,            // (in) the file descriptor for the socket
   int    size,          // (in) the amount of data to send across the socket
   void*  data           // (in) the data to send
   )\end{verbatim}
{\sf DESCRIPTION:\\ }


      Sends the specified amount of specified data across the specified
      socket.
   
%/////////////////////////////////////////////////////////////
 
\mbox{}\hrulefill\
 
\subsubsection{ESMCI\_WebServRecv() (Source File: ESMCI\_WebServSocketUtils.C)}


  
\bigskip{\sf INTERFACE:}
\begin{verbatim} int  ESMCI_WebServRecv(\end{verbatim}{\em RETURN VALUE:}
\begin{verbatim}     int  the total number of bytes read from the socket\end{verbatim}{\em ARGUMENTS:}
\begin{verbatim}   int    fd,            // (in) the file descriptor for the socket
   int    size,          // (in) the amount of data to read from the socket
   void*  data           // (inout) the data buffer where the read data gets put.
                         // The memory for this buffer should be allocated by the
                         // calling function with enough space to handle the
                         // specified size.
   )\end{verbatim}
{\sf DESCRIPTION:\\ }


      Reads the specified amount of data from the specified socket and
      puts the data into the specified data buffer.
   
%/////////////////////////////////////////////////////////////
 
\mbox{}\hrulefill\
 
\subsubsection{ESMCI\_WebServRecv() (Source File: ESMCI\_WebServSocketUtils.C)}


  
\bigskip{\sf INTERFACE:}
\begin{verbatim} int  ESMCI_WebServRecv(\end{verbatim}{\em RETURN VALUE:}
\begin{verbatim}     int  the total number of bytes read from the socket\end{verbatim}{\em ARGUMENTS:}
\begin{verbatim}   int    fd,            // (in) the file descriptor for the socket
   const char*  s  // (inout) the data buffer where the read data gets put.
                         // The memory for this buffer should be allocated by the
                         // calling function with enough space to handle the data.
   )\end{verbatim}
{\sf DESCRIPTION:\\ }


      Reads all of the available data from the socket and puts the data
      into the specified data buffer.
  
%...............................................................
\setlength{\parskip}{\oldparskip}
\setlength{\parindent}{\oldparindent}
\setlength{\baselineskip}{\oldbaselineskip}
