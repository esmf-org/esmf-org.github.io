%                **** IMPORTANT NOTICE *****
% This LaTeX file has been automatically produced by ProTeX v. 1.1
% Any changes made to this file will likely be lost next time
% this file is regenerated from its source. Send questions 
% to Arlindo da Silva, dasilva@gsfc.nasa.gov
 
\setlength{\oldparskip}{\parskip}
\setlength{\parskip}{1.5ex}
\setlength{\oldparindent}{\parindent}
\setlength{\parindent}{0pt}
\setlength{\oldbaselineskip}{\baselineskip}
\setlength{\baselineskip}{11pt}
 
%--------------------- SHORT-HAND MACROS ----------------------
\def\bv{\begin{verbatim}}
\def\ev{\end{verbatim}}
\def\be{\begin{equation}}
\def\ee{\end{equation}}
\def\bea{\begin{eqnarray}}
\def\eea{\end{eqnarray}}
\def\bi{\begin{itemize}}
\def\ei{\end{itemize}}
\def\bn{\begin{enumerate}}
\def\en{\end{enumerate}}
\def\bd{\begin{description}}
\def\ed{\end{description}}
\def\({\left (}
\def\){\right )}
\def\[{\left [}
\def\]{\right ]}
\def\<{\left  \langle}
\def\>{\right \rangle}
\def\cI{{\cal I}}
\def\diag{\mathop{\rm diag}}
\def\tr{\mathop{\rm tr}}
%-------------------------------------------------------------

\markboth{Left}{Source File: ESMCI\_WebServRegistrarClient.C,  Date: Tue May  5 21:00:15 MDT 2020
}

 
%/////////////////////////////////////////////////////////////
\subsubsection{ESMCI\_WebServRegistrarClient::ESMCI\_WebServRegistrarClient() (Source File: ESMCI\_WebServRegistrarClient.C)}


  
\bigskip{\sf INTERFACE:}
\begin{verbatim} ESMCI_WebServRegistrarClient::ESMCI_WebServRegistrarClient(\end{verbatim}{\em ARGUMENTS:}
\begin{verbatim}   const char*  host,   // (in) the name of the host machine running the
                        // component service
   int          port    // (in) the port number of the component service
                        // to which this client will connect
   )\end{verbatim}
{\sf DESCRIPTION:\\ }


      Initialize the Registrar client with the name of the host and port
      where the Registrar service is running.
   
%/////////////////////////////////////////////////////////////
 
\mbox{}\hrulefill\
 
\subsubsection{ESMCI\_WebServRegistrarClient::~ESMCI\_WebServRegistrarClient() (Source File: ESMCI\_WebServRegistrarClient.C)}


  
\bigskip{\sf INTERFACE:}
\begin{verbatim} ESMCI_WebServRegistrarClient::~ESMCI_WebServRegistrarClient(\end{verbatim}{\em ARGUMENTS:}
\begin{verbatim}   )\end{verbatim}
{\sf DESCRIPTION:\\ }


      Cleans up the Registrar client by disconnecting from the service.
   
%/////////////////////////////////////////////////////////////
 
\mbox{}\hrulefill\
 
\subsubsection{ESMCI\_WebServRegistrarClient::registerComp() (Source File: ESMCI\_WebServRegistrarClient.C)}


  
\bigskip{\sf INTERFACE:}
\begin{verbatim} int  ESMCI_WebServRegistrarClient::registerComp(\end{verbatim}{\em RETURN VALUE:}
\begin{verbatim}      int  current state of the component service on the Registrar;
           ESMF_FAILURE if an error occurs\end{verbatim}{\em ARGUMENTS:}
\begin{verbatim}   const char*  clientId,        // (in) the id for the client who created the svc
   const char*  hostName,        // (in) the host name of the component svc scheduler
   const char*  portNum          // (in) the port number of the component svc
   )\end{verbatim}
{\sf DESCRIPTION:\\ }


      Registers the component service identified by the input parameters
      with the Registrar.
   
%/////////////////////////////////////////////////////////////
 
\mbox{}\hrulefill\
 
\subsubsection{ESMCI\_WebServRegistrarClient::compSubmitted() (Source File: ESMCI\_WebServRegistrarClient.C)}


  
\bigskip{\sf INTERFACE:}
\begin{verbatim} int  ESMCI_WebServRegistrarClient::compSubmitted(\end{verbatim}{\em RETURN VALUE:}
\begin{verbatim}      int  number of bytes read from the socket;
           ESMF_FAILURE if an error occurs\end{verbatim}{\em ARGUMENTS:}
\begin{verbatim}   const char*  clientId,        // (in) the id for the client who created the service
   const char*  jobId               // (in) the id of the job associated with the
                            //      scheduling of the service
   )\end{verbatim}
{\sf DESCRIPTION:\\ }


      Notifies the Registrar that the process controller has submitted the
      job to the job scheduler and sets the job id returned from the job
      scheduler.
   
%/////////////////////////////////////////////////////////////
 
\mbox{}\hrulefill\
 
\subsubsection{ESMCI\_WebServRegistrarClient::compStarted() (Source File: ESMCI\_WebServRegistrarClient.C)}


  
\bigskip{\sf INTERFACE:}
\begin{verbatim} int  ESMCI_WebServRegistrarClient::compStarted(\end{verbatim}{\em RETURN VALUE:}
\begin{verbatim}      int  number of bytes read from the socket;
           ESMF_FAILURE if an error occurs\end{verbatim}{\em ARGUMENTS:}
\begin{verbatim}   const char*  clientId,                // (in) the id of the client who created the svc
   const char*  compName,                // (in) the name of the component service
   const char*  compDesc,                // (in) the description of the component service
   const char*  physHostName     // (in) the name of actual host on which the
                               //      component service is running
   )\end{verbatim}
{\sf DESCRIPTION:\\ }


      Notifies the Registrar that the component service has successfully
      been started.
   
%/////////////////////////////////////////////////////////////
 
\mbox{}\hrulefill\
 
\subsubsection{ESMCI\_WebServRegistrarClient::getComponent() (Source File: ESMCI\_WebServRegistrarClient.C)}


  
\bigskip{\sf INTERFACE:}
\begin{verbatim} int  ESMCI_WebServRegistrarClient::getComponent(\end{verbatim}{\em RETURN VALUE:}
\begin{verbatim}      int  number of bytes read from the socket;
           ESMF_FAILURE if an error occurs\end{verbatim}{\em ARGUMENTS:}
\begin{verbatim}   const char*               clientId,      // (in) the id for the client who
                                           //    created the service
   ESMCI_WebServCompSvrInfo*  compSvrInfo  // (out) structure into which the
                                           //    server information is put
   )\end{verbatim}
{\sf DESCRIPTION:\\ }


      Notifies the Registrar that the component service has successfully
      been started.
   
%/////////////////////////////////////////////////////////////
 
\mbox{}\hrulefill\
 
\subsubsection{ESMCI\_WebServRegistrarClient::getStatus() (Source File: ESMCI\_WebServRegistrarClient.C)}


  
\bigskip{\sf INTERFACE:}
\begin{verbatim} int  ESMCI_WebServRegistrarClient::getStatus(\end{verbatim}{\em RETURN VALUE:}
\begin{verbatim}      int  number of bytes read from the socket;
           ESMF_FAILURE if an error occurs\end{verbatim}{\em ARGUMENTS:}
\begin{verbatim}   const char*  clientId                 // (in) the id for the client who created the service
   )\end{verbatim}
{\sf DESCRIPTION:\\ }


      Notifies the Registrar that the component service has successfully
      been started.
   
%/////////////////////////////////////////////////////////////
 
\mbox{}\hrulefill\
 
\subsubsection{ESMCI\_WebServRegistrarClient::setStatus() (Source File: ESMCI\_WebServRegistrarClient.C)}


  
\bigskip{\sf INTERFACE:}
\begin{verbatim} int  ESMCI_WebServRegistrarClient::setStatus(\end{verbatim}{\em RETURN VALUE:}
\begin{verbatim}      int  number of bytes read from the socket.
           ESMF_FAILURE if an error occurs\end{verbatim}{\em ARGUMENTS:}
\begin{verbatim}   const char*  clientId,   // (in) the id of the client whose component svc is
                               //      to be unregistered
   const char*  status           // (in) the status to be set
   )\end{verbatim}
{\sf DESCRIPTION:\\ }


      Sets the status of the component server in the Registrar.
   
%/////////////////////////////////////////////////////////////
 
\mbox{}\hrulefill\
 
\subsubsection{ESMCI\_WebServRegistrarClient::unregisterComp() (Source File: ESMCI\_WebServRegistrarClient.C)}


  
\bigskip{\sf INTERFACE:}
\begin{verbatim} int  ESMCI_WebServRegistrarClient::unregisterComp(\end{verbatim}{\em RETURN VALUE:}
\begin{verbatim}      int  number of bytes read from the socket.
           ESMF_FAILURE if an error occurs\end{verbatim}{\em ARGUMENTS:}
\begin{verbatim}   const char*  clientId          // (in) the id of the client whose component service is
                    //      to be unregistered
   )\end{verbatim}
{\sf DESCRIPTION:\\ }


      Unregisters the component service identified by the input parameters
      from the Registrar.
  
%...............................................................
\setlength{\parskip}{\oldparskip}
\setlength{\parindent}{\oldparindent}
\setlength{\baselineskip}{\oldbaselineskip}
