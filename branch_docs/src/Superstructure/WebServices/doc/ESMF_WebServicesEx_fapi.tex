%                **** IMPORTANT NOTICE *****
% This LaTeX file has been automatically produced by ProTeX v. 1.1
% Any changes made to this file will likely be lost next time
% this file is regenerated from its source. Send questions 
% to Arlindo da Silva, dasilva@gsfc.nasa.gov
 
\setlength{\oldparskip}{\parskip}
\setlength{\parskip}{1.5ex}
\setlength{\oldparindent}{\parindent}
\setlength{\parindent}{0pt}
\setlength{\oldbaselineskip}{\baselineskip}
\setlength{\baselineskip}{11pt}
 
%--------------------- SHORT-HAND MACROS ----------------------
\def\bv{\begin{verbatim}}
\def\ev{\end{verbatim}}
\def\be{\begin{equation}}
\def\ee{\end{equation}}
\def\bea{\begin{eqnarray}}
\def\eea{\end{eqnarray}}
\def\bi{\begin{itemize}}
\def\ei{\end{itemize}}
\def\bn{\begin{enumerate}}
\def\en{\end{enumerate}}
\def\bd{\begin{description}}
\def\ed{\end{description}}
\def\({\left (}
\def\){\right )}
\def\[{\left [}
\def\]{\right ]}
\def\<{\left  \langle}
\def\>{\right \rangle}
\def\cI{{\cal I}}
\def\diag{\mathop{\rm diag}}
\def\tr{\mathop{\rm tr}}
%-------------------------------------------------------------

\markboth{Left}{Source File: ESMF\_WebServicesEx.F90,  Date: Tue May  5 21:00:16 MDT 2020
}

 
%/////////////////////////////////////////////////////////////

  \subsubsection{Making a Component available through ESMF Web Services}
        
    In this example, a standard ESMF Component is made available through
    the Web Services interface. 
%/////////////////////////////////////////////////////////////

    The first step is to make sure your callback routines for initialize, run
    and finalize are setup.  This is done by creating a register routine that
    sets the entry points for each of these callbacks.  In this example, we've
    packaged it all up into a separate module.   
%/////////////////////////////////////////////////////////////

 \begin{verbatim}
module ESMF_WebServUserModel

  ! ESMF Framework module
  use ESMF

  implicit none

  public ESMF_WebServUserModelRegister

  contains

  !-------------------------------------------------------------------------
  !  The Registration routine
  !  
  subroutine ESMF_WebServUserModelRegister(comp, rc)
    type(ESMF_GridComp)  :: comp
    integer, intent(out) :: rc

    ! Initialize return code
    rc = ESMF_SUCCESS

    print *, "User Comp1 Register starting"

    ! Register the callback routines.

    call ESMF_GridCompSetEntryPoint(comp, ESMF_METHOD_INITIALIZE, &
                                    userRoutine=user_init, rc=rc)
    if (rc/=ESMF_SUCCESS) return ! bail out

    call ESMF_GridCompSetEntryPoint(comp, ESMF_METHOD_RUN, &
                                    userRoutine=user_run, rc=rc)
    if (rc/=ESMF_SUCCESS) return ! bail out

    call ESMF_GridCompSetEntryPoint(comp, ESMF_METHOD_FINALIZE, &
                                    userRoutine=user_final, rc=rc)
    if (rc/=ESMF_SUCCESS) return ! bail out

    print *, "Registered Initialize, Run, and Finalize routines"
    print *, "User Comp1 Register returning"

  end subroutine

  !-------------------------------------------------------------------------
  !  The Initialization routine
  !  
  subroutine user_init(comp, importState, exportState, clock, rc)
    type(ESMF_GridComp)  :: comp
    type(ESMF_State)     :: importState, exportState
    type(ESMF_Clock)     :: clock
    integer, intent(out) :: rc

    ! Initialize return code
    rc = ESMF_SUCCESS

    print *, "User Comp1 Init"

  end subroutine user_init

  !-------------------------------------------------------------------------
  !  The Run routine
  !  
  subroutine user_run(comp, importState, exportState, clock, rc)
    type(ESMF_GridComp)  :: comp
    type(ESMF_State)     :: importState, exportState
    type(ESMF_Clock)     :: clock
    integer, intent(out) :: rc

    ! Initialize return code
    rc = ESMF_SUCCESS

    print *, "User Comp1 Run"

  end subroutine user_run

  !-------------------------------------------------------------------------
  !  The Finalization routine
  !  
  subroutine user_final(comp, importState, exportState, clock, rc)
    type(ESMF_GridComp)  :: comp
    type(ESMF_State)     :: importState, exportState
    type(ESMF_Clock)     :: clock
    integer, intent(out) :: rc

    ! Initialize return code
    rc = ESMF_SUCCESS

    print *, "User Comp1 Final"

  end subroutine user_final

end module ESMF_WebServUserModel
 
\end{verbatim}
 
%/////////////////////////////////////////////////////////////

    The actual driver code then becomes very simple; ESMF is initialized,
    the component is created, the callback functions for the component are
    registered, and the Web Service loop is started. 
%/////////////////////////////////////////////////////////////

 \begin{verbatim}
program WebServicesEx
#include "ESMF.h"

  ! ESMF Framework module
  use ESMF
  use ESMF_TestMod

  use ESMF_WebServMod
  use ESMF_WebServUserModel

  implicit none

  ! Local variables
  type(ESMF_GridComp) :: comp1     !! Grid Component
  integer             :: rc        !! Return Code
  integer             :: finalrc  !! Final return code
  integer             :: portNum   !! The port number for the listening socket
 
\end{verbatim}
 
%/////////////////////////////////////////////////////////////

    A listening socket will be created on the local machine with the specified
    port number.  This socket is used by the service to
    wait for and receive requests from the client.  Check with your system
    administrator to determine an appropriate port to use for your service. 
%/////////////////////////////////////////////////////////////

 \begin{verbatim}
  finalrc = ESMF_SUCCESS

  call ESMF_Initialize(defaultlogfilename="WebServicesEx.Log", &
                    logkindflag=ESMF_LOGKIND_MULTI, rc=rc)
 
\end{verbatim}
 
%/////////////////////////////////////////////////////////////

 \begin{verbatim}
  ! create the grid component 
  comp1 = ESMF_GridCompCreate(name="My Component", rc=rc)
 
\end{verbatim}
 
%/////////////////////////////////////////////////////////////

 \begin{verbatim}
  ! Set up the register routine 
  call ESMF_GridCompSetServices(comp1, &
          userRoutine=ESMF_WebServUserModelRegister, rc=rc)
 
\end{verbatim}
 
%/////////////////////////////////////////////////////////////

 \begin{verbatim}
  portNum = 27060

  ! Call the Web Services Loop and wait for requests to come in
  !call ESMF_WebServicesLoop(comp1, portNum, rc=rc)
 
\end{verbatim}
 
%/////////////////////////////////////////////////////////////

    The call to ESMF\_WebServicesLoop will setup the listening socket for your
    service and will wait for requests from a client.  As requests are received,
    the Web Services software will process the requests and then return to the
    loop to continue to wait. 
%/////////////////////////////////////////////////////////////

    The 3 main requests processed are INIT, RUN, and FINAL.  These requests 
    will then call the appropriate callback routine as specified in your 
    register routine (as specified in the ESMF\_GridCompSetServices call).
    In this example, when the INIT request is received, the user\_init routine
    found in the ESMF\_WebServUserModel module is called. 
%/////////////////////////////////////////////////////////////

    One other request is also processed by the Component Service, and that is
    the EXIT request.  When this request is received, the Web Services loop
    is terminated and the remainder of the code after the ESMF\_WebServicesLoop
    call is executed. 
%/////////////////////////////////////////////////////////////

 \begin{verbatim}
  call ESMF_Finalize(rc=rc)
 
\end{verbatim}
 
%/////////////////////////////////////////////////////////////

 \begin{verbatim}
end program WebServicesEx
 
\end{verbatim}

%...............................................................
\setlength{\parskip}{\oldparskip}
\setlength{\parindent}{\oldparindent}
\setlength{\baselineskip}{\oldbaselineskip}
