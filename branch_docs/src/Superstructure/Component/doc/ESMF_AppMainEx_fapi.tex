%                **** IMPORTANT NOTICE *****
% This LaTeX file has been automatically produced by ProTeX v. 1.1
% Any changes made to this file will likely be lost next time
% this file is regenerated from its source. Send questions 
% to Arlindo da Silva, dasilva@gsfc.nasa.gov
 
\setlength{\oldparskip}{\parskip}
\setlength{\parskip}{1.5ex}
\setlength{\oldparindent}{\parindent}
\setlength{\parindent}{0pt}
\setlength{\oldbaselineskip}{\baselineskip}
\setlength{\baselineskip}{11pt}
 
%--------------------- SHORT-HAND MACROS ----------------------
\def\bv{\begin{verbatim}}
\def\ev{\end{verbatim}}
\def\be{\begin{equation}}
\def\ee{\end{equation}}
\def\bea{\begin{eqnarray}}
\def\eea{\end{eqnarray}}
\def\bi{\begin{itemize}}
\def\ei{\end{itemize}}
\def\bn{\begin{enumerate}}
\def\en{\end{enumerate}}
\def\bd{\begin{description}}
\def\ed{\end{description}}
\def\({\left (}
\def\){\right )}
\def\[{\left [}
\def\]{\right ]}
\def\<{\left  \langle}
\def\>{\right \rangle}
\def\cI{{\cal I}}
\def\diag{\mathop{\rm diag}}
\def\tr{\mathop{\rm tr}}
%-------------------------------------------------------------

\markboth{Left}{Source File: ESMF\_AppMainEx.F90,  Date: Tue May  5 21:00:11 MDT 2020
}

 
%/////////////////////////////////////////////////////////////

  
{\sf DESCRIPTION:\\ }


   Example of what a main program which uses ESMF might look like, along
    with 2 Gridded Components and a Coupler Component.  
    (In a real application each Component would probably be in separate files.)
   
%/////////////////////////////////////////////////////////////

 \begin{verbatim}
!-------------------------------------------------------------------------
!   ! Example Gridded Component that the main program will call.

    module PHYS_mod

    use ESMF
    public PHYS_SetServices
    contains

!   ! Public subroutine which the main program will call to register the
!   ! various user-supplied subroutines which make up this Component.
    subroutine PHYS_SetServices(gcomp, rc)
      type(ESMF_GridComp) :: gcomp
      integer, intent(out) :: rc

       call ESMF_GridCompSetEntryPoint(gcomp, ESMF_METHOD_INITIALIZE, my_init, rc=rc)
       call ESMF_GridCompSetEntryPoint(gcomp, ESMF_METHOD_RUN, my_run, rc=rc)
       call ESMF_GridCompSetEntryPoint(gcomp, ESMF_METHOD_FINALIZE, my_final, rc=rc)
      
    end subroutine PHYS_SetServices
      
!   ! User-written Initialization routine
    subroutine my_init(gcomp, importState, exportState, externalclock, rc)
      type(ESMF_GridComp) :: gcomp
      type(ESMF_State) :: importState
      type(ESMF_State) :: exportState
      type(ESMF_Clock) :: externalclock
      integer, intent(out) :: rc

      print *, "PHYS initialize routine called"
      rc = ESMF_SUCCESS

    end subroutine my_init

!   ! User-written Run routine
    subroutine my_run(gcomp, importState, exportState, externalclock, rc)
      type(ESMF_GridComp) :: gcomp
      type(ESMF_State) :: importState
      type(ESMF_State) :: exportState
      type(ESMF_Clock) :: externalclock
      integer, intent(out) :: rc

      print *, "PHYS run routine called"
      rc = ESMF_SUCCESS

    end subroutine my_run

!   ! User-written Finalization routine
    subroutine my_final(gcomp, importState, exportState, externalclock, rc)
      type(ESMF_GridComp) :: gcomp
      type(ESMF_State) :: importState
      type(ESMF_State) :: exportState
      type(ESMF_Clock) :: externalclock
      integer, intent(out) :: rc

      print *, "PHYS finalize routine called"
      rc = ESMF_SUCCESS

    end subroutine my_final

    end module
!   ! End of Gridded Component module
!-------------------------------------------------------------------------

!-------------------------------------------------------------------------
!   ! Start of a Second Gridded Component module
    module DYNM_mod

    use ESMF
    public DYNM_SetServices
    contains

!   ! Public subroutine which the main program will call to register the
!   ! various user-supplied subroutines which make up this Component.
    subroutine DYNM_SetServices(gcomp, rc)
      type(ESMF_GridComp) :: gcomp
      integer, intent(out) :: rc

       call ESMF_GridCompSetEntryPoint(gcomp, ESMF_METHOD_INITIALIZE, my_init, rc=rc)
       call ESMF_GridCompSetEntryPoint(gcomp, ESMF_METHOD_RUN, my_run, rc=rc)
       call ESMF_GridCompSetEntryPoint(gcomp, ESMF_METHOD_FINALIZE, my_final, rc=rc)
      
    end subroutine DYNM_SetServices
      
!   ! User-written Initialization routine
    subroutine my_init(gcomp, importState, exportState, externalclock, rc)
      type(ESMF_GridComp) :: gcomp
      type(ESMF_State) :: importState
      type(ESMF_State) :: exportState
      type(ESMF_Clock) :: externalclock
      integer, intent(out) :: rc

      print *, "DYNM initialize routine called"
      rc = ESMF_SUCCESS

    end subroutine my_init

!   ! User-written Run routine
    subroutine my_run(gcomp, importState, exportState, externalclock, rc)
      type(ESMF_GridComp) :: gcomp
      type(ESMF_State) :: importState
      type(ESMF_State) :: exportState
      type(ESMF_Clock) :: externalclock
      integer, intent(out) :: rc

      print *, "DYNM run routine called"
      rc = ESMF_SUCCESS

    end subroutine my_run

!   ! User-written Finalization routine
    subroutine my_final(gcomp, importState, exportState, externalclock, rc)
      type(ESMF_GridComp) :: gcomp
      type(ESMF_State) :: importState
      type(ESMF_State) :: exportState
      type(ESMF_Clock) :: externalclock
      integer, intent(out) :: rc

      print *, "DYNM finalize routine called"
      rc = ESMF_SUCCESS

    end subroutine my_final

    end module
!   ! End of Second Gridded Component module
!-------------------------------------------------------------------------

!-------------------------------------------------------------------------
!   ! Start of a Coupler Component module
    module CPLR_mod

    use ESMF

    public CPLR_SetServices
    contains

!   ! Public subroutine which the main program will call to register the
!   ! various user-supplied subroutines which make up this Component.
    subroutine CPLR_SetServices(cpl, rc)
      type(ESMF_CplComp) :: cpl
      integer, intent(out) :: rc

       call ESMF_CplCompSetEntryPoint(cpl, ESMF_METHOD_INITIALIZE, my_init, rc=rc)
       call ESMF_CplCompSetEntryPoint(cpl, ESMF_METHOD_RUN, my_run, rc=rc)
       call ESMF_CplCompSetEntryPoint(cpl, ESMF_METHOD_FINALIZE, my_final, rc=rc)
      
    end subroutine CPLR_SetServices
      
!   ! User-written Initialization routine
    subroutine my_init(cpl, importStatelist, exportStatelist, externalclock, rc)
      type(ESMF_CplComp) :: cpl
      type(ESMF_State) :: importStatelist
      type(ESMF_State) :: exportStatelist
      type(ESMF_Clock) :: externalclock
      integer, intent(out) :: rc

      print *, "CPLR initialize routine called"
      rc = ESMF_SUCCESS

    end subroutine my_init

!   ! User-written Run routine
    subroutine my_run(cpl, importStatelist, exportStatelist, externalclock, rc)
      type(ESMF_CplComp) :: cpl
      type(ESMF_State) :: importStatelist
      type(ESMF_State) :: exportStatelist
      type(ESMF_Clock) :: externalclock
      integer, intent(out) :: rc

      print *, "CPLR run routine called"
      rc = ESMF_SUCCESS

    end subroutine my_run

!   ! User-written Finalization routine
    subroutine my_final(cpl, importStatelist, exportStatelist, externalclock, rc)
      type(ESMF_CplComp) :: cpl
      type(ESMF_State) :: importStatelist
      type(ESMF_State) :: exportStatelist
      type(ESMF_Clock) :: externalclock
      integer, intent(out) :: rc

      print *, "CPLR finalize routine called"
      rc = ESMF_SUCCESS

    end subroutine my_final

    end module
!   ! End of Gridded Component module
!-------------------------------------------------------------------------

!-------------------------------------------------------------------------
!   ! Start of the main program.
    program ESMF_AppMainEx
#include "ESMF.h"
    
!   ! The ESMF Framework module
    use ESMF
    use ESMF_TestMod
    
!   ! User supplied modules, using only the public registration routine.
    use PHYS_Mod, only: PHYS_SetServices
    use DYNM_Mod, only: DYNM_SetServices
    use CPLR_Mod, only: CPLR_SetServices
    implicit none
    
!   ! Local variables
    integer :: rc
    logical :: finished
    type(ESMF_Clock) :: tclock
    type(ESMF_Calendar) :: gregorianCalendar
    type(ESMF_TimeInterval) :: timeStep
    type(ESMF_Time) :: startTime, stopTime
    character(ESMF_MAXSTR) :: cname, cname1, cname2
    type(ESMF_VM) :: vm
    type(ESMF_State) :: states(2)
    type(ESMF_GridComp) :: top
    type(ESMF_GridComp) :: gcomp1, gcomp2
    type(ESMF_CplComp) :: cpl
        
 
\end{verbatim}
 
%/////////////////////////////////////////////////////////////

 \begin{verbatim}
!-------------------------------------------------------------------------
!   ! Initialize the Framework, and get the default VM
    call ESMF_Initialize(vm=vm, defaultlogfilename="AppMainEx.Log", &
                    logkindflag=ESMF_LOGKIND_MULTI, rc=rc)
    if (rc .ne. ESMF_SUCCESS) then
        print *, "Unable to initialize ESMF Framework"
        print *, "FAIL: ESMF_AppMainEx.F90"
        call ESMF_Finalize(endflag=ESMF_END_ABORT)
    endif


!-------------------------------------------------------------------------
!   !
!   !  Create, Init, Run, Finalize, Destroy Components.
 
    print *, "Application Example 1:"

    ! Create the top level application component

    cname = "Top Level Atmosphere Model Component"
    top = ESMF_GridCompCreate(name=cname, configFile="setup.rc", rc=rc)  
 
\end{verbatim}
 
%/////////////////////////////////////////////////////////////

 \begin{verbatim}
    cname1 = "Atmosphere Physics"
    gcomp1 = ESMF_GridCompCreate(name=cname1, rc=rc)  
 
\end{verbatim}
 
%/////////////////////////////////////////////////////////////

 \begin{verbatim}

    ! This single user-supplied subroutine must be a public entry point 
    !  and can renamed with the 'use localname => modulename' syntax if
    !  the name is not unique.
    call ESMF_GridCompSetServices(gcomp1, userRoutine=PHYS_SetServices, rc=rc)
 
\end{verbatim}
 
%/////////////////////////////////////////////////////////////

 \begin{verbatim}
    ! (see below for what the SetServices routine will need to do.)

    print *, "Comp Create returned, name = ", trim(cname1)

    cname2 = "Atmosphere Dynamics"
    gcomp2 = ESMF_GridCompCreate(name=cname2, rc=rc)  
 
\end{verbatim}
 
%/////////////////////////////////////////////////////////////

 \begin{verbatim}

    ! This single user-supplied subroutine must be a public entry point.
    call ESMF_GridCompSetServices(gcomp2, userRoutine=DYNM_SetServices, rc=rc)
 
\end{verbatim}
 
%/////////////////////////////////////////////////////////////

 \begin{verbatim}

    print *, "Comp Create returned, name = ", trim(cname2)

    cname = "Atmosphere Coupler"
    cpl = ESMF_CplCompCreate(name=cname, rc=rc)
 
\end{verbatim}
 
%/////////////////////////////////////////////////////////////

 \begin{verbatim}

    ! This single user-supplied subroutine must be a public entry point.
    call ESMF_CplCompSetServices(cpl, userRoutine=CPLR_SetServices, rc=rc)
 
\end{verbatim}
 
%/////////////////////////////////////////////////////////////

 \begin{verbatim}

    print *, "Comp Create returned, name = ", trim(cname)

    ! Create the necessary import and export states used to pass data
    !  between components.

    states(1) = ESMF_StateCreate(name=cname1,  &
                                 stateintent=ESMF_STATEINTENT_EXPORT, rc=rc)
 
\end{verbatim}
 
%/////////////////////////////////////////////////////////////

 \begin{verbatim}
    states(2) = ESMF_StateCreate(name=cname2,  &
                                 stateintent=ESMF_STATEINTENT_IMPORT, rc=rc)
 
\end{verbatim}
 
%/////////////////////////////////////////////////////////////

 \begin{verbatim}
    ! See the TimeMgr document for the details on the actual code needed
    !  to set up a clock.
    ! initialize calendar to be Gregorian type
    gregorianCalendar = ESMF_CalendarCreate(ESMF_CALKIND_GREGORIAN, name="Gregorian", rc=rc)
 
\end{verbatim}
 
%/////////////////////////////////////////////////////////////

 \begin{verbatim}

    ! initialize time interval to 6 hours
    call ESMF_TimeIntervalSet(timeStep, h=6, rc=rc)
 
\end{verbatim}
 
%/////////////////////////////////////////////////////////////

 \begin{verbatim}

    ! initialize start time to 5/1/2003
    call ESMF_TimeSet(startTime, yy=2003, mm=5, dd=1, &
                      calendar=gregorianCalendar, rc=rc)
 
\end{verbatim}
 
%/////////////////////////////////////////////////////////////

 \begin{verbatim}

    ! initialize stop time to 5/2/2003
    call ESMF_TimeSet(stopTime, yy=2003, mm=5, dd=2, &
                      calendar=gregorianCalendar, rc=rc)
 
\end{verbatim}
 
%/////////////////////////////////////////////////////////////

 \begin{verbatim}

    ! initialize the clock with the above values
    tclock = ESMF_ClockCreate(timeStep, startTime, stopTime=stopTime, &
                              name="top clock", rc=rc)
 
\end{verbatim}
 
%/////////////////////////////////////////////////////////////

 \begin{verbatim}
     
    ! Call each Init routine in turn.  There is an optional index number
    !  for those components which have multiple entry points.
    call ESMF_GridCompInitialize(gcomp1, exportState=states(1), clock=tclock, &
                                 rc=rc)
 
\end{verbatim}
 
%/////////////////////////////////////////////////////////////

 \begin{verbatim}
    call ESMF_GridCompInitialize(gcomp2, importState=states(2), clock=tclock, &
                                 rc=rc)
 
\end{verbatim}
 
%/////////////////////////////////////////////////////////////

 \begin{verbatim}
    call ESMF_CplCompInitialize(cpl, importState=states(1), &
      exportState=states(2), clock=tclock, rc=rc)
 
\end{verbatim}
 
%/////////////////////////////////////////////////////////////

 \begin{verbatim}
    print *, "Comp Initialize complete"

    ! Main run loop.
    finished = .false.
    do while (.not. finished)
        call ESMF_GridCompRun(gcomp1, exportState=states(1), clock=tclock, rc=rc)
 
\end{verbatim}
 
%/////////////////////////////////////////////////////////////

 \begin{verbatim}
        call ESMF_CplCompRun(cpl, importState=states(1), &
                                  exportState=states(2), clock=tclock, rc=rc)
 
\end{verbatim}
 
%/////////////////////////////////////////////////////////////

 \begin{verbatim}
        call ESMF_GridCompRun(gcomp2, importState=states(2), clock=tclock, rc=rc)
 
\end{verbatim}
 
%/////////////////////////////////////////////////////////////

 \begin{verbatim}
        call ESMF_ClockAdvance(tclock, timeStep=timestep)
        ! query clock for current time
        if (ESMF_ClockIsStopTime(tclock)) finished = .true.
    enddo
    print *, "Comp Run complete"

    ! Give each component a chance to write out final results, clean up.
    ! Call each Finalize routine in turn.  There is an optional index number
    !  for those components which have multiple entry points.
    call ESMF_GridCompFinalize(gcomp1, exportState=states(1), clock=tclock, rc=rc)
 
\end{verbatim}
 
%/////////////////////////////////////////////////////////////

 \begin{verbatim}
    call ESMF_GridCompFinalize(gcomp2, importState=states(2), clock=tclock, rc=rc)
 
\end{verbatim}
 
%/////////////////////////////////////////////////////////////

 \begin{verbatim}
    call ESMF_CplCompFinalize(cpl, importState=states(1), &
      exportState=states(2), clock=tclock, rc=rc)
 
\end{verbatim}
 
%/////////////////////////////////////////////////////////////

 \begin{verbatim}
    print *, "Comp Finalize complete"

    ! Destroy objects
    call ESMF_StateDestroy(states(1), rc=rc)
 
\end{verbatim}
 
%/////////////////////////////////////////////////////////////

 \begin{verbatim}
    call ESMF_StateDestroy(states(2), rc=rc)
 
\end{verbatim}
 
%/////////////////////////////////////////////////////////////

 \begin{verbatim}
    call ESMF_ClockDestroy(tclock, rc=rc)
 
\end{verbatim}
 
%/////////////////////////////////////////////////////////////

 \begin{verbatim}
    call ESMF_CalendarDestroy(gregorianCalendar, rc=rc)
 
\end{verbatim}
 
%/////////////////////////////////////////////////////////////

 \begin{verbatim}
    call ESMF_GridCompDestroy(gcomp1, rc=rc)
 
\end{verbatim}
 
%/////////////////////////////////////////////////////////////

 \begin{verbatim}
    call ESMF_GridCompDestroy(gcomp2, rc=rc)
 
\end{verbatim}
 
%/////////////////////////////////////////////////////////////

 \begin{verbatim}
    call ESMF_CplCompDestroy(cpl, rc=rc)
 
\end{verbatim}
 
%/////////////////////////////////////////////////////////////

 \begin{verbatim}
    call ESMF_GridCompDestroy(top, rc=rc)
 
\end{verbatim}
 
%/////////////////////////////////////////////////////////////

 \begin{verbatim}
    print *, "Comp Destroy returned"

    print *, "Application Example 1 finished"

 
\end{verbatim}
 
%/////////////////////////////////////////////////////////////

 \begin{verbatim}

    call ESMF_Finalize(rc=rc)
 
\end{verbatim}
 
%/////////////////////////////////////////////////////////////

 \begin{verbatim}

    end program ESMF_AppMainEx
!   ! End of main program
!-------------------------------------------------------------------------

    ! Each Component must supply a SetServices routine which makes the
    !  following types of calls:
    !
    !! call ESMF_GridCompSetEntryPoint(gcomp1, ESMF_METHOD_INITIALIZE, PHYS_Init, 1, rc=rc)
    !! call ESMF_GridCompSetEntryPoint(gcomp1, ESMF_METHOD_INITIALIZE, PHYS_InitPhase2, 2, rc=rc)
    !! call ESMF_GridCompSetEntryPoint(gcomp1, ESMF_METHOD_RUN, PHYS_Run, 0, rc=rc)
    !! call ESMF_GridCompSetEntryPoint(gcomp1, ESMF_METHOD_FINALIZE, PHYS_Final, 0, rc=rc)
    !
    ! The arguments are: the component, the type of routine, 
    !  the name of the internal subroutine which contains the user code, 
    !  and a "phase" or index number to support multiple entry points 
    !  of the same type for codes which need to compute part of the process
    !  and then allow another component to run before completing the function.

 
\end{verbatim}

%...............................................................
\setlength{\parskip}{\oldparskip}
\setlength{\parindent}{\oldparindent}
\setlength{\baselineskip}{\oldbaselineskip}
