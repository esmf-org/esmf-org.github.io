%                **** IMPORTANT NOTICE *****
% This LaTeX file has been automatically produced by ProTeX v. 1.1
% Any changes made to this file will likely be lost next time
% this file is regenerated from its source. Send questions 
% to Arlindo da Silva, dasilva@gsfc.nasa.gov
 
\setlength{\oldparskip}{\parskip}
\setlength{\parskip}{1.5ex}
\setlength{\oldparindent}{\parindent}
\setlength{\parindent}{0pt}
\setlength{\oldbaselineskip}{\baselineskip}
\setlength{\baselineskip}{11pt}
 
%--------------------- SHORT-HAND MACROS ----------------------
\def\bv{\begin{verbatim}}
\def\ev{\end{verbatim}}
\def\be{\begin{equation}}
\def\ee{\end{equation}}
\def\bea{\begin{eqnarray}}
\def\eea{\end{eqnarray}}
\def\bi{\begin{itemize}}
\def\ei{\end{itemize}}
\def\bn{\begin{enumerate}}
\def\en{\end{enumerate}}
\def\bd{\begin{description}}
\def\ed{\end{description}}
\def\({\left (}
\def\){\right )}
\def\[{\left [}
\def\]{\right ]}
\def\<{\left  \langle}
\def\>{\right \rangle}
\def\cI{{\cal I}}
\def\diag{\mathop{\rm diag}}
\def\tr{\mathop{\rm tr}}
%-------------------------------------------------------------

\markboth{Left}{Source File: ESMC\_SciComp.h,  Date: Tue May  5 21:00:10 MDT 2020
}

 
%/////////////////////////////////////////////////////////////
\subsubsection [ESMC\_SciCompCreate] {ESMC\_SciCompCreate - Create a Science Component}


  
\bigskip{\sf INTERFACE:}
\begin{verbatim} ESMC_SciComp ESMC_SciCompCreate(
   const char *name,                    // in 
   int *rc                              // out
 );\end{verbatim}{\em RETURN VALUE:}
\begin{verbatim}    Newly created ESMC_SciComp object.\end{verbatim}
{\sf DESCRIPTION:\\ }


  
    This interface creates an {\tt ESMC\_SciComp} object. 
  
    The arguments are:
    \begin{description}
    \item[name]
      Name of the newly-created {\tt ESMC\_SciComp}.
    \item[{[rc]}]
     Return code; equals {\tt ESMF\_SUCCESS} if there are no errors.
    \end{description}
   
%/////////////////////////////////////////////////////////////
 
\mbox{}\hrulefill\ 
 
\subsubsection [ESMC\_SciCompDestroy] {ESMC\_SciCompDestroy - Destroy a Science Component}


  
\bigskip{\sf INTERFACE:}
\begin{verbatim} int ESMC_SciCompDestroy(
   ESMC_SciComp *comp               // inout
 );\end{verbatim}{\em RETURN VALUE:}
\begin{verbatim}    Return code; equals ESMF_SUCCESS if there are no errors.\end{verbatim}
{\sf DESCRIPTION:\\ }


  
    Releases all resources associated with this {\tt ESMC\_SciComp}.
  
    The arguments are:
    \begin{description}
    \item[comp]
      Release all resources associated with this {\tt ESMC\_SciComp} and mark
      the object as invalid. It is an error to pass this object into any other
      routines after being destroyed. 
   \end{description}
   
%/////////////////////////////////////////////////////////////
 
\mbox{}\hrulefill\ 
 
\subsubsection [ESMC\_SciCompPrint] {ESMC\_SciCompPrint - Print the contents of a SciComp}


  
\bigskip{\sf INTERFACE:}
\begin{verbatim} int ESMC_SciCompPrint(
   ESMC_SciComp comp     // in
 );\end{verbatim}{\em RETURN VALUE:}
\begin{verbatim}    Return code; equals ESMF_SUCCESS if there are no errors.\end{verbatim}
{\sf DESCRIPTION:\\ }


  
    Prints information about an {\tt ESMC\_SciComp} to {\tt stdout}.
  
    The arguments are:
    \begin{description}
    \item[comp]
      An {\tt ESMC\_SciComp} object.
   \end{description}
  
%...............................................................
\setlength{\parskip}{\oldparskip}
\setlength{\parindent}{\oldparindent}
\setlength{\baselineskip}{\oldbaselineskip}
