%                **** IMPORTANT NOTICE *****
% This LaTeX file has been automatically produced by ProTeX v. 1.1
% Any changes made to this file will likely be lost next time
% this file is regenerated from its source. Send questions 
% to Arlindo da Silva, dasilva@gsfc.nasa.gov
 
\setlength{\oldparskip}{\parskip}
\setlength{\parskip}{1.5ex}
\setlength{\oldparindent}{\parindent}
\setlength{\parindent}{0pt}
\setlength{\oldbaselineskip}{\baselineskip}
\setlength{\baselineskip}{11pt}
 
%--------------------- SHORT-HAND MACROS ----------------------
\def\bv{\begin{verbatim}}
\def\ev{\end{verbatim}}
\def\be{\begin{equation}}
\def\ee{\end{equation}}
\def\bea{\begin{eqnarray}}
\def\eea{\end{eqnarray}}
\def\bi{\begin{itemize}}
\def\ei{\end{itemize}}
\def\bn{\begin{enumerate}}
\def\en{\end{enumerate}}
\def\bd{\begin{description}}
\def\ed{\end{description}}
\def\({\left (}
\def\){\right )}
\def\[{\left [}
\def\]{\right ]}
\def\<{\left  \langle}
\def\>{\right \rangle}
\def\cI{{\cal I}}
\def\diag{\mathop{\rm diag}}
\def\tr{\mathop{\rm tr}}
%-------------------------------------------------------------

\markboth{Left}{Source File: ESMF\_CompTunnelEx.F90,  Date: Tue May  5 21:00:11 MDT 2020
}

 
%/////////////////////////////////////////////////////////////

  \subsubsection{Creating an {\em actual} Component}
   
   \label{sec:CompTunnelActualCreate}
  
   The creation process of an {\em actual} Gridded Component, which will become
   one of the two end points of a Component Tunnel, is identical to the creation
   of a regular Gridded Component. On the actual side, an actual Component is 
   very similar to a regular Component. Here the actual Component is created
   with a custom {\tt petList}. 
%/////////////////////////////////////////////////////////////

 \begin{verbatim}
  petList = (/0,1,2/)
  actualComp = ESMF_GridCompCreate(petList=petList, name="actual", rc=rc)
 
\end{verbatim}
 
%/////////////////////////////////////////////////////////////

  \subsubsection{Creating a {\em dual} Component}
   
   \label{sec:CompTunnelDualCreate}
  
   The same way an actual Component appears as a regular Component in
   the context of the actual side application, a {\em dual} Component
   is created as a regular Component on the dual side.
   A dual Gridded Component with custom {\tt petList} is created using the
   regular create call. 
%/////////////////////////////////////////////////////////////

 \begin{verbatim}
  petList = (/4,3,5/)
  dualComp = ESMF_GridCompCreate(petList=petList, name="dual", rc=rc)
 
\end{verbatim}
 
%/////////////////////////////////////////////////////////////

  \subsubsection{Setting up the {\em actual} side of a Component Tunnel}
   
   \label{sec:CompTunnelActualSide}
  
   After creation, the regular procedure for registering the standard Component
   methods is followed for the actual Gridded Component. 
%/////////////////////////////////////////////////////////////

 \begin{verbatim}
  call ESMF_GridCompSetServices(actualComp, userRoutine=setservices, &
    userRc=userRc, rc=rc)
 
\end{verbatim}
 
%/////////////////////////////////////////////////////////////

   So far the {\tt actualComp} object is no different from a regular Gridded
   Component. In order to turn it into the {\em actual} end point of a Component
   Tunnel the {\tt ServiceLoop()} method is called. Here the socket-based
   implementation is chosen. 
%/////////////////////////////////////////////////////////////

 \begin{verbatim}
  call ESMF_GridCompServiceLoop(actualComp, port=61010, timeout=20, rc=rc)
 
\end{verbatim}
 
%/////////////////////////////////////////////////////////////

   This call opens the actual side of the Component Tunnel in form of a
   socket-based server, listening on {\tt port} 61010. The {\tt timeout} argument
   specifies how long the actual side will wait for the dual side
   to connect, before the actual side returns with a time out condition. The
   time out is set to 20 seconds.
  
   At this point, before a dual Component connects to the other side of the 
   Component Tunnel, it is
   possible to manually connect to the waiting actual Component. This can be
   useful when debugging connection issues. A convenient tool for this is the 
   standard {\tt telnet} application. Below is a transcript of such a connection.
   The manually typed commands are separate from the previous responses by a 
   blank line.
  
   \begin{verbatim}
   $ telnet localhost 61010
   Trying 127.0.0.1...
   Connected to localhost.
   Escape character is '^]'.
   Hello from ESMF Actual Component server!
  
   date
   Tue Apr  3 21:53:03 2012
   
   version
   ESMF_VERSION_STRING: 5.3.0
   \end{verbatim}
  
   If at any point the {\tt telnet} session is manually shut down, the 
   {\tt ServiceLoop()} on the actual side will return with an error condition. 
   The clean way to
   disconnect the {\tt telnet} session, and to have the {\tt ServiceLoop()}
   wait for a new connection, e.g. from a dual Component, is to send the
   {\tt reconnect} command. This will automatically shut down the {\tt telnet}
   connection.
  
   \begin{verbatim}
   reconnect
   Actual Component server will reconnect now!
   Connection closed by foreign host.
   $
   \end{verbatim}
  
   At this point the actual Component is back in listening mode, with a time out
   of 20 seconds, as specified during the ServiceLoop() call.
  
   \begin{sloppypar}
   Before moving on to the dual side of the GridComp based Component Tunnel 
   example, it should be pointed out that the exact same procedure is used to
   set up the actual side of a {\em CplComp} based Component Tunnel. Assuming
   that {\tt actualCplComp} is a CplComp object for which SetServices has already
   been called, the actual side uses {\tt ESMF\_CplCompServiceLoop()} to start
   listening for connections from the dual side.
   \end{sloppypar} 
%/////////////////////////////////////////////////////////////

 \begin{verbatim}
  call ESMF_CplCompServiceLoop(actualCplComp, port=61011, timeout=2, &
    timeoutFlag=timeoutFlag, rc=rc)
 
\end{verbatim}
 
%/////////////////////////////////////////////////////////////

   Here the {\tt timeoutFlag} is specified in order to prevent the expected
   time-out condition to be indicated through the return code. Instead, when
   {\tt timeoutFlag} is present, the return code is still {\tt ESMF\_SUCCESS}, 
   but {\tt timeoutFlag} is set to {\tt .true.} when a time-out occurs. 
%/////////////////////////////////////////////////////////////

  \subsubsection{Setting up the {\em dual} side of a Component Tunnel}
   
   \label{sec:CompTunnelDualSide}
  
   On the dual side, the {\tt dualComp} object needs to be connected to the
   actual Component in order to complete the Component Tunnel. Instead of
   registering standard Component methods locally, a special variant of the
   {\tt SetServices()} call is used to connect to the actual Component. 
%/////////////////////////////////////////////////////////////

 \begin{verbatim}
  call ESMF_GridCompSetServices(dualComp, port=61010, server="localhost", &
    timeout=10, timeoutFlag=timeoutFlag, rc=rc)
 
\end{verbatim}
 
%/////////////////////////////////////////////////////////////

   The {\tt port} and {\tt server} arguments are used to connect to the desired
   actual Component. The time out of 10 seconds ensures that if the actual
   Component is not available, a time out condition is returned instead of
   resulting in a hang. The {\tt timeoutFlag} argument further absorbs the time
   out condition, either returning as {\tt .true.} or {\tt .false.}. In this mode
   the standard {\tt rc} will indicate success even when a time out condition
   was reached. 
%/////////////////////////////////////////////////////////////

  \subsubsection{Invoking standard Component methods through a Component Tunnel}
   
   \label{sec:CompTunnelInvoking}
  
   Once a Component Tunnel is established, the actual Component is fully under
   the control of the dual Component. A standard Component method invoked on the
   dual Component is not executed by the dual Component itself, but by the 
   actual Component instead. In fact, it is the entry points registered with
   the actual Component that are executed when standard methods are invoked on
   the dual Component. The connected {\tt dualComp} object serves as a portal
   through which the connected {\tt actualComp} becomes accessible on the dual
   side.
  
   Typically the first standard method called is the {\tt CompInitialize()}
   routine. 
%/////////////////////////////////////////////////////////////

 \begin{verbatim}
  call ESMF_GridCompInitialize(dualComp, timeout=10, timeoutFlag=timeoutFlag, &
    userRc=userRc, rc=rc)
 
\end{verbatim}
 
%/////////////////////////////////////////////////////////////

   Again, the {\tt timeout} argument serves to prevent the dual side from 
   hanging if the actual Component application has experienced a catastrophic
   condition and is no longer available, or takes longer than expected. The
   presence of the {\tt timeoutFlag} allows time out conditions to be caught
   gracefully, so the dual side can deal with it in an orderly fashion, instead
   of triggering an application abort due to an error condition.
  
   The {\tt CompRun()} and {\tt CompFinalize()} methods follow the same format.
   
%/////////////////////////////////////////////////////////////

 \begin{verbatim}
  call ESMF_GridCompRun(dualComp, timeout=10, timeoutFlag=timeoutFlag, &
    userRc=userRc, rc=rc)
 
\end{verbatim}
 
%/////////////////////////////////////////////////////////////

 \begin{verbatim}
  call ESMF_GridCompFinalize(dualComp, timeout=10, timeoutFlag=timeoutFlag, &
    userRc=userRc, rc=rc)
 
\end{verbatim}
 
%/////////////////////////////////////////////////////////////

  \subsubsection{The non-blocking option to invoke standard Component methods through a Component Tunnel}
   
   \label{sec:CompTunnelInvokingNonblocking}
  
   Standard Component methods called on a connected dual Component are executed
   on the actual side, across the PETs of the actual Component. By default the
   dual Component PETs are blocked until the actual Component has finished
   executing the invoked Component method, or until a time out condition has been
   reached. In many practical applications a more loose synchronization between
   dual and actual Components is useful. Having the PETs of a dual
   Component return immediately from a standard Component method allows multiple
   dual Component, on the same PETs, to control multiple actual Components. 
   If the actual Components are executing in separate executables, or the same 
   executable but on exclusive sets of PETs, they can execute concurrently, even
   with the controlling dual Components all running on the same PETs.
   The non-blocking dual side regains control over the actual Component by 
   synchronizing through the CompWait() call.
  
   Any of the standard Component methods can be called in non-blocking mode
   by setting the optional {\tt syncflag} argument to 
   {\tt ESMF\_SYNC\_NONBLOCKING}.  
%/////////////////////////////////////////////////////////////

 \begin{verbatim}
  call ESMF_GridCompInitialize(dualComp, syncflag=ESMF_SYNC_NONBLOCKING, rc=rc)
 
\end{verbatim}
 
%/////////////////////////////////////////////////////////////

  
   {\em If} communication between the dual and the actual Component was successful, 
   this call will return immediately on all of the dual Component PETs, while the
   actual Component continues to execute the invoked Component method.
   However, if the dual Component has difficulties reaching the actual Component,
   the call will block on all dual PETs until successful contact was made, or the
   default time out (3600 seconds, i.e. 1 hour) has been reached. In most cases a 
   shorter time out condition is desired with the non-blocking option, as shown
   below.
  
   First the dual Component must wait for the outstanding method. 
%/////////////////////////////////////////////////////////////

 \begin{verbatim}
  call ESMF_GridCompWait(dualComp, rc=rc)
 
\end{verbatim}
 
%/////////////////////////////////////////////////////////////

  
   Now the same non-blocking CompInitialize() call is issued again, but this time
   with an explicit 10 second time out. 
%/////////////////////////////////////////////////////////////

 \begin{verbatim}
  call ESMF_GridCompInitialize(dualComp, syncflag=ESMF_SYNC_NONBLOCKING, &
    timeout=10, timeoutFlag=timeoutFlag, rc=rc)
 
\end{verbatim}
 
%/////////////////////////////////////////////////////////////

  
   This call is guaranteed to return within 10 seconds, or less, on the dual Component
   PETs, either without time out condition, indicating that the actual Component
   has been contacted successfully, or with time out condition, indicating that
   the actual Component was unreachable at the time. Either way, the dual 
   Component PETs are back under user control quickly.
  
   Calling the CompWait() method on the dual Component causes the dual Component
   PETs to block until the actual Component method has returned, or a time out
   condition has been reached. 
%/////////////////////////////////////////////////////////////

 \begin{verbatim}
  call ESMF_GridCompWait(dualComp, userRc=userRc, rc=rc)
 
\end{verbatim}
 
%/////////////////////////////////////////////////////////////

  
   The default time out for CompWait() is 3600 seconds, i.e. 1 hour, just like
   for the other Component methods. However, the semantics of a time out 
   condition under CompWait() is different from the other Component methods. Typically the {\tt timeout} is simply the
   maximum time that any communication between dual and actual Component is allowed 
   to take before a time out condition is raised. For CompWait(), the {\tt timeout}
   is the maximum time that an actual Component is allowed to execute before
   reporting back to the dual Component. Here, even with the default time out, 
   the dual Component would return from CompWait() immediately with a time out
   condition if the actual Component has already been executing for over 1 hour, 
   and is not already waiting to report back when the dual Component calls 
   CompWait(). On the other hand, if it has only been 30 minutes since 
   CompInitialize() was called on the dual Component, then the actual Component
   still has 30 minutes before CompWait() returns with a time out condition.
   During this time (or until the actual Component returns) the dual Component
   PETs are blocked.
  
   A standard Component method is invoked in non-blocking mode. 
%/////////////////////////////////////////////////////////////

 \begin{verbatim}
  call ESMF_GridCompRun(dualComp, syncflag=ESMF_SYNC_NONBLOCKING, &
    timeout=10, timeoutFlag=timeoutFlag, rc=rc)
 
\end{verbatim}
 
%/////////////////////////////////////////////////////////////

  
   Once the user code on the dual side is ready to regain control over the
   actual Component it calls CompWait() on the dual Component. Here a
   {\tt timeout} of 60s is specified, meaning that the total execution time the
   actual Component spends in the registered Run() routine may not exceed 60s
   before CompWait() returns with a time out condition. 
%/////////////////////////////////////////////////////////////

 \begin{verbatim}
  call ESMF_GridCompWait(dualComp, timeout=60, userRc=userRc, rc=rc)
 
\end{verbatim}
 
%/////////////////////////////////////////////////////////////

  \subsubsection{Destroying a connected {\em dual} Component}
   
   \label{sec:CompTunnelDualDestroy}
  
   A dual Component that is connected to an actual Component through a Component
   Tunnel is destroyed the same way a regular Component is. The only
   difference is that a connected dual Component may specify a {\tt timeout}
   argument to the {\tt CompDestroy()} call. 
%/////////////////////////////////////////////////////////////

 \begin{verbatim}
  call ESMF_GridCompDestroy(dualComp, timeout=10, rc=rc)
 
\end{verbatim}
 
%/////////////////////////////////////////////////////////////

   The {\tt timeout} argument again ensures that the dual side does not hang
   indefinitely in case the actual Component has become unavailable. If the
   actual Component is available, the destroy call will indicate to the actual
   Component that it should break out of the {\tt ServiceLoop()}. Either way,
   the local dual Component is destroyed. 
%/////////////////////////////////////////////////////////////

  \subsubsection{Destroying a connected {\em actual} Component}
   
   \label{sec:CompTunnelActualDestroy}
  
   An actual Component that is in a {\tt ServiceLoop()} must first return from 
   that call before it can be destroyed. This can either happen when a connected
   dual Component calls its {\tt CompDestroy()} method, or if the
   {\tt ServiceLoop()} reaches the specified time out condition. Either way,
   once control has been returned to the user code, the actual Component is 
   destroyed in the same way a regular Component is, by calling the destroy
   method. 
%/////////////////////////////////////////////////////////////

 \begin{verbatim}
  call ESMF_GridCompDestroy(actualComp, rc=rc)
 
\end{verbatim}

%...............................................................
\setlength{\parskip}{\oldparskip}
\setlength{\parindent}{\oldparindent}
\setlength{\baselineskip}{\oldbaselineskip}
