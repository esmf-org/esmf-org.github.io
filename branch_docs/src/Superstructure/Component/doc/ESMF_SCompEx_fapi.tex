%                **** IMPORTANT NOTICE *****
% This LaTeX file has been automatically produced by ProTeX v. 1.1
% Any changes made to this file will likely be lost next time
% this file is regenerated from its source. Send questions 
% to Arlindo da Silva, dasilva@gsfc.nasa.gov
 
\setlength{\oldparskip}{\parskip}
\setlength{\parskip}{1.5ex}
\setlength{\oldparindent}{\parindent}
\setlength{\parindent}{0pt}
\setlength{\oldbaselineskip}{\baselineskip}
\setlength{\baselineskip}{11pt}
 
%--------------------- SHORT-HAND MACROS ----------------------
\def\bv{\begin{verbatim}}
\def\ev{\end{verbatim}}
\def\be{\begin{equation}}
\def\ee{\end{equation}}
\def\bea{\begin{eqnarray}}
\def\eea{\end{eqnarray}}
\def\bi{\begin{itemize}}
\def\ei{\end{itemize}}
\def\bn{\begin{enumerate}}
\def\en{\end{enumerate}}
\def\bd{\begin{description}}
\def\ed{\end{description}}
\def\({\left (}
\def\){\right )}
\def\[{\left [}
\def\]{\right ]}
\def\<{\left  \langle}
\def\>{\right \rangle}
\def\cI{{\cal I}}
\def\diag{\mathop{\rm diag}}
\def\tr{\mathop{\rm tr}}
%-------------------------------------------------------------

\markboth{Left}{Source File: ESMF\_SCompEx.F90,  Date: Tue May  5 21:00:11 MDT 2020
}

 
%/////////////////////////////////////////////////////////////

   \subsubsection{Use ESMF\_SciComp and Attach Attributes}
   \label{sec:component:usage:scicomp}
  
  \begin{sloppypar}
   This example illustrates the use of the ESMF\_SciComp to attach Attributes
   within a Component hierarchy.  The hierarchy includes Coupler, Gridded,
   and Science Components and Attributes are attached to the Science Components.
   For demonstrable purposes, we'll add some CIM Component attributes to
   the Gridded Component.  However, for a complete example of the CIM
   Attribute packages supplied by ESMF, see the example in the ESMF Attributes
   section \ref{sec:attribute:usage:cimAttPack}.
  \end{sloppypar} 
%/////////////////////////////////////////////////////////////

  \begin{sloppypar}
      Create the top 2 levels of the Component hierarchy.  This example creates
      a parent Coupler Component and 2 Gridded Components as children.
  \end{sloppypar} 
%/////////////////////////////////////////////////////////////

 \begin{verbatim}
      ! Create top-level Coupler Component
      cplcomp = ESMF_CplCompCreate(name="coupler_component", rc=rc)

      ! Create Gridded Component for Atmosphere
      atmcomp = ESMF_GridCompCreate(name="Atmosphere", rc=rc)

      ! Create Gridded Component for Ocean
      ocncomp = ESMF_GridCompCreate(name="Ocean", rc=rc)

      ! Link the attributes for the parent and child components
      call ESMF_AttributeLink(cplcomp, atmcomp, rc=rc)
      call ESMF_AttributeLink(cplcomp, ocncomp, rc=rc)

 
\end{verbatim}
 
%/////////////////////////////////////////////////////////////

  \begin{sloppypar}
      Now add CIM Attribute packages to the Component.  Also, add
      a CIM Component Properties package, to contain two custom attributes.
  \end{sloppypar} 
%/////////////////////////////////////////////////////////////

 \begin{verbatim}
      convCIM = 'CIM 1.5'
      purpComp = 'ModelComp'
      purpProp = 'CompProp'
      purpField = 'Inputs'
      purpPlatform = 'Platform'

      convISO = 'ISO 19115'
      purpRP = 'RespParty'
      purpCitation = 'Citation'

      ! Add CIM Attribute package to the Science Component
      call ESMF_AttributeAdd(atmcomp, convention=convCIM, &
        purpose=purpComp, attpack=attpack, rc=rc)
 
\end{verbatim}
 
%/////////////////////////////////////////////////////////////

  \begin{sloppypar}
      The Attribute package can also be retrieved in a multi-Component
      setting like this:
  \end{sloppypar} 
%/////////////////////////////////////////////////////////////

 \begin{verbatim}
      call ESMF_AttributeGetAttPack(atmcomp, convCIM, purpComp, &
                                    attpack=attpack, rc=rc)
 
\end{verbatim}
 
%/////////////////////////////////////////////////////////////

  \begin{sloppypar}
       Now, add some CIM Component attributes to the Atmosphere Grid Component.
  \end{sloppypar} 
%/////////////////////////////////////////////////////////////

 \begin{verbatim}
      !
      ! Top-level model component attributes, set on gridded component
      !
      call ESMF_AttributeSet(atmcomp, 'ShortName', 'EarthSys_Atmos', &
        attpack=attpack, rc=rc)
 
\end{verbatim}
 
%/////////////////////////////////////////////////////////////

 \begin{verbatim}

      call ESMF_AttributeSet(atmcomp, 'LongName', &
        'Earth System High Resolution Global Atmosphere Model', &
        attpack=attpack, rc=rc)
 
\end{verbatim}
 
%/////////////////////////////////////////////////////////////

 \begin{verbatim}

      call ESMF_AttributeSet(atmcomp, 'Description', &
        'EarthSys brings together expertise from the global ' // &
        'community in a concerted effort to develop coupled ' // &
        'climate models with increased horizontal resolutions.  ' // &
        'Increasing the horizontal resolution of coupled climate ' // &
        'models will allow us to capture climate processes and ' // &
        'weather systems in much greater detail.', &
        attpack=attpack, rc=rc)
 
\end{verbatim}
 
%/////////////////////////////////////////////////////////////

 \begin{verbatim}

      call ESMF_AttributeSet(atmcomp, 'Version', '2.0', &
        attpack=attpack, rc=rc)
 
\end{verbatim}
 
%/////////////////////////////////////////////////////////////

 \begin{verbatim}

      call ESMF_AttributeSet(atmcomp, 'ReleaseDate', '2009-01-01T00:00:00Z', &
        attpack=attpack, rc=rc)
 
\end{verbatim}
 
%/////////////////////////////////////////////////////////////

 \begin{verbatim}

      call ESMF_AttributeSet(atmcomp, 'ModelType', 'aerosol', &
        attpack=attpack, rc=rc)
 
\end{verbatim}
 
%/////////////////////////////////////////////////////////////

 \begin{verbatim}

      call ESMF_AttributeSet(atmcomp, 'URL', &
        'www.earthsys.org', attpack=attpack, rc=rc)
 
\end{verbatim}
 
%/////////////////////////////////////////////////////////////

  \begin{sloppypar}
      Now create a set of Science Components as a children of the Atmosphere 
      Gridded Component. The hierarchy is as follows:
      \begin{itemize}
         \item Atmosphere
         \begin{itemize}
            \item AtmosDynamicalCore
            \begin{itemize}
               \item AtmosAdvection
            \end{itemize}
            \item AtmosRadiation
         \end{itemize}
      \end{itemize}
      After each Component is created, we need to link it with its parent
      Component.  We then add some standard CIM Component properties as well 
      as Scientific Properties to each of these components.
  \end{sloppypar} 
%/////////////////////////////////////////////////////////////

 \begin{verbatim}
    !
    ! Atmosphere Dynamical Core Science Component
    !
    dc_scicomp = ESMF_SciCompCreate(name="AtmosDynamicalCore", rc=rc)
 
\end{verbatim}
 
%/////////////////////////////////////////////////////////////

 \begin{verbatim}
    call ESMF_AttributeLink(atmcomp, dc_scicomp, rc=rc)
 
\end{verbatim}
 
%/////////////////////////////////////////////////////////////

 \begin{verbatim}
    call ESMF_AttributeAdd(dc_scicomp,  &
                           convention=convCIM, purpose=purpComp, &
                           attpack=attpack, rc=rc)

    call ESMF_AttributeSet(dc_scicomp, "ShortName", "AtmosDynamicalCore", &
                           attpack=attpack, rc=rc)
    call ESMF_AttributeSet(dc_scicomp, "LongName", &
                           "Atmosphere Dynamical Core", &
                           attpack=attpack, rc=rc)
 
\end{verbatim}
 
%/////////////////////////////////////////////////////////////

 \begin{verbatim}
    purpSci = 'SciProp'

    dc_sciPropAtt(1) = 'TopBoundaryCondition'
    dc_sciPropAtt(2) = 'HeatTreatmentAtTop'
    dc_sciPropAtt(3) = 'WindTreatmentAtTop'

    call ESMF_AttributeAdd(dc_scicomp,  &
                           convention=convCIM, purpose=purpSci, &
                           attrList=dc_sciPropAtt, &
                           attpack=attpack, rc=rc)

    call ESMF_AttributeSet(dc_scicomp, 'TopBoundaryCondition', &
                           'radiation boundary condition', &
                           attpack=attpack, rc=rc)
    call ESMF_AttributeSet(dc_scicomp, 'HeatTreatmentAtTop', &
                           'some heat treatment', &
                           attpack=attpack, rc=rc)
    call ESMF_AttributeSet(dc_scicomp, 'WindTreatmentAtTop', &
                           'some wind treatment', &
                           attpack=attpack, rc=rc)
 
\end{verbatim}
 
%/////////////////////////////////////////////////////////////

 \begin{verbatim}
    !
    ! Atmosphere Advection Science Component
    !
    adv_scicomp = ESMF_SciCompCreate(name="AtmosAdvection", rc=rc)
 
\end{verbatim}
 
%/////////////////////////////////////////////////////////////

 \begin{verbatim}
    call ESMF_AttributeLink(dc_scicomp, adv_scicomp, rc=rc)
 
\end{verbatim}
 
%/////////////////////////////////////////////////////////////

 \begin{verbatim}
    call ESMF_AttributeAdd(adv_scicomp,  &
                           convention=convCIM, purpose=purpComp, &
                           attpack=attpack, rc=rc)

    call ESMF_AttributeSet(adv_scicomp, "ShortName", "AtmosAdvection", &
                           attpack=attpack, rc=rc)
    call ESMF_AttributeSet(adv_scicomp, "LongName", "Atmosphere Advection", &
                           attpack=attpack, rc=rc)
 
\end{verbatim}
 
%/////////////////////////////////////////////////////////////

 \begin{verbatim}
    adv_sciPropAtt(1) = 'TracersSchemeName'
    adv_sciPropAtt(2) = 'TracersSchemeCharacteristics'
    adv_sciPropAtt(3) = 'MomentumSchemeName'

    call ESMF_AttributeAdd(adv_scicomp,  &
                           convention=convCIM, purpose=purpSci, &
                           attrList=adv_sciPropAtt, &
                           attpack=attpack, rc=rc)

    call ESMF_AttributeSet(adv_scicomp, 'TracersSchemeName', 'Prather', &
                           attpack=attpack, rc=rc)
    call ESMF_AttributeSet(adv_scicomp, 'TracersSchemeCharacteristics', &
                           'modified Euler', &
                           attpack=attpack, rc=rc)
    call ESMF_AttributeSet(adv_scicomp, 'MomentumSchemeName', 'Van Leer', &
                           attpack=attpack, rc=rc)
 
\end{verbatim}
 
%/////////////////////////////////////////////////////////////

 \begin{verbatim}
    !
    ! Atmosphere Radiation Science Component
    !
    rad_scicomp = ESMF_SciCompCreate(name="AtmosRadiation", rc=rc)
 
\end{verbatim}
 
%/////////////////////////////////////////////////////////////

 \begin{verbatim}
    call ESMF_AttributeLink(atmcomp, rad_scicomp, rc=rc)
 
\end{verbatim}
 
%/////////////////////////////////////////////////////////////

 \begin{verbatim}
    call ESMF_AttributeAdd(rad_scicomp,  &
                           convention=convCIM, purpose=purpComp, &
                           attpack=attpack, rc=rc)

    call ESMF_AttributeSet(rad_scicomp, "ShortName", "AtmosRadiation", &
                           attpack=attpack, rc=rc)
    call ESMF_AttributeSet(rad_scicomp, "LongName", &
                           "Atmosphere Radiation", &
                           attpack=attpack, rc=rc)
 
\end{verbatim}
 
%/////////////////////////////////////////////////////////////

 \begin{verbatim}
    rad_sciPropAtt(1) = 'LongwaveSchemeType'
    rad_sciPropAtt(2) = 'LongwaveSchemeMethod'

    call ESMF_AttributeAdd(rad_scicomp,  &
                           convention=convCIM, purpose=purpSci, &
                           attrList=rad_sciPropAtt, &
                           attpack=attpack, rc=rc)

    call ESMF_AttributeSet(rad_scicomp, &
                           'LongwaveSchemeType', &
                           'wide-band model', &
                           attpack=attpack, rc=rc)
    call ESMF_AttributeSet(rad_scicomp, &
                           'LongwaveSchemeMethod', &
                           'two-stream', &
                           attpack=attpack, rc=rc)
 
\end{verbatim}
 
 
%/////////////////////////////////////////////////////////////

  \begin{sloppypar}
       Write the entire CIM Attribute hierarchy, beginning at the top of the
       Component hierarchy (the Coupler Component), to an XML file formatted 
       to conform to CIM specifications.  The file is written to the examples 
       execution directory.
  \end{sloppypar} 
%/////////////////////////////////////////////////////////////

 \begin{verbatim}
        call ESMF_AttributeWrite(cplcomp, convCIM, purpComp, &
          attwriteflag=ESMF_ATTWRITE_XML,rc=rc)
 
\end{verbatim}
 
%/////////////////////////////////////////////////////////////

  \begin{sloppypar}
       Finally, destroy all of the Components.
  \end{sloppypar} 
%/////////////////////////////////////////////////////////////

 \begin{verbatim}
      call ESMF_SciCompDestroy(rad_scicomp, rc=rc)
      call ESMF_SciCompDestroy(adv_scicomp, rc=rc)
      call ESMF_SciCompDestroy(dc_scicomp, rc=rc)
      call ESMF_GridCompDestroy(atmcomp, rc=rc)
      call ESMF_GridCompDestroy(ocncomp, rc=rc)
      call ESMF_CplCompDestroy(cplcomp, rc=rc)
 
\end{verbatim}

%...............................................................
\setlength{\parskip}{\oldparskip}
\setlength{\parindent}{\oldparindent}
\setlength{\baselineskip}{\oldbaselineskip}
