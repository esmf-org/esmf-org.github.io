%                **** IMPORTANT NOTICE *****
% This LaTeX file has been automatically produced by ProTeX v. 1.1
% Any changes made to this file will likely be lost next time
% this file is regenerated from its source. Send questions 
% to Arlindo da Silva, dasilva@gsfc.nasa.gov
 
\setlength{\oldparskip}{\parskip}
\setlength{\parskip}{1.5ex}
\setlength{\oldparindent}{\parindent}
\setlength{\parindent}{0pt}
\setlength{\oldbaselineskip}{\baselineskip}
\setlength{\baselineskip}{11pt}
 
%--------------------- SHORT-HAND MACROS ----------------------
\def\bv{\begin{verbatim}}
\def\ev{\end{verbatim}}
\def\be{\begin{equation}}
\def\ee{\end{equation}}
\def\bea{\begin{eqnarray}}
\def\eea{\end{eqnarray}}
\def\bi{\begin{itemize}}
\def\ei{\end{itemize}}
\def\bn{\begin{enumerate}}
\def\en{\end{enumerate}}
\def\bd{\begin{description}}
\def\ed{\end{description}}
\def\({\left (}
\def\){\right )}
\def\[{\left [}
\def\]{\right ]}
\def\<{\left  \langle}
\def\>{\right \rangle}
\def\cI{{\cal I}}
\def\diag{\mathop{\rm diag}}
\def\tr{\mathop{\rm tr}}
%-------------------------------------------------------------

\markboth{Left}{Source File: ESMF\_SciComp.F90,  Date: Tue May  5 21:00:10 MDT 2020
}

 
%/////////////////////////////////////////////////////////////
\subsubsection [ESMF\_SciCompAssignment(=)] {ESMF\_SciCompAssignment(=) - SciComp assignment}


  
\bigskip{\sf INTERFACE:}
\begin{verbatim}     interface assignment(=)
     scicomp1 = scicomp2\end{verbatim}{\em ARGUMENTS:}
\begin{verbatim}     type(ESMF_SciComp) :: scicomp1
     type(ESMF_SciComp) :: scicomp2\end{verbatim}
{\sf DESCRIPTION:\\ }


     Assign scicomp1 as an alias to the same ESMF SciComp object in memory
     as scicomp2. If scicomp2 is invalid, then scicomp1 will be equally 
     invalid after the assignment.
  
     The arguments are:
     \begin{description}
     \item[scicomp1]
       The {\tt ESMF\_SciComp} object on the left hand side of the assignment.
     \item[scicomp2]
       The {\tt ESMF\_SciComp} object on the right hand side of the assignment.
     \end{description}
   
%/////////////////////////////////////////////////////////////
 
\mbox{}\hrulefill\ 
 
\subsubsection [ESMF\_SciCompOperator(==)] {ESMF\_SciCompOperator(==) - SciComp equality operator}


  
\bigskip{\sf INTERFACE:}
\begin{verbatim}   interface operator(==)
     if (scicomp1 == scicomp2) then ... endif
               OR
     result = (scicomp1 == scicomp2)\end{verbatim}{\em RETURN VALUE:}
\begin{verbatim}     logical :: result\end{verbatim}{\em ARGUMENTS:}
\begin{verbatim}     type(ESMF_SciComp), intent(in) :: scicomp1
     type(ESMF_SciComp), intent(in) :: scicomp2\end{verbatim}
{\sf DESCRIPTION:\\ }


     Test whether scicomp1 and scicomp2 are valid aliases to the same ESMF
     SciComp object in memory. For a more general comparison of two ESMF 
     SciComps, going beyond the simple alias test, the ESMF\_SciCompMatch() 
     function (not yet implemented) must be used.
  
     The arguments are:
     \begin{description}
     \item[scicomp1]
       The {\tt ESMF\_SciComp} object on the left hand side of the equality
       operation.
     \item[scicomp2]
       The {\tt ESMF\_SciComp} object on the right hand side of the equality
       operation.
     \end{description}
   
%/////////////////////////////////////////////////////////////
 
\mbox{}\hrulefill\ 
 
\subsubsection [ESMF\_SciCompOperator(/=)] {ESMF\_SciCompOperator(/=) - SciComp not equal operator}


  
\bigskip{\sf INTERFACE:}
\begin{verbatim}   interface operator(/=)
     if (scicomp1 /= scicomp2) then ... endif
               OR
     result = (scicomp1 /= scicomp2)\end{verbatim}{\em RETURN VALUE:}
\begin{verbatim}     logical :: result\end{verbatim}{\em ARGUMENTS:}
\begin{verbatim}     type(ESMF_SciComp), intent(in) :: scicomp1
     type(ESMF_SciComp), intent(in) :: scicomp2\end{verbatim}
{\sf DESCRIPTION:\\ }


     Test whether scicomp1 and scicomp2 are {\it not} valid aliases to the
     same ESMF SciComp object in memory. For a more general comparison of two 
     ESMF SciComps, going beyond the simple alias test, the ESMF\_SciCompMatch()
     function (not yet implemented) must be used.
  
     The arguments are:
     \begin{description}
     \item[scicomp1]
       The {\tt ESMF\_SciComp} object on the left hand side of the non-equality
       operation.
     \item[scicomp2]
       The {\tt ESMF\_SciComp} object on the right hand side of the non-equality
       operation.
     \end{description}
   
%/////////////////////////////////////////////////////////////
 
\mbox{}\hrulefill\ 
 
\subsubsection [ESMF\_SciCompCreate] {ESMF\_SciCompCreate - Create a SciComp}


  
\bigskip{\sf INTERFACE:}
\begin{verbatim}   recursive function ESMF_SciCompCreate(name, rc)\end{verbatim}{\em RETURN VALUE:}
\begin{verbatim}     type(ESMF_SciComp) :: ESMF_SciCompCreate\end{verbatim}{\em ARGUMENTS:}
\begin{verbatim} -- The following arguments require argument keyword syntax (e.g. rc=rc). --
     character(len=*),        intent(in),    optional :: name
     integer,                 intent(out),   optional :: rc\end{verbatim}
{\sf DESCRIPTION:\\ }


   This interface creates an {\tt ESMF\_SciComp} object. 
   The return value is the new {\tt ESMF\_SciComp}.
     
   The arguments are:
   \begin{description}
   \item[{[name]}]
     Name of the newly-created {\tt ESMF\_SciComp}.  This name can be altered
     from within the {\tt ESMF\_SciComp} code once the initialization routine
     is called.
   \item[{[rc]}]
     Return code; equals {\tt ESMF\_SUCCESS} if there are no errors.
   \end{description}
   
%/////////////////////////////////////////////////////////////
 
\mbox{}\hrulefill\ 
 
\subsubsection [ESMF\_SciCompDestroy] {ESMF\_SciCompDestroy - Release resources associated with a SciComp}


  
\bigskip{\sf INTERFACE:}
\begin{verbatim}   subroutine ESMF_SciCompDestroy(scicomp, rc)\end{verbatim}{\em ARGUMENTS:}
\begin{verbatim}     type(ESMF_SciComp), intent(inout)           :: scicomp
 -- The following arguments require argument keyword syntax (e.g. rc=rc). --
     integer,             intent(out),  optional :: rc\end{verbatim}
{\sf DESCRIPTION:\\ }


   Destroys an {\tt ESMF\_SciComp}, releasing the resources associated
   with the object.
  
   The arguments are:
   \begin{description}
   \item[scicomp]
     Release all resources associated with this {\tt ESMF\_SciComp}
     and mark the object as invalid.  It is an error to pass this
     object into any other routines after being destroyed.
   \item[{[rc]}]
     Return code; equals {\tt ESMF\_SUCCESS} if there are no errors.
   \end{description}
   
%/////////////////////////////////////////////////////////////
 
\mbox{}\hrulefill\ 
 
\subsubsection [ESMF\_SciCompGet] {ESMF\_SciCompGet - Get SciComp information}


  
\bigskip{\sf INTERFACE:}
\begin{verbatim}   subroutine ESMF_SciCompGet(scicomp, name, rc)\end{verbatim}{\em ARGUMENTS:}
\begin{verbatim}     type(ESMF_SciComp),       intent(in)            :: scicomp
 -- The following arguments require argument keyword syntax (e.g. rc=rc). --
     character(len=*),         intent(out), optional :: name
     integer,                  intent(out), optional :: rc\end{verbatim}
{\sf DESCRIPTION:\\ }


   Get information about an {\tt ESMF\_SciComp} object.
    
   The arguments are:
   \begin{description}
   \item[scicomp]
     The {\tt ESMF\_SciComp} object being queried.
   \item[{[name]}]
     Return the name of the {\tt ESMF\_SciComp}.
   \item[{[rc]}]
     Return code; equals {\tt ESMF\_SUCCESS} if there are no errors.
   \end{description}
   
%/////////////////////////////////////////////////////////////
 
\mbox{}\hrulefill\ 
 
\subsubsection [ESMF\_SciCompIsCreated] {ESMF\_SciCompIsCreated - Check whether a SciComp object has been created}


 
\bigskip{\sf INTERFACE:}
\begin{verbatim}   function ESMF_SciCompIsCreated(scicomp, rc)\end{verbatim}{\em RETURN VALUE:}
\begin{verbatim}     logical :: ESMF_SciCompIsCreated\end{verbatim}{\em ARGUMENTS:}
\begin{verbatim}     type(ESMF_SciComp), intent(in)            :: scicomp
 -- The following arguments require argument keyword syntax (e.g. rc=rc). --
     integer,             intent(out), optional :: rc
 \end{verbatim}
{\sf DESCRIPTION:\\ }


     Return {\tt .true.} if the {\tt scicomp} has been created. Otherwise return
     {\tt .false.}. If an error occurs, i.e. {\tt rc /= ESMF\_SUCCESS} is
     returned, the return value of the function will also be {\tt .false.}.
  
   The arguments are:
     \begin{description}
     \item[scicomp]
       {\tt ESMF\_SciComp} queried.
     \item[{[rc]}]
       Return code; equals {\tt ESMF\_SUCCESS} if there are no errors.
     \end{description}
   
%/////////////////////////////////////////////////////////////
 
\mbox{}\hrulefill\ 
 
\subsubsection [ESMF\_SciCompPrint] {ESMF\_SciCompPrint - Print SciComp information}


  
\bigskip{\sf INTERFACE:}
\begin{verbatim}   subroutine ESMF_SciCompPrint(scicomp, rc)\end{verbatim}{\em ARGUMENTS:}
\begin{verbatim}     type(ESMF_SciComp), intent(in)             :: scicomp
 -- The following arguments require argument keyword syntax (e.g. rc=rc). --
     integer,             intent(out), optional :: rc\end{verbatim}
{\sf DESCRIPTION:\\ }


   Prints information about an {\tt ESMF\_SciComp} to {\tt stdout}. \\
  
   The arguments are:
   \begin{description}
   \item[scicomp]
     {\tt ESMF\_SciComp} to print.
   \item[{[rc]}]
     Return code; equals {\tt ESMF\_SUCCESS} if there are no errors.
   \end{description}
   
%/////////////////////////////////////////////////////////////
 
\mbox{}\hrulefill\ 
 
\subsubsection [ESMF\_SciCompSet] {ESMF\_SciCompSet - Set or reset information about the SciComp}


  
\bigskip{\sf INTERFACE:}
\begin{verbatim}   subroutine ESMF_SciCompSet(scicomp, name, rc)\end{verbatim}{\em ARGUMENTS:}
\begin{verbatim}     type(ESMF_SciComp), intent(inout)          :: scicomp
 -- The following arguments require argument keyword syntax (e.g. rc=rc). --
     character(len=*),    intent(in),  optional :: name
     integer,             intent(out), optional :: rc\end{verbatim}
{\sf DESCRIPTION:\\ }


   Sets or resets information about an {\tt ESMF\_SciComp}.
  
   The arguments are:
   \begin{description}
   \item[scicomp]
     {\tt ESMF\_SciComp} to change.
   \item[{[name]}]
     Set the name of the {\tt ESMF\_SciComp}.
   \item[{[rc]}]
     Return code; equals {\tt ESMF\_SUCCESS} if there are no errors.
   \end{description}
   
%/////////////////////////////////////////////////////////////
 
\mbox{}\hrulefill\ 
 
\subsubsection [ESMF\_SciCompValidate] {ESMF\_SciCompValidate - Check validity of a SciComp}


  
\bigskip{\sf INTERFACE:}
\begin{verbatim}   subroutine ESMF_SciCompValidate(scicomp, rc)\end{verbatim}{\em ARGUMENTS:}
\begin{verbatim}     type(ESMF_SciComp), intent(in)             :: scicomp
 -- The following arguments require argument keyword syntax (e.g. rc=rc). --
     integer,             intent(out), optional :: rc\end{verbatim}
{\sf DESCRIPTION:\\ }


   Currently all this method does is to check that the {\tt scicomp}
   was created.
  
   The arguments are:
   \begin{description}
   \item[scicomp]
     {\tt ESMF\_SciComp} to validate.
   \item[{[rc]}]
     Return code; equals {\tt ESMF\_SUCCESS} if there are no errors.
   \end{description}
  
%...............................................................
\setlength{\parskip}{\oldparskip}
\setlength{\parindent}{\oldparindent}
\setlength{\baselineskip}{\oldbaselineskip}
