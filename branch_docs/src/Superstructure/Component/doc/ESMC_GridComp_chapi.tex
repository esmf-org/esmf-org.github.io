%                **** IMPORTANT NOTICE *****
% This LaTeX file has been automatically produced by ProTeX v. 1.1
% Any changes made to this file will likely be lost next time
% this file is regenerated from its source. Send questions 
% to Arlindo da Silva, dasilva@gsfc.nasa.gov
 
\setlength{\oldparskip}{\parskip}
\setlength{\parskip}{1.5ex}
\setlength{\oldparindent}{\parindent}
\setlength{\parindent}{0pt}
\setlength{\oldbaselineskip}{\baselineskip}
\setlength{\baselineskip}{11pt}
 
%--------------------- SHORT-HAND MACROS ----------------------
\def\bv{\begin{verbatim}}
\def\ev{\end{verbatim}}
\def\be{\begin{equation}}
\def\ee{\end{equation}}
\def\bea{\begin{eqnarray}}
\def\eea{\end{eqnarray}}
\def\bi{\begin{itemize}}
\def\ei{\end{itemize}}
\def\bn{\begin{enumerate}}
\def\en{\end{enumerate}}
\def\bd{\begin{description}}
\def\ed{\end{description}}
\def\({\left (}
\def\){\right )}
\def\[{\left [}
\def\]{\right ]}
\def\<{\left  \langle}
\def\>{\right \rangle}
\def\cI{{\cal I}}
\def\diag{\mathop{\rm diag}}
\def\tr{\mathop{\rm tr}}
%-------------------------------------------------------------

\markboth{Left}{Source File: ESMC\_GridComp.h,  Date: Tue May  5 21:00:10 MDT 2020
}

 
%/////////////////////////////////////////////////////////////
\subsubsection [ESMC\_GridCompCreate] {ESMC\_GridCompCreate - Create a Gridded Component}


  
\bigskip{\sf INTERFACE:}
\begin{verbatim} ESMC_GridComp ESMC_GridCompCreate(
   const char *name,                    // in 
   const char *configFile,              // in
   ESMC_Clock clock,                    // in
   int *rc                              // out
 );\end{verbatim}{\em RETURN VALUE:}
\begin{verbatim}    Newly created ESMC_GridComp object.\end{verbatim}
{\sf DESCRIPTION:\\ }


  
    This interface creates an {\tt ESMC\_GridComp} object. By default, a
    separate VM context will be created for each component.  This implies
    creating a new MPI communicator and allocating additional memory to
    manage the VM resources.
  
    The arguments are:
    \begin{description}
    \item[name]
      Name of the newly-created {\tt ESMC\_GridComp}.
    \item[configFile]
     The filename of an {\tt ESMC\_Config} format file. If specified, this file
     is opened an {\tt ESMC\_Config}  configuration object is created for the
     file, and attached to the new component. 
    \item[clock]
     Component-specific {\tt ESMC\_Clock}. This clock is available to be queried
     and updated by the new {\tt ESMC\_GridComp} as it chooses. This should not
     be the parent component clock, which should be maintained and passed down
     to the initialize/run/finalize routines separately. 
    \item[{[rc]}]
     Return code; equals {\tt ESMF\_SUCCESS} if there are no errors.
    \end{description}
   
%/////////////////////////////////////////////////////////////
 
\mbox{}\hrulefill\ 
 
\subsubsection [ESMC\_GridCompDestroy] {ESMC\_GridCompDestroy - Destroy a Gridded Component}


  
\bigskip{\sf INTERFACE:}
\begin{verbatim} int ESMC_GridCompDestroy(
   ESMC_GridComp *comp               // inout
 );\end{verbatim}{\em RETURN VALUE:}
\begin{verbatim}    Return code; equals ESMF_SUCCESS if there are no errors.\end{verbatim}
{\sf DESCRIPTION:\\ }


  
    Releases all resources associated with this {\tt ESMC\_GridComp}.
  
    The arguments are:
    \begin{description}
    \item[comp]
      Release all resources associated with this {\tt ESMC\_GridComp} and mark
      the object as invalid. It is an error to pass this object into any other
      routines after being destroyed. 
   \end{description}
   
%/////////////////////////////////////////////////////////////
 
\mbox{}\hrulefill\ 
 
\subsubsection [ESMC\_GridCompFinalize] {ESMC\_GridCompFinalize - Finalize a Gridded Component}


  
\bigskip{\sf INTERFACE:}
\begin{verbatim} int ESMC_GridCompFinalize(
   ESMC_GridComp comp,           // inout
   ESMC_State importState,       // inout
   ESMC_State exportState,       // inout 
   ESMC_Clock clock,             // in
   int phase,                    // in
   int *userRc                   // out
 );\end{verbatim}{\em RETURN VALUE:}
\begin{verbatim}    Return code; equals ESMF_SUCCESS if there are no errors.\end{verbatim}
{\sf DESCRIPTION:\\ }


  
    Call the associated user finalize code for a GridComp.
  
    The arguments are:
    \begin{description}
    \item[comp]
      {\tt ESMC\_GridComp} to call finalize routine for.
    \item[importState]
      {\tt ESMC\_State} containing import data for coupling.
    \item[exportState]
      {\tt ESMC\_State} containing export data for coupling.
    \item[clock]
      External {\tt ESMC\_Clock} for passing in time information. This is 
      generally the parent component's clock, and will be treated as read-only
      by the child component. The child component can maintain a private clock
      for its own internal time computations.
    \item[phase]
      Component providers must document whether each of their routines are 
      {\tt single-phase} or {\tt multi-phase}. Single-phase routines require 
      only one invocation to complete their work. Multi-phase routines provide
      multiple subroutines to accomplish the work, accommodating components
      which must complete part of their work, return to the caller and allow 
      other processing to occur, and then continue the original operation. 
      For multiple-phase child components, this is the integer phase number to
      be invoked. For single-phase child components this argument must be 1.
    \item[{[userRc]}]
      Return code set by {\tt userRoutine} before returning.
    \end{description}
   
%/////////////////////////////////////////////////////////////
 
\mbox{}\hrulefill\ 
 
\subsubsection [ESMC\_GridCompGetInternalState] {ESMC\_GridCompGetInternalState - Get the Internal State of a Gridded Component}


  
\bigskip{\sf INTERFACE:}
\begin{verbatim} void *ESMC_GridCompGetInternalState(
   ESMC_GridComp comp,           // in
   int *rc                       // out
 );\end{verbatim}{\em RETURN VALUE:}
\begin{verbatim}    Pointer to private data block that is stored in the internal state.\end{verbatim}
{\sf DESCRIPTION:\\ }


  
    Available to be called by an {\tt ESMC\_GridComp} at any time after 
    {\tt ESMC\_GridCompSetInternalState} has been called. Since init, run, and
    finalize must be separate subroutines, data that they need to share in 
    common can either be global data, or can be allocated in a private data
    block and the address of that block can be registered with the framework 
    and retrieved by this call. When running multiple instantiations of an 
    {\tt ESMC\_GridComp}, for example during ensemble runs, it may be simpler 
    to maintain private data specific to each run with private data blocks. A 
    corresponding {\tt ESMC\_GridCompSetInternalState} call sets the data
    pointer to this block, and this call retrieves the data pointer. 
  
    Only the {\em last} data block set via {\tt ESMC\_GridCompSetInternalState}
    will be accessible. 
  
    The arguments are:
    \begin{description}
    \item[comp]
      An {\tt ESMC\_GridComp} object.
    \item[{[rc]}]
      Return code; equals {\tt ESMF\_SUCCESS} if there are no errors.
   \end{description}
   
%/////////////////////////////////////////////////////////////
 
\mbox{}\hrulefill\ 
 
\subsubsection [ESMC\_GridCompInitialize] {ESMC\_GridCompInitialize - Initialize a Gridded Component}


  
\bigskip{\sf INTERFACE:}
\begin{verbatim} int ESMC_GridCompInitialize(
   ESMC_GridComp comp,           // inout
   ESMC_State importState,       // inout
   ESMC_State exportState,       // inout 
   ESMC_Clock clock,             // in
   int phase,                    // in
   int *userRc                   // out
 );\end{verbatim}{\em RETURN VALUE:}
\begin{verbatim}    Return code; equals ESMF_SUCCESS if there are no errors.\end{verbatim}
{\sf DESCRIPTION:\\ }


  
    Call the associated user initialization code for a GridComp.
  
    The arguments are:
    \begin{description}
    \item[comp]
      {\tt ESMC\_GridComp} to call initialize routine for.
    \item[importState]
      {\tt ESMC\_State} containing import data for coupling.
    \item[exportState]
      {\tt ESMC\_State} containing export data for coupling.
    \item[clock]
      External {\tt ESMC\_Clock} for passing in time information. This is 
      generally the parent component's clock, and will be treated as read-only
      by the child component. The child component can maintain a private clock
      for its own internal time computations.
    \item[phase]
      Component providers must document whether each of their routines are 
      {\tt single-phase} or {\tt multi-phase}. Single-phase routines require 
      only one invocation to complete their work. Multi-phase routines provide
      multiple subroutines to accomplish the work, accommodating components
      which must complete part of their work, return to the caller and allow 
      other processing to occur, and then continue the original operation. 
      For multiple-phase child components, this is the integer phase number to
      be invoked. For single-phase child components this argument must be 1.
    \item[{[userRc]}]
      Return code set by {\tt userRoutine} before returning.
    \end{description}
   
%/////////////////////////////////////////////////////////////
 
\mbox{}\hrulefill\ 
 
\subsubsection [ESMC\_GridCompPrint] {ESMC\_GridCompPrint - Print the contents of a GridComp}


  
\bigskip{\sf INTERFACE:}
\begin{verbatim} int ESMC_GridCompPrint(
   ESMC_GridComp comp     // in
 );\end{verbatim}{\em RETURN VALUE:}
\begin{verbatim}    Return code; equals ESMF_SUCCESS if there are no errors.\end{verbatim}
{\sf DESCRIPTION:\\ }


  
    Prints information about an {\tt ESMC\_GridComp} to {\tt stdout}.
  
    The arguments are:
    \begin{description}
    \item[comp]
      An {\tt ESMC\_GridComp} object.
   \end{description}
   
%/////////////////////////////////////////////////////////////
 
\mbox{}\hrulefill\ 
 
\subsubsection [ESMC\_GridCompRun] {ESMC\_GridCompRun - Run a Gridded Component}


  
\bigskip{\sf INTERFACE:}
\begin{verbatim} int ESMC_GridCompRun(
   ESMC_GridComp comp,           // inout
   ESMC_State importState,       // inout
   ESMC_State exportState,       // inout 
   ESMC_Clock clock,             // in
   int phase,                    // in
   int *userRc                   // out
 );\end{verbatim}{\em RETURN VALUE:}
\begin{verbatim}    Return code; equals ESMF_SUCCESS if there are no errors.\end{verbatim}
{\sf DESCRIPTION:\\ }


  
    Call the associated user run code for a GridComp.
  
    The arguments are:
    \begin{description}
    \item[comp]
      {\tt ESMC\_GridComp} to call run routine for.
    \item[importState]
      {\tt ESMC\_State} containing import data for coupling.
    \item[exportState]
      {\tt ESMC\_State} containing export data for coupling.
    \item[clock]
      External {\tt ESMC\_Clock} for passing in time information. This is 
      generally the parent component's clock, and will be treated as read-only
      by the child component. The child component can maintain a private clock
      for its own internal time computations.
    \item[phase]
      Component providers must document whether each of their routines are 
      {\tt single-phase} or {\tt multi-phase}. Single-phase routines require 
      only one invocation to complete their work. Multi-phase routines provide
      multiple subroutines to accomplish the work, accommodating components
      which must complete part of their work, return to the caller and allow 
      other processing to occur, and then continue the original operation. 
      For multiple-phase child components, this is the integer phase number to
      be invoked. For single-phase child components this argument must be 1.
    \item[{[userRc]}]
      Return code set by {\tt userRoutine} before returning.
    \end{description}
   
%/////////////////////////////////////////////////////////////
 
\mbox{}\hrulefill\ 
 
\subsubsection [ESMC\_GridCompSetEntryPoint] {ESMC\_GridCompSetEntryPoint - Set user routine as entry point for standard Component method}


  
\bigskip{\sf INTERFACE:}
\begin{verbatim} int ESMC_GridCompSetEntryPoint(
   ESMC_GridComp comp,                                               // in
   enum ESMC_Method method,                                          // in
   void (*userRoutine)                                               // in
     (ESMC_GridComp, ESMC_State, ESMC_State, ESMC_Clock *, int *),
   int phase                                                         // in
 );\end{verbatim}{\em RETURN VALUE:}
\begin{verbatim}    Return code; equals ESMF_SUCCESS if there are no errors.\end{verbatim}
{\sf DESCRIPTION:\\ }


  
    Registers a user-supplied {\tt userRoutine} as the entry point for one of 
    the predefined Component methods. After this call the {\tt userRoutine} 
    becomes accessible via the standard Component method API.
  
    The arguments are:
    \begin{description}
    \item[comp]
      An {\tt ESMC\_GridComp} object. 
    \item[method]
    \begin{sloppypar}
      One of a set of predefined Component methods 
      - e.g. {\tt ESMF\_METHOD\_INITIALIZE}, {\tt ESMF\_METHOD\_RUN},
      {\tt ESMF\_METHOD\_FINALIZE}. See section~\ref{const:cmethod}
      for a complete list of valid method options. 
    \end{sloppypar}
    \item[userRoutine]
      The user-supplied subroutine to be associated for this Component 
      {\tt method}. This subroutine does not have to be public. 
    \item[phase]
      The phase number for multi-phase methods.
    \end{description}
   
%/////////////////////////////////////////////////////////////
 
\mbox{}\hrulefill\ 
 
\subsubsection [ESMC\_GridCompSetInternalState] {ESMC\_GridCompSetInternalState - Set the Internal State of a Gridded Component}


  
\bigskip{\sf INTERFACE:}
\begin{verbatim} int ESMC_GridCompSetInternalState(
   ESMC_GridComp comp,           // inout
   void *data                    // in
 );\end{verbatim}{\em RETURN VALUE:}
\begin{verbatim}    Return code; equals ESMF_SUCCESS if there are no errors.\end{verbatim}
{\sf DESCRIPTION:\\ }


  
    Available to be called by an {\tt ESMC\_GridComp} at any time, but
    expected to be most useful when called during the registration process, 
    or initialization. Since init, run, and finalize must be separate
    subroutines, data that they need to share in common can either be global
    data, or can be allocated in a private data block and the address of that 
    block can be registered with the framework and retrieved by subsequent
    calls.
    When running multiple instantiations of an {\tt ESMC\_GridComp}, 
    for example during ensemble runs, it may be simpler to maintain private 
    data specific to each run with private data blocks.  A corresponding 
    {\tt ESMC\_GridCompGetInternalState} call retrieves the data pointer.
     
    Only the {\em last} data block set via
    {\tt ESMC\_GridCompSetInternalState} will be accessible.
  
    The arguments are:
    \begin{description}
    \item[comp]
      An {\tt ESMC\_GridComp} object.
    \item[data]
      Pointer to private data block to be stored.
   \end{description}
   
%/////////////////////////////////////////////////////////////
 
\mbox{}\hrulefill\ 
 
\subsubsection [ESMC\_GridCompSetServices] {ESMC\_GridCompSetServices - Call user routine to register GridComp methods}


  
\bigskip{\sf INTERFACE:}
\begin{verbatim} int ESMC_GridCompSetServices(
   ESMC_GridComp comp,                           // in
   void (*userRoutine)(ESMC_GridComp, int *),    // in
   int *userRc                                   // out
 );\end{verbatim}{\em RETURN VALUE:}
\begin{verbatim}    Return code; equals ESMF_SUCCESS if there are no errors.\end{verbatim}
{\sf DESCRIPTION:\\ }


  
    Call into user provided {\tt userRoutine} which is responsible for setting
    Component's Initialize(), Run() and Finalize() services.
  
    The arguments are:
    \begin{description}
    \item[comp]
      Gridded Component. 
    \item[userRoutine]
      Routine to be called.
    \item[userRc]
      Return code set by {\tt userRoutine} before returning.
    \end{description}
    
    The Component writer must supply a subroutine with the exact interface shown
    above for the {\tt userRoutine} argument.
  
    The {\tt userRoutine}, when called by the framework, must make successive
    calls to {\tt ESMC\_GridCompSetEntryPoint()} to preset callback routines for
    standard Component Initialize(), Run() and Finalize() methods. 
  
%...............................................................
\setlength{\parskip}{\oldparskip}
\setlength{\parindent}{\oldparindent}
\setlength{\baselineskip}{\oldbaselineskip}
