%                **** IMPORTANT NOTICE *****
% This LaTeX file has been automatically produced by ProTeX v. 1.1
% Any changes made to this file will likely be lost next time
% this file is regenerated from its source. Send questions 
% to Arlindo da Silva, dasilva@gsfc.nasa.gov
 
\setlength{\oldparskip}{\parskip}
\setlength{\parskip}{1.5ex}
\setlength{\oldparindent}{\parindent}
\setlength{\parindent}{0pt}
\setlength{\oldbaselineskip}{\baselineskip}
\setlength{\baselineskip}{11pt}
 
%--------------------- SHORT-HAND MACROS ----------------------
\def\bv{\begin{verbatim}}
\def\ev{\end{verbatim}}
\def\be{\begin{equation}}
\def\ee{\end{equation}}
\def\bea{\begin{eqnarray}}
\def\eea{\end{eqnarray}}
\def\bi{\begin{itemize}}
\def\ei{\end{itemize}}
\def\bn{\begin{enumerate}}
\def\en{\end{enumerate}}
\def\bd{\begin{description}}
\def\ed{\end{description}}
\def\({\left (}
\def\){\right )}
\def\[{\left [}
\def\]{\right ]}
\def\<{\left  \langle}
\def\>{\right \rangle}
\def\cI{{\cal I}}
\def\diag{\mathop{\rm diag}}
\def\tr{\mathop{\rm tr}}
%-------------------------------------------------------------

\markboth{Left}{Source File: ESMF\_InternalStateEx.F90,  Date: Tue May  5 21:00:11 MDT 2020
}

 
%/////////////////////////////////////////////////////////////

  \subsubsection{Set and Get the Internal State}  
  
     ESMF provides the concept of an Internal State that is associated with
     a Component. Through the Internal State API a user can attach a private
     data block to a Component, and later retrieve a pointer to this memory
     allocation. Setting and getting of Internal State information are
     supported from anywhere in the Component's SetServices, Initialize, Run,
     or Finalize code.
  
     The code below demonstrates the basic Internal State API
     of {\tt ESMF\_<Grid|Cpl>SetInternalState()} and
     {\tt ESMF\_<Grid|Cpl>GetInternalState()}. Notice that an extra level of
     indirection to the user data is necessary!
   
%/////////////////////////////////////////////////////////////

 \begin{verbatim}
  ! ESMF Framework module
  use ESMF
  use ESMF_TestMod
  implicit none
  
  type(ESMF_GridComp) :: comp
  integer :: rc, finalrc

  ! Internal State Variables
  type testData
  sequence
    integer :: testValue
    real    :: testScaling
  end type

  type dataWrapper
  sequence
    type(testData), pointer :: p
  end type

  type(dataWrapper) :: wrap1, wrap2
  type(testData), target :: data
  type(testData), pointer :: datap  ! extra level of indirection
 
\end{verbatim}
 
%/////////////////////////////////////////////////////////////

 \begin{verbatim}
!-------------------------------------------------------------------------
        
  call ESMF_Initialize(defaultlogfilename="InternalStateEx.Log", &
                    logkindflag=ESMF_LOGKIND_MULTI, rc=rc)
  if (rc /= ESMF_SUCCESS) call ESMF_Finalize(endflag=ESMF_END_ABORT)

!-------------------------------------------------------------------------

  !  Creation of a Component
  comp = ESMF_GridCompCreate(name="test", rc=rc)  
  if (rc .ne. ESMF_SUCCESS) finalrc = ESMF_FAILURE 

!-------------------------------------------------------------------------
! This could be called, for example, during a Component's initialize phase.

    ! Initialize private data block
  data%testValue = 4567
  data%testScaling = 0.5

  ! Set Internal State
  wrap1%p => data
  call ESMF_GridCompSetInternalState(comp, wrap1, rc)
  if (rc .ne. ESMF_SUCCESS) finalrc = ESMF_FAILURE 

!-------------------------------------------------------------------------
! This could be called, for example, during a Component's run phase.

  ! Get Internal State
  call ESMF_GridCompGetInternalState(comp, wrap2, rc)
  if (rc .ne. ESMF_SUCCESS) finalrc = ESMF_FAILURE 

  ! Access private data block and verify data
  datap => wrap2%p 
  if ((datap%testValue .ne. 4567) .or. (datap%testScaling .ne. 0.5)) then
    print *, "did not get same values back"
    finalrc = ESMF_FAILURE
  else
    print *, "got same values back from GetInternalState as original"
  endif

 
\end{verbatim}

%...............................................................
\setlength{\parskip}{\oldparskip}
\setlength{\parindent}{\oldparindent}
\setlength{\baselineskip}{\oldbaselineskip}
