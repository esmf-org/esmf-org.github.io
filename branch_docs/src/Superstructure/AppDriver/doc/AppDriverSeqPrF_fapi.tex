%                **** IMPORTANT NOTICE *****
% This LaTeX file has been automatically produced by ProTeX v. 1.1
% Any changes made to this file will likely be lost next time
% this file is regenerated from its source. Send questions 
% to Arlindo da Silva, dasilva@gsfc.nasa.gov
 
\setlength{\oldparskip}{\parskip}
\setlength{\parskip}{1.5ex}
\setlength{\oldparindent}{\parindent}
\setlength{\parindent}{0pt}
\setlength{\oldbaselineskip}{\baselineskip}
\setlength{\baselineskip}{11pt}
 
%--------------------- SHORT-HAND MACROS ----------------------
\def\bv{\begin{verbatim}}
\def\ev{\end{verbatim}}
\def\be{\begin{equation}}
\def\ee{\end{equation}}
\def\bea{\begin{eqnarray}}
\def\eea{\end{eqnarray}}
\def\bi{\begin{itemize}}
\def\ei{\end{itemize}}
\def\bn{\begin{enumerate}}
\def\en{\end{enumerate}}
\def\bd{\begin{description}}
\def\ed{\end{description}}
\def\({\left (}
\def\){\right )}
\def\[{\left [}
\def\]{\right ]}
\def\<{\left  \langle}
\def\>{\right \rangle}
\def\cI{{\cal I}}
\def\diag{\mathop{\rm diag}}
\def\tr{\mathop{\rm tr}}
%-------------------------------------------------------------

\markboth{Left}{Source File: AppDriver.F90,  Date: Tue May  5 21:00:19 MDT 2020
}

 
%/////////////////////////////////////////////////////////////

  \begin{verbatim}
 
  ---------------------------------------------------------------------------
  ---------------------------------------------------------------------------
   EXAMPLE:  This is an AppDriver.F90 file for a sequential ESMF application.
  ---------------------------------------------------------------------------
  ---------------------------------------------------------------------------
  
    The ChangeMe.F90 file that's included below contains a number of
    definitions that are used by the AppDriver, such as the name of the
    application's main configuration file and the name of the application's
    SetServices routine.  This file is in the same directory as the
    AppDriver.F90 file.
  ---------------------------------------------------------------------------
 
 #include "ChangeMe.F90"
 
     program ESMF_AppDriver
 #define ESMF_METHOD "program ESMF_AppDriver"
 
 #include "ESMF.h"
 
     ! ESMF module, defines all ESMF data types and procedures
     use ESMF
 
     ! Gridded Component registration routines.  Defined in "ChangeMe.F90"
     use USER_APP_Mod, only : SetServices => USER_APP_SetServices
 
     implicit none
 
  ---------------------------------------------------------------------------
    Define local variables
  ---------------------------------------------------------------------------
 
     ! Components and States
     type(ESMF_GridComp) :: compGridded
     type(ESMF_State) :: defaultstate
 
     ! Configuration information
     type(ESMF_Config) :: config
 
     ! A common Grid
     type(ESMF_Grid) :: grid
 
     ! A Clock, a Calendar, and timesteps
     type(ESMF_Clock) :: clock
     type(ESMF_TimeInterval) :: timeStep
     type(ESMF_Time) :: startTime
     type(ESMF_Time) :: stopTime
 
     ! Variables related to the Grid
     integer :: i_max, j_max
 
     ! Return codes for error checks
     integer :: rc, localrc
 
  ---------------------------------------------------------------------------
    Initialize ESMF.  Note that an output Log is created by default.
  ---------------------------------------------------------------------------
 
     call ESMF_Initialize(defaultCalKind=ESMF_CALKIND_GREGORIAN, rc=localrc)
     if (ESMF_LogFoundError(localrc, ESMF_ERR_PASSTHRU, &
         ESMF_CONTEXT, rcToReturn=rc)) &
         call ESMF_Finalize(rc=localrc, endflag=ESMF_END_ABORT)
 
     call ESMF_LogWrite("ESMF AppDriver start", ESMF_LOGMSG_INFO)
 
  ---------------------------------------------------------------------------
    Create and load a configuration file.
    The USER_CONFIG_FILE is set to sample.rc in the ChangeMe.F90 file.
    The sample.rc file is also included in the directory with the
    AppDriver.F90 file.
  ---------------------------------------------------------------------------
 
     config = ESMF_ConfigCreate(rc=localrc)
     if (ESMF_LogFoundError(localrc, ESMF_ERR_PASSTHRU, &
         ESMF_CONTEXT, rcToReturn=rc)) &
         call ESMF_Finalize(rc=localrc, endflag=ESMF_END_ABORT)
 
     call ESMF_ConfigLoadFile(config, USER_CONFIG_FILE, rc = localrc)
     if (ESMF_LogFoundError(localrc, ESMF_ERR_PASSTHRU, &
         ESMF_CONTEXT, rcToReturn=rc)) &
         call ESMF_Finalize(rc=localrc, endflag=ESMF_END_ABORT)
 
  ---------------------------------------------------------------------------
    Get configuration information.
  
    A configuration file like sample.rc might include:
    - size and coordinate information needed to create the default Grid.
    - the default start time, stop time, and running intervals
      for the main time loop.
  ---------------------------------------------------------------------------
 
     call ESMF_ConfigGetAttribute(config, i_max, label='I Counts:', &
       default=10, rc=localrc)
     if (ESMF_LogFoundError(localrc, ESMF_ERR_PASSTHRU, &
         ESMF_CONTEXT, rcToReturn=rc)) &
         call ESMF_Finalize(rc=localrc, endflag=ESMF_END_ABORT)
     call ESMF_ConfigGetAttribute(config, j_max, label='J Counts:', &
       default=40, rc=localrc)
     if (ESMF_LogFoundError(localrc, ESMF_ERR_PASSTHRU, &
         ESMF_CONTEXT, rcToReturn=rc)) &
         call ESMF_Finalize(rc=localrc, endflag=ESMF_END_ABORT)
 
  ---------------------------------------------------------------------------
    Create the top Gridded Component.
  ---------------------------------------------------------------------------
 
     compGridded = ESMF_GridCompCreate(name="ESMF Gridded Component", &
         rc=localrc)
     if (ESMF_LogFoundError(localrc, ESMF_ERR_PASSTHRU, &
         ESMF_CONTEXT, rcToReturn=rc)) &
         call ESMF_Finalize(rc=localrc, endflag=ESMF_END_ABORT)
 
     call ESMF_LogWrite("Component Create finished", ESMF_LOGMSG_INFO)
 
  ----------------------------------------------------------------------------
    Register the set services method for the top Gridded Component.
  ----------------------------------------------------------------------------
 
     call ESMF_GridCompSetServices(compGridded, userRoutine=SetServices, rc=rc)
     if (ESMF_LogFoundError(rc, msg="Registration failed", rcToReturn=rc)) &
         call ESMF_Finalize(rc=localrc, endflag=ESMF_END_ABORT)
 
  ----------------------------------------------------------------------------
    Create and initialize a Clock.
  ----------------------------------------------------------------------------
 
       call ESMF_TimeIntervalSet(timeStep, s=2, rc=localrc)
       if (ESMF_LogFoundError(localrc, ESMF_ERR_PASSTHRU, &
             ESMF_CONTEXT, rcToReturn=rc)) &
             call ESMF_Finalize(rc=localrc, endflag=ESMF_END_ABORT)
 
       call ESMF_TimeSet(startTime, yy=2004, mm=9, dd=25, rc=localrc)
       if (ESMF_LogFoundError(localrc, ESMF_ERR_PASSTHRU, &
             ESMF_CONTEXT, rcToReturn=rc)) &
             call ESMF_Finalize(rc=localrc, endflag=ESMF_END_ABORT)
 
       call ESMF_TimeSet(stopTime, yy=2004, mm=9, dd=26, rc=localrc)
       if (ESMF_LogFoundError(localrc, ESMF_ERR_PASSTHRU, &
             ESMF_CONTEXT, rcToReturn=rc)) &
             call ESMF_Finalize(rc=localrc, endflag=ESMF_END_ABORT)
 
       clock = ESMF_ClockCreate(timeStep, startTime, stopTime=stopTime, &
                 name="Application Clock", rc=localrc)
       if (ESMF_LogFoundError(localrc, ESMF_ERR_PASSTHRU, &
             ESMF_CONTEXT, rcToReturn=rc)) &
             call ESMF_Finalize(rc=localrc, endflag=ESMF_END_ABORT)
 
  ----------------------------------------------------------------------------
    Create and initialize a Grid.
  
    The default lower indices for the Grid are (/1,1/).
    The upper indices for the Grid are read in from the sample.rc file,
    where they are set to (/10,40/).  This means a Grid will be
    created with 10 grid cells in the x direction and 40 grid cells in the
    y direction.  The Grid section in the Reference Manual shows how to set
    coordinates.
  ----------------------------------------------------------------------------
 
       grid = ESMF_GridCreateNoPeriDim(maxIndex=(/i_max, j_max/), &
                              name="source grid", rc=localrc)
       if (ESMF_LogFoundError(localrc, ESMF_ERR_PASSTHRU, &
             ESMF_CONTEXT, rcToReturn=rc)) &
             call ESMF_Finalize(rc=localrc, endflag=ESMF_END_ABORT)
 
       ! Attach the grid to the Component
       call ESMF_GridCompSet(compGridded, grid=grid, rc=localrc)
       if (ESMF_LogFoundError(localrc, ESMF_ERR_PASSTHRU, &
             ESMF_CONTEXT, rcToReturn=rc)) &
             call ESMF_Finalize(rc=localrc, endflag=ESMF_END_ABORT)
 
  ----------------------------------------------------------------------------
    Create and initialize a State to use for both import and export.
    In a real code, separate import and export States would normally be
    created.
  ----------------------------------------------------------------------------
 
       defaultstate = ESMF_StateCreate(name="Default State", rc=localrc)
       if (ESMF_LogFoundError(localrc, ESMF_ERR_PASSTHRU, &
             ESMF_CONTEXT, rcToReturn=rc)) &
             call ESMF_Finalize(rc=localrc, endflag=ESMF_END_ABORT)
 
  ----------------------------------------------------------------------------
    Call the initialize, run, and finalize methods of the top component.
    When the initialize method of the top component is called, it will in
    turn call the initialize methods of all its child components, they
    will initialize their children, and so on.  The same is true of the
    run and finalize methods.
  ----------------------------------------------------------------------------
 
       call ESMF_GridCompInitialize(compGridded, importState=defaultstate, &
         exportState=defaultstate, clock=clock, rc=localrc)
       if (ESMF_LogFoundError(rc, msg="Initialize failed", rcToReturn=rc)) &
           call ESMF_Finalize(rc=localrc, endflag=ESMF_END_ABORT)
 
       call ESMF_GridCompRun(compGridded, importState=defaultstate, &
         exportState=defaultstate, clock=clock, rc=localrc)
       if (ESMF_LogFoundError(rc, msg="Run failed", rcToReturn=rc)) &
           call ESMF_Finalize(rc=localrc, endflag=ESMF_END_ABORT)
 
       call ESMF_GridCompFinalize(compGridded, importState=defaultstate, &
         exportState=defaultstate, clock=clock, rc=localrc)
       if (ESMF_LogFoundError(rc, msg="Finalize failed", rcToReturn=rc)) &
           call ESMF_Finalize(rc=localrc, endflag=ESMF_END_ABORT)
 
 
  ----------------------------------------------------------------------------
    Destroy objects.
  ----------------------------------------------------------------------------
 
       call ESMF_ClockDestroy(clock, rc=localrc)
       if (ESMF_LogFoundError(localrc, ESMF_ERR_PASSTHRU, &
         ESMF_CONTEXT, rcToReturn=rc)) &
         call ESMF_Finalize(rc=localrc, endflag=ESMF_END_ABORT)
 
       call ESMF_StateDestroy(defaultstate, rc=localrc)
       if (ESMF_LogFoundError(localrc, ESMF_ERR_PASSTHRU, &
         ESMF_CONTEXT, rcToReturn=rc)) &
         call ESMF_Finalize(rc=localrc, endflag=ESMF_END_ABORT)
 
       call ESMF_GridCompDestroy(compGridded, rc=localrc)
       if (ESMF_LogFoundError(localrc, ESMF_ERR_PASSTHRU, &
         ESMF_CONTEXT, rcToReturn=rc)) &
         call ESMF_Finalize(rc=localrc, endflag=ESMF_END_ABORT)
 
  ----------------------------------------------------------------------------
    Finalize and clean up.
  ----------------------------------------------------------------------------
 
     call ESMF_Finalize()
 
     end program ESMF_AppDriver
 
  \end{verbatim}
%...............................................................
\setlength{\parskip}{\oldparskip}
\setlength{\parindent}{\oldparindent}
\setlength{\baselineskip}{\oldbaselineskip}
