%
% This is the !DESCRIPTION: entry for the MAPL_HistoryGridComp
%

{\tt MAPL\_HistoryGridCompMod} is an internal MAPL gridded component
used to manange output streams from a MAPL hierarchy. It write Fields
in the Export states of all MAPL components in a hierarchy to file
collections during the course of a run. It also has the some limited
capability to interpolate the fields horizontally and/or vertically
beofore outputing them.

It is usually one of the two gridded components in the ``cap'' or main
program of a MAPL application, the other being the root of the MAPL
hierarchy it is servicing. It is instanciated and all its registered
methods are run automatically by {\tt MAPL\_Cap,} if that is used.  If
writing a custom cap, {\tt MAPL\_HistoryGridCompMod}'s SetServices can
be called anytime after ESMF is initialized.  Its Initialize method
should be executed before entering the time loop, and its Run method
at the bottom of each time loop, after advancing the Clock. Finalize
simply cleans-up memory.

The component has no true export state, since its products are
diagnostic file collections. It does have both Import and Internal
states, which can be treated as in any other MAPL component, but it
generally makes no sense to checkpoint and restart these.

The behavior of {\tt MAPL\_HistoryGridCompMod} is controlled through
its configuration, which as in any MAPL gridded component, is open and
available in the GC. It is placed there by the cap and usually
contained in a HISTORY.rc file.

{\tt MAPL\_HistoryGridCompMod} uses {\tt MAPL\_CFIO} for creating and
writing its files; it thus obeys all {\tt MAPL\_CFIO} rules. In
particular, an application can write either Grads style flat files
together with the Grads .ctl file description files, or one of two
self-describing format (netcdf or HDF), which ever is linked with the
application.

Each collection to be produced is described in the HISTORY.rc file and
can have the following properties:
\begin{itemize}
\item Its fields may be "instantaneous" or "time-averaged", but all fields within
      a collection use the same time discretization. 
\item A beginning and an end time may be specified for each collection .
\item Collections are a set of files with a common name template. 
\item Files in a collection have a fixed number of time groups in them.
\item Data in each time group are "time-stamped"; for time-averaged data,
 the center of the averaging period is used.
\item Files in a collection can have time-templated names. The template
      values correspond to the times on the first group in the file.
\end{itemize}

The body of the HISTORY.rc file usually begins with two character
string attributes under the config labels {\tt EXPID:} and {\tt
  EXPDSC:} that are identifiers for the full set of collections. These
are followed by a list of collection names under the config label {\tt
  COLLECTIONS:}. Note the conventional use of colons to terminate
labels in the HISTORY.rc.

The remainder of the file contains the attributes for each collection.
Attribute labels consist of the attribute name with the collection
name prepended; the two are separated by a '.'.

Attributes are listed below. A special attribute is {\tt {\em
collection}.fields:} which is the label for the list of fields
that will be in the collection.  Each item (line) in the field list
consists of a comma separated list with the field's name (as it
appears in the corresponding ESMF field in the EXPORT of the
component), the name of the component that produces it, and the alias
to use for it in the file. The alias may be omitted, in which case it
defaults to the true name.

Files in a collection are named using the collection name, the
template attribute described below, and the {\tt EXDID:} attribute
value. A filename extension may also be added to identify the type of
file (e.g., .nc4).
\begin{quote}
    {\tt [expid.]collection[.template][.ext]}
\end{quote}
The extension is not added automatically, it is up to the user to add the appropriate one.
If the format is CFIO or CFIOasync and the extension is absent or .nc a NETCDF4 classic file
will be produced. Is the extentions is .nc4 a NETCDF4 file will be produced.
If it is "flat", the data files have whatever extension you provide and
the ``control file'' has the .ctl extension, but with no {\tt
template}. The expid is always prepended, unless it is an empty
string.

The following are the valid collection attributes:
\begin{quote}
\begin{trivlist}
\item[\tt template]      Character string defining the time stamping template that is appended 
                         to {\tt collection} to create a particular file name. 
                         The template uses GrADS convensions. 
                         The default value depends on the {\tt duration} of the file.
\item[\tt descr]         Character string describing the collection. Defaults to "expdsc".
\item[\tt format]        Character string to select file format ("CFIO", "CFIOasync", "flat").  "CFIO" 
                         uses MAPL\_CFIO and produces netcdf output. "CFIOasync" uses MAPL\_CFIO but
                         delegates the actual I/O to the MAPL\_CFIOServer (see MAPL\_CFIOServer documenation for details).
                         Default = "flat".
\item[\tt frequency]     Integer (HHHHMMSS) for the frequency of time groups in the collection.
                         Default = 060000.
\item[\tt mode]          Character string equal to "instantaneous" or "time-averaged".
                         Default = "instantaneous".
\item[\tt acc\_interval] Integer (HHHHMMSS) for the acculation interval (<= frequency)
                         for time-averaged diagnostics.Default = {\tt frequency}; ignored
                         if {\tt mode} is "instantaneous".
\item[\tt ref\_date]     Integer (YYYYMMDD) reference date for {\em frquency};
                         also the beginning date for
                         the collection. Default is the Start date on the Clock.
\item[\tt ref\_time]     Integer (HHMMSS) Same a {\tt ref\_date}.
\item[\tt end\_date]     Integer (YYYYMMDD) ending date to stop diagnostic output.
                         Default: no end
\item[\tt end\_time]     Integer (HHMMSS) ending time to stop diagnostic output.
                         Default: no end.
\item[\tt duration]      Integer (HHHHMMSS) for the duration of each file. 
                         Default = 00000000 (everything in one file).
\item[\tt resolution]    Optional resolution (IM JM) for the ouput stream.
                         Transforms betwee two regulate LogRect grid in index space. 
                         Default is the native resolution.
\item[\tt xyoffset]      Optional Flag for output grid offset when interpolating. Must be
                         between 0 and 3. (Cryptic Meaning: 0:DcPc, 1:DePc, 2:DcPe, 3:DePe).
                         Ignored when {\tt resolution} results in no interpolation (native).
                         Default: 0 (DatelinCenterPoleCenter). 
\item[\tt levels]        Optional list of output levels (Default is all levels on Native Grid).
                         If {\tt vvars} is not specified, these are layer indeces. Otherwise
                         see {\tt vvars, vunits, vscale}.
\item[\tt vvars]         Optional field to use as the vertical coordinate and functional form
                         of vertical interpolation. A second argument specifies 
                         the component the field comes from. 
                         Example 1: the entry 'log(PLE)','DYN' uses PLE from the
                         DYN component as the vertical coordinate and interpolates
                         to {\tt levels} linearly in its log. Example 2: 'THETA','DYN'
                         a way of producing isentropic output.
                         Only {\tt log}($\cdot$), {\tt pow}($\cdot$,{\em real number})
                         and straight linear interpolation are supported.
\item[\tt vunit]         Character string to use for units attribute of the vertical 
                         coordinate in file. 
                         The default is the MAPL\_CFIO default. 
                         This affects only the name in the file.
                         It does not do the conversion. See {\tt vscale}
\item[\tt vscale]        Optional Scaling to convert VVARS units to VUNIT units.
                         Default: no conversion.
\item[\tt regrid\_exch]  Name of the exchange grid that can be used for interpolation
                         between two LogRect grids or from a tile grid to a LogRect grid.
                         Default: no exchange grid interpolation.
                         irregular grid.
\item[\tt regrid\_name]  Name of the Log-Rect grid to interpolate to when going from a tile 
                         to Field to a gridde output. {\tt regrid\_exch} must be set, otherwise
                         it is ignored.
\item[\tt conservative]  Set to a non-zero integer to turn on conservative regridding when going
                         from a native cube-sphere grid to lat-lon output.
                         Default: 0
\item[\tt deflate]       Set deflate level (0-9) of NETCDF output when format is CFIO or CFIOasync.
                         Default: 0
\item[\tt subset]        Optional subset (lonMin lonMax latMin latMax) for the output when performing
                         non-conservative cube-sphere to lat-lon regridding of the output.
\item[\tt chunksize]     Optional user specified chunking of NETCDF output when format is CFIO or CFIOasync, 
                         (Lon chunksize, Lat chunksize, Lev chunksize, Time chunksize)
\end{trivlist}
\end{quote}

The following is a sample HISORY.rc take from the FV\_HeldSuarez test.
\begin{verbatim}
EXPID:  fvhs_example
EXPDSC: fvhs_(ESMF07_EXAMPLE)_5x4_Deg

COLLECTIONS:
      'dynamics_vars_eta'
      'dynamics_vars_p'
      ::

dynamics_vars_eta.template:   '%y4%m2%d2_%h2%n2z',
dynamics_vars_eta.format:     'CFIO',
dynamics_vars_eta.frequency:  240000,
dynamics_vars_eta.duration:   240000,
dynamics_vars_eta.fields:     'T_EQ'     , 'HSPHYSICS'           ,
                              'U'        , 'FVDYNAMICS'          ,
                              'V'        , 'FVDYNAMICS'          ,
                              'T'        , 'FVDYNAMICS'          ,
                              'PLE'      , 'FVDYNAMICS'          ,
                      ::

dynamics_vars_p.template:   '%y4%m2%d2_%h2%n2z',
dynamics_vars_p.format:     'flat',
dynamics_vars_p.frequency:  240000,
dynamics_vars_p.duration:   240000,
dynamics_vars_p.vscale:     100.0,
dynamics_vars_p.vunit:      'hPa',
dynamics_vars_p.vvars:      'log(PLE)' , 'FVDYNAMICS'          ,   
dynamics_vars_p.levels:      1000 900 850 750 500 300 250 150 1,
dynamics_vars_p.fields:     'T_EQ'     , 'HSPHYSICS'           ,
                            'U'        , 'FVDYNAMICS'          ,
                            'V'        , 'FVDYNAMICS'          ,
                            'T'        , 'FVDYNAMICS'          ,
                            'PLE'      , 'FVDYNAMICS'          ,
                      ::
\end{verbatim}

\subsubsection*{BUGS:}

 \begin{enumerate}
 \item It may not be well behaved if more than one instance exists in an application.
 \item Its use for servicing a non-MAPL gridded components is not documented.
 \item GrADS output is currently done through specialized calls, rather than through the 
       CFIO library.
 \item Horizontal and vertical interpolation correctly 
       rely on CFIO, and so are not yet available for GrADS files.
 \item If resolution attribute is used to INCREASE resolution, code may break.   
 \item Grid offsetting is very limited and does not allow for arbitrary rotations
       in longitude 
 \item Currently, this component only handles MAPL supported grids, which currently are 
       the regular lat-lon and the cubed-sphere ESMF grids that tile the entire sphere.

 \end{enumerate}

  
