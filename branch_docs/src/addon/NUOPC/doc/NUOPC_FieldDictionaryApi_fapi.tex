%                **** IMPORTANT NOTICE *****
% This LaTeX file has been automatically produced by ProTeX v. 1.1
% Any changes made to this file will likely be lost next time
% this file is regenerated from its source. Send questions 
% to Arlindo da Silva, dasilva@gsfc.nasa.gov
 
\setlength{\oldparskip}{\parskip}
\setlength{\parskip}{1.5ex}
\setlength{\oldparindent}{\parindent}
\setlength{\parindent}{0pt}
\setlength{\oldbaselineskip}{\baselineskip}
\setlength{\baselineskip}{11pt}
 
%--------------------- SHORT-HAND MACROS ----------------------
\def\bv{\begin{verbatim}}
\def\ev{\end{verbatim}}
\def\be{\begin{equation}}
\def\ee{\end{equation}}
\def\bea{\begin{eqnarray}}
\def\eea{\end{eqnarray}}
\def\bi{\begin{itemize}}
\def\ei{\end{itemize}}
\def\bn{\begin{enumerate}}
\def\en{\end{enumerate}}
\def\bd{\begin{description}}
\def\ed{\end{description}}
\def\({\left (}
\def\){\right )}
\def\[{\left [}
\def\]{\right ]}
\def\<{\left  \langle}
\def\>{\right \rangle}
\def\cI{{\cal I}}
\def\diag{\mathop{\rm diag}}
\def\tr{\mathop{\rm tr}}
%-------------------------------------------------------------

\markboth{Left}{Source File: NUOPC\_FieldDictionaryApi.F90,  Date: Tue May  5 21:00:27 MDT 2020
}

 
%/////////////////////////////////////////////////////////////
\subsubsection [NUOPC\_FieldDictionaryAddEntry] {NUOPC\_FieldDictionaryAddEntry - Add an entry to the NUOPC Field dictionary}


\bigskip{\sf INTERFACE:}
\begin{verbatim}   subroutine NUOPC_FieldDictionaryAddEntry(standardName, canonicalUnits, rc)\end{verbatim}{\em ARGUMENTS:}
\begin{verbatim}     character(*),                 intent(in)            :: standardName
     character(*),                 intent(in)            :: canonicalUnits
     integer,                      intent(out), optional :: rc\end{verbatim}
{\sf DESCRIPTION:\\ }


     Add an entry to the NUOPC Field dictionary. If necessary the dictionary is
     first set up. 
%/////////////////////////////////////////////////////////////
 
\mbox{}\hrulefill\ 
 
\subsubsection [NUOPC\_FieldDictionaryEgest] {NUOPC\_FieldDictionaryEgest - Egest NUOPC Field dictionary into FreeFormat}


\bigskip{\sf INTERFACE:}
\begin{verbatim}   subroutine NUOPC_FieldDictionaryEgest(freeFormat, iofmt, rc)\end{verbatim}{\em ARGUMENTS:}
\begin{verbatim}     type(NUOPC_FreeFormat), intent(out)           :: freeFormat
     type(ESMF_IOFmt_Flag),  intent(in),  optional :: iofmt
     integer,                intent(out), optional :: rc\end{verbatim}
{\sf DESCRIPTION:\\ }


     Egest the contents of the NUOPC Field dictionary into a FreeFormat object.
     If I/O format option {\tt iofmt} is provided and equal to {\tt ESMF\_IOFMT\_YAML},
     the FreeFormat object will contain the NUOPC Field dictionary expressed in YAML
     format. Other values for {\tt iofmt} are ignored and this method behaves as if
     the optional {\tt iofmt} argument were missing. In such a case, {\tt freeFormat}
     will contain NUOPC Field dictionary entries in the traditional format.
     It is the caller's responsibility to destroy the created {\tt freeFormat}
     object. 
%/////////////////////////////////////////////////////////////
 
\mbox{}\hrulefill\ 
 
\subsubsection [NUOPC\_FieldDictionaryGetEntry] {NUOPC\_FieldDictionaryGetEntry - Get information about a NUOPC Field dictionary entry}


\bigskip{\sf INTERFACE:}
\begin{verbatim}   subroutine NUOPC_FieldDictionaryGetEntry(standardName, canonicalUnits, rc)\end{verbatim}{\em ARGUMENTS:}
\begin{verbatim}     character(*),                 intent(in)            :: standardName
     character(*),                 intent(out), optional :: canonicalUnits
     integer,                      intent(out), optional :: rc\end{verbatim}
{\sf DESCRIPTION:\\ }


     Return the canonical units that the NUOPC Field dictionary associates with
     the {\tt standardName}. 
%/////////////////////////////////////////////////////////////
 
\mbox{}\hrulefill\ 
 
\subsubsection [NUOPC\_FieldDictionaryHasEntry] {NUOPC\_FieldDictionaryHasEntry - Check whether the NUOPC Field dictionary has a specific entry}


\bigskip{\sf INTERFACE:}
\begin{verbatim}   function NUOPC_FieldDictionaryHasEntry(standardName, rc)\end{verbatim}{\em RETURN VALUE:}
\begin{verbatim}     logical :: NUOPC_FieldDictionaryHasEntry\end{verbatim}{\em ARGUMENTS:}
\begin{verbatim}     character(*),                 intent(in)            :: standardName
     integer,                      intent(out), optional :: rc\end{verbatim}
{\sf DESCRIPTION:\\ }


     Return {\tt .true.} if the NUOPC Field dictionary has an entry with the
     specified {\tt standardName}, {\tt .false.} otherwise. 
%/////////////////////////////////////////////////////////////
 
\mbox{}\hrulefill\ 
 
\subsubsection [NUOPC\_FieldDictionaryMatchSyno] {NUOPC\_FieldDictionaryMatchSyno - Check whether the NUOPC Field dictionary considers the standard names synonyms}


\bigskip{\sf INTERFACE:}
\begin{verbatim}   function NUOPC_FieldDictionaryMatchSyno(standardName1, standardName2, rc)\end{verbatim}{\em RETURN VALUE:}
\begin{verbatim}     logical :: NUOPC_FieldDictionaryMatchSyno\end{verbatim}{\em ARGUMENTS:}
\begin{verbatim}     character(*),                 intent(in)            :: standardName1
     character(*),                 intent(in)            :: standardName2
     integer,                      intent(out), optional :: rc\end{verbatim}
{\sf DESCRIPTION:\\ }


     Return {\tt .true.} if the NUOPC Field dictionary considers
     {\tt standardName1} and {\tt standardName2} synonyms, {\tt .false.} 
     otherwise. Also, if {\tt standardName1} and/or {\tt standardName2} do not 
     correspond to an existing dictionary entry, {.false.} will be returned. 
%/////////////////////////////////////////////////////////////
 
\mbox{}\hrulefill\ 
 
\subsubsection [NUOPC\_FieldDictionarySetSyno] {NUOPC\_FieldDictionarySetSyno - Set synonyms in the NUOPC Field dictionary}


\bigskip{\sf INTERFACE:}
\begin{verbatim}   subroutine NUOPC_FieldDictionarySetSyno(standardNames, rc)\end{verbatim}{\em ARGUMENTS:}
\begin{verbatim}     character(*),                 intent(in)            :: standardNames(:)
     integer,                      intent(out), optional :: rc\end{verbatim}
{\sf DESCRIPTION:\\ }


     Set all of the elements of the {\tt standardNames} argument to be considered
     synonyms by the field dictionary. Every element in {\tt standardNames} must
     correspond to the standard name of already existing entries in the field 
     dictionary, or else an error will be returned. 
%/////////////////////////////////////////////////////////////
 
\mbox{}\hrulefill\ 
 
\subsubsection [NUOPC\_FieldDictionarySetup] {NUOPC\_FieldDictionarySetup - Setup the default NUOPC Field dictionary}


\bigskip{\sf INTERFACE:}
\begin{verbatim}   ! Private name; call using NUOPC_FieldDictionarySetup()
   subroutine NUOPC_FieldDictionarySetupDefault(rc)\end{verbatim}{\em ARGUMENTS:}
\begin{verbatim}     integer,      intent(out), optional   :: rc\end{verbatim}
{\sf DESCRIPTION:\\ }


     Setup the default NUOPC Field dictionary. 
%/////////////////////////////////////////////////////////////
 
\mbox{}\hrulefill\ 
 
\subsubsection [NUOPC\_FieldDictionarySetup] {NUOPC\_FieldDictionarySetup - Setup the NUOPC Field dictionary from file}


\bigskip{\sf INTERFACE:}
\begin{verbatim}   ! Private name; call using NUOPC_FieldDictionarySetup()
   subroutine NUOPC_FieldDictionarySetupFile(fileName, rc)\end{verbatim}{\em ARGUMENTS:}
\begin{verbatim}     character(len=*),      intent(in)              :: fileName
     integer,               intent(out), optional   :: rc\end{verbatim}
{\sf DESCRIPTION:\\ }


     Setup the NUOPC Field dictionary by reading its content from YAML file.
     If the NUOPC Field dictionary already exists, remove it and create a new one.
     This feature requires ESMF built with YAML support. Please see the
     ESMF User's Guide for details.
%...............................................................
\setlength{\parskip}{\oldparskip}
\setlength{\parindent}{\oldparindent}
\setlength{\baselineskip}{\oldbaselineskip}
