%                **** IMPORTANT NOTICE *****
% This LaTeX file has been automatically produced by ProTeX v. 1.1
% Any changes made to this file will likely be lost next time
% this file is regenerated from its source. Send questions 
% to Arlindo da Silva, dasilva@gsfc.nasa.gov
 
\setlength{\oldparskip}{\parskip}
\setlength{\parskip}{1.5ex}
\setlength{\oldparindent}{\parindent}
\setlength{\parindent}{0pt}
\setlength{\oldbaselineskip}{\baselineskip}
\setlength{\baselineskip}{11pt}
 
%--------------------- SHORT-HAND MACROS ----------------------
\def\bv{\begin{verbatim}}
\def\ev{\end{verbatim}}
\def\be{\begin{equation}}
\def\ee{\end{equation}}
\def\bea{\begin{eqnarray}}
\def\eea{\end{eqnarray}}
\def\bi{\begin{itemize}}
\def\ei{\end{itemize}}
\def\bn{\begin{enumerate}}
\def\en{\end{enumerate}}
\def\bd{\begin{description}}
\def\ed{\end{description}}
\def\({\left (}
\def\){\right )}
\def\[{\left [}
\def\]{\right ]}
\def\<{\left  \langle}
\def\>{\right \rangle}
\def\cI{{\cal I}}
\def\diag{\mathop{\rm diag}}
\def\tr{\mathop{\rm tr}}
%-------------------------------------------------------------

\markboth{Left}{Source File: NUOPC\_Connector.F90,  Date: Tue May  5 21:00:28 MDT 2020
}

 
%/////////////////////////////////////////////////////////////
\subsubsection [NUOPC\_ConnectorGet] {NUOPC\_ConnectorGet - Get parameters from a Connector}


  
\bigskip{\sf INTERFACE:}
\begin{verbatim}   subroutine NUOPC_ConnectorGet(connector, srcFields, dstFields, rh, state, &
     CplSet, cplSetList, srcVM, dstVM, driverClock, rc)\end{verbatim}{\em ARGUMENTS:}
\begin{verbatim}     type(ESMF_CplComp)                            :: connector
     type(ESMF_FieldBundle), intent(out), optional :: srcFields
     type(ESMF_FieldBundle), intent(out), optional :: dstFields
     type(ESMF_RouteHandle), intent(out), optional :: rh
     type(ESMF_State),       intent(out), optional :: state
     character(*),           intent(in),  optional :: CplSet
     character(ESMF_MAXSTR), pointer,     optional :: cplSetList(:)
     type(ESMF_VM),          intent(out), optional :: srcVM
     type(ESMF_VM),          intent(out), optional :: dstVM
     type(ESMF_Clock),       intent(out), optional :: driverClock
     integer,                intent(out), optional :: rc\end{verbatim}
{\sf DESCRIPTION:\\ }


     Get parameters from the {\tt connector} internal state.
  
     The Connector keeps information about the connection that it implements 
     in its internal state. When customizing a Connector, it is often necessary
     to access and sometimes modify these data objects.
  
     The arguments are:
     \begin{description}
     \item[connector]
       The Connector component.
     \item[{[srcFields]}]
       The FieldBundle under which the Connector keeps track of all connected
       source side fields. The order in which the fields are stored
       in {\tt srcFields} is significant, as it corresponds to the order of
       fields in {\tt dstFields}. Consequently, when accessing and modifying
       the fields inside of {\tt srcFields}, it is important to use the
       {\tt itemorderflag=ESMF\_ITEMORDER\_ADDORDER} option to
       {\tt ESMF\_FieldBundleGet()}.
     \item[{[dstFields]}]
       The FieldBundle under which the Connector keeps track of all connected
       destination side fields. The order in which the fields are stored
       in {\tt dstFields} is significant, as it corresponds to the order of
       fields in {\tt srcFields}. Consequently, when accessing and modifying
       the fields inside of {\tt dstFields}, it is important to use the
       {\tt itemorderflag=ESMF\_ITEMORDER\_ADDORDER} option to
       {\tt ESMF\_FieldBundleGet()}.
     \item[{[rh]}]
       The RouteHandle that the Connector uses to move data from {\tt srcFields}
       to {\tt dstFields}.
     \item[{[state]}]
       A State object that the Connector keeps to make customization of the 
       Connector more convenient. The generic Connector code handles creation
       and destruction of {\tt state}, but does {\em not} access it directly 
       for information.
     \item[{[CplSet]}]
       If present, all of the returned information is specific to the specified
       coupling set.
     \item[{[cplSetList]}]
       The list of coupling sets currently known to the Connector. This argument
       must enter the call {\em unassociated} or an error is returned. This means
       that the user code must explicitly call {\tt nullify()} or use the
       {\tt => null()} syntax on the variable passed in as {\tt cplSetList}
       argument. On return, the {\tt cplSetList} argument will be associated, 
       potentially of size zero. The responsibility for deallocation transfers
       to the caller.
     \item[{[srcVM]}]
       The VM of the source side component.
     \item[{[dstVM]}]
       The VM of the destination side component.
     \item[{[driverClock]}]
       The Clock object used by the current RunSequence level to drive this
       component.
     \item[{[rc]}]
       Return code; equals {\tt ESMF\_SUCCESS} if there are no errors.
     \end{description}
   
%/////////////////////////////////////////////////////////////
 
\mbox{}\hrulefill\ 
 
\subsubsection [NUOPC\_ConnectorSet] {NUOPC\_ConnectorSet - Set parameters in a Connector}


  
\bigskip{\sf INTERFACE:}
\begin{verbatim}   subroutine NUOPC_ConnectorSet(connector, srcFields, dstFields, rh, state, &
     CplSet, srcVM, dstVM, rc)\end{verbatim}{\em ARGUMENTS:}
\begin{verbatim}     type(ESMF_CplComp)                            :: connector
     type(ESMF_FieldBundle), intent(in),  optional :: srcFields
     type(ESMF_FieldBundle), intent(in),  optional :: dstFields
     type(ESMF_RouteHandle), intent(in),  optional :: rh
     type(ESMF_State),       intent(in),  optional :: state
     character(*),           intent(in),  optional :: CplSet
     type(ESMF_VM),          intent(in),  optional :: srcVM
     type(ESMF_VM),          intent(in),  optional :: dstVM
     integer,                intent(out), optional :: rc\end{verbatim}
{\sf DESCRIPTION:\\ }


     Set parameters in the {\tt connector} internal state.
  
     The Connector keeps information about the connection that it implements 
     in its internal state. When customizing a Connector, it is often necessary
     to access and sometimes modify these data objects.
  
     The arguments are:
     \begin{description}
     \item[connector]
       The Connector component.
     \item[{[srcFields]}]
       The FieldBundle under which the Connector keeps track of all connected
       source side fields. The order in which the fields are stored
       in {\tt srcFields} is significant, as it corresponds to the order of
       fields in {\tt dstFields}. Consequently, when setting {\tt srcFields}, it
       is important to add them in the same order as for {\tt dstFields}.
     \item[{[dstFields]}]
       The FieldBundle under which the Connector keeps track of all connected
       destination side fields. The order in which the fields are stored
       in {\tt dstFields} is significant, as it corresponds to the order of
       fields in {\tt srcFields}. Consequently, when setting {\tt dstFields}, it
       is important to add them in the same order as for {\tt srcFields}.
     \item[{[rh]}]
       The RouteHandle that the Connector uses to move data from {\tt srcFields}
       to {\tt dstFields}.
     \item[{[state]}]
       A State object that the Connector keeps to make customization of the 
       Connector more convenient. Only in very rare cases would the user want
       to replace the {\tt state} that is managed by the generic Connector
       implementation. If {\tt state} is set by this call, the user essentially
       claims ownership of the previous {\tt state} object, and becomes 
       responsible for its destruction. Ownership of the new {\tt state} is 
       transferred to the Connector and must not be explicitly destroyed by the
       user code.
     \item[{[CplSet]}]
       If present, all of the passed in information is set under the specified
       coupling set.
     \item[{[srcVM]}]
       The VM of the source side component.
     \item[{[dstVM]}]
       The VM of the destination side component.
     \item[{[rc]}]
       Return code; equals {\tt ESMF\_SUCCESS} if there are no errors.
     \end{description}
  
%...............................................................
\setlength{\parskip}{\oldparskip}
\setlength{\parindent}{\oldparindent}
\setlength{\baselineskip}{\oldbaselineskip}
