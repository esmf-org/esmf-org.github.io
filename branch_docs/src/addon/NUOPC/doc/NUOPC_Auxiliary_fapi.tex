%                **** IMPORTANT NOTICE *****
% This LaTeX file has been automatically produced by ProTeX v. 1.1
% Any changes made to this file will likely be lost next time
% this file is regenerated from its source. Send questions 
% to Arlindo da Silva, dasilva@gsfc.nasa.gov
 
\setlength{\oldparskip}{\parskip}
\setlength{\parskip}{1.5ex}
\setlength{\oldparindent}{\parindent}
\setlength{\parindent}{0pt}
\setlength{\oldbaselineskip}{\baselineskip}
\setlength{\baselineskip}{11pt}
 
%--------------------- SHORT-HAND MACROS ----------------------
\def\bv{\begin{verbatim}}
\def\ev{\end{verbatim}}
\def\be{\begin{equation}}
\def\ee{\end{equation}}
\def\bea{\begin{eqnarray}}
\def\eea{\end{eqnarray}}
\def\bi{\begin{itemize}}
\def\ei{\end{itemize}}
\def\bn{\begin{enumerate}}
\def\en{\end{enumerate}}
\def\bd{\begin{description}}
\def\ed{\end{description}}
\def\({\left (}
\def\){\right )}
\def\[{\left [}
\def\]{\right ]}
\def\<{\left  \langle}
\def\>{\right \rangle}
\def\cI{{\cal I}}
\def\diag{\mathop{\rm diag}}
\def\tr{\mathop{\rm tr}}
%-------------------------------------------------------------

\markboth{Left}{Source File: NUOPC\_Auxiliary.F90,  Date: Tue May  5 21:00:28 MDT 2020
}

 
%/////////////////////////////////////////////////////////////
\subsubsection [NUOPC\_Write] {NUOPC\_Write - Write a distributed interpolation matrix to file in SCRIP format}


\bigskip{\sf INTERFACE:}
\begin{verbatim}   ! Private name; call using NUOPC_Write()
   subroutine NUOPC_SCRIPWrite(factorList, factorIndexList, fileName, &
     relaxedflag, rc)\end{verbatim}{\em ARGUMENTS:}
\begin{verbatim}     real(ESMF_KIND_R8), intent(in), target    :: factorList(:)
     integer,            intent(in), target    :: factorIndexList(:,:) 
     character(*),       intent(in)            :: fileName
     logical,            intent(in),  optional :: relaxedflag
     integer,            intent(out), optional :: rc\end{verbatim}
{\sf DESCRIPTION:\\ }


     \label{api_NUOPC_SCRIPWrite}
     Write the destributed interpolaton matrix provided by {\tt factorList} 
     and {\tt factorIndexList} to a SCRIP formatted NetCDF file. Each PET calls
     with its local list of factors and indices. The call then writes the 
     distributed factors into a single file. If the file already exists, the
     contents is replaced by this call.
  
     The arguments are:
     \begin{description}
     \item[factorList]
       The distributed factor list.
     \item[factorIndexList]
       The distributed list of source and destination indices.
     \item[fileName]
       The name of the file to be written to.
     \item[{[relaxedflag]}]
       If {\tt .true.}, then no error is returned even if the call cannot write
       the file due to library limitations. Default is {\tt .false.}.
     \item[{[rc]}]
       Return code; equals {\tt ESMF\_SUCCESS} if there are no errors.
     \end{description}
   
%/////////////////////////////////////////////////////////////
 
\mbox{}\hrulefill\ 
 
\subsubsection [NUOPC\_Write] {NUOPC\_Write - Write a distributed factorList to file}


\bigskip{\sf INTERFACE:}
\begin{verbatim}   ! Private name; call using NUOPC_Write()
   subroutine NUOPC_FactorsWrite(factorList, fileName, rc)\end{verbatim}{\em ARGUMENTS:}
\begin{verbatim}     real(ESMF_KIND_R8), pointer               :: factorList(:)
     character(*),       intent(in)            :: fileName
     integer,            intent(out), optional :: rc\end{verbatim}
{\sf DESCRIPTION:\\ }


  
     THIS METHOD IS DEPRECATED. Use \ref{api_NUOPC_SCRIPWrite} instead.
   
     Write the destributed {\tt factorList} to file. Each PET calls with its 
     local list of factors. The call then writes the distributed factors into
     a single file. The order of the factors in the file is first by PET, and 
     within each PET the PET-local order is preserved. Changing the number of 
     PETs for the same regrid operation will likely change the order of factors
     across PETs, and therefore files written will differ.
  
     The arguments are:
     \begin{description}
     \item[factorList]
       The distributed factor list.
     \item[fileName]
       The name of the file to be written to.
     \item[{[rc]}]
       Return code; equals {\tt ESMF\_SUCCESS} if there are no errors.
     \end{description}
   
%/////////////////////////////////////////////////////////////
 
\mbox{}\hrulefill\ 
 
\subsubsection [NUOPC\_Write] {NUOPC\_Write - Write Field data to file}


\bigskip{\sf INTERFACE:}
\begin{verbatim}   ! Private name; call using NUOPC_Write()
   subroutine NUOPC_FieldWrite(field, fileName, overwrite, status, timeslice, &
     iofmt, relaxedflag, rc)\end{verbatim}{\em ARGUMENTS:}
\begin{verbatim}     type(ESMF_Field),           intent(in)            :: field
     character(*),               intent(in)            :: fileName
     logical,                    intent(in),  optional :: overwrite
     type(ESMF_FileStatus_Flag), intent(in),  optional :: status
     integer,                    intent(in),  optional :: timeslice
     type(ESMF_IOFmt_Flag),      intent(in),  optional :: iofmt
     logical,                    intent(in),  optional :: relaxedflag
     integer,                    intent(out), optional :: rc\end{verbatim}
{\sf DESCRIPTION:\\ }


     Write the data in {\tt field} to {\tt file} under the field's "StandardName" 
     attribute if supported by the {\tt iofmt}.
  
     The arguments are:
     \begin{description}
     \item[field]
       The {\tt ESMF\_Field} object whose data is to be written.
     \item[fileName]
       The name of the file to write to.
     \item[{[overwrite]}]
        A logical flag, the default is .false., i.e., existing Field data may
        {\em not} be overwritten. If .true., the
        data corresponding to each field's name will be
        be overwritten. If the {\tt timeslice} option is given, only data for
        the given timeslice may be overwritten.
        Note that it is always an error to attempt to overwrite a NetCDF
        variable with data which has a different shape.
     \item[{[status]}]
        The file status. Valid options are {\tt ESMF\_FILESTATUS\_NEW}, 
        {\tt ESMF\_FILESTATUS\_OLD}, {\tt ESMF\_FILESTATUS\_REPLACE}, and
        {\tt ESMF\_FILESTATUS\_UNKNOWN} (default).
     \item[{[timeslice]}]
       Time slice counter. Must be positive. The behavior of this
       option may depend on the setting of the {\tt overwrite} flag:
       \begin{description}
       \item[{\tt overwrite = .false.}:]\ If the timeslice value is
       less than the maximum time already in the file, the write will fail.
       \item[{\tt overwrite = .true.}:]\ Any positive timeslice value is valid.
       \end{description}
       By default, i.e. by omitting the {\tt timeslice} argument, no
       provisions for time slicing are made in the output file,
       however, if the file already contains a time axis for the variable,
       a timeslice one greater than the maximum will be written.
     \item[{[iofmt]}]
      The I/O format.  Valid options are  {\tt ESMF\_IOFMT\_BIN} and
      {\tt ESMF\_IOFMT\_NETCDF}. If not present, file names with a {\tt .bin} 
      extension will use {\tt ESMF\_IOFMT\_BIN}, and file names with a {\tt .nc}
      extension will use {\tt ESMF\_IOFMT\_NETCDF}.  Other files default to
      {\tt ESMF\_IOFMT\_NETCDF}.
     \item[{[relaxedflag]}]
       If {\tt .true.}, then no error is returned even if the call cannot write
       the file due to library limitations, or because {\tt field} does not 
       contain any data. Default is {\tt .false.}.
     \item[{[rc]}]
       Return code; equals {\tt ESMF\_SUCCESS} if there are no errors.
     \end{description}
   
%/////////////////////////////////////////////////////////////
 
\mbox{}\hrulefill\ 
 
\subsubsection [NUOPC\_Write] {NUOPC\_Write - Write the Fields within a State to NetCDF files}


\bigskip{\sf INTERFACE:}
\begin{verbatim}   ! Private name; call using NUOPC_Write()
   subroutine NUOPC_StateWrite(state, fieldNameList, fileNamePrefix, overwrite, &
     status, timeslice, relaxedflag, rc)\end{verbatim}{\em ARGUMENTS:}
\begin{verbatim}     type(ESMF_State),           intent(in)            :: state
     character(len=*),           intent(in),  optional :: fieldNameList(:)
     character(len=*),           intent(in),  optional :: fileNamePrefix
     logical,                    intent(in),  optional :: overwrite
     type(ESMF_FileStatus_Flag), intent(in),  optional :: status
     integer,                    intent(in),  optional :: timeslice
     logical,                    intent(in),  optional :: relaxedflag
     integer,                    intent(out), optional :: rc\end{verbatim}
{\sf DESCRIPTION:\\ }


     Write the data of the fields within a {\tt state} to NetCDF files. Each 
     field is written to an individual file using the "StandardName" attribute
     as NetCDF attribute.
  
     The arguments are:
     \begin{description}
     \item[state]
       The {\tt ESMF\_State} object containing the fields.
     \item[{[fieldNameList]}]
       List of names of the fields to be written. By default write all the fields
       in {\tt state}.
     \item[{[fileNamePrefix]}]
       File name prefix, common to all the files written.
     \item[{[overwrite]}]
        A logical flag, the default is .false., i.e., existing Field data may
        {\em not} be overwritten. If .true., the
        data corresponding to each field's name will be
        be overwritten. If the {\tt timeslice} option is given, only data for
        the given timeslice may be overwritten.
        Note that it is always an error to attempt to overwrite a NetCDF
        variable with data which has a different shape.
     \item[{[status]}]
        The file status. Valid options are {\tt ESMF\_FILESTATUS\_NEW}, 
        {\tt ESMF\_FILESTATUS\_OLD}, {\tt ESMF\_FILESTATUS\_REPLACE}, and
        {\tt ESMF\_FILESTATUS\_UNKNOWN} (default).
     \item[{[timeslice]}]
       Time slice counter. Must be positive. The behavior of this
       option may depend on the setting of the {\tt overwrite} flag:
       \begin{description}
       \item[{\tt overwrite = .false.}:]\ If the timeslice value is
       less than the maximum time already in the file, the write will fail.
       \item[{\tt overwrite = .true.}:]\ Any positive timeslice value is valid.
       \end{description}
       By default, i.e. by omitting the {\tt timeslice} argument, no
       provisions for time slicing are made in the output file,
       however, if the file already contains a time axis for the variable,
       a timeslice one greater than the maximum will be written.
     \item[{[relaxedflag]}]
       If {\tt .true.}, then no error is returned even if the call cannot write
       the file due to library limitations. Default is {\tt .false.}.
     \item[{[rc]}]
       Return code; equals {\tt ESMF\_SUCCESS} if there are no errors.
     \end{description}
  
%...............................................................
\setlength{\parskip}{\oldparskip}
\setlength{\parindent}{\oldparindent}
\setlength{\baselineskip}{\oldbaselineskip}
