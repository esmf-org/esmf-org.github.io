\label{ConnectorCompMeta}
The Connector Component metadata is implemented as an ESMF Attribute Package:

\begin{itemize}
    \item Convention: NUOPC
    \item Purpose: General
    \item Includes:
    \begin{itemize}
        \item ESG General (see for example the \htmladdnormallink{Component Attribute packages}{http://www.earthsystemmodeling.org/esmf\_releases/public/ESMF\_5\_2\_0rp2/ESMF\_refdoc/node6.html\#SECTION06022100000000000000} section in the ESMF v5.2.0rp2 documentation)
    \end{itemize} 
\end{itemize}

\begin{longtable}{|p{.22\textwidth}|p{.6\textwidth}|p{.2\textwidth}|}
     \hline\hline
     {\bf Attribute name} & {\bf Definition} & {\bf Controlled vocabulary}\\
     \hline\hline
     {\tt Kind} & String value indicating component kind.& Connector\\ \hline
     {\tt Verbosity} & String value, converted into an integer, and interpreted as a bit field. The lower 16 bits are reserved to control verbosity of the generic component implementation. Higher bits are available for user level verbosity control. \newline
                       {\bf bit 0}: Intro/Extro of methods with indentation.\newline
                       {\bf bit 1}: Intro/Extro with memory info.\newline
                       {\bf bit 2}: Intro/Extro with garbage collection info.\newline
                       {\bf bit 3}: Intro/Extro with local VM info.\newline
                       {\bf bit 8}: Log FieldTransferPolicy.\newline
                       {\bf bit 9}: Log bond level info.\newline
                       {\bf bit 10}: Log CplList construction.\newline
                       {\bf bit 11}: Log GeomObject Transfer.\newline
                       {\bf bit 12}: Log looping over all elements in CplList for RouteHandle computation, FieldSharing, and Timestamp propagation.\newline
                       {\bf bit 13}: Log Run phase with $>>>$, $<<<$, and currTime.\newline
                       {\bf bit 14}: Log info about RouteHandle execution.\newline
                       {\bf bit 15}: Log info about RouteHandle release.
                     & 0, 1, 2, ... \newline
                       "off" = 0 (default), \newline
                       "low": some verbosity, \newline
                       "high": more verbosity, \newline
                       "max": all lower 16 bits\\ \hline
     {\tt Profiling} & String value, converted into an integer, and interpreted as a bit field. The lower 16 bits are reserved to control profiling of the generic component implementation. Higher bits are available for user level profiling control. \newline
                       {\bf bit 0}: Run method execution timings.\newline
                       {\bf bit 1}: Reconcile timings.
                     & 0, 1, 2, ... \newline
                       "max" = set all lower 16 bits\\ \hline
     {\tt Diagnostic} & String value, converted into an integer, and interpreted as a bit field. The lower 16 bits are reserved to control diagnostic of the generic component implementation. Higher bits are available for user level diagnostic control. \newline
                       {\bf bit 0}: Dump fields of the importState on entering the Initialize method.\newline
                       {\bf bit 1}: Dump fields of the exportState on entering the Initialize method.\newline
                       {\bf bit 2}: Dump fields of the importState on exiting the Initialize method.\newline
                       {\bf bit 3}: Dump fields of the exportState on exiting the Initialize method.\newline
                       {\bf bit 4}: Dump fields of the importState on entering the Run method.\newline
                       {\bf bit 5}: Dump fields of the exportState on entering the Run method.\newline
                       {\bf bit 6}: Dump fields of the importState on exiting the Run method.\newline
                       {\bf bit 7}: Dump fields of the exportState on exiting the Run method.\newline
                       {\bf bit 8}: Dump fields of the importState on entering the Finalize method.\newline
                       {\bf bit 9}: Dump fields of the exportState on entering the Finalize method.\newline
                       {\bf bit 10}: Dump fields of the importState on exiting the Finalize method.\newline
                       {\bf bit 11}: Dump fields of the exportState on exiting the Finalize method.
                     & 0, 1, 2, ... \newline
                       "max" = set all lower 16 bits\\ \hline
     {\tt CompLabel} & String value holding the label under which the component was added to its parent driver.& {\em no restriction}\\ \hline
     {\tt InitializePhaseMap} & List of string values, mapping the logical NUOPC initialize phases, of a specific Initialize Phase Definition (IPD) version, to the actual ESMF initialize phase number under which the entry point is registered.& IPDvXXpY=Z, where XX = two-digit revision number, e.g. 01, Y = logical NUOPC phase number, Z = actual ESMF phase number, with Y, Z > 0 and Y, Z < 10 \\ \hline
     {\tt RunPhaseMap} & List of string values, mapping the logical NUOPC run phases to the actual ESMF run phase number under which the entry point is registered.& {\em label-string}=Z, where {\em label-string} can be chosen freely, and Z = actual ESMF phase number. \\ \hline
     {\tt FinalizePhaseMap} & List of string values, mapping the logical NUOPC finalize phases to the actual ESMF finalize phase number under which the entry point is registered.& {\em label-string}=Z, where {\em label-string} can be chosen freely, and Z = actual ESMF phase number. \\ \hline
     {\tt CplList} & List of StandardNames of the connected Fields. Each StandardName entry may be followed by a colon separated list of connection options. The details are discussed in section \ref{connection_options} & {\em Standard names} as per field dictionary, followed by {\em connection options} defined in section \ref{connection_options}.\\ \hline
     {\tt CplSetList} & List of coupling sets. Each coupling set is identified by a string value.& {\em no restriction}\\ \hline
     {\tt ConnectionOptions} & Connection options to be applied to all the fields in the {\tt CplList} by default.& {\em Connection options} defined in section \ref{connection_options}.\\ \hline
     \hline
\end{longtable}
