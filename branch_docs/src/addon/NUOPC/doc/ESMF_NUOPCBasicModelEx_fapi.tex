%                **** IMPORTANT NOTICE *****
% This LaTeX file has been automatically produced by ProTeX v. 1.1
% Any changes made to this file will likely be lost next time
% this file is regenerated from its source. Send questions 
% to Arlindo da Silva, dasilva@gsfc.nasa.gov
 
\setlength{\oldparskip}{\parskip}
\setlength{\parskip}{1.5ex}
\setlength{\oldparindent}{\parindent}
\setlength{\parindent}{0pt}
\setlength{\oldbaselineskip}{\baselineskip}
\setlength{\baselineskip}{11pt}
 
%--------------------- SHORT-HAND MACROS ----------------------
\def\bv{\begin{verbatim}}
\def\ev{\end{verbatim}}
\def\be{\begin{equation}}
\def\ee{\end{equation}}
\def\bea{\begin{eqnarray}}
\def\eea{\end{eqnarray}}
\def\bi{\begin{itemize}}
\def\ei{\end{itemize}}
\def\bn{\begin{enumerate}}
\def\en{\end{enumerate}}
\def\bd{\begin{description}}
\def\ed{\end{description}}
\def\({\left (}
\def\){\right )}
\def\[{\left [}
\def\]{\right ]}
\def\<{\left  \langle}
\def\>{\right \rangle}
\def\cI{{\cal I}}
\def\diag{\mathop{\rm diag}}
\def\tr{\mathop{\rm tr}}
%-------------------------------------------------------------

\markboth{Left}{Source File: ESMF\_NUOPCBasicModelEx.F90,  Date: Tue May  5 21:00:28 MDT 2020
}

 
%/////////////////////////////////////////////////////////////

  \subsection{Example NUOPC Model cap}
  \label{sec:basicexamplecap}
  
   The following code is a starting point for creating a basic NUOPC 
   Model cap.  
    
%/////////////////////////////////////////////////////////////

 \begin{verbatim}
module MYMODEL

  !-----------------------------------------------------------------------------
  ! Basic NUOPC Model cap
  !-----------------------------------------------------------------------------

  use ESMF
  use NUOPC
  use NUOPC_Model, &
    model_routine_SS    => SetServices, &
    model_label_Advance => label_Advance
    
  ! add use statements for your model's initialization
  ! and run subroutines
  
  implicit none
  
  private
  
  public :: SetServices
  
  !-----------------------------------------------------------------------------
  contains
  !-----------------------------------------------------------------------------
  
  subroutine SetServices(model, rc)
    type(ESMF_GridComp)  :: model
    integer, intent(out) :: rc
    
    rc = ESMF_SUCCESS
    
    ! the NUOPC model component will register the generic methods
    call NUOPC_CompDerive(model, model_routine_SS, rc=rc)
    if (ESMF_LogFoundError(rcToCheck=rc, msg=ESMF_LOGERR_PASSTHRU, &
      line=__LINE__, &
      file=__FILE__)) &
      return  ! bail out
    
    ! set entry point for methods that require specific implementation
    call NUOPC_CompSetEntryPoint(model, ESMF_METHOD_INITIALIZE, &
      phaseLabelList=(/"IPDv04p1"/), userRoutine=AdvertiseFields, rc=rc)
    if (ESMF_LogFoundError(rcToCheck=rc, msg=ESMF_LOGERR_PASSTHRU, &
      line=__LINE__, &
      file=__FILE__)) &
      return  ! bail out
    call NUOPC_CompSetEntryPoint(model, ESMF_METHOD_INITIALIZE, &
      phaseLabelList=(/"IPDv04p3"/), userRoutine=RealizeFields, rc=rc)
    if (ESMF_LogFoundError(rcToCheck=rc, msg=ESMF_LOGERR_PASSTHRU, &
      line=__LINE__, &
      file=__FILE__)) &
      return  ! bail out
    
    ! attach specializing method(s)
    call NUOPC_CompSpecialize(model, specLabel=model_label_Advance, &
      specRoutine=ModelAdvance, rc=rc)
    if (ESMF_LogFoundError(rcToCheck=rc, msg=ESMF_LOGERR_PASSTHRU, &
      line=__LINE__, &
      file=__FILE__)) &
      return  ! bail out
    
  end subroutine
  
  !-----------------------------------------------------------------------------

  subroutine AdvertiseFields(model, importState, exportState, clock, rc)
    type(ESMF_GridComp)  :: model
    type(ESMF_State)     :: importState, exportState
    type(ESMF_Clock)     :: clock
    integer, intent(out) :: rc
    
    rc = ESMF_SUCCESS 
    
    ! Eventually, you will advertise your model's import and
    ! export fields in this phase.  For now, however, call
    ! your model's initialization routine(s).
    
    ! call my_model_init()
    
  end subroutine
  
  !-----------------------------------------------------------------------------

  subroutine RealizeFields(model, importState, exportState, clock, rc)
    type(ESMF_GridComp)  :: model
    type(ESMF_State)     :: importState, exportState
    type(ESMF_Clock)     :: clock
    integer, intent(out) :: rc
    
    rc = ESMF_SUCCESS  
    
    ! Eventually, you will realize your model's fields here,
    ! but leave empty for now.

  end subroutine
  
  !-----------------------------------------------------------------------------

  subroutine ModelAdvance(model, rc)
    type(ESMF_GridComp)  :: model
    integer, intent(out) :: rc
    
    ! local variables
    type(ESMF_Clock)              :: clock
    type(ESMF_State)              :: importState, exportState

    rc = ESMF_SUCCESS
    
    ! query the Component for its clock, importState and exportState
    call NUOPC_ModelGet(model, modelClock=clock, importState=importState, &
      exportState=exportState, rc=rc)
    if (ESMF_LogFoundError(rcToCheck=rc, msg=ESMF_LOGERR_PASSTHRU, &
      line=__LINE__, &
      file=__FILE__)) &
      return  ! bail out

    ! HERE THE MODEL ADVANCES: currTime -> currTime + timeStep
    
    ! Because of the way that the internal Clock was set by default,
    ! its timeStep is equal to the parent timeStep. As a consequence the
    ! currTime + timeStep is equal to the stopTime of the internal Clock
    ! for this call of the ModelAdvance() routine.

    call ESMF_ClockPrint(clock, options="currTime", &
      preString="------>Advancing MODEL from: ", rc=rc)
    if (ESMF_LogFoundError(rcToCheck=rc, msg=ESMF_LOGERR_PASSTHRU, &
      line=__LINE__, &
      file=__FILE__)) &
      return  ! bail out
    
    call ESMF_ClockPrint(clock, options="stopTime", &
      preString="--------------------------------> to: ", rc=rc)
    if (ESMF_LogFoundError(rcToCheck=rc, msg=ESMF_LOGERR_PASSTHRU, &
      line=__LINE__, &
      file=__FILE__)) &
      return  ! bail out

    ! Call your model's timestep routine here
    
    ! call my_model_update()
      
  end subroutine

end module

 
\end{verbatim}

%...............................................................
\setlength{\parskip}{\oldparskip}
\setlength{\parindent}{\oldparindent}
\setlength{\baselineskip}{\oldbaselineskip}
