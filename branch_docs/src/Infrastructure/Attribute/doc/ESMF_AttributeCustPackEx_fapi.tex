%                **** IMPORTANT NOTICE *****
% This LaTeX file has been automatically produced by ProTeX v. 1.1
% Any changes made to this file will likely be lost next time
% this file is regenerated from its source. Send questions 
% to Arlindo da Silva, dasilva@gsfc.nasa.gov
 
\setlength{\oldparskip}{\parskip}
\setlength{\parskip}{1.5ex}
\setlength{\oldparindent}{\parindent}
\setlength{\parindent}{0pt}
\setlength{\oldbaselineskip}{\baselineskip}
\setlength{\baselineskip}{11pt}
 
%--------------------- SHORT-HAND MACROS ----------------------
\def\bv{\begin{verbatim}}
\def\ev{\end{verbatim}}
\def\be{\begin{equation}}
\def\ee{\end{equation}}
\def\bea{\begin{eqnarray}}
\def\eea{\end{eqnarray}}
\def\bi{\begin{itemize}}
\def\ei{\end{itemize}}
\def\bn{\begin{enumerate}}
\def\en{\end{enumerate}}
\def\bd{\begin{description}}
\def\ed{\end{description}}
\def\({\left (}
\def\){\right )}
\def\[{\left [}
\def\]{\right ]}
\def\<{\left  \langle}
\def\>{\right \rangle}
\def\cI{{\cal I}}
\def\diag{\mathop{\rm diag}}
\def\tr{\mathop{\rm tr}}
%-------------------------------------------------------------

\markboth{Left}{Source File: ESMF\_AttributeCustPackEx.F90,  Date: Tue May  5 21:00:07 MDT 2020
}

 
%/////////////////////////////////////////////////////////////

   \subsubsection{Custom Attribute package}  \label{ex:AttributeCustPackEx}
  
   This example illustrates how to create a user-defined, custom Attribute 
   package.  The package is created on a gridded Component with three custom
   Attributes. 
%/////////////////////////////////////////////////////////////

      We must construct the ESMF gridded Component object that will be 
      responsible for the custom Attribute package we will be manipulating. 
%/////////////////////////////////////////////////////////////

 \begin{verbatim}
      if (petCount<4) then
        gridcomp = ESMF_GridCompCreate(name="gridded_comp_ex3", &
          petList=(/0/), rc=rc)
      else 
        gridcomp = ESMF_GridCompCreate(name="gridded_comp_ex3", &
          petList=(/0,1,2,3/), rc=rc)
      endif
 
\end{verbatim}
 
%/////////////////////////////////////////////////////////////

      Now we can add a custom Attribute package to the gridded Component object. 
%/////////////////////////////////////////////////////////////

 \begin{verbatim}
      customConv = 'CustomConvention'
      customPurp = 'CustomPurpose'

      customAttrList(1) = 'CustomAttrName1'
      customAttrList(2) = 'CustomAttrName2'
      customAttrList(3) = 'CustomAttrName3'

      call ESMF_AttributeAdd(gridcomp, convention=customConv, &
        purpose=customPurp, attrList=customAttrList, rc=rc)

 
\end{verbatim}
 
%/////////////////////////////////////////////////////////////

 \begin{verbatim}

 
\end{verbatim}
 
%/////////////////////////////////////////////////////////////

       We must set the Attribute values of our custom Attribute package. 
%/////////////////////////////////////////////////////////////

 \begin{verbatim}
    call ESMF_AttributeSet(gridcomp, 'CustomAttrName1', 'CustomAttrValue1', &
      convention=customConv, purpose=customPurp, rc=rc)
 
\end{verbatim}
 
%/////////////////////////////////////////////////////////////

 \begin{verbatim}
    call ESMF_AttributeSet(gridcomp, 'CustomAttrName2', 'CustomAttrValue2', &
      convention=customConv, purpose=customPurp, rc=rc)
 
\end{verbatim}
 
%/////////////////////////////////////////////////////////////

 \begin{verbatim}
    call ESMF_AttributeSet(gridcomp, 'CustomAttrName3', 'CustomAttrValue3', &
      convention=customConv, purpose=customPurp, rc=rc)

 
\end{verbatim}
 
%/////////////////////////////////////////////////////////////

       Write out the contents of our custom Attribute package to an XML file,
       which is generated with a .xml file extension in the execution directory.  
%/////////////////////////////////////////////////////////////

 \begin{verbatim}
      call ESMF_AttributeWrite(gridcomp,customConv,customPurp, &
        attwriteflag=ESMF_ATTWRITE_XML,rc=rc)

 
\end{verbatim}

%...............................................................
\setlength{\parskip}{\oldparskip}
\setlength{\parindent}{\oldparindent}
\setlength{\baselineskip}{\oldbaselineskip}
