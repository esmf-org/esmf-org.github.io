%                **** IMPORTANT NOTICE *****
% This LaTeX file has been automatically produced by ProTeX v. 1.1
% Any changes made to this file will likely be lost next time
% this file is regenerated from its source. Send questions 
% to Arlindo da Silva, dasilva@gsfc.nasa.gov
 
\setlength{\oldparskip}{\parskip}
\setlength{\parskip}{1.5ex}
\setlength{\oldparindent}{\parindent}
\setlength{\parindent}{0pt}
\setlength{\oldbaselineskip}{\baselineskip}
\setlength{\baselineskip}{11pt}
 
%--------------------- SHORT-HAND MACROS ----------------------
\def\bv{\begin{verbatim}}
\def\ev{\end{verbatim}}
\def\be{\begin{equation}}
\def\ee{\end{equation}}
\def\bea{\begin{eqnarray}}
\def\eea{\end{eqnarray}}
\def\bi{\begin{itemize}}
\def\ei{\end{itemize}}
\def\bn{\begin{enumerate}}
\def\en{\end{enumerate}}
\def\bd{\begin{description}}
\def\ed{\end{description}}
\def\({\left (}
\def\){\right )}
\def\[{\left [}
\def\]{\right ]}
\def\<{\left  \langle}
\def\>{\right \rangle}
\def\cI{{\cal I}}
\def\diag{\mathop{\rm diag}}
\def\tr{\mathop{\rm tr}}
%-------------------------------------------------------------

\markboth{Left}{Source File: ESMF\_AttributeInternalInfoEx.F90,  Date: Tue May  5 21:00:07 MDT 2020
}

 
%/////////////////////////////////////////////////////////////

   \subsubsection{Accessing object information through Attribute} 
   \label{ex:AttributeInternalInfoEx}
  
   This example demonstrates the ability to access object information through
   the Attribute class.  This capability is enabled only in the Grid class
   at this point.  Internal Grid information is retrieved through the 
   ESMF\_AttributeGet() interface by specifying the name as a character
   string holding the keyword of the desired piece of Grid information.
   Information that requires input arguments is retrieved by
   specifying the input argument in a character array.
  
   Some examples of this capability are given in this section.  The first
   shows how to get the name of a Grid, and the second shows how
   to get a more complex parameter which requires inputs.  First, we must
   initialize ESMF, declare some variables, and create a Grid: 
%/////////////////////////////////////////////////////////////

 \begin{verbatim}
      ! Use ESMF framework module
      use ESMF
      use ESMF_TestMod
      implicit none

      ! Local variables  
      integer                 :: rc, finalrc, petCount, localPet, result
      type(ESMF_VM)           :: vm
      type(ESMF_Grid)         :: grid
      type(ESMF_DistGrid)     :: distgrid
      character(ESMF_MAXSTR)  :: name
      character(ESMF_MAXSTR),dimension(3) :: inputList 
      integer(ESMF_KIND_I4)  :: exclusiveLBound(2), exclusiveUBound(2)
      integer(ESMF_KIND_I4)  :: exclusiveCount(2)

      character(ESMF_MAXSTR)               :: testname
      character(ESMF_MAXSTR)               :: failMsg
 
\end{verbatim}
 
%/////////////////////////////////////////////////////////////

 \begin{verbatim}
      ! initialize ESMF
      finalrc = ESMF_SUCCESS
      call ESMF_Initialize(vm=vm, &
                defaultlogfilename="AttributeInternalInfoEx.Log", &
                logkindflag=ESMF_LOGKIND_MULTI, rc=rc)

      distgrid=ESMF_DistGridCreate(minIndex=(/1,1/),maxIndex=(/10,10/), rc=rc)
 
\end{verbatim}
 
%/////////////////////////////////////////////////////////////

 \begin{verbatim}
      grid=ESMF_GridCreate(distgrid=distgrid, &
                       coordTypeKind=ESMF_TYPEKIND_I4, &
                       name="AttributeTestGrid", rc=rc)
 
\end{verbatim}
 
%/////////////////////////////////////////////////////////////

    This first call shows how to retrieve the name of a Grid.  The 
    return value is a character string in this case, which must be
    provided as the argument to 'value'.  The 'name' of the Attribute
    is specified as a character string whose value is the keyword of the piece
    of Grid information to retrieve preceded by a special tag.
    This tag, 'ESMF:', tells the ESMF\_AttributeGet() routine that it
    should be looking for class information, rather than an Attribute
    that was previously created with the ESMF\_AttributeSet() call. 
%/////////////////////////////////////////////////////////////

 \begin{verbatim}

  call ESMF_AttributeGet(grid, name="ESMF:name", value=name, rc=rc)

 
\end{verbatim}
 
%/////////////////////////////////////////////////////////////

    This second call demonstrates how to retrieve the exclusiveCount from
    a Grid.  As before, the 'name' of the Attribute is specified as the
    keyword of the information to retrieve, preceded by the 'ESMF:' tag.
    The value is an integer array, which must be allocated to a sufficient
    size to hold all of the requested information.  The exclusiveCount of
    a Grid requires three pieces of input information: localDe, itemflag, and
    staggerloc.  These are specified in an array of character strings.  The
    name of the input parameter is separated from the value by a ':'. 
%/////////////////////////////////////////////////////////////

 \begin{verbatim}
  inputList(:) = ''
  inputList(1) = 'localDe:0'
  inputList(2) = 'itemflag:ESMF_GRIDITEM_MASK'
  inputList(3) = 'staggerloc:ESMF_STAGGERLOC_CENTER'
  call ESMF_AttributeGet(grid, name="ESMF:exclusiveCount", &
                         valueList=exclusiveCount, inputList=inputList, rc=rc)
 
\end{verbatim}
 
%/////////////////////////////////////////////////////////////

    That all there is to it!  Now we just have to Finalize ESMF: 
%/////////////////////////////////////////////////////////////

 \begin{verbatim}
    call ESMF_Finalize(rc=rc)
 
\end{verbatim}

%...............................................................
\setlength{\parskip}{\oldparskip}
\setlength{\parindent}{\oldparindent}
\setlength{\baselineskip}{\oldbaselineskip}
