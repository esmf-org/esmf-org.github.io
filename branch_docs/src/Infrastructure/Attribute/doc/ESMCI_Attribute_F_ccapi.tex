%                **** IMPORTANT NOTICE *****
% This LaTeX file has been automatically produced by ProTeX v. 1.1
% Any changes made to this file will likely be lost next time
% this file is regenerated from its source. Send questions 
% to Arlindo da Silva, dasilva@gsfc.nasa.gov
 
\setlength{\oldparskip}{\parskip}
\setlength{\parskip}{1.5ex}
\setlength{\oldparindent}{\parindent}
\setlength{\parindent}{0pt}
\setlength{\oldbaselineskip}{\baselineskip}
\setlength{\baselineskip}{11pt}
 
%--------------------- SHORT-HAND MACROS ----------------------
\def\bv{\begin{verbatim}}
\def\ev{\end{verbatim}}
\def\be{\begin{equation}}
\def\ee{\end{equation}}
\def\bea{\begin{eqnarray}}
\def\eea{\end{eqnarray}}
\def\bi{\begin{itemize}}
\def\ei{\end{itemize}}
\def\bn{\begin{enumerate}}
\def\en{\end{enumerate}}
\def\bd{\begin{description}}
\def\ed{\end{description}}
\def\({\left (}
\def\){\right )}
\def\[{\left [}
\def\]{\right ]}
\def\<{\left  \langle}
\def\>{\right \rangle}
\def\cI{{\cal I}}
\def\diag{\mathop{\rm diag}}
\def\tr{\mathop{\rm tr}}
%-------------------------------------------------------------

\markboth{Left}{Source File: ESMCI\_Attribute\_F.C,  Date: Tue May  5 21:00:05 MDT 2020
}

 
%/////////////////////////////////////////////////////////////
\subsubsection [c\_ESMC\_AttPackGet] {c\_ESMC\_AttPackGet - get an attpack handle}


  
\bigskip{\sf INTERFACE:}
\begin{verbatim}       void FTN_X(c_esmc_attpackget)(
 #undef  ESMC_METHOD
 #define ESMC_METHOD "c_esmc_attpackget()"\end{verbatim}{\em RETURN VALUE:}
\begin{verbatim}      none.  return code is passed thru the parameter list
   \end{verbatim}{\em ARGUMENTS:}
\begin{verbatim}       ESMC_Base **base,              // in/out - base object
       ESMCI::Attribute **attpack,    // in/out - attpack to return
       int *count,                    // in - number of value(s)
       char *specList,                // in - char string
       int *lens,                     // in - lengths
       ESMC_AttNest_Flag *anflag,     // in - attnest flag
       ESMC_Logical *present,         // out/out - present flag
       int *rc,                       // in - return code
       ESMCI_FortranStrLenArg slen) { // hidden/in - strlen count for specList
   \end{verbatim}
{\sf DESCRIPTION:\\ }


       Retrieve an attribute package from any Attribute containing object.
   
%/////////////////////////////////////////////////////////////
 
\mbox{}\hrulefill\ 
 
\subsubsection [c\_ESMC\_AttPackAddAttribute] {c\_ESMC\_AttPackAddAttribute - add an attribute to an attpack}


  
\bigskip{\sf INTERFACE:}
\begin{verbatim}       void FTN_X(c_esmc_attpackaddatt)(
 #undef  ESMC_METHOD
 #define ESMC_METHOD "c_esmc_attpackaddatt()"\end{verbatim}{\em RETURN VALUE:}
\begin{verbatim}      none.  return code is passed thru the parameter list
   \end{verbatim}{\em ARGUMENTS:}
\begin{verbatim}       char *name,                    // in - F90, non-null terminated string
       ESMCI::Attribute **attpack,            // in - attpack for attributes
       int *rc,                       // in - return code
       ESMCI_FortranStrLenArg nlen) { // hidden/in - strlen count for name
   \end{verbatim}
{\sf DESCRIPTION:\\ }


       Associate a convention, purpose, and object type with an attribute package
   
%/////////////////////////////////////////////////////////////
 
\mbox{}\hrulefill\ 
 
\subsubsection [c\_ESMC\_AttPackAddAttribute] {c\_ESMC\_AttPackAddAttribute - add an attribute to an attpackv}


  
\bigskip{\sf INTERFACE:}
\begin{verbatim}       void FTN_X(c_esmc_attpackaddattribute)(
 #undef  ESMC_METHOD
 #define ESMC_METHOD "c_esmc_attpackaddattribute()"\end{verbatim}{\em RETURN VALUE:}
\begin{verbatim}      none.  return code is passed thru the parameter list\end{verbatim}{\em ARGUMENTS:}
\begin{verbatim}       ESMC_Base **base,              // in/out - base object
       char *name,                    // in - F90, non-null terminated string
       int *count,                    // in - number of value(s)
       char *specList,                // in - char string
       int *lens,                     // in - lengths
       int *rc,                       // in - return code
       ESMCI_FortranStrLenArg nlen,   // hidden/in - strlen count for name
       ESMCI_FortranStrLenArg slen) { // hidden/in - strlen count for specList\end{verbatim}
{\sf DESCRIPTION:\\ }


       Associate a convention, purpose, and object type with an attribute package
   
%/////////////////////////////////////////////////////////////
 
\mbox{}\hrulefill\ 
 
\subsubsection [c\_ESMC\_AttPackCreateCustom] {c\_ESMC\_AttPackCreateCustom - Setup the attribute package}


  
\bigskip{\sf INTERFACE:}
\begin{verbatim}       void FTN_X(c_esmc_attpackcreatecustom)(
 #undef  ESMC_METHOD
 #define ESMC_METHOD "c_esmc_attpackcreatecustom()"\end{verbatim}{\em RETURN VALUE:}
\begin{verbatim}      none.  return code is passed thru the parameter list
   \end{verbatim}{\em ARGUMENTS:}
\begin{verbatim}       ESMC_Base **base,              // in/out - base object
       int *count,                    // in - number of value(s)
       char *specList,                // in - char string
       int *lens,                     // in - lengths
       ESMCI::Attribute **attpack,    // out - attpack created
       int *rc,                       // in - return code
       ESMCI_FortranStrLenArg slen) { // hidden/in - strlen count for specList
   \end{verbatim}
{\sf DESCRIPTION:\\ }


       Associate a convention, purpose, and object type with an attribute package
   
%/////////////////////////////////////////////////////////////
 
\mbox{}\hrulefill\ 
 
\subsubsection [c\_ESMC\_AttPackCreateStandard] {c\_ESMC\_AttPackCreateStandard - Setup the attribute package}


  
\bigskip{\sf INTERFACE:}
\begin{verbatim}       void FTN_X(c_esmc_attpackcreatestandard)(
 #undef  ESMC_METHOD
 #define ESMC_METHOD "c_esmc_attpackcreatestandard()"\end{verbatim}{\em RETURN VALUE:}
\begin{verbatim}      none.  return code is passed thru the parameter list
   \end{verbatim}{\em ARGUMENTS:}
\begin{verbatim}       ESMC_Base **base,              // in/out - base object
       int *count,                    // in - number of value(s)
       char *specList,                // in - char string
       int *lens,                     // in - lengths
       int *rc,                       // in - return code
       ESMCI_FortranStrLenArg slen) { // hidden/in - strlen count for specList
   \end{verbatim}
{\sf DESCRIPTION:\\ }


       Associate a convention, purpose, and object type with an attribute package
   
%/////////////////////////////////////////////////////////////
 
\mbox{}\hrulefill\ 
 
\subsubsection [c\_ESMC\_AttPackNest] {c\_ESMC\_AttPackNest - Setup the attribute package}


  
\bigskip{\sf INTERFACE:}
\begin{verbatim}       void FTN_X(c_esmc_attpacknest)(
 #undef  ESMC_METHOD
 #define ESMC_METHOD "c_esmc_attpacknest()"\end{verbatim}{\em RETURN VALUE:}
\begin{verbatim}      none.  return code is passed thru the parameter list
   \end{verbatim}{\em ARGUMENTS:}
\begin{verbatim}       ESMC_Base **base,               // in/out - base object
       int *count,                     // in - number of value(s)
       char *specList,                 // in - char string
       int *lens,                      // in - lengths
       int  *nestCount,                // in - number of nested attpacks (child nodes)
       char *nestConvention,           // in - nest convention list
       char *nestPurpose,              // in - nest purpose list
       int  *nestConvLens,             // in - length of each nestConvention
       int  *nestPurpLens,             // in - length of each nestPurpose
       int  *rc,                       // in - return code
       ESMCI_FortranStrLenArg slen,    // hidden/in - strlen count for specList
       ESMCI_FortranStrLenArg nclen,   // hidden/in - strlen count for nestConvention
       ESMCI_FortranStrLenArg nplen) { // hidden/in - strlen count for nestPurpose
   \end{verbatim}
{\sf DESCRIPTION:\\ }


       Associate a convention, purpose, and object type with an attribute package
   
%/////////////////////////////////////////////////////////////
 
\mbox{}\hrulefill\ 
 
\subsubsection [c\_ESMC\_AttPackCreateStdNest] {c\_ESMC\_AttPackCreateStdNest - Setup a standard nested attribute package}


  
\bigskip{\sf INTERFACE:}
\begin{verbatim}       void FTN_X(c_esmc_attpackcreatestdnest)(
 #undef  ESMC_METHOD
 #define ESMC_METHOD "c_esmc_attpackcreatestdnest()"\end{verbatim}{\em RETURN VALUE:}
\begin{verbatim}      none.  return code is passed thru the parameter list
   \end{verbatim}{\em ARGUMENTS:}
\begin{verbatim}       ESMC_Base **base,          // in/out - base object
       char *convention,          // in - convention
       char *purpose,             // in - purpose
       char *object,              // in - object type
       char *nestConvention,      // in - nest convention list
       char *nestPurpose,         // in - nest purpose list
       int  *nestConvLens,        // in - length of each nestConvention
       int  *nestPurpLens,        // in - length of each nestPurpose
       int  *nestAttPackInstanceCountList, // in - number of desired instances
                                           //   of each (conv,purp) attpack type
       int  *nestCount,           // in - number of nested attpacks (child nodes)
       char *nestAttPackInstanceNameList,  // out - attpack instance name list
       int  *nestAttPackInstanceNameLens,  // inout - length of each inst name
       int  *nestAttPackInstanceNameSize,  // in - number of elements in 
                                      //      attPackInstanceNameList
       int  *nestAttPackInstanceNameCount, // out - number of attpack 
                                           //   instance names
       int  *rc,                  // in - return code
       ESMCI_FortranStrLenArg clen,// hidden/in - strlen count for convention
       ESMCI_FortranStrLenArg plen,// hidden/in - strlen count for purpose
       ESMCI_FortranStrLenArg olen,// hidden/in - strlen count for object
       ESMCI_FortranStrLenArg nclen,// hidden/in - strlen count for nestConvention
       ESMCI_FortranStrLenArg nplen,// hidden/in - strlen count for nestPurpose
       ESMCI_FortranStrLenArg napinlen) { // hidden/in - strlen count for 
                                          //   nestAttPackInstanceNameList
   \end{verbatim}
{\sf DESCRIPTION:\\ }


       Create a standard nested attpack with a specified number of instances
       of each attpack type (convention,purpose).  Return a list of their names.
   
%/////////////////////////////////////////////////////////////
 
\mbox{}\hrulefill\ 
 
\subsubsection [c\_ESMC\_AttPackRemove] {c\_ESMC\_AttPackRemove - Remove the attribute package}


  
\bigskip{\sf INTERFACE:}
\begin{verbatim}       void FTN_X(c_esmc_attpackremove)(
 #undef  ESMC_METHOD
 #define ESMC_METHOD "c_esmc_attpackremove()"\end{verbatim}{\em RETURN VALUE:}
\begin{verbatim}      none.  return code is passed thru the parameter list
   \end{verbatim}{\em ARGUMENTS:}
\begin{verbatim}       ESMC_Base **base,           // in/out - base object
       ESMCI::Attribute **attpack, // in - attribute package
       int *rc) {                  // in - return code
   \end{verbatim}
{\sf DESCRIPTION:\\ }


      Remove an attribute package
   
%/////////////////////////////////////////////////////////////
 
\mbox{}\hrulefill\ 
 
\subsubsection [c\_ESMC\_AttPackRemoveAttribute] {c\_ESMC\_AttPackRemoveAttribute - Remove an attribute from an}


                                                attribute package
  
\bigskip{\sf INTERFACE:}
\begin{verbatim}       void FTN_X(c_esmc_attpackremoveattribute)(
 #undef  ESMC_METHOD
 #define ESMC_METHOD "c_esmc_attpackremoveattribute()"\end{verbatim}{\em RETURN VALUE:}
\begin{verbatim}      none.  return code is passed thru the parameter list
   \end{verbatim}{\em ARGUMENTS:}
\begin{verbatim}       ESMC_Base **base,              // in/out - base object
       char *name,                    // in - F90, non-null terminated string
       ESMCI::Attribute **attpack,    // in - attribute package
       ESMC_AttNest_Flag *anflag,     // in - attnest flag
       int *rc,                       // in - return code
       ESMCI_FortranStrLenArg nlen) { // hidden/in - strlen count for name
   \end{verbatim}
{\sf DESCRIPTION:\\ }


      Remove an attribute package
   
%/////////////////////////////////////////////////////////////
 
\mbox{}\hrulefill\ 
 
\subsubsection [c\_ESMC\_AttPackGetCharList] {c\_ESMC\_AttPackGetCharList - get attribute from an attpack}


  
\bigskip{\sf INTERFACE:}
\begin{verbatim}       void FTN_X(c_esmc_attpackgetcharlist)(
 #undef  ESMC_METHOD
 #define ESMC_METHOD "c_esmc_attpackgetcharlist()"\end{verbatim}{\em RETURN VALUE:}
\begin{verbatim}      none.  return code is passed thru the parameter list
   \end{verbatim}{\em ARGUMENTS:}
\begin{verbatim}       ESMC_Base **base,              // in/out - base object
       char *name,                    // in - F90, non-null terminated string
       ESMCI::Attribute **attpack,    // in - Attribute package
       ESMC_TypeKind_Flag *tk,        // in - typekind
       int *count,                    // in - must match actual length
       ESMC_AttNest_Flag *anflag,     // in - attnest flag
       int *lens,                     // in/out - length of strings
       char *valueList,               // out - character values
       int *rc,                       // in - return code
       ESMCI_FortranStrLenArg nlen,   // hidden/in - strlen count for name
       ESMCI_FortranStrLenArg vlen) { // hidden/in - strlen count for value
   \end{verbatim}
{\sf DESCRIPTION:\\ }


       Retrieve a (name,value) pair from any object type in the system.
   
%/////////////////////////////////////////////////////////////
 
\mbox{}\hrulefill\ 
 
\subsubsection [c\_ESMC\_AttPackGetValue] {c\_ESMC\_AttPackGetValue - get attribute from an attpack}


  
\bigskip{\sf INTERFACE:}
\begin{verbatim}       void FTN_X(c_esmc_attpackgetvalue)(
 #undef  ESMC_METHOD
 #define ESMC_METHOD "c_esmc_attpackgetvalue()"\end{verbatim}{\em RETURN VALUE:}
\begin{verbatim}      none.  return code is passed thru the parameter list
   \end{verbatim}{\em ARGUMENTS:}
\begin{verbatim}       ESMC_Base **base,              // in/out - base object
       char *name,                    // in - F90, non-null terminated string
       ESMCI::Attribute **attpack,    // in - attribute package
       ESMC_TypeKind_Flag *tk,        // in - typekind
       int *count,                    // in - must match actual length
       ESMC_AttNest_Flag *anflag,     // in - attnest flag
       void *value,                   // out - value
       int *rc,                       // in/out - return code
       ESMCI_FortranStrLenArg nlen) { // hidden/in - strlen count for name
   \end{verbatim}
{\sf DESCRIPTION:\\ }


       Return the (name,value) pair from any object type in the system.
   
%/////////////////////////////////////////////////////////////
 
\mbox{}\hrulefill\ 
 
\subsubsection [c\_ESMC\_AttPackGetAPinstNames] {c\_ESMC\_AttPackGetAPinstNames - get attpack instance names}


  
\bigskip{\sf INTERFACE:}
\begin{verbatim}       void FTN_X(c_esmc_attpackgetapinstnames)(
 #undef  ESMC_METHOD
 #define ESMC_METHOD "c_esmc_attpackgetapinstnames()"\end{verbatim}{\em RETURN VALUE:}
\begin{verbatim}      none.  return code is passed thru the parameter list
   \end{verbatim}{\em ARGUMENTS:}
\begin{verbatim}       ESMC_Base **base,                  // in/out - base object
       ESMCI::Attribute **attpack,        // in - attribute package
       char *attPackInstanceNameList,     // out - attpack instance names
       int *attPackInstanceNameLens,      // inout - lengths of attpack inst names
       int *attPackInstanceNameSize,      // in - number of elements in 
                                          //      attPackInstanceNameList
       int *attPackInstanceNameCount,     // out - number of attpack instance names
       ESMC_AttNest_Flag *anflag,         // in - attnest flag
       int *rc,                           // in - return code
       ESMCI_FortranStrLenArg napinlen) { // hidden/in - strlen count for attPackInstanceNameList
   \end{verbatim}
{\sf DESCRIPTION:\\ }


       Return the attpack instance names for (convention,purpose)
   
%/////////////////////////////////////////////////////////////
 
\mbox{}\hrulefill\ 
 
\subsubsection [c\_ESMC\_AttPackIsPresent] {c\_ESMC\_AttPackIsPresent - Query for an Attribute package Attribute}


  
\bigskip{\sf INTERFACE:}
\begin{verbatim}       void FTN_X(c_esmc_attpackispresent)(
 #undef  ESMC_METHOD
 #define ESMC_METHOD "c_esmc_attpackispresent()"\end{verbatim}{\em RETURN VALUE:}
\begin{verbatim}      none.  return code is passed thru the parameter list\end{verbatim}{\em ARGUMENTS:}
\begin{verbatim}       ESMC_Base **base,              // in/out - base object
       ESMCI::Attribute **attpack,    // in - Attribute package
       ESMC_Logical *present,         // out/out - present flag
       int *rc) {                     // out - return code
 \end{verbatim}
{\sf DESCRIPTION:\\ }


       Query an Attribute package for the presence of an Attribute.
   
%/////////////////////////////////////////////////////////////
 
\mbox{}\hrulefill\ 
 
\subsubsection [c\_ESMC\_AttPackIsPresentAtt] {c\_ESMC\_AttPackIsPresentAtt - Query for an Attribute package Attribute}


  
\bigskip{\sf INTERFACE:}
\begin{verbatim}       void FTN_X(c_esmc_attpackispresentatt)(
 #undef  ESMC_METHOD
 #define ESMC_METHOD "c_esmc_attpackispresentatt()"\end{verbatim}{\em RETURN VALUE:}
\begin{verbatim}      none.  return code is passed thru the parameter list
   \end{verbatim}{\em ARGUMENTS:}
\begin{verbatim}       ESMC_Base **base,              // in/out - base object
       char *name,                    // in - F90, non-null terminated string
       ESMCI::Attribute **attpack,    // in - Attribute package
       ESMC_Logical *present,         // out/out - present flag 
       int *rc,                       // in/out - return code
       ESMCI_FortranStrLenArg nlen) { // hidden/in - strlen count for name
 
   \end{verbatim}
{\sf DESCRIPTION:\\ }


       Query an Attribute package for the presence of an Attribute.
   
%/////////////////////////////////////////////////////////////
 
\mbox{}\hrulefill\ 
 
\subsubsection [c\_ESMC\_AttPackIsPresentIndex] {c\_ESMC\_AttPackIsPresentIndex - Query for an Attribute package}


                                              Attribute by Index
  
\bigskip{\sf INTERFACE:}
\begin{verbatim}       void FTN_X(c_esmc_attpackispresentindex)(
 #undef  ESMC_METHOD
 #define ESMC_METHOD "c_esmc_attpackispresentindex()"\end{verbatim}{\em RETURN VALUE:}
\begin{verbatim}      none.  return code is passed thru the parameter list
   \end{verbatim}{\em ARGUMENTS:}
\begin{verbatim}       ESMC_Base **base,              // in/out - base object
       int *num,                      // in - position of attribute to check
       ESMCI::Attribute **attpack,    // in - Attribute package
       ESMC_Logical *present,         // out/out - present flag
       int *rc) {                     // in/out - return code
 \end{verbatim}
{\sf DESCRIPTION:\\ }


       Query an Attribute package for the presence of an Attribute.
   
%/////////////////////////////////////////////////////////////
 
\mbox{}\hrulefill\ 
 
\subsubsection [c\_ESMC\_AttributeMove] {c\_ESMC\_AttributeMove - Move an attribute between objects}


  
\bigskip{\sf INTERFACE:}
\begin{verbatim}       void FTN_X(c_esmc_attributemove)(
 #undef  ESMC_METHOD
 #define ESMC_METHOD "c_esmc_attributemove()"\end{verbatim}{\em RETURN VALUE:}
\begin{verbatim}      none.  return code is passed thru the parameter list\end{verbatim}{\em ARGUMENTS:}
\begin{verbatim}       ESMC_Base **source,              // in/out - base object
       ESMC_Base **destination,         // in/out - base object
       int *rc) {                       // in/out - return code
   \end{verbatim}
{\sf DESCRIPTION:\\ }


       Swap the Attribute hierarchy from Base1 to Base2
   
%/////////////////////////////////////////////////////////////
 
\mbox{}\hrulefill\ 
 
\subsubsection [c\_esmc\_attpacksetcharlist] {c\_esmc\_attpacksetcharlist - Set attributes in the attribute package}


  
\bigskip{\sf INTERFACE:}
\begin{verbatim}       void FTN_X(c_esmc_attpacksetcharlist)(
 #undef  ESMC_METHOD
 #define ESMC_METHOD "c_esmc_attpacksetcharlist()"\end{verbatim}{\em RETURN VALUE:}
\begin{verbatim}      none.  return code is passed thru the parameter list
   \end{verbatim}{\em ARGUMENTS:}
\begin{verbatim}       ESMC_Base **base,              // in/out - base object
       char *name,                    // in - F90, non-null terminated string
       ESMC_TypeKind_Flag *tk,        // in - typekind
       int *count,                    // in - number of items
       char *valueList,               // in - F90, non-null terminated string
       int *lens,                     // in - length of the char*s
       ESMCI::Attribute **attpack,    // in - attribute package
       ESMC_AttNest_Flag *anflag,     // in - attnest flag
       int *rc,                       // in - return code
       ESMCI_FortranStrLenArg nlen,  // hidden/in - strlen count for name
       ESMCI_FortranStrLenArg vlen) { // hidden/in - strlen count for value
   \end{verbatim}
{\sf DESCRIPTION:\\ }


       Set the convention, purpose, and object type on an attribute package
   
%/////////////////////////////////////////////////////////////
 
\mbox{}\hrulefill\ 
 
\subsubsection [c\_esmc\_attpacksetvalue] {c\_esmc\_attpacksetvalue - Set attributes in the attribute package}


  
\bigskip{\sf INTERFACE:}
\begin{verbatim}       void FTN_X(c_esmc_attpacksetvalue)(
 #undef  ESMC_METHOD
 #define ESMC_METHOD "c_esmc_attpacksetvalue()"\end{verbatim}{\em RETURN VALUE:}
\begin{verbatim}      none.  return code is passed thru the parameter list
   \end{verbatim}{\em ARGUMENTS:}
\begin{verbatim}       ESMC_Base **base,              // in/out - base object
       char *name,                    // in - F90, non-null terminated string
       ESMC_TypeKind_Flag *tk,        // in - typekind
       int *count,                    // in - item count
       void *value,                   // in - F90, non-null terminated string
       ESMCI::Attribute **attpack,    // in - attribute package
       ESMC_AttNest_Flag *anflag,     // in - attnest flag
       int *rc,                       // in - return code
       ESMCI_FortranStrLenArg nlen) { // hidden/in - strlen count for name
   \end{verbatim}
{\sf DESCRIPTION:\\ }


       Set the convention, purpose, and object type on an attribute package
   
%/////////////////////////////////////////////////////////////
 
\mbox{}\hrulefill\ 
 
\subsubsection [c\_ESMC\_AttributeCopy] {c\_ESMC\_AttributeCopy - copy an attribute between objects}


  
\bigskip{\sf INTERFACE:}
\begin{verbatim}       void FTN_X(c_esmc_attributecopy)(
 #undef  ESMC_METHOD
 #define ESMC_METHOD "c_esmc_attributecopy()"\end{verbatim}{\em RETURN VALUE:}
\begin{verbatim}      none.  return code is passed thru the parameter list
   \end{verbatim}{\em ARGUMENTS:}
\begin{verbatim}       ESMC_Base **source,              // in/out - base object
       ESMC_Base **destination,         // in/out - base object
       ESMC_AttCopyFlag *attcopyflag,   // in - attcopyflag
       int *rc) {                       // in/out - return code
   \end{verbatim}
{\sf DESCRIPTION:\\ }


       Copy the Attribute hierarchy from Base1 to Base2
   
%/////////////////////////////////////////////////////////////
 
\mbox{}\hrulefill\ 
 
\subsubsection [c\_ESMC\_AttributeGetCharList] {c\_ESMC\_AttributeGetCharList - get attribute list from an ESMF type}


  
\bigskip{\sf INTERFACE:}
\begin{verbatim}       void FTN_X(c_esmc_attributegetcharlist)(
 #undef  ESMC_METHOD
 #define ESMC_METHOD "c_esmc_attributegetcharlist()"\end{verbatim}{\em RETURN VALUE:}
\begin{verbatim}      none.  return code is passed thru the parameter list
   \end{verbatim}{\em ARGUMENTS:}
\begin{verbatim}       ESMC_Base **base,              // in/out - base object
       char *name,                    // in - F90, non-null terminated string
       ESMC_TypeKind_Flag *tk,        // in - typekind
       int *count,                    // in - must match actual length
       int *lens,                     // in/out - length of strings
       char *valueList,               // out - character values
       int *rc,                       // out - return code
       ESMCI_FortranStrLenArg nlen,   // hidden/in - strlen count for name
       ESMCI_FortranStrLenArg vlen) { // hidden/in - strlen count for valueList
   \end{verbatim}
{\sf DESCRIPTION:\\ }


       Retrieve a (name,value) pair from any object type in the system.
   
%/////////////////////////////////////////////////////////////
 
\mbox{}\hrulefill\ 
 
\subsubsection [c\_ESMC\_AttributeGetValue] {c\_ESMC\_AttributeGetValue - get attribute from an ESMF type}


  
\bigskip{\sf INTERFACE:}
\begin{verbatim}       void FTN_X(c_esmc_attributegetvalue)(
 #undef  ESMC_METHOD
 #define ESMC_METHOD "c_esmc_attributegetvalue()"\end{verbatim}{\em RETURN VALUE:}
\begin{verbatim}      none.  return code is passed thru the parameter list
   \end{verbatim}{\em ARGUMENTS:}
\begin{verbatim}       ESMC_Base **base,              // in/out - base object
       char *name,                    // in - F90, non-null terminated string
       ESMC_TypeKind_Flag *tk,        // in - typekind
       int *items,                    // in - must match actual length
       void *value,                   // out - value
       int *rc,                       // out - return code
       ESMCI_FortranStrLenArg nlen) { // hidden/in - strlen count for name
   \end{verbatim}
{\sf DESCRIPTION:\\ }


       Return the (name,value) pair from any object type in the system.
   
%/////////////////////////////////////////////////////////////
 
\mbox{}\hrulefill\ 
 
\subsubsection [c\_ESMC\_AttPackGetInfoName] {c\_ESMC\_AttPackGetInfoName - get type and number of items in an attpackattr}


  
\bigskip{\sf INTERFACE:}
\begin{verbatim}       void FTN_X(c_esmc_attpackgetinfoname)(
 #undef  ESMC_METHOD
 #define ESMC_METHOD "c_esmc_attpackgetinfoname()"\end{verbatim}{\em RETURN VALUE:}
\begin{verbatim}      none.  return code is passed thru the parameter list
   \end{verbatim}{\em ARGUMENTS:}
\begin{verbatim}       ESMC_Base **base,              // in/out - base object
       char *name,                    // in - F90, non-null terminated string
       ESMCI::Attribute **attpack,    // in - Attribute package
       ESMC_AttNest_Flag *anflag,     // in - attnest flag
       ESMC_TypeKind_Flag *tk,        // out - typekind
       int *count,                    // out - item count
       int *rc,                       // out - return code
       ESMCI_FortranStrLenArg nlen) { // hidden/in - strlen count for name
   \end{verbatim}
{\sf DESCRIPTION:\\ }


     Return the typekind, count of items in the (name,value) pair from any 
     object type in the system.
   
%/////////////////////////////////////////////////////////////
 
\mbox{}\hrulefill\ 
 
\subsubsection [c\_ESMC\_AttributeGetInfoName] {c\_ESMC\_AttributeGetInfoName - get type and number of items in an attr}


  
\bigskip{\sf INTERFACE:}
\begin{verbatim}       void FTN_X(c_esmc_attributegetinfoname)(
 #undef  ESMC_METHOD
 #define ESMC_METHOD "c_esmc_attributegetinfoname()"\end{verbatim}{\em RETURN VALUE:}
\begin{verbatim}      none.  return code is passed thru the parameter list
   \end{verbatim}{\em ARGUMENTS:}
\begin{verbatim}       ESMC_Base **base,         // in/out - base object
       char *name,               // in - F90, non-null terminated string
       ESMC_TypeKind_Flag *tk,   // out - typekind
       int *count,               // out - item count
       int *rc,                  // in - return code
       ESMCI_FortranStrLenArg nlen) { // hidden/in - strlen count for name
   \end{verbatim}
{\sf DESCRIPTION:\\ }


     Return the typekind, count of items in the (name,value) pair from any 
     object type in the system.
   
%/////////////////////////////////////////////////////////////
 
\mbox{}\hrulefill\ 
 
\subsubsection [c\_ESMC\_AttPackGetInfoNum] {c\_ESMC\_AttPackGetInfoNum - get type and number of items in an attpack}


  
\bigskip{\sf INTERFACE:}
\begin{verbatim}       void FTN_X(c_esmc_attpackgetinfonum)(
 #undef  ESMC_METHOD
 #define ESMC_METHOD "c_esmc_attpackgetinfonum()"\end{verbatim}{\em RETURN VALUE:}
\begin{verbatim}      none.  return code is passed thru the parameter list
   \end{verbatim}{\em ARGUMENTS:}
\begin{verbatim}       ESMC_Base **base,              // in/out - base object
       int *num,                      // in - attr number
       ESMCI::Attribute **attpack,    // in - Attribute package
       char *name,                    // out - F90, non-null terminated string
       ESMC_AttNest_Flag *anflag,     // in - attnest flag
       ESMC_TypeKind_Flag *tk,        // out - typekind
       int *count,                    // out - item count
       int *rc,                       // in - return code
       ESMCI_FortranStrLenArg nlen) { // hidden/in - strlen count for name
   \end{verbatim}
{\sf DESCRIPTION:\\ }


     Return the name, type, count of items in the (name,value) pair from any 
     object type in the system.
   
%/////////////////////////////////////////////////////////////
 
\mbox{}\hrulefill\ 
 
\subsubsection [c\_ESMC\_AttributeGetInfoNum] {c\_ESMC\_AttributeGetInfoNum - get type and number of items in an attr}


  
\bigskip{\sf INTERFACE:}
\begin{verbatim}       void FTN_X(c_esmc_attributegetinfonum)(
 #undef  ESMC_METHOD
 #define ESMC_METHOD "c_esmc_attributegetinfonum()"\end{verbatim}{\em RETURN VALUE:}
\begin{verbatim}      none.  return code is passed thru the parameter list
   \end{verbatim}{\em ARGUMENTS:}
\begin{verbatim}       ESMC_Base **base,         // in/out - base object
       int *num,                 // in - attr number
       char *name,               // out - F90, non-null terminated string
       ESMC_TypeKind_Flag *tk,        // out - typekind
       int *count,               // out - item count
       int *rc,                  // in - return code
       ESMCI_FortranStrLenArg nlen) { // hidden/in - strlen count for name
   \end{verbatim}
{\sf DESCRIPTION:\\ }


     Return the name, type, count of items in the (name,value) pair from any
     object type in the system.
   
%/////////////////////////////////////////////////////////////
 
\mbox{}\hrulefill\ 
 
\subsubsection [c\_ESMC\_AttributeGetCount] {c\_ESMC\_AttributeGetCount - get number of attrs}


  
\bigskip{\sf INTERFACE:}
\begin{verbatim}       void FTN_X(c_esmc_attributegetcount)(
 #undef  ESMC_METHOD
 #define ESMC_METHOD "c_esmc_attributegetcount()"\end{verbatim}{\em RETURN VALUE:}
\begin{verbatim}      none.  return code is passed thru the parameter list\end{verbatim}{\em ARGUMENTS:}
\begin{verbatim}       ESMC_Base **base,              // in/out - base object
       int *count,                    // out - attribute count
       ESMC_AttGetCountFlag *gcflag,  // in - attgetcount flag
       int *rc) {                     // out - return code\end{verbatim}
{\sf DESCRIPTION:\\ }


     Return the count of attributes for any object type in the system.
   
%/////////////////////////////////////////////////////////////
 
\mbox{}\hrulefill\ 
 
\subsubsection [c\_ESMC\_AttributeGetCountAttPack] {c\_ESMC\_AttributeGetCountAttPack - get number of attrs}


  
\bigskip{\sf INTERFACE:}
\begin{verbatim}       void FTN_X(c_esmc_attributegetcountattpack)(
 #undef  ESMC_METHOD
 #define ESMC_METHOD "c_esmc_attributegetcountattpack()"\end{verbatim}{\em RETURN VALUE:}
\begin{verbatim}      none.  return code is passed thru the parameter list\end{verbatim}{\em ARGUMENTS:}
\begin{verbatim}       ESMCI::Attribute **attpack,    // in/out - attpack object
       int *count,                    // out - attribute count
       ESMC_AttGetCountFlag *gcflag,  // in - attgetcount flag
       ESMC_AttNest_Flag *anflag,     // in - attnest flag
       int *rc) {                     // out - return code\end{verbatim}
{\sf DESCRIPTION:\\ }


     Return the count of attributes for an attpack.
   
%/////////////////////////////////////////////////////////////
 
\mbox{}\hrulefill\ 
 
\subsubsection [c\_ESMC\_AttributeIsPresent] {c\_ESMC\_AttributeIsPresent - Query for an Attribute}


  
\bigskip{\sf INTERFACE:}
\begin{verbatim}       void FTN_X(c_esmc_attributeispresent)(
 #undef  ESMC_METHOD
 #define ESMC_METHOD "c_esmc_attributeispresent()"\end{verbatim}{\em RETURN VALUE:}
\begin{verbatim}      none.  return code is passed thru the parameter list\end{verbatim}{\em ARGUMENTS:}
\begin{verbatim}       ESMC_Base **base,          // in/out - base object
       char *name,                // in - F90, non-null terminated string
       ESMC_Logical *present,     // out/out - present flag
       int *rc,                   // in/out - return code
       ESMCI_FortranStrLenArg nlen) { // hidden/in - strlen count for name\end{verbatim}
{\sf DESCRIPTION:\\ }


       Query for the presence of an Attribute.
   
%/////////////////////////////////////////////////////////////
 
\mbox{}\hrulefill\ 
 
\subsubsection [c\_ESMC\_AttributeIsPresentIndex] {c\_ESMC\_AttributeIsPresentIndex - Query for an Attribute by index}


  
\bigskip{\sf INTERFACE:}
\begin{verbatim}       void FTN_X(c_esmc_attributeispresentindex)(
 #undef  ESMC_METHOD
 #define ESMC_METHOD "c_esmc_attributeispresentindex()"\end{verbatim}{\em RETURN VALUE:}
\begin{verbatim}      none.  return code is passed thru the parameter list
   \end{verbatim}{\em ARGUMENTS:}
\begin{verbatim}       ESMC_Base **base,      // in/out - base object
       int *num,              // in - position of attribute to check
       ESMC_Logical *present, // out/out - present flag
       int *rc) {             // in/out - return code
   \end{verbatim}
{\sf DESCRIPTION:\\ }


       Query for the presence of an Attribute.
   
%/////////////////////////////////////////////////////////////
 
\mbox{}\hrulefill\ 
 
\subsubsection [c\_ESMC\_AttributeLink] {c\_ESMC\_AttributeLink - Link an Attribute hierarchy}


  
\bigskip{\sf INTERFACE:}
\begin{verbatim}       void FTN_X(c_esmc_attributelink)(
 #undef  ESMC_METHOD
 #define ESMC_METHOD "c_esmc_attributelink()"\end{verbatim}{\em RETURN VALUE:}
\begin{verbatim}      none.  return code is passed thru the parameter list
   \end{verbatim}{\em ARGUMENTS:}
\begin{verbatim}       ESMC_Base **source,       // in/out - base object
       ESMC_Base **destination,  // in/out - base destination object
       ESMC_Logical *linkChange, // in/out - link changes boolean
       int *rc) {                // in/out - return value 
   \end{verbatim}
{\sf DESCRIPTION:\\ }


       Set a link in an attribute hierarchy.
   
%/////////////////////////////////////////////////////////////
 
\mbox{}\hrulefill\ 
 
\subsubsection [c\_ESMC\_AttributeLinkRemove] {c\_ESMC\_AttributeLinkRemove - Remove a link an Attribute hierarchy}


  
\bigskip{\sf INTERFACE:}
\begin{verbatim}       void FTN_X(c_esmc_attributelinkremove)(
 #undef  ESMC_METHOD
 #define ESMC_METHOD "c_esmc_attributelinkremove()"\end{verbatim}{\em RETURN VALUE:}
\begin{verbatim}      none.  return code is passed thru the parameter list
   \end{verbatim}{\em ARGUMENTS:}
\begin{verbatim}       ESMC_Base **source,       // in/out - base object
       ESMC_Base **destination,  // in/out - base destination object
       ESMC_Logical *linkChange, // in/out - link changes boolean
       int *rc) {                // in/out - return value 
   \end{verbatim}
{\sf DESCRIPTION:\\ }


       Remove a link in an attribute hierarchy.
   
%/////////////////////////////////////////////////////////////
 
\mbox{}\hrulefill\ 
 
\subsubsection [c\_ESMC\_AttributeRemove] {c\_ESMC\_AttributeRemove - Remove the attribute}


  
\bigskip{\sf INTERFACE:}
\begin{verbatim}       void FTN_X(c_esmc_attributeremove)(
 #undef  ESMC_METHOD
 #define ESMC_METHOD "c_esmc_attributeremove()"\end{verbatim}{\em RETURN VALUE:}
\begin{verbatim}      none.  return code is passed thru the parameter list
   \end{verbatim}{\em ARGUMENTS:}
\begin{verbatim}       ESMC_Base **base,          // in/out - base object
       char *name,                // in - F90, non-null terminated string
       int *rc,                   // in - return code     
       ESMCI_FortranStrLenArg nlen) { // hidden/in - strlen count for name
   \end{verbatim}
{\sf DESCRIPTION:\\ }


      Remove an attribute package
   
%/////////////////////////////////////////////////////////////
 
\mbox{}\hrulefill\ 
 
\subsubsection [c\_ESMC\_AttributeSetCharList] {c\_ESMC\_AttributeSetCharList - Set String Attribute List on an ESMF type}


  
\bigskip{\sf INTERFACE:}
\begin{verbatim}       void FTN_X(c_esmc_attributesetcharlist)(
 #undef  ESMC_METHOD
 #define ESMC_METHOD "c_esmc_attributesetcharlist()"\end{verbatim}{\em RETURN VALUE:}
\begin{verbatim}      none.  return code is passed thru the parameter list
   \end{verbatim}{\em ARGUMENTS:}
\begin{verbatim}       ESMC_Base **base,         // in/out - base object
       char *name,               // in - F90, non-null terminated string
       ESMC_TypeKind_Flag *tk,        // in - typekind
       int *count,               // in - number of value(s)
       char *valueList,          // in - char string
       int *lens,                // in - lengths
       int *rc,                  // in - return code
       ESMCI_FortranStrLenArg nlen,// hidden/in - strlen count for name
       ESMCI_FortranStrLenArg vlen) { // hidden/in - strlen count for valueList
   \end{verbatim}
{\sf DESCRIPTION:\\ }


       Associate a (name,value) pair with any object type in the system.
       Character strings have this special version since they come in
       with an additional hidden length argument.
   
%/////////////////////////////////////////////////////////////
 
\mbox{}\hrulefill\ 
 
\subsubsection [c\_ESMC\_AttributeSetValue] {c\_ESMC\_AttributeSetValue - Set Attribute on an ESMF type}


  
\bigskip{\sf INTERFACE:}
\begin{verbatim}       void FTN_X(c_esmc_attributesetvalue)(
 #undef  ESMC_METHOD
 #define ESMC_METHOD "c_esmc_attributesetvalue()"\end{verbatim}{\em RETURN VALUE:}
\begin{verbatim}      none.  return code is passed thru the parameter list
   \end{verbatim}{\em ARGUMENTS:}
\begin{verbatim}       ESMC_Base **base,         // in/out - base object
       char *name,               // in - F90, non-null terminated string
       ESMC_TypeKind_Flag *tk,        // in - typekind
       int *count,               // in - number of value(s)
       void *value,              // in - any value or list of values
       int *rc,                  // in - return code
       ESMCI_FortranStrLenArg nlen) { // hidden/in - strlen count for name
   \end{verbatim}
{\sf DESCRIPTION:\\ }


       Associate a (name,value) pair with any object type in the system.
       Any type or list of types can be passed except character strings
       since they come with an additional hidden length argument.
   
%/////////////////////////////////////////////////////////////
 
\mbox{}\hrulefill\ 
 
\subsubsection [c\_ESMC\_AttributeSetObjsInTree] {c\_ESMC\_AttributeSetObjsInTree - Set an Attribute on all objects}


                                                 in an Attribute hierarchy
  
\bigskip{\sf INTERFACE:}
\begin{verbatim}       void FTN_X(c_esmc_attributesetobjsintree)(
 #undef  ESMC_METHOD
 #define ESMC_METHOD "c_esmc_attributesetobjsintree()"\end{verbatim}{\em RETURN VALUE:}
\begin{verbatim}      none.  return code is passed thru the parameter list
   \end{verbatim}{\em ARGUMENTS:}
\begin{verbatim}       ESMC_Base **base,         // in/out - base object
       char *object,             // in - F90, object of the Attribute
       char *name,               // in - F90, non-null terminated string
       ESMC_TypeKind_Flag *tk,        // in - typekind of the Attribute
       int *count,               // in - items
       void *value,              // in - value
       int *rc,                  // in - return code
       ESMCI_FortranStrLenArg olen,// hidden/in - strlen count for object
       ESMCI_FortranStrLenArg nlen) { // hidden/in - strlen count for name
   \end{verbatim}
{\sf DESCRIPTION:\\ }


       Change the Attribute values for Attribute <name> on all <object>s.
   
%/////////////////////////////////////////////////////////////
 
\mbox{}\hrulefill\ 
 
\subsubsection [c\_ESMC\_AttributeSetObjChrInTree] {c\_ESMC\_AttributeSetObjChrInTree - Set a Char Attribute on all}


                                             objects in an Attribute hierarchy
  
\bigskip{\sf INTERFACE:}
\begin{verbatim}       void FTN_X(c_esmc_attributesetobjchrintree)(
 #undef  ESMC_METHOD
 #define ESMC_METHOD "c_esmc_attributesetobjchrintree()"\end{verbatim}{\em RETURN VALUE:}
\begin{verbatim}      none.  return code is passed thru the parameter list
   \end{verbatim}{\em ARGUMENTS:}
\begin{verbatim}       ESMC_Base **base,         // in/out - base object
       char *object,             // in - F90, object of the Attribute
       char *name,               // in - F90, non-null terminated string
       char *value,              // in - value
       int *rc,                  // in - return code
       ESMCI_FortranStrLenArg olen, // hidden/in - strlen count for object
       ESMCI_FortranStrLenArg nlen, // hidden/in - strlen count for name
       ESMCI_FortranStrLenArg vlen) { // hidden/in - strlen count for value
   \end{verbatim}
{\sf DESCRIPTION:\\ }


       Change the Attribute values for Attribute <name> on all <object>s.
   
%/////////////////////////////////////////////////////////////
 
\mbox{}\hrulefill\ 
 
\subsubsection [c\_ESMC\_AttributeUpdate] {c\_ESMC\_AttributeUpdate - Update an Attribute}


  
\bigskip{\sf INTERFACE:}
\begin{verbatim}       void FTN_X(c_esmc_attributeupdate)(
 #undef  ESMC_METHOD
 #define ESMC_METHOD "c_esmc_attributeupdate()"\end{verbatim}{\em RETURN VALUE:}
\begin{verbatim}      none.  return code is passed thru the parameter list
   \end{verbatim}{\em ARGUMENTS:}
\begin{verbatim}       ESMC_Base **base,         // in/out - base object
       ESMCI::VM **vm,           // in - VM that this Attribute lives on
       int *rootList,            // in - root PET list
       int *count,               // in - count of rootList
       ESMCI::InterArray<int> *petList,  // in - list of participating PETs
       ESMC_Logical *reconcile,  // in - reconcile flag
       int *rc) {                // in - return code
   \end{verbatim}
{\sf DESCRIPTION:\\ }


       Update an Attribute.
   
%/////////////////////////////////////////////////////////////
 
\mbox{}\hrulefill\ 
 
\subsubsection [c\_ESMC\_AttributeUpdateReset] {c\_ESMC\_AttributeUpdateReset - Reset flags in an Attribute hierarchy}


  
\bigskip{\sf INTERFACE:}
\begin{verbatim}       void FTN_X(c_esmc_attributeupdatereset)(
 #undef  ESMC_METHOD
 #define ESMC_METHOD "c_esmc_attributeupdatereset()"\end{verbatim}{\em RETURN VALUE:}
\begin{verbatim}      none.  return code is passed thru the parameter list
   \end{verbatim}{\em ARGUMENTS:}
\begin{verbatim}       ESMC_Base **base,         // in/out - base object
       int *rc) {                // in - return code
   \end{verbatim}
{\sf DESCRIPTION:\\ }


       Reset flags after updating an Attribute hierarchy.
   
%/////////////////////////////////////////////////////////////
 
\mbox{}\hrulefill\ 
 
\subsubsection [c\_ESMC\_AttributeRead] {c\_ESMC\_AttributeRead - Read the attribute package}


  
\bigskip{\sf INTERFACE:}
\begin{verbatim} void FTN_X(c_esmc_attributeread)(ESMC_Base **base,
                                int *fileNameLen,
                                const char *fileName,
                                int *schemaFileNameLen,
                                const char *schemaFileName,
                                int *status,
                                ESMCI_FortranStrLenArg fnlen,
                                ESMCI_FortranStrLenArg sfnlen) {
 #undef  ESMC_METHOD
 #define ESMC_METHOD "c_esmc_attributeread()"
 
   TODO convention, purpose, basename?
   TODO match formatting of rest of this file
 
    ESMF_CHECK_POINTER(*base, status)
 
    // Read the attributes into the object.
    int rc = (*base)->ESMC_BaseGetRoot()->AttributeRead(
                     *fileNameLen, // always present internal arg.
                     ESMC_NOT_PRESENT_FILTER(fileName),
                     *schemaFileNameLen, // always present internal arg.
                     ESMC_NOT_PRESENT_FILTER(schemaFileName));
                     
    if (ESMC_PRESENT(status)) *status = rc;
 }  // end c_ESMC_AttributeRead
 
  ----------------------------------------------------------------------------- 
%/////////////////////////////////////////////////////////////
 
\mbox{}\hrulefill\ 
 
\subsubsection [c\_ESMC\_AttributeWrite] {c\_ESMC\_AttributeWrite - write out attributes to file}


  
\bigskip{\sf INTERFACE:}
\begin{verbatim} void FTN_X(c_esmc_attributewrite)(
 #undef  ESMC_METHOD
 #define ESMC_METHOD "c_esmc_attributewrite()"\end{verbatim}{\em RETURN VALUE:}
\begin{verbatim}      none.  return code is passed thru the parameter list
   \end{verbatim}{\em ARGUMENTS:}
\begin{verbatim}       ESMC_Base **base,          // in/out - base object
       char *convention,          // in - convention
       char *purpose,             // in - purpose
       char *object,              // in - object type
       char *targetobj,           // in - target object for writing
       ESMC_AttWriteFlag *attwriteflag, // in - attwriteflag
       int *rc,                   // in - return code
       ESMCI_FortranStrLenArg clen,// hidden/in - strlen count for convention
       ESMCI_FortranStrLenArg plen,// hidden/in - strlen count for purpose
       ESMCI_FortranStrLenArg olen,// hidden/in - strlen count for object
       ESMCI_FortranStrLenArg tlen) { // hidden/in - strlen count for target object
   \end{verbatim}
{\sf DESCRIPTION:\\ }


       Associate a convention, purpose, and object type with an attribute package
   
%/////////////////////////////////////////////////////////////
 
\mbox{}\hrulefill\ 
 
\subsubsection [c\_ESMC\_AttPackStreamJSON] {c\_ESMC\_AttPackStreamJSON - write json from attpack to string}


  
\bigskip{\sf INTERFACE:}
\begin{verbatim} void FTN_X(c_esmc_attpackstreamjson)(
 #undef  ESMC_METHOD
 #define ESMC_METHOD "c_esmc_attpackstreamjson()"\end{verbatim}{\em RETURN VALUE:}
\begin{verbatim}      none.  return code is passed thru the parameter list\end{verbatim}{\em ARGUMENTS:}
\begin{verbatim}         ESMCI::Attribute **attpack,    // in - attpack
         int *flattenPackList,          // in - should nested attribute packs be flattened
         int *includeUnset,             // in - should unset attributes be included
         int *includeLinks,             // in - should recurse through linked attributes
         char *output,                  // out - output string
         int *rc,                       // out - return code
         ESMCI_FortranStrLenArg olen) { // hidden/in - strlen count for target object\end{verbatim}
{\sf DESCRIPTION:\\ }


       Write attpack contents to json formatted output
   
%/////////////////////////////////////////////////////////////
 
\mbox{}\hrulefill\ 
 
\subsubsection [c\_ESMC\_AttPackStreamJSONstrlen] {c\_ESMC\_AttPackStreamJSONstrlen - get the string length required}


                                                to write json from attpack to
                                                string
  
\bigskip{\sf INTERFACE:}
\begin{verbatim} void FTN_X(c_esmc_attpackstreamjsonstrlen)(
 #undef  ESMC_METHOD
 #define ESMC_METHOD "c_esmc_attpackstreamjsonstrlen()"\end{verbatim}{\em RETURN VALUE:}
\begin{verbatim}      none.  return code is passed thru the parameter list\end{verbatim}{\em ARGUMENTS:}
\begin{verbatim}         ESMCI::Attribute **attpack,    // in - attpack
         int *flattenPackList,          // in - should nested attribute packs be flattened
         int *includeUnset,             // in - should unset attributes be included
         int *includeLinks,             // in - should recurse through linked attributes
         int *jsonstrlen,               // out - output stringlength
         int *rc) {                     // out - return code\end{verbatim}
{\sf DESCRIPTION:\\ }


       Get the string length required to write attpack contents to json
       formatted output
  
%...............................................................
\setlength{\parskip}{\oldparskip}
\setlength{\parindent}{\oldparindent}
\setlength{\baselineskip}{\oldbaselineskip}
