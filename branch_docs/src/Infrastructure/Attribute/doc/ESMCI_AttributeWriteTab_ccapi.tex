%                **** IMPORTANT NOTICE *****
% This LaTeX file has been automatically produced by ProTeX v. 1.1
% Any changes made to this file will likely be lost next time
% this file is regenerated from its source. Send questions 
% to Arlindo da Silva, dasilva@gsfc.nasa.gov
 
\setlength{\oldparskip}{\parskip}
\setlength{\parskip}{1.5ex}
\setlength{\oldparindent}{\parindent}
\setlength{\parindent}{0pt}
\setlength{\oldbaselineskip}{\baselineskip}
\setlength{\baselineskip}{11pt}
 
%--------------------- SHORT-HAND MACROS ----------------------
\def\bv{\begin{verbatim}}
\def\ev{\end{verbatim}}
\def\be{\begin{equation}}
\def\ee{\end{equation}}
\def\bea{\begin{eqnarray}}
\def\eea{\end{eqnarray}}
\def\bi{\begin{itemize}}
\def\ei{\end{itemize}}
\def\bn{\begin{enumerate}}
\def\en{\end{enumerate}}
\def\bd{\begin{description}}
\def\ed{\end{description}}
\def\({\left (}
\def\){\right )}
\def\[{\left [}
\def\]{\right ]}
\def\<{\left  \langle}
\def\>{\right \rangle}
\def\cI{{\cal I}}
\def\diag{\mathop{\rm diag}}
\def\tr{\mathop{\rm tr}}
%-------------------------------------------------------------

\markboth{Left}{Source File: ESMCI\_AttributeWriteTab.C,  Date: Tue May  5 21:00:05 MDT 2020
}

 
%/////////////////////////////////////////////////////////////

  
%/////////////////////////////////////////////////////////////
 
\mbox{}\hrulefill\
 
\subsubsection [AttributeWriteTab] {AttributeWriteTab - write Attributes in Tab delimited format}


  
\bigskip{\sf INTERFACE:}
\begin{verbatim} int Attribute::AttributeWriteTab(\end{verbatim}{\em RETURN VALUE:}
\begin{verbatim}      {\tt ESMF\_SUCCESS} or error code on failure.\end{verbatim}{\em ARGUMENTS:}
\begin{verbatim}         const string &convention,             //  in - convention
         const string &purpose,                //  in - purpose
         const string &object,                 //  in - object
         const string &varobj,                 //  in - variable object
         const string &basename) const{        //  in - basename\end{verbatim}
{\sf DESCRIPTION:\\ }


      Write the contents on an {\tt Attribute} hierarchy in Tab delimited format.
      Expected to be called internally.
   
%/////////////////////////////////////////////////////////////
 
\mbox{}\hrulefill\
 
\subsubsection [AttributeWriteTabTraverse] {AttributeWriteTabTraverse - write Attributes in Tab delimited format}


                                               recursive function
  
\bigskip{\sf INTERFACE:}
\begin{verbatim} int Attribute::AttributeWriteTabTraverse(\end{verbatim}{\em RETURN VALUE:}
\begin{verbatim}      {\tt ESMF\_SUCCESS} or error code on failure.\end{verbatim}{\em ARGUMENTS:}
\begin{verbatim}         FILE *tab,                                //  in - file to write
         const string &convention,                 //  in - convention
         const string &purpose,                    //  in - purpose
         int &index,                               //  in - counter
         const int &columns,                       //  in - columns
         int *attrLens,                            //  in - column widths
         const vector<string> &attrNames) const{   //  inout - column headings\end{verbatim}
{\sf DESCRIPTION:\\ }


      Write the contents on an {\tt Attribute} hierarchy in Tab delimited format.
   
%/////////////////////////////////////////////////////////////
 
\mbox{}\hrulefill\
 
\subsubsection [AttributeWriteTabBuffer] {AttributeWriteTabBuffer - write Attributes in Tab delimited format}


                                               recursive function
  
\bigskip{\sf INTERFACE:}
\begin{verbatim} int Attribute::AttributeWriteTabBuffer(\end{verbatim}{\em RETURN VALUE:}
\begin{verbatim}      {\tt ESMF\_SUCCESS} or error code on failure.\end{verbatim}{\em ARGUMENTS:}
\begin{verbatim}         FILE *tab,                               //  in - file to write
         int &index,                              //  in - index counter
         const int &columns,                      //  in - columns
         int *attrLens,                           //  in - integer array of attribute lengths
         const vector<string> &attrNames) const{  //  in - attribute names\end{verbatim}
{\sf DESCRIPTION:\\ }


      Write the contents on an {\tt Attribute} hierarchy in Tab delimited format.
      Expected to be called internally.
  
%...............................................................
\setlength{\parskip}{\oldparskip}
\setlength{\parindent}{\oldparindent}
\setlength{\baselineskip}{\oldbaselineskip}
