%                **** IMPORTANT NOTICE *****
% This LaTeX file has been automatically produced by ProTeX v. 1.1
% Any changes made to this file will likely be lost next time
% this file is regenerated from its source. Send questions 
% to Arlindo da Silva, dasilva@gsfc.nasa.gov
 
\setlength{\oldparskip}{\parskip}
\setlength{\parskip}{1.5ex}
\setlength{\oldparindent}{\parindent}
\setlength{\parindent}{0pt}
\setlength{\oldbaselineskip}{\baselineskip}
\setlength{\baselineskip}{11pt}
 
%--------------------- SHORT-HAND MACROS ----------------------
\def\bv{\begin{verbatim}}
\def\ev{\end{verbatim}}
\def\be{\begin{equation}}
\def\ee{\end{equation}}
\def\bea{\begin{eqnarray}}
\def\eea{\end{eqnarray}}
\def\bi{\begin{itemize}}
\def\ei{\end{itemize}}
\def\bn{\begin{enumerate}}
\def\en{\end{enumerate}}
\def\bd{\begin{description}}
\def\ed{\end{description}}
\def\({\left (}
\def\){\right )}
\def\[{\left [}
\def\]{\right ]}
\def\<{\left  \langle}
\def\>{\right \rangle}
\def\cI{{\cal I}}
\def\diag{\mathop{\rm diag}}
\def\tr{\mathop{\rm tr}}
%-------------------------------------------------------------

\markboth{Left}{Source File: ESMF\_AttributeUpdateEx.F90,  Date: Tue May  5 21:00:07 MDT 2020
}

 
%/////////////////////////////////////////////////////////////

   \subsubsection{Updating Attributes in a distributed environment}
  
   This advanced example illustrates the proper methods of Attribute manipulation
   in a distributed environment to ensure consistency of metadata across the VM.
   This example is much more complicated than the previous two because we will
   be following the flow of control of a typical model run with two gridded Components
   and one coupling Component.  We will start out in the application driver, declaring
   Components, States, and the routines used to initialize, run and finalize the user's
   model Components.  Then we will follow the control flow into the actual Component level
   through initialize, run, and finalize examining how Attributes are used to organize the
   metadata.
  
   This example follows a simple user model with two gridded Components and one coupling Component.
   The initialize routines are used to set up the application data and the run
   routines are used to manipulate the data.  Accordingly, most of the Attribute manipulation
   will take place in the initialize phase of each of the three Components.  The two gridded
   Components will be running on exclusive pieces of the VM and the coupler Component will
   encompass the entire VM so that it can handle the Attribute communications.
  
   The control flow of this
   example will start in the application driver, after which it will complete three cycles
   through the three Components.  The first cycle will be through the initialize routines,
   from the first gridded Component to the second gridded Component to the coupler Component.  The
   second cycle will go through the run routines, from the first gridded Component to the
   coupler Component to the second Gridded component.  The third cycle will be through the
   finalize routines in the same order as the first cycle.
   
%/////////////////////////////////////////////////////////////

   In the application driver, we must now construct some ESMF objects,
   such as the gridded Components, the coupler Component, and the States.  This
   is also where it is determined which subsets of the PETs of the VM the
   Components will be using to run their initialize, run, and finalize routines. 
%/////////////////////////////////////////////////////////////

 \begin{verbatim}
        gridcomp1 = ESMF_GridCompCreate(name="gridcomp1", &
          petList=(/0,1/), rc=rc)
 
\end{verbatim}
 
%/////////////////////////////////////////////////////////////

 \begin{verbatim}
        gridcomp2 = ESMF_GridCompCreate(name="gridcomp2", &
          petList=(/2,3/), rc=rc)
 
\end{verbatim}
 
%/////////////////////////////////////////////////////////////

 \begin{verbatim}
        cplcomp = ESMF_CplCompCreate(name="cplcomp", &
          petList=(/0,1,2,3/), rc=rc)
 
\end{verbatim}
 
%/////////////////////////////////////////////////////////////

 \begin{verbatim}
      endif

        c1exp = ESMF_StateCreate(name="Comp1 exportState", &
                               stateintent=ESMF_STATEINTENT_EXPORT, rc=rc)
 
\end{verbatim}
 
 
%/////////////////////////////////////////////////////////////

   Before the individual components are initialized, run, and finalized Attributes should be set at the
   Component level.  Here we are going to use the ESG Attribute package on
   the first gridded Component.  The Attribute package is added, and then
   each of the Attributes is set.  The Attribute hierarchy of the Component
   is then linked to the Attribute hierarchy of the export State in a
   manual fashion. 
%/////////////////////////////////////////////////////////////

 \begin{verbatim}
      convESMF = 'ESMF'
      purpGen = 'General'
      call ESMF_AttributeAdd(gridcomp1, &
        convention=convESMF, purpose=purpGen, attpack=attpack, &
        rc=rc)
 
\end{verbatim}
 
%/////////////////////////////////////////////////////////////

 \begin{verbatim}

    call ESMF_AttributeSet(gridcomp1, 'Agency', 'NASA', attpack, rc=rc)
 
\end{verbatim}
 
%/////////////////////////////////////////////////////////////

 \begin{verbatim}

    call ESMF_AttributeSet(gridcomp1, 'Author', 'Max Suarez', &
      convention=convESMF, purpose=purpGen, rc=rc)
 
\end{verbatim}
 
%/////////////////////////////////////////////////////////////

 \begin{verbatim}

    call ESMF_AttributeSet(gridcomp1, 'CodingLanguage', &
      'Fortran 90', convention=convESMF, purpose=purpGen, rc=rc)
 
\end{verbatim}
 
%/////////////////////////////////////////////////////////////

 \begin{verbatim}

    call ESMF_AttributeSet(gridcomp1, 'Discipline', &
      'Atmosphere', convention=convESMF, purpose=purpGen, rc=rc)
 
\end{verbatim}
 
%/////////////////////////////////////////////////////////////

 \begin{verbatim}

    call ESMF_AttributeSet(gridcomp1, 'ComponentLongName', &
   'Goddard Earth Observing System Version 5 Finite Volume Dynamical Core', &
        convention=convESMF, purpose=purpGen, rc=rc)
 
\end{verbatim}
 
%/////////////////////////////////////////////////////////////

 \begin{verbatim}

    call ESMF_AttributeSet(gridcomp1, 'ModelComponentFramework', &
      'ESMF', &
      convention=convESMF, purpose=purpGen, rc=rc)
 
\end{verbatim}
 
%/////////////////////////////////////////////////////////////

 \begin{verbatim}

    call ESMF_AttributeSet(gridcomp1, 'ComponentShortName', &
      'GEOS-5 FV dynamical core', convention=convESMF, purpose=purpGen, rc=rc)
 
\end{verbatim}
 
%/////////////////////////////////////////////////////////////

 \begin{verbatim}

    call ESMF_AttributeSet(gridcomp1, 'PhysicalDomain', &
      'Earth system', convention=convESMF, purpose=purpGen, rc=rc)
 
\end{verbatim}
 
%/////////////////////////////////////////////////////////////

 \begin{verbatim}

    call ESMF_AttributeSet(gridcomp1, 'Version', &
      'GEOSagcm-EROS-beta7p12', convention=convESMF, purpose=purpGen, rc=rc)
 
\end{verbatim}
 
%/////////////////////////////////////////////////////////////

 \begin{verbatim}


    call ESMF_AttributeLink(gridcomp1, c1exp, rc=rc)
 
\end{verbatim}
 
%/////////////////////////////////////////////////////////////

 \begin{verbatim}


    call ESMF_AttributeLinkRemove(gridcomp1, c1exp, rc=rc)
 
\end{verbatim}
 
%/////////////////////////////////////////////////////////////

   Now the individual Components will be run.  First we will initialize the two
   gridded Components, then we will initialize the coupler Component.
   During each of these Component initialize routines Attribute
   packages will be added, and the Attributes set.  The Attribute
   hierarchies will also be linked (unlinking also demonstrated).  As the
   gridded Components will be running on exclusive portions of the VM, the
   Attributes will need to be made available across the VM using an
   {\tt ESMF\_StateReconcile()} call in the coupler Component.  The majority of
   the work with Attributes will take place in this portion of the model run, as
   metadata rarely needs to be changed during run time.
  
   What
   follows are the calls from the driver code that run the initialize, run, and finalize routines
   for each of the Components.  After these calls we will step through the first
   cycle as explained in the introduction, through the initialize routines of
   gridded Component 1 to gridded Component 2 to the coupler Component. 
%/////////////////////////////////////////////////////////////

 \begin{verbatim}
      call ESMF_GridCompInitialize(gridcomp1, exportState=c1exp, rc=rc)
      call ESMF_GridCompInitialize(gridcomp2, importState=c2imp, rc=rc)
      call ESMF_CplCompInitialize(cplcomp, importState=c1exp, &
        exportState=c2imp, rc=rc)
 
\end{verbatim}
 
%/////////////////////////////////////////////////////////////

 \begin{verbatim}


      call ESMF_GridCompRun(gridcomp1, exportState=c1exp, rc=rc)
 
\end{verbatim}
 
%/////////////////////////////////////////////////////////////

 \begin{verbatim}

      call ESMF_CplCompRun(cplcomp, importState=c1exp, &
        exportState=c2imp, userRc=urc, rc=rc)
 
\end{verbatim}
 
%/////////////////////////////////////////////////////////////

 \begin{verbatim}


      call ESMF_GridCompRun(gridcomp2, importState=c2imp, rc=rc)
 
\end{verbatim}
 
%/////////////////////////////////////////////////////////////

 \begin{verbatim}


      call ESMF_GridCompFinalize(gridcomp1, exportState=c1exp, rc=rc)
 
\end{verbatim}
 
%/////////////////////////////////////////////////////////////

 \begin{verbatim}

      call ESMF_GridCompFinalize(gridcomp2, importState=c2imp, rc=rc)
 
\end{verbatim}
 
%/////////////////////////////////////////////////////////////

 \begin{verbatim}

      call ESMF_CplCompFinalize(cplcomp, importState=c1exp, &
        exportState=c2imp, rc=rc)

 
\end{verbatim}

%...............................................................
\setlength{\parskip}{\oldparskip}
\setlength{\parindent}{\oldparindent}
\setlength{\baselineskip}{\oldbaselineskip}
