%                **** IMPORTANT NOTICE *****
% This LaTeX file has been automatically produced by ProTeX v. 1.1
% Any changes made to this file will likely be lost next time
% this file is regenerated from its source. Send questions 
% to Arlindo da Silva, dasilva@gsfc.nasa.gov
 
\setlength{\oldparskip}{\parskip}
\setlength{\parskip}{1.5ex}
\setlength{\oldparindent}{\parindent}
\setlength{\parindent}{0pt}
\setlength{\oldbaselineskip}{\baselineskip}
\setlength{\baselineskip}{11pt}
 
%--------------------- SHORT-HAND MACROS ----------------------
\def\bv{\begin{verbatim}}
\def\ev{\end{verbatim}}
\def\be{\begin{equation}}
\def\ee{\end{equation}}
\def\bea{\begin{eqnarray}}
\def\eea{\end{eqnarray}}
\def\bi{\begin{itemize}}
\def\ei{\end{itemize}}
\def\bn{\begin{enumerate}}
\def\en{\end{enumerate}}
\def\bd{\begin{description}}
\def\ed{\end{description}}
\def\({\left (}
\def\){\right )}
\def\[{\left [}
\def\]{\right ]}
\def\<{\left  \langle}
\def\>{\right \rangle}
\def\cI{{\cal I}}
\def\diag{\mathop{\rm diag}}
\def\tr{\mathop{\rm tr}}
%-------------------------------------------------------------

\markboth{Left}{Source File: ESMF\_FieldBundleRedistEx.F90,  Date: Tue May  5 21:00:04 MDT 2020
}

 
%/////////////////////////////////////////////////////////////

   \subsubsection{Redistribute data from a source FieldBundle to a destination FieldBundle}
   \label{sec:fieldbundle:usage:redist_1dptr}
  
   The {\tt ESMF\_FieldBundleRedist} interface can be used to redistribute data from
   source FieldBundle to destination FieldBundle. This interface is overloaded by type and kind;
   In the version of {\tt ESMF\_FieldBundleRedist} without factor argument, a default value
   of factor 1 is used.
   
   In this example, we first create two FieldBundles, a source FieldBundle and a destination
   FieldBundle. Then we use {\tt ESMF\_FieldBundleRedist} to
   redistribute data from source FieldBundle to destination FieldBundle. 
%/////////////////////////////////////////////////////////////

 \begin{verbatim}
    ! perform redist
    call ESMF_FieldBundleRedistStore(srcFieldBundle, dstFieldBundle, &
         routehandle, rc=rc)
 
\end{verbatim}
 
%/////////////////////////////////////////////////////////////

 \begin{verbatim}
    call ESMF_FieldBundleRedist(srcFieldBundle, dstFieldBundle, &
         routehandle, rc=rc)
 
\end{verbatim}
 
%/////////////////////////////////////////////////////////////

   \subsubsection{Redistribute data from a packed source FieldBundle to a packed destination FieldBundle}
   \label{sec:fieldbundle:usage:redist_packed}
  
   The {\tt ESMF\_FieldBundleRedist} interface can be used to redistribute data from
   source FieldBundle to destination FieldBundle when both Bundles are packed with same
   number of fields.  
  
   In this example, we first create two packed FieldBundles, a source FieldBundle and a destination
   FieldBundle. Then we use {\tt ESMF\_FieldBundleRedist} to
   redistribute data from source FieldBundle to destination FieldBundle.
  
   The same Grid is used where the source and destination packed FieldBundle are built upon. Source
   and destination Bundle have different memory layout. 
%/////////////////////////////////////////////////////////////

 \begin{verbatim}
    allocate(srcfptr(3,5,10), dstfptr(10,5,3))
    srcfptr = lpe
    srcFieldBundle = ESMF_FieldBundleCreate((/'field01', 'field02', 'field03'/), &
      srcfptr, grid, 1, gridToFieldMap=(/2,3/), rc=rc)
 
\end{verbatim}
 
%/////////////////////////////////////////////////////////////

 \begin{verbatim}
    dstFieldBundle = ESMF_FieldBundleCreate((/'field01', 'field02', 'field03'/), &
      dstfptr, grid, 3, gridToFieldMap=(/2,1/), rc=rc)
 
\end{verbatim}
 
%/////////////////////////////////////////////////////////////

 \begin{verbatim}
    ! perform redist
    call ESMF_FieldBundleRedistStore(srcFieldBundle, dstFieldBundle, &
         routehandle, rc=rc)
 
\end{verbatim}
 
%/////////////////////////////////////////////////////////////

 \begin{verbatim}
    call ESMF_FieldBundleRedist(srcFieldBundle, dstFieldBundle, &
         routehandle, rc=rc)
 
\end{verbatim}

%...............................................................
\setlength{\parskip}{\oldparskip}
\setlength{\parindent}{\oldparindent}
\setlength{\baselineskip}{\oldbaselineskip}
