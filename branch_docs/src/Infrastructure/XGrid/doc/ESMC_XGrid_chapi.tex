%                **** IMPORTANT NOTICE *****
% This LaTeX file has been automatically produced by ProTeX v. 1.1
% Any changes made to this file will likely be lost next time
% this file is regenerated from its source. Send questions 
% to Arlindo da Silva, dasilva@gsfc.nasa.gov
 
\setlength{\oldparskip}{\parskip}
\setlength{\parskip}{1.5ex}
\setlength{\oldparindent}{\parindent}
\setlength{\parindent}{0pt}
\setlength{\oldbaselineskip}{\baselineskip}
\setlength{\baselineskip}{11pt}
 
%--------------------- SHORT-HAND MACROS ----------------------
\def\bv{\begin{verbatim}}
\def\ev{\end{verbatim}}
\def\be{\begin{equation}}
\def\ee{\end{equation}}
\def\bea{\begin{eqnarray}}
\def\eea{\end{eqnarray}}
\def\bi{\begin{itemize}}
\def\ei{\end{itemize}}
\def\bn{\begin{enumerate}}
\def\en{\end{enumerate}}
\def\bd{\begin{description}}
\def\ed{\end{description}}
\def\({\left (}
\def\){\right )}
\def\[{\left [}
\def\]{\right ]}
\def\<{\left  \langle}
\def\>{\right \rangle}
\def\cI{{\cal I}}
\def\diag{\mathop{\rm diag}}
\def\tr{\mathop{\rm tr}}
%-------------------------------------------------------------

\markboth{Left}{Source File: ESMC\_XGrid.h,  Date: Tue May  5 20:59:57 MDT 2020
}

 
%/////////////////////////////////////////////////////////////
\subsubsection [ESMC\_XGridCreate] {ESMC\_XGridCreate - Create an XGrid}


  
\bigskip{\sf INTERFACE:}
\begin{verbatim}   ESMC_XGrid ESMC_XGridCreate(
                               int sideAGridCount,  ESMC_Grid *sideAGrid, // in
                               int sideAMeshCount,  ESMC_Mesh *sideAMesh, // in
                               int sideBGridCount,  ESMC_Grid *sideBGrid, // in
                               int sideBMeshCount,  ESMC_Mesh *sideBMesh, // in
                               ESMC_InterArrayInt *sideAGridPriority,     // in
                               ESMC_InterArrayInt *sideAMeshPriority,     // in
                               ESMC_InterArrayInt *sideBGridPriority,     // in
                               ESMC_InterArrayInt *sideBMeshPriority,     // in
                               ESMC_InterArrayInt *sideAMaskValues,       // in
                               ESMC_InterArrayInt *sideBMaskValues,       // in
                               int storeOverlay,                          // in
                               int *rc                                    // out
 );
 \end{verbatim}{\em RETURN VALUE:}
\begin{verbatim}    Newly created ESMC_XGrid object.\end{verbatim}
{\sf DESCRIPTION:\\ }


  
        Create an {\tt ESMC\_XGrid} object from user supplied input: the list of Grids or Meshes on side A and side B, 
    and other optional arguments. A user can supply both Grids and Meshes on one side to create
    the XGrid. By default, the Grids have a higher priority over Meshes but the order of priority 
    can be adjusted by the optional GridPriority and MeshPriority arguments. The priority order
    of Grids and Meshes can also be interleaved by rearranging the optional 
    GridPriority and MeshPriority arguments accordingly.
    
    Sparse matrix multiplication coefficients are internally computed and
    uniquely determined by the Grids or Meshes provided in {\tt sideA} and {\tt sideB}. User can supply
    a single {\tt ESMC\_Grid} or an array of {\tt ESMC\_Grid} on either side of the 
    {\tt ESMC\_XGrid}. For an array of {\tt ESMC\_Grid} or {\tt ESMC\_Mesh} in {\tt sideA} or {\tt sideB},
    a merging process concatenates all the {\tt ESMC\_Grid}s and {\tt ESMC\_Mesh}es 
    into a super mesh represented
    by {\tt ESMC\_Mesh}. The super mesh is then used to compute the XGrid. 
    Grid or Mesh objects in {\tt sideA} and {\tt sideB} arguments must have coordinates defined for
    the corners of a Grid or Mesh cell. XGrid creation can be potentially memory expensive given the
    size of the input Grid and Mesh objects. By default, the super mesh is not stored
    to reduce memory usage. 
   
    If {\tt sideA} and {\tt sideB} have a single 
    Grid or Mesh object, it's erroneous
    if the two Grids or Meshes are spatially disjoint. 
    It is also erroneous to specify a Grid or Mesh object in {\tt sideA} or {\tt sideB}
    that is spatially disjoint from the {\tt ESMC\_XGrid}. 
  
    This call is {\em collective} across the current VM. For more details please refer to the description 
    \ref{sec:xgrid:desc} of the XGrid class. 
  
       The arguments are:
       \begin{description}
       \item [sideAGridCount]
             The number of Grids in the {\tt sideAGrid} array.
       \item [{[sideAGrid]}]
             Parametric 2D Grids on side A, for example, 
             these Grids can be either Cartesian 2D or Spherical.
       \item [sideAMeshCount]
             The number of Meshes in the {\tt sideAMesh} array.
       \item [{[sideAMesh]}]
             Parametric 2D Meshes on side A, for example, 
             these Meshes can be either Cartesian 2D or Spherical.
       \item [sideBGridCount]
             The number of Grids in the {\tt sideBGrid} array.
       \item [{[sideBGrid]}]
             Parametric 2D Grids on side B, for example, 
             these Grids can be either Cartesian 2D or Spherical.
       \item [sideBMeshCount]
             The number of Meshes in the {\tt sideBMesh} array.
       \item [{[sideBMesh]}]
             Parametric 2D Meshes on side B, for example, 
             these Meshes can be either Cartesian 2D or Spherical.
       \item [{[sideAGridPriority]}]
             Priority array of Grids on {\tt sideA} during overlay generation.
             The priority arrays describe the priorities of Grids at the overlapping region.
             Flux contributions at the overlapping region are computed in the order from the Grid of the
             highest priority to the lowest priority.
       \item [{[sideAMeshPriority]}]
             Priority array of Meshes on {\tt sideA} during overlay generation.
             The priority arrays describe the priorities of Meshes at the overlapping region.
             Flux contributions at the overlapping region are computed in the order from the Mesh of the
             highest priority to the lowest priority.
       \item [{[sideBGridPriority]}]
             Priority of Grids on {\tt sideB} during overlay generation
             The priority arrays describe the priorities of Grids at the overlapping region.
             Flux contributions at the overlapping region are computed in the order from the Grid of the
             highest priority to the lowest priority.
       \item [{[sideBMeshPriority]}]
             Priority array of Meshes on {\tt sideB} during overlay generation.
             The priority arrays describe the priorities of Meshes at the overlapping region.
             Flux contributions at the overlapping region are computed in the order from the Mesh of the
             highest priority to the lowest priority.
       \item [{[sideAMaskValues]}]
             Mask information can be set in the Grid (see~\ref{sec:usage:items}) or Mesh
             upon which the Field is built. The {\tt sideAMaskValues} argument specifies the values in that 
             mask information which indicate a point should be masked out. In other words, a location is masked if and only if the
             value for that location in the mask information matches one of the values listed in {\tt sideAMaskValues}.  
             If {\tt sideAMaskValues} is not specified, no masking on side A will occur. 
       \item [{[sideBMaskValues]}]
             Mask information can be set in the Grid (see~\ref{sec:usage:items}) or Mesh
             upon which the Field is built. The {\tt sideBMaskValues} argument specifies the values in that 
             mask information which indicate a point should be masked out. In other words, a location is masked if and only if the
             value for that location in the mask information matches one of the values listed in {\tt sideBMaskValues}.  
             If {\tt sideBMaskValues} is not specified, no masking on side B will occur. 
       \item [storeOverlay]
             Setting the {\tt storeOverlay} optional argument to 0. 
             allows a user to bypass storage of the Mesh used to represent the XGrid.
             Only a DistGrid is stored to allow Field to be built on the XGrid.
             If the temporary mesh object is of interest, {\tt storeOverlay} can be set to a value not equal to 0.
             so a user can retrieve it for future use.
       \item [{[rc]}]
             Return code; equals {\tt ESMF\_SUCCESS} only if the {\tt ESMF\_XGrid} 
             is created.
       \end{description}
   
%/////////////////////////////////////////////////////////////
 
\mbox{}\hrulefill\ 
 
\subsubsection [ESMC\_XGridDestroy] {ESMC\_XGridDestroy - Destroy a XGrid}


  
\bigskip{\sf INTERFACE:}
\begin{verbatim} int ESMC_XGridDestroy(
   ESMC_XGrid *xgrid     // inout
 );
 \end{verbatim}{\em RETURN VALUE:}
\begin{verbatim}    Return code; equals ESMF_SUCCESS if there are no errors.\end{verbatim}
{\sf DESCRIPTION:\\ }


  
    Releases all resources associated with this {\tt ESMC\_XGrid}.
      Return code; equals {\tt ESMF\_SUCCESS} if there are no errors.
  
    The arguments are:
    \begin{description}
    \item[xgrid]
      Destroy contents of this {\tt ESMC\_XGrid}.
    \end{description}
   
%/////////////////////////////////////////////////////////////
 
\mbox{}\hrulefill\ 
 
\subsubsection [ESMC\_XGridGetSideAGridCount] {ESMC\_XGridGetSideAGridCount - Get the number of Grids on side A.}


  
\bigskip{\sf INTERFACE:}
\begin{verbatim} int ESMC_XGridGetSideAGridCount(
   ESMC_XGrid xgrid,     // in
   int *rc               // out
 );
 \end{verbatim}{\em RETURN VALUE:}
\begin{verbatim}    The number of Grids on side A. \end{verbatim}
{\sf DESCRIPTION:\\ }


  
    Get the number of Grids on side A. 
  
    The arguments are:
    \begin{description}
    \item[xgrid]
      The XGrid from which to get the information.
    \item[{[rc]}]
      Return code; equals {\tt ESMF\_SUCCESS} if there are no errors.
    \end{description}
   
%/////////////////////////////////////////////////////////////
 
\mbox{}\hrulefill\ 
 
\subsubsection [ESMC\_XGridGetSideAMeshCount] {ESMC\_XGridGetSideAMeshCount - Get the number of Meshes on side A.}


  
\bigskip{\sf INTERFACE:}
\begin{verbatim} int ESMC_XGridGetSideAMeshCount(
   ESMC_XGrid xgrid,     // in
   int *rc               // out
 );
 \end{verbatim}{\em RETURN VALUE:}
\begin{verbatim}    The number of Meshes on side A. \end{verbatim}
{\sf DESCRIPTION:\\ }


  
    Get the number of Meshes on side A. 
  
    The arguments are:
    \begin{description}
    \item[xgrid]
      The XGrid from which to get the information.
    \item[{[rc]}]
      Return code; equals {\tt ESMF\_SUCCESS} if there are no errors.
    \end{description}
   
%/////////////////////////////////////////////////////////////
 
\mbox{}\hrulefill\ 
 
\subsubsection [ESMC\_XGridGetSideBGridCount] {ESMC\_XGridGetSideBGridCount - Get the number of Grids on side B.}


  
\bigskip{\sf INTERFACE:}
\begin{verbatim} int ESMC_XGridGetSideBGridCount(
   ESMC_XGrid xgrid,     // in
   int *rc               // out
 );
 \end{verbatim}{\em RETURN VALUE:}
\begin{verbatim}    The number of Grids on side B. \end{verbatim}
{\sf DESCRIPTION:\\ }


  
    Get the number of Grids on side B. 
  
    The arguments are:
    \begin{description}
    \item[xgrid]
      The XGrid from which to get the information.
    \item[{[rc]}]
      Return code; equals {\tt ESMF\_SUCCESS} if there are no errors.
    \end{description}
   
%/////////////////////////////////////////////////////////////
 
\mbox{}\hrulefill\ 
 
\subsubsection [ESMC\_XGridGetSideBMeshCount] {ESMC\_XGridGetSideBMeshCount - Get the number of Meshes on side B.}


  
\bigskip{\sf INTERFACE:}
\begin{verbatim} int ESMC_XGridGetSideBMeshCount(
   ESMC_XGrid xgrid,     // in
   int *rc               // out
 );
 \end{verbatim}{\em RETURN VALUE:}
\begin{verbatim}    The number of Meshes on side B. \end{verbatim}
{\sf DESCRIPTION:\\ }


  
    Get the number of Meshes on side B. 
  
    The arguments are:
    \begin{description}
    \item[xgrid]
      The XGrid from which to get the information.
    \item[{[rc]}]
      Return code; equals {\tt ESMF\_SUCCESS} if there are no errors.
    \end{description}
   
%/////////////////////////////////////////////////////////////
 
\mbox{}\hrulefill\ 
 
\subsubsection [ESMC\_XGridGetDimCount] {ESMC\_XGridGetDimCount - Get the dimension of the XGrid.}


  
\bigskip{\sf INTERFACE:}
\begin{verbatim} int ESMC_XGridGetDimCount(
   ESMC_XGrid xgrid,     // in
   int *rc               // out
 );
 \end{verbatim}{\em RETURN VALUE:}
\begin{verbatim}    The dimension of the XGrid.\end{verbatim}
{\sf DESCRIPTION:\\ }


  
     Get the dimension of the XGrid.
  
    The arguments are:
    \begin{description}
    \item[xgrid]
      The XGrid from which to get the information.
    \item[{[rc]}]
      Return code; equals {\tt ESMF\_SUCCESS} if there are no errors.
    \end{description}
   
%/////////////////////////////////////////////////////////////
 
\mbox{}\hrulefill\ 
 
\subsubsection [ESMC\_XGridGetElementCount] {ESMC\_XGridGetElementCount - Get the number of elements in the XGrid.}


  
\bigskip{\sf INTERFACE:}
\begin{verbatim} int ESMC_XGridGetElementCount(
   ESMC_XGrid xgrid,     // in
   int *rc               // out
 );
 \end{verbatim}{\em RETURN VALUE:}
\begin{verbatim}       The number of elements in the XGrid.\end{verbatim}
{\sf DESCRIPTION:\\ }


  
      Get the number of elements in the XGrid.
  
    The arguments are:
    \begin{description}
    \item[xgrid]
      The XGrid from which to get the information.
    \item[{[rc]}]
      Return code; equals {\tt ESMF\_SUCCESS} if there are no errors.
    \end{description}
   
%/////////////////////////////////////////////////////////////
 
\mbox{}\hrulefill\ 
 
\subsubsection [ESMC\_XGridGetMesh] {ESMC\_XGridGetMesh - Get the Mesh representation of the XGrid. }


  
\bigskip{\sf INTERFACE:}
\begin{verbatim} ESMC_Mesh ESMC_XGridGetMesh(
   ESMC_XGrid xgrid,     // in
   int *rc               // out
 );
 \end{verbatim}{\em RETURN VALUE:}
\begin{verbatim}    The ESMC_Mesh object representing the XGrid. \end{verbatim}
{\sf DESCRIPTION:\\ }


  
    Get the ESMC\_Mesh object representing the XGrid. 
  
    The arguments are:
    \begin{description}
    \item[xgrid]
      The xgrid from which to get the information. 
    \item[{[rc]}]
      Return code; equals {\tt ESMF\_SUCCESS} if there are no errors.
    \end{description}
   
%/////////////////////////////////////////////////////////////
 
\mbox{}\hrulefill\ 
 
\subsubsection [ESMC\_XGridGetElementArea] {ESMC\_XGridGetElementArea - Get the area of elements in the XGrid.}


  
\bigskip{\sf INTERFACE:}
\begin{verbatim} void ESMC_XGridGetElementArea(
   ESMC_XGrid xgrid,     // in
   ESMC_R8 *area,        // out
   int *rc               // out
 );
 \end{verbatim}{\em RETURN VALUE:}
\begin{verbatim}       The number of elements in the XGrid.\end{verbatim}
{\sf DESCRIPTION:\\ }


  
      Get the number of elements in the XGrid.
  
    The arguments are:
    \begin{description}
    \item[xgrid]
      The XGrid from which to get the information.
    \item[area]
      An array to fill with element areas. The array must be allocated
      to size elementCount.
    \item[{[rc]}]
      Return code; equals {\tt ESMF\_SUCCESS} if there are no errors.
    \end{description}
   
%/////////////////////////////////////////////////////////////
 
\mbox{}\hrulefill\ 
 
\subsubsection [ESMC\_XGridGetElementCentroid] {ESMC\_XGridGetElementCentroid - Get the centroid of elements in the XGrid.}


  
\bigskip{\sf INTERFACE:}
\begin{verbatim} void ESMC_XGridGetElementCentroid(
   ESMC_XGrid xgrid,     // in
   ESMC_R8 *centroid,    // out
   int *rc               // out
 );
 \end{verbatim}{\em RETURN VALUE:}
\begin{verbatim}       The number of elements in the XGrid.\end{verbatim}
{\sf DESCRIPTION:\\ }


  
      Get the centroid for each element in the exchange grid. 
  
    The arguments are:
    \begin{description}
    \item[xgrid]
      The XGrid from which to get the information.
    \item[centroid]
      An array to fill with element centroids. The array must be allocated
      to size elementCount*dimCount.
    \item[{[rc]}]
      Return code; equals {\tt ESMF\_SUCCESS} if there are no errors.
    \end{description}
   
%/////////////////////////////////////////////////////////////
 
\mbox{}\hrulefill\ 
 
\subsubsection [ESMC\_XGridGetSparseMatA2X] {ESMC\_XGridGetSparseMatA2X - Get the sparse matrix that goes from a side A grid to the exchange grid.}


 
  
\bigskip{\sf INTERFACE:}
\begin{verbatim} void ESMC_XGridGetSparseMatA2X(
                                ESMC_XGrid xgrid,      // in
                                int sideAIndex,        // in
                                int *factorListCount,  // out
                                double **factorList,   // out
                                int **factorIndexList, // out
                                int *rc);
 \end{verbatim}{\em RETURN VALUE:}
\begin{verbatim}       N/A\end{verbatim}
{\sf DESCRIPTION:\\ }


   
      Get the sparse matrix that goes from a side A grid to the exchange grid.
  
    The arguments are:
    \begin{description}
    \item[xgrid]
      The XGrid from which to get the information.
    \item[sideAIndex]
      The 0 based index of the Grid/Mesh on side A to get the sparse matrix for.
      If a priority has been specified for Grids and Meshes, then this index is 
      in priority order. If no priority was specified, then the all the Grids are
      first followed by all the Meshes in the order they were passed into the XGrid 
      create call. 
    \item[factorListCount]
      The size of the sparse matrix.
    \item[factorList]
      A pointer to the list of factors for the requested sparse matrix. 
      The list is of size {\tt factorListCount}. To save space
      this is a pointer to the internal xgrid memory for this list. 
      Don't deallocate it. 
    \item[factorIndexList]
      A pointer to the list of indices for the requested sparse matrix. 
      The list is of size 2*{\tt factorListCount}. For each pair of entries
      in this array: entry 0 is the sequence index in the source grid, and entry 1 is
      the sequence index in the destination grid. To save space
      this is a pointer to the internal xgrid memory for this list. 
      Don't deallocate it. 
    \item[{[rc]}]
      Return code; equals {\tt ESMF\_SUCCESS} if there are no errors.
    \end{description}
   
%/////////////////////////////////////////////////////////////
 
\mbox{}\hrulefill\ 
 
\subsubsection [ESMC\_XGridGetSparseMatA2X] {ESMC\_XGridGetSparseMatA2X - Get the sparse matrix that goes from the exchange grid to a side A grid.}


 
  
\bigskip{\sf INTERFACE:}
\begin{verbatim} void ESMC_XGridGetSparseMatX2A(
                                ESMC_XGrid xgrid,      // in
                                int sideAIndex,        // in
                                int *factorListCount,  // out
                                double **factorList,   // out
                                int **factorIndexList, // out
                                int *rc);
 \end{verbatim}{\em RETURN VALUE:}
\begin{verbatim}       N/A\end{verbatim}
{\sf DESCRIPTION:\\ }


   
      Get the sparse matrix that goes from the exchange grid to a side A grid. 
  
    The arguments are:
    \begin{description}
    \item[xgrid]
      The XGrid from which to get the information.
    \item[sideAIndex]
      The 0 based index of the Grid/Mesh on side A to get the sparse matrix for.
      If a priority has been specified for Grids and Meshes, then this index is 
      in priority order. If no priority was specified, then the all the Grids are
      first followed by all the Meshes in the order they were passed into the XGrid 
      create call. 
    \item[factorListCount]
      The size of the sparse matrix.
    \item[factorList]
      A pointer to the list of factors for the requested sparse matrix. 
      The list is of size {\tt factorListCount}. To save space
      this is a pointer to the internal xgrid memory for this list. 
      Don't deallocate it. 
    \item[factorIndexList]
      A pointer to the list of indices for the requested sparse matrix. 
      The list is of size 2*{\tt factorListCount}. For each pair of entries
      in this array: entry 0 is the sequence index in the source grid, and entry 1 is
      the sequence index in the destination grid. To save space 
      this is a pointer to the internal xgrid memory for this list. 
      Don't deallocate it. 
    \item[{[rc]}]
      Return code; equals {\tt ESMF\_SUCCESS} if there are no errors.
    \end{description}
   
%/////////////////////////////////////////////////////////////
 
\mbox{}\hrulefill\ 
 
\subsubsection [ESMC\_XGridGetSparseMatB2X] {ESMC\_XGridGetSparseMatB2X - Get the sparse matrix that goes from a side B grid to the exchange grid.}


 
  
\bigskip{\sf INTERFACE:}
\begin{verbatim} void ESMC_XGridGetSparseMatB2X(
                                ESMC_XGrid xgrid,      // in
                                int sideBIndex,        // in
                                int *factorListCount,  // out
                                double **factorList,   // out
                                int **factorIndexList, // out
                                int *rc);
 \end{verbatim}{\em RETURN VALUE:}
\begin{verbatim}       N/A\end{verbatim}
{\sf DESCRIPTION:\\ }


   
      Get the sparse matrix that goes from a side B grid to the exchange grid.
  
    The arguments are:
    \begin{description}
    \item[xgrid]
      The XGrid from which to get the information.
    \item[sideBIndex]
      The 0 based index of the Grid/Mesh on side B to get the sparse matrix for.
      If a priority has been specified for Grids and Meshes, then this index is 
      in priority order. If no priority was specified, then the all the Grids are
      first followed by all the Meshes in the order they were passed into the XGrid 
      create call. 
    \item[factorListCount]
      The size of the sparse matrix.
    \item[factorList]
      A pointer to the list of factors for the requested sparse matrix. 
      The list is of size {\tt factorListCount}. To save space
      this is a pointer to the internal xgrid memory for this list. 
      Don't deallocate it. 
    \item[factorIndexList]
      A pointer to the list of indices for the requested sparse matrix. 
      The list is of size 2*{\tt factorListCount}. For each pair of entries
      in this array: entry 0 is the sequence index in the source grid, and entry 1 is
      the sequence index in the destination grid. To save space
      this is a pointer to the internal xgrid memory for this list. 
      Don't deallocate it. 
    \item[{[rc]}]
      Return code; equals {\tt ESMF\_SUCCESS} if there are no errors.
    \end{description}
   
%/////////////////////////////////////////////////////////////
 
\mbox{}\hrulefill\ 
 
\subsubsection [ESMC\_XGridGetSparseMatB2X] {ESMC\_XGridGetSparseMatB2X - Get the sparse matrix that goes from the exchange grid to a side B grid.}


 
  
\bigskip{\sf INTERFACE:}
\begin{verbatim} void ESMC_XGridGetSparseMatX2B(
                                ESMC_XGrid xgrid,      // in
                                int sideBIndex,        // in
                                int *factorListCount,  // out
                                double **factorList,   // out
                                int **factorIndexList, // out
                                int *rc);
 \end{verbatim}{\em RETURN VALUE:}
\begin{verbatim}       N/A\end{verbatim}
{\sf DESCRIPTION:\\ }


   
      Get the sparse matrix that goes from the exchange grid to a side B grid. 
  
    The arguments are:
    \begin{description}
    \item[xgrid]
      The XGrid from which to get the information.
    \item[sideBIndex]
      The 0 based index of the Grid/Mesh on side B to get the sparse matrix for.
      If a priority has been specified for Grids and Meshes, then this index is 
      in priority order. If no priority was specified, then the all the Grids are
      first followed by all the Meshes in the order they were passed into the XGrid 
      create call. 
    \item[factorListCount]
      The size of the sparse matrix.
    \item[factorList]
      A pointer to the list of factors for the requested sparse matrix. 
      The list is of size {\tt factorListCount}. To save space
      this is a pointer to the internal xgrid memory for this list. 
      Don't deallocate it. 
    \item[factorIndexList]
      A pointer to the list of indices for the requested sparse matrix. 
      The list is of size 2*{\tt factorListCount}. For each pair of entries
      in this array: entry 0 is the sequence index in the source grid, and entry 1 is
      the sequence index in the destination grid. To save space 
      this is a pointer to the internal xgrid memory for this list. 
      Don't deallocate it. 
    \item[{[rc]}]
      Return code; equals {\tt ESMF\_SUCCESS} if there are no errors.
    \end{description}
  
%...............................................................
\setlength{\parskip}{\oldparskip}
\setlength{\parindent}{\oldparindent}
\setlength{\baselineskip}{\oldbaselineskip}
