%                **** IMPORTANT NOTICE *****
% This LaTeX file has been automatically produced by ProTeX v. 1.1
% Any changes made to this file will likely be lost next time
% this file is regenerated from its source. Send questions 
% to Arlindo da Silva, dasilva@gsfc.nasa.gov
 
\setlength{\oldparskip}{\parskip}
\setlength{\parskip}{1.5ex}
\setlength{\oldparindent}{\parindent}
\setlength{\parindent}{0pt}
\setlength{\oldbaselineskip}{\baselineskip}
\setlength{\baselineskip}{11pt}
 
%--------------------- SHORT-HAND MACROS ----------------------
\def\bv{\begin{verbatim}}
\def\ev{\end{verbatim}}
\def\be{\begin{equation}}
\def\ee{\end{equation}}
\def\bea{\begin{eqnarray}}
\def\eea{\end{eqnarray}}
\def\bi{\begin{itemize}}
\def\ei{\end{itemize}}
\def\bn{\begin{enumerate}}
\def\en{\end{enumerate}}
\def\bd{\begin{description}}
\def\ed{\end{description}}
\def\({\left (}
\def\){\right )}
\def\[{\left [}
\def\]{\right ]}
\def\<{\left  \langle}
\def\>{\right \rangle}
\def\cI{{\cal I}}
\def\diag{\mathop{\rm diag}}
\def\tr{\mathop{\rm tr}}
%-------------------------------------------------------------

\markboth{Left}{Source File: ESMC\_Array.h,  Date: Tue May  5 20:59:42 MDT 2020
}

 
%/////////////////////////////////////////////////////////////
\subsubsection [ESMC\_ArrayCreate] {ESMC\_ArrayCreate - Create an Array}


  
\bigskip{\sf INTERFACE:}
\begin{verbatim} ESMC_Array ESMC_ArrayCreate(
   ESMC_ArraySpec arrayspec,   // in
   ESMC_DistGrid distgrid,     // in
   const char* name,           // in
   int *rc                     // out
 );\end{verbatim}{\em RETURN VALUE:}
\begin{verbatim}    Newly created ESMC_Array object.\end{verbatim}
{\sf DESCRIPTION:\\ }


  
    Create an {\tt ESMC\_Array} object.
  
    The arguments are:
    \begin{description}
    \item[arrayspec]
      {\tt ESMC\_ArraySpec} object containing the type/kind/rank information.
    \item[distgrid]
      {\tt ESMC\_DistGrid} object that describes how the Array is decomposed and
      distributed over DEs. The dimCount of distgrid must be smaller or equal
      to the rank specified in arrayspec, otherwise a runtime ESMF error will be
      raised.
    \item[{[name]}]
      The name for the Array object. If not specified, i.e. NULL,
      a default unique name will be generated: "ArrayNNN" where NNN
      is a unique sequence number from 001 to 999.
    \item[{[rc]}]
      Return code; equals {\tt ESMF\_SUCCESS} if there are no errors.
    \end{description}
   
%/////////////////////////////////////////////////////////////
 
\mbox{}\hrulefill\ 
 
\subsubsection [ESMC\_ArrayDestroy] {ESMC\_ArrayDestroy - Destroy an Array}


  
\bigskip{\sf INTERFACE:}
\begin{verbatim} int ESMC_ArrayDestroy(
   ESMC_Array *array           // inout
 );\end{verbatim}{\em RETURN VALUE:}
\begin{verbatim}    Return code; equals ESMF_SUCCESS if there are no errors.\end{verbatim}
{\sf DESCRIPTION:\\ }


  
    Destroy an {\tt ESMC\_Array} object.
  
    The arguments are:
    \begin{description}
    \item[array] 
      {\tt ESMC\_Array} object to be destroyed.
    \end{description}
   
%/////////////////////////////////////////////////////////////
 
\mbox{}\hrulefill\ 
 
\subsubsection [ESMC\_ArrayGetName] {ESMC\_ArrayGetName - Get the name of an Array}


  
\bigskip{\sf INTERFACE:}
\begin{verbatim} const char *ESMC_ArrayGetName(
   ESMC_Array array,           // in
   int *rc                     // out
 );\end{verbatim}{\em RETURN VALUE:}
\begin{verbatim}    Pointer to the Array name string.\end{verbatim}
{\sf DESCRIPTION:\\ }


  
    Get the name of the specified {\tt ESMC\_Array} object.
  
    The arguments are:
    \begin{description}
    \item[array] 
      {\tt ESMC\_Array} object to be queried.
    \item[{[rc]}]
      Return code; equals {\tt ESMF\_SUCCESS} if there are no errors.
    \end{description}
   
%/////////////////////////////////////////////////////////////
 
\mbox{}\hrulefill\ 
 
\subsubsection [ESMC\_ArrayGetPtr] {ESMC\_ArrayGetPtr - Get pointer to Array data.}


  
\bigskip{\sf INTERFACE:}
\begin{verbatim} void *ESMC_ArrayGetPtr(
   ESMC_Array array,           // in
   int localDe,                // in
   int *rc                     // out
 );\end{verbatim}{\em RETURN VALUE:}
\begin{verbatim}    Pointer to the Array data.\end{verbatim}
{\sf DESCRIPTION:\\ }


  
    Get pointer to the data of the specified {\tt ESMC\_Array} object.
  
    The arguments are:
    \begin{description}
    \item[array] 
      {\tt ESMC\_Array} object to be queried.
    \item[localDe] 
      Local De for which to data pointer is queried.
    \item[{[rc]}]
      Return code; equals {\tt ESMF\_SUCCESS} if there are no errors.
    \end{description}
   
%/////////////////////////////////////////////////////////////
 
\mbox{}\hrulefill\ 
 
\subsubsection [ESMC\_ArrayPrint] {ESMC\_ArrayPrint - Print an Array}


  
\bigskip{\sf INTERFACE:}
\begin{verbatim} int ESMC_ArrayPrint(
   ESMC_Array array            // in
 );\end{verbatim}{\em RETURN VALUE:}
\begin{verbatim}    Return code; equals ESMF_SUCCESS if there are no errors.\end{verbatim}
{\sf DESCRIPTION:\\ }


  
    Print internal information of the specified {\tt ESMC\_Array} object.
  
    The arguments are:
    \begin{description}
    \item[array] 
      {\tt ESMC\_Array} object to be printed.
    \end{description}
  
%...............................................................
\setlength{\parskip}{\oldparskip}
\setlength{\parindent}{\oldparindent}
\setlength{\baselineskip}{\oldbaselineskip}
