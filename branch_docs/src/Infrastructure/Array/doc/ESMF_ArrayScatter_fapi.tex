%                **** IMPORTANT NOTICE *****
% This LaTeX file has been automatically produced by ProTeX v. 1.1
% Any changes made to this file will likely be lost next time
% this file is regenerated from its source. Send questions 
% to Arlindo da Silva, dasilva@gsfc.nasa.gov
 
\setlength{\oldparskip}{\parskip}
\setlength{\parskip}{1.5ex}
\setlength{\oldparindent}{\parindent}
\setlength{\parindent}{0pt}
\setlength{\oldbaselineskip}{\baselineskip}
\setlength{\baselineskip}{11pt}
 
%--------------------- SHORT-HAND MACROS ----------------------
\def\bv{\begin{verbatim}}
\def\ev{\end{verbatim}}
\def\be{\begin{equation}}
\def\ee{\end{equation}}
\def\bea{\begin{eqnarray}}
\def\eea{\end{eqnarray}}
\def\bi{\begin{itemize}}
\def\ei{\end{itemize}}
\def\bn{\begin{enumerate}}
\def\en{\end{enumerate}}
\def\bd{\begin{description}}
\def\ed{\end{description}}
\def\({\left (}
\def\){\right )}
\def\[{\left [}
\def\]{\right ]}
\def\<{\left  \langle}
\def\>{\right \rangle}
\def\cI{{\cal I}}
\def\diag{\mathop{\rm diag}}
\def\tr{\mathop{\rm tr}}
%-------------------------------------------------------------

\markboth{Left}{Source File: ESMF\_ArrayScatter.F90,  Date: Tue May  5 20:59:43 MDT 2020
}

 
%/////////////////////////////////////////////////////////////

   \subsubsection [ESMF\_ArrayScatter] {ESMF\_ArrayScatter - Scatter a Fortran array across the ESMF\_Array }


   
\bigskip{\sf INTERFACE:}
\begin{verbatim}   subroutine ESMF_ArrayScatter(array, farray, rootPet, tile, vm, rc) 
   \end{verbatim}{\em ARGUMENTS:}
\begin{verbatim}   type(ESMF_Array), intent(inout) :: array 
   <type> (ESMF_KIND_<kind>), intent(in), target :: farray(<rank>) 
   integer, intent(in) :: rootPet 
 -- The following arguments require argument keyword syntax (e.g. rc=rc). --
   integer, intent(in), optional :: tile 
   type(ESMF_VM), intent(in), optional :: vm 
   integer, intent(out), optional :: rc 
   \end{verbatim}
{\sf STATUS:}
   \begin{itemize} 
   \item\apiStatusCompatibleVersion{5.2.0r} 
   \end{itemize} 
   
{\sf DESCRIPTION:\\ }

 
   Scatter the data of {\tt farray} located on {\tt rootPET} 
   across an {ESMF\_Array} object. A single {\tt farray} must be 
   scattered across a single DistGrid tile in Array. The optional {\tt tile} 
   argument allows selection of the tile. For Arrays defined on a single 
   tile DistGrid the default selection (tile 1) will be correct. The 
   shape of {\tt farray} must match the shape of the tile in Array. 
   
   If the Array contains replicating DistGrid dimensions data will be 
   scattered across all of the replicated pieces. 
   
   This version of the interface implements the PET-based blocking paradigm: 
   Each PET of the VM must issue this call exactly once for {\em all} of its 
   DEs. The call will block until all PET-local data objects are accessible. 
   
   The arguments are: 
   \begin{description} 
   \item[array] 
   The {\tt ESMF\_Array} object across which data will be scattered. 
   \item[\{farray\}] 
   The Fortran array that is to be scattered. Only root 
   must provide a valid {\tt farray}, the other PETs may treat 
   {\tt farray} as an optional argument. 
   \item[rootPet] 
   PET that holds the valid data in {\tt farray}. 
   \item[{[tile]}] 
   The DistGrid tile in {\tt array} into which to scatter {\tt farray}. 
   By default {\tt farray} will be scattered into tile 1. 
   \item[{[vm]}] 
   Optional {\tt ESMF\_VM} object of the current context. Providing the 
   VM of the current context will lower the method's overhead. 
   \item[{[rc]}] 
   Return code; equals {\tt ESMF\_SUCCESS} if there are no errors. 
   \end{description} 
   
%...............................................................
\setlength{\parskip}{\oldparskip}
\setlength{\parindent}{\oldparindent}
\setlength{\baselineskip}{\oldbaselineskip}
