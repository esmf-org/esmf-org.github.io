%                **** IMPORTANT NOTICE *****
% This LaTeX file has been automatically produced by ProTeX v. 1.1
% Any changes made to this file will likely be lost next time
% this file is regenerated from its source. Send questions 
% to Arlindo da Silva, dasilva@gsfc.nasa.gov
 
\setlength{\oldparskip}{\parskip}
\setlength{\parskip}{1.5ex}
\setlength{\oldparindent}{\parindent}
\setlength{\parindent}{0pt}
\setlength{\oldbaselineskip}{\baselineskip}
\setlength{\baselineskip}{11pt}
 
%--------------------- SHORT-HAND MACROS ----------------------
\def\bv{\begin{verbatim}}
\def\ev{\end{verbatim}}
\def\be{\begin{equation}}
\def\ee{\end{equation}}
\def\bea{\begin{eqnarray}}
\def\eea{\end{eqnarray}}
\def\bi{\begin{itemize}}
\def\ei{\end{itemize}}
\def\bn{\begin{enumerate}}
\def\en{\end{enumerate}}
\def\bd{\begin{description}}
\def\ed{\end{description}}
\def\({\left (}
\def\){\right )}
\def\[{\left [}
\def\]{\right ]}
\def\<{\left  \langle}
\def\>{\right \rangle}
\def\cI{{\cal I}}
\def\diag{\mathop{\rm diag}}
\def\tr{\mathop{\rm tr}}
%-------------------------------------------------------------

\markboth{Left}{Source File: ESMF\_DynamicMask.F90,  Date: Tue May  5 20:59:42 MDT 2020
}

 
%/////////////////////////////////////////////////////////////
\subsubsection [ESMF\_DynamicMaskSetR8R8R8] {ESMF\_DynamicMaskSetR8R8R8 - Set DynamicMask for R8R8R8}


\bigskip{\sf INTERFACE:}
\begin{verbatim}   subroutine ESMF_DynamicMaskSetR8R8R8(dynamicMask, dynamicMaskRoutine, &
     handleAllElements, dynamicSrcMaskValue, &
     dynamicDstMaskValue, rc)\end{verbatim}{\em ARGUMENTS:}
\begin{verbatim}     type(ESMF_DynamicMask), intent(out)           :: dynamicMask
     procedure(ESMF_DynamicMaskRoutineR8R8R8)      :: dynamicMaskRoutine
 -- The following arguments require argument keyword syntax (e.g. rc=rc). --
     logical,                intent(in),  optional :: handleAllElements
     real(ESMF_KIND_R8),     intent(in),  optional :: dynamicSrcMaskValue
     real(ESMF_KIND_R8),     intent(in),  optional :: dynamicDstMaskValue
     integer,                intent(out), optional :: rc
           \end{verbatim}
{\sf DESCRIPTION:\\ }


     \label{api:DynamicMaskSetR8R8R8}
     Set an {\tt ESMF\_DynamicMask} object suitable for 
     destination element typekind {\tt ESMF\_TYPEKIND\_R8},
     factor typekind {\tt ESMF\_TYPEKIND\_R8}, and
     source element typekind {\tt ESMF\_TYPEKIND\_R8}.
     
     All values in {\tt dynamicMask} will be reset by this call.
  
     See section \ref{RH:DynMask} for a general discussion of dynamic masking.
  
     The arguments are:
     \begin{description}
     \item[dynamicMask] 
       DynamicMask object.
     \item [dynamicMaskRoutine]
       The routine responsible for handling dynamically masked source and 
       destination elements. See section \ref{RH:DynMask} for the precise
       definition of the {\tt dynamicMaskRoutine} procedure interface.
       The routine is only called on PETs where {\em at least one} interpolation 
       element is identified for special handling.
     \item [{[handleAllElements]}]
       If set to {\tt .true.}, all local elements, regardless of their dynamic
       masking status, are made available to {\tt dynamicMaskRoutine} for
       handling. This option can be used to implement fully customized
       interpolations based on the information provided by ESMF.
       The default is {\tt .false.}, meaning that only elements affected by
       dynamic masking will be handed to {\tt dynamicMaskRoutine}.
     \item [{[dynamicSrcMaskValue]}]
       The value for which a source element is considered dynamically
       masked.
       The default is to {\em not} consider any source elements as
       dynamically masked.
     \item [{[dynamicDstMaskValue]}]
       The value for which a destination element is considered dynamically
       masked.
       The default is to {\em not} consider any destination elements as
       dynamically masked.
     \item[{[rc]}] 
       Return code; equals {\tt ESMF\_SUCCESS} if there are no errors.
     \end{description}
   
%/////////////////////////////////////////////////////////////
 
\mbox{}\hrulefill\ 
 
\subsubsection [ESMF\_DynamicMaskSetR8R8R8V] {ESMF\_DynamicMaskSetR8R8R8V - Set DynamicMask for R8R8R8 with vectorization}


\bigskip{\sf INTERFACE:}
\begin{verbatim}   subroutine ESMF_DynamicMaskSetR8R8R8V(dynamicMask, dynamicMaskRoutine, &
     handleAllElements, dynamicSrcMaskValue, &
     dynamicDstMaskValue, rc)\end{verbatim}{\em ARGUMENTS:}
\begin{verbatim}     type(ESMF_DynamicMask), intent(out)           :: dynamicMask
     procedure(ESMF_DynamicMaskRoutineR8R8R8V)     :: dynamicMaskRoutine
 -- The following arguments require argument keyword syntax (e.g. rc=rc). --
     logical,                intent(in),  optional :: handleAllElements
     real(ESMF_KIND_R8),     intent(in),  optional :: dynamicSrcMaskValue
     real(ESMF_KIND_R8),     intent(in),  optional :: dynamicDstMaskValue
     integer,                intent(out), optional :: rc
           \end{verbatim}
{\sf DESCRIPTION:\\ }


     \label{api:DynamicMaskSetR8R8R8V}
     Set an {\tt ESMF\_DynamicMask} object suitable for 
     destination element typekind {\tt ESMF\_TYPEKIND\_R8},
     factor typekind {\tt ESMF\_TYPEKIND\_R8}, and
     source element typekind {\tt ESMF\_TYPEKIND\_R8}.
     
     All values in {\tt dynamicMask} will be reset by this call.
  
     See section \ref{RH:DynMask} for a general discussion of dynamic masking.
  
     The arguments are:
     \begin{description}
     \item[dynamicMask] 
       DynamicMask object.
     \item [dynamicMaskRoutine]
       The routine responsible for handling dynamically masked source and 
       destination elements. See section \ref{RH:DynMask} for the precise
       definition of the {\tt dynamicMaskRoutine} procedure interface.
       The routine is only called on PETs where {\em at least one} interpolation 
       element is identified for special handling.
     \item [{[handleAllElements]}]
       If set to {\tt .true.}, all local elements, regardless of their dynamic
       masking status, are made available to {\tt dynamicMaskRoutine} for
       handling. This option can be used to implement fully customized
       interpolations based on the information provided by ESMF.
       The default is {\tt .false.}, meaning that only elements affected by
       dynamic masking will be handed to {\tt dynamicMaskRoutine}.
     \item [{[dynamicSrcMaskValue]}]
       The value for which a source element is considered dynamically
       masked.
       The default is to {\em not} consider any source elements as
       dynamically masked.
     \item [{[dynamicDstMaskValue]}]
       The value for which a destination element is considered dynamically
       masked.
       The default is to {\em not} consider any destination elements as
       dynamically masked.
     \item[{[rc]}] 
       Return code; equals {\tt ESMF\_SUCCESS} if there are no errors.
     \end{description}
   
%/////////////////////////////////////////////////////////////
 
\mbox{}\hrulefill\ 
 
\subsubsection [ESMF\_DynamicMaskSetR4R8R4] {ESMF\_DynamicMaskSetR4R8R4 - Set DynamicMask for R4R8R4}


\bigskip{\sf INTERFACE:}
\begin{verbatim}   subroutine ESMF_DynamicMaskSetR4R8R4(dynamicMask, dynamicMaskRoutine, &
     handleAllElements, dynamicSrcMaskValue, &
     dynamicDstMaskValue, rc)\end{verbatim}{\em ARGUMENTS:}
\begin{verbatim}     type(ESMF_DynamicMask), intent(out)           :: dynamicMask
     procedure(ESMF_DynamicMaskRoutineR4R8R4)      :: dynamicMaskRoutine
 -- The following arguments require argument keyword syntax (e.g. rc=rc). --
     logical,                intent(in),  optional :: handleAllElements
     real(ESMF_KIND_R4),     intent(in),  optional :: dynamicSrcMaskValue
     real(ESMF_KIND_R4),     intent(in),  optional :: dynamicDstMaskValue
     integer,                intent(out), optional :: rc
           \end{verbatim}
{\sf DESCRIPTION:\\ }


     \label{api:DynamicMaskSetR8R8R8}
     Set an {\tt ESMF\_DynamicMask} object suitable for 
     destination element typekind {\tt ESMF\_TYPEKIND\_R4},
     factor typekind {\tt ESMF\_TYPEKIND\_R8}, and
     source element typekind {\tt ESMF\_TYPEKIND\_R4}.
     
     All values in {\tt dynamicMask} will be reset by this call.
  
     See section \ref{RH:DynMask} for a general discussion of dynamic masking.
  
     The arguments are:
     \begin{description}
     \item[dynamicMask] 
       DynamicMask object.
     \item [dynamicMaskRoutine]
       The routine responsible for handling dynamically masked source and 
       destination elements. See section \ref{RH:DynMask} for the precise
       definition of the {\tt dynamicMaskRoutine} procedure interface.
       The routine is only called on PETs where {\em at least one} interpolation 
       element is identified for special handling.
     \item [{[handleAllElements]}]
       If set to {\tt .true.}, all local elements, regardless of their dynamic
       masking status, are made available to {\tt dynamicMaskRoutine} for
       handling. This option can be used to implement fully customized
       interpolations based on the information provided by ESMF.
       The default is {\tt .false.}, meaning that only elements affected by
       dynamic masking will be handed to {\tt dynamicMaskRoutine}.
     \item [{[dynamicSrcMaskValue]}]
       The value for which a source element is considered dynamically
       masked.
       The default is to {\em not} consider any source elements as
       dynamically masked.
     \item [{[dynamicDstMaskValue]}]
       The value for which a destination element is considered dynamically
       masked.
       The default is to {\em not} consider any destination elements as
       dynamically masked.
     \item[{[rc]}] 
       Return code; equals {\tt ESMF\_SUCCESS} if there are no errors.
     \end{description}
   
%/////////////////////////////////////////////////////////////
 
\mbox{}\hrulefill\ 
 
\subsubsection [ESMF\_DynamicMaskSetR4R8R4V] {ESMF\_DynamicMaskSetR4R8R4V - Set DynamicMask for R4R8R4 with vectorization}


\bigskip{\sf INTERFACE:}
\begin{verbatim}   subroutine ESMF_DynamicMaskSetR4R8R4V(dynamicMask, dynamicMaskRoutine, &
     handleAllElements, dynamicSrcMaskValue, &
     dynamicDstMaskValue, rc)\end{verbatim}{\em ARGUMENTS:}
\begin{verbatim}     type(ESMF_DynamicMask), intent(out)           :: dynamicMask
     procedure(ESMF_DynamicMaskRoutineR4R8R4V)     :: dynamicMaskRoutine
 -- The following arguments require argument keyword syntax (e.g. rc=rc). --
     logical,                intent(in),  optional :: handleAllElements
     real(ESMF_KIND_R4),     intent(in),  optional :: dynamicSrcMaskValue
     real(ESMF_KIND_R4),     intent(in),  optional :: dynamicDstMaskValue
     integer,                intent(out), optional :: rc
           \end{verbatim}
{\sf DESCRIPTION:\\ }


     \label{api:DynamicMaskSetR4R8R4V}
     Set an {\tt ESMF\_DynamicMask} object suitable for 
     destination element typekind {\tt ESMF\_TYPEKIND\_R4},
     factor typekind {\tt ESMF\_TYPEKIND\_R8}, and
     source element typekind {\tt ESMF\_TYPEKIND\_R4}.
     
     All values in {\tt dynamicMask} will be reset by this call.
  
     See section \ref{RH:DynMask} for a general discussion of dynamic masking.
  
     The arguments are:
     \begin{description}
     \item[dynamicMask] 
       DynamicMask object.
     \item [dynamicMaskRoutine]
       The routine responsible for handling dynamically masked source and 
       destination elements. See section \ref{RH:DynMask} for the precise
       definition of the {\tt dynamicMaskRoutine} procedure interface.
       The routine is only called on PETs where {\em at least one} interpolation 
       element is identified for special handling.
     \item [{[handleAllElements]}]
       If set to {\tt .true.}, all local elements, regardless of their dynamic
       masking status, are made available to {\tt dynamicMaskRoutine} for
       handling. This option can be used to implement fully customized
       interpolations based on the information provided by ESMF.
       The default is {\tt .false.}, meaning that only elements affected by
       dynamic masking will be handed to {\tt dynamicMaskRoutine}.
     \item [{[dynamicSrcMaskValue]}]
       The value for which a source element is considered dynamically
       masked.
       The default is to {\em not} consider any source elements as
       dynamically masked.
     \item [{[dynamicDstMaskValue]}]
       The value for which a destination element is considered dynamically
       masked.
       The default is to {\em not} consider any destination elements as
       dynamically masked.
     \item[{[rc]}] 
       Return code; equals {\tt ESMF\_SUCCESS} if there are no errors.
     \end{description}
   
%/////////////////////////////////////////////////////////////
 
\mbox{}\hrulefill\ 
 
\subsubsection [ESMF\_DynamicMaskSetR4R4R4] {ESMF\_DynamicMaskSetR4R4R4 - Set DynamicMask for R4R4R4}


\bigskip{\sf INTERFACE:}
\begin{verbatim}   subroutine ESMF_DynamicMaskSetR4R4R4(dynamicMask, dynamicMaskRoutine, &
     handleAllElements, dynamicSrcMaskValue, &
     dynamicDstMaskValue, rc)\end{verbatim}{\em ARGUMENTS:}
\begin{verbatim}     type(ESMF_DynamicMask), intent(out)           :: dynamicMask
     procedure(ESMF_DynamicMaskRoutineR4R4R4)      :: dynamicMaskRoutine
 -- The following arguments require argument keyword syntax (e.g. rc=rc). --
     logical,                intent(in),  optional :: handleAllElements
     real(ESMF_KIND_R4),     intent(in),  optional :: dynamicSrcMaskValue
     real(ESMF_KIND_R4),     intent(in),  optional :: dynamicDstMaskValue
     integer,                intent(out), optional :: rc
           \end{verbatim}
{\sf DESCRIPTION:\\ }


     \label{api:DynamicMaskSetR8R8R8}
     Set an {\tt ESMF\_DynamicMask} object suitable for 
     destination element typekind {\tt ESMF\_TYPEKIND\_R4},
     factor typekind {\tt ESMF\_TYPEKIND\_R4}, and
     source element typekind {\tt ESMF\_TYPEKIND\_R4}.
     
     All values in {\tt dynamicMask} will be reset by this call.
  
     See section \ref{RH:DynMask} for a general discussion of dynamic masking.
  
     The arguments are:
     \begin{description}
     \item[dynamicMask] 
       DynamicMask object.
     \item [dynamicMaskRoutine]
       The routine responsible for handling dynamically masked source and 
       destination elements. See section \ref{RH:DynMask} for the precise
       definition of the {\tt dynamicMaskRoutine} procedure interface.
       The routine is only called on PETs where {\em at least one} interpolation 
       element is identified for special handling.
     \item [{[handleAllElements]}]
       If set to {\tt .true.}, all local elements, regardless of their dynamic
       masking status, are made available to {\tt dynamicMaskRoutine} for
       handling. This option can be used to implement fully customized
       interpolations based on the information provided by ESMF.
       The default is {\tt .false.}, meaning that only elements affected by
       dynamic masking will be handed to {\tt dynamicMaskRoutine}.
     \item [{[dynamicSrcMaskValue]}]
       The value for which a source element is considered dynamically
       masked.
       The default is to {\em not} consider any source elements as
       dynamically masked.
     \item [{[dynamicDstMaskValue]}]
       The value for which a destination element is considered dynamically
       masked.
       The default is to {\em not} consider any destination elements as
       dynamically masked.
     \item[{[rc]}] 
       Return code; equals {\tt ESMF\_SUCCESS} if there are no errors.
     \end{description}
   
%/////////////////////////////////////////////////////////////
 
\mbox{}\hrulefill\ 
 
\subsubsection [ESMF\_DynamicMaskSetR4R4R4V] {ESMF\_DynamicMaskSetR4R4R4V - Set DynamicMask for R4R4R4 with vectorization}


\bigskip{\sf INTERFACE:}
\begin{verbatim}   subroutine ESMF_DynamicMaskSetR4R4R4V(dynamicMask, dynamicMaskRoutine, &
     handleAllElements, dynamicSrcMaskValue, &
     dynamicDstMaskValue, rc)\end{verbatim}{\em ARGUMENTS:}
\begin{verbatim}     type(ESMF_DynamicMask), intent(out)           :: dynamicMask
     procedure(ESMF_DynamicMaskRoutineR4R4R4V)     :: dynamicMaskRoutine
 -- The following arguments require argument keyword syntax (e.g. rc=rc). --
     logical,                intent(in),  optional :: handleAllElements
     real(ESMF_KIND_R4),     intent(in),  optional :: dynamicSrcMaskValue
     real(ESMF_KIND_R4),     intent(in),  optional :: dynamicDstMaskValue
     integer,                intent(out), optional :: rc
           \end{verbatim}
{\sf DESCRIPTION:\\ }


     \label{api:DynamicMaskSetR4R4R4V}
     Set an {\tt ESMF\_DynamicMask} object suitable for 
     destination element typekind {\tt ESMF\_TYPEKIND\_R4},
     factor typekind {\tt ESMF\_TYPEKIND\_R4}, and
     source element typekind {\tt ESMF\_TYPEKIND\_R4}.
     
     All values in {\tt dynamicMask} will be reset by this call.
  
     See section \ref{RH:DynMask} for a general discussion of dynamic masking.
  
     The arguments are:
     \begin{description}
     \item[dynamicMask] 
       DynamicMask object.
     \item [dynamicMaskRoutine]
       The routine responsible for handling dynamically masked source and 
       destination elements. See section \ref{RH:DynMask} for the precise
       definition of the {\tt dynamicMaskRoutine} procedure interface.
       The routine is only called on PETs where {\em at least one} interpolation 
       element is identified for special handling.
     \item [{[handleAllElements]}]
       If set to {\tt .true.}, all local elements, regardless of their dynamic
       masking status, are made available to {\tt dynamicMaskRoutine} for
       handling. This option can be used to implement fully customized
       interpolations based on the information provided by ESMF.
       The default is {\tt .false.}, meaning that only elements affected by
       dynamic masking will be handed to {\tt dynamicMaskRoutine}.
     \item [{[dynamicSrcMaskValue]}]
       The value for which a source element is considered dynamically
       masked.
       The default is to {\em not} consider any source elements as
       dynamically masked.
     \item [{[dynamicDstMaskValue]}]
       The value for which a destination element is considered dynamically
       masked.
       The default is to {\em not} consider any destination elements as
       dynamically masked.
     \item[{[rc]}] 
       Return code; equals {\tt ESMF\_SUCCESS} if there are no errors.
     \end{description}
  
%...............................................................
\setlength{\parskip}{\oldparskip}
\setlength{\parindent}{\oldparindent}
\setlength{\baselineskip}{\oldbaselineskip}
