%                **** IMPORTANT NOTICE *****
% This LaTeX file has been automatically produced by ProTeX v. 1.1
% Any changes made to this file will likely be lost next time
% this file is regenerated from its source. Send questions 
% to Arlindo da Silva, dasilva@gsfc.nasa.gov
 
\setlength{\oldparskip}{\parskip}
\setlength{\parskip}{1.5ex}
\setlength{\oldparindent}{\parindent}
\setlength{\parindent}{0pt}
\setlength{\oldbaselineskip}{\baselineskip}
\setlength{\baselineskip}{11pt}
 
%--------------------- SHORT-HAND MACROS ----------------------
\def\bv{\begin{verbatim}}
\def\ev{\end{verbatim}}
\def\be{\begin{equation}}
\def\ee{\end{equation}}
\def\bea{\begin{eqnarray}}
\def\eea{\end{eqnarray}}
\def\bi{\begin{itemize}}
\def\ei{\end{itemize}}
\def\bn{\begin{enumerate}}
\def\en{\end{enumerate}}
\def\bd{\begin{description}}
\def\ed{\end{description}}
\def\({\left (}
\def\){\right )}
\def\[{\left [}
\def\]{\right ]}
\def\<{\left  \langle}
\def\>{\right \rangle}
\def\cI{{\cal I}}
\def\diag{\mathop{\rm diag}}
\def\tr{\mathop{\rm tr}}
%-------------------------------------------------------------

\markboth{Left}{Source File: ESMF\_ArrayFarrayHaloEx.F90,  Date: Tue May  5 20:59:44 MDT 2020
}

 
%/////////////////////////////////////////////////////////////

   \subsubsection{Array from native Fortran array with extra elements for halo or padding}
   \label{Array:fpadding}
  
   The example of the previous section showed how easy it is to create an Array
   object from existing PET-local Fortran arrays. The example did, however, not
   define any halo elements around the DE-local regions. The following code
   demonstrates how an Array object with space for a halo can be set up. 
%/////////////////////////////////////////////////////////////

 \begin{verbatim}
program ESMF_ArrayFarrayHaloEx
#include "ESMF.h"

  use ESMF
  use ESMF_TestMod
  
  implicit none
  
 
\end{verbatim}
 
%/////////////////////////////////////////////////////////////

   The allocatable array {\tt farrayA} will be used to provide the PET-local
   Fortran array for this example. 
%/////////////////////////////////////////////////////////////

 \begin{verbatim}
  ! local variables
  real(ESMF_KIND_R8), allocatable :: farrayA(:,:) ! allocatable Fortran array
  real(ESMF_KIND_R8), pointer :: farrayPtr(:,:)   ! matching Fortran array ptr
  type(ESMF_DistGrid)         :: distgrid         ! DistGrid object
  type(ESMF_Array)            :: array            ! Array object
  integer                     :: rc, i, j
  real(ESMF_KIND_R8)          :: localSum
  
 
\end{verbatim}
 
%/////////////////////////////////////////////////////////////

 \begin{verbatim}
  call ESMF_Initialize(defaultlogfilename="ArrayFarrayHaloEx.Log", &
                    logkindflag=ESMF_LOGKIND_MULTI, rc=rc)
  if (rc /= ESMF_SUCCESS) call ESMF_Finalize(endflag=ESMF_END_ABORT)
  
 
\end{verbatim}
 
%/////////////////////////////////////////////////////////////

   The Array is to cover the exact same index space as in the previous
   example. Furthermore decomposition and distribution are also kept the same.
   Hence the same DistGrid object will be created and it is expected to 
   execute this example with 4 PETs.
   
%/////////////////////////////////////////////////////////////

 \begin{verbatim}
  distgrid = ESMF_DistGridCreate(minIndex=(/1,1/), maxIndex=(/40,10/), rc=rc)
 
\end{verbatim}
 
%/////////////////////////////////////////////////////////////

   This DistGrid describes a 40 x 10 index space that will be decomposed into 
   4 DEs when executed on 4 PETs, associating 1 DE per PET. Each DE-local 
   exclusive region contains 10 x 10 elements. The DistGrid also stores and provides
   information about the relationship between DEs in index space, however,
   DistGrid does not contain information about halos. Arrays contain halo 
   information and it is possible to create multiple Arrays covering the same
   index space with identical decomposition and distribution using the same
   DistGrid object, while defining different, Array-specific halo regions.
  
   The extra memory required to cover the halo in the Array object must be 
   taken into account when allocating the PET-local {\tt farrayA} arrays. For
   a halo of 2 elements in each direction the following allocation will suffice. 
%/////////////////////////////////////////////////////////////

 \begin{verbatim}
  allocate(farrayA(14,14))    ! Fortran array with halo: 14 = 10 + 2 * 2
 
\end{verbatim}
 
%/////////////////////////////////////////////////////////////

   The {\tt farrayA} can now be used to create an Array object with enough space
   for a two element halo in each direction. The Array creation method checks for 
   each PET that the local Fortran array can accommodate the requested regions.
  
   The default behavior of ArrayCreate() is to center the exclusive region within
   the total region. Consequently the following call will provide the 2 extra 
   elements on each side of the exclusive 10 x 10 region without having to specify
   any additional arguments. 
%/////////////////////////////////////////////////////////////

 \begin{verbatim}
  array = ESMF_ArrayCreate(farray=farrayA, distgrid=distgrid, &
    indexflag=ESMF_INDEX_DELOCAL, rc=rc)
 
\end{verbatim}
 
%/////////////////////////////////////////////////////////////

   The exclusive Array region on each PET can be accessed through a suitable
   Fortran array pointer. See section \ref{Array_regions_and_default_bounds}
   for more details on Array regions. 
%/////////////////////////////////////////////////////////////

 \begin{verbatim}
  call ESMF_ArrayGet(array, farrayPtr=farrayPtr, rc=rc)
 
\end{verbatim}
 
%/////////////////////////////////////////////////////////////

   Following Array bounds convention, which by default puts the beginning of 
   the exclusive region at (1, 1, ...), the following loop will add up the 
   values of the local exclusive region for each DE, regardless of how the bounds
   were chosen for the original PET-local {\tt farrayA} arrays. 
%/////////////////////////////////////////////////////////////

 \begin{verbatim}
  localSum = 0.
  do j=1, 10
    do i=1, 10
      localSum = localSum + farrayPtr(i, j)
    enddo
  enddo
 
\end{verbatim}
 
%/////////////////////////////////////////////////////////////

   Elements with $i$ or $j$ in the [-1,0] or [11,12] ranges are located outside the
   exclusive region and may be used to define extra computational points or 
   halo operations.
  
   Cleanup and shut down ESMF. 
%/////////////////////////////////////////////////////////////

 \begin{verbatim}
  call ESMF_ArrayDestroy(array, rc=rc)
 
\end{verbatim}
 
%/////////////////////////////////////////////////////////////

 \begin{verbatim}
  deallocate(farrayA)
  call ESMF_DistGridDestroy(distgrid, rc=rc)
 
\end{verbatim}
 
%/////////////////////////////////////////////////////////////

 \begin{verbatim}
  call ESMF_Finalize(rc=rc)
 
\end{verbatim}
 
%/////////////////////////////////////////////////////////////

 \begin{verbatim}
end program
 
\end{verbatim}

%...............................................................
\setlength{\parskip}{\oldparskip}
\setlength{\parindent}{\oldparindent}
\setlength{\baselineskip}{\oldbaselineskip}
