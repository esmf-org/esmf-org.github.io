%                **** IMPORTANT NOTICE *****
% This LaTeX file has been automatically produced by ProTeX v. 1.1
% Any changes made to this file will likely be lost next time
% this file is regenerated from its source. Send questions 
% to Arlindo da Silva, dasilva@gsfc.nasa.gov
 
\setlength{\oldparskip}{\parskip}
\setlength{\parskip}{1.5ex}
\setlength{\oldparindent}{\parindent}
\setlength{\parindent}{0pt}
\setlength{\oldbaselineskip}{\baselineskip}
\setlength{\baselineskip}{11pt}
 
%--------------------- SHORT-HAND MACROS ----------------------
\def\bv{\begin{verbatim}}
\def\ev{\end{verbatim}}
\def\be{\begin{equation}}
\def\ee{\end{equation}}
\def\bea{\begin{eqnarray}}
\def\eea{\end{eqnarray}}
\def\bi{\begin{itemize}}
\def\ei{\end{itemize}}
\def\bn{\begin{enumerate}}
\def\en{\end{enumerate}}
\def\bd{\begin{description}}
\def\ed{\end{description}}
\def\({\left (}
\def\){\right )}
\def\[{\left [}
\def\]{\right ]}
\def\<{\left  \langle}
\def\>{\right \rangle}
\def\cI{{\cal I}}
\def\diag{\mathop{\rm diag}}
\def\tr{\mathop{\rm tr}}
%-------------------------------------------------------------

\markboth{Left}{Source File: ESMF\_ArrayGather.F90,  Date: Tue May  5 20:59:43 MDT 2020
}

 
%/////////////////////////////////////////////////////////////

   \subsubsection [ESMF\_ArrayGather] {ESMF\_ArrayGather - Gather a Fortran array from an ESMF\_Array }


   
\bigskip{\sf INTERFACE:}
\begin{verbatim}   subroutine ESMF_ArrayGather(array, farray, rootPet, tile, vm, rc) 
   \end{verbatim}{\em ARGUMENTS:}
\begin{verbatim}   type(ESMF_Array), intent(in) :: array 
   <type>(ESMF_KIND_<kind>), intent(out), target :: farray(<rank>) 
   integer, intent(in) :: rootPet 
 -- The following arguments require argument keyword syntax (e.g. rc=rc). --
   integer, intent(in), optional :: tile 
   type(ESMF_VM), intent(in), optional :: vm 
   integer, intent(out), optional :: rc 
   \end{verbatim}
{\sf STATUS:}
   \begin{itemize} 
   \item\apiStatusCompatibleVersion{5.2.0r} 
   \end{itemize} 
   
{\sf DESCRIPTION:\\ }

 
   Gather the data of an {ESMF\_Array} object into the {\tt farray} located on 
   {\tt rootPET}. A single DistGrid tile of {\tt array} must be 
   gathered into {\tt farray}. The optional {\tt tile} 
   argument allows selection of the tile. For Arrays defined on a single 
   tile DistGrid the default selection (tile 1) will be correct. The 
   shape of {\tt farray} must match the shape of the tile in Array. 
   
   If the Array contains replicating DistGrid dimensions data will be 
   gathered from the numerically higher DEs. Replicated data elements in 
   numerically lower DEs will be ignored. 
   
   This version of the interface implements the PET-based blocking paradigm: 
   Each PET of the VM must issue this call exactly once for {\em all} of its 
   DEs. The call will block until all PET-local data objects are accessible. 
   
   The arguments are: 
   \begin{description} 
   \item[array] 
   The {\tt ESMF\_Array} object from which data will be gathered. 
   \item[\{farray\}] 
   The Fortran array into which to gather data. Only root 
   must provide a valid {\tt farray}, the other PETs may treat 
   {\tt farray} as an optional argument. 
   \item[rootPet] 
   PET that holds the valid destination array, i.e. {\tt farray}. 
   \item[{[tile]}] 
   The DistGrid tile in {\tt array} from which to gather {\tt farray}. 
   By default {\tt farray} will be gathered from tile 1. 
   \item[{[vm]}] 
   Optional {\tt ESMF\_VM} object of the current context. Providing the 
   VM of the current context will lower the method's overhead. 
   \item[{[rc]}] 
   Return code; equals {\tt ESMF\_SUCCESS} if there are no errors. 
   \end{description} 
   
%...............................................................
\setlength{\parskip}{\oldparskip}
\setlength{\parindent}{\oldparindent}
\setlength{\baselineskip}{\oldbaselineskip}
