%                **** IMPORTANT NOTICE *****
% This LaTeX file has been automatically produced by ProTeX v. 1.1
% Any changes made to this file will likely be lost next time
% this file is regenerated from its source. Send questions 
% to Arlindo da Silva, dasilva@gsfc.nasa.gov
 
\setlength{\oldparskip}{\parskip}
\setlength{\parskip}{1.5ex}
\setlength{\oldparindent}{\parindent}
\setlength{\parindent}{0pt}
\setlength{\oldbaselineskip}{\baselineskip}
\setlength{\baselineskip}{11pt}
 
%--------------------- SHORT-HAND MACROS ----------------------
\def\bv{\begin{verbatim}}
\def\ev{\end{verbatim}}
\def\be{\begin{equation}}
\def\ee{\end{equation}}
\def\bea{\begin{eqnarray}}
\def\eea{\end{eqnarray}}
\def\bi{\begin{itemize}}
\def\ei{\end{itemize}}
\def\bn{\begin{enumerate}}
\def\en{\end{enumerate}}
\def\bd{\begin{description}}
\def\ed{\end{description}}
\def\({\left (}
\def\){\right )}
\def\[{\left [}
\def\]{\right ]}
\def\<{\left  \langle}
\def\>{\right \rangle}
\def\cI{{\cal I}}
\def\diag{\mathop{\rm diag}}
\def\tr{\mathop{\rm tr}}
%-------------------------------------------------------------

\markboth{Left}{Source File: ESMF\_ArrayGet.F90,  Date: Tue May  5 20:59:43 MDT 2020
}

 
%/////////////////////////////////////////////////////////////
\subsubsection [ESMF\_ArrayGet] {ESMF\_ArrayGet - Get object-wide Array information}


\bigskip{\sf INTERFACE:}
\begin{verbatim}   ! Private name; call using ESMF_ArrayGet()
   subroutine ESMF_ArrayGetDefault(array, arrayspec, typekind, &
     rank, localarrayList, indexflag, distgridToArrayMap, &
     distgridToPackedArrayMap, arrayToDistGridMap, undistLBound, &
     undistUBound, exclusiveLBound, exclusiveUBound, computationalLBound, &
     computationalUBound, totalLBound, totalUBound, computationalLWidth, &
     computationalUWidth, totalLWidth, totalUWidth, distgrid, dimCount, &
     tileCount, minIndexPTile, maxIndexPTile, deToTileMap, indexCountPDe, &
     delayout, deCount, localDeCount, ssiLocalDeCount, localDeToDeMap, &
     localDeList, & ! DEPRECATED ARGUMENT
     name, vm, rc)\end{verbatim}{\em ARGUMENTS:}
\begin{verbatim}     type(ESMF_Array), intent(in) :: array
 -- The following arguments require argument keyword syntax (e.g. rc=rc). --
     type(ESMF_ArraySpec), intent(out), optional :: arrayspec
     type(ESMF_TypeKind_Flag), intent(out), optional :: typekind
     integer, intent(out), optional :: rank
     type(ESMF_LocalArray), target, intent(out), optional :: localarrayList(:)
     type(ESMF_Index_Flag), intent(out), optional :: indexflag
     integer, target, intent(out), optional :: distgridToArrayMap(:)
     integer, target, intent(out), optional :: distgridToPackedArrayMap(:)
     integer, target, intent(out), optional :: arrayToDistGridMap(:)
     integer, target, intent(out), optional :: undistLBound(:)
     integer, target, intent(out), optional :: undistUBound(:)
     integer, target, intent(out), optional :: exclusiveLBound(:,:)
     integer, target, intent(out), optional :: exclusiveUBound(:,:)
     integer, target, intent(out), optional :: computationalLBound(:,:)
     integer, target, intent(out), optional :: computationalUBound(:,:)
     integer, target, intent(out), optional :: totalLBound(:,:)
     integer, target, intent(out), optional :: totalUBound(:,:)
     integer, target, intent(out), optional :: computationalLWidth(:,:)
     integer, target, intent(out), optional :: computationalUWidth(:,:)
     integer, target, intent(out), optional :: totalLWidth(:,:)
     integer, target, intent(out), optional :: totalUWidth(:,:)
     type(ESMF_DistGrid), intent(out), optional :: distgrid
     integer, intent(out), optional :: dimCount
     integer, intent(out), optional :: tileCount
     integer, intent(out), optional :: minIndexPTile(:,:)
     integer, intent(out), optional :: maxIndexPTile(:,:)
     integer, intent(out), optional :: deToTileMap(:)
     integer, intent(out), optional :: indexCountPDe(:,:)
     type(ESMF_DELayout), intent(out), optional :: delayout
     integer, intent(out), optional :: deCount
     integer, intent(out), optional :: localDeCount
     integer, intent(out), optional :: ssiLocalDeCount
     integer, intent(out), optional :: localDeToDeMap(:)
     integer, intent(out), optional :: localDeList(:) ! DEPRECATED ARGUMENT
     character(len=*), intent(out), optional :: name
     type(ESMF_VM), intent(out), optional :: vm
     integer, intent(out), optional :: rc\end{verbatim}
{\sf STATUS:}
   \begin{itemize}
   \item\apiStatusCompatibleVersion{5.2.0r}
   \item\apiStatusModifiedSinceVersion{5.2.0r}
   \begin{description}
   \item[5.2.0rp1] Added argument {\tt localDeToDeMap}.
   Started to deprecate argument {\tt localDeList}.
   The new argument name correctly uses the {\tt Map} suffix and
   better describes the returned information.
   This was pointed out by user request.
   \item[8.0.0] Added argument {\tt ssiLocalDeCount} to support DE sharing
   between PETs on the same single system image (SSI).\newline
   Added argument {\tt vm} in order to offer information about the
   VM on which the Array was created.
   \end{description}
   \end{itemize}
  
{\sf DESCRIPTION:\\ }


   Get internal information.
  
   This interface works for any number of DEs per PET.
  
   The arguments are:
   \begin{description}
   \item[array]
   Queried {\tt ESMF\_Array} object.
   \item[{[arrayspec]}]
   {\tt ESMF\_ArraySpec} object containing the type/kind/rank information
   of the Array object.
   \item[{[typekind]}]
   TypeKind of the Array object.
   \item[{[rank]}]
   Rank of the Array object.
   \item[{[localarrayList]}]
   Upon return this holds a list of the associated {\tt ESMC\_LocalArray}
   objects. {\tt localarrayList} must be allocated to be of size
   {\tt localDeCount} or {\tt ssiLocalDeCount}.
   \item[{[indexflag]}]
   Upon return this flag indicates how the DE-local indices are defined.
   See section \ref{const:indexflag} for a list of possible return values.
   \item[{[distgridToArrayMap]}]
   Upon return this list holds the Array dimensions against which the
   DistGrid dimensions are mapped. {\tt distgridToArrayMap} must be allocated
   to be of size {\tt dimCount}. An entry of zero indicates that the
   respective DistGrid dimension is replicating the Array across the DEs
   along this direction.
   \item[{[distgridToPackedArrayMap]}]
   Upon return this list holds the indices of the Array dimensions in packed
   format against which the DistGrid dimensions are mapped.
   {\tt distgridToPackedArrayMap} must be allocated to be of size
   {\tt dimCount}. An entry of zero indicates that the respective DistGrid
   dimension is replicating the Array across the DEs along this direction.
   \item[{[arrayToDistGridMap]}]
   Upon return this list holds the DistGrid dimensions against which the
   Array dimensions are mapped. {\tt arrayToDistGridMap} must be allocated
   to be of size {\tt rank}. An entry of zero indicates that the respective
   Array dimension is not decomposed, rendering it a tensor dimension.
   \item[{[undistLBound]}]
   \begin{sloppypar}
   Upon return this array holds the lower bounds of the undistributed
   dimensions of the Array. {\tt UndistLBound} must be allocated to be
   of size {\tt rank-dimCount}.
   \end{sloppypar}
   \item[{[undistUBound]}]
   \begin{sloppypar}
   Upon return this array holds the upper bounds of the undistributed
   dimensions of the Array. {\tt UndistUBound} must be allocated to be
   of size {\tt rank-dimCount}.
   \end{sloppypar}
   \item[{[exclusiveLBound]}]
   \begin{sloppypar}
   Upon return this holds the lower bounds of the exclusive regions for
   all PET-local DEs. {\tt exclusiveLBound} must be allocated to be
   of size {\tt (dimCount, localDeCount)} or
   {\tt (dimCount, ssiLocalDeCount)}.
   \end{sloppypar}
   \item[{[exclusiveUBound]}]
   \begin{sloppypar}
   Upon return this holds the upper bounds of the exclusive regions for
   all PET-local DEs. {\tt exclusiveUBound} must be allocated to be
   of size {\tt (dimCount, localDeCount)} or
   {\tt (dimCount, ssiLocalDeCount)}.
   \end{sloppypar}
   \item[{[computationalLBound]}]
   Upon return this holds the lower bounds of the computational regions for
   all PET-local DEs. {\tt computationalLBound} must be allocated to be
   of size {\tt (dimCount, localDeCount)} or
   {\tt (dimCount, ssiLocalDeCount)}.
   \item[{[computationalUBound]}]
   Upon return this holds the upper bounds of the computational regions for
   all PET-local DEs. {\tt computationalUBound} must be allocated to be
   of size {\tt (dimCount, localDeCount)} or
   {\tt (dimCount, ssiLocalDeCount)}.
   \item[{[totalLBound]}]
   Upon return this holds the lower bounds of the total regions for
   all PET-local DEs. {\tt totalLBound} must be allocated to be
   of size {\tt (dimCount, localDeCount)} or
   {\tt (dimCount, ssiLocalDeCount)}.
   \item[{[totalUBound]}]
   Upon return this holds the upper bounds of the total regions for
   all PET-local DEs. {\tt totalUBound} must be allocated to be
   of size {\tt (dimCount, localDeCount)} or
   {\tt (dimCount, ssiLocalDeCount)}.
   \item[{[computationalLWidth]}]
   Upon return this holds the lower width of the computational regions for
   all PET-local DEs. {\tt computationalLWidth} must be allocated to be
   of size {\tt (dimCount, localDeCount)} or
   {\tt (dimCount, ssiLocalDeCount)}.
   \item[{[computationalUWidth]}]
   Upon return this holds the upper width of the computational regions for
   all PET-local DEs. {\tt computationalUWidth} must be allocated to be
   of size {\tt (dimCount, localDeCount)} or
   {\tt (dimCount, ssiLocalDeCount)}.
   \item[{[totalLWidth]}]
   \begin{sloppypar}
   Upon return this holds the lower width of the total memory regions for
   all PET-local DEs. {\tt totalLWidth} must be allocated to be
   of size {\tt (dimCount, localDeCount)} or
   {\tt (dimCount, ssiLocalDeCount)}.
   \end{sloppypar}
   \item[{[totalUWidth]}]
   \begin{sloppypar}
   Upon return this holds the upper width of the total memory regions for
   all PET-local DEs. {\tt totalUWidth} must be allocated to be
   of size {\tt (dimCount, localDeCount)} or
   {\tt (dimCount, ssiLocalDeCount)}.
   \end{sloppypar}
   \item[{[distgrid]}]
   Upon return this holds the associated {\tt ESMF\_DistGrid} object.
   \item[{[dimCount]}]
   Number of dimensions (rank) of {\tt distgrid}.
   \item[{[tileCount]}]
   Number of tiles in {\tt distgrid}.
   \item[{[minIndexPTile]}]
   Lower index space corner per {\tt dim}, per {\tt tile}, with
   {\tt size(minIndexPTile) == (/dimCount, tileCount/)}.
   \item[{[maxIndexPTile]}]
   Upper index space corner per {\tt dim}, per {\tt tile}, with
   {\tt size(maxIndexPTile) == (/dimCount, tileCount/)}.
   \item[{[deToTileMap]}]
   List of tile id numbers, one for each DE, with
   {\tt size(deToTileMap) == (/deCount/)}
   \item[{[indexCountPDe]}]
   \begin{sloppypar}
   Array of extents per {\tt dim}, per {\tt de}, with
   {\tt size(indexCountPDe) == (/dimCount, deCount/)}.
   \end{sloppypar}
   \item[{[delayout]}]
   The associated {\tt ESMF\_DELayout} object.
   \item[{[deCount]}]
   The total number of DEs in the Array.
   \item[{[localDeCount]}]
   The number of DEs in the Array associated with the local PET.
   \item[{[ssiLocalDeCount]}]
   The number of DEs in the Array available to the local PET. This
   includes DEs that are local to other PETs on the same SSI, that are
   accessible via shared memory.
   \item[{[localDeToDeMap]}]
   Mapping between localDe indices and the (global) DEs associated with
   the local PET. The localDe index variables are discussed in sections
   \ref{DELayout_general_mapping} and \ref{Array_native_language_localde}.
   The provided actual argument must be of size {\tt localDeCount}, or
   {\tt ssiLocalDeCount}, and will be filled accordingly.
   \item[{[localDeList]}]
   \apiDeprecatedArgWithReplacement{localDeToDeMap}
   \item [{[name]}]
   Name of the Array object.
   \item [{[vm}]
   The VM on which the Array object was created.
   \item[{[rc]}]
   Return code; equals {\tt ESMF\_SUCCESS} if there are no errors.
   \end{description}
   
%/////////////////////////////////////////////////////////////
 
\mbox{}\hrulefill\ 
 
\subsubsection [ESMF\_ArrayGet] {ESMF\_ArrayGet - Get DE-local Array information for a specific dimension}


\bigskip{\sf INTERFACE:}
\begin{verbatim}   ! Private name; call using ESMF_ArrayGet()
   subroutine ESMF_ArrayGetPLocalDePDim(array, dim, localDe, &
     indexCount, indexList, rc)\end{verbatim}{\em ARGUMENTS:}
\begin{verbatim}     type(ESMF_Array), intent(in) :: array
     integer, intent(in) :: dim
 -- The following arguments require argument keyword syntax (e.g. rc=rc). --
     integer, intent(in), optional :: localDe
     integer, intent(out), optional :: indexCount
     integer, intent(out), optional :: indexList(:)
     integer, intent(out), optional :: rc\end{verbatim}
{\sf STATUS:}
   \begin{itemize}
   \item\apiStatusCompatibleVersion{5.2.0r}
   \end{itemize}
  
{\sf DESCRIPTION:\\ }


   Get internal information per local DE, per dim.
  
   This interface works for any number of DEs per PET.
  
   The arguments are:
   \begin{description}
   \item[array]
   Queried {\tt ESMF\_Array} object.
   \item[dim]
   Dimension for which information is requested. {\tt [1,..,dimCount]}
   \item[{[localDe]}]
   Local DE for which information is requested. {\tt [0,..,localDeCount-1]}.
   For {\tt localDeCount==1} the {\tt localDe} argument may be omitted,
   in which case it will default to {\tt localDe=0}.
   \item[{[indexCount]}]
   DistGrid indexCount associated with {\tt localDe, dim}.
   \item[{[indexList]}]
   List of DistGrid tile-local indices for {\tt localDe} along dimension
   {\tt dim}.
   \item[{[rc]}]
   Return code; equals {\tt ESMF\_SUCCESS} if there are no errors.
   \end{description}
   
%/////////////////////////////////////////////////////////////
 
\mbox{}\hrulefill\ 
 
\subsubsection [ESMF\_ArrayGet] {ESMF\_ArrayGet - Get a DE-local Fortran array pointer from an Array }


 
\bigskip{\sf INTERFACE:}
\begin{verbatim}   ! Private name; call using ESMF_ArrayGet() 
   subroutine ESMF_ArrayGetFPtr<rank><type><kind>(array, localDe, & 
   farrayPtr, rc) 
   \end{verbatim}{\em ARGUMENTS:}
\begin{verbatim}   type(ESMF_Array), intent(in) :: array 
 -- The following arguments require argument keyword syntax (e.g. rc=rc). --
   integer, intent(in), optional :: localDe 
   <type> (ESMF_KIND_<kind>), pointer :: farrayPtr(<rank>) 
   integer, intent(out), optional :: rc 
   \end{verbatim}
{\sf STATUS:}
   \begin{itemize} 
   \item\apiStatusCompatibleVersion{5.2.0r} 
   \end{itemize} 
   
{\sf DESCRIPTION:\\ }

 
   Access Fortran array pointer to the specified DE-local memory allocation of 
   the Array object. 
   
   The arguments are: 
   \begin{description} 
   \item[array] 
   Queried {\tt ESMF\_Array} object. 
   \item[{[localDe]}] 
   Local DE for which information is requested. {\tt [0,..,localDeCount-1]}. 
   For {\tt localDeCount==1} the {\tt localDe} argument may be omitted, 
   in which case it will default to {\tt localDe=0}. 
   \item[farrayPtr] 
   Upon return, {\tt farrayPtr} points to the DE-local data allocation of 
   {\tt localDe} in {\tt array}. It depends on the specific entry point 
   of {\tt ESMF\_ArrayCreate()} used during {\tt array} creation, which 
   Fortran operations are supported on the returned {\tt farrayPtr}. See 
   \ref{Array:rest} for more details. 
   \item[{[rc]}] 
   Return code; equals {\tt ESMF\_SUCCESS} if there are no errors. 
   \end{description} 
    
%/////////////////////////////////////////////////////////////
 
\mbox{}\hrulefill\ 
 
\subsubsection [ESMF\_ArrayGet] {ESMF\_ArrayGet - Get a DE-local LocalArray object from an Array}


\bigskip{\sf INTERFACE:}
\begin{verbatim}   ! Private name; call using ESMF_ArrayGet()
   subroutine ESMF_ArrayGetLocalArray(array, localDe, localarray, rc)\end{verbatim}{\em ARGUMENTS:}
\begin{verbatim}     type(ESMF_Array), intent(in) :: array
 -- The following arguments require argument keyword syntax (e.g. rc=rc). --
     integer, intent(in), optional :: localDe
     type(ESMF_LocalArray), intent(inout) :: localarray
     integer, intent(out), optional :: rc\end{verbatim}
{\sf STATUS:}
   \begin{itemize}
   \item\apiStatusCompatibleVersion{5.2.0r}
   \end{itemize}
  
{\sf DESCRIPTION:\\ }


   Provide access to {\tt ESMF\_LocalArray} object that holds data for
   the specified local DE.
  
   The arguments are:
   \begin{description}
   \item[array]
   Queried {\tt ESMF\_Array} object.
   \item[{[localDe]}]
   Local DE for which information is requested. {\tt [0,..,localDeCount-1]}.
   For {\tt localDeCount==1} the {\tt localDe} argument may be omitted,
   in which case it will default to {\tt localDe=0}.
   \item[localarray]
   Upon return {\tt localarray} refers to the DE-local data allocation of
   {\tt array}.
   \item[{[rc]}]
   Return code; equals {\tt ESMF\_SUCCESS} if there are no errors.
   \end{description}
  
%...............................................................
\setlength{\parskip}{\oldparskip}
\setlength{\parindent}{\oldparindent}
\setlength{\baselineskip}{\oldbaselineskip}
