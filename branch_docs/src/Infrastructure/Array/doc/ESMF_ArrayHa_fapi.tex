%                **** IMPORTANT NOTICE *****
% This LaTeX file has been automatically produced by ProTeX v. 1.1
% Any changes made to this file will likely be lost next time
% this file is regenerated from its source. Send questions 
% to Arlindo da Silva, dasilva@gsfc.nasa.gov
 
\setlength{\oldparskip}{\parskip}
\setlength{\parskip}{1.5ex}
\setlength{\oldparindent}{\parindent}
\setlength{\parindent}{0pt}
\setlength{\oldbaselineskip}{\baselineskip}
\setlength{\baselineskip}{11pt}
 
%--------------------- SHORT-HAND MACROS ----------------------
\def\bv{\begin{verbatim}}
\def\ev{\end{verbatim}}
\def\be{\begin{equation}}
\def\ee{\end{equation}}
\def\bea{\begin{eqnarray}}
\def\eea{\end{eqnarray}}
\def\bi{\begin{itemize}}
\def\ei{\end{itemize}}
\def\bn{\begin{enumerate}}
\def\en{\end{enumerate}}
\def\bd{\begin{description}}
\def\ed{\end{description}}
\def\({\left (}
\def\){\right )}
\def\[{\left [}
\def\]{\right ]}
\def\<{\left  \langle}
\def\>{\right \rangle}
\def\cI{{\cal I}}
\def\diag{\mathop{\rm diag}}
\def\tr{\mathop{\rm tr}}
%-------------------------------------------------------------

\markboth{Left}{Source File: ESMF\_ArrayHa.F90,  Date: Tue May  5 20:59:42 MDT 2020
}

 
%/////////////////////////////////////////////////////////////
\subsubsection [ESMF\_ArrayHalo] {ESMF\_ArrayHalo - Execute an Array halo operation}


  
\bigskip{\sf INTERFACE:}
\begin{verbatim}   subroutine ESMF_ArrayHalo(array, routehandle, &
     routesyncflag, finishedflag, cancelledflag, checkflag, rc)\end{verbatim}{\em ARGUMENTS:}
\begin{verbatim}     type(ESMF_Array),          intent(inout)         :: array
     type(ESMF_RouteHandle),    intent(inout)         :: routehandle
 -- The following arguments require argument keyword syntax (e.g. rc=rc). --
     type(ESMF_RouteSync_Flag), intent(in),  optional :: routesyncflag
     logical,                   intent(out), optional :: finishedflag
     logical,                   intent(out), optional :: cancelledflag
     logical,                   intent(in),  optional :: checkflag
     integer,                   intent(out), optional :: rc\end{verbatim}
{\sf STATUS:}
   \begin{itemize}
   \item\apiStatusCompatibleVersion{5.2.0r}
   \end{itemize}
  
{\sf DESCRIPTION:\\ }


     Execute a precomputed Array halo operation for {\tt array}.
     The {\tt array} argument must match the respective Array
     used during {\tt ESMF\_ArrayHaloStore()} in {\em type}, {\em kind}, and 
     memory layout of the {\em distributed} dimensions. However, the size,
     number, and index order of {\em undistributed} dimensions may be different.
     See section \ref{RH:Reusability} for a more detailed discussion of
     RouteHandle reusability.
  
     See {\tt ESMF\_ArrayHaloStore()} on how to precompute {\tt routehandle}.
  
     This call is {\em collective} across the current VM.
  
     \begin{description}
     \item [array]
       {\tt ESMF\_Array} containing data to be haloed.
     \item [routehandle]
       Handle to the precomputed Route.
     \item [{[routesyncflag]}]
       Indicate communication option. Default is {\tt ESMF\_ROUTESYNC\_BLOCKING},
       resulting in a blocking operation.
       See section \ref{const:routesync} for a complete list of valid settings.
     \item [{[finishedflag]}]
       \begin{sloppypar}
       Used in combination with {\tt routesyncflag = ESMF\_ROUTESYNC\_NBTESTFINISH}.
       Returned {\tt finishedflag} equal to {\tt .true.} indicates that all
       operations have finished. A value of {\tt .false.} indicates that there
       are still unfinished operations that require additional calls with
       {\tt routesyncflag = ESMF\_ROUTESYNC\_NBTESTFINISH}, or a final call with
       {\tt routesyncflag = ESMF\_ROUTESYNC\_NBWAITFINISH}. For all other {\tt routesyncflag}
       settings the returned value in {\tt finishedflag} is always {\tt .true.}.
       \end{sloppypar}
     \item [{[cancelledflag]}]
       A value of {\tt .true.} indicates that were cancelled communication
       operations. In this case the data in the {\tt dstArray} must be considered
       invalid. It may have been partially modified by the call. A value of
       {\tt .false.} indicates that none of the communication operations was
       cancelled. The data in {\tt dstArray} is valid if {\tt finishedflag} 
       returns equal {\tt .true.}.
     \item [{[checkflag]}]
       If set to {\tt .TRUE.} the input Array pair will be checked for
       consistency with the precomputed operation provided by {\tt routehandle}.
       If set to {\tt .FALSE.} {\em (default)} only a very basic input check
       will be performed, leaving many inconsistencies undetected. Set
       {\tt checkflag} to {\tt .FALSE.} to achieve highest performance.
     \item [{[rc]}]
       Return code; equals {\tt ESMF\_SUCCESS} if there are no errors.
     \end{description}
   
%/////////////////////////////////////////////////////////////
 
\mbox{}\hrulefill\ 
 
\subsubsection [ESMF\_ArrayHaloRelease] {ESMF\_ArrayHaloRelease - Release resources associated with Array halo operation}


  
\bigskip{\sf INTERFACE:}
\begin{verbatim}   subroutine ESMF_ArrayHaloRelease(routehandle, noGarbage, rc)\end{verbatim}{\em ARGUMENTS:}
\begin{verbatim}     type(ESMF_RouteHandle), intent(inout)         :: routehandle
 -- The following arguments require argument keyword syntax (e.g. rc=rc). --
     logical,                intent(in),  optional :: noGarbage
     integer,                intent(out), optional :: rc\end{verbatim}
{\sf STATUS:}
   \begin{itemize}
   \item\apiStatusCompatibleVersion{5.2.0r}
   \item\apiStatusModifiedSinceVersion{5.2.0r}
   \begin{description}
   \item[8.0.0] Added argument {\tt noGarbage}.
     The argument provides a mechanism to override the default garbage collection
     mechanism when destroying an ESMF object.
   \end{description}
   \end{itemize}
  
{\sf DESCRIPTION:\\ }


     Release resources associated with an Array halo operation. 
     After this call {\tt routehandle} becomes invalid.
  
     \begin{description}
     \item [routehandle]
       Handle to the precomputed Route.
     \item[{[noGarbage]}]
       If set to {\tt .TRUE.} the object will be fully destroyed and removed
       from the ESMF garbage collection system. Note however that under this 
       condition ESMF cannot protect against accessing the destroyed object 
       through dangling aliases -- a situation which may lead to hard to debug 
       application crashes.
   
       It is generally recommended to leave the {\tt noGarbage} argument
       set to {\tt .FALSE.} (the default), and to take advantage of the ESMF 
       garbage collection system which will prevent problems with dangling
       aliases or incorrect sequences of destroy calls. However this level of
       support requires that a small remnant of the object is kept in memory
       past the destroy call. This can lead to an unexpected increase in memory
       consumption over the course of execution in applications that use 
       temporary ESMF objects. For situations where the repeated creation and 
       destruction of temporary objects leads to memory issues, it is 
       recommended to call with {\tt noGarbage} set to {\tt .TRUE.}, fully 
       removing the entire temporary object from memory.
     \item [{[rc]}]
       Return code; equals {\tt ESMF\_SUCCESS} if there are no errors.
     \end{description}
   
%/////////////////////////////////////////////////////////////
 
\mbox{}\hrulefill\ 
 
\subsubsection [ESMF\_ArrayHaloStore] {ESMF\_ArrayHaloStore - Precompute an Array halo operation}


  
\bigskip{\sf INTERFACE:}
\begin{verbatim}     subroutine ESMF_ArrayHaloStore(array, routehandle, &
       startregion, haloLDepth, haloUDepth, pipelineDepth, rc)\end{verbatim}{\em ARGUMENTS:}
\begin{verbatim}     type(ESMF_Array),            intent(inout)           :: array
     type(ESMF_RouteHandle),      intent(inout)           :: routehandle
 -- The following arguments require argument keyword syntax (e.g. rc=rc). --
     type(ESMF_StartRegion_Flag), intent(in),    optional :: startregion
     integer,                     intent(in),    optional :: haloLDepth(:)
     integer,                     intent(in),    optional :: haloUDepth(:)
     integer,                     intent(inout), optional :: pipelineDepth
     integer,                     intent(out),   optional :: rc\end{verbatim}
{\sf STATUS:}
   \begin{itemize}
   \item\apiStatusCompatibleVersion{5.2.0r}
   \item\apiStatusModifiedSinceVersion{5.2.0r}
   \begin{description}
   \item[6.1.0] Added argument {\tt pipelineDepth}.
                The new argument provide access to the tuning parameter
                affecting the sparse matrix execution.
   \end{description}
   \end{itemize}
  
{\sf DESCRIPTION:\\ }


     Store an Array halo operation over the data in {\tt array}. By default,
     i.e. without specifying {\tt startregion}, {\tt haloLDepth} and
     {\tt haloUDepth}, all elements in the total Array region that lie outside
     the exclusive region will be considered potential destination elements for
     halo. However, only those elements that have a corresponding halo source
     element, i.e. an exclusive element on one of the DEs, will be updated under
     the halo operation. Elements that have no associated source remain 
     unchanged under halo.
  
     Specifying {\tt startregion} allows the shape of the effective halo region 
     to be changed from the inside. Setting this flag to
     {\tt ESMF\_STARTREGION\_COMPUTATIONAL} means that only elements outside 
     the computational region of the Array are considered for potential
     destination elements for the halo operation. The default is 
     {\tt ESMF\_STARTREGION\_EXCLUSIVE}.
  
     The {\tt haloLDepth} and {\tt haloUDepth} arguments allow to reduce
     the extent of the effective halo region. Starting at the region specified
     by {\tt startregion}, the {\tt haloLDepth} and {\tt haloUDepth}
     define a halo depth in each direction. Note that the maximum halo region is
     limited by the total Array region, independent of the actual
     {\tt haloLDepth} and {\tt haloUDepth} setting. The total Array region is
     local DE specific. The {\tt haloLDepth} and {\tt haloUDepth} are interpreted
     as the maximum desired extent, reducing the potentially larger region
     available for the halo operation.
  
     The routine returns an {\tt ESMF\_RouteHandle} that can be used to call 
     {\tt ESMF\_ArrayHalo()} on any Array that matches 
     {\tt array} in {\em type}, {\em kind}, and 
     memory layout of the {\em distributed} dimensions. However, the size,
     number, and index order of {\em undistributed} dimensions may be different.
     See section \ref{RH:Reusability} for a more detailed discussion of
     RouteHandle reusability.
  
     This call is {\em collective} across the current VM.  
  
     \begin{description}
     \item [array]
       {\tt ESMF\_Array} containing data to be haloed. The data in the halo
       region may be destroyed by this call.
     \item [routehandle]
       Handle to the precomputed Route.
     \item [{[startregion]}]
       \begin{sloppypar}
       The start of the effective halo region on every DE. The default
       setting is {\tt ESMF\_STARTREGION\_EXCLUSIVE}, rendering all non-exclusive
       elements potential halo destination elements.
       See section \ref{const:startregion} for a complete list of
       valid settings.
       \end{sloppypar}
     \item[{[haloLDepth]}] 
       This vector specifies the lower corner of the effective halo
       region with respect to the lower corner of {\tt startregion}.
       The size of {\tt haloLDepth} must equal the number of distributed Array
       dimensions.
     \item[{[haloUDepth]}] 
       This vector specifies the upper corner of the effective halo
       region with respect to the upper corner of {\tt startregion}.
       The size of {\tt haloUDepth} must equal the number of distributed Array
       dimensions.
     \item [{[pipelineDepth]}]
       The {\tt pipelineDepth} parameter controls how many messages a PET
       may have outstanding during a halo exchange. Larger values
       of {\tt pipelineDepth} typically lead to better performance. However,
       on some systems too large a value may lead to performance degradation,
       or runtime errors.
  
       Note that the pipeline depth has no effect on the bit-for-bit
       reproducibility of the results. However, it may affect the performance
       reproducibility of the exchange.
  
       The {\tt ESMF\_ArraySMMStore()} method implements an auto-tuning scheme
       for the {\tt pipelineDepth} parameter. The intent on the 
       {\tt pipelineDepth} argument is "{\tt inout}" in order to 
       support both overriding and accessing the auto-tuning parameter.
       If an argument $>= 0$ is specified, it is used for the 
       {\tt pipelineDepth} parameter, and the auto-tuning phase is skipped.
       In this case the {\tt pipelineDepth} argument is not modified on
       return. If the provided argument is $< 0$, the {\tt pipelineDepth}
       parameter is determined internally using the auto-tuning scheme. In this
       case the {\tt pipelineDepth} argument is re-set to the internally
       determined value on return. Auto-tuning is also used if the optional 
       {\tt pipelineDepth} argument is omitted.
  
     \item [{[rc]}]
       Return code; equals {\tt ESMF\_SUCCESS} if there are no errors.
     \end{description}
   
%/////////////////////////////////////////////////////////////
 
\mbox{}\hrulefill\ 
 
\subsubsection [ESMF\_ArrayIsCreated] {ESMF\_ArrayIsCreated - Check whether an Array object has been created}


 
\bigskip{\sf INTERFACE:}
\begin{verbatim}   function ESMF_ArrayIsCreated(array, rc)\end{verbatim}{\em RETURN VALUE:}
\begin{verbatim}     logical :: ESMF_ArrayIsCreated\end{verbatim}{\em ARGUMENTS:}
\begin{verbatim}     type(ESMF_Array), intent(in)            :: array
 -- The following arguments require argument keyword syntax (e.g. rc=rc). --
     integer,          intent(out), optional :: rc
 \end{verbatim}
{\sf DESCRIPTION:\\ }


     Return {\tt .true.} if the {\tt array} has been created. Otherwise return 
     {\tt .false.}. If an error occurs, i.e. {\tt rc /= ESMF\_SUCCESS} is 
     returned, the return value of the function will also be {\tt .false.}.
  
   The arguments are:
     \begin{description}
     \item[array]
       {\tt ESMF\_Array} queried.
     \item[{[rc]}]
       Return code; equals {\tt ESMF\_SUCCESS} if there are no errors.
     \end{description}
   
%/////////////////////////////////////////////////////////////
 
\mbox{}\hrulefill\ 
 
\subsubsection [ESMF\_ArrayPrint] {ESMF\_ArrayPrint - Print Array information}


 
\bigskip{\sf INTERFACE:}
\begin{verbatim}   subroutine ESMF_ArrayPrint(array, rc)\end{verbatim}{\em ARGUMENTS:}
\begin{verbatim}     type(ESMF_Array), intent(in)            :: array
 -- The following arguments require argument keyword syntax (e.g. rc=rc). --
     integer,          intent(out), optional :: rc  \end{verbatim}
{\sf STATUS:}
   \begin{itemize}
   \item\apiStatusCompatibleVersion{5.2.0r}
   \end{itemize}
  
{\sf DESCRIPTION:\\ }


     Print internal information of the specified {\tt ESMF\_Array} object. \\
  
     The arguments are:
     \begin{description}
     \item[array] 
       {\tt ESMF\_Array} object.
     \item[{[rc]}] 
       Return code; equals {\tt ESMF\_SUCCESS} if there are no errors.
     \end{description}
   
%/////////////////////////////////////////////////////////////
 
\mbox{}\hrulefill\ 
 
\subsubsection [ESMF\_ArrayRead] {ESMF\_ArrayRead - Read Array data from a file}


   \label{api:ArrayRead}
  
\bigskip{\sf INTERFACE:}
\begin{verbatim}   subroutine ESMF_ArrayRead(array, fileName, variableName, &
     timeslice, iofmt, rc)
     ! We need to terminate the strings on the way to C++\end{verbatim}{\em ARGUMENTS:}
\begin{verbatim}     type(ESMF_Array),      intent(inout)         :: array
     character(*),          intent(in)            :: fileName
 -- The following arguments require argument keyword syntax (e.g. rc=rc). --
     character(*),          intent(in),  optional :: variableName
     integer,               intent(in),  optional :: timeslice
     type(ESMF_IOFmt_Flag), intent(in),  optional :: iofmt
     integer,               intent(out), optional :: rc\end{verbatim}
{\sf DESCRIPTION:\\ }


     Read Array data from file and put it into an {\tt ESMF\_Array} object.
     For this API to be functional, the environment variable {\tt ESMF\_PIO}
     should be set to "internal" when the ESMF library is built.
     Please see the section on Data I/O,~\ref{io:dataio}.
   
     Limitations:
     \begin{itemize}
       \item Only single tile Arrays are supported.
       \item Not supported in {\tt ESMF\_COMM=mpiuni} mode.
     \end{itemize}
  
    The arguments are:
    \begin{description}
     \item[array]
      The {\tt ESMF\_Array} object in which the read data is returned.
     \item[fileName]
      The name of the file from which Array data is read.
     \item[{[variableName]}]
      Variable name in the file; default is the "name" of Array.
      Use this argument only in the I/O format (such as NetCDF) that
      supports variable name. If the I/O format does not support this
      (such as binary format), ESMF will return an error code.
     \item[{[timeslice]}]
      The time-slice number of the variable read from file.
     \item[{[iofmt]}]
      \begin{sloppypar}
      The I/O format.  Please see Section~\ref{opt:iofmtflag} for the list
      of options. If not present, file names with a {\tt .bin} extension will
      use {\tt ESMF\_IOFMT\_BIN}, and file names with a {\tt .nc} extension
      will use {\tt ESMF\_IOFMT\_NETCDF}.  Other files default to
      {\tt ESMF\_IOFMT\_NETCDF}.
      \end{sloppypar}
     \item[{[rc]}]
      Return code; equals {\tt ESMF\_SUCCESS} if there are no errors.
    \end{description}
   
%/////////////////////////////////////////////////////////////
 
\mbox{}\hrulefill\ 
 
\subsubsection [ESMF\_ArrayRedist] {ESMF\_ArrayRedist - Execute an Array redistribution}


  
\bigskip{\sf INTERFACE:}
\begin{verbatim}   subroutine ESMF_ArrayRedist(srcArray, dstArray, routehandle, &
     routesyncflag, finishedflag, cancelledflag, zeroregion, checkflag, rc)\end{verbatim}{\em ARGUMENTS:}
\begin{verbatim}     type(ESMF_Array),          intent(in),    optional :: srcArray
     type(ESMF_Array),          intent(inout), optional :: dstArray
     type(ESMF_RouteHandle),    intent(inout)           :: routehandle
 -- The following arguments require argument keyword syntax (e.g. rc=rc). --
     type(ESMF_RouteSync_Flag), intent(in),    optional :: routesyncflag
     logical,                   intent(out),   optional :: finishedflag
     logical,                   intent(out),   optional :: cancelledflag
     type(ESMF_Region_Flag),    intent(in),    optional :: zeroregion
     logical,                   intent(in),    optional :: checkflag
     integer,                   intent(out),   optional :: rc\end{verbatim}
{\sf STATUS:}
   \begin{itemize}
   \item\apiStatusCompatibleVersion{5.2.0r}
   \item\apiStatusModifiedSinceVersion{5.2.0r}
   \begin{description}
   \item[7.1.0r] Added argument {\tt zeroregion} to allow user to control
                how the destination array is zero'ed out. This is especially
                useful in cases where the source and destination arrays do not
                cover the identical index space.
   \end{description}
   \end{itemize}
  
{\sf DESCRIPTION:\\ }


     \begin{sloppypar}
     Execute a precomputed Array redistribution from {\tt srcArray}
     to {\tt dstArray}.
     Both {\tt srcArray} and {\tt dstArray} must match the respective Arrays
     used during {\tt ESMF\_ArrayRedisttore()} in {\em type}, {\em kind}, and 
     memory layout of the {\em distributed} dimensions. However, the size,
     number, and index order of {\em undistributed} dimensions may be different.
     See section \ref{RH:Reusability} for a more detailed discussion of
     RouteHandle reusability.
     \end{sloppypar}
  
     The {\tt srcArray} and {\tt dstArray} arguments are optional in support of
     the situation where {\tt srcArray} and/or {\tt dstArray} are not defined on
     all PETs. The {\tt srcArray} and {\tt dstArray} must be specified on those
     PETs that hold source or destination DEs, respectively, but may be omitted
     on all other PETs. PETs that hold neither source nor destination DEs may
     omit both arguments.
  
     It is erroneous to specify the identical Array object for {\tt srcArray} and
     {\tt dstArray} arguments.
  
     See {\tt ESMF\_ArrayRedistStore()} on how to precompute 
     {\tt routehandle}.
  
     This call is {\em collective} across the current VM.
  
     \begin{description}
     \item [{[srcArray]}]
       {\tt ESMF\_Array} with source data.
     \item [{[dstArray]}]
       {\tt ESMF\_Array} with destination data.
     \item [routehandle]
       Handle to the precomputed Route.
     \item [{[routesyncflag]}]
       Indicate communication option. Default is {\tt ESMF\_ROUTESYNC\_BLOCKING},
       resulting in a blocking operation.
       See section \ref{const:routesync} for a complete list of valid settings.
     \item [{[finishedflag]}]
       \begin{sloppypar}
       Used in combination with {\tt routesyncflag = ESMF\_ROUTESYNC\_NBTESTFINISH}.
       Returned {\tt finishedflag} equal to {\tt .true.} indicates that all
       operations have finished. A value of {\tt .false.} indicates that there
       are still unfinished operations that require additional calls with
       {\tt routesyncflag = ESMF\_ROUTESYNC\_NBTESTFINISH}, or a final call with
       {\tt routesyncflag = ESMF\_ROUTESYNC\_NBWAITFINISH}. For all other {\tt routesyncflag}
       settings the returned value in {\tt finishedflag} is always {\tt .true.}.
       \end{sloppypar}
     \item [{[cancelledflag]}]
       A value of {\tt .true.} indicates that were cancelled communication
       operations. In this case the data in the {\tt dstArray} must be considered
       invalid. It may have been partially modified by the call. A value of
       {\tt .false.} indicates that none of the communication operations was
       cancelled. The data in {\tt dstArray} is valid if {\tt finishedflag} 
       returns equal {\tt .true.}.
     \item [{[zeroregion]}]
       \begin{sloppypar}
       If set to {\tt ESMF\_REGION\_TOTAL} the total regions of
       all DEs in {\tt dstArray} will be initialized to zero before updating the 
       elements with the results of the sparse matrix multiplication. If set to
       {\tt ESMF\_REGION\_EMPTY} the elements in {\tt dstArray} will not be
       modified prior to the sparse matrix multiplication and results will be
       added to the incoming element values. Setting {\tt zeroregion} to 
       {\tt ESMF\_REGION\_SELECT} will only zero out those elements in the 
       destination Array that will be updated by the sparse matrix
       multiplication. See section \ref{const:region} for a complete list of
       valid settings. The default is {\tt ESMF\_REGION\_SELECT}.
       \end{sloppypar}
     \item [{[checkflag]}]
       If set to {\tt .TRUE.} the input Array pair will be checked for
       consistency with the precomputed operation provided by {\tt routehandle}.
       If set to {\tt .FALSE.} {\em (default)} only a very basic input check
       will be performed, leaving many inconsistencies undetected. Set
       {\tt checkflag} to {\tt .FALSE.} to achieve highest performance.
     \item [{[rc]}]
       Return code; equals {\tt ESMF\_SUCCESS} if there are no errors.
     \end{description}
   
%/////////////////////////////////////////////////////////////
 
\mbox{}\hrulefill\ 
 
\subsubsection [ESMF\_ArrayRedistRelease] {ESMF\_ArrayRedistRelease - Release resources associated with Array redistribution}


  
\bigskip{\sf INTERFACE:}
\begin{verbatim}   subroutine ESMF_ArrayRedistRelease(routehandle, noGarbage, rc)
   \end{verbatim}{\em ARGUMENTS:}
\begin{verbatim}     type(ESMF_RouteHandle), intent(inout)         :: routehandle
 -- The following arguments require argument keyword syntax (e.g. rc=rc). --
     logical,                intent(in),  optional :: noGarbage
     integer,                intent(out), optional :: rc\end{verbatim}
{\sf STATUS:}
   \begin{itemize}
   \item\apiStatusCompatibleVersion{5.2.0r}
   \item\apiStatusModifiedSinceVersion{5.2.0r}
   \begin{description}
   \item[8.0.0] Added argument {\tt noGarbage}.
     The argument provides a mechanism to override the default garbage collection
     mechanism when destroying an ESMF object.
   \end{description}
   \end{itemize}
  
{\sf DESCRIPTION:\\ }


     Release resources associated with an Array redistribution. After this call
     {\tt routehandle} becomes invalid.
  
     \begin{description}
     \item [routehandle]
       Handle to the precomputed Route.
     \item[{[noGarbage]}]
       If set to {\tt .TRUE.} the object will be fully destroyed and removed
       from the ESMF garbage collection system. Note however that under this 
       condition ESMF cannot protect against accessing the destroyed object 
       through dangling aliases -- a situation which may lead to hard to debug 
       application crashes.
   
       It is generally recommended to leave the {\tt noGarbage} argument
       set to {\tt .FALSE.} (the default), and to take advantage of the ESMF 
       garbage collection system which will prevent problems with dangling
       aliases or incorrect sequences of destroy calls. However this level of
       support requires that a small remnant of the object is kept in memory
       past the destroy call. This can lead to an unexpected increase in memory
       consumption over the course of execution in applications that use 
       temporary ESMF objects. For situations where the repeated creation and 
       destruction of temporary objects leads to memory issues, it is 
       recommended to call with {\tt noGarbage} set to {\tt .TRUE.}, fully 
       removing the entire temporary object from memory.
     \item [{[rc]}]
       Return code; equals {\tt ESMF\_SUCCESS} if there are no errors.
     \end{description}
   
%/////////////////////////////////////////////////////////////
 
\mbox{}\hrulefill\ 
 
\subsubsection [ESMF\_ArrayRedistStore] {ESMF\_ArrayRedistStore - Precompute Array redistribution with local factor argument}


  
\bigskip{\sf INTERFACE:}
\begin{verbatim}   ! Private name; call using ESMF_ArrayRedistStore()
   subroutine ESMF_ArrayRedistStore<type><kind>(srcArray, dstArray, &
     routehandle, factor, srcToDstTransposeMap, &
     ignoreUnmatchedIndices, pipelineDepth, rc)\end{verbatim}{\em ARGUMENTS:}
\begin{verbatim}     type(ESMF_Array),       intent(in)              :: srcArray
     type(ESMF_Array),       intent(inout)           :: dstArray
     type(ESMF_RouteHandle), intent(inout)           :: routehandle
     <type>(ESMF_KIND_<kind>),intent(in)             :: factor
 -- The following arguments require argument keyword syntax (e.g. rc=rc). --
     integer,                intent(in),    optional :: srcToDstTransposeMap(:)
     logical,                intent(in),    optional :: ignoreUnmatchedIndices
     integer,                intent(inout), optional :: pipelineDepth
     integer,                intent(out),   optional :: rc\end{verbatim}
{\sf STATUS:}
   \begin{itemize}
   \item\apiStatusCompatibleVersion{5.2.0r}
   \item\apiStatusModifiedSinceVersion{5.2.0r}
   \begin{description}
   \item[6.1.0] Added argument {\tt pipelineDepth}.
                The new argument provide access to the tuning parameter
                affecting the sparse matrix execution.
   \item[7.0.0] Added argument {\tt transposeRoutehandle} to allow a handle to
                the transposed redist operation to be returned.\newline
                Added argument {\tt ignoreUnmatchedIndices} to support situations 
                where not all elements between source and destination Arrays 
                match.
   \item[7.1.0r] Removed argument {\tt transposeRoutehandle} and provide it
                via interface overloading instead. This allows argument 
                {\tt srcArray} to stay strictly intent(in) for this entry point.
   \end{description}
   \end{itemize}
  
{\sf DESCRIPTION:\\ }


   \label{ArrayRedistStoreTK}
   {\tt ESMF\_ArrayRedistStore()} is a collective method across all PETs of the
   current Component. The interface of the method is overloaded, allowing 
   -- in principle -- each PET to call into {\tt ESMF\_ArrayRedistStore()}
   through a different entry point. Restrictions apply as to which combinations
   are sensible. All other combinations result in ESMF run time errors. The
   complete semantics of the {\tt ESMF\_ArrayRedistStore()} method, as provided
   through the separate entry points shown in \ref{ArrayRedistStoreTK} and
   \ref{ArrayRedistStoreNF}, is described in the following paragraphs as a whole.
  
   Store an Array redistribution operation from {\tt srcArray} to {\tt dstArray}.
   Interface \ref{ArrayRedistStoreTK} allows PETs to specify a {\tt factor}
   argument. PETs not specifying a {\tt factor} argument call into interface
   \ref{ArrayRedistStoreNF}. If multiple PETs specify the {\tt factor} argument,
   its type and kind, as well as its value must match across all PETs. If none
   of the PETs specify a {\tt factor} argument the default will be a factor of
   1. The resulting factor is applied to all of the source data during
   redistribution, allowing scaling of the data, e.g. for unit transformation.
    
   Both {\tt srcArray} and {\tt dstArray} are interpreted as sequentialized 
   vectors. The sequence is defined by the order of DistGrid dimensions and the
   order of tiles within the DistGrid or by user-supplied arbitrary sequence
   indices. See section \ref{Array:SparseMatMul} for details on the definition
   of {\em sequence indices}.
  
   Source Array, destination Array, and the factor may be of different
   <type><kind>. Further, source and destination Arrays may differ in shape,
   however, the number of elements must match. 
    
   If {\tt srcToDstTransposeMap} is not specified the redistribution corresponds
   to an identity mapping of the sequentialized source Array to the
   sequentialized destination Array. If the {\tt srcToDstTransposeMap}
   argument is provided it must be identical on all PETs. The
   {\tt srcToDstTransposeMap} allows source and destination Array dimensions to
   be transposed during the redistribution. The number of source and destination
   Array dimensions must be equal under this condition and the size of mapped
   dimensions must match.
    
   It is erroneous to specify the identical Array object for {\tt srcArray} and
   {\tt dstArray} arguments. 
  
     The routine returns an {\tt ESMF\_RouteHandle} that can be used to call 
     {\tt ESMF\_ArrayRedist()} on any pair of Arrays that matches 
     {\tt srcArray} and {\tt dstArray} in {\em type}, {\em kind}, and 
     memory layout of the {\em distributed} dimensions. However, the size,
     number, and index order of {\em undistributed} dimensions may be different.
     See section \ref{RH:Reusability} for a more detailed discussion of
     RouteHandle reusability.
  
   This method is overloaded for:\newline
   {\tt ESMF\_TYPEKIND\_I4}, {\tt ESMF\_TYPEKIND\_I8},\newline 
   {\tt ESMF\_TYPEKIND\_R4}, {\tt ESMF\_TYPEKIND\_R8}.
   \newline
    
   This call is {\em collective} across the current VM.  
  
     \begin{description}
  
     \item [srcArray]
       {\tt ESMF\_Array} with source data.
  
     \item [dstArray]
       {\tt ESMF\_Array} with destination data. The data in this Array may be
       destroyed by this call.
  
     \item [routehandle]
       Handle to the precomputed Route.
  
     \item [factor]
       Factor by which to multiply source data.
  
     \item [{[srcToDstTransposeMap]}]
       List with as many entries as there are dimensions in {\tt srcArray}. Each
       entry maps the corresponding {\tt srcArray} dimension against the 
       specified {\tt dstArray} dimension. Mixing of distributed and
       undistributed dimensions is supported.
  
     \item [{[ignoreUnmatchedIndices]}]
       A logical flag that affects the behavior for when not all elements match
       between the {\tt srcArray} and {\tt dstArray} side. The default setting
       is {\tt .false.}, indicating that it is an error when such a situation is 
       encountered. Setting {\tt ignoreUnmatchedIndices} to {\tt .true.} ignores
       unmatched indices.
  
     \item [{[pipelineDepth]}]
       The {\tt pipelineDepth} parameter controls how many messages a PET
       may have outstanding during a redist exchange. Larger values
       of {\tt pipelineDepth} typically lead to better performance. However,
       on some systems too large a value may lead to performance degradation,
       or runtime errors.
  
       Note that the pipeline depth has no effect on the bit-for-bit
       reproducibility of the results. However, it may affect the performance
       reproducibility of the exchange.
  
       The {\tt ESMF\_ArraySMMStore()} method implements an auto-tuning scheme
       for the {\tt pipelineDepth} parameter. The intent on the 
       {\tt pipelineDepth} argument is "{\tt inout}" in order to 
       support both overriding and accessing the auto-tuning parameter.
       If an argument $>= 0$ is specified, it is used for the 
       {\tt pipelineDepth} parameter, and the auto-tuning phase is skipped.
       In this case the {\tt pipelineDepth} argument is not modified on
       return. If the provided argument is $< 0$, the {\tt pipelineDepth}
       parameter is determined internally using the auto-tuning scheme. In this
       case the {\tt pipelineDepth} argument is re-set to the internally
       determined value on return. Auto-tuning is also used if the optional 
       {\tt pipelineDepth} argument is omitted.
  
     \item [{[rc]}]
       Return code; equals {\tt ESMF\_SUCCESS} if there are no errors.
     \end{description}
   
%/////////////////////////////////////////////////////////////
 
\mbox{}\hrulefill\ 
 
\subsubsection [ESMF\_ArrayRedistStore] {ESMF\_ArrayRedistStore - Precompute Array redistribution and transpose with local factor argument}


  
\bigskip{\sf INTERFACE:}
\begin{verbatim}   ! Private name; call using ESMF_ArrayRedistStore()
   subroutine ESMF_ArrayRedistStore<type><kind>TP(srcArray, dstArray, &
     routehandle, transposeRoutehandle, factor, &
     srcToDstTransposeMap, ignoreUnmatchedIndices, pipelineDepth, rc)\end{verbatim}{\em ARGUMENTS:}
\begin{verbatim}     type(ESMF_Array),       intent(inout)           :: srcArray
     type(ESMF_Array),       intent(inout)           :: dstArray
     type(ESMF_RouteHandle), intent(inout)           :: routehandle
     type(ESMF_RouteHandle), intent(inout)           :: transposeRoutehandle
     <type>(ESMF_KIND_<kind>),intent(in)             :: factor
 -- The following arguments require argument keyword syntax (e.g. rc=rc). --
     integer,                intent(in),    optional :: srcToDstTransposeMap(:)
     logical,                intent(in),    optional :: ignoreUnmatchedIndices
     integer,                intent(inout), optional :: pipelineDepth
     integer,                intent(out),   optional :: rc\end{verbatim}
{\sf DESCRIPTION:\\ }


   \label{ArrayRedistStoreTK}
   {\tt ESMF\_ArrayRedistStore()} is a collective method across all PETs of the
   current Component. The interface of the method is overloaded, allowing 
   -- in principle -- each PET to call into {\tt ESMF\_ArrayRedistStore()}
   through a different entry point. Restrictions apply as to which combinations
   are sensible. All other combinations result in ESMF run time errors. The
   complete semantics of the {\tt ESMF\_ArrayRedistStore()} method, as provided
   through the separate entry points shown in \ref{ArrayRedistStoreTK} and
   \ref{ArrayRedistStoreNF}, is described in the following paragraphs as a whole.
  
   Store an Array redistribution operation from {\tt srcArray} to {\tt dstArray}.
   Interface \ref{ArrayRedistStoreTK} allows PETs to specify a {\tt factor}
   argument. PETs not specifying a {\tt factor} argument call into interface
   \ref{ArrayRedistStoreNF}. If multiple PETs specify the {\tt factor} argument,
   its type and kind, as well as its value must match across all PETs. If none
   of the PETs specify a {\tt factor} argument the default will be a factor of
   1. The resulting factor is applied to all of the source data during
   redistribution, allowing scaling of the data, e.g. for unit transformation.
    
   Both {\tt srcArray} and {\tt dstArray} are interpreted as sequentialized 
   vectors. The sequence is defined by the order of DistGrid dimensions and the
   order of tiles within the DistGrid or by user-supplied arbitrary sequence
   indices. See section \ref{Array:SparseMatMul} for details on the definition
   of {\em sequence indices}.
  
   Source Array, destination Array, and the factor may be of different
   <type><kind>. Further, source and destination Arrays may differ in shape,
   however, the number of elements must match. 
    
   If {\tt srcToDstTransposeMap} is not specified the redistribution corresponds
   to an identity mapping of the sequentialized source Array to the
   sequentialized destination Array. If the {\tt srcToDstTransposeMap}
   argument is provided it must be identical on all PETs. The
   {\tt srcToDstTransposeMap} allows source and destination Array dimensions to
   be transposed during the redistribution. The number of source and destination
   Array dimensions must be equal under this condition and the size of mapped
   dimensions must match.
    
   It is erroneous to specify the identical Array object for {\tt srcArray} and
   {\tt dstArray} arguments. 
  
     The routine returns an {\tt ESMF\_RouteHandle} that can be used to call 
     {\tt ESMF\_ArrayRedist()} on any pair of Arrays that matches 
     {\tt srcArray} and {\tt dstArray} in {\em type}, {\em kind}, and 
     memory layout of the {\em distributed} dimensions. However, the size,
     number, and index order of {\em undistributed} dimensions may be different.
     See section \ref{RH:Reusability} for a more detailed discussion of
     RouteHandle reusability.
  
   This method is overloaded for:\newline
   {\tt ESMF\_TYPEKIND\_I4}, {\tt ESMF\_TYPEKIND\_I8},\newline 
   {\tt ESMF\_TYPEKIND\_R4}, {\tt ESMF\_TYPEKIND\_R8}.
   \newline
    
   This call is {\em collective} across the current VM.  
  
     \begin{description}
  
     \item [srcArray]
       {\tt ESMF\_Array} with source data. The data in this Array may be
       destroyed by this call.
  
     \item [dstArray]
       {\tt ESMF\_Array} with destination data. The data in this Array may be
       destroyed by this call.
  
     \item [routehandle]
       Handle to the precomputed Route.
  
     \item [transposeRoutehandle]
       Handle to the transposed matrix operation. The transposed operation goes
       from {\tt dstArray} to {\tt srcArray}.
  
     \item [factor]
       Factor by which to multiply source data.
  
     \item [{[srcToDstTransposeMap]}]
       List with as many entries as there are dimensions in {\tt srcArray}. Each
       entry maps the corresponding {\tt srcArray} dimension against the 
       specified {\tt dstArray} dimension. Mixing of distributed and
       undistributed dimensions is supported.
  
     \item [{[ignoreUnmatchedIndices]}]
       A logical flag that affects the behavior for when not all elements match
       between the {\tt srcArray} and {\tt dstArray} side. The default setting
       is {\tt .false.}, indicating that it is an error when such a situation is 
       encountered. Setting {\tt ignoreUnmatchedIndices} to {\tt .true.} ignores
       unmatched indices.
  
     \item [{[pipelineDepth]}]
       The {\tt pipelineDepth} parameter controls how many messages a PET
       may have outstanding during a redist exchange. Larger values
       of {\tt pipelineDepth} typically lead to better performance. However,
       on some systems too large a value may lead to performance degradation,
       or runtime errors.
  
       Note that the pipeline depth has no effect on the bit-for-bit
       reproducibility of the results. However, it may affect the performance
       reproducibility of the exchange.
  
       The {\tt ESMF\_ArraySMMStore()} method implements an auto-tuning scheme
       for the {\tt pipelineDepth} parameter. The intent on the 
       {\tt pipelineDepth} argument is "{\tt inout}" in order to 
       support both overriding and accessing the auto-tuning parameter.
       If an argument $>= 0$ is specified, it is used for the 
       {\tt pipelineDepth} parameter, and the auto-tuning phase is skipped.
       In this case the {\tt pipelineDepth} argument is not modified on
       return. If the provided argument is $< 0$, the {\tt pipelineDepth}
       parameter is determined internally using the auto-tuning scheme. In this
       case the {\tt pipelineDepth} argument is re-set to the internally
       determined value on return. Auto-tuning is also used if the optional 
       {\tt pipelineDepth} argument is omitted.
  
     \item [{[rc]}]
       Return code; equals {\tt ESMF\_SUCCESS} if there are no errors.
     \end{description}
   
%/////////////////////////////////////////////////////////////
 
\mbox{}\hrulefill\ 
 
\subsubsection [ESMF\_ArrayRedistStore] {ESMF\_ArrayRedistStore - Precompute Array redistribution without local factor argument}


  
\bigskip{\sf INTERFACE:}
\begin{verbatim}   ! Private name; call using ESMF_ArrayRedistStore()
   subroutine ESMF_ArrayRedistStoreNF(srcArray, dstArray, routehandle, &
     srcToDstTransposeMap, ignoreUnmatchedIndices, &
     pipelineDepth, rc)\end{verbatim}{\em ARGUMENTS:}
\begin{verbatim}     type(ESMF_Array),       intent(in)              :: srcArray
     type(ESMF_Array),       intent(inout)           :: dstArray
     type(ESMF_RouteHandle), intent(inout)           :: routehandle
 -- The following arguments require argument keyword syntax (e.g. rc=rc). --
     integer,                intent(in),    optional :: srcToDstTransposeMap(:)
     logical,                intent(in),    optional :: ignoreUnmatchedIndices
     integer,                intent(inout), optional :: pipelineDepth
     integer,                intent(out),   optional :: rc\end{verbatim}
{\sf STATUS:}
   \begin{itemize}
   \item\apiStatusCompatibleVersion{5.2.0r}
   \item\apiStatusModifiedSinceVersion{5.2.0r}
   \begin{description}
   \item[6.1.0] Added argument {\tt pipelineDepth}.
                The new argument provide access to the tuning parameter
                affecting the sparse matrix execution.
   \item[7.0.0] Added argument {\tt transposeRoutehandle} to allow a handle to
                the transposed redist operation to be returned.\newline
                Added argument {\tt ignoreUnmatchedIndices} to support situations 
                where not all elements between source and destination Arrays 
                match.
   \item[7.1.0r] Removed argument {\tt transposeRoutehandle} and provide it
                via interface overloading instead. This allows argument 
                {\tt srcArray} to stay strictly intent(in) for this entry point.
   \end{description}
   \end{itemize}
  
{\sf DESCRIPTION:\\ }


   \label{ArrayRedistStoreNF}
   {\tt ESMF\_ArrayRedistStore()} is a collective method across all PETs of the
   current Component. The interface of the method is overloaded, allowing 
   -- in principle -- each PET to call into {\tt ESMF\_ArrayRedistStore()}
   through a different entry point. Restrictions apply as to which combinations
   are sensible. All other combinations result in ESMF run time errors. The
   complete semantics of the {\tt ESMF\_ArrayRedistStore()} method, as provided
   through the separate entry points shown in \ref{ArrayRedistStoreTK} and
   \ref{ArrayRedistStoreNF}, is described in the following paragraphs as a whole.
  
   Store an Array redistribution operation from {\tt srcArray} to {\tt dstArray}.
   Interface \ref{ArrayRedistStoreTK} allows PETs to specify a {\tt factor}
   argument. PETs not specifying a {\tt factor} argument call into interface
   \ref{ArrayRedistStoreNF}. If multiple PETs specify the {\tt factor} argument,
   its type and kind, as well as its value must match across all PETs. If none
   of the PETs specify a {\tt factor} argument the default will be a factor of
   1. The resulting factor is applied to all of the source data during
   redistribution, allowing scaling of the data, e.g. for unit transformation.
    
   Both {\tt srcArray} and {\tt dstArray} are interpreted as sequentialized 
   vectors. The sequence is defined by the order of DistGrid dimensions and the
   order of tiles within the DistGrid or by user-supplied arbitrary sequence
   indices. See section \ref{Array:SparseMatMul} for details on the definition
   of {\em sequence indices}.
  
   Source Array, destination Array, and the factor may be of different
   <type><kind>. Further, source and destination Arrays may differ in shape,
   however, the number of elements must match. 
    
   If {\tt srcToDstTransposeMap} is not specified the redistribution corresponds
   to an identity mapping of the sequentialized source Array to the
   sequentialized destination Array. If the {\tt srcToDstTransposeMap}
   argument is provided it must be identical on all PETs. The
   {\tt srcToDstTransposeMap} allows source and destination Array dimensions to
   be transposed during the redistribution. The number of source and destination
   Array dimensions must be equal under this condition and the size of mapped
   dimensions must match.
    
   It is erroneous to specify the identical Array object for {\tt srcArray} and
   {\tt dstArray} arguments. 
  
     The routine returns an {\tt ESMF\_RouteHandle} that can be used to call 
     {\tt ESMF\_ArrayRedist()} on any pair of Arrays that matches 
     {\tt srcArray} and {\tt dstArray} in {\em type}, {\em kind}, and 
     memory layout of the {\em distributed} dimensions. However, the size,
     number, and index order of {\em undistributed} dimensions may be different.
     See section \ref{RH:Reusability} for a more detailed discussion of
     RouteHandle reusability.
  
   This call is {\em collective} across the current VM.  
  
     \begin{description}
  
     \item [srcArray]
       {\tt ESMF\_Array} with source data.
  
     \item [dstArray]
       {\tt ESMF\_Array} with destination data. The data in this Array may be
       destroyed by this call.
  
     \item [routehandle]
       Handle to the precomputed Route.
  
     \item [{[srcToDstTransposeMap]}]
       List with as many entries as there are dimensions in {\tt srcArray}. Each
       entry maps the corresponding {\tt srcArray} dimension against the 
       specified {\tt dstArray} dimension. Mixing of distributed and
       undistributed dimensions is supported.
  
     \item [{[ignoreUnmatchedIndices]}]
       A logical flag that affects the behavior for when not all elements match
       between the {\tt srcArray} and {\tt dstArray} side. The default setting
       is {\tt .false.}, indicating that it is an error when such a situation is 
       encountered. Setting {\tt ignoreUnmatchedIndices} to {\tt .true.} ignores
       unmatched indices.
  
     \item [{[pipelineDepth]}]
       The {\tt pipelineDepth} parameter controls how many messages a PET
       may have outstanding during a redist exchange. Larger values
       of {\tt pipelineDepth} typically lead to better performance. However,
       on some systems too large a value may lead to performance degradation,
       or runtime errors.
  
       Note that the pipeline depth has no effect on the bit-for-bit
       reproducibility of the results. However, it may affect the performance
       reproducibility of the exchange.
  
       The {\tt ESMF\_ArraySMMStore()} method implements an auto-tuning scheme
       for the {\tt pipelineDepth} parameter. The intent on the 
       {\tt pipelineDepth} argument is "{\tt inout}" in order to 
       support both overriding and accessing the auto-tuning parameter.
       If an argument $>= 0$ is specified, it is used for the 
       {\tt pipelineDepth} parameter, and the auto-tuning phase is skipped.
       In this case the {\tt pipelineDepth} argument is not modified on
       return. If the provided argument is $< 0$, the {\tt pipelineDepth}
       parameter is determined internally using the auto-tuning scheme. In this
       case the {\tt pipelineDepth} argument is re-set to the internally
       determined value on return. Auto-tuning is also used if the optional 
       {\tt pipelineDepth} argument is omitted.
  
     \item [{[rc]}]
       Return code; equals {\tt ESMF\_SUCCESS} if there are no errors.
     \end{description}
   
%/////////////////////////////////////////////////////////////
 
\mbox{}\hrulefill\ 
 
\subsubsection [ESMF\_ArrayRedistStore] {ESMF\_ArrayRedistStore - Precompute Array redistribution and transpose without local factor argument}


  
\bigskip{\sf INTERFACE:}
\begin{verbatim}   ! Private name; call using ESMF_ArrayRedistStore()
   subroutine ESMF_ArrayRedistStoreNFTP(srcArray, dstArray, routehandle, &
     transposeRoutehandle, srcToDstTransposeMap, &
     ignoreUnmatchedIndices, pipelineDepth, rc)\end{verbatim}{\em ARGUMENTS:}
\begin{verbatim}     type(ESMF_Array),       intent(inout)           :: srcArray
     type(ESMF_Array),       intent(inout)           :: dstArray
     type(ESMF_RouteHandle), intent(inout)           :: routehandle
     type(ESMF_RouteHandle), intent(inout)           :: transposeRoutehandle
 -- The following arguments require argument keyword syntax (e.g. rc=rc). --
     integer,                intent(in),    optional :: srcToDstTransposeMap(:)
     logical,                intent(in),    optional :: ignoreUnmatchedIndices
     integer,                intent(inout), optional :: pipelineDepth
     integer,                intent(out),   optional :: rc\end{verbatim}
{\sf DESCRIPTION:\\ }


   \label{ArrayRedistStoreNF}
   {\tt ESMF\_ArrayRedistStore()} is a collective method across all PETs of the
   current Component. The interface of the method is overloaded, allowing 
   -- in principle -- each PET to call into {\tt ESMF\_ArrayRedistStore()}
   through a different entry point. Restrictions apply as to which combinations
   are sensible. All other combinations result in ESMF run time errors. The
   complete semantics of the {\tt ESMF\_ArrayRedistStore()} method, as provided
   through the separate entry points shown in \ref{ArrayRedistStoreTK} and
   \ref{ArrayRedistStoreNF}, is described in the following paragraphs as a whole.
  
   Store an Array redistribution operation from {\tt srcArray} to {\tt dstArray}.
   Interface \ref{ArrayRedistStoreTK} allows PETs to specify a {\tt factor}
   argument. PETs not specifying a {\tt factor} argument call into interface
   \ref{ArrayRedistStoreNF}. If multiple PETs specify the {\tt factor} argument,
   its type and kind, as well as its value must match across all PETs. If none
   of the PETs specify a {\tt factor} argument the default will be a factor of
   1. The resulting factor is applied to all of the source data during
   redistribution, allowing scaling of the data, e.g. for unit transformation.
    
   Both {\tt srcArray} and {\tt dstArray} are interpreted as sequentialized 
   vectors. The sequence is defined by the order of DistGrid dimensions and the
   order of tiles within the DistGrid or by user-supplied arbitrary sequence
   indices. See section \ref{Array:SparseMatMul} for details on the definition
   of {\em sequence indices}.
  
   Source Array, destination Array, and the factor may be of different
   <type><kind>. Further, source and destination Arrays may differ in shape,
   however, the number of elements must match. 
    
   If {\tt srcToDstTransposeMap} is not specified the redistribution corresponds
   to an identity mapping of the sequentialized source Array to the
   sequentialized destination Array. If the {\tt srcToDstTransposeMap}
   argument is provided it must be identical on all PETs. The
   {\tt srcToDstTransposeMap} allows source and destination Array dimensions to
   be transposed during the redistribution. The number of source and destination
   Array dimensions must be equal under this condition and the size of mapped
   dimensions must match.
    
   It is erroneous to specify the identical Array object for {\tt srcArray} and
   {\tt dstArray} arguments. 
  
     The routine returns an {\tt ESMF\_RouteHandle} that can be used to call 
     {\tt ESMF\_ArrayRedist()} on any pair of Arrays that matches 
     {\tt srcArray} and {\tt dstArray} in {\em type}, {\em kind}, and 
     memory layout of the {\em distributed} dimensions. However, the size,
     number, and index order of {\em undistributed} dimensions may be different.
     See section \ref{RH:Reusability} for a more detailed discussion of
     RouteHandle reusability.
  
   This call is {\em collective} across the current VM.  
  
     \begin{description}
  
     \item [srcArray]
       {\tt ESMF\_Array} with source data. The data in this Array may be
       destroyed by this call.
  
     \item [dstArray]
       {\tt ESMF\_Array} with destination data. The data in this Array may be
       destroyed by this call.
  
     \item [routehandle]
       Handle to the precomputed Route.
  
     \item [transposeRoutehandle]
       Handle to the transposed matrix operation. The transposed operation goes
       from {\tt dstArray} to {\tt srcArray}.
  
     \item [{[srcToDstTransposeMap]}]
       List with as many entries as there are dimensions in {\tt srcArray}. Each
       entry maps the corresponding {\tt srcArray} dimension against the 
       specified {\tt dstArray} dimension. Mixing of distributed and
       undistributed dimensions is supported.
  
     \item [{[ignoreUnmatchedIndices]}]
       A logical flag that affects the behavior for when not all elements match
       between the {\tt srcArray} and {\tt dstArray} side. The default setting
       is {\tt .false.}, indicating that it is an error when such a situation is 
       encountered. Setting {\tt ignoreUnmatchedIndices} to {\tt .true.} ignores
       unmatched indices.
  
     \item [{[pipelineDepth]}]
       The {\tt pipelineDepth} parameter controls how many messages a PET
       may have outstanding during a redist exchange. Larger values
       of {\tt pipelineDepth} typically lead to better performance. However,
       on some systems too large a value may lead to performance degradation,
       or runtime errors.
  
       Note that the pipeline depth has no effect on the bit-for-bit
       reproducibility of the results. However, it may affect the performance
       reproducibility of the exchange.
  
       The {\tt ESMF\_ArraySMMStore()} method implements an auto-tuning scheme
       for the {\tt pipelineDepth} parameter. The intent on the 
       {\tt pipelineDepth} argument is "{\tt inout}" in order to 
       support both overriding and accessing the auto-tuning parameter.
       If an argument $>= 0$ is specified, it is used for the 
       {\tt pipelineDepth} parameter, and the auto-tuning phase is skipped.
       In this case the {\tt pipelineDepth} argument is not modified on
       return. If the provided argument is $< 0$, the {\tt pipelineDepth}
       parameter is determined internally using the auto-tuning scheme. In this
       case the {\tt pipelineDepth} argument is re-set to the internally
       determined value on return. Auto-tuning is also used if the optional 
       {\tt pipelineDepth} argument is omitted.
  
     \item [{[rc]}]
       Return code; equals {\tt ESMF\_SUCCESS} if there are no errors.
     \end{description}
  
%...............................................................
\setlength{\parskip}{\oldparskip}
\setlength{\parindent}{\oldparindent}
\setlength{\baselineskip}{\oldbaselineskip}
