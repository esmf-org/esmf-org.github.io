%                **** IMPORTANT NOTICE *****
% This LaTeX file has been automatically produced by ProTeX v. 1.1
% Any changes made to this file will likely be lost next time
% this file is regenerated from its source. Send questions 
% to Arlindo da Silva, dasilva@gsfc.nasa.gov
 
\setlength{\oldparskip}{\parskip}
\setlength{\parskip}{1.5ex}
\setlength{\oldparindent}{\parindent}
\setlength{\parindent}{0pt}
\setlength{\oldbaselineskip}{\baselineskip}
\setlength{\baselineskip}{11pt}
 
%--------------------- SHORT-HAND MACROS ----------------------
\def\bv{\begin{verbatim}}
\def\ev{\end{verbatim}}
\def\be{\begin{equation}}
\def\ee{\end{equation}}
\def\bea{\begin{eqnarray}}
\def\eea{\end{eqnarray}}
\def\bi{\begin{itemize}}
\def\ei{\end{itemize}}
\def\bn{\begin{enumerate}}
\def\en{\end{enumerate}}
\def\bd{\begin{description}}
\def\ed{\end{description}}
\def\({\left (}
\def\){\right )}
\def\[{\left [}
\def\]{\right ]}
\def\<{\left  \langle}
\def\>{\right \rangle}
\def\cI{{\cal I}}
\def\diag{\mathop{\rm diag}}
\def\tr{\mathop{\rm tr}}
%-------------------------------------------------------------

\markboth{Left}{Source File: ESMF\_ArrayCommNBEx.F90,  Date: Tue May  5 20:59:43 MDT 2020
}

 
%/////////////////////////////////////////////////////////////

  
   \subsubsection{Non-blocking Communications}
   \label{Array:CommNB}
   
   All {\tt ESMF\_RouteHandle} based communication methods, like 
   {\tt ESMF\_ArrayRedist()}, {\tt ESMF\_ArrayHalo()} and {\tt ESMF\_ArraySMM()}, 
   can be executed in blocking or non-blocking mode. The non-blocking feature is
   useful, for example, to overlap computation with communication, or to
   implement a more loosely synchronized inter-Component interaction scheme than
   is possible with the blocking communication mode.
  
   Access to the non-blocking execution mode is provided uniformly across all
   RouteHandle based communication calls. Every such call contains the optional
   {\tt routesyncflag} argument of type {\tt ESMF\_RouteSync\_Flag}. Section
   \ref{const:routesync} lists all of the valid settings for this flag.
  
   It is an execution time decision to select whether to invoke a precomputed
   communication pattern, stored in a RouteHandle, in the blocking or
   non-blocking mode. Neither requires specifically precomputed RouteHandles
   - i.e. a RouteHandle is neither specifically blocking nor specifically
   non-blocking. 
%/////////////////////////////////////////////////////////////

 \begin{verbatim}
  call ESMF_ArrayRedistStore(srcArray=srcArray, dstArray=dstArray, &
    routehandle=routehandle, rc=rc)
 
\end{verbatim}
 
%/////////////////////////////////////////////////////////////

   The returned RouteHandle {\tt routehandle} can be used in blocking or 
   non-blocking execution calls. The application is free to switch between
   both modes for the same RouteHandle.
  
   By default {\tt routesyncflag} is set to {\tt ESMF\_ROUTESYNC\_BLOCKING} in all of the
   RouteHandle execution methods, and the behavior is that of the VM-wide
   collective communication calls described in the previous sections. In the
   blocking mode the user must assume that the communication call will not
   return until all PETs have exchanged the precomputed information. On the
   other hand, the user has no guarantee about the exact synchronization 
   behavior, and it is unsafe to make specific assumptions. What is guaranteed
   in the blocking communication mode is that when the call returns on the
   local PET, all data exchanges associated with all local DEs have finished.
   This means that all in-bound data elements are valid and that all out-bound
   data elements can safely be overwritten by the user. 
%/////////////////////////////////////////////////////////////

 \begin{verbatim}
  call ESMF_ArrayRedist(srcArray=srcArray, dstArray=dstArray, &
    routehandle=routehandle, routesyncflag=ESMF_ROUTESYNC_BLOCKING, rc=rc)
 
\end{verbatim}
 
%/////////////////////////////////////////////////////////////

   The same exchange pattern, that is encoded in {\tt routehandle}, can be 
   executed in non-blocking mode, simply by setting the appropriate
   {\tt routesyncflag} when calling into {\tt ESMF\_ArrayRedist()}.
  
   At first sight there are obvious similarities between the non-blocking
   RouteHandle based execution paradigm and the non-blocking message passing
   calls provided by MPI. However, there are significant differences in
   the behavior of the non-blocking point-to-point calls that MPI defines and
   the non-blocking mode of the collective exchange patterns described by ESMF
   RouteHandles.
  
   Setting {\tt routesyncflag} to {\tt ESMF\_ROUTESYNC\_NBSTART} in any RouteHandle
   execution call returns immediately after all out-bound data has been moved
   into ESMF internal transfer buffers and the exchange has been initiated. 
%/////////////////////////////////////////////////////////////

 \begin{verbatim}
  call ESMF_ArrayRedist(srcArray=srcArray, dstArray=dstArray, &
    routehandle=routehandle, routesyncflag=ESMF_ROUTESYNC_NBSTART, rc=rc)
 
\end{verbatim}
 
%/////////////////////////////////////////////////////////////

   Once a call with {\tt routesyncflag = ESMF\_ROUTESYNC\_NBSTART} returns, it is safe
   to modify the out-bound data elements in the {\tt srcArray} object. However,
   no guarantees are made for the in-bound data elements in {\tt dstArray} at
   this phase of the non-blocking execution. It is unsafe to access these
   elements until the exchange has finished locally.
  
   \begin{sloppypar}
   One way to ensure that the exchange has finished locally is to call 
   with {\tt routesyncflag} set to {\tt ESMF\_ROUTESYNC\_NBWAITFINISH}.
   \end{sloppypar} 
%/////////////////////////////////////////////////////////////

 \begin{verbatim}
  call ESMF_ArrayRedist(srcArray=srcArray, dstArray=dstArray, &
    routehandle=routehandle, routesyncflag=ESMF_ROUTESYNC_NBWAITFINISH, rc=rc)
 
\end{verbatim}
 
%/////////////////////////////////////////////////////////////

   Calling with {\tt routesyncflag = ESMF\_ROUTESYNC\_NBWAITFINISH} instructs the
   communication method to wait and block until the previously started
   exchange has finished, and has been processed locally according to 
   the RouteHandle. Once the call returns, it is safe to access both in-bound
   and out-bound data elements in {\tt dstArray} and {\tt srcArray}, 
   respectively.
  
   \begin{sloppypar}
   Some situations require more flexibility than is provided by the 
   {\tt ESMF\_ROUTESYNC\_NBSTART} - {\tt ESMF\_ROUTESYNC\_NBWAITFINISH} pair. For
   instance, a Component that needs to interact with several other Components,
   virtually simultaneously, would initiated several different exchanges with 
   {\tt ESMF\_ROUTESYNC\_NBSTART}. Calling with {\tt ESMF\_ROUTESYNC\_NBWAITFINISH} for
   any of the outstanding exchanges may potentially block for a long time, 
   lowering the throughput. In the worst case a dead lock situation may arise.
   Calling with {\tt routesyncflag = ESMF\_ROUTESYNC\_NBTESTFINISH} addresses this problem.
   \end{sloppypar} 
%/////////////////////////////////////////////////////////////

 \begin{verbatim}
  call ESMF_ArrayRedist(srcArray=srcArray, dstArray=dstArray, &
    routehandle=routehandle, routesyncflag=ESMF_ROUTESYNC_NBTESTFINISH, &
    finishedflag=finishflag, rc=rc)
 
\end{verbatim}
 
%/////////////////////////////////////////////////////////////

   This call tests the locally outstanding data transfer operation in 
   {\tt routehandle}, and finishes the exchange as much as currently possible.
   It does not block until the entire exchange has finished locally, instead
   it returns immediately after one round of testing has been
   completed. The optional return argument {\tt finishedflag} is set to 
   {\tt .true.} if the exchange is completely finished locally, and set to 
   {\tt .false.} otherwise.
  
   The user code must decide, depending on the value of the returned
   {\tt finishedflag}, whether additional calls are required to finish an
   outstanding non-blocking exchange. If so, it can be done by 
   calling {\tt ESMF\_ArrayRedist()} repeatedly with 
   {\tt ESMF\_ROUTESYNC\_NBTESTFINISH} until 
   {\tt finishedflag} comes back with a value of {\tt .true.}. Such a loop
   allows other pieces of user code to be executed between the calls. 
   A call with {\tt ESMF\_ROUTESYNC\_NBWAITFINISH} can alternatively be used to
   block until the exchange has locally finished.
  
   {\em Noteworthy property.}
   It is allowable to invoke a RouteHandle based communication call
   with {\tt routesyncflag} set to 
   {\tt ESMF\_ROUTESYNC\_NBTESTFINISH} or
   {\tt ESMF\_ROUTESYNC\_NBWAITFINISH} on a specific RouteHandle without there 
   being an outstanding non-blocking exchange. As a matter of fact, it is not
   required that there was ever a call made with {\tt ESMF\_ROUTESYNC\_NBSTART} for
   the RouteHandle. In these cases the calls made with
   {\tt ESMF\_ROUTESYNC\_NBTESTFINISH} or {\tt ESMF\_ROUTESYNC\_NBWAITFINISH}  will
   simply return immediately (with {\tt finishedflag} set to {\tt .true.}).
  
   {\em Noteworthy property.}
   It is fine to mix blocking and non-blocking invocations of the same 
   RouteHandle based communication call across the PETs. This means that it is
   fine for some PETs to issue the call with {\tt ESMF\_ROUTESYNC\_BLOCKING}
   (or using the default), while other PETs call the same communication call
   with {\tt ESMF\_ROUTESYNC\_NBSTART}.
  
   {\em Noteworthy restriction.}
   A RouteHandle that is currently involved in an outstanding non-blocking
   exchange may {\em not} be used to start any further exchanges, neither
   blocking nor non-blocking. This restriction is independent of whether the
   newly started RouteHandle based exchange is made for the same or for 
   different data objects.
   
%...............................................................
\setlength{\parskip}{\oldparskip}
\setlength{\parindent}{\oldparindent}
\setlength{\baselineskip}{\oldbaselineskip}
