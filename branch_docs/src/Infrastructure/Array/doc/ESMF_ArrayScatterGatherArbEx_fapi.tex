%                **** IMPORTANT NOTICE *****
% This LaTeX file has been automatically produced by ProTeX v. 1.1
% Any changes made to this file will likely be lost next time
% this file is regenerated from its source. Send questions 
% to Arlindo da Silva, dasilva@gsfc.nasa.gov
 
\setlength{\oldparskip}{\parskip}
\setlength{\parskip}{1.5ex}
\setlength{\oldparindent}{\parindent}
\setlength{\parindent}{0pt}
\setlength{\oldbaselineskip}{\baselineskip}
\setlength{\baselineskip}{11pt}
 
%--------------------- SHORT-HAND MACROS ----------------------
\def\bv{\begin{verbatim}}
\def\ev{\end{verbatim}}
\def\be{\begin{equation}}
\def\ee{\end{equation}}
\def\bea{\begin{eqnarray}}
\def\eea{\end{eqnarray}}
\def\bi{\begin{itemize}}
\def\ei{\end{itemize}}
\def\bn{\begin{enumerate}}
\def\en{\end{enumerate}}
\def\bd{\begin{description}}
\def\ed{\end{description}}
\def\({\left (}
\def\){\right )}
\def\[{\left [}
\def\]{\right ]}
\def\<{\left  \langle}
\def\>{\right \rangle}
\def\cI{{\cal I}}
\def\diag{\mathop{\rm diag}}
\def\tr{\mathop{\rm tr}}
%-------------------------------------------------------------

\markboth{Left}{Source File: ESMF\_ArrayScatterGatherArbEx.F90,  Date: Tue May  5 20:59:43 MDT 2020
}

 
%/////////////////////////////////////////////////////////////

  
   \subsubsection{Communication -- Scatter and Gather, revisited}
   \label{Array:ScatterGatherRevisited}
   
   The {\tt ESMF\_ArrayScatter()} and {\tt ESMF\_ArrayGather()} calls, 
   introduced in section \ref{Array:ScatterGather}, provide a convenient
   way of communicating data between a Fortran array and all of the DEs of
   a single Array tile. A key requirement of {\tt ESMF\_ArrayScatter()}
   and {\tt ESMF\_ArrayGather()} is that the {\em shape} of the Fortran array
   and the Array tile must match. This means that the {\tt dimCount} must be
   equal, and that the size of each dimension must match. Element reordering
   during scatter and gather is only supported on a per dimension level,
   based on the {\tt decompflag} option available during DistGrid creation.
  
   While the {\tt ESMF\_ArrayScatter()} and {\tt ESMF\_ArrayGather()} methods
   cover a broad, and important spectrum of cases, there are situations that
   require a different set of rules to scatter and gather data between a
   Fortran array and an ESMF Array object. For instance, it is often convenient
   to create an Array on a DistGrid that was created with arbitrary,
   user-supplied sequence indices. See section \ref{DistGrid:ArbitrarySeqInd}
   for more background on DistGrids with arbitrary sequence indices. 
%/////////////////////////////////////////////////////////////

 \begin{verbatim}
  allocate(arbSeqIndexList(10))   ! each PET will have 10 elements
  
  do i=1, 10
    arbSeqIndexList(i) = (i-1)*petCount + localPet+1 ! initialize unique 
                                                     ! seq. indices
  enddo
  
  distgrid = ESMF_DistGridCreate(arbSeqIndexList=arbSeqIndexList, rc=rc)
 
\end{verbatim}
 
%/////////////////////////////////////////////////////////////

 \begin{verbatim}
  deallocate(arbSeqIndexList)
  
  call ESMF_ArraySpecSet(arrayspec, typekind=ESMF_TYPEKIND_I4, rank=1, rc=rc)
 
\end{verbatim}
 
%/////////////////////////////////////////////////////////////

 \begin{verbatim}
  array = ESMF_ArrayCreate(arrayspec=arrayspec, distgrid=distgrid, rc=rc)
 
\end{verbatim}
 
%/////////////////////////////////////////////////////////////

   This {\tt array} object holds 10 elements on each DE, and there is one DE
   per PET, for a total element count of 10 x {\tt petCount}. The
   {\tt arbSeqIndexList}, used during DistGrid creation, was constructed cyclic
   across all DEs. DE 0, for example, on a 4 PET run, would hold sequence
   indices 1, 5, 9, ... . DE 1 would hold 2, 6, 10, ..., and so on.
  
   The usefulness of the user-specified arbitrary sequence indices becomes
   clear when they are interpreted as global element ids. The ArrayRedist()
   and ArraySMM() communication methods are based on sequence index mapping
   between source and destination Arrays. Other than providing a canonical
   sequence index order via the default sequence scheme, outlined in
   \ref{Array:SparseMatMul}, ESMF does not place any restrictions on the
   sequence indices. Objects that were not created with user supplied
   sequence indices default to the ESMF sequence index order.
  
   A common, and useful interpretation of the arbitrary sequence indices, 
   specified during DistGrid creation, is that of relating them to the 
   canonical ESMF sequence index order of another data object. Within this
   interpretation the {\tt array} object created above could be viewed as an
   arbitrary distribution of a ({\tt petCount} x 10) 2D array. 
   
%/////////////////////////////////////////////////////////////

 \begin{verbatim}
  if (localPet == 0) then
    allocate(farray(petCount,10)) ! allocate 2D Fortran array petCount x 10
    do j=1, 10
      do i=1, petCount
        farray(i,j) = 100 + (j-1)*petCount + i    ! initialize to something
      enddo
    enddo
  else
    allocate(farray(0,0)) ! must allocate an array of size 0 on all other PETs
  endif
 
\end{verbatim}
 
%/////////////////////////////////////////////////////////////

   For a 4 PET run, {\tt farray} on PET 0 now holds the following data.
   \begin{verbatim}
     -----1----2----3------------10-----> j
     |
     1   101, 105, 109, ....  , 137
     |
     2   102, 106, 110, ....  , 138
     |
     3   103, 107, 111, ....  , 139
     |
     4   104, 108, 112, ....  , 140
     |
     |
     v
    i
   \end{verbatim}
  
   On all other PETs {\tt farray} has a zero size allocation.
  
   Following the sequence index interpretation from above, scattering the data
   contained in {\tt farray} on PET 0 across the {\tt array} object created
   further up, seems like a well defined operation. Looking at it a bit closer,
   it becomes clear that it is in fact more of a redistribution than a simple
   scatter operation. The general rule for such a "redist-scatter"  operation,
   of a Fortran array, located on a single PET, into an ESMF Array, is to 
   use the canonical ESMF sequence index scheme to label the elements of the
   Fortran array, and to send the data to the Array element with the same
   sequence index.
  
   The just described "redist-scatter" operation is much more general than
   the standard {\tt ESMF\_ArrayScatter()} method. It does not require shape
   matching, and supports full element reordering based on the sequence indices.
   Before {\tt farray} can be scattered across {\tt array} in the described way,
   it must be wrapped into an ESMF Array object itself, essentially labeling the
   array elements according to the canonical sequence index scheme.
    
%/////////////////////////////////////////////////////////////

 \begin{verbatim}
  distgridAux = ESMF_DistGridCreate(minIndex=(/1,1/), &
    maxIndex=(/petCount,10/), &
    regDecomp=(/1,1/), rc=rc) ! DistGrid with only 1 DE
 
\end{verbatim}
 
%/////////////////////////////////////////////////////////////

   The first step is to create a DistGrid object with only a single DE. This
   DE must be located on the PET on which the Fortran data array resides.
   In this example {\tt farray} holds data on PET 0, which is where the default
   DELayout will place the single DE defined in the DistGrid. If the {\tt farray}
   was setup on a different PET, an explicit DELayout would need to be created
   first, mapping the only DE to the PET on which the data is defined.
  
   Next the Array wrapper object can be created from the {\tt farray} and the
   just created DistGrid object. 
%/////////////////////////////////////////////////////////////

 \begin{verbatim}
  arrayAux = ESMF_ArrayCreate(farray=farray, distgrid=distgridAux, &
    indexflag=ESMF_INDEX_DELOCAL, rc=rc)
 
\end{verbatim}
 
%/////////////////////////////////////////////////////////////

   At this point all of the pieces are in place to use {\tt ESMF\_ArrayRedist()}
   to do the "redist-scatter" operation. The typical store/execute/release
   pattern must be followed. 
%/////////////////////////////////////////////////////////////

 \begin{verbatim}
  call ESMF_ArrayRedistStore(srcArray=arrayAux, dstArray=array, &
    routehandle=scatterHandle, rc=rc)
 
\end{verbatim}
 
%/////////////////////////////////////////////////////////////

 \begin{verbatim}
  call ESMF_ArrayRedist(srcArray=arrayAux, dstArray=array, &
    routehandle=scatterHandle, rc=rc)
 
\end{verbatim}
 
%/////////////////////////////////////////////////////////////

   In this example, after {\tt ESMF\_ArrayRedist()} was called, the content
   of {\tt array} on a 4 PET run would look like this:
   \begin{verbatim}
    PET 0:   101, 105, 109, ....  , 137
    PET 1:   102, 106, 110, ....  , 138
    PET 2:   103, 107, 111, ....  , 139
    PET 3:   104, 108, 112, ....  , 140
   \end{verbatim}
  
   Once set up, {\tt scatterHandle} can be used repeatedly to scatter data
   from {\tt farray} on PET 0 to all the DEs of {\tt array}. All of the
   resources should be released once {\tt scatterHandle} is no longer needed. 
%/////////////////////////////////////////////////////////////

 \begin{verbatim}
  call ESMF_ArrayRedistRelease(routehandle=scatterHandle, rc=rc)
 
\end{verbatim}
 
%/////////////////////////////////////////////////////////////

   The opposite operation, i.e. {\em gathering} of the {\tt array} data
   into {\tt farray} on PET 0, follows a very similar setup. In fact, the
   {\tt arrayAux} object already constructed for the scatter direction, can
   directly be re-used. The only thing that is different for the "redist-gather",
   are the {\tt srcArray} and {\tt dstArray} argument assignments, reflecting
   the opposite direction of data movement. 
%/////////////////////////////////////////////////////////////

 \begin{verbatim}
  call ESMF_ArrayRedistStore(srcArray=array, dstArray=arrayAux, &
    routehandle=gatherHandle, rc=rc)
 
\end{verbatim}
 
%/////////////////////////////////////////////////////////////

 \begin{verbatim}
  call ESMF_ArrayRedist(srcArray=array, dstArray=arrayAux, &
    routehandle=gatherHandle, rc=rc)
 
\end{verbatim}
 
%/////////////////////////////////////////////////////////////

   Just as for the scatter case, the {\tt gatherHandle} can be used repeatedly
   to gather data from {\tt array} into {\tt farray} on PET 0. All of the
   resources should be released once {\tt gatherHandle} is no longer needed. 
%/////////////////////////////////////////////////////////////

 \begin{verbatim}
  call ESMF_ArrayRedistRelease(routehandle=gatherHandle, rc=rc)
 
\end{verbatim}
 
%/////////////////////////////////////////////////////////////

   Finally the wrapper Array {\tt arrayAux} and the associated DistGrid object
   can also be destroyed. 
%/////////////////////////////////////////////////////////////

 \begin{verbatim}
  call ESMF_ArrayDestroy(arrayAux, rc=rc)
 
\end{verbatim}
 
%/////////////////////////////////////////////////////////////

 \begin{verbatim}
  call ESMF_DistGridDestroy(distgridAux, rc=rc)
 
\end{verbatim}
 
%/////////////////////////////////////////////////////////////

   Further, the primary data objects of this example must be deallocated
   and destroyed. 
%/////////////////////////////////////////////////////////////

 \begin{verbatim}
  deallocate(farray)
  
  call ESMF_ArrayDestroy(array, rc=rc)
 
\end{verbatim}
 
%/////////////////////////////////////////////////////////////

 \begin{verbatim}
  call ESMF_DistGridDestroy(distgrid, rc=rc)
 
\end{verbatim}

%...............................................................
\setlength{\parskip}{\oldparskip}
\setlength{\parindent}{\oldparindent}
\setlength{\baselineskip}{\oldbaselineskip}
