%                **** IMPORTANT NOTICE *****
% This LaTeX file has been automatically produced by ProTeX v. 1.1
% Any changes made to this file will likely be lost next time
% this file is regenerated from its source. Send questions 
% to Arlindo da Silva, dasilva@gsfc.nasa.gov
 
\setlength{\oldparskip}{\parskip}
\setlength{\parskip}{1.5ex}
\setlength{\oldparindent}{\parindent}
\setlength{\parindent}{0pt}
\setlength{\oldbaselineskip}{\baselineskip}
\setlength{\baselineskip}{11pt}
 
%--------------------- SHORT-HAND MACROS ----------------------
\def\bv{\begin{verbatim}}
\def\ev{\end{verbatim}}
\def\be{\begin{equation}}
\def\ee{\end{equation}}
\def\bea{\begin{eqnarray}}
\def\eea{\end{eqnarray}}
\def\bi{\begin{itemize}}
\def\ei{\end{itemize}}
\def\bn{\begin{enumerate}}
\def\en{\end{enumerate}}
\def\bd{\begin{description}}
\def\ed{\end{description}}
\def\({\left (}
\def\){\right )}
\def\[{\left [}
\def\]{\right ]}
\def\<{\left  \langle}
\def\>{\right \rangle}
\def\cI{{\cal I}}
\def\diag{\mathop{\rm diag}}
\def\tr{\mathop{\rm tr}}
%-------------------------------------------------------------

\markboth{Left}{Source File: ESMF\_ArrayRedistEx.F90,  Date: Tue May  5 20:59:44 MDT 2020
}

 
%/////////////////////////////////////////////////////////////

  
   \subsubsection{Communication -- Redist}
   \label{Array:Redist}
   
   Arrays used in different models often cover the same index space region,
   however, the distribution of the Arrays may be different, e.g. the models
   run on exclusive sets of PETs. Even if the Arrays are defined on the same
   list of PETs the decomposition may be different. 
%/////////////////////////////////////////////////////////////

 \begin{verbatim}
  srcDistgrid = ESMF_DistGridCreate(minIndex=(/1,1/), maxIndex=(/10,20/), &
    regDecomp=(/4,1/), rc=rc)
 
\end{verbatim}
 
%/////////////////////////////////////////////////////////////

 \begin{verbatim}
  dstDistgrid = ESMF_DistGridCreate(minIndex=(/1,1/), maxIndex=(/10,20/), &
    regDecomp=(/1,4/), rc=rc)
 
\end{verbatim}
 
%/////////////////////////////////////////////////////////////

   The number of elements covered by {\tt srcDistgrid} is identical to the number
   of elements covered by {\tt dstDistgrid} -- in fact the index space regions
   covered by both DistGrid objects are congruent. However, the decomposition
   defined by {\tt regDecomp}, and consequently the distribution of source and
   destination, are different. 
%/////////////////////////////////////////////////////////////

 \begin{verbatim}
  call ESMF_ArraySpecSet(arrayspec, typekind=ESMF_TYPEKIND_R8, rank=2, rc=rc)
 
\end{verbatim}
 
%/////////////////////////////////////////////////////////////

 \begin{verbatim}
  srcArray = ESMF_ArrayCreate(arrayspec=arrayspec, distgrid=srcDistgrid, rc=rc)
 
\end{verbatim}
 
%/////////////////////////////////////////////////////////////

 \begin{verbatim}
  dstArray = ESMF_ArrayCreate(arrayspec=arrayspec, distgrid=dstDistgrid, rc=rc)
 
\end{verbatim}
 
%/////////////////////////////////////////////////////////////

   By construction {\tt srcArray} and {\tt dstArray} are of identical type and
   kind. Further the number of exclusive elements matches between both Arrays.
   These are the prerequisites for the application of an Array redistribution
   in default mode. In order to increase performance of the actual 
   redistribution the communication pattern is precomputed once, and stored in
   an {\tt ESMF\_RouteHandle} object. 
%/////////////////////////////////////////////////////////////

 \begin{verbatim}
  call ESMF_ArrayRedistStore(srcArray=srcArray, dstArray=dstArray, &
    routehandle=redistHandle, rc=rc)
 
\end{verbatim}
 
%/////////////////////////////////////////////////////////////
 
%/////////////////////////////////////////////////////////////

   The {\tt redistHandle} can now be used repeatedly to transfer data from
   {\tt srcArray} to {\tt dstArray}. 
%/////////////////////////////////////////////////////////////

 \begin{verbatim}
  call ESMF_ArrayRedist(srcArray=srcArray, dstArray=dstArray, &
    routehandle=redistHandle, rc=rc)
 
\end{verbatim}
 
%/////////////////////////////////////////////////////////////

   \begin{sloppypar}
   The use of the precomputed {\tt redistHandle} is {\em not} restricted to
   the ({\tt srcArray}, {\tt dstArray}) pair. Instead the {\tt redistHandle}
   can be used to redistribute data between any two Arrays that are compatible
   with the Array pair used during precomputation. I.e. any pair of Arrays that
   matches {\tt srcArray} and {\tt dstArray} in {\em type}, {\em kind}, and 
   memory layout of the {\em distributed} dimensions. However, the size, number, 
   and index order of {\em undistributed} dimensions may be different.
   See section \ref{RH:Reusability} for a more detailed discussion of
   RouteHandle reusability.
   \end{sloppypar}
  
   The transferability of RouteHandles between Array pairs can greatly reduce
   the number of communication store calls needed.
   In a typical application Arrays are often defined on the same decomposition,
   typically leading to congruent distributed dimensions. For these Arrays, while
   they may not have the same shape or size in the undistributed dimensions,
   RouteHandles are reusable.
  
   For the current case, the {\tt redistHandle} was precomputed for simple 2D
   Arrays without undistributed dimensions. The RouteHandle transferability
   rule allows us to use this same RouteHandle to redistribute between two 
   3D Array that are built on the same 2D DistGrid, but have an undistributed
   dimension. Note that the undistributed dimension does not have to be in the
   same position on source and destination. Here the undistributed dimension is
   in position 2 for {\tt srcArray1}, and in position 1 for {\tt dstArray1}. 
%/////////////////////////////////////////////////////////////

 \begin{verbatim}
  call ESMF_ArraySpecSet(arrayspec3d, typekind=ESMF_TYPEKIND_R8, rank=3, rc=rc)
 
\end{verbatim}
 
%/////////////////////////////////////////////////////////////

 \begin{verbatim}
  srcArray1 = ESMF_ArrayCreate(arrayspec=arrayspec3d, distgrid=srcDistgrid, &
    distgridToArrayMap=(/1,3/), undistLBound=(/1/), undistUBound=(/10/), rc=rc)
 
\end{verbatim}
 
%/////////////////////////////////////////////////////////////

 \begin{verbatim}
  dstArray1 = ESMF_ArrayCreate(arrayspec=arrayspec3d, distgrid=dstDistgrid, &
    distgridToArrayMap=(/2,3/), undistLBound=(/1/), undistUBound=(/10/), rc=rc)
 
\end{verbatim}
 
%/////////////////////////////////////////////////////////////

  
%/////////////////////////////////////////////////////////////

 \begin{verbatim}
  call ESMF_ArrayRedist(srcArray=srcArray1, dstArray=dstArray1, &
    routehandle=redistHandle, rc=rc)
 
\end{verbatim}
 
%/////////////////////////////////////////////////////////////

   The following variation of the code shows that the same RouteHandle can be
   applied to an Array pair where the number of undistributed dimensions does
   not match between source and destination Array. Here we prepare a source
   Array with {\em two} undistributed dimensions, in position 1 and 3, that 
   multiply out to 2x5=10 undistributed elements. The destination array is the
   same as before with only a {\em single} undistributed dimension in position 1
   of size 10. 
%/////////////////////////////////////////////////////////////

 \begin{verbatim}
  call ESMF_ArraySpecSet(arrayspec4d, typekind=ESMF_TYPEKIND_R8, rank=4, rc=rc)
 
\end{verbatim}
 
%/////////////////////////////////////////////////////////////

 \begin{verbatim}
  srcArray2 = ESMF_ArrayCreate(arrayspec=arrayspec4d, distgrid=srcDistgrid, &
    distgridToArrayMap=(/2,4/), undistLBound=(/1,1/), undistUBound=(/2,5/), &
    rc=rc)
 
\end{verbatim}
 
%/////////////////////////////////////////////////////////////

 \begin{verbatim}
  call ESMF_ArrayRedist(srcArray=srcArray2, dstArray=dstArray1, &
    routehandle=redistHandle, rc=rc)
 
\end{verbatim}
 
%/////////////////////////////////////////////////////////////

   When done, the resources held by {\tt redistHandle} need to be deallocated
   by the user code before the RouteHandle becomes inaccessible. 
%/////////////////////////////////////////////////////////////

 \begin{verbatim}
  call ESMF_ArrayRedistRelease(routehandle=redistHandle, rc=rc)
 
\end{verbatim}
 
%/////////////////////////////////////////////////////////////

   \begin{sloppypar}
   In {\em default} mode, i.e. without providing the optional
   {\tt srcToDstTransposeMap} argument, {\tt ESMF\_ArrayRedistStore()} does not
   require equal number of dimensions in source and destination Array. Only the
   total number of elements must match.
   Specifying {\tt srcToDstTransposeMap} switches {\tt ESMF\_ArrayRedistStore()}
   into {\em transpose} mode. In this mode each dimension of {\tt srcArray}
   is uniquely associated with a dimension in {\tt dstArray}, and the sizes of 
   associated dimensions must match for each pair.
   \end{sloppypar}
    
%/////////////////////////////////////////////////////////////

 \begin{verbatim}
  dstDistgrid = ESMF_DistGridCreate(minIndex=(/1,1/), maxIndex=(/20,10/), &
      rc=rc)
 
\end{verbatim}
 
%/////////////////////////////////////////////////////////////

 \begin{verbatim}
  dstArray = ESMF_ArrayCreate(arrayspec=arrayspec, distgrid=dstDistgrid, rc=rc)
 
\end{verbatim}
 
%/////////////////////////////////////////////////////////////

   This {\tt dstArray} object covers a 20 x 10 index space while the
   {\tt srcArray}, defined further up, covers a 10 x 20 index space. Setting
   {\tt srcToDstTransposeMap = (/2,1/)} will associate the first and second 
   dimension of {\tt srcArray} with the second and first dimension of
   {\tt dstArray}, respectively. This corresponds to a transpose of dimensions.
   Since the decomposition and distribution of dimensions may be different for
   source and destination redistribution may occur at the same time. 
%/////////////////////////////////////////////////////////////

 \begin{verbatim}
  call ESMF_ArrayRedistStore(srcArray=srcArray, dstArray=dstArray, &
    routehandle=redistHandle, srcToDstTransposeMap=(/2,1/), rc=rc)
 
\end{verbatim}
 
%/////////////////////////////////////////////////////////////

 \begin{verbatim}
  call ESMF_ArrayRedist(srcArray=srcArray, dstArray=dstArray, &
    routehandle=redistHandle, rc=rc)
 
\end{verbatim}
 
%/////////////////////////////////////////////////////////////

   \begin{sloppypar}
   The transpose mode of {\tt ESMF\_ArrayRedist()} is not limited to
   distributed dimensions of Arrays. The {\tt srcToDstTransposeMap} argument
   can be used to transpose undistributed dimensions in the same manner.
   Furthermore transposing distributed and undistributed dimensions between
   Arrays is also supported.
   \end{sloppypar}
  
   The {\tt srcArray} used in the following examples is of rank 4 with 2 
   distributed and 2 undistributed dimensions. The distributed dimensions
   are the two first dimensions of the Array and are distributed according to the
   {\tt srcDistgrid} which describes a total index space region of 100 x 200
   elements. The last two Array dimensions are undistributed dimensions of size
   2 and 3, respectively. 
%/////////////////////////////////////////////////////////////

 \begin{verbatim}
  call ESMF_ArraySpecSet(arrayspec, typekind=ESMF_TYPEKIND_R8, rank=4, rc=rc)
 
\end{verbatim}
 
%/////////////////////////////////////////////////////////////

 \begin{verbatim}
  srcDistgrid = ESMF_DistGridCreate(minIndex=(/1,1/), maxIndex=(/100,200/), &
    rc=rc)
 
\end{verbatim}
 
%/////////////////////////////////////////////////////////////

 \begin{verbatim}
  srcArray = ESMF_ArrayCreate(arrayspec=arrayspec, distgrid=srcDistgrid, &
    undistLBound=(/1,1/), undistUBound=(/2,3/), rc=rc)
 
\end{verbatim}
 
%/////////////////////////////////////////////////////////////

   The first {\tt dstArray} to consider is defined on a DistGrid that also 
   describes a 100 x 200 index space region. The distribution indicated
   by {\tt dstDistgrid} may be different from the source distribution. Again
   the first two Array dimensions are associated with the DistGrid dimensions in
   sequence. Furthermore, the last two Array dimensions are undistributed
   dimensions, however, the sizes are 3 and 2, respectively. 
%/////////////////////////////////////////////////////////////

 \begin{verbatim}
  dstDistgrid = ESMF_DistGridCreate(minIndex=(/1,1/), maxIndex=(/100,200/), &
    rc=rc)
 
\end{verbatim}
 
%/////////////////////////////////////////////////////////////

 \begin{verbatim}
  dstArray = ESMF_ArrayCreate(arrayspec=arrayspec, distgrid=dstDistgrid, &
    undistLBound=(/1,1/), undistUBound=(/3,2/), rc=rc)
 
\end{verbatim}
 
%/////////////////////////////////////////////////////////////

   The desired mapping between {\tt srcArray} and {\tt dstArray} dimensions
   is expressed by {\tt srcToDstTransposeMap = (/1,2,4,3/)}, transposing only
   the two undistributed dimensions. 
%/////////////////////////////////////////////////////////////

 \begin{verbatim}
  call ESMF_ArrayRedistStore(srcArray=srcArray, dstArray=dstArray, &
    routehandle=redistHandle, srcToDstTransposeMap=(/1,2,4,3/), rc=rc)
 
\end{verbatim}
 
%/////////////////////////////////////////////////////////////

 \begin{verbatim}
  call ESMF_ArrayRedist(srcArray=srcArray, dstArray=dstArray, &
    routehandle=redistHandle, rc=rc)
 
\end{verbatim}
 
%/////////////////////////////////////////////////////////////

   Next consider a {\tt dstArray} that is defined on the same {\tt dstDistgrid},
   but with a different order of Array dimensions. The desired order is
   specified during Array creation using the argument 
   {\tt distgridToArrayMap = (/2,3/)}. This map associates the first and second
   DistGrid dimensions with the second and third Array dimensions, respectively,
   leaving Array dimensions one and four undistributed. 
%/////////////////////////////////////////////////////////////

 \begin{verbatim}
  dstArray = ESMF_ArrayCreate(arrayspec=arrayspec, distgrid=dstDistgrid, &
    distgridToArrayMap=(/2,3/), undistLBound=(/1,1/), undistUBound=(/3,2/), &
    rc=rc)
 
\end{verbatim}
 
%/////////////////////////////////////////////////////////////

   Again the sizes of the undistributed dimensions are chosen in reverse order
   compared to {\tt srcArray}. The desired transpose mapping in this case will
   be {\tt srcToDstTransposeMap = (/2,3,4,1/)}. 
%/////////////////////////////////////////////////////////////

 \begin{verbatim}
  call ESMF_ArrayRedistStore(srcArray=srcArray, dstArray=dstArray, &
    routehandle=redistHandle, srcToDstTransposeMap=(/2,3,4,1/), rc=rc)
 
\end{verbatim}
 
%/////////////////////////////////////////////////////////////

 \begin{verbatim}
  call ESMF_ArrayRedist(srcArray=srcArray, dstArray=dstArray, &
    routehandle=redistHandle, rc=rc)
 
\end{verbatim}
 
%/////////////////////////////////////////////////////////////

   Finally consider the case where {\tt dstArray} is constructed on a 
   200 x 3 index space and where the undistributed dimensions are of size
   100 and 2. 
%/////////////////////////////////////////////////////////////

 \begin{verbatim}
  dstDistgrid = ESMF_DistGridCreate(minIndex=(/1,1/), maxIndex=(/200,3/), &
    rc=rc)
 
\end{verbatim}
 
%/////////////////////////////////////////////////////////////

 \begin{verbatim}
  dstArray = ESMF_ArrayCreate(arrayspec=arrayspec, distgrid=dstDistgrid, &
    undistLBound=(/1,1/), undistUBound=(/100,2/), rc=rc)
 
\end{verbatim}
 
%/////////////////////////////////////////////////////////////

   By construction {\tt srcArray} and {\tt dstArray} hold the same number of
   elements, albeit in a very different layout. Nevertheless, with a
   {\tt srcToDstTransposeMap} that maps matching dimensions from source to
   destination an Array redistribution becomes a well defined operation between
   {\tt srcArray} and {\tt dstArray}. 
%/////////////////////////////////////////////////////////////

 \begin{verbatim}
  call ESMF_ArrayRedistStore(srcArray=srcArray, dstArray=dstArray, &
    routehandle=redistHandle, srcToDstTransposeMap=(/3,1,4,2/), rc=rc)
 
\end{verbatim}
 
%/////////////////////////////////////////////////////////////

 \begin{verbatim}
  call ESMF_ArrayRedist(srcArray=srcArray, dstArray=dstArray, &
    routehandle=redistHandle, rc=rc)
 
\end{verbatim}
 
%/////////////////////////////////////////////////////////////

   \begin{sloppypar}
   The default mode of Array redistribution, i.e. without providing a
   {\tt srcToDstTransposeMap} to {\tt ESMF\_ArrayRedistStore()}, also supports
   undistributed Array dimensions. The requirement in this case is that the 
   total undistributed element count, i.e. the product of the sizes of all
   undistributed dimensions, be the same for source and destination Array.
   In this mode the number of undistributed dimensions need not match between
   source and destination.
   \end{sloppypar} 
%/////////////////////////////////////////////////////////////

 \begin{verbatim}
  call ESMF_ArraySpecSet(arrayspec, typekind=ESMF_TYPEKIND_R8, rank=4, rc=rc)
 
\end{verbatim}
 
%/////////////////////////////////////////////////////////////

 \begin{verbatim}
  srcDistgrid = ESMF_DistGridCreate(minIndex=(/1,1/), maxIndex=(/10,20/), &
    regDecomp=(/4,1/), rc=rc)
 
\end{verbatim}
 
%/////////////////////////////////////////////////////////////

 \begin{verbatim}
  srcArray = ESMF_ArrayCreate(arrayspec=arrayspec, distgrid=srcDistgrid, &
    undistLBound=(/1,1/), undistUBound=(/2,4/), rc=rc)
 
\end{verbatim}
 
%/////////////////////////////////////////////////////////////

 \begin{verbatim}
  dstDistgrid = ESMF_DistGridCreate(minIndex=(/1,1/), maxIndex=(/10,20/), &
    regDecomp=(/1,4/), rc=rc)
 
\end{verbatim}
 
%/////////////////////////////////////////////////////////////
 
%/////////////////////////////////////////////////////////////

 \begin{verbatim}
  dstArray = ESMF_ArrayCreate(arrayspec=arrayspec, distgrid=dstDistgrid, &
    distgridToArrayMap=(/2,3/), undistLBound=(/1,1/), undistUBound=(/2,4/), &
    rc=rc)
 
\end{verbatim}
 
%/////////////////////////////////////////////////////////////

   Both {\tt srcArray} and {\tt dstArray} have two undistributed dimensions and
   a total count of undistributed elements of $ 2 \times 4 = 8$.
  
   The Array redistribution operation is defined in terms of sequentialized
   undistributed dimensions. In the above case this means that a unique sequence
   index will be assigned to each of the 8 undistributed elements. The sequence
   indices will be 1, 2, ..., 8, where sequence index 1 is assigned to the first
   element in the first (i.e. fastest varying in memory) undistributed dimension.
   The following undistributed elements are labeled in consecutive order as they
   are stored in memory. 
%/////////////////////////////////////////////////////////////

 \begin{verbatim}
  call ESMF_ArrayRedistStore(srcArray=srcArray, dstArray=dstArray, &
    routehandle=redistHandle, rc=rc)
 
\end{verbatim}
 
%/////////////////////////////////////////////////////////////
 
%/////////////////////////////////////////////////////////////

   The redistribution operation by default applies the identity operation between
   the elements of undistributed dimensions. This means that source element with
   sequence index 1 will be mapped against destination element with sequence
   index 1 and so forth. Because of the way source and destination Arrays
   in the current example were constructed this corresponds to a mapping of
   dimensions 3 and 4 on {\tt srcArray} to dimensions 1 and 4 on {\tt dstArray},
   respectively. 
%/////////////////////////////////////////////////////////////

 \begin{verbatim}
  call ESMF_ArrayRedist(srcArray=srcArray, dstArray=dstArray, &
    routehandle=redistHandle, rc=rc)
 
\end{verbatim}
 
%/////////////////////////////////////////////////////////////

   Array redistribution does {\em not} require the same number of undistributed
   dimensions in source and destination Array, merely the total number of
   undistributed elements must match. 
%/////////////////////////////////////////////////////////////

 \begin{verbatim}
  call ESMF_ArraySpecSet(arrayspec, typekind=ESMF_TYPEKIND_R8, rank=3, rc=rc)
 
\end{verbatim}
 
%/////////////////////////////////////////////////////////////

 \begin{verbatim}
  dstArray = ESMF_ArrayCreate(arrayspec=arrayspec, distgrid=dstDistgrid, &
    distgridToArrayMap=(/1,3/), undistLBound=(/11/), undistUBound=(/18/), &
    rc=rc)
 
\end{verbatim}
 
%/////////////////////////////////////////////////////////////

   This {\tt dstArray} object only has a single undistributed dimension, while
   the {\tt srcArray}, defined further back, has two undistributed dimensions.
   However, the total undistributed element count for both Arrays is 8. 
%/////////////////////////////////////////////////////////////

 \begin{verbatim}
  call ESMF_ArrayRedistStore(srcArray=srcArray, dstArray=dstArray, &
    routehandle=redistHandle, rc=rc)
 
\end{verbatim}
 
%/////////////////////////////////////////////////////////////
 
%/////////////////////////////////////////////////////////////

   In this case the default identity operation between the elements of
   undistributed dimensions corresponds to a {\em merging} of dimensions
   3 and 4 on {\tt srcArray} into dimension 2 on {\tt dstArray}. 
%/////////////////////////////////////////////////////////////

 \begin{verbatim}
  call ESMF_ArrayRedist(srcArray=srcArray, dstArray=dstArray, &
    routehandle=redistHandle, rc=rc)
 
\end{verbatim}

%...............................................................
\setlength{\parskip}{\oldparskip}
\setlength{\parindent}{\oldparindent}
\setlength{\baselineskip}{\oldbaselineskip}
