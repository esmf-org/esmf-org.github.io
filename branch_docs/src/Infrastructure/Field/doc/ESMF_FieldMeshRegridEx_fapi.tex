%                **** IMPORTANT NOTICE *****
% This LaTeX file has been automatically produced by ProTeX v. 1.1
% Any changes made to this file will likely be lost next time
% this file is regenerated from its source. Send questions 
% to Arlindo da Silva, dasilva@gsfc.nasa.gov
 
\setlength{\oldparskip}{\parskip}
\setlength{\parskip}{1.5ex}
\setlength{\oldparindent}{\parindent}
\setlength{\parindent}{0pt}
\setlength{\oldbaselineskip}{\baselineskip}
\setlength{\baselineskip}{11pt}
 
%--------------------- SHORT-HAND MACROS ----------------------
\def\bv{\begin{verbatim}}
\def\ev{\end{verbatim}}
\def\be{\begin{equation}}
\def\ee{\end{equation}}
\def\bea{\begin{eqnarray}}
\def\eea{\end{eqnarray}}
\def\bi{\begin{itemize}}
\def\ei{\end{itemize}}
\def\bn{\begin{enumerate}}
\def\en{\end{enumerate}}
\def\bd{\begin{description}}
\def\ed{\end{description}}
\def\({\left (}
\def\){\right )}
\def\[{\left [}
\def\]{\right ]}
\def\<{\left  \langle}
\def\>{\right \rangle}
\def\cI{{\cal I}}
\def\diag{\mathop{\rm diag}}
\def\tr{\mathop{\rm tr}}
%-------------------------------------------------------------

\markboth{Left}{Source File: ESMF\_FieldMeshRegridEx.F90,  Date: Tue May  5 21:00:02 MDT 2020
}

 
%/////////////////////////////////////////////////////////////

  \subsubsection{Field regrid example: Mesh to Mesh}
   This example demonstrates the regridding process between Fields created on Meshes. First
   the Meshes are created. This example omits the setup of the arrays describing the Mesh, but please see
   Section~\ref{sec:mesh:usage:meshCreation} for examples of this. After creation Fields are constructed on the Meshes, 
   and then ESMF\_FieldRegridStore() is called to construct a RouteHandle implementing the regrid operation. Finally, ESMF\_FieldRegrid() is
   called with the Fields and the RouteHandle to do the interpolation between the source Field and 
   destination Field.  Note the coordinates of the source and destination Mesh should be in degrees.
    
%/////////////////////////////////////////////////////////////

 \begin{verbatim}

  !!!!!!!!!!!!!!!!!!!!!!!!!!!!!!!!!!!!!!!!!!!!!!!
  ! Create Source Mesh
  !!!!!!!!!!!!!!!!!!!!!!!!!!!!!!!!!!!!!!!!!!!!!!!

  ! Create the Mesh structure.
  ! For brevity's sake, the code to fill the Mesh creation 
  ! arrays is omitted from this example. However, here
  ! is a brief description of the arrays:
  ! srcNodeIds    - the global ids for the src nodes
  ! srcNodeCoords - the coordinates for the src nodes
  ! srcNodeOwners - which PET owns each src node
  ! srcElemIds    - the global ids of the src elements
  ! srcElemTypes  - the topological shape of each src element
  ! srcElemConn   - how to connect the nodes to form the elements
  !                 in the source mesh
  ! Several examples of setting up these arrays can be seen in
  ! the Mesh Section "Mesh Creation". 
  srcMesh=ESMF_MeshCreate(parametricDim=2,spatialDim=2, &
         nodeIds=srcNodeIds, nodeCoords=srcNodeCoords, &
         nodeOwners=srcNodeOwners, elementIds=srcElemIds,&
         elementTypes=srcElemTypes, elementConn=srcElemConn, rc=rc)

  if (rc /= ESMF_SUCCESS) call ESMF_Finalize(endflag=ESMF_END_ABORT)



  !!!!!!!!!!!!!!!!!!!!!!!!!!!!!!!!!!!!!!!!!!!!!!!
  ! Create and Fill Source Field
  !!!!!!!!!!!!!!!!!!!!!!!!!!!!!!!!!!!!!!!!!!!!!!!

  ! Set description of source Field
  call ESMF_ArraySpecSet(arrayspec, 1, ESMF_TYPEKIND_R8, rc=rc)

  if (rc /= ESMF_SUCCESS) call ESMF_Finalize(endflag=ESMF_END_ABORT)

  ! Create source Field
  srcField = ESMF_FieldCreate(srcMesh, arrayspec, &
                        name="source", rc=rc)

  if (rc /= ESMF_SUCCESS) call ESMF_Finalize(endflag=ESMF_END_ABORT)

  ! Get source Field data pointer to put data into
  call ESMF_FieldGet(srcField, 0, fptr1D,  rc=rc)

  if (rc /= ESMF_SUCCESS) call ESMF_Finalize(endflag=ESMF_END_ABORT)

  ! Get number of local nodes to allocate space
  ! to hold local node coordinates
  call ESMF_MeshGet(srcMesh, &
         numOwnedNodes=numOwnedNodes, rc=rc)

  if (rc /= ESMF_SUCCESS) call ESMF_Finalize(endflag=ESMF_END_ABORT)

  ! Allocate space to hold local node coordinates
  ! (spatial dimension of Mesh*number of local nodes)
  allocate(ownedNodeCoords(2*numOwnedNodes))

  ! Get local node coordinates
  call ESMF_MeshGet(srcMesh, &
         ownedNodeCoords=ownedNodeCoords, rc=rc)

  if (rc /= ESMF_SUCCESS) call ESMF_Finalize(endflag=ESMF_END_ABORT)

  ! Set the source Field to the function 20.0+x+y
  do i=1,numOwnedNodes
    ! Get coordinates
    x=ownedNodeCoords(2*i-1)
    y=ownedNodeCoords(2*i)

   ! Set source function
   fptr1D(i) = 20.0+x+y
  enddo

  ! Deallocate local node coordinates
  deallocate(ownedNodeCoords)


  !!!!!!!!!!!!!!!!!!!!!!!!!!!!!!!!!!!!!!!!!!!!!!!
  ! Create Destination Mesh
  !!!!!!!!!!!!!!!!!!!!!!!!!!!!!!!!!!!!!!!!!!!!!!!

  ! Create the Mesh structure.
  ! For brevity's sake, the code to fill the Mesh creation 
  ! arrays is omitted from this example. However, here
  ! is a brief description of the arrays:
  ! dstNodeIds    - the global ids for the dst nodes
  ! dstNodeCoords - the coordinates for the dst nodes
  ! dstNodeOwners - which PET owns each dst node
  ! dstElemIds    - the global ids of the dst elements
  ! dstElemTypes  - the topological shape of each dst element
  ! dstElemConn   - how to connect the nodes to form the elements
  !                 in the destination mesh
  ! Several examples of setting up these arrays can be seen in
  ! the Mesh Section "Mesh Creation". 
  dstMesh=ESMF_MeshCreate(parametricDim=2,spatialDim=2, &
         nodeIds=dstNodeIds, nodeCoords=dstNodeCoords, &
         nodeOwners=dstNodeOwners, elementIds=dstElemIds,&
         elementTypes=dstElemTypes, elementConn=dstElemConn, rc=rc)

  if (rc /= ESMF_SUCCESS) call ESMF_Finalize(endflag=ESMF_END_ABORT)


  !!!!!!!!!!!!!!!!!!!!!!!!!!!!!!!!!!!!!!!!!!!!!!!
  ! Create Destination Field
  !!!!!!!!!!!!!!!!!!!!!!!!!!!!!!!!!!!!!!!!!!!!!!!

  ! Set description of source Field
  call ESMF_ArraySpecSet(arrayspec, 1, ESMF_TYPEKIND_R8, rc=rc)

  if (rc /= ESMF_SUCCESS) call ESMF_Finalize(endflag=ESMF_END_ABORT)

  ! Create destination Field
  dstField = ESMF_FieldCreate(dstMesh, arrayspec, &
                        name="destination", rc=rc)

  if (rc /= ESMF_SUCCESS) call ESMF_Finalize(endflag=ESMF_END_ABORT)

  !!!!!!!!!!!!!!!!!!!!!!!!!!!!!!!!!!!!!!!!!!!!!!!
  ! Do Regrid
  !!!!!!!!!!!!!!!!!!!!!!!!!!!!!!!!!!!!!!!!!!!!!!!

  ! Compute RouteHandle which contains the regrid operation
  call ESMF_FieldRegridStore( &
          srcField, &
          dstField=dstField, &
          routeHandle=routeHandle, &
          regridmethod=ESMF_REGRIDMETHOD_BILINEAR, &
          rc=rc)

  if (rc /= ESMF_SUCCESS) call ESMF_Finalize(endflag=ESMF_END_ABORT)

  ! Perform Regrid operation moving data from srcField to dstField
  call ESMF_FieldRegrid(srcField, dstField, routeHandle, rc=rc)


  if (rc /= ESMF_SUCCESS) call ESMF_Finalize(endflag=ESMF_END_ABORT)

  !!!!!!!!!!!!!!!!!!!!!!!!!!!!!!!!!!!!!!!!!!!!!!!
  ! dstField now contains the interpolated data.
  ! If the Meshes don't change, then routeHandle
  ! may be used repeatedly to interpolate from 
  ! srcField to dstField.  
  !!!!!!!!!!!!!!!!!!!!!!!!!!!!!!!!!!!!!!!!!!!!!!!

   
  ! User code to use the routeHandle, Fields, and
  ! Meshes goes here before they are freed below.


  !!!!!!!!!!!!!!!!!!!!!!!!!!!!!!!!!!!!!!!!!!!!!!!
  ! Free the objects created in the example.
  !!!!!!!!!!!!!!!!!!!!!!!!!!!!!!!!!!!!!!!!!!!!!!!

  ! Free the RouteHandle
  call ESMF_FieldRegridRelease(routeHandle, rc=rc)

  if (rc /= ESMF_SUCCESS) call ESMF_Finalize(endflag=ESMF_END_ABORT)

  ! Free the Fields
  call ESMF_FieldDestroy(srcField, rc=rc)

  if (rc /= ESMF_SUCCESS) call ESMF_Finalize(endflag=ESMF_END_ABORT)

  call ESMF_FieldDestroy(dstField, rc=rc)

  if (rc /= ESMF_SUCCESS) call ESMF_Finalize(endflag=ESMF_END_ABORT)

  ! Free the Meshes
  call ESMF_MeshDestroy(dstMesh, rc=rc)

  if (rc /= ESMF_SUCCESS) call ESMF_Finalize(endflag=ESMF_END_ABORT)

  call ESMF_MeshDestroy(srcMesh, rc=rc)
 
  if (rc /= ESMF_SUCCESS) call ESMF_Finalize(endflag=ESMF_END_ABORT)

 
\end{verbatim}

%...............................................................
\setlength{\parskip}{\oldparskip}
\setlength{\parindent}{\oldparindent}
\setlength{\baselineskip}{\oldbaselineskip}
