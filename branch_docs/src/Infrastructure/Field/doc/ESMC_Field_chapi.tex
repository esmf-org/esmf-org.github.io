%                **** IMPORTANT NOTICE *****
% This LaTeX file has been automatically produced by ProTeX v. 1.1
% Any changes made to this file will likely be lost next time
% this file is regenerated from its source. Send questions 
% to Arlindo da Silva, dasilva@gsfc.nasa.gov
 
\setlength{\oldparskip}{\parskip}
\setlength{\parskip}{1.5ex}
\setlength{\oldparindent}{\parindent}
\setlength{\parindent}{0pt}
\setlength{\oldbaselineskip}{\baselineskip}
\setlength{\baselineskip}{11pt}
 
%--------------------- SHORT-HAND MACROS ----------------------
\def\bv{\begin{verbatim}}
\def\ev{\end{verbatim}}
\def\be{\begin{equation}}
\def\ee{\end{equation}}
\def\bea{\begin{eqnarray}}
\def\eea{\end{eqnarray}}
\def\bi{\begin{itemize}}
\def\ei{\end{itemize}}
\def\bn{\begin{enumerate}}
\def\en{\end{enumerate}}
\def\bd{\begin{description}}
\def\ed{\end{description}}
\def\({\left (}
\def\){\right )}
\def\[{\left [}
\def\]{\right ]}
\def\<{\left  \langle}
\def\>{\right \rangle}
\def\cI{{\cal I}}
\def\diag{\mathop{\rm diag}}
\def\tr{\mathop{\rm tr}}
%-------------------------------------------------------------

\markboth{Left}{Source File: ESMC\_Field.h,  Date: Tue May  5 21:00:00 MDT 2020
}

 
%/////////////////////////////////////////////////////////////
\subsubsection [ESMC\_FieldCreateGridArraySpec] {ESMC\_FieldCreateGridArraySpec - Create a Field from Grid and ArraySpec}


  
\bigskip{\sf INTERFACE:}
\begin{verbatim} ESMC_Field ESMC_FieldCreateGridArraySpec(
   ESMC_Grid grid,                           // in
   ESMC_ArraySpec arrayspec,                 // in
   enum ESMC_StaggerLoc staggerloc,          // in
   ESMC_InterArrayInt *gridToFieldMap,       // in
   ESMC_InterArrayInt *ungriddedLBound,      // in
   ESMC_InterArrayInt *ungriddedUBound,      // in
   const char *name,                         // in
   int *rc                                   // out
 );
 \end{verbatim}{\em RETURN VALUE:}
\begin{verbatim}    Newly created ESMC_Field object.\end{verbatim}
{\sf DESCRIPTION:\\ }


  
    Creates a {\tt ESMC\_Field} object.
  
    The arguments are:
    \begin{description}
    \item[grid]
      A {\tt ESMC\_Grid} object.
    \item[arrayspec]
      A {\tt ESMC\_ArraySpec} object describing data type and kind specification.
    \item[staggerloc]
      Stagger location of data in grid cells. The default value is 
      ESMF\_STAGGERLOC\_CENTER.
    \item[gridToFieldMap]
      List with number of elements equal to the grid's dimCount. The list
      elements map each dimension of the grid to a dimension in the field by
      specifying the appropriate field dimension index. The default is to map all of
      the grid's dimensions against the lowest dimensions of the field in sequence,
      i.e. gridToFieldMap = (/1,2,3,.../). The values of all gridToFieldMap entries
      must be greater than or equal to one and smaller than or equal to the field
      rank. It is erroneous to specify the same gridToFieldMap entry multiple times.
      The total ungridded dimensions in the field  are the total field dimensions
      less the dimensions in the grid. Ungridded dimensions must be in the same order
      they are stored in the field. If the Field dimCount is less than the Mesh
      dimCount then the default gridToFieldMap will contain zeros for the rightmost
      entries. A zero entry in the gridToFieldMap indicates that the particular Mesh
      dimension will be replicating the Field across the DEs along this direction.
    \item[ungriddedLBound]
      Lower bounds of the ungridded dimensions of the field. The number of elements
      in the ungriddedLBound is equal to the number of ungridded dimensions in the
      field. All ungridded dimensions of the field are also undistributed. When field
      dimension count is greater than grid dimension count, both ungriddedLBound and
      ungriddedUBound must be specified. When both are specified the values are
      checked for consistency. Note that the the ordering of these ungridded
      dimensions is the same as their order in the field.  
    \item[ungriddedUBound]
      Upper bounds of the ungridded dimensions of the field. The number of elements
      in the ungriddedUBound is equal to the number of ungridded dimensions in the
      field. All ungridded dimensions of the field are also undistributed. When field
      dimension count is greater than grid dimension count, both ungriddedLBound and
      ungriddedUBound must be specified. When both are specified the values are
      checked for consistency. Note that the the ordering of these ungridded
      dimensions is the same as their order in the field.  
    \item[{[name]}]
      The name for the newly created field.  If not specified, i.e. NULL,
      a default unique name will be generated: "FieldNNN" where NNN
      is a unique sequence number from 001 to 999.
    \item[{[rc]}]
      Return code; equals {\tt ESMF\_SUCCESS} if there are no errors.
    \end{description}
   
%/////////////////////////////////////////////////////////////
 
\mbox{}\hrulefill\ 
 
\subsubsection [ESMC\_FieldCreateGridTypeKind] {ESMC\_FieldCreateGridTypeKind - Create a Field from Grid and typekind}


  
\bigskip{\sf INTERFACE:}
\begin{verbatim} ESMC_Field ESMC_FieldCreateGridTypeKind(
   ESMC_Grid grid,                           // in
   enum ESMC_TypeKind_Flag typekind,         // in
   enum ESMC_StaggerLoc staggerloc,          // in
   ESMC_InterArrayInt *gridToFieldMap,       // in
   ESMC_InterArrayInt *ungriddedLBound,      // in
   ESMC_InterArrayInt *ungriddedUBound,      // in
   const char *name,                         // in
   int *rc                                   // out
 );
 \end{verbatim}{\em RETURN VALUE:}
\begin{verbatim}    Newly created ESMC_Field object.\end{verbatim}
{\sf DESCRIPTION:\\ }


  
    Creates a {\tt ESMC\_Field} object.
  
    The arguments are:
    \begin{description}
    \item[grid]
      A {\tt ESMC\_Grid} object.
    \item[typekind]
      The ESMC\_TypeKind\_Flag that describes this Field data.
    \item[staggerloc]
      Stagger location of data in grid cells. The default value is 
      ESMF\_STAGGERLOC\_CENTER.
    \item[gridToFieldMap]
      List with number of elements equal to the grid's dimCount. The list
      elements map each dimension of the grid to a dimension in the field by
      specifying the appropriate field dimension index. The default is to map all of
      the grid's dimensions against the lowest dimensions of the field in sequence,
      i.e. gridToFieldMap = (/1,2,3,.../). The values of all gridToFieldMap entries
      must be greater than or equal to one and smaller than or equal to the field
      rank. It is erroneous to specify the same gridToFieldMap entry multiple times.
      The total ungridded dimensions in the field  are the total field dimensions
      less the dimensions in the grid. Ungridded dimensions must be in the same order
      they are stored in the field. If the Field dimCount is less than the Mesh
      dimCount then the default gridToFieldMap will contain zeros for the rightmost
      entries. A zero entry in the gridToFieldMap indicates that the particular Mesh
      dimension will be replicating the Field across the DEs along this direction.
    \item[ungriddedLBound]
      Lower bounds of the ungridded dimensions of the field. The number of elements
      in the ungriddedLBound is equal to the number of ungridded dimensions in the
      field. All ungridded dimensions of the field are also undistributed. When field
      dimension count is greater than grid dimension count, both ungriddedLBound and
      ungriddedUBound must be specified. When both are specified the values are
      checked for consistency. Note that the the ordering of these ungridded
      dimensions is the same as their order in the field.  
    \item[ungriddedUBound]
      Upper bounds of the ungridded dimensions of the field. The number of elements
      in the ungriddedUBound is equal to the number of ungridded dimensions in the
      field. All ungridded dimensions of the field are also undistributed. When field
      dimension count is greater than grid dimension count, both ungriddedLBound and
      ungriddedUBound must be specified. When both are specified the values are
      checked for consistency. Note that the the ordering of these ungridded
      dimensions is the same as their order in the field.  
    \item[{[name]}]
      The name for the newly created field.  If not specified, i.e. NULL,
      a default unique name will be generated: "FieldNNN" where NNN
      is a unique sequence number from 001 to 999.
    \item[{[rc]}]
      Return code; equals {\tt ESMF\_SUCCESS} if there are no errors.
    \end{description}
   
%/////////////////////////////////////////////////////////////
 
\mbox{}\hrulefill\ 
 
\subsubsection [ESMC\_FieldCreateMeshArraySpec] {ESMC\_FieldCreateMeshArraySpec - Create a Field from Mesh and ArraySpec}


  
\bigskip{\sf INTERFACE:}
\begin{verbatim} ESMC_Field ESMC_FieldCreateMeshArraySpec(
   ESMC_Mesh mesh,                           // in
   ESMC_ArraySpec arrayspec,                 // in
   ESMC_InterArrayInt *gridToFieldMap,       // in
   ESMC_InterArrayInt *ungriddedLBound,      // in
   ESMC_InterArrayInt *ungriddedUBound,      // in
   const char *name,                         // in
   int *rc                                   // out
 );
 \end{verbatim}{\em RETURN VALUE:}
\begin{verbatim}    Newly created ESMC_Field object.\end{verbatim}
{\sf DESCRIPTION:\\ }


  
    Creates a {\tt ESMC\_Field} object.
  
    The arguments are:
    \begin{description}
    \item[mesh]
      A {\tt ESMC\_Mesh} object.
    \item[arrayspec]
      A {\tt ESMC\_ArraySpec} object describing data type and kind specification.
    \item[gridToFieldMap]
      List with number of elements equal to the grid's dimCount. The list
      elements map each dimension of the grid to a dimension in the field by
      specifying the appropriate field dimension index. The default is to map all of
      the grid's dimensions against the lowest dimensions of the field in sequence,
      i.e. gridToFieldMap = (/1,2,3,.../). The values of all gridToFieldMap entries
      must be greater than or equal to one and smaller than or equal to the field
      rank. It is erroneous to specify the same gridToFieldMap entry multiple times.
      The total ungridded dimensions in the field  are the total field dimensions
      less the dimensions in the grid. Ungridded dimensions must be in the same order
      they are stored in the field. If the Field dimCount is less than the Mesh
      dimCount then the default gridToFieldMap will contain zeros for the rightmost
      entries. A zero entry in the gridToFieldMap indicates that the particular Mesh
      dimension will be replicating the Field across the DEs along this direction.
    \item[ungriddedLBound]
      Lower bounds of the ungridded dimensions of the field. The number of elements
      in the ungriddedLBound is equal to the number of ungridded dimensions in the
      field. All ungridded dimensions of the field are also undistributed. When field
      dimension count is greater than grid dimension count, both ungriddedLBound and
      ungriddedUBound must be specified. When both are specified the values are
      checked for consistency. Note that the the ordering of these ungridded
      dimensions is the same as their order in the field.  
    \item[ungriddedUBound]
      Upper bounds of the ungridded dimensions of the field. The number of elements
      in the ungriddedUBound is equal to the number of ungridded dimensions in the
      field. All ungridded dimensions of the field are also undistributed. When field
      dimension count is greater than grid dimension count, both ungriddedLBound and
      ungriddedUBound must be specified. When both are specified the values are
      checked for consistency. Note that the the ordering of these ungridded
      dimensions is the same as their order in the field.  
    \item[{[name]}]
      The name for the newly created field.  If not specified, i.e. NULL,
      a default unique name will be generated: "FieldNNN" where NNN
      is a unique sequence number from 001 to 999.
    \item[{[rc]}]
      Return code; equals {\tt ESMF\_SUCCESS} if there are no errors.
    \end{description}
   
%/////////////////////////////////////////////////////////////
 
\mbox{}\hrulefill\ 
 
\subsubsection [ESMC\_FieldCreateMeshTypeKind] {ESMC\_FieldCreateMeshTypeKind - Create a Field from Mesh and typekind}


  
\bigskip{\sf INTERFACE:}
\begin{verbatim} ESMC_Field ESMC_FieldCreateMeshTypeKind(
   ESMC_Mesh mesh,                           // in
   enum ESMC_TypeKind_Flag typekind,         // in
   enum ESMC_MeshLoc_Flag meshloc,           // in
   ESMC_InterArrayInt *gridToFieldMap,       // in
   ESMC_InterArrayInt *ungriddedLBound,      // in
   ESMC_InterArrayInt *ungriddedUBound,      // in
   const char *name,                         // in
   int *rc                                   // out
 );
 \end{verbatim}{\em RETURN VALUE:}
\begin{verbatim}    Newly created ESMC_Field object.\end{verbatim}
{\sf DESCRIPTION:\\ }


  
    Creates a {\tt ESMC\_Field} object.
  
    The arguments are:
    \begin{description}
    \item[mesh]
      A {\tt ESMC\_Mesh} object.
    \item[typekind]
      The ESMC\_TypeKind\_Flag that describes this Field data.
    \item[meshloc]
      The ESMC\_MeshLoc\_Flag that describes this Field data.
    \item[gridToFieldMap]
      List with number of elements equal to the grid's dimCount. The list
      elements map each dimension of the grid to a dimension in the field by
      specifying the appropriate field dimension index. The default is to map all of
      the grid's dimensions against the lowest dimensions of the field in sequence,
      i.e. gridToFieldMap = (/1,2,3,.../). The values of all gridToFieldMap entries
      must be greater than or equal to one and smaller than or equal to the field
      rank. It is erroneous to specify the same gridToFieldMap entry multiple times.
      The total ungridded dimensions in the field  are the total field dimensions
      less the dimensions in the grid. Ungridded dimensions must be in the same order
      they are stored in the field. If the Field dimCount is less than the Mesh
      dimCount then the default gridToFieldMap will contain zeros for the rightmost
      entries. A zero entry in the gridToFieldMap indicates that the particular Mesh
      dimension will be replicating the Field across the DEs along this direction.
    \item[ungriddedLBound]
      Lower bounds of the ungridded dimensions of the field. The number of elements
      in the ungriddedLBound is equal to the number of ungridded dimensions in the
      field. All ungridded dimensions of the field are also undistributed. When field
      dimension count is greater than grid dimension count, both ungriddedLBound and
      ungriddedUBound must be specified. When both are specified the values are
      checked for consistency. Note that the the ordering of these ungridded
      dimensions is the same as their order in the field.  
    \item[ungriddedUBound]
      Upper bounds of the ungridded dimensions of the field. The number of elements
      in the ungriddedUBound is equal to the number of ungridded dimensions in the
      field. All ungridded dimensions of the field are also undistributed. When field
      dimension count is greater than grid dimension count, both ungriddedLBound and
      ungriddedUBound must be specified. When both are specified the values are
      checked for consistency. Note that the the ordering of these ungridded
      dimensions is the same as their order in the field.  
    \item[{[name]}]
      The name for the newly created field.  If not specified, i.e. NULL,
      a default unique name will be generated: "FieldNNN" where NNN
      is a unique sequence number from 001 to 999.
    \item[{[rc]}]
      Return code; equals {\tt ESMF\_SUCCESS} if there are no errors.
    \end{description}
   
%/////////////////////////////////////////////////////////////
 
\mbox{}\hrulefill\ 
 
\subsubsection [ESMC\_FieldCreateLocStreamArraySpec] {ESMC\_FieldCreateLocStreamArraySpec - Create a Field from LocStream and ArraySpec}


  
\bigskip{\sf INTERFACE:}
\begin{verbatim} ESMC_Field ESMC_FieldCreateLocStreamArraySpec(
   ESMC_LocStream locstream,                 // in
   ESMC_ArraySpec arrayspec,                 // in
   ESMC_InterArrayInt *gridToFieldMap,       // in
   ESMC_InterArrayInt *ungriddedLBound,      // in
   ESMC_InterArrayInt *ungriddedUBound,      // in
   const char *name,                         // in
   int *rc                                   // out
 );
 \end{verbatim}{\em RETURN VALUE:}
\begin{verbatim}    Newly created ESMC_Field object.\end{verbatim}
{\sf DESCRIPTION:\\ }


  
    Creates a {\tt ESMC\_Field} object.
  
    The arguments are:
    \begin{description}
    \item[locstream]
      A {\tt ESMC\_LocStream} object.
    \item[arrayspec]
      A {\tt ESMC\_ArraySpec} object describing data type and kind specification.
    \item[gridToFieldMap]
      List with number of elements equal to the grid's dimCount. The list
      elements map each dimension of the grid to a dimension in the field by
      specifying the appropriate field dimension index. The default is to map all of
      the grid's dimensions against the lowest dimensions of the field in sequence,
      i.e. gridToFieldMap = (/1,2,3,.../). The values of all gridToFieldMap entries
      must be greater than or equal to one and smaller than or equal to the field
      rank. It is erroneous to specify the same gridToFieldMap entry multiple times.
      The total ungridded dimensions in the field  are the total field dimensions
      less the dimensions in the grid. Ungridded dimensions must be in the same order
      they are stored in the field. If the Field dimCount is less than the Mesh
      dimCount then the default gridToFieldMap will contain zeros for the rightmost
      entries. A zero entry in the gridToFieldMap indicates that the particular Mesh
      dimension will be replicating the Field across the DEs along this direction.
    \item[ungriddedLBound]
      Lower bounds of the ungridded dimensions of the field. The number of elements
      in the ungriddedLBound is equal to the number of ungridded dimensions in the
      field. All ungridded dimensions of the field are also undistributed. When field
      dimension count is greater than grid dimension count, both ungriddedLBound and
      ungriddedUBound must be specified. When both are specified the values are
      checked for consistency. Note that the the ordering of these ungridded
      dimensions is the same as their order in the field.  
    \item[ungriddedUBound]
      Upper bounds of the ungridded dimensions of the field. The number of elements
      in the ungriddedUBound is equal to the number of ungridded dimensions in the
      field. All ungridded dimensions of the field are also undistributed. When field
      dimension count is greater than grid dimension count, both ungriddedLBound and
      ungriddedUBound must be specified. When both are specified the values are
      checked for consistency. Note that the the ordering of these ungridded
      dimensions is the same as their order in the field.  
    \item[{[name]}]
      The name for the newly created field.  If not specified, i.e. NULL,
      a default unique name will be generated: "FieldNNN" where NNN
      is a unique sequence number from 001 to 999.
    \item[{[rc]}]
      Return code; equals {\tt ESMF\_SUCCESS} if there are no errors.
    \end{description}
   
%/////////////////////////////////////////////////////////////
 
\mbox{}\hrulefill\ 
 
\subsubsection [ESMC\_FieldCreateLocStreamTypeKind] {ESMC\_FieldCreateLocStreamTypeKind - Create a Field from LocStream and typekind}


  
\bigskip{\sf INTERFACE:}
\begin{verbatim} ESMC_Field ESMC_FieldCreateLocStreamTypeKind(
   ESMC_LocStream locstream,                 // in
   enum ESMC_TypeKind_Flag typekind,         // in
   ESMC_InterArrayInt *gridToFieldMap,       // in
   ESMC_InterArrayInt *ungriddedLBound,      // in
   ESMC_InterArrayInt *ungriddedUBound,      // in
   const char *name,                         // in
   int *rc                                   // out
 );
 \end{verbatim}{\em RETURN VALUE:}
\begin{verbatim}    Newly created ESMC_Field object.\end{verbatim}
{\sf DESCRIPTION:\\ }


  
    Creates a {\tt ESMC\_Field} object.
  
    The arguments are:
    \begin{description}
    \item[locstream]
      A {\tt ESMC\_LocStream} object.
    \item[typekind]
      The ESMC\_TypeKind\_Flag that describes this Field data.
    \item[gridToFieldMap]
      List with number of elements equal to the grid's dimCount. The list
      elements map each dimension of the grid to a dimension in the field by
      specifying the appropriate field dimension index. The default is to map all of
      the grid's dimensions against the lowest dimensions of the field in sequence,
      i.e. gridToFieldMap = (/1,2,3,.../). The values of all gridToFieldMap entries
      must be greater than or equal to one and smaller than or equal to the field
      rank. It is erroneous to specify the same gridToFieldMap entry multiple times.
      The total ungridded dimensions in the field  are the total field dimensions
      less the dimensions in the grid. Ungridded dimensions must be in the same order
      they are stored in the field. If the Field dimCount is less than the Mesh
      dimCount then the default gridToFieldMap will contain zeros for the rightmost
      entries. A zero entry in the gridToFieldMap indicates that the particular Mesh
      dimension will be replicating the Field across the DEs along this direction.
    \item[ungriddedLBound]
      Lower bounds of the ungridded dimensions of the field. The number of elements
      in the ungriddedLBound is equal to the number of ungridded dimensions in the
      field. All ungridded dimensions of the field are also undistributed. When field
      dimension count is greater than grid dimension count, both ungriddedLBound and
      ungriddedUBound must be specified. When both are specified the values are
      checked for consistency. Note that the the ordering of these ungridded
      dimensions is the same as their order in the field.  
    \item[ungriddedUBound]
      Upper bounds of the ungridded dimensions of the field. The number of elements
      in the ungriddedUBound is equal to the number of ungridded dimensions in the
      field. All ungridded dimensions of the field are also undistributed. When field
      dimension count is greater than grid dimension count, both ungriddedLBound and
      ungriddedUBound must be specified. When both are specified the values are
      checked for consistency. Note that the the ordering of these ungridded
      dimensions is the same as their order in the field.  
    \item[{[name]}]
      The name for the newly created field.  If not specified, i.e. NULL,
      a default unique name will be generated: "FieldNNN" where NNN
      is a unique sequence number from 001 to 999.
    \item[{[rc]}]
      Return code; equals {\tt ESMF\_SUCCESS} if there are no errors.
    \end{description}
   
%/////////////////////////////////////////////////////////////
 
\mbox{}\hrulefill\ 
 
\subsubsection [ESMC\_FieldDestroy] {ESMC\_FieldDestroy - Destroy a Field}


  
\bigskip{\sf INTERFACE:}
\begin{verbatim} int ESMC_FieldDestroy(
   ESMC_Field *field     // inout
 );
 \end{verbatim}{\em RETURN VALUE:}
\begin{verbatim}    Return code; equals ESMF_SUCCESS if there are no errors.\end{verbatim}
{\sf DESCRIPTION:\\ }


  
    Releases all resources associated with this {\tt ESMC\_Field}.
      Return code; equals {\tt ESMF\_SUCCESS} if there are no errors.
  
    The arguments are:
    \begin{description}
    \item[field]
      Destroy contents of this {\tt ESMC\_Field}.
    \end{description}
   
%/////////////////////////////////////////////////////////////
 
\mbox{}\hrulefill\ 
 
\subsubsection [ESMC\_FieldGetArray] {ESMC\_FieldGetArray - Get the internal Array stored in the Field}


  
\bigskip{\sf INTERFACE:}
\begin{verbatim} ESMC_Array ESMC_FieldGetArray(
   ESMC_Field field,     // in
   int *rc               // out
 );
 \end{verbatim}{\em RETURN VALUE:}
\begin{verbatim}    The ESMC_Array object stored in the ESMC_Field.\end{verbatim}
{\sf DESCRIPTION:\\ }


  
    Get the internal Array stored in the {\tt ESMC\_Field}.
  
    The arguments are:
    \begin{description}
    \item[field]
      Get the internal Array stored in this {\tt ESMC\_Field}.
    \item[{[rc]}]
      Return code; equals {\tt ESMF\_SUCCESS} if there are no errors.
    \end{description}
   
%/////////////////////////////////////////////////////////////
 
\mbox{}\hrulefill\ 
 
\subsubsection [ESMC\_FieldGetMesh] {ESMC\_FieldGetMesh - Get the internal Mesh stored in the Field}


  
\bigskip{\sf INTERFACE:}
\begin{verbatim} ESMC_Mesh ESMC_FieldGetMesh(
   ESMC_Field field,     // in
   int *rc               // out
 );
 \end{verbatim}{\em RETURN VALUE:}
\begin{verbatim}    The ESMC_Mesh object stored in the ESMC_Field.\end{verbatim}
{\sf DESCRIPTION:\\ }


  
    Get the internal Mesh stored in the {\tt ESMC\_Field}.
  
    The arguments are:
    \begin{description}
    \item[field]
      Get the internal Mesh stored in this {\tt ESMC\_Field}.
    \item[{[rc]}]
      Return code; equals {\tt ESMF\_SUCCESS} if there are no errors.
    \end{description}
   
%/////////////////////////////////////////////////////////////
 
\mbox{}\hrulefill\ 
 
\subsubsection [ESMC\_FieldGetPtr] {ESMC\_FieldGetPtr - Get the internal Fortran data pointer stored in the Field}


  
\bigskip{\sf INTERFACE:}
\begin{verbatim} void *ESMC_FieldGetPtr(
   ESMC_Field field,     // in
   int localDe,          // in
   int *rc               // out
 );
 \end{verbatim}{\em RETURN VALUE:}
\begin{verbatim}    The Fortran data pointer stored in the ESMC_Field.\end{verbatim}
{\sf DESCRIPTION:\\ }


  
    Get the internal Fortran data pointer stored in the {\tt ESMC\_Field}.
  
    The arguments are:
    \begin{description}
    \item[field]
      Get the internal Fortran data pointer stored in this {\tt ESMC\_Field}.
    \item[localDe]
      Local DE for which information is requested. {\tt [0,..,localDeCount-1]}. 
    \item[{[rc]}]
      Return code; equals {\tt ESMF\_SUCCESS} if there are no errors.
    \end{description}
   
%/////////////////////////////////////////////////////////////
 
\mbox{}\hrulefill\ 
 
\subsubsection [ESMC\_FieldGetBounds] {ESMC\_FieldGetBounds - Get the Field bounds}


  
\bigskip{\sf INTERFACE:}
\begin{verbatim} int ESMC_FieldGetBounds(
   ESMC_Field field,      // in
   int *localDe,
   int *exclusiveLBound,
   int *exclusiveUBound,
   int rank
 );
 \end{verbatim}{\em RETURN VALUE:}
\begin{verbatim}    Return code; equals ESMF_SUCCESS if there are no errors.\end{verbatim}
{\sf DESCRIPTION:\\ }


  
    Get the Field bounds from the {\tt ESMC\_Field}.
  
    The arguments are:
    \begin{description}
    \item[field]
      {\tt ESMC\_Field} whose bounds will be returned
    \item[localDe]
      The local DE of the {\tt ESMC\_Field} (not implemented)
    \item[exclusiveLBound]
      The exclusive lower bounds of the {\tt ESMC\_Field}
    \item[exclusiveUBound]
      The exclusive upper bounds of the {\tt ESMC\_Field}
    \item[rank]
      The rank of the {\tt ESMC\_Field}, to size the bounds arrays
    \end{description}
   
%/////////////////////////////////////////////////////////////
 
\mbox{}\hrulefill\ 
 
\subsubsection [ESMC\_FieldPrint] {ESMC\_FieldPrint - Print the internal information of a Field}


  
\bigskip{\sf INTERFACE:}
\begin{verbatim} int ESMC_FieldPrint(
   ESMC_Field field      // in
 );
 \end{verbatim}{\em RETURN VALUE:}
\begin{verbatim}    Return code; equals ESMF_SUCCESS if there are no errors.\end{verbatim}
{\sf DESCRIPTION:\\ }


  
    Print the internal information within this {\tt ESMC\_Field}.
  
    The arguments are:
    \begin{description}
    \item[field]
      Print contents of this {\tt ESMC\_Field}.
    \end{description}
   
%/////////////////////////////////////////////////////////////
 
\mbox{}\hrulefill\ 
 
\subsubsection [ESMC\_FieldRegridGetArea] {ESMC\_FieldRegridGetArea - Get the area of the cells used for }


                                        conservative interpolation
  
\bigskip{\sf INTERFACE:}
\begin{verbatim} int ESMC_FieldRegridGetArea(
   ESMC_Field field      // in
 );
 \end{verbatim}{\em RETURN VALUE:}
\begin{verbatim}    Return code; equals ESMF_SUCCESS if there are no errors.\end{verbatim}
{\sf DESCRIPTION:\\ }


       This subroutine gets the area of the cells used for conservative interpolation for the grid object 
       associated with {\tt areaField} and puts them into {\tt areaField}. If created on a 2D Grid, it must 
       be built on the {\tt ESMF\_STAGGERLOC\_CENTER} stagger location. 
       If created on a 3D Grid, it must be built on the {\tt ESMF\_STAGGERLOC\_CENTER\_VCENTER} stagger 
       location. If created on a Mesh, it must be built on the {\tt ESMF\_MESHLOC\_ELEMENT} mesh location. 
  
       The arguments are:
       \begin{description}
       \item [areaField]
             The Field to put the area values in. 
       \end{description}
   
%/////////////////////////////////////////////////////////////
 
\mbox{}\hrulefill\ 
 
\subsubsection [ESMC\_FieldRegridStore] {ESMC\_FieldRegridStore - Precompute a Field regridding operation and return a RouteHandle}


  
\bigskip{\sf INTERFACE:}
\begin{verbatim} int ESMC_FieldRegridStore( 
     ESMC_Field srcField,                           // in
     ESMC_Field dstField,                           // in
     ESMC_InterArrayInt *srcMaskValues,             // in
     ESMC_InterArrayInt *dstMaskValues,             // in
     ESMC_RouteHandle *routehandle,                 // inout
     enum ESMC_RegridMethod_Flag *regridmethod,     // in
     enum ESMC_PoleMethod_Flag *polemethod,         // in
     int *regridPoleNPnts,                          // in
     enum ESMC_LineType_Flag *lineType,             // in
     enum ESMC_NormType_Flag *normType,             // in
     enum ESMC_ExtrapMethod_Flag *extrapMethod,     // in
     int *extrapNumSrcPnts,                         // in
     float *extrapDistExponent,                     // in
     enum ESMC_UnmappedAction_Flag *unmappedaction, // in
     enum ESMC_Logical *ignoreDegenerate,           // in
     double **factorList,                           // inout
     int **factorIndexList,                         // inout
     int *numFactors,                               // inout
     ESMC_Field *srcFracField,                      // inout
     ESMC_Field *dstFracField);                     // inout
 \end{verbatim}{\em RETURN VALUE:}
\begin{verbatim}     Return code; equals ESMF_SUCCESS if there are no errors.\end{verbatim}
{\sf DESCRIPTION:\\ }


  
     Creates a sparse matrix operation (stored in routehandle) that contains 
     the calculations and communications necessary to interpolate from srcField 
     to dstField. The routehandle can then be used in the call ESMC\_FieldRegrid() 
     to interpolate between the Fields. 
  
    The arguments are:
    \begin{description}
    \item[srcField]
      ESMC\_Field with source data.
    \item[dstField]
      ESMC\_Field with destination data.
    \item[srcMaskValues]
      List of values that indicate a source point should be masked out. 
      If not specified, no masking will occur.
    \item[dstMaskValues]
      List of values that indicate a destination point should be masked out. 
      If not specified, no masking will occur.
    \item[routehandle]
      The handle that implements the regrid, to be used in {\tt ESMC\_FieldRegrid()}.
    \item[regridmethod]
      The type of interpolation. If not specified, defaults to {\tt ESMF\_REGRIDMETHOD\_BILINEAR}.
    \item [polemethod]
      Which type of artificial pole
      to construct on the source Grid for regridding. 
      If not specified, defaults to {\tt ESMF\_POLEMETHOD\_ALLAVG} for non-conservative regrid methods, 
      and {\tt ESMF\_POLEMETHOD\_NONE} for conservative methods. 
      If not specified, defaults to {\tt ESMC\_POLEMETHOD\_ALLAVG}. 
    \item [regridPoleNPnts]
      If {\tt polemethod} is {\tt ESMC\_POLEMETHOD\_NPNTAVG}.
      This parameter indicates how many points should be averaged
      over. Must be specified if {\tt polemethod} is 
      {\tt ESMC\_POLEMETHOD\_NPNTAVG}.
    \item [{[lineType]}]
      This argument controls the path of the line which connects two points on a sphere surface. This in
      turn controls the path along which distances are calculated and the shape of the edges that make
      up a cell. Both of these quantities can influence how interpolation weights are calculated.
      As would be expected, this argument is only applicable when {\tt srcField} and {\tt dstField} are
      built on grids which lie on the surface of a sphere. Section~\ref{opt:lineType} shows a
      list of valid options for this argument. If not specified, the default depends on the
      regrid method. Section~\ref{opt:lineType} has the defaults by line type.
    \item[normType]
      This argument controls the type of normalization used when generating conservative weights.
      This option only applies to weights generated with {\tt regridmethod=ESMF\_REGRIDMETHOD\_CONSERVE}.
      If not specified normType defaults to {\tt ESMF\_NORMTYPE\_DSTAREA}.
    \item [{[extrapMethod]}]
      The type of extrapolation. Please see Section~\ref{opt:cextrapmethod} 
      for a list of valid options. If not specified, defaults to 
      {\tt ESMC\_EXTRAPMETHOD\_NONE}.
    \item [{[extrapNumSrcPnts]}] 
      The number of source points to use for the extrapolation methods that use more than one source point 
      (e.g. {\tt ESMC\_EXTRAPMETHOD\_NEAREST\_IDAVG}). If not specified, defaults to 8.
    \item [{[extrapDistExponent]}] 
      The exponent to raise the distance to when calculating weights for 
      the {\tt ESMC\_EXTRAPMETHOD\_NEAREST\_IDAVG} extrapolation method. A higher value reduces the influence 
      of more distant points. If not specified, defaults to 2.0.
    \item[unmappedaction]
      Specifies what should happen if there are destination points that can't 
      be mapped to a source cell. Options are {\tt ESMF\_UNMAPPEDACTION\_ERROR} or
      {\tt ESMF\_UNMAPPEDACTION\_IGNORE}. If not specified, defaults to {\tt ESMF\_UNMAPPEDACTION\_ERROR}.
    \item [{[factorList]}] 
      The list of coefficients for a sparse matrix which interpolates from {\tt srcField} to 
      {\tt dstField}. The array coming out of this variable is in the appropriate format to be used
      in other ESMF sparse matrix multiply calls, for example {\tt ESMC\_FieldSMMStore()}. 
      The {\tt factorList} array is allocated by the method and the user is responsible for 
      deallocating it. 
    \item [{[factorIndexList]}] 
      The indices for a sparse matrix which interpolates from {\tt srcField} to 
      {\tt dstField}. This argument is a 2D array containing pairs of source and destination
      sequence indices corresponding to the coefficients in the {\tt factorList} argument. 
      The first dimension of {\tt factorIndexList} is of size 2. {\tt factorIndexList(1,:)} specifies 
      the sequence index of the source element in the {\tt srcField}. {\tt factorIndexList(2,:)} specifies 
      the sequence index of the destination element in the {\tt dstField}. The second dimension of 
      {\tt factorIndexList} steps through the list of pairs, i.e. {\tt size(factorIndexList,2)==size(factorList)}.
      The array coming out of this variable is in the appropriate format to be used
      in other ESMF sparse matrix multiply calls, for example {\tt ESMC\_FieldSMMStore()}. 
      The {\tt factorIndexList} array is allocated by the method and the user is responsible for deallocating it. 
    \item [{[numFactors]}] 
      The number of factors returned in {\tt factorList}.
    \item [{[srcFracField]}] 
      The fraction of each source cell participating in the regridding. Only 
      valid when regridmethod is {\tt ESMC\_REGRIDMETHOD\_CONSERVE}.
      This Field needs to be created on the same location (e.g staggerloc) 
      as the srcField.
    \item [{[dstFracField]}] 
      The fraction of each destination cell participating in the regridding. Only 
      valid when regridmethod is {\tt ESMF\_REGRIDMETHOD\_CONSERVE}.
      This Field needs to be created on the same location (e.g staggerloc) 
      as the dstField.
    \end{description}
   
%/////////////////////////////////////////////////////////////
 
\mbox{}\hrulefill\ 
 
\subsubsection [ESMC\_FieldRegridStoreFile] {ESMC\_FieldRegridStoreFile - Precompute a Field regridding operation and return a RouteHandle}


  
\bigskip{\sf INTERFACE:}
\begin{verbatim} int ESMC_FieldRegridStoreFile(
     ESMC_Field srcField,                           // in
     ESMC_Field dstField,                           // in
     const char *filename,                          // in
     ESMC_InterArrayInt *srcMaskValues,             // in
     ESMC_InterArrayInt *dstMaskValues,             // in
     ESMC_RouteHandle *routehandle,                 // inout
     enum ESMC_RegridMethod_Flag *regridmethod,     // in
     enum ESMC_PoleMethod_Flag *polemethod,         // in
     int *regridPoleNPnts,                          // in
     enum ESMC_LineType_Flag *lineType,             // in
     enum ESMC_NormType_Flag *normType,             // in
     enum ESMC_UnmappedAction_Flag *unmappedaction, // in
     enum ESMC_Logical *ignoreDegenerate,           // in
     enum ESMC_Logical *create_rh,                  // in
     ESMC_FileMode_Flag *filemode,                  // in
     const char *srcFile,                           // in
     const char *dstFile,                           // in
     enum ESMC_FileFormat_Flag *srcFileType,        // in
     enum ESMC_FileFormat_Flag *dstFileType,        // in
     ESMC_Field *srcFracField,                      // out
     ESMC_Field *dstFracField);                     // out
 \end{verbatim}{\em RETURN VALUE:}
\begin{verbatim}     Return code; equals ESMF_SUCCESS if there are no errors.\end{verbatim}
{\sf DESCRIPTION:\\ }


  
     Creates a sparse matrix operation (stored in routehandle) that contains
     the calculations and communications necessary to interpolate from srcField
     to dstField. The routehandle can then be used in the call ESMC\_FieldRegrid()
     to interpolate between the Fields. The weights will be output to the file
     with name {\tt filename}.
  
    The arguments are:
    \begin{description}
    \item[srcField]
      ESMC\_Field with source data.
    \item[dstField]
      ESMC\_Field with destination data.
    \item[{[filename]}]
      The output filename for the factorList and factorIndexList.
    \item[{[srcMaskValues]}]
      List of values that indicate a source point should be masked out.
      If not specified, no masking will occur.
    \item[{[dstMaskValues]}]
      List of values that indicate a destination point should be masked out.
      If not specified, no masking will occur.
    \item[{[routehandle]}]
      The handle that implements the regrid, to be used in {\tt ESMC\_FieldRegrid()}.
    \item[{[regridmethod]}]
      The type of interpolation. If not specified, defaults to {\tt ESMC\_REGRIDMETHOD\_BILINEAR}.
    \item [{[polemethod]}]
      Which type of artificial pole
      to construct on the source Grid for regridding.
      If not specified, defaults to {\tt ESMC\_POLEMETHOD\_ALLAVG} for non-conservative regrid methods,
      and {\tt ESMC\_POLEMETHOD\_NONE} for conservative methods.
      If not specified, defaults to {\tt ESMC\_POLEMETHOD\_ALLAVG}.
    \item [{[regridPoleNPnts]}]
      If {\tt polemethod} is {\tt ESMC\_POLEMETHOD\_NPNTAVG}.
      This parameter indicates how many points should be averaged
      over. Must be specified if {\tt polemethod} is
      {\tt ESMC\_POLEMETHOD\_NPNTAVG}.
    \item [{[lineType]}]
      This argument controls the path of the line which connects two points on a sphere surface. This in
      turn controls the path along which distances are calculated and the shape of the edges that make
      up a cell. Both of these quantities can influence how interpolation weights are calculated.
      As would be expected, this argument is only applicable when {\tt srcField} and {\tt dstField} are
      built on grids which lie on the surface of a sphere. Section~\ref{opt:lineType} shows a
      list of valid options for this argument. If not specified, the default depends on the
      regrid method. Section~\ref{opt:lineType} has the defaults by line type.
    \item[{[normType]}]
      This argument controls the type of normalization used when generating conservative weights.
      This option only applies to weights generated with {\tt regridmethod=ESMC\_REGRIDMETHOD\_CONSERVE}.
      If not specified normType defaults to {\tt ESMC\_NORMTYPE\_DSTAREA}.
    \item[{[unmappedaction]}]
      Specifies what should happen if there are destination points that can't
      be mapped to a source cell. Options are {\tt ESMC\_UNMAPPEDACTION\_ERROR} or
      {\tt ESMC\_UNMAPPEDACTION\_IGNORE}. If not specified, defaults to {\tt ESMC\_UNMAPPEDACTION\_ERROR}.
    \item[{create\_rh}]
      Specifies whether or not to create a routehandle, or just write weights to file.
      If not specified, defaults to {\tt ESMF\_TRUE}.
    \item[{filemode}]
      Specifies the mode to use when creating the weight file. Options are
      {\tt ESMC\_FILEMODE\_BASIC} and {\tt ESMC\_FILEMODE\_WITHAUX}, which will
      write a file that includes center coordinates of the grids. The default 
      value is {\tt ESMC\_FILEMODE\_BASIC}.
    \item[{srcFile}]
      The name of the source file used to create the {\tt ESMC\_Grid} used
      in this regridding operation.
    \item[{dstFile}]
      The name of the destination file used to create the {\tt ESMC\_Grid} used
      in this regridding operation.
    \item[{srcFileType}]
      The type of the file used to represent the source grid.
    \item[{dstFileType}]
      The type of the file used to represent the destination grid.
    \item [{[srcFracField]}]
      The fraction of each source cell participating in the regridding. Only
      valid when regridmethod is {\tt ESMC\_REGRIDMETHOD\_CONSERVE}.
      This Field needs to be created on the same location (e.g staggerloc)
      as the srcField.
    \item [{[dstFracField]}]
      The fraction of each destination cell participating in the regridding. Only
      valid when regridmethod is {\tt ESMC\_REGRIDMETHOD\_CONSERVE}.
      This Field needs to be created on the same location (e.g staggerloc)
      as the dstField.
    \end{description}
   
%/////////////////////////////////////////////////////////////
 
\mbox{}\hrulefill\ 
 
\subsubsection [ESMC\_FieldRegrid] {ESMC\_FieldRegrid - Compute a regridding operation}


  
\bigskip{\sf INTERFACE:}
\begin{verbatim}   int ESMC_FieldRegrid( 
     ESMC_Field srcField,                  // in
     ESMC_Field dstField,                  // inout
     ESMC_RouteHandle routehandle,         // in
     enum ESMC_Region_Flag *zeroregion);   // in
 \end{verbatim}{\em RETURN VALUE:}
\begin{verbatim}    Return code; equals ESMF_SUCCESS if there are no errors.\end{verbatim}
{\sf DESCRIPTION:\\ }


  
    Execute the precomputed regrid operation stored in routehandle to interpolate 
    from srcField to dstField. See ESMF\_FieldRegridStore() on how to precompute
    the routehandle.  It is erroneous to specify the identical Field object for
    srcField and dstField arguments.  This call is collective across the 
    current VM.
  
    The arguments are:
    \begin{description}
    \item[srcField]
      ESMC\_Field with source data.
    \item[dstField]
      ESMC\_Field with destination data.
    \item[routehandle]
      Handle to the precomputed Route.
    \item [{[zeroregion]}]
      \begin{sloppypar}
      If set to {\tt ESMC\_REGION\_TOTAL} {\em (default)} the total regions of
      all DEs in {\tt dstField} will be initialized to zero before updating the 
      elements with the results of the sparse matrix multiplication. If set to
      {\tt ESMC\_REGION\_EMPTY} the elements in {\tt dstField} will not be
      modified prior to the sparse matrix multiplication and results will be
      added to the incoming element values. Setting {\tt zeroregion} to 
      {\tt ESMC\_REGION\_SELECT} will only zero out those elements in the 
      destination Array that will be updated by the sparse matrix
      multiplication.
      \end{sloppypar}
    \end{description}
   
%/////////////////////////////////////////////////////////////
 
\mbox{}\hrulefill\ 
 
\subsubsection [ESMC\_FieldRegridRelease] {ESMC\_FieldRegridRelease - Free resources used by a regridding operation}


  
\bigskip{\sf INTERFACE:}
\begin{verbatim}   int ESMC_FieldRegridRelease(ESMC_RouteHandle *routehandle);  // inout
 \end{verbatim}{\em RETURN VALUE:}
\begin{verbatim}    Return code; equals ESMF_SUCCESS if there are no errors.\end{verbatim}
{\sf DESCRIPTION:\\ }


  
    Free resources used by regrid object
  
    The arguments are:
    \begin{description}
    \item[routehandle]
      Handle carrying the sparse matrix
    \end{description}
   
%/////////////////////////////////////////////////////////////
 
\mbox{}\hrulefill\ 
 
\subsubsection [ESMC\_FieldSMMStore] {ESMC\_FieldSMMStore - Precompute a Field regridding operation and return a RouteHandle}


  
\bigskip{\sf INTERFACE:}
\begin{verbatim} int ESMC_FieldSMMStore(
     ESMC_Field srcField,                           // in
     ESMC_Field dstField,                           // in
     const char *filename,                          // in
     ESMC_RouteHandle *routehandle,                 // out
     enum ESMC_Logical *ignoreUnmatchedIndices,     // in
     int *srcTermProcessing,                        // in
     int *pipeLineDepth);                           // in
 \end{verbatim}{\em RETURN VALUE:}
\begin{verbatim}     Return code; equals ESMF_SUCCESS if there are no errors.\end{verbatim}
{\sf DESCRIPTION:\\ }


  
     Creates a sparse matrix operation (stored in routehandle) that contains
     the calculations and communications necessary to interpolate from srcField
     to dstField. The routehandle can then be used in the call ESMC\_FieldRegrid()
     to interpolate between the Fields.
  
    The arguments are:
    \begin{description}
    \item[srcField]
      ESMC\_Field with source data.
    \item[dstField]
      ESMC\_Field with destination data.
    \item [filename]
      Path to the file containing weights for creating an {\tt ESMC\_RouteHandle}.
      Only "row", "col", and "S" variables are required. They
      must be one-dimensionsal with dimension "n\_s".
    \item[routehandle]
      The handle that implements the regrid, to be used in {\tt ESMC\_FieldRegrid()}.
    \item [{[ignoreUnmatchedIndices]}]
      A logical flag that affects the behavior for when sequence indices
      in the sparse matrix are encountered that do not have a match on the
      {\tt srcField} or {\tt dstField} side. The default setting is
      {\tt .false.}, indicating that it is an error when such a situation is
      encountered. Setting {\tt ignoreUnmatchedIndices} to {\tt .true.} ignores
      entries with unmatched indices.
    \item [{[srcTermProcessing]}]
      The {\tt srcTermProcessing} parameter controls how many source terms,
      located on the same PET and summing into the same destination element,
      are summed into partial sums on the source PET before being transferred
      to the destination PET. A value of 0 indicates that the entire arithmetic
      is done on the destination PET; source elements are neither multiplied
      by their factors nor added into partial sums before being sent off by the
      source PET. A value of 1 indicates that source elements are multiplied
      by their factors on the source side before being sent to the destination
      PET. Larger values of {\tt srcTermProcessing} indicate the maximum number
      of terms in the partial sums on the source side.
      Note that partial sums may lead to bit-for-bit differences in the results.
      See section \ref{RH:bfb} for an in-depth discussion of {\em all}
      bit-for-bit reproducibility aspects related to route-based communication
      methods.
      The {\tt ESMC\_FieldSMMStore()} method implements an auto-tuning scheme
      for the {\tt srcTermProcessing} parameter. The intent on the
      {\tt srcTermProcessing} argument is "{\tt inout}" in order to
      support both overriding and accessing the auto-tuning parameter.
      If an argument $>= 0$ is specified, it is used for the
      {\tt srcTermProcessing} parameter, and the auto-tuning phase is skipped.
      In this case the {\tt srcTermProcessing} argument is not modified on
      return. If the provided argument is $< 0$, the {\tt srcTermProcessing}
      parameter is determined internally using the auto-tuning scheme. In this
      case the {\tt srcTermProcessing} argument is re-set to the internally
      determined value on return. Auto-tuning is also used if the optional
      {\tt srcTermProcessing} argument is omitted.
    \item [{[pipelineDepth]}]
      The {\tt pipelineDepth} parameter controls how many messages a PET
      may have outstanding during a sparse matrix exchange. Larger values
      of {\tt pipelineDepth} typically lead to better performance. However,
      on some systems too large a value may lead to performance degradation,
      or runtime errors.
      Note that the pipeline depth has no effect on the bit-for-bit
      reproducibility of the results. However, it may affect the performance
      reproducibility of the exchange.
      The {\tt ESMC\_FieldSMMStore()} method implements an auto-tuning scheme
      for the {\tt pipelineDepth} parameter. The intent on the
      {\tt pipelineDepth} argument is "{\tt inout}" in order to
      support both overriding and accessing the auto-tuning parameter.
      If an argument $>= 0$ is specified, it is used for the
      {\tt pipelineDepth} parameter, and the auto-tuning phase is skipped.
      In this case the {\tt pipelineDepth} argument is not modified on
      return. If the provided argument is $< 0$, the {\tt pipelineDepth}
      parameter is determined internally using the auto-tuning scheme. In this
      case the {\tt pipelineDepth} argument is re-set to the internally
      determined value on return. Auto-tuning is also used if the optional
      {\tt pipelineDepth} argument is omitted.
    \end{description}
  
%...............................................................
\setlength{\parskip}{\oldparskip}
\setlength{\parindent}{\oldparindent}
\setlength{\baselineskip}{\oldbaselineskip}
