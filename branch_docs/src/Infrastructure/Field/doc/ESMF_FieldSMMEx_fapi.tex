%                **** IMPORTANT NOTICE *****
% This LaTeX file has been automatically produced by ProTeX v. 1.1
% Any changes made to this file will likely be lost next time
% this file is regenerated from its source. Send questions 
% to Arlindo da Silva, dasilva@gsfc.nasa.gov
 
\setlength{\oldparskip}{\parskip}
\setlength{\parskip}{1.5ex}
\setlength{\oldparindent}{\parindent}
\setlength{\parindent}{0pt}
\setlength{\oldbaselineskip}{\baselineskip}
\setlength{\baselineskip}{11pt}
 
%--------------------- SHORT-HAND MACROS ----------------------
\def\bv{\begin{verbatim}}
\def\ev{\end{verbatim}}
\def\be{\begin{equation}}
\def\ee{\end{equation}}
\def\bea{\begin{eqnarray}}
\def\eea{\end{eqnarray}}
\def\bi{\begin{itemize}}
\def\ei{\end{itemize}}
\def\bn{\begin{enumerate}}
\def\en{\end{enumerate}}
\def\bd{\begin{description}}
\def\ed{\end{description}}
\def\({\left (}
\def\){\right )}
\def\[{\left [}
\def\]{\right ]}
\def\<{\left  \langle}
\def\>{\right \rangle}
\def\cI{{\cal I}}
\def\diag{\mathop{\rm diag}}
\def\tr{\mathop{\rm tr}}
%-------------------------------------------------------------

\markboth{Left}{Source File: ESMF\_FieldSMMEx.F90,  Date: Tue May  5 21:00:02 MDT 2020
}

 
%/////////////////////////////////////////////////////////////

   \subsubsection{Sparse matrix multiplication from source Field to destination Field}
   \label{sec:field:usage:smm_1dptr}
  
   The {\tt ESMF\_FieldSMM()} interface can be used to perform sparse matrix multiplication
   from
   source Field to destination Field. This interface is overloaded by type and kind;
  
   In this example, we first create two 1D Fields, a source Field and a destination
   Field. Then we use {\tt ESMF\_FieldSMM} to
   perform sparse matrix multiplication from source Field to destination Field.
  
   The source and destination Field data are arranged such that each of the 4 PETs has 4
   data elements. Moreover, the source Field has all its data elements initialized to a linear
   function based on local PET number.
   Then collectively on each PET, a SMM according to the following formula
   is preformed: \newline
   $dstField(i) = i * srcField(i), i = 1 ... 4$ \newline
   \newline
  
   Because source Field data are initialized to a linear function based on local PET number,
   the formula predicts that
   the result destination Field data on each PET is {1,2,3,4}. This is verified in the
   example.
  
   Section \ref{Array:SparseMatMul} provides a detailed discussion of the
   sparse matrix multiplication operation implemented in ESMF.
   
%/////////////////////////////////////////////////////////////

 \begin{verbatim}

    ! Get current VM and pet number
    call ESMF_VMGetCurrent(vm, rc=rc)
    if (rc /= ESMF_SUCCESS) call ESMF_Finalize(endflag=ESMF_END_ABORT)

    call ESMF_VMGet(vm, localPet=lpe, rc=rc)
    if (rc /= ESMF_SUCCESS) call ESMF_Finalize(endflag=ESMF_END_ABORT)

    ! create distgrid and grid
    distgrid = ESMF_DistGridCreate(minIndex=(/1/), maxIndex=(/16/), &
        regDecomp=(/4/), &
        rc=rc)
    if (rc /= ESMF_SUCCESS) call ESMF_Finalize(endflag=ESMF_END_ABORT)

    grid = ESMF_GridCreate(distgrid=distgrid, &
        name="grid", rc=rc)
    if (rc /= ESMF_SUCCESS) call ESMF_Finalize(endflag=ESMF_END_ABORT)

    call ESMF_GridGetFieldBounds(grid, localDe=0, totalCount=fa_shape, rc=rc)
    if (rc /= ESMF_SUCCESS) call ESMF_Finalize(endflag=ESMF_END_ABORT)

    ! create src\_farray, srcArray, and srcField
    ! +--------+--------+--------+--------+
    !      1        2        3        4            ! value
    ! 1        4        8        12       16       ! bounds
    allocate(src_farray(fa_shape(1)) )
    src_farray = lpe+1
    srcArray = ESMF_ArrayCreate(distgrid, src_farray, &
                indexflag=ESMF_INDEX_DELOCAL, rc=rc)
    if (rc /= ESMF_SUCCESS) call ESMF_Finalize(endflag=ESMF_END_ABORT)

    srcField = ESMF_FieldCreate(grid, srcArray, rc=rc)
    if (rc /= ESMF_SUCCESS) call ESMF_Finalize(endflag=ESMF_END_ABORT)

    ! create dst_farray, dstArray, and dstField
    ! +--------+--------+--------+--------+
    !      0        0        0        0            ! value
    ! 1        4        8        12       16       ! bounds
    allocate(dst_farray(fa_shape(1)) )
    dst_farray = 0
    dstArray = ESMF_ArrayCreate(distgrid, dst_farray, &
                indexflag=ESMF_INDEX_DELOCAL, rc=rc)
    if (rc /= ESMF_SUCCESS) call ESMF_Finalize(endflag=ESMF_END_ABORT)

    dstField = ESMF_FieldCreate(grid, dstArray, rc=rc)
    if (rc /= ESMF_SUCCESS) call ESMF_Finalize(endflag=ESMF_END_ABORT)

    ! perform sparse matrix multiplication
    ! 1. setup routehandle from source Field to destination Field
    ! initialize factorList and factorIndexList
    allocate(factorList(4))
    allocate(factorIndexList(2,4))
    factorList = (/1,2,3,4/)
    factorIndexList(1,:) = (/lpe*4+1,lpe*4+2,lpe*4+3,lpe*4+4/)
    factorIndexList(2,:) = (/lpe*4+1,lpe*4+2,lpe*4+3,lpe*4+4/)

    call ESMF_FieldSMMStore(srcField, dstField, routehandle, &
        factorList, factorIndexList, rc=localrc)
    if (localrc /= ESMF_SUCCESS) call ESMF_Finalize(endflag=ESMF_END_ABORT)

    ! 2. use precomputed routehandle to perform SMM
    call ESMF_FieldSMM(srcfield, dstField, routehandle, rc=rc)
    if (rc /= ESMF_SUCCESS) call ESMF_Finalize(endflag=ESMF_END_ABORT)

    ! verify sparse matrix multiplication
    call ESMF_FieldGet(dstField, localDe=0, farrayPtr=fptr, rc=rc)
    if (rc /= ESMF_SUCCESS) call ESMF_Finalize(endflag=ESMF_END_ABORT)

    ! Verify that the result data in dstField is correct.
    ! Before the SMM op, the dst Field contains all 0.
    ! The SMM op reset the values to the index value, verify this is the case.
    ! +--------+--------+--------+--------+
    !  1 2 3 4  2 4 6 8  3 6 9 12  4 8 12 16       ! value
    ! 1        4        8        12       16       ! bounds
    do i = lbound(fptr, 1), ubound(fptr, 1)
        if(fptr(i) /= i*(lpe+1)) rc = ESMF_FAILURE
    enddo
 
\end{verbatim}
 
%/////////////////////////////////////////////////////////////

   Field sparse matrix multiplication can also be applied between Fields 
   that matche the original Fields in {\em type}, {\em kind}, and 
   memory layout of the {\em gridded} dimensions. However, the size, number, 
   and index order of {\em ungridded} dimensions may be different. See section
   \ref{RH:Reusability} for a more detailed discussion of RouteHandle 
   reusability 
%/////////////////////////////////////////////////////////////

 \begin{verbatim}
    call ESMF_ArraySpecSet(arrayspec, typekind=ESMF_TYPEKIND_I4, rank=2, rc=rc)
 
\end{verbatim}
 
%/////////////////////////////////////////////////////////////

   Create two fields with ungridded dimensions using the Grid created previously.
   The new Field pair has matching number of elements. The ungridded dimension
   is mapped to the first dimension of either Field. 
%/////////////////////////////////////////////////////////////

 \begin{verbatim}
    srcFieldA = ESMF_FieldCreate(grid, arrayspec, gridToFieldMap=(/2/), &
        ungriddedLBound=(/1/), ungriddedUBound=(/10/), rc=rc)
 
\end{verbatim}
 
%/////////////////////////////////////////////////////////////

 \begin{verbatim}
    dstFieldA = ESMF_FieldCreate(grid, arrayspec, gridToFieldMap=(/2/), &
        ungriddedLBound=(/1/), ungriddedUBound=(/10/), rc=rc)
 
\end{verbatim}
 
%/////////////////////////////////////////////////////////////

   Using the previously computed routehandle, the sparse matrix multiplication
   can be performed between the Fields. 
%/////////////////////////////////////////////////////////////

 \begin{verbatim}
    call ESMF_FieldSMM(srcfieldA, dstFieldA, routehandle, rc=rc)
 
\end{verbatim}
 
%/////////////////////////////////////////////////////////////

 \begin{verbatim}
    ! release route handle
    call ESMF_FieldSMMRelease(routehandle, rc=rc)
 
\end{verbatim}
 
%/////////////////////////////////////////////////////////////

   In the following discussion, we demonstrate how to set up a SMM routehandle
   between a pair of Fields that are different in number of gridded dimensions
   and the size of those gridded dimensions. The source Field has a 1D decomposition
   with 16 total elements; the destination Field has a 2D decomposition with
   12 total elements. For ease of understanding of the actual matrix calculation,
   a global indexing scheme is used. 
%/////////////////////////////////////////////////////////////

 \begin{verbatim}
    distgrid = ESMF_DistGridCreate(minIndex=(/1/), maxIndex=(/16/), &
        indexflag=ESMF_INDEX_GLOBAL, &
        regDecomp=(/4/), &
        rc=rc)
    if (rc /= ESMF_SUCCESS) call ESMF_Finalize(endflag=ESMF_END_ABORT)

    grid = ESMF_GridCreate(distgrid=distgrid, &
        indexflag=ESMF_INDEX_GLOBAL, &
        name="grid", rc=rc)
    if (rc /= ESMF_SUCCESS) call ESMF_Finalize(endflag=ESMF_END_ABORT)

    call ESMF_GridGetFieldBounds(grid, localDe=0, totalLBound=tlb, &
                       totalUBound=tub, rc=rc)
    if (rc /= ESMF_SUCCESS) call ESMF_Finalize(endflag=ESMF_END_ABORT)
 
\end{verbatim}
 
%/////////////////////////////////////////////////////////////

   create 1D src\_farray, srcArray, and srcField
  \begin{verbatim}
   +  PET0  +  PET1  +  PET2  +  PET3  +
   +--------+--------+--------+--------+
        1        2        3        4            ! value
   1        4        8        12       16       ! bounds of seq indices
  \end{verbatim} 
%/////////////////////////////////////////////////////////////

 \begin{verbatim}
    allocate(src_farray2(tlb(1):tub(1)) )
    src_farray2 = lpe+1
    srcArray = ESMF_ArrayCreate(distgrid, src_farray2, &
                  indexflag=ESMF_INDEX_GLOBAL, &
      rc=rc)
    if (rc /= ESMF_SUCCESS) call ESMF_Finalize(endflag=ESMF_END_ABORT)
    !print *, lpe, '+', tlb, tub, '+', src_farray2

    srcField = ESMF_FieldCreate(grid, srcArray, rc=rc)
    if (rc /= ESMF_SUCCESS) call ESMF_Finalize(endflag=ESMF_END_ABORT)

 
\end{verbatim}
 
%/////////////////////////////////////////////////////////////

   Create 2D dstField on the following distribution
   (numbers are the sequence indices):
  \begin{verbatim}
   +  PET0  +  PET1  +  PET2  +  PET3  +
   +--------+--------+--------+--------+
   |        |        |        |        |
   |   1    |   4    |   7    |   10   |
   |        |        |        |        |
   +--------+--------+--------+--------+
   |        |        |        |        |
   |   2    |   5    |   8    |   11   |
   |        |        |        |        |
   +--------+--------+--------+--------+
   |        |        |        |        |
   |   3    |   6    |   9    |   12   |
   |        |        |        |        |
   +--------+--------+--------+--------+
  \end{verbatim} 
%/////////////////////////////////////////////////////////////

 \begin{verbatim}

    ! Create the destination Grid
    dstGrid = ESMF_GridCreateNoPeriDim(minIndex=(/1,1/), maxIndex=(/3,4/), &
      indexflag = ESMF_INDEX_GLOBAL, &
      regDecomp = (/1,4/), &
      rc=rc)
    if (rc /= ESMF_SUCCESS) call ESMF_Finalize(endflag=ESMF_END_ABORT)

    dstField = ESMF_FieldCreate(dstGrid, typekind=ESMF_TYPEKIND_R4, &
      indexflag=ESMF_INDEX_GLOBAL, &
      rc=rc)
    if (rc /= ESMF_SUCCESS) call ESMF_Finalize(endflag=ESMF_END_ABORT)
 
\end{verbatim}
 
%/////////////////////////////////////////////////////////////

   Perform sparse matrix multiplication $dst_i$ = $M_{i,j}$ * $src_j$
   First setup routehandle from source Field to destination Field
   with prescribed factorList and factorIndexList.
  
   The sparse matrix is of size 12x16, however only the following entries
   are filled:
   \begin{verbatim}
   M(3,1) = 0.1
   M(3,10) = 0.4
   M(8,2) = 0.25
   M(8,16) = 0.5
   M(12,1) = 0.3
   M(12,16) = 0.7
   \end{verbatim}
  
   By the definition of matrix calculation, the 8th element on PET2 in the
   dstField equals to 0.25*srcField(2) + 0.5*srcField(16) = 0.25*1+0.5*4=2.25
   For simplicity, we will load the factorList and factorIndexList on
   PET 0 and 1, the SMMStore engine will load balance the parameters on all 4
   PETs internally for optimal performance. 
%/////////////////////////////////////////////////////////////

 \begin{verbatim}
    if(lpe == 0) then
      allocate(factorList(3), factorIndexList(2,3))
      factorList=(/0.1,0.4,0.25/)
      factorIndexList(1,:)=(/1,10,2/)
      factorIndexList(2,:)=(/3,3,8/)
      call ESMF_FieldSMMStore(srcField, dstField, routehandle=routehandle, &
          factorList=factorList, factorIndexList=factorIndexList, rc=localrc)
      if (localrc /= ESMF_SUCCESS) call ESMF_Finalize(endflag=ESMF_END_ABORT)
    else if(lpe == 1) then
      allocate(factorList(3), factorIndexList(2,3))
      factorList=(/0.5,0.3,0.7/)
      factorIndexList(1,:)=(/16,1,16/)
      factorIndexList(2,:)=(/8,12,12/)
      call ESMF_FieldSMMStore(srcField, dstField, routehandle=routehandle, &
          factorList=factorList, factorIndexList=factorIndexList, rc=localrc)
      if (localrc /= ESMF_SUCCESS) call ESMF_Finalize(endflag=ESMF_END_ABORT)
    else
      call ESMF_FieldSMMStore(srcField, dstField, routehandle=routehandle, &
          rc=localrc)
      if (localrc /= ESMF_SUCCESS) call ESMF_Finalize(endflag=ESMF_END_ABORT)
    endif

    ! 2. use precomputed routehandle to perform SMM
    call ESMF_FieldSMM(srcfield, dstField, routehandle=routehandle, rc=rc)
    if (rc /= ESMF_SUCCESS) call ESMF_Finalize(endflag=ESMF_END_ABORT)
 
\end{verbatim}

%...............................................................
\setlength{\parskip}{\oldparskip}
\setlength{\parindent}{\oldparindent}
\setlength{\baselineskip}{\oldbaselineskip}
