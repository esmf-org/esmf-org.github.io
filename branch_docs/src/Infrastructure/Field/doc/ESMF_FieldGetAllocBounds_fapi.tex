%                **** IMPORTANT NOTICE *****
% This LaTeX file has been automatically produced by ProTeX v. 1.1
% Any changes made to this file will likely be lost next time
% this file is regenerated from its source. Send questions 
% to Arlindo da Silva, dasilva@gsfc.nasa.gov
 
\setlength{\oldparskip}{\parskip}
\setlength{\parskip}{1.5ex}
\setlength{\oldparindent}{\parindent}
\setlength{\parindent}{0pt}
\setlength{\oldbaselineskip}{\baselineskip}
\setlength{\baselineskip}{11pt}
 
%--------------------- SHORT-HAND MACROS ----------------------
\def\bv{\begin{verbatim}}
\def\ev{\end{verbatim}}
\def\be{\begin{equation}}
\def\ee{\end{equation}}
\def\bea{\begin{eqnarray}}
\def\eea{\end{eqnarray}}
\def\bi{\begin{itemize}}
\def\ei{\end{itemize}}
\def\bn{\begin{enumerate}}
\def\en{\end{enumerate}}
\def\bd{\begin{description}}
\def\ed{\end{description}}
\def\({\left (}
\def\){\right )}
\def\[{\left [}
\def\]{\right ]}
\def\<{\left  \langle}
\def\>{\right \rangle}
\def\cI{{\cal I}}
\def\diag{\mathop{\rm diag}}
\def\tr{\mathop{\rm tr}}
%-------------------------------------------------------------

\markboth{Left}{Source File: ESMF\_FieldGetAllocBounds.F90,  Date: Tue May  5 21:00:00 MDT 2020
}

 
%/////////////////////////////////////////////////////////////
\subsubsection [ESMF\_GridGetFieldBounds] {ESMF\_GridGetFieldBounds - Get precomputed DE-local Fortran data array bounds for creating a Field from a Grid and Fortran array}


 
\bigskip{\sf INTERFACE:}
\begin{verbatim}     subroutine ESMF_GridGetFieldBounds(grid, &
         localDe, staggerloc, gridToFieldMap, &
         ungriddedLBound, ungriddedUBound, &
         totalLWidth, totalUWidth, &
         totalLBound, totalUBound, totalCount, rc)
     \end{verbatim}{\em ARGUMENTS:}
\begin{verbatim}     type(ESMF_Grid),       intent(in)            :: grid     
 -- The following arguments require argument keyword syntax (e.g. rc=rc). --
     integer,               intent(in),  optional :: localDe
     type(ESMF_StaggerLoc), intent(in),  optional :: staggerloc 
     integer,               intent(in),  optional :: gridToFieldMap(:)    
     integer,               intent(in),  optional :: ungriddedLBound(:)
     integer,               intent(in),  optional :: ungriddedUBound(:)
     integer,               intent(in),  optional :: totalLWidth(:)
     integer,               intent(in),  optional :: totalUWidth(:)
     integer,               intent(out), optional :: totalLBound(:)
     integer,               intent(out), optional :: totalUBound(:)
     integer,               intent(out), optional :: totalCount(:)
     integer,               intent(out), optional :: rc     
 \end{verbatim}
{\sf STATUS:}
   \begin{itemize}
   \item\apiStatusCompatibleVersion{5.2.0r}
   \end{itemize}
  
{\sf DESCRIPTION:\\ }


   Compute the lower and upper bounds of Fortran data array that can later
   be used in FieldCreate interface to create a {\tt ESMF\_Field} from a
   {\tt ESMF\_Grid} and the Fortran data array. For an example and
   associated documentation using this method see section 
   \ref{sec:field:usage:create_5dgrid_7dptr_2dungridded}.
  
   The arguments are:
   \begin{description}
   \item [grid]
         {\tt ESMF\_Grid}.
   \item [{[localDe]}]
         Local DE for which information is requested. {\tt [0,..,localDeCount-1]}.
         For {\tt localDeCount==1} the {\tt localDe} argument may be omitted,
         in which case it will default to {\tt localDe=0}.
   \item [{[staggerloc]}]
         Stagger location of data in grid cells.  For valid
         predefined values and interpretation of results see
         section \ref{const:staggerloc}.
   \item [{[gridToFieldMap]}]
         List with number of elements equal to the
         {\tt grid}|s dimCount.  The list elements map each dimension
         of the {\tt grid} to a dimension in the {\tt field} by
         specifying the appropriate {\tt field} dimension index. The default is to
         map all of the {\tt grid}|s dimensions against the lowest dimensions of
         the {\tt field} in sequence, i.e. {\tt gridToFieldMap} = (/1,2,3,.../).
         The values of all {\tt gridToFieldMap} entries must be greater than or equal
         to one and smaller than or equal to the {\tt field} rank.
         It is erroneous to specify the same {\tt gridToFieldMap} entry
         multiple times. The total ungridded dimensions in the {\tt field}
         are the total {\tt field} dimensions less
         the dimensions in
         the {\tt grid}.  Ungridded dimensions must be in the same order they are
         stored in the {\tt field}.  
   \item [{[ungriddedLBound]}]
         Lower bounds of the ungridded dimensions of the {\tt field}.
         The number of elements in the {\tt ungriddedLBound} is equal to the number of ungridded
         dimensions in the {\tt field}.  All ungridded dimensions of the
         {\tt field} are also undistributed. When field dimension count is
         greater than grid dimension count, both ungriddedLBound and ungriddedUBound
         must be specified. When both are specified the values are checked
         for consistency.  Note that the the ordering of
         these ungridded dimensions is the same as their order in the {\tt field}.
   \item [{[ungriddedUBound]}]
         Upper bounds of the ungridded dimensions of the {\tt field}.
         The number of elements in the {\tt ungriddedUBound} is equal to the number of ungridded
         dimensions in the {\tt field}.  All ungridded dimensions of the
         {\tt field} are also undistributed. When field dimension count is
         greater than grid dimension count, both ungriddedLBound and ungriddedUBound
         must be specified. When both are specified the values are checked
         for consistency.  Note that the the ordering of
         these ungridded dimensions is the same as their order in the {\tt field}.
   \item [{[totalLWidth]}]
         Lower bound of halo region.  The size of this array is the number
         of dimensions in the {\tt grid}.  However, ordering of the elements
         needs to be the same as they appear in the {\tt field}.  Values default
         to 0.  If values for totalLWidth are specified they must be reflected in
         the size of the {\tt field}.  That is, for each gridded dimension the
         {\tt field} size should be max( {\tt totalLWidth} + {\tt totalUWidth}
         + {\tt computationalCount}, {\tt exclusiveCount} ).
   \item [{[totalUWidth]}]
         Upper bound of halo region.  The size of this array is the number
         of dimensions in the {\tt grid}.  However, ordering of the elements
         needs to be the same as they appear in the {\tt field}.  Values default
         to 0.  If values for totalUWidth are specified they must be reflected in
         the size of the {\tt field}.  That is, for each gridded dimension the
         {\tt field} size should max( {\tt totalLWidth} + {\tt totalUWidth}
         + {\tt computationalCount}, {\tt exclusiveCount} ).
   \item [{[totalLBound]}]
         \begin{sloppypar}
         The relative lower bounds of Fortran data array to be used
         later in {\tt ESMF\_FieldCreate} from {\tt ESMF\_Grid} and Fortran data array.
         This is an output variable from this user interface.
         \end{sloppypar}
         The relative lower bounds of Fortran data array to be used
   \item [{[totalUBound]}]
         \begin{sloppypar}
         The relative upper bounds of Fortran data array to be used
         later in {\tt ESMF\_FieldCreate} from {\tt ESMF\_Grid} and Fortran data array.
         This is an output variable from this user interface.
         \end{sloppypar}
   \item [{[totalCount]}]
         Number of elements need to be allocated for Fortran data array to be used
         later in {\tt ESMF\_FieldCreate} from {\tt ESMF\_Grid} and Fortran data array.
         This is an output variable from this user interface.
  
   \item[{[rc]}]
       Return code; equals {\tt ESMF\_SUCCESS} if there are no errors.
   \end{description} 
%/////////////////////////////////////////////////////////////
 
\mbox{}\hrulefill\ 
 
\subsubsection [ESMF\_LocStreamGetFieldBounds] {ESMF\_LocStreamGetFieldBounds - Get precomputed DE-local Fortran data array bounds for creating a Field from a LocStream and Fortran array}


 
\bigskip{\sf INTERFACE:}
\begin{verbatim}     subroutine ESMF_LocStreamGetFieldBounds(locstream, &
         localDe, gridToFieldMap, &
         ungriddedLBound, ungriddedUBound, &
         totalLBound, totalUBound, totalCount, rc)
     \end{verbatim}{\em ARGUMENTS:}
\begin{verbatim}     type(ESMF_LocStream), intent(in)            :: locstream     
 -- The following arguments require argument keyword syntax (e.g. rc=rc). --
     integer,              intent(in),  optional :: localDe
     integer,              intent(in),  optional :: gridToFieldMap(:)    
     integer,              intent(in),  optional :: ungriddedLBound(:)
     integer,              intent(in),  optional :: ungriddedUBound(:)
     integer,              intent(out), optional :: totalLBound(:)
     integer,              intent(out), optional :: totalUBound(:)
     integer,              intent(out), optional :: totalCount(:)
     integer,              intent(out), optional :: rc     
 \end{verbatim}
{\sf DESCRIPTION:\\ }


   Compute the lower and upper bounds of Fortran data array that can later
   be used in FieldCreate interface to create a {\tt ESMF\_Field} from a
   {\tt ESMF\_LocStream} and the Fortran data array.  For an example and
   associated documentation using this method see section 
   \ref{sec:field:usage:create_5dgrid_7dptr_2dungridded}.
  
   The arguments are:
   \begin{description}
   \item [locstream]
         {\tt ESMF\_LocStream}.
   \item [{[localDe]}]
         Local DE for which information is requested. {\tt [0,..,localDeCount-1]}.
         For {\tt localDeCount==1} the {\tt localDe} argument may be omitted,
         in which case it will default to {\tt localDe=0}.
   \item [{[gridToFieldMap]}]
         List with number of elements equal to 1.
         The list elements map the dimension
         of the {\tt locstream} to a dimension in the {\tt field} by
         specifying the appropriate {\tt field} dimension index. The default is to
         map the {\tt locstream}|s dimension against the lowest dimension of
         the {\tt field} in sequence, i.e. {\tt gridToFieldMap} = (/1/).
         The values of all {\tt gridToFieldMap} entries must be greater than or equal
         to one and smaller than or equal to the {\tt field} rank.
         The total ungridded dimensions in the {\tt field}
         are the total {\tt field} dimensions less
         the dimensions in
         the {\tt grid}.  Ungridded dimensions must be in the same order they are
         stored in the {\t field}.  
   \item [{[ungriddedLBound]}]
         Lower bounds of the ungridded dimensions of the {\tt field}.
         The number of elements in the {\tt ungriddedLBound} is equal to the number of ungridded
         dimensions in the {\tt field}.  All ungridded dimensions of the
         {\tt field} are also undistributed. When field dimension count is
         greater than 1, both ungriddedLBound and ungriddedUBound
         must be specified. When both are specified the values are checked
         for consistency.  Note that the the ordering of
         these ungridded dimensions is the same as their order in the {\tt field}.
   \item [{[ungriddedUBound]}]
         Upper bounds of the ungridded dimensions of the {\tt field}.
         The number of elements in the {\tt ungriddedUBound} is equal to the number of ungridded
         dimensions in the {\tt field}.  All ungridded dimensions of the
         {\tt field} are also undistributed. When field dimension count is
         greater than 1, both ungriddedLBound and ungriddedUBound
         must be specified. When both are specified the values are checked
         for consistency.  Note that the the ordering of
         these ungridded dimensions is the same as their order in the {\tt field}.
   \item [{[totalLBound]}]
         \begin{sloppypar}
         The relative lower bounds of Fortran data array to be used
         later in {\tt ESMF\_FieldCreate} from {\tt ESMF\_LocStream} and Fortran data array.
         This is an output variable from this user interface.
         \end{sloppypar}
   \item [{[totalUBound]}]
         \begin{sloppypar}
         The relative upper bounds of Fortran data array to be used
         later in {\tt ESMF\_FieldCreate} from {\tt ESMF\_LocStream} and Fortran data array.
         This is an output variable from this user interface.
         \end{sloppypar}
   \item [{[totalCount]}]
         Number of elements need to be allocated for Fortran data array to be used
         later in {\tt ESMF\_FieldCreate} from {\tt ESMF\_LocStream} and Fortran data array.
         This is an output variable from this user interface.
  
   \item[{[rc]}]
       Return code; equals {\tt ESMF\_SUCCESS} if there are no errors.
   \end{description} 
%/////////////////////////////////////////////////////////////
 
\mbox{}\hrulefill\ 
 
\subsubsection [ESMF\_MeshGetFieldBounds] {ESMF\_MeshGetFieldBounds - Get precomputed DE-local Fortran data array bounds for creating a Field from a Mesh and a Fortran array}


 
\bigskip{\sf INTERFACE:}
\begin{verbatim}     subroutine ESMF_MeshGetFieldBounds(mesh, &
         meshloc, &
         localDe, gridToFieldMap, &
         ungriddedLBound, ungriddedUBound, &
         totalLBound, totalUBound, totalCount, rc)
     \end{verbatim}{\em ARGUMENTS:}
\begin{verbatim}     type(ESMF_Mesh), intent(in)            :: mesh     
 -- The following arguments require argument keyword syntax (e.g. rc=rc). --
     type(ESMF_MeshLoc),intent(in),optional :: meshloc
     integer,         intent(in),  optional :: localDe
     integer,         intent(in),  optional :: gridToFieldMap(:)    
     integer,         intent(in),  optional :: ungriddedLBound(:)
     integer,         intent(in),  optional :: ungriddedUBound(:)
     integer,         intent(out), optional :: totalLBound(:)
     integer,         intent(out), optional :: totalUBound(:)
     integer,         intent(out), optional :: totalCount(:)
     integer,         intent(out), optional :: rc     
 \end{verbatim}
{\sf DESCRIPTION:\\ }


   Compute the lower and upper bounds of Fortran data array that can later
   be used in FieldCreate interface to create a {\tt ESMF\_Field} from a
   {\tt ESMF\_Mesh} and the Fortran data array. For an example and
   associated documentation using this method see section 
   \ref{sec:field:usage:create_5dgrid_7dptr_2dungridded}.
  
   The arguments are:
   \begin{description}
   \item [mesh]
         {\tt ESMF\_Mesh}.
   \item [{[meshloc]}]
         \begin{sloppypar}
         Which part of the mesh to build the Field on. Can be set to either
         {\tt ESMF\_MESHLOC\_NODE} or {\tt ESMF\_MESHLOC\_ELEMENT}. If not set,
         defaults to {\tt ESMF\_MESHLOC\_NODE}.
         \end{sloppypar}
   \item [{[localDe]}]
         Local DE for which information is requested. {\tt [0,..,localDeCount-1]}.
         For {\tt localDeCount==1} the {\tt localDe} argument may be omitted,
         in which case it will default to {\tt localDe=0}.
   \item [{[gridToFieldMap]}]
         List with number of elements equal to the
         {\tt grid}|s dimCount.  The list elements map each dimension
         of the {\tt grid} to a dimension in the {\tt field} by
         specifying the appropriate {\tt field} dimension index. The default is to
         map all of the {\tt grid}|s dimensions against the lowest dimensions of
         the {\tt field} in sequence, i.e. {\tt gridToFieldMap} = (/1,2,3,.../).
         The values of all {\tt gridToFieldMap} entries must be greater than or equal
         to one and smaller than or equal to the {\tt field} rank.
         It is erroneous to specify the same {\tt gridToFieldMap} entry
         multiple times. The total ungridded dimensions in the {\tt field}
         are the total {\tt field} dimensions less
         the dimensions in
         the {\tt grid}.  Ungridded dimensions must be in the same order they are
         stored in the {\t field}.  
   \item [{[ungriddedLBound]}]
         Lower bounds of the ungridded dimensions of the {\tt field}.
         The number of elements in the {\tt ungriddedLBound} is equal to the number of ungridded
         dimensions in the {\tt field}.  All ungridded dimensions of the
         {\tt field} are also undistributed. When field dimension count is
         greater than grid dimension count, both ungriddedLBound and ungriddedUBound
         must be specified. When both are specified the values are checked
         for consistency.  Note that the the ordering of
         these ungridded dimensions is the same as their order in the {\tt field}.
   \item [{[ungriddedUBound]}]
         Upper bounds of the ungridded dimensions of the {\tt field}.
         The number of elements in the {\tt ungriddedUBound} is equal to the number of ungridded
         dimensions in the {\tt field}.  All ungridded dimensions of the
         {\tt field} are also undistributed. When field dimension count is
         greater than grid dimension count, both ungriddedLBound and ungriddedUBound
         must be specified. When both are specified the values are checked
         for consistency.  Note that the the ordering of
         these ungridded dimensions is the same as their order in the {\tt field}.
   \item [{[totalLBound]}]
         \begin{sloppypar}
         The relative lower bounds of Fortran data array to be used
         later in {\tt ESMF\_FieldCreate} from {\tt ESMF\_Mesh} and Fortran data array.
         This is an output variable from this user interface.
         \end{sloppypar}
   \item [{[totalUBound]}]
         \begin{sloppypar}
         The relative upper bounds of Fortran data array to be used
         later in {\tt ESMF\_FieldCreate} from {\tt ESMF\_Mesh} and Fortran data array.
         This is an output variable from this user interface.
         \end{sloppypar}
   \item [{[totalCount]}]
         Number of elements need to be allocated for Fortran data array to be used
         later in {\tt ESMF\_FieldCreate} from {\tt ESMF\_Mesh} and Fortran data array.
         This is an output variable from this user interface.
  
   \item[{[rc]}]
       Return code; equals {\tt ESMF\_SUCCESS} if there are no errors.
   \end{description} 
%/////////////////////////////////////////////////////////////
 
\mbox{}\hrulefill\ 
 
\subsubsection [ESMF\_XGridGetFieldBounds] {ESMF\_XGridGetFieldBounds - Get precomputed DE-local Fortran data array bounds for creating a Field from an XGrid and a Fortran array}


 
\bigskip{\sf INTERFACE:}
\begin{verbatim}     subroutine ESMF_XGridGetFieldBounds(xgrid, &
         xgridside, gridindex, localDe, gridToFieldMap, &
         ungriddedLBound, ungriddedUBound, &
         totalLBound, totalUBound, totalCount, rc)
     \end{verbatim}{\em ARGUMENTS:}
\begin{verbatim}     type(ESMF_XGrid),          intent(in)            :: xgrid
 -- The following arguments require argument keyword syntax (e.g. rc=rc). --
     type(ESMF_XGridSide_Flag), intent(in),  optional :: xgridside
     integer,                   intent(in),  optional :: gridindex
     integer,                   intent(in),  optional :: localDe
     integer,                   intent(in),  optional :: gridToFieldMap(:)    
     integer,                   intent(in),  optional :: ungriddedLBound(:)
     integer,                   intent(in),  optional :: ungriddedUBound(:)
     integer,                   intent(out), optional :: totalLBound(:)
     integer,                   intent(out), optional :: totalUBound(:)
     integer,                   intent(out), optional :: totalCount(:)
     integer,                   intent(out), optional :: rc     
 \end{verbatim}
{\sf DESCRIPTION:\\ }


   Compute the lower and upper bounds of Fortran data array that can later
   be used in FieldCreate interface to create a {\tt ESMF\_Field} from a
   {\tt ESMF\_XGrid} and the Fortran data array.  For an example and
   associated documentation using this method see section 
   \ref{sec:field:usage:create_5dgrid_7dptr_2dungridded}.
  
   The arguments are:
   \begin{description}
   \item [xgrid]
          {\tt ESMF\_XGrid} object.
   \item [{[xgridside]}]
         Which side of the XGrid to create the Field on (either ESMF\_XGRIDSIDE\_A,
         ESMF\_XGRIDSIDE\_B, or ESMF\_XGRIDSIDE\_BALANCED). If not passed in then
         defaults to ESMF\_XGRIDSIDE\_BALANCED.
   \item [{[gridindex]}]
         If xgridside is ESMF\_XGRIDSIDE\_A or ESMF\_XGRIDSIDE\_B then this index tells which Grid on
         that side to create the Field on. If not provided, defaults to 1.
   \item [{[localDe]}]
         Local DE for which information is requested. {\tt [0,..,localDeCount-1]}.
         For {\tt localDeCount==1} the {\tt localDe} argument may be omitted,
         in which case it will default to {\tt localDe=0}.
   \item [{[gridToFieldMap]}]
         List with number of elements equal to 1.
         The list elements map the dimension
         of the {\tt locstream} to a dimension in the {\tt field} by
         specifying the appropriate {\tt field} dimension index. The default is to
         map the {\tt locstream}|s dimension against the lowest dimension of
         the {\tt field} in sequence, i.e. {\tt gridToFieldMap} = (/1/).
         The values of all {\tt gridToFieldMap} entries must be greater than or equal
         to one and smaller than or equal to the {\tt field} rank.
         The total ungridded dimensions in the {\tt field}
         are the total {\tt field} dimensions less
         the dimensions in
         the {\tt grid}.  Ungridded dimensions must be in the same order they are
         stored in the {\t field}.  
   \item [{[ungriddedLBound]}]
         Lower bounds of the ungridded dimensions of the {\tt field}.
         The number of elements in the {\tt ungriddedLBound} is equal to the number of ungridded
         dimensions in the {\tt field}.  All ungridded dimensions of the
         {\tt field} are also undistributed. When field dimension count is
         greater than 1, both ungriddedLBound and ungriddedUBound
         must be specified. When both are specified the values are checked
         for consistency.  Note that the the ordering of
         these ungridded dimensions is the same as their order in the {\tt field}.
   \item [{[ungriddedUBound]}]
         Upper bounds of the ungridded dimensions of the {\tt field}.
         The number of elements in the {\tt ungriddedUBound} is equal to the number of ungridded
         dimensions in the {\tt field}.  All ungridded dimensions of the
         {\tt field} are also undistributed. When field dimension count is
         greater than 1, both ungriddedLBound and ungriddedUBound
         must be specified. When both are specified the values are checked
         for consistency.  Note that the the ordering of
         these ungridded dimensions is the same as their order in the {\tt field}.
   \item [{[totalLBound]}]
         \begin{sloppypar}
         The relative lower bounds of Fortran data array to be used
         later in {\tt ESMF\_FieldCreate} from {\tt ESMF\_LocStream} and Fortran data array.
         This is an output variable from this user interface.
         \end{sloppypar}
   \item [{[totalUBound]}]
         \begin{sloppypar}
         The relative upper bounds of Fortran data array to be used
         later in {\tt ESMF\_FieldCreate} from {\tt ESMF\_LocStream} and Fortran data array.
         This is an output variable from this user interface.
         \end{sloppypar}
   \item [{[totalCount]}]
         Number of elements need to be allocated for Fortran data array to be used
         later in {\tt ESMF\_FieldCreate} from {\tt ESMF\_LocStream} and Fortran data array.
         This is an output variable from this user interface.
  
   \item[{[rc]}]
       Return code; equals {\tt ESMF\_SUCCESS} if there are no errors.
   \end{description}
%...............................................................
\setlength{\parskip}{\oldparskip}
\setlength{\parindent}{\oldparindent}
\setlength{\baselineskip}{\oldbaselineskip}
