%                **** IMPORTANT NOTICE *****
% This LaTeX file has been automatically produced by ProTeX v. 1.1
% Any changes made to this file will likely be lost next time
% this file is regenerated from its source. Send questions 
% to Arlindo da Silva, dasilva@gsfc.nasa.gov
 
\setlength{\oldparskip}{\parskip}
\setlength{\parskip}{1.5ex}
\setlength{\oldparindent}{\parindent}
\setlength{\parindent}{0pt}
\setlength{\oldbaselineskip}{\baselineskip}
\setlength{\baselineskip}{11pt}
 
%--------------------- SHORT-HAND MACROS ----------------------
\def\bv{\begin{verbatim}}
\def\ev{\end{verbatim}}
\def\be{\begin{equation}}
\def\ee{\end{equation}}
\def\bea{\begin{eqnarray}}
\def\eea{\end{eqnarray}}
\def\bi{\begin{itemize}}
\def\ei{\end{itemize}}
\def\bn{\begin{enumerate}}
\def\en{\end{enumerate}}
\def\bd{\begin{description}}
\def\ed{\end{description}}
\def\({\left (}
\def\){\right )}
\def\[{\left [}
\def\]{\right ]}
\def\<{\left  \langle}
\def\>{\right \rangle}
\def\cI{{\cal I}}
\def\diag{\mathop{\rm diag}}
\def\tr{\mathop{\rm tr}}
%-------------------------------------------------------------

\markboth{Left}{Source File: ESMF\_FieldScatter.F90,  Date: Tue May  5 21:00:02 MDT 2020
}

 
%/////////////////////////////////////////////////////////////

   \subsubsection [ESMF\_FieldScatter] {ESMF\_FieldScatter - Scatter a Fortran array across the ESMF\_Field }


   
\bigskip{\sf INTERFACE:}
\begin{verbatim}   subroutine ESMF_FieldScatter<rank><type><kind>(field, farray, & 
   rootPet, tile, vm, rc) 
   \end{verbatim}{\em ARGUMENTS:}
\begin{verbatim}   type(ESMF_Field), intent(inout) :: field 
   mtype (ESMF_KIND_mtypekind),intent(in), target :: farray(mdim) 
   integer, intent(in) :: rootPet 
 -- The following arguments require argument keyword syntax (e.g. rc=rc). --
   integer, intent(in), optional :: tile 
   type(ESMF_VM), intent(in), optional :: vm 
   integer, intent(out), optional :: rc 
   
   
   \end{verbatim}
{\sf STATUS:}
   \begin{itemize} 
   \item\apiStatusCompatibleVersion{5.2.0r} 
   \end{itemize} 
   
{\sf DESCRIPTION:\\ }

 
   Scatter the data of {\tt farray} located on {\tt rootPET} 
   across an {ESMF\_Field} object. A single {\tt farray} must be 
   scattered across a single DistGrid tile in Field. The optional {\tt tile} 
   argument allows selection of the tile. For Fields defined on a single 
   tile DistGrid the default selection (tile 1) will be correct. The 
   shape of {\tt farray} must match the shape of the tile in Field. 
   
   If the Field contains replicating DistGrid dimensions data will be 
   scattered across all of the replicated pieces. 
   
   The implementation of Scatter and Gather is not sequence index based. 
   If the Field is built on arbitrarily distributed Grid, Mesh, LocStream or XGrid, 
   Scatter will not scatter data from rootPet 
   to the destination data points corresponding to the sequence index on the rootPet. 
   Instead Scatter will scatter a contiguous memory range from rootPet to 
   destination PET. The size of the memory range is equal to the number of 
   data elements on the destination PET. Vice versa for the Gather operation. 
   In this case, the user should use {\tt ESMF\_FieldRedist} to achieve 
   the same data operation result. For examples how to use {\tt ESMF\_FieldRedist} 
   to perform Gather and Scatter, please refer to 
   \ref{sec:field:usage:redist_gathering} and 
   \ref{sec:field:usage:redist_scattering}. 
   
   This version of the interface implements the PET-based blocking paradigm: 
   Each PET of the VM must issue this call exactly once for {\em all} of its 
   DEs. The call will block until all PET-local data objects are accessible. 
   
   For examples and associated documentation regarding this method see Section 
   \ref{sec:field:usage:scatter_2dptr}. 
   
   The arguments are: 
   \begin{description} 
   \item[field] 
   The {\tt ESMF\_Field} object across which data will be scattered. 
   \item[\{farray\}] 
   The Fortran array that is to be scattered. Only root 
   must provide a valid {\tt farray}, the other PETs may treat 
   {\tt farray} as an optional argument. 
   \item[rootPet] 
   PET that holds the valid data in {\tt farray}. 
   \item[{[tile]}] 
   The DistGrid tile in {\tt field} into which to scatter {\tt farray}. 
   By default {\tt farray} will be scattered into tile 1. 
   \item[{[vm]}] 
   Optional {\tt ESMF\_VM} object of the current context. Providing the 
   VM of the current context will lower the method's overhead. 
   \item[{[rc]}] 
   Return code; equals {\tt ESMF\_SUCCESS} if there are no errors. 
   \end{description} 
   
%...............................................................
\setlength{\parskip}{\oldparskip}
\setlength{\parindent}{\oldparindent}
\setlength{\baselineskip}{\oldbaselineskip}
