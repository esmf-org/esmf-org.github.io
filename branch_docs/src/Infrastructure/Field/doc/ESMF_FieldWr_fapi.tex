%                **** IMPORTANT NOTICE *****
% This LaTeX file has been automatically produced by ProTeX v. 1.1
% Any changes made to this file will likely be lost next time
% this file is regenerated from its source. Send questions 
% to Arlindo da Silva, dasilva@gsfc.nasa.gov
 
\setlength{\oldparskip}{\parskip}
\setlength{\parskip}{1.5ex}
\setlength{\oldparindent}{\parindent}
\setlength{\parindent}{0pt}
\setlength{\oldbaselineskip}{\baselineskip}
\setlength{\baselineskip}{11pt}
 
%--------------------- SHORT-HAND MACROS ----------------------
\def\bv{\begin{verbatim}}
\def\ev{\end{verbatim}}
\def\be{\begin{equation}}
\def\ee{\end{equation}}
\def\bea{\begin{eqnarray}}
\def\eea{\end{eqnarray}}
\def\bi{\begin{itemize}}
\def\ei{\end{itemize}}
\def\bn{\begin{enumerate}}
\def\en{\end{enumerate}}
\def\bd{\begin{description}}
\def\ed{\end{description}}
\def\({\left (}
\def\){\right )}
\def\[{\left [}
\def\]{\right ]}
\def\<{\left  \langle}
\def\>{\right \rangle}
\def\cI{{\cal I}}
\def\diag{\mathop{\rm diag}}
\def\tr{\mathop{\rm tr}}
%-------------------------------------------------------------

\markboth{Left}{Source File: ESMF\_FieldWr.F90,  Date: Tue May  5 21:00:00 MDT 2020
}

 
%/////////////////////////////////////////////////////////////
\subsubsection [ESMF\_FieldWrite] {ESMF\_FieldWrite - Write Field data into a file}


   \label{api:FieldWrite}
 
\bigskip{\sf INTERFACE:}
\begin{verbatim}   subroutine ESMF_FieldWrite(field, fileName,   &
       variableName, convention, purpose, overwrite, status, timeslice, iofmt, rc)\end{verbatim}{\em ARGUMENTS:}
\begin{verbatim}     type(ESMF_Field),           intent(in)             :: field 
     character(*),               intent(in)             :: fileName
 -- The following arguments require argument keyword syntax (e.g. rc=rc). --
     character(*),               intent(in),  optional  :: variableName
     character(*),               intent(in),  optional  :: convention
     character(*),               intent(in),  optional  :: purpose
     logical,                    intent(in),  optional  :: overwrite
     type(ESMF_FileStatus_Flag), intent(in),  optional  :: status
     integer,                    intent(in),  optional  :: timeslice
     type(ESMF_IOFmt_Flag),      intent(in),  optional  :: iofmt
     integer,                    intent(out), optional  :: rc\end{verbatim}
{\sf DESCRIPTION:\\ }


     Write Field data into a file.  For this API to be functional, the 
     environment variable {\tt ESMF\_PIO} should be set to "internal" when 
     the ESMF library is built.  Please see the section on 
     Data I/O,~\ref{io:dataio}.
  
     When {\tt convention} and {\tt purpose} arguments are specified,
     a NetCDF variable can be created with user-specified dimension labels and
     attributes.  Dimension labels may be defined for both gridded and
     ungridded dimensions.  Dimension labels for gridded dimensions are specified
     at the Grid level by attaching an ESMF Attribute package to it.  The Attribute
     package must contain an attribute named by the pre-defined ESMF parameter
     {\tt ESMF\_ATT\_GRIDDED\_DIM\_LABELS}.  The corresponding value is an array of
     character strings specifying the desired names of the dimensions.  Likewise,
     for ungridded dimensions, an Attribute package is attached at the Field level.
     The name of the name must be {\tt ESMF\_ATT\_UNGRIDDED\_DIM\_LABELS}.
  
     NetCDF attributes for the variable can also be specified.  As with dimension labels,
     an Attribute package is added to the Field with the desired names and values.
     A value may be either a scalar character string, or a scalar or array of type
     integer, real, or double precision.  Dimension label attributes can co-exist with
     variable attributes within a common Attribute package.
  
     Limitations:
     \begin{itemize}
       \item Only single tile Fields are supported.
       \item Not supported in {\tt ESMF\_COMM=mpiuni} mode.
     \end{itemize}
  
     The arguments are:
     \begin{description}
     \item [field]
       The {\tt ESMF\_Field} object that contains data to be written.
     \item[fileName]
       The name of the output file to which Field data is written.
     \item[{[variableName]}]
      Variable name in the output file; default is the "name" of field.
      Use this argument only in the I/O format (such as NetCDF) that
      supports variable name. If the I/O format does not support this
      (such as binary format), ESMF will return an error code.
     \item[{[convention]}]
       Specifies an Attribute package associated with the Field, used to create NetCDF
       dimension labels and attributes for the variable in the file.  When this argument is present,
       the {\tt purpose} argument must also be present.  Use this argument only with a NetCDF
       I/O format. If binary format is used, ESMF will return an error code.
     \item[{[purpose]}]
       Specifies an Attribute package associated with the Field, used to create NetCDF
       dimension labels and attributes for the variable in the file.  When this argument is present,
       the {\tt convention} argument must also be present.  Use this argument only with a NetCDF
       I/O format. If binary format is used, ESMF will return an error code.
     \item[{[overwrite]}]
      \begin{sloppypar}
        A logical flag, the default is .false., i.e., existing field data may
        {\em not} be overwritten. If .true., the overwrite behavior depends
        on the value of {\tt iofmt} as shown below:
      \begin{description}
      \item[{\tt iofmt} = {\tt ESMF\_IOFMT\_BIN}:]\ All data in the file will
        be overwritten with each field's data.
      \item[{\tt iofmt} = {\tt ESMF\_IOFMT\_NETCDF, ESMF\_IOFMT\_NETCDF\_64BIT\_OFFSET}:]\ Only the
        data corresponding to each field's name will be
        be overwritten. If the {\tt timeslice} option is given, only data for
        the given timeslice may be overwritten.
        Note that it is always an error to attempt to overwrite a NetCDF
        variable with data which has a different shape.
      \end{description}
      \end{sloppypar}
     \item[{[status]}]
      \begin{sloppypar}
      The file status. Please see Section~\ref{const:filestatusflag} for
      the list of options. If not present, defaults to
      {\tt ESMF\_FILESTATUS\_UNKNOWN}.
      \end{sloppypar}
     \item[{[timeslice]}]
      \begin{sloppypar}
      Some I/O formats (e.g. NetCDF) support the output of data in form of
      time slices.  An unlimited dimension called {\tt time} is defined in the
      file variable for this capability.
      The {\tt timeslice} argument provides access to the {\tt time} dimension,
      and must have a positive value. The behavior of this
      option may depend on the setting of the {\tt overwrite} flag:
      \begin{description}
      \item[{\tt overwrite = .false.}:]\ If the timeslice value is
      less than the maximum time already in the file, the write will fail.
      \item[{\tt overwrite = .true.}:]\ Any positive timeslice value is valid.
      \end{description}
      By default, i.e. by omitting the {\tt timeslice} argument, no
      provisions for time slicing are made in the output file,
      however, if the file already contains a time axis for the variable,
      a timeslice one greater than the maximum will be written.
      \end{sloppypar}
     \item[{[iofmt]}]
       \begin{sloppypar}
      The I/O format.  Please see Section~\ref{opt:iofmtflag} for the list
      of options. If not present, file names with a {\tt .bin} extension will
      use {\tt ESMF\_IOFMT\_BIN}, and file names with a {\tt .nc} extension
      will use {\tt ESMF\_IOFMT\_NETCDF}.  Other files default to
      {\tt ESMF\_IOFMT\_NETCDF}.
       \end{sloppypar}
     \item [{[rc]}]
       Return code; equals {\tt ESMF\_SUCCESS} if there are no errors.
     \end{description}
  
%...............................................................
\setlength{\parskip}{\oldparskip}
\setlength{\parindent}{\oldparindent}
\setlength{\baselineskip}{\oldbaselineskip}
