%                **** IMPORTANT NOTICE *****
% This LaTeX file has been automatically produced by ProTeX v. 1.1
% Any changes made to this file will likely be lost next time
% this file is regenerated from its source. Send questions 
% to Arlindo da Silva, dasilva@gsfc.nasa.gov
 
\setlength{\oldparskip}{\parskip}
\setlength{\parskip}{1.5ex}
\setlength{\oldparindent}{\parindent}
\setlength{\parindent}{0pt}
\setlength{\oldbaselineskip}{\baselineskip}
\setlength{\baselineskip}{11pt}
 
%--------------------- SHORT-HAND MACROS ----------------------
\def\bv{\begin{verbatim}}
\def\ev{\end{verbatim}}
\def\be{\begin{equation}}
\def\ee{\end{equation}}
\def\bea{\begin{eqnarray}}
\def\eea{\end{eqnarray}}
\def\bi{\begin{itemize}}
\def\ei{\end{itemize}}
\def\bn{\begin{enumerate}}
\def\en{\end{enumerate}}
\def\bd{\begin{description}}
\def\ed{\end{description}}
\def\({\left (}
\def\){\right )}
\def\[{\left [}
\def\]{\right ]}
\def\<{\left  \langle}
\def\>{\right \rangle}
\def\cI{{\cal I}}
\def\diag{\mathop{\rm diag}}
\def\tr{\mathop{\rm tr}}
%-------------------------------------------------------------

\markboth{Left}{Source File: ESMF\_FieldHaloEx.F90,  Date: Tue May  5 21:00:02 MDT 2020
}

 
%/////////////////////////////////////////////////////////////

   \subsubsection{Field Halo solving a domain decomposed heat transfer problem}
   \label{sec:field:usage:halo}
  
   The {\tt ESMF\_FieldHalo()} interface can be used to perform halo updates for a Field. This
   eases communication programming from a user perspective. By definition, the user
   program only needs to update locally owned exclusive region in each domain, then call
   FieldHalo to communicate the values in the halo region from/to neighboring domain elements.
   In this example, we solve a 1D heat transfer problem: $u_t = \alpha^2 u_{xx}$ with the
   initial condition $u(0, x) = 20$ and boundary conditions $u(t, 0) = 10, u(t, 1) = 40$.
   The temperature field $u$
   is represented by a {\tt ESMF\_Field}. A finite difference explicit time stepping scheme is employed.
   During each time step, FieldHalo update is called to communicate values in the halo region
   to neighboring domain elements. The steady state (as $t \rightarrow \infty$) solution
   is a linear temperature profile along $x$. The numerical solution is an approximation of
   the steady state solution. It can be verified to represent a linear temperature profile.
  
   Section \ref{Array:Halo} provides a discussion of the
   halo operation implemented in {\tt ESMF\_Array}.
   
%/////////////////////////////////////////////////////////////

 \begin{verbatim}
! create 1D distgrid and grid decomposed according to the following diagram:
! +------------+   +----------------+   +---------------+   +--------------+
! |   DE 0  |  |   |  |   DE 1   |  |   |  |   DE 2  |  |   |  |   DE 3    |
! |  1 x 16 |  |   |  |  1 x 16  |  |   |  |  1 x 16 |  |   |  |  1 x 16   |
! |         | 1|<->|1 |          | 1|<->|1 |         | 1|<->|1 |           |
! |         |  |   |  |          |  |   |  |         |  |   |  |           |
! +------------+   +----------------+   +---------------+   +--------------+
    distgrid = ESMF_DistGridCreate(minIndex=(/1/), maxIndex=(/npx/), &
        regDecomp=(/4/), rc=rc)
    if (rc /= ESMF_SUCCESS) call ESMF_Finalize(endflag=ESMF_END_ABORT)

    grid = ESMF_GridCreate(distgrid=distgrid, name="grid", rc=rc)
    if (rc /= ESMF_SUCCESS) call ESMF_Finalize(endflag=ESMF_END_ABORT)

    ! set up initial condition and boundary conditions of the
    ! temperature Field
    if(lpe == 0) then
        allocate(fptr(17), tmp_farray(17))
        fptr = 20.
        fptr(1) = 10.
        tmp_farray(1) = 10.
        startx = 2
        endx = 16

        field = ESMF_FieldCreate(grid, fptr, totalUWidth=(/1/), &
                name="temperature", rc=rc)
        if (rc /= ESMF_SUCCESS) call ESMF_Finalize(endflag=ESMF_END_ABORT)
    else if(lpe == 3) then
        allocate(fptr(17), tmp_farray(17))
        fptr = 20.
        fptr(17) = 40.
        tmp_farray(17) = 40.
        startx = 2
        endx = 16

        field = ESMF_FieldCreate(grid, fptr, totalLWidth=(/1/), &
                name="temperature", rc=rc)
        if (rc /= ESMF_SUCCESS) call ESMF_Finalize(endflag=ESMF_END_ABORT)
    else
        allocate(fptr(18), tmp_farray(18))
        fptr = 20.
        startx = 2
        endx = 17

        field = ESMF_FieldCreate(grid, fptr, &
            totalLWidth=(/1/), totalUWidth=(/1/), name="temperature", rc=rc)
        if (rc /= ESMF_SUCCESS) call ESMF_Finalize(endflag=ESMF_END_ABORT)
    endif

    ! compute the halo update routehandle of the decomposed temperature Field
    call ESMF_FieldHaloStore(field, routehandle=routehandle, rc=rc)
    if (rc /= ESMF_SUCCESS) call ESMF_Finalize(endflag=ESMF_END_ABORT)

    dt = 0.01
    dx = 1./npx
    alpha = 0.1

    ! Employ explicit time stepping
    ! Solution converges after about 9000 steps based on apriori knowledge.
    ! The result is a linear temperature profile stored in field.
    do iter = 1, 9000
     ! only elements in the exclusive region are updated locally
     ! in each domain
     do i = startx, endx
       tmp_farray(i) = &
       fptr(i)+alpha*alpha*dt/dx/dx*(fptr(i+1)-2.*fptr(i)+fptr(i-1))
      enddo
      fptr = tmp_farray
     ! call halo update to communicate the values in the halo region to
     ! neighboring domains
     call ESMF_FieldHalo(field, routehandle=routehandle, rc=rc)
     if (rc /= ESMF_SUCCESS) call ESMF_Finalize(endflag=ESMF_END_ABORT)
    enddo

    ! release the halo routehandle
    call ESMF_FieldHaloRelease(routehandle, rc=rc)
 
\end{verbatim}

%...............................................................
\setlength{\parskip}{\oldparskip}
\setlength{\parindent}{\oldparindent}
\setlength{\baselineskip}{\oldbaselineskip}
