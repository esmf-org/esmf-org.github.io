%                **** IMPORTANT NOTICE *****
% This LaTeX file has been automatically produced by ProTeX v. 1.1
% Any changes made to this file will likely be lost next time
% this file is regenerated from its source. Send questions 
% to Arlindo da Silva, dasilva@gsfc.nasa.gov
 
\setlength{\oldparskip}{\parskip}
\setlength{\parskip}{1.5ex}
\setlength{\oldparindent}{\parindent}
\setlength{\parindent}{0pt}
\setlength{\oldbaselineskip}{\baselineskip}
\setlength{\baselineskip}{11pt}
 
%--------------------- SHORT-HAND MACROS ----------------------
\def\bv{\begin{verbatim}}
\def\ev{\end{verbatim}}
\def\be{\begin{equation}}
\def\ee{\end{equation}}
\def\bea{\begin{eqnarray}}
\def\eea{\end{eqnarray}}
\def\bi{\begin{itemize}}
\def\ei{\end{itemize}}
\def\bn{\begin{enumerate}}
\def\en{\end{enumerate}}
\def\bd{\begin{description}}
\def\ed{\end{description}}
\def\({\left (}
\def\){\right )}
\def\[{\left [}
\def\]{\right ]}
\def\<{\left  \langle}
\def\>{\right \rangle}
\def\cI{{\cal I}}
\def\diag{\mathop{\rm diag}}
\def\tr{\mathop{\rm tr}}
%-------------------------------------------------------------

\markboth{Left}{Source File: ESMF\_FieldGather.F90,  Date: Tue May  5 21:00:01 MDT 2020
}

 
%/////////////////////////////////////////////////////////////
\subsubsection [ESMF\_FieldFill] {ESMF\_FieldFill - Fill data into a Field}


\bigskip{\sf INTERFACE:}
\begin{verbatim}   subroutine ESMF_FieldFill(field, dataFillScheme, &
     const1, member, step, &
     param1I4, param2I4, param3I4, &
     param1R4, param2R4, param3R4, &
     param1R8, param2R8, param3R8, &
     rc)\end{verbatim}{\em ARGUMENTS:}
\begin{verbatim}     type(ESMF_Field), intent(inout) :: field
 -- The following arguments require argument keyword syntax (e.g. rc=rc). --
     character(len=*), intent(in), optional :: dataFillScheme
     real(ESMF_KIND_R8), intent(in), optional :: const1
     integer, intent(in), optional :: member
     integer, intent(in), optional :: step
     integer(ESMF_KIND_I4), intent(in), optional :: param1I4
     integer(ESMF_KIND_I4), intent(in), optional :: param2I4
     integer(ESMF_KIND_I4), intent(in), optional :: param3I4
     real(ESMF_KIND_R4), intent(in), optional :: param1R4
     real(ESMF_KIND_R4), intent(in), optional :: param2R4
     real(ESMF_KIND_R4), intent(in), optional :: param3R4
     real(ESMF_KIND_R8), intent(in), optional :: param1R8
     real(ESMF_KIND_R8), intent(in), optional :: param2R8
     real(ESMF_KIND_R8), intent(in), optional :: param3R8
     integer, intent(out), optional :: rc\end{verbatim}
{\sf DESCRIPTION:\\ }


   \label{ESMF_FieldFill}
   Fill {\tt field} with data according to {\tt dataFillScheme}. Depending
   on the chosen fill scheme, the {\tt member} and {\tt step} arguments are
   used to provide differing fill data patterns.
  
   The arguments are:
   \begin{description}
   \item[field]
   The {\tt ESMF\_Field} object to fill with data.
   \item[{[dataFillScheme]}]
   The fill scheme. The available options are "sincos", "one", and "const".
   Defaults to "sincos".
   \item[{[const1]}]
   Constant of real type. Defaults to 0.
   \item[{[member]}]
   Member incrementor. Defaults to 1.
   \item[{[step]}]
   Step incrementor. Defaults to 1.
   \item[{[param1I4]}]
   Optional parameter of typekind I4.
   The default depends on the specified {\tt dataFillScheme}.
   \item[{[param2I4]}]
   Optional parameter of typekind I4.
   The default depends on the specified {\tt dataFillScheme}.
   \item[{[param3I4]}]
   Optional parameter of typekind I4.
   The default depends on the specified {\tt dataFillScheme}.
   \item[{[param1R4]}]
   Optional parameter of typekind R4.
   The default depends on the specified {\tt dataFillScheme}.
   \item[{[param2R4]}]
   Optional parameter of typekind R4.
   The default depends on the specified {\tt dataFillScheme}.
   \item[{[param3R4]}]
   Optional parameter of typekind R4.
   The default depends on the specified {\tt dataFillScheme}.
   \item[{[param1R8]}]
   Optional parameter of typekind R8.
   The default depends on the specified {\tt dataFillScheme}.
   \item[{[param2R8]}]
   Optional parameter of typekind R8.
   The default depends on the specified {\tt dataFillScheme}.
   \item[{[param3R8]}]
   Optional parameter of typekind R8.
   The default depends on the specified {\tt dataFillScheme}.
   \item[{[rc]}]
   Return code; equals {\tt ESMF\_SUCCESS} if there are no errors.
   \end{description}
   
%/////////////////////////////////////////////////////////////
 
\mbox{}\hrulefill\ 
 

   \subsubsection [ESMF\_FieldGather] {ESMF\_FieldGather - Gather a Fortran array from an ESMF\_Field }


   
\bigskip{\sf INTERFACE:}
\begin{verbatim}   subroutine ESMF_FieldGather<rank><type><kind>(field, farray, & 
   rootPet, tile, vm, rc) 
   \end{verbatim}{\em ARGUMENTS:}
\begin{verbatim}   type(ESMF_Field), intent(in) :: field 
   <type>(ESMF_KIND_<kind>), intent(out), target :: farray(<rank>) 
   integer, intent(in) :: rootPet 
 -- The following arguments require argument keyword syntax (e.g. rc=rc). --
   integer, intent(in), optional :: tile 
   type(ESMF_VM), intent(in), optional :: vm 
   integer, intent(out), optional :: rc 
   
   
   \end{verbatim}
{\sf STATUS:}
   \begin{itemize} 
   \item\apiStatusCompatibleVersion{5.2.0r} 
   \end{itemize} 
   
{\sf DESCRIPTION:\\ }

 
   Gather the data of an {ESMF\_Field} object into the {\tt farray} located on 
   {\tt rootPET}. A single DistGrid tile of {\tt array} must be 
   gathered into {\tt farray}. The optional {\tt tile} 
   argument allows selection of the tile. For Fields defined on a single 
   tile DistGrid the default selection (tile 1) will be correct. The 
   shape of {\tt farray} must match the shape of the tile in Field. 
   
   If the Field contains replicating DistGrid dimensions data will be 
   gathered from the numerically higher DEs. Replicated data elements in 
   numericaly lower DEs will be ignored. 
   
   The implementation of Scatter and Gather is not sequence index based. 
   If the Field is built on arbitrarily distributed Grid, Mesh, LocStream or XGrid, 
   Gather will not gather data to rootPet 
   from source data points corresponding to the sequence index on rootPet. 
   Instead Gather will gather a contiguous memory range from source PET to 
   rootPet. The size of the memory range is equal to the number of 
   data elements on the source PET. Vice versa for the Scatter operation. 
   In this case, the user should use {\tt ESMF\_FieldRedist} to achieve 
   the same data operation result. For examples how to use {\tt ESMF\_FieldRedist} 
   to perform Gather and Scatter, please refer to 
   \ref{sec:field:usage:redist_gathering} and 
   \ref{sec:field:usage:redist_scattering}. 
   
   This version of the interface implements the PET-based blocking paradigm: 
   Each PET of the VM must issue this call exactly once for {\em all} of its 
   DEs. The call will block until all PET-local data objects are accessible. 
   
   For examples and associated documentation regarding this method see Section 
   \ref{sec:field:usage:gather_2dptr}. 
   
   The arguments are: 
   \begin{description} 
   \item[field] 
   The {\tt ESMF\_Field} object from which data will be gathered. 
   \item[\{farray\}] 
   The Fortran array into which to gather data. Only root 
   must provide a valid {\tt farray}, the other PETs may treat 
   {\tt farray} as an optional argument. 
   \item[rootPet] 
   PET that holds the valid destination array, i.e. {\tt farray}. 
   \item[{[tile]}] 
   The DistGrid tile in {\tt field} from which to gather {\tt farray}. 
   By default {\tt farray} will be gathered from tile 1. 
   \item[{[vm]}] 
   Optional {\tt ESMF\_VM} object of the current context. Providing the 
   VM of the current context will lower the method's overhead. 
   \item[{[rc]}] 
   Return code; equals {\tt ESMF\_SUCCESS} if there are no errors. 
   \end{description} 
   
%...............................................................
\setlength{\parskip}{\oldparskip}
\setlength{\parindent}{\oldparindent}
\setlength{\baselineskip}{\oldbaselineskip}
