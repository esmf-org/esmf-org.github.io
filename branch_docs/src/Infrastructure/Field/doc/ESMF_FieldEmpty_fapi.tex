%                **** IMPORTANT NOTICE *****
% This LaTeX file has been automatically produced by ProTeX v. 1.1
% Any changes made to this file will likely be lost next time
% this file is regenerated from its source. Send questions 
% to Arlindo da Silva, dasilva@gsfc.nasa.gov
 
\setlength{\oldparskip}{\parskip}
\setlength{\parskip}{1.5ex}
\setlength{\oldparindent}{\parindent}
\setlength{\parindent}{0pt}
\setlength{\oldbaselineskip}{\baselineskip}
\setlength{\baselineskip}{11pt}
 
%--------------------- SHORT-HAND MACROS ----------------------
\def\bv{\begin{verbatim}}
\def\ev{\end{verbatim}}
\def\be{\begin{equation}}
\def\ee{\end{equation}}
\def\bea{\begin{eqnarray}}
\def\eea{\end{eqnarray}}
\def\bi{\begin{itemize}}
\def\ei{\end{itemize}}
\def\bn{\begin{enumerate}}
\def\en{\end{enumerate}}
\def\bd{\begin{description}}
\def\ed{\end{description}}
\def\({\left (}
\def\){\right )}
\def\[{\left [}
\def\]{\right ]}
\def\<{\left  \langle}
\def\>{\right \rangle}
\def\cI{{\cal I}}
\def\diag{\mathop{\rm diag}}
\def\tr{\mathop{\rm tr}}
%-------------------------------------------------------------

\markboth{Left}{Source File: ESMF\_FieldEmpty.F90,  Date: Tue May  5 21:00:01 MDT 2020
}

 
%/////////////////////////////////////////////////////////////
\subsubsection [ESMF\_FieldEmptyComplete] {ESMF\_FieldEmptyComplete - Complete a Field from arrayspec}


\bigskip{\sf INTERFACE:}
\begin{verbatim}   ! Private name; call using ESMF_FieldEmptyComplete()
 subroutine ESMF_FieldEmptyCompAS(field, arrayspec, &
  indexflag, gridToFieldMap, &
  ungriddedLBound, ungriddedUBound, totalLWidth, totalUWidth, rc)\end{verbatim}{\em ARGUMENTS:}
\begin{verbatim}  type(ESMF_Field), intent(inout) :: field
  type(ESMF_ArraySpec), intent(in) :: arrayspec
 -- The following arguments require argument keyword syntax (e.g. rc=rc). --
  type(ESMF_Index_Flag), intent(in), optional :: indexflag
  integer, intent(in), optional :: gridToFieldMap(:)
  integer, intent(in), optional :: ungriddedLBound(:)
  integer, intent(in), optional :: ungriddedUBound(:)
  integer, intent(in), optional :: totalLWidth(:)
  integer, intent(in), optional :: totalUWidth(:)
  integer, intent(out), optional :: rc\end{verbatim}
{\sf STATUS:}
   \begin{itemize}
   \item\apiStatusCompatibleVersion{5.2.0r}
   \end{itemize}
  
{\sf DESCRIPTION:\\ }


   Complete an {\tt ESMF\_Field} and allocate space internally for an
   {\tt ESMF\_Array} based on arrayspec.
   The input {\tt ESMF\_Field} must have a status of
   {\tt ESMF\_FIELDSTATUS\_GRIDSET}. After this call the completed {\tt ESMF\_Field}
   has a status of {\tt ESMF\_FIELDSTATUS\_COMPLETE}.
  
   The arguments are:
   \begin{description}
   \item[field]
   The input {\tt ESMF\_Field} with a status of
   {\tt ESMF\_FIELDSTATUS\_GRIDSET}.
   \item[arrayspec]
   Data type and kind specification.
   \item[{[indexflag]}]
   Indicate how DE-local indices are defined. By default each DE's
   exclusive region is placed to start at the local index space origin,
   i.e. (1, 1, ..., 1). Alternatively the DE-local index space can be
   aligned with the global index space, if a global index space is well
   defined by the associated Grid. See section \ref{const:indexflag}
   for a list of valid indexflag options.
   \item [{[gridToFieldMap]}]
   List with number of elements equal to the
   {\tt grid}'s dimCount. The list elements map each dimension
   of the {\tt grid} to a dimension in the {\tt field} by
   specifying the appropriate {\tt field} dimension index. The default is to
   map all of the {\tt grid}'s dimensions against the lowest dimensions of
   the {\tt field} in sequence, i.e. {\tt gridToFieldMap} = (/1,2,3,.../).
   The values of all {\tt gridToFieldMap} entries must be greater than or equal
   to one and smaller than or equal to the {\tt field} rank.
   It is erroneous to specify the same {\tt gridToFieldMap} entry
   multiple times. The total ungridded dimensions in the {\tt field}
   are the total {\tt field} dimensions less
   the dimensions in
   the {\tt grid}. Ungridded dimensions must be in the same order they are
   stored in the {\t field}.
   If the Field dimCount is less than the Grid dimCount then the default
   gridToFieldMap will contain zeros for the rightmost entries. A zero
   entry in the {\tt gridToFieldMap} indicates that the particular
   Grid dimension will be replicating the Field across the DEs along
   this direction.
   \item [{[ungriddedLBound]}]
   Lower bounds of the ungridded dimensions of the {\tt field}.
   The number of elements in the {\tt ungriddedLBound} is equal to the number of ungridded
   dimensions in the {\tt field}. All ungridded dimensions of the
   {\tt field} are also undistributed. When field dimension count is
   greater than grid dimension count, both ungriddedLBound and ungriddedUBound
   must be specified. When both are specified the values are checked
   for consistency. Note that the the ordering of
   these ungridded dimensions is the same as their order in the {\tt field}.
   \item [{[ungriddedUBound]}]
   Upper bounds of the ungridded dimensions of the {\tt field}.
   The number of elements in the {\tt ungriddedUBound} is equal to the number of ungridded
   dimensions in the {\tt field}. All ungridded dimensions of the
   {\tt field} are also undistributed. When field dimension count is
   greater than grid dimension count, both ungriddedLBound and ungriddedUBound
   must be specified. When both are specified the values are checked
   for consistency. Note that the the ordering of
   these ungridded dimensions is the same as their order in the {\tt field}.
   \item [{[totalLWidth]}]
   Lower bound of halo region. The size of this array is the number
   of gridded dimensions in the Field. However, ordering of the elements
   needs to be the same as they appear in the {\tt field}. Values default
   to 0. If values for totalLWidth are specified they must be reflected in
   the size of the {\tt field}. That is, for each gridded dimension the
   {\tt field} size should be max( {\tt totalLWidth} + {\tt totalUWidth}
   + {\tt computationalCount}, {\tt exclusiveCount} ).
   \item [{[totalUWidth]}]
   Upper bound of halo region. The size of this array is the number
   of gridded dimensions in the Field. However, ordering of the elements
   needs to be the same as they appear in the {\tt field}. Values default
   to 0. If values for totalUWidth are specified they must be reflected in
   the size of the {\tt field}. That is, for each gridded dimension the
   {\tt field} size should max( {\tt totalLWidth} + {\tt totalUWidth}
   + {\tt computationalCount}, {\tt exclusiveCount} ).
   \item [{[rc]}]
   Return code; equals {\tt ESMF\_SUCCESS} if there are no errors.
   \end{description} 
%/////////////////////////////////////////////////////////////
 
\mbox{}\hrulefill\ 
 
\subsubsection [ESMF\_FieldEmptyComplete] {ESMF\_FieldEmptyComplete - Complete a Field from typekind}


\bigskip{\sf INTERFACE:}
\begin{verbatim}   ! Private name; call using ESMF_FieldEmptyComplete()
 subroutine ESMF_FieldEmptyCompTK(field, typekind, &
  indexflag, gridToFieldMap, &
  ungriddedLBound, ungriddedUBound, totalLWidth, totalUWidth, rc)\end{verbatim}{\em ARGUMENTS:}
\begin{verbatim}  type(ESMF_Field), intent(inout) :: field
  type(ESMF_TypeKind_Flag), intent(in) :: typekind
 -- The following arguments require argument keyword syntax (e.g. rc=rc). --
  type(ESMF_Index_Flag), intent(in), optional :: indexflag
  integer, intent(in), optional :: gridToFieldMap(:)
  integer, intent(in), optional :: ungriddedLBound(:)
  integer, intent(in), optional :: ungriddedUBound(:)
  integer, intent(in), optional :: totalLWidth(:)
  integer, intent(in), optional :: totalUWidth(:)
  integer, intent(out), optional :: rc\end{verbatim}
{\sf STATUS:}
   \begin{itemize}
   \item\apiStatusCompatibleVersion{5.2.0r}
   \end{itemize}
  
{\sf DESCRIPTION:\\ }


   \begin{sloppypar}
   Complete an {\tt ESMF\_Field} and allocate space internally for an
   {\tt ESMF\_Array} based on typekind.
   The input {\tt ESMF\_Field} must have a status of
   {\tt ESMF\_FIELDSTATUS\_GRIDSET}. After this call the completed {\tt ESMF\_Field}
   has a status of {\tt ESMF\_FIELDSTATUS\_COMPLETE}.
   \end{sloppypar}
  
   For an example and
   associated documentation using this method see section
   \ref{sec:field:usage:partial_creation}.
  
  
   The arguments are:
   \begin{description}
   \item[field]
   \begin{sloppypar}
   The input {\tt ESMF\_Field} with a status of
   {\tt ESMF\_FIELDSTATUS\_GRIDSET}.
   \end{sloppypar}
   \item[typekind]
   Data type and kind specification.
   \item[{[indexflag]}]
   Indicate how DE-local indices are defined. By default each DE's
   exclusive region is placed to start at the local index space origin,
   i.e. (1, 1, ..., 1). Alternatively the DE-local index space can be
   aligned with the global index space, if a global index space is well
   defined by the associated Grid. See section \ref{const:indexflag}
   for a list of valid indexflag options.
   \item [{[gridToFieldMap]}]
   List with number of elements equal to the
   {\tt grid}'s dimCount. The list elements map each dimension
   of the {\tt grid} to a dimension in the {\tt field} by
   specifying the appropriate {\tt field} dimension index. The default is to
   map all of the {\tt grid}'s dimensions against the lowest dimensions of
   the {\tt field} in sequence, i.e. {\tt gridToFieldMap} = (/1,2,3,.../).
   The values of all {\tt gridToFieldMap} entries must be greater than or equal
   to one and smaller than or equal to the {\tt field} rank.
   It is erroneous to specify the same {\tt gridToFieldMap} entry
   multiple times. The total ungridded dimensions in the {\tt field}
   are the total {\tt field} dimensions less
   the dimensions in
   the {\tt grid}. Ungridded dimensions must be in the same order they are
   stored in the {\t field}.
   If the Field dimCount is less than the Grid dimCount then the default
   gridToFieldMap will contain zeros for the rightmost entries. A zero
   entry in the {\tt gridToFieldMap} indicates that the particular
   Grid dimension will be replicating the Field across the DEs along
   this direction.
   \item [{[ungriddedLBound]}]
   Lower bounds of the ungridded dimensions of the {\tt field}.
   The number of elements in the {\tt ungriddedLBound} is equal to the number of ungridded
   dimensions in the {\tt field}. All ungridded dimensions of the
   {\tt field} are also undistributed. When field dimension count is
   greater than grid dimension count, both ungriddedLBound and ungriddedUBound
   must be specified. When both are specified the values are checked
   for consistency. Note that the the ordering of
   these ungridded dimensions is the same as their order in the {\tt field}.
   \item [{[ungriddedUBound]}]
   Upper bounds of the ungridded dimensions of the {\tt field}.
   The number of elements in the {\tt ungriddedUBound} is equal to the number of ungridded
   dimensions in the {\tt field}. All ungridded dimensions of the
   {\tt field} are also undistributed. When field dimension count is
   greater than grid dimension count, both ungriddedLBound and ungriddedUBound
   must be specified. When both are specified the values are checked
   for consistency. Note that the the ordering of
   these ungridded dimensions is the same as their order in the {\tt field}.
   \item [{[totalLWidth]}]
   Lower bound of halo region. The size of this array is the number
   of gridded dimensions in the Field. However, ordering of the elements
   needs to be the same as they appear in the {\tt field}. Values default
   to 0. If values for totalLWidth are specified they must be reflected in
   the size of the {\tt field}. That is, for each gridded dimension the
   {\tt field} size should be max( {\tt totalLWidth} + {\tt totalUWidth}
   + {\tt computationalCount}, {\tt exclusiveCount} ).
   \item [{[totalUWidth]}]
   Upper bound of halo region. The size of this array is the number
   of gridded dimensions in the Field. However, ordering of the elements
   needs to be the same as they appear in the {\tt field}. Values default
   to 0. If values for totalUWidth are specified they must be reflected in
   the size of the {\tt field}. That is, for each gridded dimension the
   {\tt field} size should max( {\tt totalLWidth} + {\tt totalUWidth}
   + {\tt computationalCount}, {\tt exclusiveCount} ).
   \item [{[rc]}]
   Return code; equals {\tt ESMF\_SUCCESS} if there are no errors.
   \end{description} 
%/////////////////////////////////////////////////////////////
 
\mbox{}\hrulefill\ 
 
\subsubsection [ESMF\_FieldEmptyComplete] {ESMF\_FieldEmptyComplete - Complete a Field from Fortran array }


   
\bigskip{\sf INTERFACE:}
\begin{verbatim}   ! Private name; call using ESMF_FieldEmptyComplete() 
   subroutine ESMF_FieldEmptyComp<rank><type><kind>(field, & 
   farray, indexflag, datacopyflag, gridToFieldMap, & 
   ungriddedLBound, ungriddedUBound, totalLWidth, totalUWidth, rc) 
   \end{verbatim}{\em ARGUMENTS:}
\begin{verbatim}   type(ESMF_Field), intent(inout) :: field 
   <type> (ESMF_KIND_<kind>),intent(in), target :: farray(<rank>) 
   type(ESMF_Index_Flag), intent(in) :: indexflag 
 -- The following arguments require argument keyword syntax (e.g. rc=rc). --
   type(ESMF_DataCopy_Flag), intent(in), optional :: datacopyflag 
   integer, intent(in), optional :: gridToFieldMap(:) 
   integer, intent(in), optional :: ungriddedLBound(:) 
   integer, intent(in), optional :: ungriddedUBound(:) 
   integer, intent(in), optional :: totalLWidth(:) 
   integer, intent(in), optional :: totalUWidth(:) 
   integer, intent(out), optional :: rc 
   
   \end{verbatim}
{\sf STATUS:}
   \begin{itemize} 
   \item\apiStatusCompatibleVersion{5.2.0r} 
   \end{itemize} 
   
{\sf DESCRIPTION:\\ }

 
   \begin{sloppypar} 
   Complete an {\tt ESMF\_Field} and allocate space internally for an 
   {\tt ESMF\_Array} based on typekind. 
   The input {\tt ESMF\_Field} must have a status of 
   {\tt ESMF\_FIELDSTATUS\_GRIDSET}. After this call the completed {\tt ESMF\_Field} 
   has a status of {\tt ESMF\_FIELDSTATUS\_COMPLETE}. 
   \end{sloppypar} 
   
   The Fortran data pointer inside {\tt ESMF\_Field} can be queried but deallocating 
   the retrieved data pointer is not allowed. 
   
   For an example and 
   associated documentation using this method see section 
   \ref{sec:field:usage:create_empty}. 
   
   
   The arguments are: 
   \begin{description} 
   \item [field] 
   The input {\tt ESMF\_Field} with a status of 
   {\tt ESMF\_FIELDSTATUS\_GRIDSET}. 
   The {\tt ESMF\_Field} will have the same dimension 
   (dimCount) as the rank of the {\tt farray}. 
   \item [farray] 
   Native Fortran data array to be copied/referenced in the {\tt field}. 
   The {\tt field} dimension (dimCount) will be the same as the dimCount 
   for the {\tt farray}. 
   \item [indexflag] 
   Indicate how DE-local indices are defined. See section 
   \ref{const:indexflag} for a list of valid indexflag options. 
   \item [{[datacopyflag]}] 
   Indicates whether to copy the {\tt farray} or reference it directly. 
   For valid values see \ref{const:datacopyflag}. The default is 
   {\tt ESMF\_DATACOPY\_REFERENCE}. 
   \item [{[gridToFieldMap]}] 
   List with number of elements equal to the 
   {\tt grid}'s dimCount. The list elements map each dimension 
   of the {\tt grid} to a dimension in the {\tt farray} by 
   specifying the appropriate {\tt farray} dimension index. The 
   default is to map all of the {\tt grid}'s dimensions against the 
   lowest dimensions of the {\tt farray} in sequence, i.e. 
   {\tt gridToFieldMap} = (/1,2,3,.../). 
   Unmapped {\tt farray} dimensions are undistributed Field 
   dimensions. 
   All {\tt gridToFieldMap} entries must be greater than or equal 
   to zero and smaller than or equal to the Field dimCount. It is erroneous 
   to specify the same entry multiple times unless it is zero. 
   If the Field dimCount is less than the Grid dimCount then the default 
   gridToFieldMap will contain zeros for the rightmost entries. A zero 
   entry in the {\tt gridToFieldMap} indicates that the particular 
   Grid dimension will be replicating the Field across the DEs along 
   this direction. 
   \item [{[ungriddedLBound]}] 
   Lower bounds of the ungridded dimensions of the {\tt field}. 
   The number of elements in the {\tt ungriddedLBound} is equal to the number of ungridded 
   dimensions in the {\tt field}. All ungridded dimensions of the 
   {\tt field} are also undistributed. When field dimension count is 
   greater than grid dimension count, both ungriddedLBound and ungriddedUBound 
   must be specified. When both are specified the values are checked 
   for consistency. Note that the the ordering of 
   these ungridded dimensions is the same as their order in the {\tt field}. 
   \item [{[ungriddedUBound]}] 
   Upper bounds of the ungridded dimensions of the {\tt field}. 
   The number of elements in the {\tt ungriddedUBound} is equal to the number of ungridded 
   dimensions in the {\tt field}. All ungridded dimensions of the 
   {\tt field} are also undistributed. When field dimension count is 
   greater than grid dimension count, both ungriddedLBound and ungriddedUBound 
   must be specified. When both are specified the values are checked 
   for consistency. Note that the the ordering of 
   these ungridded dimensions is the same as their order in the {\tt field}. 
   \item [{[totalLWidth]}] 
   Lower bound of halo region. The size of this array is the number 
   of gridded dimensions in the {\tt field}. However, ordering of the elements 
   needs to be the same as they appear in the {\tt field}. Values default 
   to 0. If values for totalLWidth are specified they must be reflected in 
   the size of the {\tt field}. That is, for each gridded dimension the 
   {\tt field} size should be max( {\tt totalLWidth} + {\tt totalUWidth} 
   + {\tt computationalCount}, {\tt exclusiveCount} ). 
   \item [{[totalUWidth]}] 
   Upper bound of halo region. The size of this array is the number 
   of gridded dimensions in the {\tt field}. However, ordering of the elements 
   needs to be the same as they appear in the {\tt field}. Values default 
   to 0. If values for totalUWidth are specified they must be reflected in 
   the size of the {\tt field}. That is, for each gridded dimension the 
   {\tt field} size should max( {\tt totalLWidth} + {\tt totalUWidth} 
   + {\tt computationalCount}, {\tt exclusiveCount} ). 
   \item [{[rc]}] 
   Return code; equals {\tt ESMF\_SUCCESS} if there are no errors. 
   \end{description} 
    
%/////////////////////////////////////////////////////////////
 
\mbox{}\hrulefill\ 
 
\subsubsection [ESMF\_FieldEmptyComplete] {ESMF\_FieldEmptyComplete - Complete a Field from Fortran array pointer }


   
\bigskip{\sf INTERFACE:}
\begin{verbatim}   ! Private name; call using ESMF_FieldEmptyComplete() 
   subroutine ESMF_FieldEmptyCompPtr<rank><type><kind>(field, & 
   farrayPtr, datacopyflag, gridToFieldMap, & 
   totalLWidth, totalUWidth, rc) 
   \end{verbatim}{\em ARGUMENTS:}
\begin{verbatim}   type(ESMF_Field), intent(inout) :: field 
   <type> (ESMF_KIND_<kind>), pointer :: farrayPtr(<rank>) 
 -- The following arguments require argument keyword syntax (e.g. rc=rc). --
   type(ESMF_DataCopy_Flag), intent(in), optional :: datacopyflag 
   integer, intent(in), optional :: gridToFieldMap(:) 
   integer, intent(in), optional :: totalLWidth(:) 
   integer, intent(in), optional :: totalUWidth(:) 
   integer, intent(out), optional :: rc 
   
   \end{verbatim}
{\sf STATUS:}
   \begin{itemize} 
   \item\apiStatusCompatibleVersion{5.2.0r} 
   \end{itemize} 
   
{\sf DESCRIPTION:\\ }

 
   \begin{sloppypar} 
   Complete an {\tt ESMF\_Field} and allocate space internally for an 
   {\tt ESMF\_Array} based on typekind. 
   The input {\tt ESMF\_Field} must have a status of 
   {\tt ESMF\_FIELDSTATUS\_GRIDSET}. After this call the completed {\tt ESMF\_Field} 
   has a status of {\tt ESMF\_FIELDSTATUS\_COMPLETE}. 
   
   The Fortran data pointer inside {\tt ESMF\_Field} can be queried and deallocated when 
   datacopyflag is {\tt ESMF\_DATACOPY\_REFERENCE}. Note that the {\tt ESMF\_FieldDestroy} call does not deallocate 
   the Fortran data pointer in this case. This gives user more flexibility over memory management. 
   \end{sloppypar} 
   
   The arguments are: 
   \begin{description} 
   \item [field] 
   The input {\tt ESMF\_Field} with a status of 
   {\tt ESMF\_FIELDSTATUS\_GRIDSET}. 
   The {\tt ESMF\_Field} will have the same dimension 
   (dimCount) as the rank of the {\tt farrayPtr}. 
   \item [farrayPtr] 
   Native Fortran data pointer to be copied/referenced in the {\tt field}. 
   The {\tt field} dimension (dimCount) will be the same as the dimCount 
   for the {\tt farrayPtr}. 
   \item [{[datacopyflag]}] 
   Indicates whether to copy the {\tt farrayPtr} or reference it directly. 
   For valid values see \ref{const:datacopyflag}. The default is 
   {\tt ESMF\_DATACOPY\_REFERENCE}. 
   \item [{[gridToFieldMap]}] 
   List with number of elements equal to the 
   {\tt grid}'s dimCount. The list elements map each dimension 
   of the {\tt grid} to a dimension in the {\tt farrayPtr} by 
   specifying the appropriate {\tt farrayPtr} dimension index. The 
   default is to map all of the {\tt grid}'s dimensions against the 
   lowest dimensions of the {\tt farrayPtr} in sequence, i.e. 
   {\tt gridToFieldMap} = (/1,2,3,.../). 
   Unmapped {\tt farrayPtr} dimensions are undistributed Field 
   dimensions. 
   All {\tt gridToFieldMap} entries must be greater than or equal 
   to zero and smaller than or equal to the Field dimCount. It is erroneous 
   to specify the same entry multiple times unless it is zero. 
   If the Field dimCount is less than the Grid dimCount then the default 
   gridToFieldMap will contain zeros for the rightmost entries. A zero 
   entry in the {\tt gridToFieldMap} indicates that the particular 
   Grid dimension will be replicating the Field across the DEs along 
   this direction. 
   \item [{[totalLWidth]}] 
   Lower bound of halo region. The size of this array is the number 
   of gridded dimensions in the {\tt field}. However, ordering of the elements 
   needs to be the same as they appear in the {\tt field}. Values default 
   to 0. If values for totalLWidth are specified they must be reflected in 
   the size of the {\tt field}. That is, for each gridded dimension the 
   {\tt field} size should be max( {\tt totalLWidth} + {\tt totalUWidth} 
   + {\tt computationalCount}, {\tt exclusiveCount} ). 
   \item [{[totalUWidth]}] 
   Upper bound of halo region. The size of this array is the number 
   of gridded dimensions in the {\tt field}. However, ordering of the elements 
   needs to be the same as they appear in the {\tt field}. Values default 
   to 0. If values for totalUWidth are specified they must be reflected in 
   the size of the {\tt field}. That is, for each gridded dimension the 
   {\tt field} size should max( {\tt totalLWidth} + {\tt totalUWidth} 
   + {\tt computationalCount}, {\tt exclusiveCount} ). 
   \item [{[rc]}] 
   Return code; equals {\tt ESMF\_SUCCESS} if there are no errors. 
   \end{description} 
    
%/////////////////////////////////////////////////////////////
 
\mbox{}\hrulefill\ 
 
\subsubsection [ESMF\_FieldEmptyComplete] {ESMF\_FieldEmptyComplete - Complete a Field from Grid started with FieldEmptyCreate }


   
\bigskip{\sf INTERFACE:}
\begin{verbatim}   ! Private name; call using ESMF_FieldEmptyComplete() 
   subroutine ESMF_FieldEmptyCompGrid<rank><type><kind>(field, grid, & 
   farray, indexflag, datacopyflag, staggerloc, gridToFieldMap, & 
   ungriddedLBound, ungriddedUBound, totalLWidth, totalUWidth, rc) 
   \end{verbatim}{\em ARGUMENTS:}
\begin{verbatim}   type(ESMF_Field), intent(inout) :: field 
   type(ESMF_Grid), intent(in) :: grid 
   <type> (ESMF_KIND_<kind>),intent(in), target :: farray(<rank>) 
   type(ESMF_Index_Flag), intent(in) :: indexflag 
 -- The following arguments require argument keyword syntax (e.g. rc=rc). --
   type(ESMF_DataCopy_Flag), intent(in), optional :: datacopyflag 
   type(ESMF_STAGGERLOC), intent(in), optional :: staggerloc 
   integer, intent(in), optional :: gridToFieldMap(:) 
   integer, intent(in), optional :: ungriddedLBound(:) 
   integer, intent(in), optional :: ungriddedUBound(:) 
   integer, intent(in), optional :: totalLWidth(:) 
   integer, intent(in), optional :: totalUWidth(:) 
   integer, intent(out), optional :: rc 
   \end{verbatim}
{\sf STATUS:}
   \begin{itemize} 
   \item\apiStatusCompatibleVersion{5.2.0r} 
   \end{itemize} 
   
{\sf DESCRIPTION:\\ }

 
   This call completes an {\tt ESMF\_Field} allocated with the 
   {\tt ESMF\_FieldEmptyCreate()} call. 
   
   The Fortran data pointer inside {\tt ESMF\_Field} can be queried but deallocating 
   the retrieved data pointer is not allowed. 
   
   The arguments are: 
   \begin{description} 
   \item [field] 
   The {\tt ESMF\_Field} object to be completed and 
   committed in this call. The {\tt field} will have the same dimension 
   (dimCount) as the rank of the {\tt farray}. 
   \item [grid] 
   The {\tt ESMF\_Grid} object to complete the Field. 
   \item [farray] 
   Native Fortran data array to be copied/referenced in the {\tt field}. 
   The {\tt field} dimension (dimCount) will be the same as the dimCount 
   for the {\tt farray}. 
   \item [indexflag] 
   Indicate how DE-local indices are defined. See section 
   \ref{const:indexflag} for a list of valid indexflag options. 
   \item [{[datacopyflag]}] 
   Indicates whether to copy the {\tt farray} or reference it directly. 
   For valid values see \ref{const:datacopyflag}. The default is 
   {\tt ESMF\_DATACOPY\_REFERENCE}. 
   \item [{[staggerloc]}] 
   Stagger location of data in grid cells. For valid 
   predefined values see section \ref{const:staggerloc}. 
   To create a custom stagger location see section 
   \ref{sec:usage:staggerloc:adv}. The default 
   value is {\tt ESMF\_STAGGERLOC\_CENTER}. 
   \item [{[gridToFieldMap]}] 
   List with number of elements equal to the 
   {\tt grid}'s dimCount. The list elements map each dimension 
   of the {\tt grid} to a dimension in the {\tt farray} by 
   specifying the appropriate {\tt farray} dimension index. The 
   default is to map all of the {\tt grid}'s dimensions against the 
   lowest dimensions of the {\tt farray} in sequence, i.e. 
   {\tt gridToFieldMap} = (/1,2,3,.../). 
   Unmapped {\tt farray} dimensions are undistributed Field 
   dimensions. 
   All {\tt gridToFieldMap} entries must be greater than or equal 
   to zero and smaller than or equal to the Field dimCount. It is erroneous 
   to specify the same entry multiple times unless it is zero. 
   If the Field dimCount is less than the Grid dimCount then the default 
   gridToFieldMap will contain zeros for the rightmost entries. A zero 
   entry in the {\tt gridToFieldMap} indicates that the particular 
   Grid dimension will be replicating the Field across the DEs along 
   this direction. 
   \item [{[ungriddedLBound]}] 
   Lower bounds of the ungridded dimensions of the {\tt field}. 
   The number of elements in the {\tt ungriddedLBound} is equal to the number of ungridded 
   dimensions in the {\tt field}. All ungridded dimensions of the 
   {\tt field} are also undistributed. When field dimension count is 
   greater than grid dimension count, both ungriddedLBound and ungriddedUBound 
   must be specified. When both are specified the values are checked 
   for consistency. Note that the the ordering of 
   these ungridded dimensions is the same as their order in the {\tt field}. 
   \item [{[ungriddedUBound]}] 
   Upper bounds of the ungridded dimensions of the {\tt field}. 
   The number of elements in the {\tt ungriddedUBound} is equal to the number of ungridded 
   dimensions in the {\tt field}. All ungridded dimensions of the 
   {\tt field} are also undistributed. When field dimension count is 
   greater than grid dimension count, both ungriddedLBound and ungriddedUBound 
   must be specified. When both are specified the values are checked 
   for consistency. Note that the the ordering of 
   these ungridded dimensions is the same as their order in the {\tt field}. 
   \item [{[totalLWidth]}] 
   Lower bound of halo region. The size of this array is the number 
   of gridded dimensions in the {\tt field}. However, ordering of the elements 
   needs to be the same as they appear in the {\tt field}. Values default 
   to 0. If values for totalLWidth are specified they must be reflected in 
   the size of the {\tt field}. That is, for each gridded dimension the 
   {\tt field} size should be max( {\tt totalLWidth} + {\tt totalUWidth} 
   + {\tt computationalCount}, {\tt exclusiveCount} ). 
   \item [{[totalUWidth]}] 
   Upper bound of halo region. The size of this array is the number 
   of gridded dimensions in the {\tt field}. However, ordering of the elements 
   needs to be the same as they appear in the {\tt field}. Values default 
   to 0. If values for totalUWidth are specified they must be reflected in 
   the size of the {\tt field}. That is, for each gridded dimension the 
   {\tt field} size should max( {\tt totalLWidth} + {\tt totalUWidth} 
   + {\tt computationalCount}, {\tt exclusiveCount} ). 
   \item [{[rc]}] 
   Return code; equals {\tt ESMF\_SUCCESS} if there are no errors. 
   \end{description} 
    
%/////////////////////////////////////////////////////////////
 
\mbox{}\hrulefill\ 
 
\subsubsection [ESMF\_FieldEmptyComplete] {ESMF\_FieldEmptyComplete - Complete a Field from Grid started with FieldEmptyCreate }


   
\bigskip{\sf INTERFACE:}
\begin{verbatim}   ! Private name; call using ESMF_FieldEmptyComplete() 
   subroutine ESMF_FieldEmptyCompGridPtr<rank><type><kind>(field, grid, & 
   farrayPtr, datacopyflag, staggerloc, gridToFieldMap, & 
   totalLWidth, totalUWidth, rc) 
   \end{verbatim}{\em ARGUMENTS:}
\begin{verbatim}   type(ESMF_Field), intent(inout) :: field 
   type(ESMF_Grid), intent(in) :: grid 
   <type> (ESMF_KIND_<kind>), pointer :: farrayPtr(<rank>) 
 -- The following arguments require argument keyword syntax (e.g. rc=rc). --
   type(ESMF_DataCopy_Flag), intent(in), optional :: datacopyflag 
   type(ESMF_STAGGERLOC), intent(in), optional :: staggerloc 
   integer, intent(in), optional :: gridToFieldMap(:) 
   integer, intent(in), optional :: totalLWidth(:) 
   integer, intent(in), optional :: totalUWidth(:) 
   integer, intent(out), optional :: rc 
   \end{verbatim}
{\sf STATUS:}
   \begin{itemize} 
   \item\apiStatusCompatibleVersion{5.2.0r} 
   \end{itemize} 
   
{\sf DESCRIPTION:\\ }

 
   This call completes an {\tt ESMF\_Field} allocated with the 
   {\tt ESMF\_FieldEmptyCreate()} call. 
   
   \begin{sloppypar} 
   The Fortran data pointer inside {\tt ESMF\_Field} can be queried and deallocated when 
   datacopyflag is {\tt ESMF\_DATACOPY\_REFERENCE}. Note that the {\tt ESMF\_FieldDestroy} call does not deallocate 
   the Fortran data pointer in this case. This gives user more flexibility over memory management. 
   \end{sloppypar} 
   The Fortran data pointer inside {\tt ESMF\_Field} can be queried and deallocated when 
   
   The arguments are: 
   \begin{description} 
   \item [field] 
   The {\tt ESMF\_Field} object to be completed and 
   committed in this call. The {\tt field} will have the same dimension 
   (dimCount) as the rank of the {\tt farrayPtr}. 
   \item [grid] 
   The {\tt ESMF\_Grid} object to complete the Field. 
   \item [farrayPtr] 
   Native Fortran data pointer to be copied/referenced in the {\tt field}. 
   The {\tt field} dimension (dimCount) will be the same as the dimCount 
   for the {\tt farrayPtr}. 
   \item [{[datacopyflag]}] 
   Indicates whether to copy the {\tt farrayPtr} or reference it directly. 
   For valid values see \ref{const:datacopyflag}. The default is 
   {\tt ESMF\_DATACOPY\_REFERENCE}. 
   \item [{[staggerloc]}] 
   Stagger location of data in grid cells. For valid 
   predefined values see section \ref{const:staggerloc}. 
   To create a custom stagger location see section 
   \ref{sec:usage:staggerloc:adv}. The default 
   value is {\tt ESMF\_STAGGERLOC\_CENTER}. 
   \item [{[gridToFieldMap]}] 
   List with number of elements equal to the 
   {\tt grid}'s dimCount. The list elements map each dimension 
   of the {\tt grid} to a dimension in the {\tt farrayPtr} by 
   specifying the appropriate {\tt farrayPtr} dimension index. The 
   default is to map all of the {\tt grid}'s dimensions against the 
   lowest dimensions of the {\tt farrayPtr} in sequence, i.e. 
   {\tt gridToFieldMap} = (/1,2,3,.../). 
   Unmapped {\tt farrayPtr} dimensions are undistributed Field 
   dimensions. 
   All {\tt gridToFieldMap} entries must be greater than or equal 
   to zero and smaller than or equal to the Field dimCount. It is erroneous 
   to specify the same entry multiple times unless it is zero. 
   If the Field dimCount is less than the Grid dimCount then the default 
   gridToFieldMap will contain zeros for the rightmost entries. A zero 
   entry in the {\tt gridToFieldMap} indicates that the particular 
   Grid dimension will be replicating the Field across the DEs along 
   this direction. 
   \item [{[totalLWidth]}] 
   Lower bound of halo region. The size of this array is the number 
   of gridded dimensions in the {\tt field}. However, ordering of the elements 
   needs to be the same as they appear in the {\tt field}. Values default 
   to 0. If values for totalLWidth are specified they must be reflected in 
   the size of the {\tt field}. That is, for each gridded dimension the 
   {\tt field} size should be max( {\tt totalLWidth} + {\tt totalUWidth} 
   + {\tt computationalCount}, {\tt exclusiveCount} ). 
   \item [{[totalUWidth]}] 
   Upper bound of halo region. The size of this array is the number 
   of gridded dimensions in the {\tt field}. However, ordering of the elements 
   needs to be the same as they appear in the {\tt field}. Values default 
   to 0. If values for totalUWidth are specified they must be reflected in 
   the size of the {\tt field}. That is, for each gridded dimension the 
   {\tt field} size should max( {\tt totalLWidth} + {\tt totalUWidth} 
   + {\tt computationalCount}, {\tt exclusiveCount} ). 
   \item [{[rc]}] 
   Return code; equals {\tt ESMF\_SUCCESS} if there are no errors. 
   \end{description} 
    
%/////////////////////////////////////////////////////////////
 
\mbox{}\hrulefill\ 
 
\subsubsection [ESMF\_FieldEmptyComplete] {ESMF\_FieldEmptyComplete - Complete a Field from LocStream started with FieldEmptyCreate }


   
\bigskip{\sf INTERFACE:}
\begin{verbatim}   ! Private name; call using ESMF_FieldEmptyComplete() 
   subroutine ESMF_FieldEmptyCompLS<rank><type><kind>(field, locstream, & 
   farray, indexflag, datacopyflag, gridToFieldMap, & 
   ungriddedLBound, ungriddedUBound, rc) 
   \end{verbatim}{\em ARGUMENTS:}
\begin{verbatim}   type(ESMF_Field), intent(inout) :: field 
   type(ESMF_LocStream), intent(in) :: locstream 
   <type> (ESMF_KIND_<kind>), intent(in), target :: farray(<rank>) 
   type(ESMF_Index_Flag), intent(in) :: indexflag 
 -- The following arguments require argument keyword syntax (e.g. rc=rc). --
   type(ESMF_DataCopy_Flag), intent(in), optional :: datacopyflag 
   integer, intent(in), optional :: gridToFieldMap(:) 
   integer, intent(in), optional :: ungriddedLBound(:) 
   integer, intent(in), optional :: ungriddedUBound(:) 
   integer, intent(out), optional :: rc 
   \end{verbatim}
{\sf DESCRIPTION:\\ }

 
   This call completes an {\tt ESMF\_Field} allocated with the 
   {\tt ESMF\_FieldEmptyCreate()} call. 
   
   The Fortran data pointer inside {\tt ESMF\_Field} can be queried but deallocating 
   the retrieved data pointer is not allowed. 
   
   The arguments are: 
   \begin{description} 
   \item [field] 
   The {\tt ESMF\_Field} object to be completed and 
   committed in this call. The {\tt field} will have the same dimension 
   (dimCount) as the rank of the {\tt farray}. 
   \item [locstream] 
   The {\tt ESMF\_LocStream} object to complete the Field. 
   \item [farray] 
   Native Fortran data array to be copied/referenced in the {\tt field}. 
   The {\tt field} dimension (dimCount) will be the same as the dimCount 
   for the {\tt farray}. 
   \item [indexflag] 
   Indicate how DE-local indices are defined. See section 
   \ref{const:indexflag} for a list of valid indexflag options. 
   \item [{[datacopyflag]}] 
   Indicates whether to copy the {\tt farray} or reference it directly. 
   For valid values see \ref{const:datacopyflag}. The default is 
   {\tt ESMF\_DATACOPY\_REFERENCE}. 
   \item [{[gridToFieldMap]}] 
   List with number of elements equal to the 
   {\tt locstream}'s dimCount. The list elements map each dimension 
   of the {\tt locstream} to a dimension in the {\tt farray} by 
   specifying the appropriate {\tt farray} dimension index. The 
   default is to map all of the {\tt locstream}'s dimensions against the 
   lowest dimensions of the {\tt farray} in sequence, i.e. 
   {\tt gridToFieldMap} = (/1,2,3,.../). 
   Unmapped {\tt farray} dimensions are undistributed Field 
   dimensions. 
   All {\tt gridToFieldMap} entries must be greater than or equal 
   to zero and smaller than or equal to the Field dimCount. It is erroneous 
   to specify the same entry multiple times unless it is zero. 
   If the Field dimCount is less than the LocStream dimCount then the default 
   gridToFieldMap will contain zeros for the rightmost entries. A zero 
   entry in the {\tt gridToFieldMap} indicates that the particular 
   LocStream dimension will be replicating the Field across the DEs along 
   this direction. 
   \item [{[ungriddedLBound]}] 
   Lower bounds of the ungridded dimensions of the {\tt field}. 
   The number of elements in the {\tt ungriddedLBound} is equal to the number of ungridded 
   dimensions in the {\tt field}. All ungridded dimensions of the 
   {\tt field} are also undistributed. When field dimension count is 
   greater than locstream dimension count, both ungriddedLBound and ungriddedUBound 
   must be specified. When both are specified the values are checked 
   for consistency. Note that the the ordering of 
   these ungridded dimensions is the same as their order in the {\tt field}. 
   \item [{[ungriddedUBound]}] 
   Upper bounds of the ungridded dimensions of the {\tt field}. 
   The number of elements in the {\tt ungriddedUBound} is equal to the number of ungridded 
   dimensions in the {\tt field}. All ungridded dimensions of the 
   {\tt field} are also undistributed. When field dimension count is 
   greater than locstream dimension count, both ungriddedLBound and ungriddedUBound 
   must be specified. When both are specified the values are checked 
   for consistency. Note that the the ordering of 
   these ungridded dimensions is the same as their order in the {\tt field}. 
   \item [{[rc]}] 
   Return code; equals {\tt ESMF\_SUCCESS} if there are no errors. 
   \end{description} 
    
%/////////////////////////////////////////////////////////////
 
\mbox{}\hrulefill\ 
 
\subsubsection [ESMF\_FieldEmptyComplete] {ESMF\_FieldEmptyComplete - Complete a Field from LocStream started with FieldEmptyCreate }


   
\bigskip{\sf INTERFACE:}
\begin{verbatim}   ! Private name; call using ESMF_FieldEmptyComplete() 
   subroutine ESMF_FieldEmptyCompLSPtr<rank><type><kind>(field, locstream, & 
   farrayPtr, datacopyflag, gridToFieldMap, rc) 
   \end{verbatim}{\em ARGUMENTS:}
\begin{verbatim}   type(ESMF_Field), intent(inout) :: field 
   type(ESMF_LocStream), intent(in) :: locstream 
   <type> (ESMF_KIND_<kind>), pointer :: farrayPtr(<rank>) 
 -- The following arguments require argument keyword syntax (e.g. rc=rc). --
   type(ESMF_DataCopy_Flag), intent(in), optional :: datacopyflag 
   integer, intent(in), optional :: gridToFieldMap(:) 
   integer, intent(out), optional :: rc 
   \end{verbatim}
{\sf DESCRIPTION:\\ }

 
   This call completes an {\tt ESMF\_Field} allocated with the 
   {\tt ESMF\_FieldEmptyCreate()} call. 
   
   \begin{sloppypar} 
   The Fortran data pointer inside {\tt ESMF\_Field} can be queried and deallocated when 
   datacopyflag is {\tt ESMF\_DATACOPY\_REFERENCE}. Note that the {\tt ESMF\_FieldDestroy} call does not deallocate 
   the Fortran data pointer in this case. This gives user more flexibility over memory management. 
   \end{sloppypar} 
   
   The arguments are: 
   \begin{description} 
   \item [field] 
   The {\tt ESMF\_Field} object to be completed and 
   committed in this call. The {\tt field} will have the same dimension 
   (dimCount) as the rank of the {\tt farrayPtr}. 
   \item [locstream] 
   The {\tt ESMF\_LocStream} object to complete the Field. 
   \item [farrayPtr] 
   Native Fortran data pointer to be copied/referenced in the {\tt field}. 
   The {\tt field} dimension (dimCount) will be the same as the dimCount 
   for the {\tt farrayPtr}. 
   \item [{[datacopyflag]}] 
   Indicates whether to copy the {\tt farrayPtr} or reference it directly. 
   For valid values see \ref{const:datacopyflag}. The default is 
   {\tt ESMF\_DATACOPY\_REFERENCE}. 
   \item [{[gridToFieldMap]}] 
   List with number of elements equal to the 
   {\tt locstream}'s dimCount. The list elements map each dimension 
   of the {\tt locstream} to a dimension in the {\tt farrayPtr} by 
   specifying the appropriate {\tt farrayPtr} dimension index. The 
   default is to map all of the {\tt locstream}'s dimensions against the 
   lowest dimensions of the {\tt farrayPtr} in sequence, i.e. 
   {\tt gridToFieldMap} = (/1,2,3,.../). 
   Unmapped {\tt farrayPtr} dimensions are undistributed Field 
   dimensions. 
   All {\tt gridToFieldMap} entries must be greater than or equal 
   to zero and smaller than or equal to the Field dimCount. It is erroneous 
   to specify the same entry multiple times unless it is zero. 
   If the Field dimCount is less than the LocStream dimCount then the default 
   gridToFieldMap will contain zeros for the rightmost entries. A zero 
   entry in the {\tt gridToFieldMap} indicates that the particular 
   LocStream dimension will be replicating the Field across the DEs along 
   this direction. 
   \item [{[rc]}] 
   Return code; equals {\tt ESMF\_SUCCESS} if there are no errors. 
   \end{description} 
    
%/////////////////////////////////////////////////////////////
 
\mbox{}\hrulefill\ 
 
\subsubsection [ESMF\_FieldEmptyComplete] {ESMF\_FieldEmptyComplete - Complete a Field from Mesh started with FieldEmptyCreate }


   
\bigskip{\sf INTERFACE:}
\begin{verbatim}   ! Private name; call using ESMF_FieldEmptyComplete() 
   subroutine ESMF_FieldEmptyCompMesh<rank><type><kind>(field, mesh, & 
   farray, indexflag, datacopyflag, meshloc, & 
   gridToFieldMap, ungriddedLBound, ungriddedUBound, rc) 
   \end{verbatim}{\em ARGUMENTS:}
\begin{verbatim}   type(ESMF_Field), intent(inout) :: field 
   type(ESMF_Mesh), intent(in) :: mesh 
   <type> (ESMF_KIND_<kind>), intent(in), target :: farray(<rank>) 
   type(ESMF_Index_Flag), intent(in) :: indexflag 
 -- The following arguments require argument keyword syntax (e.g. rc=rc). --
   type(ESMF_DataCopy_Flag), intent(in), optional :: datacopyflag 
   type(ESMF_MeshLoc), intent(in), optional :: meshloc 
   integer, intent(in), optional :: gridToFieldMap(:) 
   integer, intent(in), optional :: ungriddedLBound(:) 
   integer, intent(in), optional :: ungriddedUBound(:) 
   integer, intent(out), optional :: rc 
   \end{verbatim}
{\sf DESCRIPTION:\\ }

 
   This call completes an {\tt ESMF\_Field} allocated with the 
   {\tt ESMF\_FieldEmptyCreate()} call. 
   
   The Fortran data pointer inside {\tt ESMF\_Field} can be queried but deallocating 
   the retrieved data pointer is not allowed. 
   
   The arguments are: 
   \begin{description} 
   \item [field] 
   The {\tt ESMF\_Field} object to be completed and 
   committed in this call. The {\tt field} will have the same dimension 
   (dimCount) as the rank of the {\tt farray}. 
   \item [mesh] 
   The {\tt ESMF\_Mesh} object to complete the Field. 
   \item [farray] 
   Native Fortran data array to be copied/referenced in the {\tt field}. 
   The {\tt field} dimension (dimCount) will be the same as the dimCount 
   for the {\tt farray}. 
   \item [indexflag] 
   Indicate how DE-local indices are defined. See section 
   \ref{const:indexflag} for a list of valid indexflag options. 
   \item [{[datacopyflag]}] 
   Indicates whether to copy the {\tt farray} or reference it directly. 
   For valid values see \ref{const:datacopyflag}. The default is 
   {\tt ESMF\_DATACOPY\_REFERENCE}. 
   \item [{[meshloc]}] 
   \begin{sloppypar} 
   Which part of the mesh to build the Field on. Can be set to either 
   {\tt ESMF\_MESHLOC\_NODE} or {\tt ESMF\_MESHLOC\_ELEMENT}. If not set, 
   defaults to {\tt ESMF\_MESHLOC\_NODE}. 
   \end{sloppypar} 
   \item [{[gridToFieldMap]}] 
   List with number of elements equal to the 
   {\tt mesh}'s dimCount. The list elements map each dimension 
   of the {\tt mesh} to a dimension in the {\tt farray} by 
   specifying the appropriate {\tt farray} dimension index. The 
   default is to map all of the {\tt mesh}'s dimensions against the 
   lowest dimensions of the {\tt farray} in sequence, i.e. 
   {\tt gridToFieldMap} = (/1,2,3,.../). 
   Unmapped {\tt farray} dimensions are undistributed Field 
   dimensions. 
   All {\tt gridToFieldMap} entries must be greater than or equal 
   to zero and smaller than or equal to the Field dimCount. It is erroneous 
   to specify the same entry multiple times unless it is zero. 
   If the Field dimCount is less than the Mesh dimCount then the default 
   gridToFieldMap will contain zeros for the rightmost entries. A zero 
   entry in the {\tt gridToFieldMap} indicates that the particular 
   Mesh dimension will be replicating the Field across the DEs along 
   this direction. 
   \item [{[ungriddedLBound]}] 
   Lower bounds of the ungridded dimensions of the {\tt field}. 
   The number of elements in the {\tt ungriddedLBound} is equal to the number of ungridded 
   dimensions in the {\tt field}. All ungridded dimensions of the 
   {\tt field} are also undistributed. When field dimension count is 
   greater than Mesh dimension count, both ungriddedLBound and ungriddedUBound 
   must be specified. When both are specified the values are checked 
   for consistency. Note that the the ordering of 
   these ungridded dimensions is the same as their order in the {\tt field}. 
   \item [{[ungriddedUBound]}] 
   Upper bounds of the ungridded dimensions of the {\tt field}. 
   The number of elements in the {\tt ungriddedUBound} is equal to the number of ungridded 
   dimensions in the {\tt field}. All ungridded dimensions of the 
   {\tt field} are also undistributed. When field dimension count is 
   greater than Mesh dimension count, both ungriddedLBound and ungriddedUBound 
   must be specified. When both are specified the values are checked 
   for consistency. Note that the the ordering of 
   these ungridded dimensions is the same as their order in the {\tt field}. 
   \item [{[rc]}] 
   Return code; equals {\tt ESMF\_SUCCESS} if there are no errors. 
   \end{description} 
    
%/////////////////////////////////////////////////////////////
 
\mbox{}\hrulefill\ 
 
\subsubsection [ESMF\_FieldEmptyComplete] {ESMF\_FieldEmptyComplete - Complete a Field from Mesh started with FieldEmptyCreate }


   
\bigskip{\sf INTERFACE:}
\begin{verbatim}   ! Private name; call using ESMF_FieldEmptyComplete() 
   subroutine ESMF_FieldEmptyCompMeshPtr<rank><type><kind>(field, mesh, & 
   farrayPtr, datacopyflag, meshloc, gridToFieldMap, rc) 
   \end{verbatim}{\em ARGUMENTS:}
\begin{verbatim}   type(ESMF_Field), intent(inout) :: field 
   type(ESMF_Mesh), intent(in) :: mesh 
   <type> (ESMF_KIND_<kind>), pointer :: farrayPtr(<rank>) 
 -- The following arguments require argument keyword syntax (e.g. rc=rc). --
   type(ESMF_DataCopy_Flag), intent(in), optional :: datacopyflag 
   type(ESMF_MeshLoc), intent(in), optional :: meshloc 
   integer, intent(in), optional :: gridToFieldMap(:) 
   integer, intent(out), optional :: rc 
   \end{verbatim}
{\sf DESCRIPTION:\\ }

 
   This call completes an {\tt ESMF\_Field} allocated with the 
   {\tt ESMF\_FieldEmptyCreate()} call. 
   
   \begin{sloppypar} 
   The Fortran data pointer inside {\tt ESMF\_Field} can be queried and deallocated when 
   datacopyflag is {\tt ESMF\_DATACOPY\_REFERENCE}. Note that the {\tt ESMF\_FieldDestroy} call does not deallocate 
   the Fortran data pointer in this case. This gives user more flexibility over memory management. 
   \end{sloppypar} 
   
   The arguments are: 
   \begin{description} 
   \item [field] 
   The {\tt ESMF\_Field} object to be completed and 
   committed in this call. The {\tt field} will have the same dimension 
   (dimCount) as the rank of the {\tt farrayPtr}. 
   \item [mesh] 
   The {\tt ESMF\_Mesh} object to complete the Field. 
   \item [farrayPtr] 
   Native Fortran data pointer to be copied/referenced in the {\tt field}. 
   The {\tt field} dimension (dimCount) will be the same as the dimCount 
   for the {\tt farrayPtr}. 
   \item [{[datacopyflag]}] 
   Indicates whether to copy the {\tt farrayPtr} or reference it directly. 
   For valid values see \ref{const:datacopyflag}. The default is 
   {\tt ESMF\_DATACOPY\_REFERENCE}. 
   \item [{[meshloc]}] 
   \begin{sloppypar} 
   Which part of the mesh to build the Field on. Can be set to either 
   {\tt ESMF\_MESHLOC\_NODE} or {\tt ESMF\_MESHLOC\_ELEMENT}. If not set, 
   defaults to {\tt ESMF\_MESHLOC\_NODE}. 
   \end{sloppypar} 
   \item [{[gridToFieldMap]}] 
   List with number of elements equal to the 
   {\tt mesh}'s dimCount. The list elements map each dimension 
   of the {\tt mesh} to a dimension in the {\tt farrayPtr} by 
   specifying the appropriate {\tt farrayPtr} dimension index. The 
   default is to map all of the {\tt mesh}'s dimensions against the 
   lowest dimensions of the {\tt farrayPtr} in sequence, i.e. 
   {\tt gridToFieldMap} = (/1,2,3,.../). 
   Unmapped {\tt farrayPtr} dimensions are undistributed Field 
   dimensions. 
   All {\tt gridToFieldMap} entries must be greater than or equal 
   to zero and smaller than or equal to the Field dimCount. It is erroneous 
   to specify the same entry multiple times unless it is zero. 
   If the Field dimCount is less than the Mesh dimCount then the default 
   gridToFieldMap will contain zeros for the rightmost entries. A zero 
   entry in the {\tt gridToFieldMap} indicates that the particular 
   Mesh dimension will be replicating the Field across the DEs along 
   this direction. 
   \item [{[rc]}] 
   Return code; equals {\tt ESMF\_SUCCESS} if there are no errors. 
   \end{description} 
    
%/////////////////////////////////////////////////////////////
 
\mbox{}\hrulefill\ 
 
\subsubsection [ESMF\_FieldEmptyComplete] {ESMF\_FieldEmptyComplete - Complete a Field from XGrid started with FieldEmptyCreate }


   
\bigskip{\sf INTERFACE:}
\begin{verbatim}   ! Private name; call using ESMF_FieldEmptyComplete() 
   subroutine ESMF_FieldEmptyCompXG<rank><type><kind>(field, xgrid, & 
   farray, indexflag, datacopyflag, xgridside, gridindex, & 
   gridToFieldMap, & 
   ungriddedLBound, ungriddedUBound, rc) 
   \end{verbatim}{\em ARGUMENTS:}
\begin{verbatim}   type(ESMF_Field), intent(inout) :: field 
   type(ESMF_XGrid), intent(in) :: xgrid 
   <type> (ESMF_KIND_<kind>), intent(in), target :: farray(<rank>) 
   type(ESMF_Index_Flag), intent(in) :: indexflag 
 -- The following arguments require argument keyword syntax (e.g. rc=rc). --
   type(ESMF_DataCopy_Flag), intent(in), optional :: datacopyflag 
   type(ESMF_XGridSide_Flag), intent(in), optional :: xgridside 
   integer, intent(in), optional :: gridindex 
   integer, intent(in), optional :: gridToFieldMap(:) 
   integer, intent(in), optional :: ungriddedLBound(:) 
   integer, intent(in), optional :: ungriddedUBound(:) 
   integer, intent(out), optional :: rc 
   \end{verbatim}
{\sf DESCRIPTION:\\ }

 
   This call completes an {\tt ESMF\_Field} allocated with the 
   {\tt ESMF\_FieldEmptyCreate()} call. 
   
   The Fortran data pointer inside {\tt ESMF\_Field} can be queried but deallocating 
   the retrieved data pointer is not allowed. 
   
   The arguments are: 
   \begin{description} 
   \item [field] 
   The {\tt ESMF\_Field} object to be completed and 
   committed in this call. The {\tt field} will have the same dimension 
   (dimCount) as the rank of the {\tt farray}. 
   \item [xgrid] 
   The {\tt ESMF\_XGrid} object to complete the Field. 
   \item [farray] 
   Native Fortran data array to be copied/referenced in the {\tt field}. 
   The {\tt field} dimension (dimCount) will be the same as the dimCount 
   for the {\tt farray}. 
   \item [indexflag] 
   Indicate how DE-local indices are defined. See section 
   \ref{const:indexflag} for a list of valid indexflag options. 
   \item [{[datacopyflag]}] 
   Indicates whether to copy the {\tt farray} or reference it directly. 
   For valid values see \ref{const:datacopyflag}. The default is 
   {\tt ESMF\_DATACOPY\_REFERENCE}. 
   \item [{[xgridside]}] 
   Which side of the XGrid to create the Field on (either ESMF\_XGRIDSIDE\_A, 
   ESMF\_XGRIDSIDE\_B, or ESMF\_XGRIDSIDE\_BALANCED). If not passed in then 
   defaults to ESMF\_XGRIDSIDE\_BALANCED. 
   \item [{[gridindex]}] 
   If xgridSide is ESMF\_XGRIDSIDE\_A or ESMF\_XGRIDSIDE\_B then this index tells which Grid on 
   that side to create the Field on. If not provided, defaults to 1. 
   \item [{[gridToFieldMap]}] 
   List with number of elements equal to the 
   {\tt xgrid}'s dimCount. The list elements map each dimension 
   of the {\tt xgrid} to a dimension in the {\tt farray} by 
   specifying the appropriate {\tt farray} dimension index. The 
   default is to map all of the {\tt xgrid}'s dimensions against the 
   lowest dimensions of the {\tt farray} in sequence, i.e. 
   {\tt gridToFieldMap} = (/1,2,3,.../). 
   Unmapped {\tt farray} dimensions are undistributed Field 
   dimensions. 
   All {\tt gridToFieldMap} entries must be greater than or equal 
   to zero and smaller than or equal to the Field dimCount. It is erroneous 
   to specify the same entry multiple times unless it is zero. 
   If the Field dimCount is less than the XGrid dimCount then the default 
   gridToFieldMap will contain zeros for the rightmost entries. A zero 
   entry in the {\tt gridToFieldMap} indicates that the particular 
   XGrid dimension will be replicating the Field across the DEs along 
   this direction. 
   \item [{[ungriddedLBound]}] 
   Lower bounds of the ungridded dimensions of the {\tt field}. 
   The number of elements in the {\tt ungriddedLBound} is equal to the number of ungridded 
   dimensions in the {\tt field}. All ungridded dimensions of the 
   {\tt field} are also undistributed. When field dimension count is 
   greater than XGrid dimension count, both ungriddedLBound and ungriddedUBound 
   must be specified. When both are specified the values are checked 
   for consistency. Note that the the ordering of 
   these ungridded dimensions is the same as their order in the {\tt field}. 
   \item [{[ungriddedUBound]}] 
   Upper bounds of the ungridded dimensions of the {\tt field}. 
   The number of elements in the {\tt ungriddedUBound} is equal to the number of ungridded 
   dimensions in the {\tt field}. All ungridded dimensions of the 
   {\tt field} are also undistributed. When field dimension count is 
   greater than XGrid dimension count, both ungriddedLBound and ungriddedUBound 
   must be specified. When both are specified the values are checked 
   for consistency. Note that the the ordering of 
   these ungridded dimensions is the same as their order in the {\tt field}. 
   \item [{[rc]}] 
   Return code; equals {\tt ESMF\_SUCCESS} if there are no errors. 
   \end{description} 
    
%/////////////////////////////////////////////////////////////
 
\mbox{}\hrulefill\ 
 
\subsubsection [ESMF\_FieldEmptyComplete] {ESMF\_FieldEmptyComplete - Complete a Field from XGrid started with FieldEmptyCreate }


   
\bigskip{\sf INTERFACE:}
\begin{verbatim}   ! Private name; call using ESMF_FieldEmptyComplete() 
   subroutine ESMF_FieldEmptyCompXGPtr<rank><type><kind>(field, xgrid, & 
   farrayPtr, xgridside, gridindex, & 
   datacopyflag, gridToFieldMap, rc) 
   \end{verbatim}{\em ARGUMENTS:}
\begin{verbatim}   type(ESMF_Field), intent(inout) :: field 
   type(ESMF_XGrid), intent(in) :: xgrid 
   <type> (ESMF_KIND_<kind>), pointer :: farrayPtr(<rank>) 
 -- The following arguments require argument keyword syntax (e.g. rc=rc). --
   type(ESMF_DataCopy_Flag), intent(in), optional :: datacopyflag 
   type(ESMF_XGridSide_Flag), intent(in), optional :: xgridside 
   integer, intent(in), optional :: gridindex 
   integer, intent(in), optional :: gridToFieldMap(:) 
   integer, intent(out), optional :: rc 
   \end{verbatim}
{\sf DESCRIPTION:\\ }

 
   This call completes an {\tt ESMF\_Field} allocated with the 
   {\tt ESMF\_FieldEmptyCreate()} call. 
   
   \begin{sloppypar} 
   The Fortran data pointer inside {\tt ESMF\_Field} can be queried and deallocated when 
   datacopyflag is {\tt ESMF\_DATACOPY\_REFERENCE}. Note that the {\tt ESMF\_FieldDestroy} call does not deallocate 
   the Fortran data pointer in this case. This gives user more flexibility over memory management. 
   \end{sloppypar} 
   
   The arguments are: 
   \begin{description} 
   \item [field] 
   The {\tt ESMF\_Field} object to be completed and 
   committed in this call. The {\tt field} will have the same dimension 
   (dimCount) as the rank of the {\tt farrayPtr}. 
   \item [xgrid] 
   The {\tt ESMF\_XGrid} object to complete the Field. 
   \item [farrayPtr] 
   Native Fortran data pointer to be copied/referenced in the {\tt field}. 
   The {\tt field} dimension (dimCount) will be the same as the dimCount 
   for the {\tt farrayPtr}. 
   \item [{[datacopyflag]}] 
   Indicates whether to copy the {\tt farrayPtr} or reference it directly. 
   For valid values see \ref{const:datacopyflag}. The default is 
   {\tt ESMF\_DATACOPY\_REFERENCE}. 
   \item [{[xgridside]}] 
   Which side of the XGrid to create the Field on (either ESMF\_XGRIDSIDE\_A, 
   ESMF\_XGRIDSIDE\_B, or ESMF\_XGRIDSIDE\_BALANCED). If not passed in then 
   defaults to ESMF\_XGRIDSIDE\_BALANCED. 
   \item [{[gridindex]}] 
   If xgridside is ESMF\_XGRIDSIDE\_A or ESMF\_XGRIDSIDE\_B then this index tells which Grid on 
   that side to create the Field on. If not provided, defaults to 1. 
   \item [{[gridToFieldMap]}] 
   List with number of elements equal to the 
   {\tt xgrid}'s dimCount. The list elements map each dimension 
   of the {\tt xgrid} to a dimension in the {\tt farrayPtr} by 
   specifying the appropriate {\tt farrayPtr} dimension index. The 
   default is to map all of the {\tt xgrid}'s dimensions against the 
   lowest dimensions of the {\tt farrayPtr} in sequence, i.e. 
   {\tt gridToFieldMap} = (/1,2,3,.../). 
   Unmapped {\tt farrayPtr} dimensions are undistributed Field 
   dimensions. 
   All {\tt gridToFieldMap} entries must be greater than or equal 
   to zero and smaller than or equal to the Field dimCount. It is erroneous 
   to specify the same entry multiple times unless it is zero. 
   If the Field dimCount is less than the XGrid dimCount then the default 
   gridToFieldMap will contain zeros for the rightmost entries. A zero 
   entry in the {\tt gridToFieldMap} indicates that the particular 
   XGrid dimension will be replicating the Field across the DEs along 
   this direction. 
   \item [{[rc]}] 
   Return code; equals {\tt ESMF\_SUCCESS} if there are no errors. 
   \end{description} 
    
%/////////////////////////////////////////////////////////////
 
\mbox{}\hrulefill\ 
 
\subsubsection [ESMF\_FieldEmptyCreate] {ESMF\_FieldEmptyCreate - Create an empty Field}


\bigskip{\sf INTERFACE:}
\begin{verbatim}   function ESMF_FieldEmptyCreate(name, vm, rc)\end{verbatim}{\em RETURN VALUE:}
\begin{verbatim}     type(ESMF_Field) :: ESMF_FieldEmptyCreate\end{verbatim}{\em ARGUMENTS:}
\begin{verbatim} -- The following arguments require argument keyword syntax (e.g. rc=rc). --
     character (len = *), intent(in), optional :: name
     type(ESMF_VM), intent(in), optional :: vm
     integer, intent(out), optional :: rc\end{verbatim}
{\sf STATUS:}
   \begin{itemize}
   \item\apiStatusCompatibleVersion{5.2.0r}
   \item\apiStatusModifiedSinceVersion{5.2.0r}
   \begin{description}
   \item[8.0.0] Added argument {\tt vm} to support object creation on a
   different VM than that of the current context.
   \end{description}
   \end{itemize}
  
{\sf DESCRIPTION:\\ }


   \begin{sloppypar}
   This version of {\tt ESMF\_FieldCreate} builds an empty {\tt ESMF\_Field}
   and depends on later calls to add an {\tt ESMF\_Grid} and {\tt ESMF\_Array} to
   it. The empty {\tt ESMF\_Field} can be completed in one more step or two more steps by
   the {\tt ESMF\_FieldEmptySet} and {\tt ESMF\_FieldEmptyComplete} methods.
   Attributes can be added to an empty Field object. For an example and
   associated documentation using this method see section
   \ref{sec:field:usage:create_empty} and \ref{sec:field:usage:partial_creation}.
   \end{sloppypar}
  
  
   The arguments are:
   \begin{description}
   \item [{[name]}]
   Field name.
   \item[{[vm]}]
   If present, the Field object is created on the specified
   {\tt ESMF\_VM} object. The default is to create on the VM of the
   current component context.
   \item [{[rc]}]
   Return code; equals {\tt ESMF\_SUCCESS} if there are no errors.
   \end{description}
   
%/////////////////////////////////////////////////////////////
 
\mbox{}\hrulefill\ 
 
\subsubsection [ESMF\_FieldEmptySet] {ESMF\_FieldEmptySet - Set a Grid in an empty Field}


\bigskip{\sf INTERFACE:}
\begin{verbatim}   ! Private name; call using ESMF_FieldEmptySet()
   subroutine ESMF_FieldEmptySetGrid(field, grid, StaggerLoc, &
     vm, rc)\end{verbatim}{\em ARGUMENTS:}
\begin{verbatim}   type(ESMF_Field), intent(inout) :: field
   type(ESMF_Grid), intent(in) :: grid
 -- The following arguments require argument keyword syntax (e.g. rc=rc). --
   type(ESMF_STAGGERLOC), intent(in), optional :: StaggerLoc
   type(ESMF_VM), intent(in), optional :: vm
   integer, intent(out), optional :: rc\end{verbatim}
{\sf STATUS:}
   \begin{itemize}
   \item\apiStatusCompatibleVersion{5.2.0r}
   \item\apiStatusModifiedSinceVersion{5.2.0r}
   \begin{description}
   \item[7.1.0r] Added argument {\tt vm} to support object creation on a
   different VM than that of the current context.
   \end{description}
   \end{itemize}
  
{\sf DESCRIPTION:\\ }


   \begin{sloppypar}
   Set a grid and an optional staggerloc (default to center stagger
   {\tt ESMF\_STAGGERLOC\_CENTER}) in a non-completed {\tt ESMF\_Field}. The
   {\tt ESMF\_Field} must not be completed for this to succeed. After this
   operation, the {\tt ESMF\_Field} contains
   the {\tt ESMF\_Grid} internally but holds no data.
   The status of the field changes from
   {\tt ESMF\_FIELDSTATUS\_EMPTY} to {\tt ESMF\_FIELDSTATUS\_GRIDSET} or
   stays {\tt ESMF\_FIELDSTATUS\_GRIDSET}.
  
   For an example and
   associated documentation using this method see section
   \ref{sec:field:usage:partial_creation}.
   \end{sloppypar}
  
  
   The arguments are:
   \begin{description}
   \item [field]
   Empty {\tt ESMF\_Field}. After this
   operation, the {\tt ESMF\_Field} contains
   the {\tt ESMF\_Grid} internally but holds no data.
   The status of the field changes from
   {\tt ESMF\_FIELDSTATUS\_EMPTY} to {\tt ESMF\_FIELDSTATUS\_GRIDSET}.
   \item [grid]
   {\tt ESMF\_Grid} to be set in the {\tt ESMF\_Field}.
   \item [{[StaggerLoc]}]
   Stagger location of data in grid cells. For valid
   predefined values see section \ref{const:staggerloc}.
   To create a custom stagger location see section
   \ref{sec:usage:staggerloc:adv}. The default
   value is {\tt ESMF\_STAGGERLOC\_CENTER}.
   \item[{[vm]}]
   If present, the Field object will only be accessed, and the Grid object
   set, on those PETs contained in the specified {\tt ESMF\_VM} object.
   The default is to assume the VM of the current context.
   \item [{[rc]}]
   Return code; equals {\tt ESMF\_SUCCESS} if there are no errors.
   \end{description}
   
%/////////////////////////////////////////////////////////////
 
\mbox{}\hrulefill\ 
 
\subsubsection [ESMF\_FieldEmptySet] {ESMF\_FieldEmptySet - Set a Mesh in an empty Field}


\bigskip{\sf INTERFACE:}
\begin{verbatim}   ! Private name; call using ESMF_FieldEmptySet()
   subroutine ESMF_FieldEmptySetMesh(field, mesh, indexflag, meshloc, rc)\end{verbatim}{\em ARGUMENTS:}
\begin{verbatim}   type(ESMF_Field), intent(inout) :: field
   type(ESMF_Mesh), intent(in) :: mesh
 -- The following arguments require argument keyword syntax (e.g. rc=rc). --
   type(ESMF_Index_Flag),intent(in), optional :: indexflag
   type(ESMF_MeshLoc), intent(in), optional :: meshloc
   integer, intent(out), optional :: rc\end{verbatim}
{\sf DESCRIPTION:\\ }


   \begin{sloppypar}
   Set a mesh and an optional meshloc (default to center stagger
   {\tt ESMF\_MESHLOC\_NODE}) in a non-completed {\tt ESMF\_Field}. The
   {\tt ESMF\_Field} must not be completed for this to succeed. After this
   operation, the {\tt ESMF\_Field} contains
   the {\tt ESMF\_Mesh} internally but holds no data.
   The status of the field changes from
   {\tt ESMF\_FIELDSTATUS\_EMPTY} to {\tt ESMF\_FIELDSTATUS\_GRIDSET} or
   stays {\tt ESMF\_FIELDSTATUS\_GRIDSET}.
  
   \end{sloppypar}
  
  
   The arguments are:
   \begin{description}
   \item [field]
   Empty {\tt ESMF\_Field}. After this
   operation, the {\tt ESMF\_Field} contains
   the {\tt ESMF\_Mesh} internally but holds no data.
   The status of the field changes from
   {\tt ESMF\_FIELDSTATUS\_EMPTY} to {\tt ESMF\_FIELDSTATUS\_GRIDSET}.
   \item [mesh]
   {\tt ESMF\_Mesh} to be set in the {\tt ESMF\_Field}.
   \item [{[indexflag]}]
   Indicate how DE-local indices are defined. See section
   \ref{const:indexflag} for a list of valid indexflag options.
   \item [{[meshloc]}]
   \begin{sloppypar}
   Which part of the mesh to build the Field on. Can be set to either
   {\tt ESMF\_MESHLOC\_NODE} or {\tt ESMF\_MESHLOC\_ELEMENT}. If not set,
   defaults to {\tt ESMF\_MESHLOC\_NODE}.
   \end{sloppypar}
   \item [{[rc]}]
   Return code; equals {\tt ESMF\_SUCCESS} if there are no errors.
   \end{description}
   
%/////////////////////////////////////////////////////////////
 
\mbox{}\hrulefill\ 
 
\subsubsection [ESMF\_FieldEmptySet] {ESMF\_FieldEmptySet - Set a LocStream in an empty Field}


\bigskip{\sf INTERFACE:}
\begin{verbatim}   ! Private name; call using ESMF_FieldEmptySet()
   subroutine ESMF_FieldEmptySetLocStream(field, locstream, &
     vm, rc)\end{verbatim}{\em ARGUMENTS:}
\begin{verbatim}   type(ESMF_Field), intent(inout) :: field
   type(ESMF_LocStream), intent(in) :: locstream
 -- The following arguments require argument keyword syntax (e.g. rc=rc). --
   type(ESMF_VM), intent(in), optional :: vm
   integer, intent(out), optional :: rc\end{verbatim}
{\sf DESCRIPTION:\\ }


   \begin{sloppypar}
   Set a {\tt ESMF\_LocStream} in a non-completed {\tt ESMF\_Field}. The
   {\tt ESMF\_Field} must not be completed for this to succeed. After this
   operation, the {\tt ESMF\_Field} contains
   the {\tt ESMF\_LocStream} internally but holds no data.
   The status of the field changes from
   {\tt ESMF\_FIELDSTATUS\_EMPTY} to {\tt ESMF\_FIELDSTATUS\_GRIDSET} or
   stays {\tt ESMF\_FIELDSTATUS\_GRIDSET}.
  
   \end{sloppypar}
  
  
   The arguments are:
   \begin{description}
   \item [field]
   \begin{sloppypar}
   Empty {\tt ESMF\_Field}. After this
   operation, the {\tt ESMF\_Field} contains
   the {\tt ESMF\_LocStream} internally but holds no data.
   The status of the field changes from
   {\tt ESMF\_FIELDSTATUS\_EMPTY} to {\tt ESMF\_FIELDSTATUS\_GRIDSET}.
   \end{sloppypar}
   \item [locstream]
   {\tt ESMF\_LocStream} to be set in the {\tt ESMF\_Field}.
   \item[{[vm]}]
   If present, the Field object will only be accessed, and the Grid object
   set, on those PETs contained in the specified {\tt ESMF\_VM} object.
   The default is to assume the VM of the current context.
   \item [{[rc]}]
   Return code; equals {\tt ESMF\_SUCCESS} if there are no errors.
   \end{description}
   
%/////////////////////////////////////////////////////////////
 
\mbox{}\hrulefill\ 
 
\subsubsection [ESMF\_FieldEmptySet] {ESMF\_FieldEmptySet - Set an XGrid in an empty Field}


\bigskip{\sf INTERFACE:}
\begin{verbatim}   ! Private name; call using ESMF_FieldEmptySet()
   subroutine ESMF_FieldEmptySetXGrid(field, xgrid, xgridside, gridindex, rc)\end{verbatim}{\em ARGUMENTS:}
\begin{verbatim}   type(ESMF_Field), intent(inout) :: field
   type(ESMF_XGrid), intent(in) :: xgrid
 -- The following arguments require argument keyword syntax (e.g. rc=rc). --
   type(ESMF_XGridSide_Flag), intent(in), optional :: xgridside
   integer, intent(in), optional :: gridindex
   integer, intent(out), optional :: rc\end{verbatim}
{\sf DESCRIPTION:\\ }


   \begin{sloppypar}
   Set a xgrid and optional xgridside (default to balanced side
   {\tt ESMF\_XGRIDSIDE\_Balanced}) and gridindex (default to 1)
   in a non-complete {\tt ESMF\_Field}. The
   {\tt ESMF\_Field} must not be completed for this to succeed. After this
   operation, the {\tt ESMF\_Field} contains
   the {\tt ESMF\_XGrid} internally but holds no data.
   The status of the field changes from
   {\tt ESMF\_FIELDSTATUS\_EMPTY} to {\tt ESMF\_FIELDSTATUS\_GRIDSET}
   or stays {\tt ESMF\_FIELDSTATUS\_GRIDSET}.
  
   \end{sloppypar}
  
  
   The arguments are:
   \begin{description}
   \item [field]
   \begin{sloppypar}
   Empty {\tt ESMF\_Field}. After this
   operation, the {\tt ESMF\_Field} contains
   the {\tt ESMF\_XGrid} internally but holds no data.
   The status of the field changes from
   {\tt ESMF\_FIELDSTATUS\_EMPTY} to {\tt ESMF\_FIELDSTATUS\_GRIDSET}.
   \end{sloppypar}
   \item [xgrid]
   {\tt ESMF\_XGrid} to be set in the {\tt ESMF\_Field}.
   \item [{[xgridside]}]
   Side of XGrid to retrieve a DistGrid. For valid
   predefined values see section \ref{const:xgridside}.
   The default value is {\tt ESMF\_XGRIDSIDE\_BALANCED}.
   \item [{[gridindex]}]
   Index to specify which DistGrid when on side A or side B.
   The default value is 1.
   \item [{[rc]}]
   Return code; equals {\tt ESMF\_SUCCESS} if there are no errors.
   \end{description}
  
%...............................................................
\setlength{\parskip}{\oldparskip}
\setlength{\parindent}{\oldparindent}
\setlength{\baselineskip}{\oldbaselineskip}
