%                **** IMPORTANT NOTICE *****
% This LaTeX file has been automatically produced by ProTeX v. 1.1
% Any changes made to this file will likely be lost next time
% this file is regenerated from its source. Send questions 
% to Arlindo da Silva, dasilva@gsfc.nasa.gov
 
\setlength{\oldparskip}{\parskip}
\setlength{\parskip}{1.5ex}
\setlength{\oldparindent}{\parindent}
\setlength{\parindent}{0pt}
\setlength{\oldbaselineskip}{\baselineskip}
\setlength{\baselineskip}{11pt}
 
%--------------------- SHORT-HAND MACROS ----------------------
\def\bv{\begin{verbatim}}
\def\ev{\end{verbatim}}
\def\be{\begin{equation}}
\def\ee{\end{equation}}
\def\bea{\begin{eqnarray}}
\def\eea{\end{eqnarray}}
\def\bi{\begin{itemize}}
\def\ei{\end{itemize}}
\def\bn{\begin{enumerate}}
\def\en{\end{enumerate}}
\def\bd{\begin{description}}
\def\ed{\end{description}}
\def\({\left (}
\def\){\right )}
\def\[{\left [}
\def\]{\right ]}
\def\<{\left  \langle}
\def\>{\right \rangle}
\def\cI{{\cal I}}
\def\diag{\mathop{\rm diag}}
\def\tr{\mathop{\rm tr}}
%-------------------------------------------------------------

\markboth{Left}{Source File: ESMF\_FieldRedist.F90,  Date: Tue May  5 21:00:00 MDT 2020
}

 
%/////////////////////////////////////////////////////////////
\subsubsection [ESMF\_FieldRedist] {ESMF\_FieldRedist - Execute a Field redistribution}


  
\bigskip{\sf INTERFACE:}
\begin{verbatim}   subroutine ESMF_FieldRedist(srcField, dstField, routehandle,  &
     checkflag, rc)\end{verbatim}{\em ARGUMENTS:}
\begin{verbatim}         type(ESMF_Field),       intent(in),optional     :: srcField
         type(ESMF_Field),       intent(inout),optional  :: dstField
         type(ESMF_RouteHandle), intent(inout)           :: routehandle
 -- The following arguments require argument keyword syntax (e.g. rc=rc). --
         logical,                intent(in),   optional  :: checkflag
         integer,                intent(out),  optional  :: rc\end{verbatim}
{\sf STATUS:}
   \begin{itemize}
   \item\apiStatusCompatibleVersion{5.2.0r}
   \end{itemize}
  
{\sf DESCRIPTION:\\ }


     Execute a precomputed Field redistribution from {\tt srcField} to
     {\tt dstField}. 
     Both {\tt srcField} and {\tt dstField} must match the respective Fields
     used during {\tt ESMF\_FieldRedistStore()} in {\em type}, {\em kind}, and 
     memory layout of the {\em gridded} dimensions. However, the size, number, 
     and index order of {\em ungridded} dimensions may be different. See section
     \ref{RH:Reusability} for a more detailed discussion of RouteHandle 
     reusability.
  
     The {\tt srcField} and {\tt dstField} arguments are optional in support of
     the situation where {\tt srcField} and/or {\tt dstField} are not defined on
     all PETs. The {\tt srcField} and {\tt dstField} must be specified on those
     PETs that hold source or destination DEs, respectively, but may be omitted
     on all other PETs. PETs that hold neither source nor destination DEs may
     omit both arguments.
  
     It is erroneous to specify the identical Field object for {\tt srcField} and
     {\tt dstField} arguments.
  
     See {\tt ESMF\_FieldRedistStore()} on how to precompute 
     {\tt routehandle}.
  
     This call is {\em collective} across the current VM.
  
     For examples and associated documentation regarding this method see Section
     \ref{sec:field:usage:redist_1dptr}. 
  
     \begin{description}
     \item [{[srcField]}]
       {\tt ESMF\_Field} with source data.
     \item [{[dstField]}]
       {\tt ESMF\_Field} with destination data.
     \item [routehandle]
       Handle to the precomputed Route.
     \item [{[checkflag]}]
       If set to {\tt .TRUE.} the input Field pair will be checked for
       consistency with the precomputed operation provided by {\tt routehandle}.
       If set to {\tt .FALSE.} {\em (default)} only a very basic input check
       will be performed, leaving many inconsistencies undetected. Set
       {\tt checkflag} to {\tt .FALSE.} to achieve highest performance.
     \item [{[rc]}]
       Return code; equals {\tt ESMF\_SUCCESS} if there are no errors.
     \end{description}
   
%/////////////////////////////////////////////////////////////
 
\mbox{}\hrulefill\ 
 
\subsubsection [ESMF\_FieldRedistRelease] {ESMF\_FieldRedistRelease - Release resources associated with Field redistribution}


  
\bigskip{\sf INTERFACE:}
\begin{verbatim}   subroutine ESMF_FieldRedistRelease(routehandle, noGarbage, rc)\end{verbatim}{\em ARGUMENTS:}
\begin{verbatim}         type(ESMF_RouteHandle), intent(inout)           :: routehandle
 -- The following arguments require argument keyword syntax (e.g. rc=rc). --
         logical,                intent(in),   optional  :: noGarbage
         integer,                intent(out),  optional  :: rc\end{verbatim}
{\sf STATUS:}
   \begin{itemize}
   \item\apiStatusCompatibleVersion{5.2.0r}
   \item\apiStatusModifiedSinceVersion{5.2.0r}
   \begin{description}
   \item[8.0.0] Added argument {\tt noGarbage}.
     The argument provides a mechanism to override the default garbage collection
     mechanism when destroying an ESMF object.
   \end{description}
   \end{itemize}
  
{\sf DESCRIPTION:\\ }


     Release resources associated with a Field redistribution. After this call
     {\tt routehandle} becomes invalid.
  
     \begin{description}
     \item [routehandle]
       Handle to the precomputed Route.
     \item[{[noGarbage]}]
       If set to {\tt .TRUE.} the object will be fully destroyed and removed
       from the ESMF garbage collection system. Note however that under this 
       condition ESMF cannot protect against accessing the destroyed object 
       through dangling aliases -- a situation which may lead to hard to debug 
       application crashes.
   
       It is generally recommended to leave the {\tt noGarbage} argument
       set to {\tt .FALSE.} (the default), and to take advantage of the ESMF 
       garbage collection system which will prevent problems with dangling
       aliases or incorrect sequences of destroy calls. However this level of
       support requires that a small remnant of the object is kept in memory
       past the destroy call. This can lead to an unexpected increase in memory
       consumption over the course of execution in applications that use 
       temporary ESMF objects. For situations where the repeated creation and 
       destruction of temporary objects leads to memory issues, it is 
       recommended to call with {\tt noGarbage} set to {\tt .TRUE.}, fully 
       removing the entire temporary object from memory.
     \item [{[rc]}]
       Return code; equals {\tt ESMF\_SUCCESS} if there are no errors.
     \end{description}
   
%/////////////////////////////////////////////////////////////
 
\mbox{}\hrulefill\ 
 
\subsubsection [ESMF\_FieldRedistStore] {ESMF\_FieldRedistStore - Precompute Field redistribution with a local factor argument }


   
\bigskip{\sf INTERFACE:}
\begin{verbatim}   ! Private name; call using ESMF_FieldRedistStore() 
   subroutine ESMF_FieldRedistStore<type><kind>(srcField, dstField, & 
          routehandle, factor, srcToDstTransposeMap, &
          ignoreUnmatchedIndices, rc) 
   \end{verbatim}{\em ARGUMENTS:}
\begin{verbatim}     type(ESMF_Field),         intent(in)            :: srcField  
     type(ESMF_Field),         intent(inout)         :: dstField  
     type(ESMF_RouteHandle),   intent(inout)         :: routehandle
     <type>(ESMF_KIND_<kind>), intent(in)            :: factor 
 -- The following arguments require argument keyword syntax (e.g. rc=rc). --
     integer,                  intent(in),  optional :: srcToDstTransposeMap(:) 
     logical,                  intent(in),  optional :: ignoreUnmatchedIndices
     integer,                  intent(out), optional :: rc 
   \end{verbatim}
{\sf STATUS:}
   \begin{itemize}
   \item\apiStatusCompatibleVersion{5.2.0r}
   \item\apiStatusModifiedSinceVersion{5.2.0r}
   \begin{description}
   \item[7.0.0] Added argument {\tt ignoreUnmatchedIndices} to support sparse 
                matrices that contain elements with indices that do not have a
                match within the source or destination Array.
   \end{description}
   \end{itemize}
  
{\sf DESCRIPTION:\\ }

 
   \label{FieldRedistStoreTK}
   {\tt ESMF\_FieldRedistStore()} is a collective method across all PETs of the
   current Component. The interface of the method is overloaded, allowing 
   -- in principle -- each PET to call into {\tt ESMF\_FieldRedistStore()}
   through a different entry point. Restrictions apply as to which combinations
   are sensible. All other combinations result in ESMF run time errors. The
   complete semantics of the {\tt ESMF\_FieldRedistStore()} method, as provided
   through the separate entry points shown in \ref{FieldRedistStoreTK} and
   \ref{FieldRedistStoreNF}, is described in the following paragraphs as a whole.
  
   Store a Field redistribution operation from {\tt srcField} to {\tt dstField}.
   Interface \ref{FieldRedistStoreTK} allows PETs to specify a {\tt factor}
   argument. PETs not specifying a {\tt factor} argument call into interface
   \ref{FieldRedistStoreNF}. If multiple PETs specify the {\tt factor} argument,
   its type and kind, as well as its value must match across all PETs. If none
   of the PETs specify a {\tt factor} argument the default will be a factor of
   1. The resulting factor is applied to all of the source data during
   redistribution, allowing scaling of the data, e.g. for unit transformation.
    
   Both {\tt srcField} and {\tt dstField} are interpreted as sequentialized 
   vectors. The sequence is defined by the order of DistGrid dimensions and the
   order of tiles within the DistGrid or by user-supplied arbitrary sequence
   indices. See section \ref{Array:SparseMatMul} for details on the definition
   of {\em sequence indices}.
  
   Source Field, destination Field, and the factor may be of different
   <type><kind>. Further, source and destination Fields may differ in shape,
   however, the number of elements must match. 
    
   If {\tt srcToDstTransposeMap} is not specified the redistribution corresponds
   to an identity mapping of the sequentialized source Field to the
   sequentialized destination Field. If the {\tt srcToDstTransposeMap}
   argument is provided it must be identical on all PETs. The
   {\tt srcToDstTransposeMap} allows source and destination Field dimensions to
   be transposed during the redistribution. The number of source and destination
   Field dimensions must be equal under this condition and the size of mapped
   dimensions must match.
    
   It is erroneous to specify the identical Field object for {\tt srcField} and
   {\tt dstField} arguments. 
  
     The routine returns an {\tt ESMF\_RouteHandle} that can be used to call 
     {\tt ESMF\_FieldRedist()} on any pair of Fields that matches 
     {\tt srcField} and {\tt dstField} in {\em type}, {\em kind}, and 
     memory layout of the {\em gridded} dimensions. However, the size, number, 
     and index order of {\em ungridded} dimensions may be different. See section
     \ref{RH:Reusability} for a more detailed discussion of RouteHandle 
     reusability.
  
   This method is overloaded for:\newline
   {\tt ESMF\_TYPEKIND\_I4}, {\tt ESMF\_TYPEKIND\_I8},\newline 
   {\tt ESMF\_TYPEKIND\_R4}, {\tt ESMF\_TYPEKIND\_R8}.
   \newline
    
   This call is {\em collective} across the current VM.  
   
   For examples and associated documentation regarding this method see Section
   \ref{sec:field:usage:redist_1dptr}. 
   
   The arguments are: 
   \begin{description} 
   \item [srcField]  
     {\tt ESMF\_Field} with source data. 
   \item [dstField] 
     {\tt ESMF\_Field} with destination data. The data in this Field may be
       destroyed by this call.
   \item [routehandle] 
     Handle to the precomputed Route. 
   \item [factor]
     Factor by which to multiply data. Default is 1. See full method
     description above for details on the interplay with other PETs.
   \item [{[srcToDstTransposeMap]}] 
     List with as many entries as there are dimensions in {\tt srcField}. Each
     entry maps the corresponding {\tt srcField} dimension against the specified
     {\tt dstField} dimension. Mixing of distributed and undistributed
     dimensions is supported.
   \item [{[ignoreUnmatchedIndices]}]
     A logical flag that affects the behavior for when not all elements match
     between the {\tt srcField} and {\tt dstField} side. The default setting
     is {\tt .false.}, indicating that it is an error when such a situation is 
     encountered. Setting {\tt ignoreUnmatchedIndices} to {\tt .true.} ignores
     unmatched indices.
   \item [{[rc]}]  
     Return code; equals {\tt ESMF\_SUCCESS} if there are no errors. 
   \end{description} 
    
%/////////////////////////////////////////////////////////////
 
\mbox{}\hrulefill\ 
 
\subsubsection [ESMF\_FieldRedistStore] {ESMF\_FieldRedistStore - Precompute Field redistribution without a local factor argument }


   
\bigskip{\sf INTERFACE:}
\begin{verbatim}   ! Private name; call using ESMF_FieldRedistStore() 
     subroutine ESMF_FieldRedistStoreNF(srcField, dstField, & 
         routehandle, srcToDstTransposeMap, &
         ignoreUnmatchedIndices, rc) \end{verbatim}{\em ARGUMENTS:}
\begin{verbatim}         type(ESMF_Field),       intent(in)            :: srcField  
         type(ESMF_Field),       intent(inout)         :: dstField  
         type(ESMF_RouteHandle), intent(inout)         :: routehandle
 -- The following arguments require argument keyword syntax (e.g. rc=rc). --
         integer,                intent(in), optional  :: srcToDstTransposeMap(:) 
         logical,                intent(in), optional  :: ignoreUnmatchedIndices
         integer,                intent(out), optional :: rc \end{verbatim}
{\sf STATUS:}
   \begin{itemize}
   \item\apiStatusCompatibleVersion{5.2.0r}
   \end{itemize}
  
{\sf DESCRIPTION:\\ }

 
   
   \label{FieldRedistStoreNF}
   {\tt ESMF\_FieldRedistStore()} is a collective method across all PETs of the
   current Component. The interface of the method is overloaded, allowing 
   -- in principle -- each PET to call into {\tt ESMF\_FieldRedistStore()}
   through a different entry point. Restrictions apply as to which combinations
   are sensible. All other combinations result in ESMF run time errors. The
   complete semantics of the {\tt ESMF\_FieldRedistStore()} method, as provided
   through the separate entry points shown in \ref{FieldRedistStoreTK} and
   \ref{FieldRedistStoreNF}, is described in the following paragraphs as a whole.
  
   Store a Field redistribution operation from {\tt srcField} to {\tt dstField}.
   Interface \ref{FieldRedistStoreTK} allows PETs to specify a {\tt factor}
   argument. PETs not specifying a {\tt factor} argument call into interface
   \ref{FieldRedistStoreNF}. If multiple PETs specify the {\tt factor} argument,
   its type and kind, as well as its value must match across all PETs. If none
   of the PETs specify a {\tt factor} argument the default will be a factor of
   1. The resulting factor is applied to all of the source data during
   redistribution, allowing scaling of the data, e.g. for unit transformation.
    
   Both {\tt srcField} and {\tt dstField} are interpreted as sequentialized 
   vectors. The sequence is defined by the order of DistGrid dimensions and the
   order of tiles within the DistGrid or by user-supplied arbitrary sequence
   indices. See section \ref{Array:SparseMatMul} for details on the definition
   of {\em sequence indices}.
  
   Source Field, destination Field, and the factor may be of different
   <type><kind>. Further, source and destination Fields may differ in shape,
   however, the number of elements must match. 
    
   If {\tt srcToDstTransposeMap} is not specified the redistribution corresponds
   to an identity mapping of the sequentialized source Field to the
   sequentialized destination Field. If the {\tt srcToDstTransposeMap}
   argument is provided it must be identical on all PETs. The
   {\tt srcToDstTransposeMap} allows source and destination Field dimensions to
   be transposed during the redistribution. The number of source and destination
   Field dimensions must be equal under this condition and the size of mapped
   dimensions must match.
    
   It is erroneous to specify the identical Field object for {\tt srcField} and
   {\tt dstField} arguments. 
  
     The routine returns an {\tt ESMF\_RouteHandle} that can be used to call 
     {\tt ESMF\_FieldRedist()} on any pair of Fields that matches 
     {\tt srcField} and {\tt dstField} in {\em type}, {\em kind}, and 
     memory layout of the {\em gridded} dimensions. However, the size, number, 
     and index order of {\em ungridded} dimensions may be different. See section
     \ref{RH:Reusability} for a more detailed discussion of RouteHandle 
     reusability.
    
   This call is {\em collective} across the current VM.  
   
   For examples and associated documentation regarding this method see Section
   \ref{sec:field:usage:redist_1dptr}. 
   
   The arguments are: 
   \begin{description} 
   \item [srcField]  
     {\tt ESMF\_Field} with source data. 
   \item [dstField] 
     {\tt ESMF\_Field} with destination data. The data in this Field may be
       destroyed by this call.
   \item [routehandle] 
     Handle to the precomputed Route. 
   \item [{[srcToDstTransposeMap]}] 
     List with as many entries as there are dimensions in {\tt srcField}. Each
     entry maps the corresponding {\tt srcField} dimension against the specified
     {\tt dstField} dimension. Mixing of distributed and undistributed
     dimensions is supported.
   \item [{[ignoreUnmatchedIndices]}]
     A logical flag that affects the behavior for when not all elements match
     between the {\tt srcField} and {\tt dstField} side. The default setting
     is {\tt .false.}, indicating that it is an error when such a situation is 
     encountered. Setting {\tt ignoreUnmatchedIndices} to {\tt .true.} ignores
     unmatched indices.
   \item [{[rc]}]  
     Return code; equals {\tt ESMF\_SUCCESS} if there are no errors. 
   \end{description} 
   
%...............................................................
\setlength{\parskip}{\oldparskip}
\setlength{\parindent}{\oldparindent}
\setlength{\baselineskip}{\oldbaselineskip}
