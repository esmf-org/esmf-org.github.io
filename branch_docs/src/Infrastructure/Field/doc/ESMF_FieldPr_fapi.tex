%                **** IMPORTANT NOTICE *****
% This LaTeX file has been automatically produced by ProTeX v. 1.1
% Any changes made to this file will likely be lost next time
% this file is regenerated from its source. Send questions 
% to Arlindo da Silva, dasilva@gsfc.nasa.gov
 
\setlength{\oldparskip}{\parskip}
\setlength{\parskip}{1.5ex}
\setlength{\oldparindent}{\parindent}
\setlength{\parindent}{0pt}
\setlength{\oldbaselineskip}{\baselineskip}
\setlength{\baselineskip}{11pt}
 
%--------------------- SHORT-HAND MACROS ----------------------
\def\bv{\begin{verbatim}}
\def\ev{\end{verbatim}}
\def\be{\begin{equation}}
\def\ee{\end{equation}}
\def\bea{\begin{eqnarray}}
\def\eea{\end{eqnarray}}
\def\bi{\begin{itemize}}
\def\ei{\end{itemize}}
\def\bn{\begin{enumerate}}
\def\en{\end{enumerate}}
\def\bd{\begin{description}}
\def\ed{\end{description}}
\def\({\left (}
\def\){\right )}
\def\[{\left [}
\def\]{\right ]}
\def\<{\left  \langle}
\def\>{\right \rangle}
\def\cI{{\cal I}}
\def\diag{\mathop{\rm diag}}
\def\tr{\mathop{\rm tr}}
%-------------------------------------------------------------

\markboth{Left}{Source File: ESMF\_FieldPr.F90,  Date: Tue May  5 21:00:00 MDT 2020
}

 
%/////////////////////////////////////////////////////////////
\subsubsection [ESMF\_FieldPrint] {ESMF\_FieldPrint - Print Field information}


 
\bigskip{\sf INTERFACE:}
\begin{verbatim}   subroutine ESMF_FieldPrint(field, rc)\end{verbatim}{\em ARGUMENTS:}
\begin{verbatim}     type(ESMF_Field), intent(in)            :: field
 -- The following arguments require argument keyword syntax (e.g. rc=rc). --
     integer,          intent(out), optional :: rc\end{verbatim}
{\sf STATUS:}
   \begin{itemize}
   \item\apiStatusCompatibleVersion{5.2.0r}
   \end{itemize}
  
{\sf DESCRIPTION:\\ }


       Prints information about the {\tt field} to {\tt stdout}.
       This subroutine goes through the internal data members of a field
       data type and prints information of each data member. \\
  
       The arguments are:
       \begin{description}
       \item [field]
             An {\tt ESMF\_Field} object.
       \item [{[rc]}]
             Return code; equals {\tt ESMF\_SUCCESS} if there are no errors.
       \end{description}
   
%/////////////////////////////////////////////////////////////
 
\mbox{}\hrulefill\ 
 
\subsubsection [ESMF\_FieldRead] {ESMF\_FieldRead - Read Field data from a file}


   \label{api:FieldRead}
 
\bigskip{\sf INTERFACE:}
\begin{verbatim}   subroutine ESMF_FieldRead(field, fileName,        &
       variableName, timeslice, iofmt, rc)\end{verbatim}{\em ARGUMENTS:}
\begin{verbatim}     type(ESMF_Field),      intent(inout)          :: field
     character(*),          intent(in)             :: fileName
 -- The following arguments require argument keyword syntax (e.g. rc=rc). --
     character(*),          intent(in),  optional  :: variableName
     integer,               intent(in),  optional  :: timeslice
     type(ESMF_IOFmt_Flag), intent(in),  optional  :: iofmt
     integer,               intent(out), optional  :: rc\end{verbatim}
{\sf DESCRIPTION:\\ }


     Read Field data from a file and put it into an {ESMF\_Field} object.
     For this API to be functional, the environment variable {\tt ESMF\_PIO}
     should be set to "internal" when the ESMF library is built.
     Please see the section on Data I/O,~\ref{io:dataio}.
  
     Limitations:
     \begin{itemize}
       \item Only single tile Fields are supported.
       \item Not supported in {\tt ESMF\_COMM=mpiuni} mode.
     \end{itemize}
  
     The arguments are:
     \begin{description}
     \item [field]
       The {\tt ESMF\_Field} object in which the read data is returned.
     \item[fileName]
       The name of the file from which Field data is read.
     \item[{[variableName]}]
      Variable name in the file; default is the "name" of Field.
      Use this argument only in the I/O format (such as NetCDF) that
      supports variable name. If the I/O format does not support this
      (such as binary format), ESMF will return an error code.
     \item[timeslice]
       Number of slices to be read from file, starting from the 1st slice
     \item[{[iofmt]}]
       \begin{sloppypar}
      The I/O format.  Please see Section~\ref{opt:iofmtflag} for the list
      of options. If not present, file names with a {\tt .bin} extension will
      use {\tt ESMF\_IOFMT\_BIN}, and file names with a {\tt .nc} extension
      will use {\tt ESMF\_IOFMT\_NETCDF}.  Other files default to
      {\tt ESMF\_IOFMT\_NETCDF}.
       \end{sloppypar}
     \item [{[rc]}]
       Return code; equals {\tt ESMF\_SUCCESS} if there are no errors.
     \end{description}
  
%...............................................................
\setlength{\parskip}{\oldparskip}
\setlength{\parindent}{\oldparindent}
\setlength{\baselineskip}{\oldbaselineskip}
