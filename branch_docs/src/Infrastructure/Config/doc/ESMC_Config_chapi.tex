%                **** IMPORTANT NOTICE *****
% This LaTeX file has been automatically produced by ProTeX v. 1.1
% Any changes made to this file will likely be lost next time
% this file is regenerated from its source. Send questions 
% to Arlindo da Silva, dasilva@gsfc.nasa.gov
 
\setlength{\oldparskip}{\parskip}
\setlength{\parskip}{1.5ex}
\setlength{\oldparindent}{\parindent}
\setlength{\parindent}{0pt}
\setlength{\oldbaselineskip}{\baselineskip}
\setlength{\baselineskip}{11pt}
 
%--------------------- SHORT-HAND MACROS ----------------------
\def\bv{\begin{verbatim}}
\def\ev{\end{verbatim}}
\def\be{\begin{equation}}
\def\ee{\end{equation}}
\def\bea{\begin{eqnarray}}
\def\eea{\end{eqnarray}}
\def\bi{\begin{itemize}}
\def\ei{\end{itemize}}
\def\bn{\begin{enumerate}}
\def\en{\end{enumerate}}
\def\bd{\begin{description}}
\def\ed{\end{description}}
\def\({\left (}
\def\){\right )}
\def\[{\left [}
\def\]{\right ]}
\def\<{\left  \langle}
\def\>{\right \rangle}
\def\cI{{\cal I}}
\def\diag{\mathop{\rm diag}}
\def\tr{\mathop{\rm tr}}
%-------------------------------------------------------------

\markboth{Left}{Source File: ESMC\_Config.h,  Date: Tue May  5 20:59:36 MDT 2020
}

 
%/////////////////////////////////////////////////////////////
\subsubsection [ESMC\_ConfigCreate] {ESMC\_ConfigCreate - Create a Config object}


  
\bigskip{\sf INTERFACE:}
\begin{verbatim} ESMC_Config ESMC_ConfigCreate(
   int* rc                    // out
 );\end{verbatim}{\em RETURN VALUE:}
\begin{verbatim}    ESMC_Config*  to newly allocated ESMC_Config\end{verbatim}
{\sf DESCRIPTION:\\ }


    Creates an {\tt ESMC\_Config} for use in subsequent calls.
  
     The arguments are:
     \begin{description}
     \item [{[rc]}]
       Return code; equals {\tt ESMF\_SUCCESS} if there are no errors.
     \end{description}
   
%/////////////////////////////////////////////////////////////
 
\mbox{}\hrulefill\ 
 
\subsubsection [ESMC\_ConfigDestroy] {ESMC\_ConfigDestroy - Destroy a Config object}


  
\bigskip{\sf INTERFACE:}
\begin{verbatim} int ESMC_ConfigDestroy(
   ESMC_Config* config        // in
 );\end{verbatim}{\em RETURN VALUE:}
\begin{verbatim}    Return code; equals ESMF_SUCCESS if there are no errors.\end{verbatim}
{\sf DESCRIPTION:\\ }


    Destroys the {\tt config} object.
  
     The arguments are:
     \begin{description}
     \item [config]
       Already created {\tt ESMC\_Config} object to destroy.
     \end{description}
   
%/////////////////////////////////////////////////////////////
 
\mbox{}\hrulefill\ 
 
\subsubsection [ESMC\_ConfigCreate] {ESMC\_ConfigCreate - Create a Config object from a section of an existing Config object}


  
\bigskip{\sf INTERFACE:}
\begin{verbatim} ESMC_Config ESMC_ConfigCreateFromSection(
   ESMC_Config config,       // in
   const char* olabel,       // in
   const char* clabel,       // in
   int* rc                   // out
 );\end{verbatim}{\em RETURN VALUE:}
\begin{verbatim}    ESMC_Config*  to newly allocated ESMC_Config\end{verbatim}
{\sf DESCRIPTION:\\ }


     Instantiates an {\tt ESMC\_Config} object from a section of an existing
     {\tt ESMC\_Config} object delimited by {\tt openlabel} and {\tt closelabel}.
     An error is returned if neither of the input labels is found in input
     {\tt config}.
  
     Note that a section is intended as the content of a given {\tt ESMC\_Config}
     object delimited by two distinct labels. Such content, as well as each of the
     surrounding labels, are still within the scope of the parent {\tt ESMC\_Config}
     object. Therefore, including in a section labels used outside that
     section should be done carefully to prevent parsing conflicts.
  
     The arguments are:
     \begin{description}
       \item[config]
         The input {\tt ESMC\_Config} object.
       \item[openlabel]
         Label marking the beginning of a section in {\tt config}.
       \item[closelabel]
         Label marking the end of a section in {\tt config}.
     \end{description}
   
%/////////////////////////////////////////////////////////////
 
\mbox{}\hrulefill\ 
 
\subsubsection [ESMC\_ConfigFindLabel] {ESMC\_ConfigFindLabel - Find a label}


  
\bigskip{\sf INTERFACE:}
\begin{verbatim} int ESMC_ConfigFindLabel(
   ESMC_Config config,        // in
   const char* label,         // in
   int *isPresent             // out
 );\end{verbatim}{\em RETURN VALUE:}
\begin{verbatim}    Return code; equals ESMF_SUCCESS if there are no errors.
    If label not found, and the {\tt isPresent} pointer is {\tt NULL},
    an error will be returned.\end{verbatim}
{\sf DESCRIPTION:\\ }


    Finds the {\tt label} (key) in the {\tt config} file. 
  
    Since the search is done by looking for a word in the 
    whole resource file, it is important to use special 
    conventions to distinguish labels from other words 
    in the resource files. The DAO convention is to finish 
    line labels by : and table labels by ::.
  
     The arguments are:
     \begin{description}
     \item [config]
       Already created {\tt ESMC\_Config} object.
     \item [label]
       Identifying label. 
     \item [{[isPresent]}]
       Label presence flag.  (optional).  If non-NULL, the target is
       set to 1 when the label is found; otherwise set to 0.
     \end{description}
   
%/////////////////////////////////////////////////////////////
 
\mbox{}\hrulefill\ 
 
\subsubsection [ESMC\_ConfigFindNextLabel] {ESMC\_ConfigFindNextLabel - Find a label}


  
\bigskip{\sf INTERFACE:}
\begin{verbatim} int ESMC_ConfigFindNextLabel(
   ESMC_Config config,        // in
   const char* label,         // in
   int *isPresent             // out
 );\end{verbatim}{\em RETURN VALUE:}
\begin{verbatim}    Return code; equals ESMF_SUCCESS if there are no errors.
    If label not found, and the {\tt isPresent} pointer is {\tt NULL},
    an error will be returned.\end{verbatim}
{\sf DESCRIPTION:\\ }

Finds the {\tt label} (key) string in the {\tt config} object, 
     starting from the current position pointer.
  
     This method is equivalent to {\tt ESMC\_ConfigFindLabel}, but the search
     is performed starting from the current position pointer.
  
     The arguments are:
     \begin{description}
     \item [config]
       Already created {\tt ESMC\_Config} object.
     \item [label]
       Identifying label.
     \item [isPresent]
       If non-NULL, the address specified is given a value of 1 if the
       label is found, and 0 when the label is not found.
     \end{description}
   
%/////////////////////////////////////////////////////////////
 
\mbox{}\hrulefill\ 
 
\subsubsection [ESMC\_ConfigGetDim] {ESMC\_ConfigGetDim - Get table sizes}


  
\bigskip{\sf INTERFACE:}
\begin{verbatim} int ESMC_ConfigGetDim(
   ESMC_Config config,        // in
   int* lineCount,            // out
   int* columnCount,          // out
   ...                        // optional argument list
 );\end{verbatim}{\em RETURN VALUE:}
\begin{verbatim}    Return code; equals ESMF_SUCCESS if there are no errors.\end{verbatim}
{\sf DESCRIPTION:\\ }


    Returns the number of lines in the table in {\tt lineCount} and
    the maximum number of words in a table line in {\tt columnCount}.
  
     The arguments are:
     \begin{description}
     \item [config]
       Already created {\tt ESMC\_Config} object.
     \item [lineCount]
       Returned number of lines in the table. 
     \item [columnCount]
       Returned maximum number of words in a table line. 
     \item [{[label]}]
       Identifying label (optional).
     \end{description}
  
    Due to this method accepting optional arguments, the final argument
    must be {\tt ESMC\_ArgLast}.
   
%/////////////////////////////////////////////////////////////
 
\mbox{}\hrulefill\ 
 
\subsubsection [ESMC\_ConfigGetLen] {ESMC\_ConfigGetLen - Get the length of the line in words}


  
\bigskip{\sf INTERFACE:}
\begin{verbatim} int ESMC_ConfigGetLen(
   ESMC_Config config,        // in
   int* wordCount,            // out
   ...                        // optional argument list
 );\end{verbatim}{\em RETURN VALUE:}
\begin{verbatim}    Return code; equals ESMF_SUCCESS if there are no errors.\end{verbatim}
{\sf DESCRIPTION:\\ }


    Gets the length of the line in words by counting words
    disregarding types.  Returns the word count as an integer.
  
     The arguments are:
     \begin{description}
     \item [config]
       Already created {\tt ESMC\_Config} object.
     \item [wordCount]
       Returned number of words in the line. 
     \item [{[label]}]
       Identifying label.  If not specified, use the current line (optional).
     \end{description}
  
    Due to this method accepting optional arguments, the final argument
    must be {\tt ESMC\_ArgLast}.
   
%/////////////////////////////////////////////////////////////
 
\mbox{}\hrulefill\ 
 
\subsubsection [ESMC\_ConfigLoadFile] {ESMC\_ConfigLoadFile - Load resource file into memory}


  
\bigskip{\sf INTERFACE:}
\begin{verbatim} int ESMC_ConfigLoadFile(
   ESMC_Config config,        // in
   const char* file,          // in
   ...                        // optional argument list
 );\end{verbatim}{\em RETURN VALUE:}
\begin{verbatim}    Return code; equals ESMF_SUCCESS if there are no errors.\end{verbatim}
{\sf DESCRIPTION:\\ }


    Resource file with {\tt filename} is loaded into memory.
  
     The arguments are:
     \begin{description}
     \item [config]
       Already created {\tt ESMC\_Config} object.
     \item [file]
       Configuration file name.
     \item [{[delayout]}]
       {\tt ESMC\_DELayout} associated with this {\tt config} object.
       **NOTE: This argument is not currently supported.
     \item [{[unique]}]
       If specified as true, uniqueness of labels are checked and 
       error code set if duplicates found (optional).
     \end{description}
  
    Due to this method accepting optional arguments, the final argument
    must be {\tt ESMC\_ArgLast}.
   
%/////////////////////////////////////////////////////////////
 
\mbox{}\hrulefill\ 
 
\subsubsection [ESMC\_ConfigNextLine] {ESMC\_ConfigNextLine - Find next line}


  
\bigskip{\sf INTERFACE:}
\begin{verbatim} int ESMC_ConfigNextLine(
   ESMC_Config config,       // in
   int *tableEnd             // out
 );\end{verbatim}{\em RETURN VALUE:}
\begin{verbatim}    Return code; equals ESMF_SUCCESS if there are no errors.\end{verbatim}
{\sf DESCRIPTION:\\ }


    Selects the next line (for tables).
  
     The arguments are:
     \begin{description}
     \item [config]
       Already created {\tt ESMC\_Config} object.
     \item [{[tableEnd]}]
       End of table mark (::) found flag.  Returns 1 when found, and 0 when
       not found.
     \end{description}
   
%/////////////////////////////////////////////////////////////
 
\mbox{}\hrulefill\ 
 
\subsubsection [ESMC\_ConfigPrint] {ESMC\_ConfigPrint - Write content of config object to standard output}


  
\bigskip{\sf INTERFACE:}
\begin{verbatim} int ESMC_ConfigPrint(
   ESMC_Config config        // in
 );\end{verbatim}{\em RETURN VALUE:}
\begin{verbatim}    Return code; equals ESMF_SUCCESS if there are no errors.\end{verbatim}
{\sf DESCRIPTION:\\ }


     Write content of a {\tt ESMC\_Config} object to standard output.
  
     The arguments are:
     \begin{description}
     \item [config]
       Already created {\tt ESMC\_Config} object.
     \end{description}
   
%/////////////////////////////////////////////////////////////
 
\mbox{}\hrulefill\ 
 
\subsubsection [ESMC\_ConfigValidate] {ESMC\_ConfigValidate - Validate a Config object}


  
\bigskip{\sf INTERFACE:}
\begin{verbatim} int ESMC_ConfigValidate(
   ESMC_Config config,        // in
   ...                        // optional argument list
 );\end{verbatim}{\em RETURN VALUE:}
\begin{verbatim}    Return code; equals ESMF_SUCCESS if there are no errors.
    Equals ESMF_RC_ATTR_UNUSED if any unused attributes are found
    with option "unusedAttributes" below.\end{verbatim}
{\sf DESCRIPTION:\\ }


     Checks whether a {\tt config} object is valid.
  
     The arguments are:
     \begin{description}
     \item [config]
       Already created {\tt ESMC\_Config} object.
     \item[{[options]}]
       If none specified:  simply check that the buffer is not full and the
         pointers are within range (optional).
       "unusedAttributes" - Report to the default logfile all attributes not
         retrieved via a call to {\tt ESMC\_ConfigGetAttribute()} or
         {\tt ESMC\_ConfigGetChar()}.  The attribute name (label) will be
         logged via {\tt ESMC\_LogErr} with the WARNING log message type.
         For an array-valued attribute, retrieving at least one value via
         {\tt ESMC\_ConfigGetAttribute()} or {\tt ESMC\_ConfigGetChar()}
         constitutes being "used."
     \end{description}
  
    Due to this method accepting optional arguments, the final argument
    must be {\tt ESMC\_ArgLast}.
  
%...............................................................
\setlength{\parskip}{\oldparskip}
\setlength{\parindent}{\oldparindent}
\setlength{\baselineskip}{\oldbaselineskip}
