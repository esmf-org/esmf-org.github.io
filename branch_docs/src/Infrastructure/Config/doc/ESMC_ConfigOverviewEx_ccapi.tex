%                **** IMPORTANT NOTICE *****
% This LaTeX file has been automatically produced by ProTeX v. 1.1
% Any changes made to this file will likely be lost next time
% this file is regenerated from its source. Send questions 
% to Arlindo da Silva, dasilva@gsfc.nasa.gov
 
\setlength{\oldparskip}{\parskip}
\setlength{\parskip}{1.5ex}
\setlength{\oldparindent}{\parindent}
\setlength{\parindent}{0pt}
\setlength{\oldbaselineskip}{\baselineskip}
\setlength{\baselineskip}{11pt}
 
%--------------------- SHORT-HAND MACROS ----------------------
\def\bv{\begin{verbatim}}
\def\ev{\end{verbatim}}
\def\be{\begin{equation}}
\def\ee{\end{equation}}
\def\bea{\begin{eqnarray}}
\def\eea{\end{eqnarray}}
\def\bi{\begin{itemize}}
\def\ei{\end{itemize}}
\def\bn{\begin{enumerate}}
\def\en{\end{enumerate}}
\def\bd{\begin{description}}
\def\ed{\end{description}}
\def\({\left (}
\def\){\right )}
\def\[{\left [}
\def\]{\right ]}
\def\<{\left  \langle}
\def\>{\right \rangle}
\def\cI{{\cal I}}
\def\diag{\mathop{\rm diag}}
\def\tr{\mathop{\rm tr}}
%-------------------------------------------------------------

\markboth{Left}{Source File: ESMC\_ConfigOverviewEx.C,  Date: Tue May  5 20:59:36 MDT 2020
}

 
%/////////////////////////////////////////////////////////////

  \subsubsection{Common Code Arguments}
 
   Common Arguments used in the following code fragments: 
%/////////////////////////////////////////////////////////////

 \begin{verbatim}
      char fname [] = "myResourceFile.rc";    // file name
      char fn1[10], fn2[10], fn3[10];
      int rc;             // error return code (0 is OK)
      ESMC_I4 n = 0;
      ESMC_R4 r = 0.0;
      ESMC_R4 table[7][3];

      ESMC_Config *cf;
 
\end{verbatim}
 
%/////////////////////////////////////////////////////////////

  \subsubsection{Creation of a Config}
 
   The first step is to create the {\tt ESMC\_Config} and load the
   ASCII resource (rc) file into memory\footnote{See next section
   for a complete description of parameters for each routine/function}: 
%/////////////////////////////////////////////////////////////

 \begin{verbatim}
      cf = ESMC_ConfigCreate(&rc);
 
\end{verbatim}
 
%/////////////////////////////////////////////////////////////

 \begin{verbatim}
      rc = ESMC_ConfigLoadFile(cf, fname, ESMC_ArgLast);
 
\end{verbatim}
 
%/////////////////////////////////////////////////////////////

  \subsubsection{Retrieval of constants}
 
   The next step is to select the label (record) of interest, say 
%/////////////////////////////////////////////////////////////

 \begin{verbatim}
      rc = ESMC_ConfigFindLabel(cf, "constants:");
 
\end{verbatim}
 
%/////////////////////////////////////////////////////////////

   Two constants, r and n, can be retrieved with the following code
   fragment: 
%/////////////////////////////////////////////////////////////

 \begin{verbatim}
      rc = ESMC_ConfigGetAttribute(cf, &r, ESMC_TYPEKIND_R4,
                      ESMC_ArgLast);  // results in r = 3.1415
      rc = ESMC_ConfigGetAttribute(cf, &n, ESMC_TYPEKIND_I4,
                      ESMC_ArgLast);  // results in n = 25
 
\end{verbatim}
 
%/////////////////////////////////////////////////////////////

  \subsubsection{Retrieval of file names}
 
   File names can be retrieved with the following code fragment: 
%/////////////////////////////////////////////////////////////

 \begin{verbatim}
      rc = ESMC_ConfigFindLabel(cf, "my_file_names:");
 
\end{verbatim}
 
%/////////////////////////////////////////////////////////////

 \begin{verbatim}
      rc = ESMC_ConfigGetAttribute(cf, fn1, ESMC_TYPEKIND_CHARACTER,
                      ESMC_ArgLast);  // results in fn1 = 'jan87.dat'
      rc = ESMC_ConfigGetAttribute(cf, fn2, ESMC_TYPEKIND_CHARACTER,
                      ESMC_ArgLast);  // results in fn2 = 'jan88.dat'
      rc = ESMC_ConfigGetAttribute(cf, fn3, ESMC_TYPEKIND_CHARACTER,
                      ESMC_ArgLast);  // results in fn3 = 'jan89.dat'
 
\end{verbatim}
 
%/////////////////////////////////////////////////////////////

  \subsubsection{Retrieval of tables}
 
   To access tabular data, the user first must use
   {\tt ESMC\_ConfigFindLabel()} to locate the beginning of the table, e.g., 
%/////////////////////////////////////////////////////////////

 \begin{verbatim}
      rc = ESMC_ConfigFindLabel(cf, "my_table_name::");
 
\end{verbatim}
 
%/////////////////////////////////////////////////////////////

   Subsequently, {\tt call ESMC\_ConfigNextLine()} can be used to gain
   access to each row of the table. Here is a code fragment to read the above
   table (7 rows, 3 columns): 
%/////////////////////////////////////////////////////////////

 \begin{verbatim}
      for (i = 0; i < 7; i++) {
        rc = ESMC_ConfigNextLine(cf, ESMC_ArgLast);
        for (j = 0; j < 3; j++) {
          rc = ESMC_ConfigGetAttribute(cf, (&table[i][j]), ESMC_TYPEKIND_R4,
                          ESMC_ArgLast);
        }
      }
 
\end{verbatim}
 
%/////////////////////////////////////////////////////////////

  \subsubsection{Destruction of a Config}
 
   The work with the configuration file {\tt cf} is finalized by call to
   {\tt ESMC\_ConfigDestroy()}: 
%/////////////////////////////////////////////////////////////

 \begin{verbatim}
      rc = ESMC_ConfigDestroy(cf);
 
\end{verbatim}

%...............................................................
\setlength{\parskip}{\oldparskip}
\setlength{\parindent}{\oldparindent}
\setlength{\baselineskip}{\oldbaselineskip}
