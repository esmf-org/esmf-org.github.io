%                **** IMPORTANT NOTICE *****
% This LaTeX file has been automatically produced by ProTeX v. 1.1
% Any changes made to this file will likely be lost next time
% this file is regenerated from its source. Send questions 
% to Arlindo da Silva, dasilva@gsfc.nasa.gov
 
\setlength{\oldparskip}{\parskip}
\setlength{\parskip}{1.5ex}
\setlength{\oldparindent}{\parindent}
\setlength{\parindent}{0pt}
\setlength{\oldbaselineskip}{\baselineskip}
\setlength{\baselineskip}{11pt}
 
%--------------------- SHORT-HAND MACROS ----------------------
\def\bv{\begin{verbatim}}
\def\ev{\end{verbatim}}
\def\be{\begin{equation}}
\def\ee{\end{equation}}
\def\bea{\begin{eqnarray}}
\def\eea{\end{eqnarray}}
\def\bi{\begin{itemize}}
\def\ei{\end{itemize}}
\def\bn{\begin{enumerate}}
\def\en{\end{enumerate}}
\def\bd{\begin{description}}
\def\ed{\end{description}}
\def\({\left (}
\def\){\right )}
\def\[{\left [}
\def\]{\right ]}
\def\<{\left  \langle}
\def\>{\right \rangle}
\def\cI{{\cal I}}
\def\diag{\mathop{\rm diag}}
\def\tr{\mathop{\rm tr}}
%-------------------------------------------------------------

\markboth{Left}{Source File: ESMF\_ConfigOverviewEx.F90,  Date: Tue May  5 20:59:36 MDT 2020
}

 
%/////////////////////////////////////////////////////////////

   This example/test code performs simple Config/Resource File routines. It does not
   include attaching a Config to a component. The important thing to remember there
   is that you can have one Config per component. 
  
   There are two methodologies for accessing data in a Resource File.  This example will
   demonstrate both.
 
   Note the API section contains a complete description of arguments in
   the methods/functions demonstrated in this example. 
%/////////////////////////////////////////////////////////////

  \subsubsection{Variable declarations}
 
   The following are the variable declarations used as arguments in the following code 
   fragments. They represent the locals names for the variables listed in the Resource 
   File (RF).  Note they do not need to be the same. 
%/////////////////////////////////////////////////////////////

 \begin{verbatim}
      character(ESMF_MAXPATHLEN) :: fname ! config file name
      character(ESMF_MAXPATHLEN) :: fn1, fn2, fn3, input_file ! strings to be read in
      integer       :: rc            ! error return code (0 is OK)
      integer       :: i_n           ! the first constant in the RF
      real          :: param_1       ! the second constant in the RF
      real          :: radius        ! radius of the earth
      real          :: table(7,3)    ! an array to hold the table in the RF

      type(ESMF_Config)   :: cf      ! the Config itself
 
\end{verbatim}
 
%/////////////////////////////////////////////////////////////

  \subsubsection{Creation of a Config}
 
   While there are two methodologies for accessing the data within a Resource File, 
   there is only one way to create the initial Config and load its ASCII text into 
   memory. This is the first step in the process.
 
   Note that subsequent calls to {\tt ESMF\_ConfigLoadFile} will OVERWRITE the current
   Config NOT append to it. There is no means of appending to a Config. 
   
%/////////////////////////////////////////////////////////////

 \begin{verbatim}
      cf = ESMF_ConfigCreate(rc=rc)             ! Create the empty Config
 
\end{verbatim}
 
%/////////////////////////////////////////////////////////////

 \begin{verbatim}
      fname = "myResourceFile.rc"                ! Name the Resource File
      call ESMF_ConfigLoadFile(cf, fname, rc=rc) ! Load the Resource File 
                                                 ! into the empty Config
 
\end{verbatim}
 
%/////////////////////////////////////////////////////////////

  \subsubsection{How to retrieve a label with a single value}
   The first method for retrieving information from the 
   Resource File takes advantage of the <label,value> relationship
   within the file and access the data in a dictionary-like manner. This is the
   simplest methodology, but it does imply the use of only one value per label
   in the Resource File.  
   
   Remember,
   that the order in which a particular label/value pair is retrieved
   is not dependent upon the order which they exist within the Resource File.  
%/////////////////////////////////////////////////////////////

 \begin{verbatim}
    call ESMF_ConfigGetAttribute(cf, radius, label='radius_of_the_earth:', &
                                 default=1.0, rc=rc)
 
\end{verbatim}
 
%/////////////////////////////////////////////////////////////

   Note that the colon must be included in the label string when using this
   methodology.  It is also important to provide a default value in case the label
   does not exist in the file 
%/////////////////////////////////////////////////////////////

   This methodology works for all types. The following is an example of retrieving a 
   string: 
%/////////////////////////////////////////////////////////////

 \begin{verbatim}
    call ESMF_ConfigGetAttribute(cf, input_file, label='input_file_name:', &
                                 default="./default.nc", rc=rc)
 
\end{verbatim}
 
%/////////////////////////////////////////////////////////////

   The same code fragment can be used to demonstrate what happens when the label is not 
   present.  Note that "file\_name" does not exist in the Resource File. The result of 
   its absence is the default value provided in the call. 
%/////////////////////////////////////////////////////////////

 \begin{verbatim}
    call ESMF_ConfigGetAttribute(cf, input_file, label='file_name:', &
                                 default="./default.nc", rc=rc)
 
\end{verbatim}
 
%/////////////////////////////////////////////////////////////

  \subsubsection{How to retrieve a label with multiple values}
   When there are multiple, mixed-typed values associated with a label, the 
   values can be retrieved in two steps:  1) Use ESMF\_ConfigFindLabel() 
   to find the label in the Config class; 2) use
   ESMF\_ConfigGetAttribute() without the optional 'label' argument to 
   retrieve the values one at a time, reading from left to right in
   the record. 
  
   A second reminder that the order in which a particular label/value pair is 
   retrieved is not dependent upon the order which they exist within the 
   Resource File. The label used in this method allows the user to skip to
   any point in the file.  
%/////////////////////////////////////////////////////////////

 \begin{verbatim}
      call ESMF_ConfigFindLabel(cf, 'constants:', rc=rc) ! Step a) Find the 
                                                         ! label 
 
\end{verbatim}
 
%/////////////////////////////////////////////////////////////

   Two constants, radius and i\_n, can now be retrieved without having to specify their
   label or use an array. They are also different types. 
%/////////////////////////////////////////////////////////////

 \begin{verbatim}
      call ESMF_ConfigGetAttribute(cf, param_1, rc=rc) ! Step b) read in the 
                                                       ! first constant in 
                                                       ! the sequence
      call ESMF_ConfigGetAttribute(cf, i_n, rc=rc)     ! Step c) read in the 
                                                       ! second constant in 
                                                       ! the sequence
 
\end{verbatim}
 
%/////////////////////////////////////////////////////////////

   This methodology also works with strings. 
%/////////////////////////////////////////////////////////////

 \begin{verbatim}
       call ESMF_ConfigFindLabel(cf, 'my_file_names:', &
               rc=rc)                       ! Step a) find the label
 
\end{verbatim}
 
%/////////////////////////////////////////////////////////////

 \begin{verbatim}
       call ESMF_ConfigGetAttribute(cf, fn1, &
                 rc=rc)                    ! Step b) retrieve the 1st filename
       call ESMF_ConfigGetAttribute(cf, fn2, &
                 rc=rc)                    ! Step c) retrieve the 2nd filename
       call ESMF_ConfigGetAttribute(cf, fn3, &
                 rc=rc)                    ! Step d) retrieve the 3rd filename
 
\end{verbatim}
 
%/////////////////////////////////////////////////////////////

  \subsubsection{How to retrieve a table}
 
   To access tabular data, the user must use the multi-value method.  
%/////////////////////////////////////////////////////////////

 \begin{verbatim}
      call ESMF_ConfigFindLabel(cf, 'my_table_name::', &
               rc=rc)        ! Step a) Set the label location to the 
                             ! beginning of the table
 
\end{verbatim}
 
%/////////////////////////////////////////////////////////////

   Subsequently, {\tt call ESMF\_ConfigNextLine()} is used to move the location 
   to the next row of the table. The example table in the Resource File contains
   7 rows and 3 columns (7,3). 
%/////////////////////////////////////////////////////////////

 \begin{verbatim}
      do i = 1, 7
        call ESMF_ConfigNextLine(cf, rc=rc) ! Step b) Increment the rows
        do j = 1, 3                         ! Step c) Fill in the table 
          call ESMF_ConfigGetAttribute(cf, table(i,j), rc=rc)
        enddo
      enddo
 
\end{verbatim}
 
%/////////////////////////////////////////////////////////////

  \subsubsection{Destruction of a Config}
 
   The work with the configuration file {\tt cf} is finalized by call to
   {\tt ESMF\_ConfigDestroy()}: 
%/////////////////////////////////////////////////////////////

 \begin{verbatim}
      call ESMF_ConfigDestroy(cf, rc=rc) ! Destroy the Config
 
\end{verbatim}

%...............................................................
\setlength{\parskip}{\oldparskip}
\setlength{\parindent}{\oldparindent}
\setlength{\baselineskip}{\oldbaselineskip}
