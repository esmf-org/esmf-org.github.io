%                **** IMPORTANT NOTICE *****
% This LaTeX file has been automatically produced by ProTeX v. 1.1
% Any changes made to this file will likely be lost next time
% this file is regenerated from its source. Send questions 
% to Arlindo da Silva, dasilva@gsfc.nasa.gov
 
\setlength{\oldparskip}{\parskip}
\setlength{\parskip}{1.5ex}
\setlength{\oldparindent}{\parindent}
\setlength{\parindent}{0pt}
\setlength{\oldbaselineskip}{\baselineskip}
\setlength{\baselineskip}{11pt}
 
%--------------------- SHORT-HAND MACROS ----------------------
\def\bv{\begin{verbatim}}
\def\ev{\end{verbatim}}
\def\be{\begin{equation}}
\def\ee{\end{equation}}
\def\bea{\begin{eqnarray}}
\def\eea{\end{eqnarray}}
\def\bi{\begin{itemize}}
\def\ei{\end{itemize}}
\def\bn{\begin{enumerate}}
\def\en{\end{enumerate}}
\def\bd{\begin{description}}
\def\ed{\end{description}}
\def\({\left (}
\def\){\right )}
\def\[{\left [}
\def\]{\right ]}
\def\<{\left  \langle}
\def\>{\right \rangle}
\def\cI{{\cal I}}
\def\diag{\mathop{\rm diag}}
\def\tr{\mathop{\rm tr}}
%-------------------------------------------------------------

\markboth{Left}{Source File: ESMF\_Config.F90,  Date: Tue May  5 20:59:36 MDT 2020
}

 
%/////////////////////////////////////////////////////////////
\subsubsection [ESMF\_ConfigAssignment(=)] {ESMF\_ConfigAssignment(=) - Config assignment}


  
\bigskip{\sf INTERFACE:}
\begin{verbatim}     interface assignment(=)
     config1 = config2\end{verbatim}{\em ARGUMENTS:}
\begin{verbatim}     type(ESMF_Config) :: config1
     type(ESMF_Config) :: config2\end{verbatim}
{\sf DESCRIPTION:\\ }


     Assign {\tt config1} as an alias to the same {\tt ESMF\_Config} object in memory
     as {\tt config2}. If {\tt config2} is invalid, then {\tt config1} will be
     equally invalid after the assignment.
  
     The arguments are:
     \begin{description}
     \item[config1]
       The {\tt ESMF\_Config} object on the left hand side of the assignment.
     \item[config2]
       The {\tt ESMF\_Config} object on the right hand side of the assignment.
     \end{description}
   
%/////////////////////////////////////////////////////////////
 
\mbox{}\hrulefill\ 
 
\subsubsection [ESMF\_ConfigOperator(==)] {ESMF\_ConfigOperator(==) - Test if Config objects are equivalent}


  
\bigskip{\sf INTERFACE:}
\begin{verbatim}       interface operator(==)
       if (config1 == config2) then ... endif
                    OR
       result = (config1 == config2)\end{verbatim}{\em RETURN VALUE:}
\begin{verbatim}       configical :: result\end{verbatim}{\em ARGUMENTS:}
\begin{verbatim}       type(ESMF_Config), intent(in) :: config1
       type(ESMF_Config), intent(in) :: config2\end{verbatim}
{\sf DESCRIPTION:\\ }


       Overloads the (==) operator for the {\tt ESMF\_Config} class.
       Compare two configs for equality; return {\tt .true.} if equal,
       {\tt .false.} otherwise. Comparison is based on whether the objects
       are distinct, as with two newly created objects, or are simply aliases
       to the same object as would be the case when assignment was involved.
  
       The arguments are:
       \begin{description}
       \item[config1]
            The {\tt ESMF\_Config} object on the left hand side of the equality
            operation.
       \item[config2]
            The {\tt ESMF\_Config} object on the right hand side of the equality
            operation.
       \end{description}
   
%/////////////////////////////////////////////////////////////
 
\mbox{}\hrulefill\ 
 
\subsubsection [ESMF\_ConfigOperator(/=)] {ESMF\_ConfigOperator(/=) - Test if Config objects are not equivalent}


  
\bigskip{\sf INTERFACE:}
\begin{verbatim}       interface operator(/=)
       if (config1 /= config2) then ... endif
                    OR
       result = (config1 /= config2)\end{verbatim}{\em RETURN VALUE:}
\begin{verbatim}       configical :: result\end{verbatim}{\em ARGUMENTS:}
\begin{verbatim}       type(ESMF_Config), intent(in) :: config1
       type(ESMF_Config), intent(in) :: config2\end{verbatim}
{\sf DESCRIPTION:\\ }


       Overloads the (/=) operator for the {\tt ESMF\_Config} class.
       Compare two configs for equality; return {\tt .true.} if not equivalent,
       {\tt .false.} otherwise. Comparison is based on whether the Config objects
       are distinct, as with two newly created objects, or are simply aliases
       to the same object as would be the case when assignment was involved.
  
       The arguments are:
       \begin{description}
       \item[config1]
            The {\tt ESMF\_Config} object on the left hand side of the equality
            operation.
       \item[config2]
            The {\tt ESMF\_Config} object on the right hand side of the equality
            operation.
       \end{description}
   
%/////////////////////////////////////////////////////////////
 
\mbox{}\hrulefill\ 
 

  \subsubsection [ESMF\_ConfigCreate] {ESMF\_ConfigCreate - Instantiate a Config object}


  
\bigskip{\sf INTERFACE:}
\begin{verbatim}       ! Private name; call using ESMF_ConfigCreate()
       type(ESMF_Config) function ESMF_ConfigCreateEmpty(rc)
 \end{verbatim}{\em ARGUMENTS:}
\begin{verbatim} -- The following arguments require argument keyword syntax (e.g. rc=rc). --
      integer,intent(out), optional              :: rc \end{verbatim}
{\sf STATUS:}
   \begin{itemize}
   \item\apiStatusCompatibleVersion{5.2.0r}
   \end{itemize}
  
{\sf DESCRIPTION:\\ }

 
     Instantiates an {\tt ESMF\_Config} object for use in subsequent calls.
  
     The arguments are:
     \begin{description}
     \item [{[rc]}]
       Return code; equals {\tt ESMF\_SUCCESS} if there are no errors.
     \end{description}
   
%/////////////////////////////////////////////////////////////
 
\mbox{}\hrulefill\ 
 

  \subsubsection [ESMF\_ConfigCreate] {ESMF\_ConfigCreate - Instantiate a new Config object from a Config section}


  
\bigskip{\sf INTERFACE:}
\begin{verbatim}     ! Private name; call using ESMF_ConfigCreate()
     type(ESMF_Config) function ESMF_ConfigCreateFromSection(config, &
       openlabel, closelabel, rc)
 \end{verbatim}{\em ARGUMENTS:}
\begin{verbatim}       type(ESMF_Config)             :: config
       character(len=*),  intent(in) :: openlabel, closelabel
 -- The following arguments require argument keyword syntax (e.g. rc=rc). --
       integer,intent(out), optional :: rc\end{verbatim}
{\sf DESCRIPTION:\\ }


     Instantiates an {\tt ESMF\_Config} object from a section of an existing
     {\tt ESMF\_Config} object delimited by {\tt openlabel} and {\tt closelabel}.
     An error is returned if neither of the input labels is found in input
     {\tt config}.
  
     Note that a section is intended as the content of a given {\tt ESMF\_Config}
     object delimited by two distinct labels. Such content, as well as each of the
     surrounding labels, are still within the scope of the parent {\tt ESMF\_Config}
     object. Therefore, including in a section labels used outside that
     section should be done carefully to prevent parsing conflicts.
  
     The arguments are:
     \begin{description}
       \item[config]
         The input {\tt ESMF\_Config} object.
       \item[openlabel]
         Label marking the beginning of a section in {\tt config}.
       \item[closelabel]
         Label marking the end of a section in {\tt config}.
       \item [{[rc]}]
         Return code; equals {\tt ESMF\_SUCCESS} if a section is found
        and a new {\tt ESMF\_Config} object returned.
     \end{description}
   
%/////////////////////////////////////////////////////////////
 
\mbox{}\hrulefill\ 
 

  \subsubsection [ESMF\_ConfigDestroy] {ESMF\_ConfigDestroy - Destroy a Config object}


  
\bigskip{\sf INTERFACE:}
\begin{verbatim}     subroutine ESMF_ConfigDestroy(config, rc)
 \end{verbatim}{\em ARGUMENTS:}
\begin{verbatim}       type(ESMF_Config), intent(inout)          :: config
 -- The following arguments require argument keyword syntax (e.g. rc=rc). --
       integer,           intent(out),  optional :: rc\end{verbatim}
{\sf STATUS:}
   \begin{itemize}
   \item\apiStatusCompatibleVersion{5.2.0r}
   \end{itemize}
  
{\sf DESCRIPTION:\\ }

 
      Destroys the {\tt config} object.
  
     The arguments are:
     \begin{description}
     \item [config]
       Already created {\tt ESMF\_Config} object.
     \item [{[rc]}]
       Return code; equals {\tt ESMF\_SUCCESS} if there are no errors.
     \end{description}
   
%/////////////////////////////////////////////////////////////
 
\mbox{}\hrulefill\ 
 

  \subsubsection [ESMF\_ConfigFindLabel] {ESMF\_ConfigFindLabel - Find a label in a Config object}


  
\bigskip{\sf INTERFACE:}
\begin{verbatim}     subroutine ESMF_ConfigFindLabel(config, label, isPresent, rc)
 \end{verbatim}{\em ARGUMENTS:}
\begin{verbatim}       type(ESMF_Config), intent(inout)           :: config 
       character(len=*),  intent(in)              :: label
 -- The following arguments require argument keyword syntax (e.g. rc=rc). --
       logical,           intent(out),  optional  :: isPresent
       integer,           intent(out),  optional  :: rc 
 \end{verbatim}
{\sf STATUS:}
   \begin{itemize}
   \item\apiStatusCompatibleVersion{5.2.0r}
   \item\apiStatusModifiedSinceVersion{5.2.0r}
   \begin{description}
   \item[6.1.0] Added the {\tt isPresent} argument.  Allows detection of
    end-of-line condition to be separate from the {\tt rc}.
   \end{description}
   \end{itemize}
  
{\sf DESCRIPTION:\\ }

Finds the {\tt label} (key) string in the {\tt config} object 
     starting from the beginning of its content.
  
     Since the search is done by looking for a string, possibly multi-worded,
     in the whole {\tt Config} object, it is important to use special 
     conventions to distinguish {\tt labels} from other words. This is done 
     in the Resource File by using the NASA/DAO convention to finish
     line labels with a colon (:) and table labels with a double colon (::).
  
  
     The arguments are:
     \begin{description}
     \item [config]
       Already created {\tt ESMF\_Config} object.
     \item [label]
       Identifying label. 
     \item [{[isPresent]}]
       Set to {\tt .true.} if the item is found.
     \item [{[rc]}]
       Return code; equals {\tt ESMF\_SUCCESS} if there are no errors.
       If the label is not found, and the {\tt isPresent} argument is
       not present, an error is returned.
     \end{description}
   
%/////////////////////////////////////////////////////////////
 
\mbox{}\hrulefill\ 
 

  \subsubsection [ESMF\_ConfigFindNextLabel] {ESMF\_ConfigFindNextLabel - Find a label in Config object starting from current position}


  
\bigskip{\sf INTERFACE:}
\begin{verbatim}     subroutine ESMF_ConfigFindNextLabel(config, label, isPresent, rc)
 \end{verbatim}{\em ARGUMENTS:}
\begin{verbatim}       type(ESMF_Config), intent(inout)           :: config
       character(len=*),  intent(in)              :: label
 -- The following arguments require argument keyword syntax (e.g. rc=rc). --
       logical,           intent(out),  optional  :: isPresent
       integer,           intent(out),  optional  :: rc
 \end{verbatim}
{\sf DESCRIPTION:\\ }

Finds the {\tt label} (key) string in the {\tt config} object, 
     starting from the current position pointer.
  
     This method is equivalent to {\tt ESMF\_ConfigFindLabel}, but the search
     is performed starting from the current position pointer.
  
     The arguments are:
     \begin{description}
     \item [config]
       Already created {\tt ESMF\_Config} object.
     \item [label]
       Identifying label.
     \item [{[isPresent]}]
       Set to {\tt .true.} if the item is found.
     \item [{[rc]}]
       Return code; equals {\tt ESMF\_SUCCESS} if there are no errors.
       If the label is not found, and the {\tt isPresent} argument is
       not present, an error is returned.
     \end{description}
   
%/////////////////////////////////////////////////////////////
 
\mbox{}\hrulefill\ 
 
\subsubsection [ESMF\_ConfigGetAttribute] {ESMF\_ConfigGetAttribute - Get an attribute value from Config object}


  
  
\bigskip{\sf INTERFACE:}
\begin{verbatim}        subroutine ESMF_ConfigGetAttribute(config, <value>, &
          label, default, rc)\end{verbatim}{\em ARGUMENTS:}
\begin{verbatim}        type(ESMF_Config), intent(inout)         :: config     
        <value argument>, see below for supported values
 -- The following arguments require argument keyword syntax (e.g. rc=rc). --
        character(len=*),  intent(in),  optional :: label 
        character(len=*),  intent(in),  optional :: default 
        integer,           intent(out), optional :: rc     \end{verbatim}
{\sf STATUS:}
   \begin{itemize}
   \item\apiStatusCompatibleVersion{5.2.0r}
   \end{itemize}
  
{\sf DESCRIPTION:\\ }

 
        Gets a value from the {\tt config} object.  When the
        value is a sequence of characters
        it will be terminated by the first white space.
        
        Supported values for <value argument> are:
        \begin{description}
        \item character(len=*), intent(out)          :: value
        \item real(ESMF\_KIND\_R4), intent(out)      :: value    
        \item real(ESMF\_KIND\_R8), intent(out)      :: value
        \item integer(ESMF\_KIND\_I4), intent(out)   :: value
        \item integer(ESMF\_KIND\_I8), intent(out)   :: value
        \item logical, intent(out)                   :: value
        \end{description}
  
     The arguments are:
     \begin{description}
     \item [config]
       Already created {\tt ESMF\_Config} object.
     \item [<value argument>]
       Returned value.
     \item [{[label]}]
       Identifing label. 
     \item [{[default]}]
       Default value if {\tt label} is not found in {\tt config} object. 
     \item [{[rc]}]
       Return code; equals {\tt ESMF\_SUCCESS} if there are no errors.
     \end{description} 
%/////////////////////////////////////////////////////////////
 
\mbox{}\hrulefill\ 
 
\subsubsection [ESMF\_ConfigGetAttribute] {ESMF\_ConfigGetAttribute - Get a list of attribute values from Config object}


  
\bigskip{\sf INTERFACE:}
\begin{verbatim}        subroutine ESMF_ConfigGetAttribute(config, <value list argument>, &
          count, label, default, rc)\end{verbatim}{\em ARGUMENTS:}
\begin{verbatim}        type(ESMF_Config), intent(inout)         :: config     
        <value list argument>, see below for values      
 -- The following arguments require argument keyword syntax (e.g. rc=rc). --
        integer,           intent(in)   optional :: count
        character(len=*),  intent(in),  optional :: label 
        character(len=*),  intent(in),  optional :: default 
        integer,           intent(out), optional :: rc     \end{verbatim}
{\sf STATUS:}
   \begin{itemize}
   \item\apiStatusCompatibleVersion{5.2.0r}
   \end{itemize}
  
{\sf DESCRIPTION:\\ }


        Gets a list of values from the {\tt config} object.  
  
        Supported values for <value list argument> are:
        \begin{description}
        \item character(len=*), intent(out)            :: valueList(:)
        \item real(ESMF\_KIND\_R4), intent(inout)      :: valueList(:)
        \item real(ESMF\_KIND\_R8), intent(inout)      :: valueList(:)
        \item integer(ESMF\_KIND\_I4), intent(inout)   :: valueList(:)
        \item integer(ESMF\_KIND\_I8), intent(inout)   :: valueList(:)
        \item logical, intent(inout)                   :: valueList(:)
        \end{description}
  
     The arguments are:
     \begin{description}
     \item [config]
       Already created {\tt ESMF\_Config} object.
     \item [<value list argument>]
       Returned value.
     \item [count]
       Number of returned values expected.  
     \item [{[label]}]
       Identifing label. 
     \item [{[default]}]
       Default value if {\tt label} is not found in {\tt config} object. 
     \item [{[rc]}]
       Return code; equals {\tt ESMF\_SUCCESS} if there are no errors.
     \end{description} 
%/////////////////////////////////////////////////////////////
 
\mbox{}\hrulefill\ 
 

  \subsubsection [ESMF\_ConfigGetChar] {ESMF\_ConfigGetChar - Get a character attribute value from Config object}


  
\bigskip{\sf INTERFACE:}
\begin{verbatim}       subroutine ESMF_ConfigGetChar(config, value, &
         label, default, rc)
 \end{verbatim}{\em ARGUMENTS:}
\begin{verbatim}       type(ESMF_Config), intent(inout)         :: config 
       character,         intent(out)           :: value
 -- The following arguments require argument keyword syntax (e.g. rc=rc). --
       character(len=*),  intent(in),  optional :: label   
       character,         intent(in),  optional :: default
       integer,           intent(out), optional :: rc    \end{verbatim}
{\sf STATUS:}
   \begin{itemize}
   \item\apiStatusCompatibleVersion{5.2.0r}
   \end{itemize}
  
{\sf DESCRIPTION:\\ }

 
    Gets a character {\tt value} from the {\tt config} object.
  
     The arguments are:
     \begin{description}
     \item [config]
       Already created {\tt ESMF\_Config} object.
     \item [value]
       Returned value. 
     \item [{[label]}]
       Identifying label. 
     \item [{[default]}]
       Default value if label is not found in configuration object. 
     \item [{[rc]}]
       Return code; equals {\tt ESMF\_SUCCESS} if there are no errors.
     \end{description}
  
   
%/////////////////////////////////////////////////////////////
 
\mbox{}\hrulefill\ 
 

  \subsubsection [ESMF\_ConfigGetDim] {ESMF\_ConfigGetDim - Get table sizes from Config object}


  
\bigskip{\sf INTERFACE:}
\begin{verbatim}     subroutine ESMF_ConfigGetDim(config, lineCount, columnCount, &
       label, rc)
 \end{verbatim}{\em ARGUMENTS:}
\begin{verbatim}       type(ESMF_Config), intent(inout)         :: config
       integer,           intent(out)           :: lineCount
       integer,           intent(out)           :: columnCount
 -- The following arguments require argument keyword syntax (e.g. rc=rc). --
       character(len=*),  intent(in),  optional :: label
       integer,           intent(out), optional :: rc\end{verbatim}
{\sf STATUS:}
   \begin{itemize}
   \item\apiStatusCompatibleVersion{5.2.0r}
   \end{itemize}
  
{\sf DESCRIPTION:\\ }

 
    Returns the number of lines in the table in {\tt lineCount} and 
    the maximum number of words in a table line in {\tt columnCount}.
  
    After the call, the line pointer is positioned to the end of the table.
    To reset it to the beginning of the table, use {\tt ESMF\_ConfigFindLabel}. 
  
     The arguments are:
     \begin{description}
     \item [config]
       Already created {\tt ESMF\_Config} object.
     \item [lineCount]
       Returned number of lines in the table. 
     \item [columnCount]
       Returned maximum number of words in a table line. 
     \item [{[label]}]
       Identifying label (if present), otherwise current line.
     \item [{[rc]}]
       Return code; equals {\tt ESMF\_SUCCESS} if there are no errors.
     \end{description}
   
%/////////////////////////////////////////////////////////////
 
\mbox{}\hrulefill\ 
 
\subsubsection [ESMF\_ConfigGetLen] {ESMF\_ConfigGetLen - Get the length of the line in words from Config object}


  
\bigskip{\sf INTERFACE:}
\begin{verbatim}     integer function ESMF_ConfigGetLen(config, label, rc)
 \end{verbatim}{\em ARGUMENTS:}
\begin{verbatim}       type(ESMF_Config), intent(inout)          :: config 
 -- The following arguments require argument keyword syntax (e.g. rc=rc). --
       character(len=*),  intent(in),   optional :: label
       integer,           intent(out),  optional :: rc         \end{verbatim}
{\sf STATUS:}
   \begin{itemize}
   \item\apiStatusCompatibleVersion{5.2.0r}
   \end{itemize}
  
{\sf DESCRIPTION:\\ }

 
   Gets the length of the line in words by counting words
   disregarding types.  Returns the word count as an integer.
  
     The arguments are:
     \begin{description}
     \item [config]
       Already created {\tt ESMF\_Config} object.
     \item [{[label]}]
       Identifying label.   If not specified, use the current line.
     \item [{[rc]}]
       Return code; equals {\tt ESMF\_SUCCESS} if there are no errors.
     \end{description}
   
%/////////////////////////////////////////////////////////////
 
\mbox{}\hrulefill\ 
 
\subsubsection [ESMF\_ConfigIsCreated] {ESMF\_ConfigIsCreated - Check whether a Config object has been created}


 
\bigskip{\sf INTERFACE:}
\begin{verbatim}   function ESMF_ConfigIsCreated(config, rc)\end{verbatim}{\em RETURN VALUE:}
\begin{verbatim}     logical :: ESMF_ConfigIsCreated\end{verbatim}{\em ARGUMENTS:}
\begin{verbatim}     type(ESMF_Config), intent(in)            :: config
 -- The following arguments require argument keyword syntax (e.g. rc=rc). --
     integer,             intent(out), optional :: rc
 \end{verbatim}
{\sf DESCRIPTION:\\ }


     Return {\tt .true.} if the {\tt config} has been created. Otherwise return 
     {\tt .false.}. If an error occurs, i.e. {\tt rc /= ESMF\_SUCCESS} is 
     returned, the return value of the function will also be {\tt .false.}.
  
   The arguments are:
     \begin{description}
     \item[config]
       {\tt ESMF\_Config} queried.
     \item[{[rc]}]
       Return code; equals {\tt ESMF\_SUCCESS} if there are no errors.
     \end{description}
   
%/////////////////////////////////////////////////////////////
 
\mbox{}\hrulefill\ 
 

  \subsubsection [ESMF\_ConfigLoadFile] {ESMF\_ConfigLoadFile - Load resource file into Config object memory}


  
\bigskip{\sf INTERFACE:}
\begin{verbatim}     subroutine ESMF_ConfigLoadFile(config, filename, &
       delayout, unique, rc)
 \end{verbatim}{\em ARGUMENTS:}
\begin{verbatim}       type(ESMF_Config),   intent(inout)         :: config     
       character(len=*),    intent(in)            :: filename 
 -- The following arguments require argument keyword syntax (e.g. rc=rc). --
       type(ESMF_DELayout), intent(in),  optional :: delayout 
       logical,             intent(in),  optional :: unique 
       integer,             intent(out), optional :: rc         \end{verbatim}
{\sf STATUS:}
   \begin{itemize}
   \item\apiStatusCompatibleVersion{5.2.0r}
   \end{itemize}
  
{\sf DESCRIPTION:\\ }

 
    Resource file with {\tt filename} is loaded into memory.
  
     The arguments are:
     \begin{description}
     \item [config]
       Already created {\tt ESMF\_Config} object.
     \item [filename]
       Configuration file name.
     \item [{[delayout]}]
       {\tt ESMF\_DELayout} associated with this {\tt config} object.
     \item [{[unique]}]
       If specified as true, uniqueness of labels are checked and 
       error code set if duplicates found.
     \item [{[rc]}]
       Return code; equals {\tt ESMF\_SUCCESS} if there are no errors.
     \end{description}
   
%/////////////////////////////////////////////////////////////
 
\mbox{}\hrulefill\ 
 

  \subsubsection [ESMF\_ConfigNextLine] {ESMF\_ConfigNextLine - Find next line in a Config object}


  
\bigskip{\sf INTERFACE:}
\begin{verbatim}     subroutine ESMF_ConfigNextLine(config, tableEnd, rc)
 \end{verbatim}{\em ARGUMENTS:}
\begin{verbatim}       type(ESMF_Config), intent(inout)          :: config 
 -- The following arguments require argument keyword syntax (e.g. rc=rc). --
       logical,           intent(out),  optional :: tableEnd
       integer,           intent(out),  optional :: rc \end{verbatim}
{\sf STATUS:}
   \begin{itemize}
   \item\apiStatusCompatibleVersion{5.2.0r}
   \end{itemize}
  
{\sf DESCRIPTION:\\ }

 
     Selects the next line (for tables).
  
     The arguments are:
     \begin{description}
     \item [config]
       Already created {\tt ESMF\_Config} object.
     \item [{[tableEnd]}]
       Returns {\tt .true.} if end of table mark (::) is encountered.
     \item [{[rc]}]
       Return code; equals {\tt ESMF\_SUCCESS} if there are no errors.
     \end{description}
  
%/////////////////////////////////////////////////////////////
 
\mbox{}\hrulefill\ 
 

  \subsubsection [ESMF\_ConfigPrint] {ESMF\_ConfigPrint - Write content of Config object to unit}


  
\bigskip{\sf INTERFACE:}
\begin{verbatim}     subroutine ESMF_ConfigPrint(config, unit, rc)
 \end{verbatim}{\em ARGUMENTS:}
\begin{verbatim}       type(ESMF_Config), intent(in)  :: config
 -- The following arguments require argument keyword syntax (e.g. rc=rc). --
       integer, optional, intent(in)  :: unit
       integer, optional, intent(out) :: rc\end{verbatim}
{\sf DESCRIPTION:\\ }


     Write content of input {\tt ESMF\_Config} object to unit {\tt unit}.
     If {\tt unit} not provided, writes to standard output.
  
     The arguments are:
     \begin{description}
       \item[config]
         The input {\tt ESMF\_Config} object.
       \item[{[unit]}]
         Output unit.
       \item [{[rc]}]
         Return code; equals {\tt ESMF\_SUCCESS} if there are no errors.
     \end{description}
   
%/////////////////////////////////////////////////////////////
 
\mbox{}\hrulefill\ 
 

  \subsubsection [ESMF\_ConfigSetAttribute] {ESMF\_ConfigSetAttribute - Set a value in Config object}


  
  
\bigskip{\sf INTERFACE:}
\begin{verbatim}       subroutine ESMF_ConfigSetAttribute(config, <value argument>, &
         label, rc)\end{verbatim}{\em ARGUMENTS:}
\begin{verbatim}       type(ESMF_Config), intent(inout)           :: config     
       <value argument>, see below for supported values
 -- The following arguments require argument keyword syntax (e.g. rc=rc). --
       character(len=*),  intent(in),   optional  :: label 
       integer,           intent(out),  optional  :: rc   \end{verbatim}
{\sf STATUS:}
   \begin{itemize}
   \item\apiStatusCompatibleVersion{5.2.0r}
   \end{itemize}
  
{\sf DESCRIPTION:\\ }

 
    Sets a value in the {\tt config} object.
  
        Supported values for <value argument> are:
        \begin{description}
        \item integer(ESMF\_KIND\_I4), intent(in)            :: value
        \end{description}
  
     The arguments are:
       \begin{description}
     \item [config]
       Already created {\tt ESMF\_Config} object.
     \item [<value argument>]
       Value to set. 
     \item [{[label]}]
       Identifying attribute label. 
     \item [{[rc]}]
       Return code; equals {\tt ESMF\_SUCCESS} if there are no errors.
     \end{description}
   
%/////////////////////////////////////////////////////////////
 
\mbox{}\hrulefill\ 
 

  \subsubsection [ESMF\_ConfigValidate] {ESMF\_ConfigValidate - Validate a Config object}


  
\bigskip{\sf INTERFACE:}
\begin{verbatim}     subroutine ESMF_ConfigValidate(config, &
       options, rc)
 \end{verbatim}{\em ARGUMENTS:}
\begin{verbatim}       type(ESMF_Config), intent(inout)          :: config 
 -- The following arguments require argument keyword syntax (e.g. rc=rc). --
       character (len=*), intent(in),   optional :: options
       integer,           intent(out),  optional :: rc \end{verbatim}
{\sf STATUS:}
   \begin{itemize}
   \item\apiStatusCompatibleVersion{5.2.0r}
   \end{itemize}
  
{\sf DESCRIPTION:\\ }

 
     Checks whether a {\tt config} object is valid.
  
     The arguments are:
     \begin{description}
     \item [config]
       {\tt ESMF\_Config} object to be validated.
     \item[{[options]}]
       \begin{sloppypar}
       If none specified:  simply check that the buffer is not full and the
         pointers are within range.
       "unusedAttributes" - Report to the default logfile all attributes not
         retrieved via a call to {\tt ESMF\_ConfigGetAttribute()} or
         {\tt ESMF\_ConfigGetChar()}.  The attribute name (label) will be
         logged via {\tt ESMF\_LogErr} with the WARNING log message type.
         For an array-valued attribute, retrieving at least one value via
         {\tt ESMF\_ConfigGetAttribute()} or {\tt ESMF\_ConfigGetChar()}
         constitutes being "used."
       \end{sloppypar}
     \item [{[rc]}]
       Return code; equals {\tt ESMF\_SUCCESS} if there are no errors.
       Equals {\tt ESMF\_RC\_ATTR\_UNUSED} if any unused attributes are found
       with option "unusedAttributes" above.
     \end{description}
 
%...............................................................
\setlength{\parskip}{\oldparskip}
\setlength{\parindent}{\oldparindent}
\setlength{\baselineskip}{\oldbaselineskip}
