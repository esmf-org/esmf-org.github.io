%                **** IMPORTANT NOTICE *****
% This LaTeX file has been automatically produced by ProTeX v. 1.1
% Any changes made to this file will likely be lost next time
% this file is regenerated from its source. Send questions 
% to Arlindo da Silva, dasilva@gsfc.nasa.gov
 
\setlength{\oldparskip}{\parskip}
\setlength{\parskip}{1.5ex}
\setlength{\oldparindent}{\parindent}
\setlength{\parindent}{0pt}
\setlength{\oldbaselineskip}{\baselineskip}
\setlength{\baselineskip}{11pt}
 
%--------------------- SHORT-HAND MACROS ----------------------
\def\bv{\begin{verbatim}}
\def\ev{\end{verbatim}}
\def\be{\begin{equation}}
\def\ee{\end{equation}}
\def\bea{\begin{eqnarray}}
\def\eea{\end{eqnarray}}
\def\bi{\begin{itemize}}
\def\ei{\end{itemize}}
\def\bn{\begin{enumerate}}
\def\en{\end{enumerate}}
\def\bd{\begin{description}}
\def\ed{\end{description}}
\def\({\left (}
\def\){\right )}
\def\[{\left [}
\def\]{\right ]}
\def\<{\left  \langle}
\def\>{\right \rangle}
\def\cI{{\cal I}}
\def\diag{\mathop{\rm diag}}
\def\tr{\mathop{\rm tr}}
%-------------------------------------------------------------

\markboth{Left}{Source File: ESMF\_ArraySpec.F90,  Date: Tue May  5 20:59:37 MDT 2020
}

 
%/////////////////////////////////////////////////////////////
\subsubsection [ESMF\_ArraySpecAssignment(=)] {ESMF\_ArraySpecAssignment(=) - Assign an ArraySpec to another ArraySpec}


  
\bigskip{\sf INTERFACE:}
\begin{verbatim}   interface assignment(=)
     arrayspec1 = arrayspec2\end{verbatim}{\em ARGUMENTS:}
\begin{verbatim}     type(ESMF_ArraySpec) :: arrayspec1
     type(ESMF_ArraySpec) :: arrayspec2\end{verbatim}
{\sf STATUS:}
   \begin{itemize}
   \item\apiStatusCompatibleVersion{5.2.0r}
   \end{itemize}
  
{\sf DESCRIPTION:\\ }


     Set {\tt arrayspec1} equal to {\tt arrayspec2}. This is the default 
     Fortran assignment, which creates a complete, independent copy of 
     {\tt arrayspec2} as {\tt arrayspec1}. If {\tt arrayspec2} is an 
     invalid {\tt ESMF\_ArraySpec} object then {\tt arrayspec1} will be 
     equally invalid after the assignment.
  
     The arguments are:
     \begin{description} 
     \item[arrayspec1] 
       The {\tt ESMF\_ArraySpec} to be set.
     \item[arrayspec2] 
       The {\tt ESMF\_ArraySpec} to be copied.
     \end{description}
   
%/////////////////////////////////////////////////////////////
 
\mbox{}\hrulefill\ 
 
\subsubsection [ESMF\_ArraySpecOperator(==)] {ESMF\_ArraySpecOperator(==) - Test if ArraySpec 1 is equal to ArraySpec 2}


  
\bigskip{\sf INTERFACE:}
\begin{verbatim}   interface operator(==)
     if (arrayspec1 == arrayspec2) then ... endif
                  OR
     result = (arrayspec1 == arrayspec2)\end{verbatim}{\em RETURN VALUE:}
\begin{verbatim}     logical :: result\end{verbatim}{\em ARGUMENTS:}
\begin{verbatim}     type(ESMF_ArraySpec), intent(in) :: arrayspec1
     type(ESMF_ArraySpec), intent(in) :: arrayspec2\end{verbatim}
{\sf STATUS:}
   \begin{itemize}
   \item\apiStatusCompatibleVersion{5.2.0r}
   \end{itemize}
  
{\sf DESCRIPTION:\\ }


     Overloads the (==) operator for the {\tt ESMF\_ArraySpec} class to return 
     {\tt .true.} if {\tt arrayspec1} and {\tt arrayspec2} specify the same
     type, kind and rank, and {\tt .false.} otherwise.
  
     The arguments are:
     \begin{description}
     \item[arrayspec1]
       First {\tt ESMF\_ArraySpec} in comparison.
     \item[arrayspec2]
       Second {\tt ESMF\_ArraySpec} in comparison.
     \end{description}
   
%/////////////////////////////////////////////////////////////
 
\mbox{}\hrulefill\ 
 
\subsubsection [ESMF\_ArraySpecOperator(/=)] {ESMF\_ArraySpecOperator(/=) - Test if ArraySpec 1 is not equal to ArraySpec 2}


  
\bigskip{\sf INTERFACE:}
\begin{verbatim}   interface operator(/=)
     if (arrayspec1 /= arrayspec2) then ... endif
                  OR
     result = (arrayspec1 /= arrayspec2)\end{verbatim}{\em RETURN VALUE:}
\begin{verbatim}     logical :: result\end{verbatim}{\em ARGUMENTS:}
\begin{verbatim}     type(ESMF_ArraySpec), intent(in) :: arrayspec1
     type(ESMF_ArraySpec), intent(in) :: arrayspec2\end{verbatim}
{\sf STATUS:}
   \begin{itemize}
   \item\apiStatusCompatibleVersion{5.2.0r}
   \end{itemize}
  
{\sf DESCRIPTION:\\ }


     Overloads the (/=) operator for the {\tt ESMF\_ArraySpec} class to return 
     {\tt .true.} if {\tt arrayspec1} and {\tt arrayspec2} do not specify the
     same type, kind or rank, and {\tt .false.} otherwise.
  
     The arguments are:
     \begin{description}
     \item[arrayspec1]
       First {\tt ESMF\_ArraySpec} in comparison.
     \item[arrayspec2]
       Second {\tt ESMF\_ArraySpec} in comparison.
     \end{description}
    
%/////////////////////////////////////////////////////////////
 
\mbox{}\hrulefill\ 
 
\subsubsection [ESMF\_ArraySpecGet] {ESMF\_ArraySpecGet - Get values from an ArraySpec}


  
\bigskip{\sf INTERFACE:}
\begin{verbatim}   subroutine ESMF_ArraySpecGet(arrayspec, rank, typekind, rc)\end{verbatim}{\em ARGUMENTS:}
\begin{verbatim}     type(ESMF_ArraySpec),     intent(in)            :: arrayspec
 -- The following arguments require argument keyword syntax (e.g. rc=rc). --
     integer,                  intent(out), optional :: rank
     type(ESMF_TypeKind_Flag), intent(out), optional :: typekind
     integer,                  intent(out), optional :: rc\end{verbatim}
{\sf STATUS:}
   \begin{itemize}
   \item\apiStatusCompatibleVersion{5.2.0r}
   \end{itemize}
  
{\sf DESCRIPTION:\\ }


     Returns information about the contents of an {\tt ESMF\_ArraySpec}.
  
     The arguments are:
     \begin{description}
     \item[arrayspec]
       The {\tt ESMF\_ArraySpec} to query.
     \item[{[rank]}]
       Array rank (dimensionality -- 1D, 2D, etc). Maximum possible is 7D.
     \item[{[typekind]}]
       Array typekind.  See section \ref{const:typekind} for valid values.
     \item[{[rc]}]
       Return code; equals {\tt ESMF\_SUCCESS} if there are no errors.
     \end{description}
   
%/////////////////////////////////////////////////////////////
 
\mbox{}\hrulefill\ 
 
\subsubsection [ESMF\_ArraySpecPrint] {ESMF\_ArraySpecPrint - Print ArraySpec information}


 
\bigskip{\sf INTERFACE:}
\begin{verbatim}   subroutine ESMF_ArraySpecPrint(arrayspec, rc)\end{verbatim}{\em ARGUMENTS:}
\begin{verbatim}     type(ESMF_ArraySpec), intent(in)            :: arrayspec
 -- The following arguments require argument keyword syntax (e.g. rc=rc). --
     integer,              intent(out), optional :: rc\end{verbatim}
{\sf STATUS:}
   \begin{itemize}
   \item\apiStatusCompatibleVersion{5.2.0r}
   \end{itemize}
  
{\sf DESCRIPTION:\\ }


       Print ArraySpec internals. \\
  
       The arguments are:
       \begin{description}
       \item[arrayspec] 
           Specified {\tt ESMF\_ArraySpec} object.
       \item[{[rc]}]
           Return code; equals {\tt ESMF\_SUCCESS} if there are no errors.
       \end{description}
   
%/////////////////////////////////////////////////////////////
 
\mbox{}\hrulefill\ 
 
\subsubsection [ESMF\_ArraySpecSet] {ESMF\_ArraySpecSet - Set values for an ArraySpec}


  
\bigskip{\sf INTERFACE:}
\begin{verbatim}   subroutine ESMF_ArraySpecSet(arrayspec, rank, typekind, rc)\end{verbatim}{\em ARGUMENTS:}
\begin{verbatim}     type(ESMF_ArraySpec),     intent(out)           :: arrayspec
     integer,                  intent(in)            :: rank
     type(ESMF_TypeKind_Flag), intent(in)            :: typekind
 -- The following arguments require argument keyword syntax (e.g. rc=rc). --
     integer,                  intent(out), optional :: rc\end{verbatim}
{\sf STATUS:}
   \begin{itemize}
   \item\apiStatusCompatibleVersion{5.2.0r}
   \end{itemize}
  
{\sf DESCRIPTION:\\ }


     Creates a description of the data -- the typekind, the rank,
     and the dimensionality.
  
     The arguments are:
     \begin{description}
     \item[arrayspec]
       The {\tt ESMF\_ArraySpec} to set.
     \item[rank]
       Array rank (dimensionality -- 1D, 2D, etc). Maximum allowed is 7D.
     \item[typekind]
       Array typekind.  See section \ref{const:typekind} for valid values.
     \item[{[rc]}]
       Return code; equals {\tt ESMF\_SUCCESS} if there are no errors.
     \end{description}
   
%/////////////////////////////////////////////////////////////
 
\mbox{}\hrulefill\ 
 
\subsubsection [ESMF\_ArraySpecValidate] {ESMF\_ArraySpecValidate - Validate ArraySpec internals}


 
\bigskip{\sf INTERFACE:}
\begin{verbatim}   subroutine ESMF_ArraySpecValidate(arrayspec, rc)\end{verbatim}{\em ARGUMENTS:}
\begin{verbatim}     type(ESMF_ArraySpec), intent(in)            :: arrayspec
 -- The following arguments require argument keyword syntax (e.g. rc=rc). --
     integer,              intent(out), optional :: rc  \end{verbatim}
{\sf STATUS:}
   \begin{itemize}
   \item\apiStatusCompatibleVersion{5.2.0r}
   \end{itemize}
  
{\sf DESCRIPTION:\\ }


     Validates that the {\tt arrayspec} is internally consistent.
     The method returns an error code if problems are found.  
  
     The arguments are:
     \begin{description}
     \item[arrayspec] 
       Specified {\tt ESMF\_ArraySpec} object.
     \item[{[rc]}] 
       Return code; equals {\tt ESMF\_SUCCESS} if there are no errors.
     \end{description}
  
%...............................................................
\setlength{\parskip}{\oldparskip}
\setlength{\parindent}{\oldparindent}
\setlength{\baselineskip}{\oldbaselineskip}
