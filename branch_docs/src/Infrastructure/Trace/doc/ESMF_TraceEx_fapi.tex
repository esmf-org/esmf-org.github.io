%                **** IMPORTANT NOTICE *****
% This LaTeX file has been automatically produced by ProTeX v. 1.1
% Any changes made to this file will likely be lost next time
% this file is regenerated from its source. Send questions 
% to Arlindo da Silva, dasilva@gsfc.nasa.gov
 
\setlength{\oldparskip}{\parskip}
\setlength{\parskip}{1.5ex}
\setlength{\oldparindent}{\parindent}
\setlength{\parindent}{0pt}
\setlength{\oldbaselineskip}{\baselineskip}
\setlength{\baselineskip}{11pt}
 
%--------------------- SHORT-HAND MACROS ----------------------
\def\bv{\begin{verbatim}}
\def\ev{\end{verbatim}}
\def\be{\begin{equation}}
\def\ee{\end{equation}}
\def\bea{\begin{eqnarray}}
\def\eea{\end{eqnarray}}
\def\bi{\begin{itemize}}
\def\ei{\end{itemize}}
\def\bn{\begin{enumerate}}
\def\en{\end{enumerate}}
\def\bd{\begin{description}}
\def\ed{\end{description}}
\def\({\left (}
\def\){\right )}
\def\[{\left [}
\def\]{\right ]}
\def\<{\left  \langle}
\def\>{\right \rangle}
\def\cI{{\cal I}}
\def\diag{\mathop{\rm diag}}
\def\tr{\mathop{\rm tr}}
%-------------------------------------------------------------

\markboth{Left}{Source File: ESMF\_TraceEx.F90,  Date: Tue May  5 20:59:31 MDT 2020
}

 
%/////////////////////////////////////////////////////////////

   \subsubsection{Tracing a simple ESMF application} \label{ex:TraceEx}
  
   This example illustrates how to trace a simple ESMF
   application and print the event stream using Babeltrace.
   The first part of the code is a module representing
   a trivial ESMF Gridded Component.  The second part is a
   main program that creates and executes the component. 
%/////////////////////////////////////////////////////////////

 \begin{verbatim}
module SimpleComp

  use ESMF
  implicit none

  private
  public SetServices

contains

  subroutine SetServices(gcomp, rc)
      type(ESMF_GridComp)   :: gcomp
      integer, intent(out)  :: rc  

      call ESMF_GridCompSetEntryPoint(gcomp, ESMF_METHOD_INITIALIZE, &
           userRoutine=Init, rc=rc)
      call ESMF_GridCompSetEntryPoint(gcomp, ESMF_METHOD_RUN, &
           userRoutine=Run, rc=rc)
      call ESMF_GridCompSetEntryPoint(gcomp, ESMF_METHOD_FINALIZE, &
           userRoutine=Finalize, rc=rc)
      
      rc = ESMF_SUCCESS
      
    end subroutine SetServices

    subroutine Init(gcomp, istate, estate, clock, rc)
      type(ESMF_GridComp):: gcomp
      type(ESMF_State):: istate, estate
      type(ESMF_Clock):: clock
      integer, intent(out):: rc
      
      print *, "Inside Init"
      
    end subroutine Init

    subroutine Run(gcomp, istate, estate, clock, rc)
      type(ESMF_GridComp):: gcomp
      type(ESMF_State):: istate, estate
      type(ESMF_Clock):: clock
      integer, intent(out):: rc
      
      print *, "Inside Run"
      
    end subroutine Run

    subroutine Finalize(gcomp, istate, estate, clock, rc)
      type(ESMF_GridComp):: gcomp
      type(ESMF_State):: istate, estate
      type(ESMF_Clock):: clock
      integer, intent(out):: rc
      
    print *, "Inside Finalize"
    
  end subroutine Finalize 

end module SimpleComp
 
\end{verbatim}
 
%/////////////////////////////////////////////////////////////

 \begin{verbatim}
program ESMF_TraceEx
 
\end{verbatim}
 
%/////////////////////////////////////////////////////////////

 \begin{verbatim}
      ! Use ESMF framework module
      use ESMF
      use SimpleComp, only: SetServices
 
\end{verbatim}
 
%/////////////////////////////////////////////////////////////

 \begin{verbatim}
      implicit none

      ! Local variables  
      integer :: rc, finalrc, i
      type(ESMF_GridComp)     :: gridcomp
 
\end{verbatim}
 
%/////////////////////////////////////////////////////////////

 \begin{verbatim}
      ! initialize ESMF
      finalrc = ESMF_SUCCESS
      call ESMF_Initialize(vm=vm, defaultlogfilename="TraceEx.Log", &
                    logkindflag=ESMF_LOGKIND_MULTI, rc=rc)
 
\end{verbatim}
 
%/////////////////////////////////////////////////////////////

 \begin{verbatim}
      ! create the component and then execute
      ! initialize, run, and finalize routines
      gridcomp = ESMF_GridCompCreate(name="test", rc=rc)
 
\end{verbatim}
 
%/////////////////////////////////////////////////////////////

 \begin{verbatim}
      call ESMF_GridCompSetServices(gridcomp, userRoutine=SetServices, rc=rc)
 
\end{verbatim}
 
%/////////////////////////////////////////////////////////////

 \begin{verbatim}
      call ESMF_GridCompInitialize(gridcomp, rc=rc)
 
\end{verbatim}
 
%/////////////////////////////////////////////////////////////

 \begin{verbatim}
      do i=1, 5
         call ESMF_GridCompRun(gridcomp, rc=rc)
      enddo
 
\end{verbatim}
 
%/////////////////////////////////////////////////////////////

 \begin{verbatim}
      call ESMF_GridCompFinalize(gridcomp, rc=rc)
 
\end{verbatim}
 
%/////////////////////////////////////////////////////////////

 \begin{verbatim}
      call ESMF_GridCompDestroy(gridcomp, rc=rc)
 
\end{verbatim}
 
%/////////////////////////////////////////////////////////////

 \begin{verbatim}
      call ESMF_Finalize(rc=rc)
 
\end{verbatim}
 
%/////////////////////////////////////////////////////////////

 \begin{verbatim}
end program ESMF_TraceEx
 
\end{verbatim}
 
%/////////////////////////////////////////////////////////////

   Assuming the code above is executed on four PETs with
   the environment variable {\tt ESMF\_RUNTIME\_TRACE} set to
   {\tt ON}, then a folder will be created in the run directory
   called {\em traceout} containing a {\em metadata} file and
   four event stream files named {\em esmf\_stream\_XXXX}
   where {\em XXXX} is the PET number.  If Babeltrace is
   available on the system, the list of events can be printed
   by executing the following from the run directory:
   \begin{verbatim}
   $ babeltrace ./traceout
   \end{verbatim}
   For details about iterating over trace events and performing
   analyses on CTF traces, see the corresponding documentation
   in the tools listed in Section \ref{sec:Tracing}.
%...............................................................
\setlength{\parskip}{\oldparskip}
\setlength{\parindent}{\oldparindent}
\setlength{\baselineskip}{\oldbaselineskip}
