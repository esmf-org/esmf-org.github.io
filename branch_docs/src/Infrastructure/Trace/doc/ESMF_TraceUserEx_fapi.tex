%                **** IMPORTANT NOTICE *****
% This LaTeX file has been automatically produced by ProTeX v. 1.1
% Any changes made to this file will likely be lost next time
% this file is regenerated from its source. Send questions 
% to Arlindo da Silva, dasilva@gsfc.nasa.gov
 
\setlength{\oldparskip}{\parskip}
\setlength{\parskip}{1.5ex}
\setlength{\oldparindent}{\parindent}
\setlength{\parindent}{0pt}
\setlength{\oldbaselineskip}{\baselineskip}
\setlength{\baselineskip}{11pt}
 
%--------------------- SHORT-HAND MACROS ----------------------
\def\bv{\begin{verbatim}}
\def\ev{\end{verbatim}}
\def\be{\begin{equation}}
\def\ee{\end{equation}}
\def\bea{\begin{eqnarray}}
\def\eea{\end{eqnarray}}
\def\bi{\begin{itemize}}
\def\ei{\end{itemize}}
\def\bn{\begin{enumerate}}
\def\en{\end{enumerate}}
\def\bd{\begin{description}}
\def\ed{\end{description}}
\def\({\left (}
\def\){\right )}
\def\[{\left [}
\def\]{\right ]}
\def\<{\left  \langle}
\def\>{\right \rangle}
\def\cI{{\cal I}}
\def\diag{\mathop{\rm diag}}
\def\tr{\mathop{\rm tr}}
%-------------------------------------------------------------

\markboth{Left}{Source File: ESMF\_TraceUserEx.F90,  Date: Tue May  5 20:59:31 MDT 2020
}

 
%/////////////////////////////////////////////////////////////

   \subsubsection{Profiling/Tracing User-defined Code Regions} \label{ex:TraceUserEx}
  
   This example illustrates how to manually instrument code with
   entry and exit points for user-defined code regions. Note that the
   API calls {\tt ESMF\_TraceRegionEnter} and {\tt ESMF\_TraceRegionExit}
   should always appear in pairs, wrapping a particular section
   of code. The environment variable {\tt ESMF\_RUNTIME\_TRACE} 
   or {\tt ESMF\_RUNTIME\_PROFILE} must  
   be set to {\tt ON} to enable these regions. If at least one is not set, the calls to
   {\tt ESMF\_TraceRegionEnter} and {\tt ESMF\_TraceRegionExit}
   will simply return immediately. For this reason, it is safe to
   leave this instrumentation in application code, even when not being profiled.   
%/////////////////////////////////////////////////////////////

 \begin{verbatim}
      ! Use ESMF framework module
      use ESMF
 
\end{verbatim}
 
%/////////////////////////////////////////////////////////////

 \begin{verbatim}
      implicit none

      ! Local variables  
      integer :: rc, finalrc
      integer :: i, j, tmp                 
 
\end{verbatim}
 
%/////////////////////////////////////////////////////////////

 \begin{verbatim}
      ! initialize ESMF
      finalrc = ESMF_SUCCESS
      call ESMF_Initialize(vm=vm, defaultlogfilename="TraceUserEx.Log", &
                    logkindflag=ESMF_LOGKIND_MULTI, rc=rc)
 
\end{verbatim}
 
%/////////////////////////////////////////////////////////////

 \begin{verbatim}
      ! record entrance into "outer_region"
      call ESMF_TraceRegionEnter("outer_region", rc=rc)

      tmp = 0
      do i=1, 10
         
         ! record entrance into "inner_region_1"
         call ESMF_TraceRegionEnter("inner_region_1", rc=rc)
         ! arbitrary computation
         do j=1,10000
            tmp=tmp+j+i
         enddo
         ! record exit from "inner_region_1"
         call ESMF_TraceRegionExit("inner_region_1", rc=rc)

         tmp = 0
         
         ! record entrance into "inner_region_2"
         call ESMF_TraceRegionEnter("inner_region_2", rc=rc)
         ! arbitrary computation
         do j=1,5000
            tmp=tmp+j+i
         enddo
         ! record exit from "inner_region_2"
         call ESMF_TraceRegionExit("inner_region_2", rc=rc)
      enddo

      ! record exit from "outer_region"
      call ESMF_TraceRegionExit("outer_region", rc=rc)
 
\end{verbatim}
 
%/////////////////////////////////////////////////////////////

 \begin{verbatim}
      call ESMF_Finalize(rc=rc)
 
\end{verbatim}

%...............................................................
\setlength{\parskip}{\oldparskip}
\setlength{\parindent}{\oldparindent}
\setlength{\baselineskip}{\oldbaselineskip}
