%                **** IMPORTANT NOTICE *****
% This LaTeX file has been automatically produced by ProTeX v. 1.1
% Any changes made to this file will likely be lost next time
% this file is regenerated from its source. Send questions 
% to Arlindo da Silva, dasilva@gsfc.nasa.gov
 
\setlength{\oldparskip}{\parskip}
\setlength{\parskip}{1.5ex}
\setlength{\oldparindent}{\parindent}
\setlength{\parindent}{0pt}
\setlength{\oldbaselineskip}{\baselineskip}
\setlength{\baselineskip}{11pt}
 
%--------------------- SHORT-HAND MACROS ----------------------
\def\bv{\begin{verbatim}}
\def\ev{\end{verbatim}}
\def\be{\begin{equation}}
\def\ee{\end{equation}}
\def\bea{\begin{eqnarray}}
\def\eea{\end{eqnarray}}
\def\bi{\begin{itemize}}
\def\ei{\end{itemize}}
\def\bn{\begin{enumerate}}
\def\en{\end{enumerate}}
\def\bd{\begin{description}}
\def\ed{\end{description}}
\def\({\left (}
\def\){\right )}
\def\[{\left [}
\def\]{\right ]}
\def\<{\left  \langle}
\def\>{\right \rangle}
\def\cI{{\cal I}}
\def\diag{\mathop{\rm diag}}
\def\tr{\mathop{\rm tr}}
%-------------------------------------------------------------

\markboth{Left}{Source File: ESMF\_Trace.F90,  Date: Tue May  5 20:59:31 MDT 2020
}

 
%/////////////////////////////////////////////////////////////
\subsubsection [ESMF\_TraceRegionEnter] {ESMF\_TraceRegionEnter - Trace user-defined region entry event}


   
\bigskip{\sf INTERFACE:}
\begin{verbatim}   subroutine ESMF_TraceRegionEnter(name, rc)\end{verbatim}{\em ARGUMENTS:}
\begin{verbatim}     character(len=*), intent(in) :: name
     integer, intent(out), optional  :: rc\end{verbatim}
{\sf DESCRIPTION:\\ }


     Record an event in the trace for this PET indicating entry
     into a user-defined region with the given name.  This call
     must be paired with a call to {\tt ESMF\_TraceRegionExit()}
     with a matching {\tt name} parameter.  User-defined regions may be
     nested.
     If tracing is disabled on the calling PET or for the application
     as a whole, no event will be recorded and
     the call will return immediately.
  
   The arguments are:
   \begin{description}
   \item[{name}]
     A user-defined name for the region of code being entered
   \item[{[rc]}]
     Return code; equals {\tt ESMF\_SUCCESS} if there are no errors.
   \end{description}     
%/////////////////////////////////////////////////////////////
 
\mbox{}\hrulefill\ 
 
\subsubsection [ESMF\_TraceRegionExit] {ESMF\_TraceRegionExit - Trace user-defined region exit event}


   
\bigskip{\sf INTERFACE:}
\begin{verbatim}   subroutine ESMF_TraceRegionExit(name, rc)\end{verbatim}{\em ARGUMENTS:}
\begin{verbatim}     character(len=*), intent(in) :: name
     integer, intent(out), optional  :: rc\end{verbatim}
{\sf DESCRIPTION:\\ }


     Record an event in the trace for this PET indicating exit
     from a user-defined region with the given name.  This call
     must appear after a call to {\tt ESMF\_TraceRegionEnter()}
     with a matching {\tt name} parameter.
     If tracing is disabled on the calling PET or for the application
     as a whole, no event will be recorded and
     the call will return immediately.
  
   The arguments are:
   \begin{description}
   \item[{name}]
     A user-defined name for the region of code being exited
   \item[{[rc]}]
     Return code; equals {\tt ESMF\_SUCCESS} if there are no errors.
   \end{description}    
%...............................................................
\setlength{\parskip}{\oldparskip}
\setlength{\parindent}{\oldparindent}
\setlength{\baselineskip}{\oldbaselineskip}
