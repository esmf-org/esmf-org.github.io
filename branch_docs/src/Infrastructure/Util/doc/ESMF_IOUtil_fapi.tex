%                **** IMPORTANT NOTICE *****
% This LaTeX file has been automatically produced by ProTeX v. 1.1
% Any changes made to this file will likely be lost next time
% this file is regenerated from its source. Send questions 
% to Arlindo da Silva, dasilva@gsfc.nasa.gov
 
\setlength{\oldparskip}{\parskip}
\setlength{\parskip}{1.5ex}
\setlength{\oldparindent}{\parindent}
\setlength{\parindent}{0pt}
\setlength{\oldbaselineskip}{\baselineskip}
\setlength{\baselineskip}{11pt}
 
%--------------------- SHORT-HAND MACROS ----------------------
\def\bv{\begin{verbatim}}
\def\ev{\end{verbatim}}
\def\be{\begin{equation}}
\def\ee{\end{equation}}
\def\bea{\begin{eqnarray}}
\def\eea{\end{eqnarray}}
\def\bi{\begin{itemize}}
\def\ei{\end{itemize}}
\def\bn{\begin{enumerate}}
\def\en{\end{enumerate}}
\def\bd{\begin{description}}
\def\ed{\end{description}}
\def\({\left (}
\def\){\right )}
\def\[{\left [}
\def\]{\right ]}
\def\<{\left  \langle}
\def\>{\right \rangle}
\def\cI{{\cal I}}
\def\diag{\mathop{\rm diag}}
\def\tr{\mathop{\rm tr}}
%-------------------------------------------------------------

\markboth{Left}{Source File: ESMF\_IOUtil.F90,  Date: Tue May  5 20:59:26 MDT 2020
}

 
%/////////////////////////////////////////////////////////////

  
%/////////////////////////////////////////////////////////////
 
\mbox{}\hrulefill\ 
 
\subsubsection [ESMF\_UtilIOUnitFlush] {ESMF\_UtilIOUnitFlush - Flush output on a unit number}


  
\bigskip{\sf INTERFACE:}
\begin{verbatim}   subroutine ESMF_UtilIOUnitFlush(unit, rc)\end{verbatim}{\em PARAMETERS:}
\begin{verbatim}     integer, intent(in)            :: unit
 -- The following arguments require argument keyword syntax (e.g. rc=rc). --
     integer, intent(out), optional :: rc\end{verbatim}
{\sf STATUS:}
   \begin{itemize}
   \item\apiStatusCompatibleVersion{5.2.0r}
   \end{itemize}
  
{\sf DESCRIPTION:\\ }


     Call the system-dependent routine to force output on a specific
     Fortran unit number.
  
       The arguments are:
       \begin{description}
       \item[unit]
         A Fortran I/O unit number.  If the unit is not connected to a file,
         no flushing occurs.
       \item[{[rc]}]
         Return code; equals {\tt ESMF\_SUCCESS} if there are no errors.
       \end{description} 
%/////////////////////////////////////////////////////////////
 
\mbox{}\hrulefill\ 
 
\subsubsection [ESMF\_UtilIOUnitGet] {ESMF\_UtilIOUnitGet - Scan for a free I/O unit number}


  
\bigskip{\sf INTERFACE:}
\begin{verbatim}   subroutine ESMF_UtilIOUnitGet(unit, rc)\end{verbatim}{\em ARGUMENTS:}
\begin{verbatim}     integer, intent(out)           :: unit
 -- The following arguments require argument keyword syntax (e.g. rc=rc). --
     integer, intent(out), optional :: rc\end{verbatim}
{\sf STATUS:}
   \begin{itemize}
   \item\apiStatusCompatibleVersion{5.2.0r}
   \end{itemize}
  
{\sf DESCRIPTION:\\ }


     Scan for, and return, a free Fortran I/O unit number.
     By default, the range of unit numbers returned is between 50 and 99
     (parameters {\tt ESMF\_LOG\_FORTRAN\_UNIT\_NUMBER} and {\tt ESMF\_LOG\_UPPER}
     respectively.) When integrating ESMF into an application where these values
     conflict with other usages, the range of values may be moved by setting the
     optional {\tt IOUnitLower} and {\tt IOUnitUpper} arguments in the initial
     {\tt ESMF\_Initialize()} call with values in a safe, alternate, range.
  
     The Fortran unit number which is returned is not reserved in any way.
     Successive calls without intervening {\tt OPEN} or {\tt CLOSE} statements
     (or other means of connecting to units), might not return a unique unit
     number.  It is recommended that an {\tt OPEN} statement immediately follow
     the call to {\tt ESMF\_IOUnitGet()} to activate the unit.
  
       The arguments are:
       \begin{description}
       \item[unit]
         A Fortran I/O unit number.
       \item[{[rc]}]
         Return code; equals {\tt ESMF\_SUCCESS} if there are no errors.
       \end{description}
%...............................................................
\setlength{\parskip}{\oldparskip}
\setlength{\parindent}{\oldparindent}
\setlength{\baselineskip}{\oldbaselineskip}
