%                **** IMPORTANT NOTICE *****
% This LaTeX file has been automatically produced by ProTeX v. 1.1
% Any changes made to this file will likely be lost next time
% this file is regenerated from its source. Send questions 
% to Arlindo da Silva, dasilva@gsfc.nasa.gov
 
\setlength{\oldparskip}{\parskip}
\setlength{\parskip}{1.5ex}
\setlength{\oldparindent}{\parindent}
\setlength{\parindent}{0pt}
\setlength{\oldbaselineskip}{\baselineskip}
\setlength{\baselineskip}{11pt}
 
%--------------------- SHORT-HAND MACROS ----------------------
\def\bv{\begin{verbatim}}
\def\ev{\end{verbatim}}
\def\be{\begin{equation}}
\def\ee{\end{equation}}
\def\bea{\begin{eqnarray}}
\def\eea{\end{eqnarray}}
\def\bi{\begin{itemize}}
\def\ei{\end{itemize}}
\def\bn{\begin{enumerate}}
\def\en{\end{enumerate}}
\def\bd{\begin{description}}
\def\ed{\end{description}}
\def\({\left (}
\def\){\right )}
\def\[{\left [}
\def\]{\right ]}
\def\<{\left  \langle}
\def\>{\right \rangle}
\def\cI{{\cal I}}
\def\diag{\mathop{\rm diag}}
\def\tr{\mathop{\rm tr}}
%-------------------------------------------------------------

\markboth{Left}{Source File: ESMCI\_Util\_F.C,  Date: Tue May  5 20:59:27 MDT 2020
}

 
%/////////////////////////////////////////////////////////////
\subsubsection [c\_ESMC\_StringSerialize] {c\_ESMC\_StringSerialize - Serialize String object}


  
\bigskip{\sf INTERFACE:}
\begin{verbatim}       void FTN_X(c_esmc_stringserialize)(\end{verbatim}{\em RETURN VALUE:}
\begin{verbatim}      none.  return code is passed thru the parameter list\end{verbatim}{\em ARGUMENTS:}
\begin{verbatim}       char *string,             // in/out - string object
       char *buf,                // in/out - really a byte stream
       int *length,              // in/out - number of allocated bytes
       int *offset,              // in/out - current offset in the stream
       ESMC_InquireFlag *inquireflag, // in - inquire flag
       int *rc,                  // out - return code
       ESMCI_FortranStrLenArg clen) { // in, hidden - string length\end{verbatim}
{\sf DESCRIPTION:\\ }


       Serialize the contents of a string object.
   
%/////////////////////////////////////////////////////////////
 
\mbox{}\hrulefill\
 
\subsubsection [c\_ESMC\_StringDeserialize] {c\_ESMC\_StringDeserialize - Deserialize String object}


  
\bigskip{\sf INTERFACE:}
\begin{verbatim}       void FTN_X(c_esmc_stringdeserialize)(\end{verbatim}{\em RETURN VALUE:}
\begin{verbatim}      none.  return code is passed thru the parameter list\end{verbatim}{\em ARGUMENTS:}
\begin{verbatim}       char *string,             // in/out - string object
       char *buf,                // in/out - really a byte stream
       int *offset,              // in/out - current offset in the stream
       int *rc,                  // out - return code
       ESMCI_FortranStrLenArg clen) { // in, hidden - string length\end{verbatim}
{\sf DESCRIPTION:\\ }


       Deserialize the contents of a base object.
   
%/////////////////////////////////////////////////////////////
 
\mbox{}\hrulefill\
 
\subsubsection [c\_ESMC\_MakeDirectory] {c\_ESMC\_MakeDirectory - Make a directory in the file system}


  
\bigskip{\sf INTERFACE:}
\begin{verbatim}       void FTN_X(c_esmc_makedirectory)(\end{verbatim}{\em RETURN VALUE:}
\begin{verbatim}      none.  return code is passed thru the parameter list\end{verbatim}{\em ARGUMENTS:}
\begin{verbatim}       const char *pathname,     // in - path name
       int *mode,                // in - protection mode
       ESMC_Logical *relaxedFlag,// in - relaxed mode
       int *rc,                  // out - return code
       ESMCI_FortranStrLenArg pathname_l) { // in, hidden - pathname length\end{verbatim}
{\sf DESCRIPTION:\\ }


       Creates a new directory in the file system.  If the directory already
       exists, and the relaxedFlag argument is set to {tt ESMF\_TRUE},
       return an rc of ESMF\_SUCCESS.
  
       On native Windows, the protection mode argument is ignored.  Default
       security attributes are used.
   
%/////////////////////////////////////////////////////////////
 
\mbox{}\hrulefill\
 
\subsubsection [c\_ESMC\_RemoveDirectory] {c\_ESMC\_RemoveDirectory - Remove a directory from the file system}


  
\bigskip{\sf INTERFACE:}
\begin{verbatim}       void FTN_X(c_esmc_removedirectory)(\end{verbatim}{\em RETURN VALUE:}
\begin{verbatim}      none.  return code is passed thru the parameter list\end{verbatim}{\em ARGUMENTS:}
\begin{verbatim}       const char *pathname,     // in - path name
       ESMC_Logical *relaxedFlag,// in - relaxed mode
       int *rc,                  // out - return code
       ESMCI_FortranStrLenArg pathname_l) { // in, hidden - pathname length\end{verbatim}
{\sf DESCRIPTION:\\ }


       Removes an existing directory in the file system.
   
%/////////////////////////////////////////////////////////////
 
\mbox{}\hrulefill\
 
\subsubsection [c\_ESMC\_UtilSystem] {c\_ESMC\_UtilSystem - Execute a command line}


  
\bigskip{\sf INTERFACE:}
\begin{verbatim}       void FTN_X(c_esmc_utilsystem)(\end{verbatim}{\em RETURN VALUE:}
\begin{verbatim}      none.  return code is passed thru the parameter list\end{verbatim}{\em ARGUMENTS:}
\begin{verbatim}       char *command,            // out - command line
       int *rc,                  // out - return code
       ESMCI_FortranStrLenArg command_l) { // in, hidden - command length\end{verbatim}
{\sf DESCRIPTION:\\ }


       Execute a command line.  Return when complete.
   
%/////////////////////////////////////////////////////////////
 
\mbox{}\hrulefill\
 
\subsubsection [c\_ESMC\_GetCWD] {c\_ESMC\_GetCWD - Get the current directory from the file system}


  
\bigskip{\sf INTERFACE:}
\begin{verbatim}       void FTN_X(c_esmc_getcwd)(\end{verbatim}{\em RETURN VALUE:}
\begin{verbatim}      none.  return code is passed thru the parameter list\end{verbatim}{\em ARGUMENTS:}
\begin{verbatim}       char *pathname,           // out - path name
       int *rc,                  // out - return code
       ESMCI_FortranStrLenArg pathname_l) { // in, hidden - pathname length\end{verbatim}
{\sf DESCRIPTION:\\ }


       Gets the current directory in the file system.
  
%...............................................................
\setlength{\parskip}{\oldparskip}
\setlength{\parindent}{\oldparindent}
\setlength{\baselineskip}{\oldbaselineskip}
