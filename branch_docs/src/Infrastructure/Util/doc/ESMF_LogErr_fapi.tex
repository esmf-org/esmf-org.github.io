%                **** IMPORTANT NOTICE *****
% This LaTeX file has been automatically produced by ProTeX v. 1.1
% Any changes made to this file will likely be lost next time
% this file is regenerated from its source. Send questions 
% to Arlindo da Silva, dasilva@gsfc.nasa.gov
 
\setlength{\oldparskip}{\parskip}
\setlength{\parskip}{1.5ex}
\setlength{\oldparindent}{\parindent}
\setlength{\parindent}{0pt}
\setlength{\oldbaselineskip}{\baselineskip}
\setlength{\baselineskip}{11pt}
 
%--------------------- SHORT-HAND MACROS ----------------------
\def\bv{\begin{verbatim}}
\def\ev{\end{verbatim}}
\def\be{\begin{equation}}
\def\ee{\end{equation}}
\def\bea{\begin{eqnarray}}
\def\eea{\end{eqnarray}}
\def\bi{\begin{itemize}}
\def\ei{\end{itemize}}
\def\bn{\begin{enumerate}}
\def\en{\end{enumerate}}
\def\bd{\begin{description}}
\def\ed{\end{description}}
\def\({\left (}
\def\){\right )}
\def\[{\left [}
\def\]{\right ]}
\def\<{\left  \langle}
\def\>{\right \rangle}
\def\cI{{\cal I}}
\def\diag{\mathop{\rm diag}}
\def\tr{\mathop{\rm tr}}
%-------------------------------------------------------------

\markboth{Left}{Source File: ESMF\_LogErr.F90,  Date: Tue May  5 20:59:26 MDT 2020
}

 
%/////////////////////////////////////////////////////////////
\subsubsection [ESMF\_LogAssignment(=)] {ESMF\_LogAssignment(=) - Log assignment}


  
\bigskip{\sf INTERFACE:}
\begin{verbatim}     interface assignment(=)
     log1 = log2\end{verbatim}{\em ARGUMENTS:}
\begin{verbatim}     type(ESMF_Log) :: log1
     type(ESMF_Log) :: log2\end{verbatim}
{\sf DESCRIPTION:\\ }


     Assign {\tt log1} as an alias to the same {\tt ESMF\_Log} object in memory
     as {\tt log2}. If {\tt log2} is invalid, then {\tt log1} will be
     equally invalid after the assignment.
  
     The arguments are:
     \begin{description}
     \item[log1]
       The {\tt ESMF\_Log} object on the left hand side of the assignment.
     \item[log2]
       The {\tt ESMF\_Log} object on the right hand side of the assignment.
     \end{description}
   
%/////////////////////////////////////////////////////////////
 
\mbox{}\hrulefill\ 
 
\subsubsection [ESMF\_LogOperator(==)] {ESMF\_LogOperator(==) - Test if Log 1 is equivalent to Log 2}


  
\bigskip{\sf INTERFACE:}
\begin{verbatim}       interface operator(==)
       if (log1 == log2) then ... endif
                    OR
       result = (log1 == log2)\end{verbatim}{\em RETURN VALUE:}
\begin{verbatim}       logical :: result\end{verbatim}{\em ARGUMENTS:}
\begin{verbatim}       type(ESMF_Log), intent(in) :: log1
       type(ESMF_Log), intent(in) :: log2\end{verbatim}
{\sf DESCRIPTION:\\ }


       Overloads the (==) operator for the {\tt ESMF\_Log} class.
       Compare two logs for equality; return {\tt .true.} if equal,
       {\tt .false.} otherwise. Comparison is based on whether the objects
       are distinct, as with two newly created logs, or are simply aliases
       to the same log as would be the case when assignment was involved.
  
       The arguments are:
       \begin{description}
       \item[log1]
            The {\tt ESMF\_Log} object on the left hand side of the equality
            operation.
       \item[log2]
            The {\tt ESMF\_Log} object on the right hand side of the equality
            operation.
       \end{description}
   
%/////////////////////////////////////////////////////////////
 
\mbox{}\hrulefill\ 
 
\subsubsection [ESMF\_LogOperator(/=)] {ESMF\_LogOperator(/=) - Test if Log 1 is not equivalent to Log 2}


  
\bigskip{\sf INTERFACE:}
\begin{verbatim}       interface operator(/=)
       if (log1 /= log2) then ... endif
                    OR
       result = (log1 /= log2)\end{verbatim}{\em RETURN VALUE:}
\begin{verbatim}       logical :: result\end{verbatim}{\em ARGUMENTS:}
\begin{verbatim}       type(ESMF_Log), intent(in) :: log1
       type(ESMF_Log), intent(in) :: log2\end{verbatim}
{\sf DESCRIPTION:\\ }


       Overloads the (/=) operator for the {\tt ESMF\_Log} class.
       Compare two logs for inequality; return {\tt .true.} if equal,
       {\tt .false.} otherwise.  Comparison is based on whether the objects
       are distinct, as with two newly created logs, or are simply aliases
       to the same log as would be the case when assignment was involved.
  
       The arguments are:
       \begin{description}
       \item[log1]
            The {\tt ESMF\_Log} object on the left hand side of the non-equality
            operation.
       \item[log2]
            The {\tt ESMF\_Log} object on the right hand side of the non-equality
            operation.
       \end{description}
   
%/////////////////////////////////////////////////////////////
 
\mbox{}\hrulefill\ 
 
\subsubsection [ESMF\_LogClose] {ESMF\_LogClose - Close Log file(s)}


 
\bigskip{\sf INTERFACE:}
\begin{verbatim}       subroutine ESMF_LogClose(log, rc)\end{verbatim}{\em ARGUMENTS:}
\begin{verbatim}       type(ESMF_Log), intent(inout), optional :: log
 -- The following arguments require argument keyword syntax (e.g. rc=rc). --
       integer,        intent(out), optional :: rc
 \end{verbatim}
{\sf STATUS:}
   \begin{itemize}
   \item\apiStatusCompatibleVersion{5.2.0r}
   \end{itemize}
  
{\sf DESCRIPTION:\\ }


        This routine closes the log file(s) associated with {\tt log}.
        If the log is not explicitly closed, it will be closed by
        {\tt ESMF\_Finalize}.
  
        The arguments are:
        \begin{description}
  
        \item [{[log]}]
              An {\tt ESMF\_Log} object.  If not specified, the default log is closed.
        \item [{[rc]}]
              Return code; equals {\tt ESMF\_SUCCESS} if there are no errors.
        \end{description}
   
%/////////////////////////////////////////////////////////////
 
\mbox{}\hrulefill\ 
 
\subsubsection [ESMF\_LogFlush] {ESMF\_LogFlush - Flush the Log file(s)}


 
\bigskip{\sf INTERFACE:}
\begin{verbatim}       subroutine ESMF_LogFlush(log, rc)\end{verbatim}{\em ARGUMENTS:}
\begin{verbatim}       type(ESMF_Log), intent(inout), optional :: log
 -- The following arguments require argument keyword syntax (e.g. rc=rc). --
       integer,        intent(out),   optional :: rc
 \end{verbatim}
{\sf STATUS:}
   \begin{itemize}
   \item\apiStatusCompatibleVersion{5.2.0r}
   \end{itemize}
  
{\sf DESCRIPTION:\\ }


        This subroutine flushes the file buffer associated with {\tt log}.
  
        The arguments are:
        \begin{description}
  
        \item [{[log]}]
              An optional {\tt ESMF\_Log} object that can be used instead
              of the default Log.
        \item [{[rc]}]
              Return code; equals {\tt ESMF\_SUCCESS} if there are no errors.
        \end{description}
   
%/////////////////////////////////////////////////////////////
 
\mbox{}\hrulefill\ 
 
\subsubsection [ESMF\_LogFoundAllocError] {ESMF\_LogFoundAllocError - Check Fortran allocation status error and write message}


 
\bigskip{\sf INTERFACE:}
\begin{verbatim}       function ESMF_LogFoundAllocError(statusToCheck,  &
                                        msg,line,file, &
                                        method,rcToReturn,log)\end{verbatim}{\em RETURN VALUE:}
\begin{verbatim}       logical                                    :: ESMF_LogFoundAllocError\end{verbatim}{\em ARGUMENTS:}
\begin{verbatim}       integer,          intent(in)              :: statusToCheck
 -- The following arguments require argument keyword syntax (e.g. rc=rc). --
       character(len=*), intent(in),    optional :: msg
       integer,          intent(in),    optional :: line
       character(len=*), intent(in),    optional :: file
       character(len=*), intent(in),    optional :: method
       integer,          intent(inout), optional :: rcToReturn
       type(ESMF_Log),   intent(inout), optional :: log
 \end{verbatim}
{\sf STATUS:}
   \begin{itemize}
   \item\apiStatusCompatibleVersion{5.2.0r}
   \end{itemize}
  
{\sf DESCRIPTION:\\ }


        This function returns {\tt .true.} when {\tt statusToCheck} indicates
        an allocation error, otherwise it returns {\tt .false.}.  The status
        value is typically returned from a Fortran ALLOCATE statement.
        If an error is indicated, a ESMF memory allocation error message
        will be written to the {\tt ESMF\_Log} along with a user added {\tt msg},
        {\tt line}, {\tt file} and {\tt method}.
  
        The arguments are:
        \begin{description}
  
        \item [statusToCheck]
              Fortran allocation status to check.  Fortran specifies
              that a status of 0 (zero) indicates success.
        \item [{[msg]}]
              User-provided message string.
        \item [{[line]}]
              Integer source line number.  Expected to be set by
              using the preprocessor {\tt \_\_LINE\_\_} macro.
        \item [{[file]}]
              User-provided source file name.
        \item [{[method]}]
              User-provided method string.
        \item [{[rcToReturn]}]
              If specified, when the allocation status indicates an error,
              set the {\tt rcToReturn} value to {\tt ESMF\_RC\_MEM}.  Otherwise,
              {\tt rcToReturn} is not modified.
        \item [{[log]}]
              An optional {\tt ESMF\_Log} object that can be used instead
              of the default Log.
  
        \end{description}
   
%/////////////////////////////////////////////////////////////
 
\mbox{}\hrulefill\ 
 
\subsubsection [ESMF\_LogFoundDeallocError] {ESMF\_LogFoundDeallocError - Check Fortran deallocation status error and write message}


 
\bigskip{\sf INTERFACE:}
\begin{verbatim}       function ESMF_LogFoundDeallocError(statusToCheck,  &
                                          msg,line,file, &
                                          method,rcToReturn,log)\end{verbatim}{\em RETURN VALUE:}
\begin{verbatim}       logical ::ESMF_LogFoundDeallocError\end{verbatim}{\em ARGUMENTS:}
\begin{verbatim}       integer,          intent(in)              :: statusToCheck
 -- The following arguments require argument keyword syntax (e.g. rc=rc). --
       character(len=*), intent(in),    optional :: msg
       integer,          intent(in),    optional :: line
       character(len=*), intent(in),    optional :: file
       character(len=*), intent(in),    optional :: method
       integer,          intent(inout), optional :: rcToReturn
       type(ESMF_Log),   intent(inout), optional :: log
 \end{verbatim}
{\sf STATUS:}
   \begin{itemize}
   \item\apiStatusCompatibleVersion{5.2.0r}
   \end{itemize}
  
{\sf DESCRIPTION:\\ }


        This function returns {\tt .true.} when {\tt statusToCheck} indicates
        a deallocation error, otherwise it returns {\tt .false.}.  The status
        value is typically returned from a Fortran DEALLOCATE statement.
        If an error is indicated, a ESMF memory allocation error message
        will be written to the {\tt ESMF\_Log} along with a user added {\tt msg},
        {\tt line}, {\tt file} and {\tt method}.
  
        The arguments are:
        \begin{description}
  
        \item [statusToCheck]
              Fortran deallocation status to check.  Fortran specifies
              that a status of 0 (zero) indicates success.
        \item [{[msg]}]
              User-provided message string.
        \item [{[line]}]
              Integer source line number.  Expected to be set by
              using the preprocessor {\tt \_\_LINE\_\_} macro.
        \item [{[file]}]
              User-provided source file name.
        \item [{[method]}]
              User-provided method string.
        \item [{[rcToReturn]}]
              If specified, when the deallocation status indicates an error,
              set the {\tt rcToReturn} value to {\tt ESMF\_RC\_MEM}.  Otherwise,
              {\tt rcToReturn} is not modified.
        \item [{[log]}]
              An optional {\tt ESMF\_Log} object that can be used instead
              of the default Log.
  
        \end{description}
   
%/////////////////////////////////////////////////////////////
 
\mbox{}\hrulefill\ 
 
\subsubsection [ESMF\_LogFoundError] {ESMF\_LogFoundError - Check ESMF return code for error and write message}


 
\bigskip{\sf INTERFACE:}
\begin{verbatim}   recursive function ESMF_LogFoundError(rcToCheck,   &
                                   msg, line, file, method, &
                                   rcToReturn, log) result (LogFoundError)\end{verbatim}{\em RETURN VALUE:}
\begin{verbatim}       logical :: LogFoundError\end{verbatim}{\em ARGUMENTS:}
\begin{verbatim}       integer,          intent(in),    optional :: rcToCheck
 -- The following arguments require argument keyword syntax (e.g. rc=rc). --
       character(len=*), intent(in),    optional :: msg
       integer,          intent(in),    optional :: line
       character(len=*), intent(in),    optional :: file
       character(len=*), intent(in),    optional :: method
       integer,          intent(inout), optional :: rcToReturn
       type(ESMF_Log),   intent(inout), optional :: log
 \end{verbatim}
{\sf STATUS:}
   \begin{itemize}
   \item\apiStatusCompatibleVersion{5.2.0r}
   \end{itemize}
  
{\sf DESCRIPTION:\\ }


        This function returns {\tt .true.} when {\tt rcToCheck} indicates
        an return code other than {\tt ESMF\_SUCCESS}, otherwise it returns
        {\tt .false.}.
        If an error is indicated, a ESMF predefined error message
        will be written to the {\tt ESMF\_Log} along with a user added {\tt msg},
        {\tt line}, {\tt file} and {\tt method}.
  
        The arguments are:
        \begin{description}
  
        \item [{[rcToCheck]}]
              Return code to check. Default is {\tt ESMF\_SUCCESS}.
        \item [{[msg]}]
              User-provided message string.
        \item [{[line]}]
              Integer source line number.  Expected to be set by
              using the preprocessor {\tt \_\_LINE\_\_} macro.
        \item [{[file]}]
              User-provided source file name.
        \item [{[method]}]
              User-provided method string.
        \item [{[rcToReturn]}]
              If specified, when {\tt rcToCheck} indicates an error,
              set the {\tt rcToReturn} to the value of {\tt rcToCheck}.
              Otherwise, {\tt rcToReturn} is not modified.
              This is not the return code for this function; it allows
              the calling code to do an assignment of the error code
              at the same time it is testing the value.
        \item [{[log]}]
              An optional {\tt ESMF\_Log} object that can be used instead
              of the default Log.
  
        \end{description}
   
%/////////////////////////////////////////////////////////////
 
\mbox{}\hrulefill\ 
 
\subsubsection [ESMF\_LogFoundNetCDFError] {ESMF\_LogFoundNetCDFError - Check NetCDF status code for success or log the associated NetCDF error message.}


 
\bigskip{\sf INTERFACE:}
\begin{verbatim} function ESMF_LogFoundNetCDFError(ncerrToCheck, msg, line, &
                                   file, method, rcToReturn, log)
 
 #if defined ESMF_NETCDF
   use netcdf
 #elif defined ESMF_PNETCDF
   use pnetcdf
 #endif
 \end{verbatim}{\em RETURN VALUE:}
\begin{verbatim}   logical :: ESMF_LogFoundNetCDFError\end{verbatim}{\em ARGUMENTS:}
\begin{verbatim}   integer,          intent(in)              :: ncerrToCheck
 -- The following arguments require argument keyword syntax (e.g. rc=rc). --
   character(len=*), intent(in),    optional :: msg
   integer,          intent(in),    optional :: line
   character(len=*), intent(in),    optional :: file
   character(len=*), intent(in),    optional :: method
   integer,          intent(inout), optional :: rcToReturn
   type(ESMF_Log),   intent(inout), optional :: log\end{verbatim}
{\sf DESCRIPTION:\\ }


        This function returns {\tt .true.} when {\tt ncerrToCheck} indicates
        an return code other than {\tt 0} (the success code from NetCDF Fortran)
        or {\tt NF\_NOERR} (the success code for PNetCDF). Otherwise it returns
        {\tt .false.}.
        If an error is indicated, a predefined ESMF error message
        will be written to the {\tt ESMF\_Log} along with a user added {\tt msg},
        {\tt line}, {\tt file} and {\tt method}. The NetCDF string error
        representation will also be logged.
  
        The arguments are:
        \begin{description}
  
        \item [{[ncerrToCheck]}]
              NetCDF error code to check.
        \item [{[msg]}]
              User-provided message string.
        \item [{[line]}]
              Integer source line number.  Expected to be set by using the
              preprocessor {\tt \_\_LINE\_\_} macro.
        \item [{[file]}]
              User-provided source file name.
        \item [{[method]}]
              User-provided method string.
        \item [{[rcToReturn]}]
              If specified, when {\tt ncerrToCheck} indicates an error,
              set {\tt rcToReturn} to {\tt ESMF\_RC\_NETCDF\_ERROR}. The string
              representation for the error code will be retrieved from the NetCDF
              Fortran library and logged alongside any user-provided message
              string.
              Otherwise, {\tt rcToReturn} is not modified.
              This is not the return code for this function; it allows the
              calling code to do an assignment of the error code at the same time
              it is testing the value.
        \item [{[log]}]
              An optional {\tt ESMF\_Log} object that can be used instead
              of the default Log.
  
        \end{description}
   
%/////////////////////////////////////////////////////////////
 
\mbox{}\hrulefill\ 
 
\subsubsection [ESMF\_LogGet] {ESMF\_LogGet - Return information about a log object}


 
\bigskip{\sf INTERFACE:}
\begin{verbatim}       subroutine ESMF_LogGet(log,  &
                              flush,    &
                              logmsgAbort, logkindflag, &
                              maxElements, trace, fileName,  &
                              highResTimestampFlag, indentCount,  &
                              noPrefix, rc)\end{verbatim}{\em ARGUMENTS:}
\begin{verbatim}       type(ESMF_Log),          intent(in),  optional :: log
 -- The following arguments require argument keyword syntax (e.g. rc=rc). --
       logical,                 intent(out), optional :: flush
       type(ESMF_LogMsg_Flag),  pointer,     optional :: logmsgAbort(:)
       type(ESMF_LogKind_Flag), intent(out), optional :: logkindflag
       integer,                 intent(out), optional :: maxElements
       logical,                 intent(out), optional :: trace
       character(*),            intent(out), optional :: fileName
       logical,                 intent(out), optional :: highResTimestampFlag
       integer,                 intent(out), optional :: indentCount
       logical,                 intent(out), optional :: noPrefix
       integer,                 intent(out), optional :: rc
 \end{verbatim}
{\sf DESCRIPTION:\\ }


        This subroutine returns properties about a Log object.
  
        The arguments are:
        \begin{description}
  
        \item [{[log]}]
              An optional {\tt ESMF\_Log} object that can be used instead
              of the default Log.
        \item [{[flush]}]
              Flush flag.
        \item [{[logmsgAbort]}]
              Returns an array containing current message halt settings.
              If the array is not pre-allocated, {\tt ESMF\_LogGet} will
              allocate an array of the correct size.  If no message types
              are defined, an array of length zero is returned.  It is the
              callers responsibility to deallocate the array.
        \item [{[logkindflag]}]
              Defines either single or multilog.
        \item [{[maxElements]}]
              Maximum number of elements in the Log.
        \item [{[trace]}]
              Current setting of the Log call tracing flag.
        \item [{[fileName]}]
              Current file name.  When the log has been opened with
              {\tt ESMF\_LOGKIND\_MULTI}, the filename has a PET number
              prefix.
        \item [{[highResTimestampFlag]}]
              Current setting of the extended elapsed timestamp flag.
        \item [{[indentCount]}]
              Current setting of the leading white space padding.
        \item [{[noPrefix]}]
              Current setting of the message prefix enable/disable flag.
        \item [{[rc]}]
              Return code; equals {\tt ESMF\_SUCCESS} if there are no errors.
        \end{description}
   
%/////////////////////////////////////////////////////////////
 
\mbox{}\hrulefill\ 
 
\subsubsection [ESMF\_LogOpen] {ESMF\_LogOpen - Open Log file(s)}


 
\bigskip{\sf INTERFACE:}
\begin{verbatim}     subroutine ESMF_LogOpen(log, filename,  &
         appendflag, logkindflag, noPrefix, rc)\end{verbatim}{\em ARGUMENTS:}
\begin{verbatim}     type(ESMF_Log),          intent(inout)         :: log
     character(len=*),        intent(in)            :: filename
 -- The following arguments require argument keyword syntax (e.g. rc=rc). --
     logical,                 intent(in),  optional :: appendFlag
     type(ESMF_LogKind_Flag), intent(in),  optional :: logkindFlag
     logical,                 intent(in),  optional :: noPrefix
     integer,                 intent(out), optional :: rc
 \end{verbatim}
{\sf DESCRIPTION:\\ }


        This routine opens a file named {\tt filename} and associates
        it with the {\tt ESMF\_Log}.  When {\tt logkindflag} is set to
        {\tt ESMF\_LOGKIND\_MULTI} or {\tt ESMF\_LOGKIND\_MULTI\_ON\_ERROR}
        the file name is prepended with PET number identification.  If the
        incoming log is already open, an error is returned.
  
        The arguments are:
        \begin{description}
        \item [log]
              An {\tt ESMF\_Log} object.
        \item [filename]
              Name of log file to be opened.
        \item [{[appendFlag]}]
              If the log file exists, setting to {\tt .false.} will set the file position
              to the beginning of the file.  Otherwise, new records will be appended to the
              end of the file.  If not specified, defaults to {\tt .true.}.
        \item [{[logkindFlag]}]
              Set the logkindflag. See section \ref{const:logkindflag} for a list of
              valid options.  When the {\tt ESMF\_LOGKIND\_MULTI\_ON\_ERROR} is selected,
              the log opening is deferred until a {\tt ESMF\_LogWrite} with log message of
              type {\tt ESMF\_LOGMSG\_ERROR} is written.
              If not specified, defaults to {\tt ESMF\_LOGKIND\_MULTI}.
        \item [{[noPrefix]}]
              Set the noPrefix flag.  If set to {\tt .false.}, log messages are prefixed
              with time stamps, message type, and PET number.  If set to {\tt .true.} the
              messages will be written without prefixes.  If not specified, defaults to
              {\tt .false.}.
        \item [{[rc]}]
              Return code; equals {\tt ESMF\_SUCCESS} if there are no errors.
        \end{description}
   
%/////////////////////////////////////////////////////////////
 
\mbox{}\hrulefill\ 
 
\subsubsection [ESMF\_LogOpen] {ESMF\_LogOpen - Open Default Log file(s)}


 
\bigskip{\sf INTERFACE:}
\begin{verbatim}   ! Private name; call using ESMF_LogOpen ()
     subroutine ESMF_LogOpenDefault (filename,  &
         appendflag, logkindflag, rc)\end{verbatim}{\em ARGUMENTS:}
\begin{verbatim}     character(len=*),        intent(in)            :: filename
 -- The following arguments require argument keyword syntax (e.g. rc=rc). --
     logical,                 intent(in),  optional :: appendflag
     type(ESMF_LogKind_Flag), intent(in),  optional :: logkindflag
     integer,                 intent(out), optional :: rc
 \end{verbatim}
{\sf DESCRIPTION:\\ }


        This routine opens a file named {\tt filename} and associates
        it with the default log.  When {\tt logkindflag} is set to
        {\tt ESMF\_LOGKIND\_MULTI} the file name is prepended with PET
        number identification.  If the incoming default log is already open,
        an error is returned.
  
        The arguments are:
        \begin{description}
        \item [filename]
              Name of DEFAULT log file to be opened.
        \item [{[appendflag]}]
              If the log file exists, setting to {\tt .false.} will set the file position
              to the beginning of the file.  Otherwise, new records will be appended to the
              end of the file.  If not specified, defaults to {\tt .true.}.
        \item [{[logkindflag]}]
              Set the logkindflag. See section \ref{const:logkindflag} for a list of
              valid options.
              If not specified, defaults to {\tt ESMF\_LOGKIND\_MULTI}.
        \item [{[rc]}]
              Return code; equals {\tt ESMF\_SUCCESS} if there are no errors.
        \end{description}
   
%/////////////////////////////////////////////////////////////
 
\mbox{}\hrulefill\ 
 
\subsubsection [ESMF\_LogSet] {ESMF\_LogSet - Set Log parameters}


 
\bigskip{\sf INTERFACE:}
\begin{verbatim}     subroutine ESMF_LogSet(log,  &
         flush,  &
         logmsgAbort, maxElements, logmsgList,  &
         errorMask, trace, highResTimestampFlag, indentCount,  &
         noPrefix, rc)\end{verbatim}{\em ARGUMENTS:}
\begin{verbatim}       type(ESMF_Log),         intent(inout), optional :: log
 -- The following arguments require argument keyword syntax (e.g. rc=rc). --
       logical,                intent(in),    optional :: flush
       type(ESMF_LogMsg_Flag), intent(in),    optional :: logmsgAbort(:)
       integer,                intent(in),    optional :: maxElements
       type(ESMF_LogMsg_Flag), intent(in),    optional :: logmsgList(:)
       integer,                intent(in),    optional :: errorMask(:)
       logical,                intent(in),    optional :: trace
       logical,                intent(in),    optional :: highResTimestampFlag
       integer,                intent(in),    optional :: indentCount
       logical,                intent(in),    optional :: noPrefix
       integer,                intent(out),   optional :: rc
 \end{verbatim}
{\sf DESCRIPTION:\\ }


        This subroutine sets the properties for the Log object.
  
        The arguments are:
        \begin{description}
  
        \item [{[log]}]
              An optional {\tt ESMF\_Log} object.  The default is to use the
              default log that was opened at {\tt ESMF\_Initialize} time.
        \item [{[flush]}]
              If set to {\tt .true.}, flush log messages immediately, rather
              than buffering them.  Default is to flush after {\tt maxElements}
              messages.
        \item [{[logmsgAbort]}]
              Sets the condition on which ESMF aborts.  The array
              can contain any combination of {\tt ESMF\_LOGMSG} named constants.  These
              named constants are described in section \ref{const:logmsgflag}.
              Default is to always continue processing.
        \item [{[maxElements]}]
              Maximum number of elements in the Log buffer before flushing occurs.
              Default is to flush when 10 messages have been accumulated.
        \item [{[logmsgList]}]
              An array of message types that will be logged.  Log write requests
              not matching the list will be ignored.  If an empty array is
              provided, no messages will be logged.
              See section \ref{const:logmsgflag} for a list of
              valid message types.  By default, all non-trace messages will be
              logged.
        \item [{[errorMask]}]
              List of error codes that will {\em not} be logged as errors.
              Default is to log all error codes.
        \item [{[trace]}]
              \begin{sloppypar}
              If set to {\tt .true.}, calls such as {\tt ESMF\_LogFoundError()},
              {\tt ESMF\_LogFoundAllocError()}, and
              {\tt ESMF\_LogFoundDeallocError()}
              will be logged in the default log files.  This option is intended
              to be used as a tool for debugging and program flow tracing
              within the ESMF library. Voluminous output may appear in the log,
              with a consequent slowdown in performance.  Therefore, it is
              recommended that this option only be enabled before a problematic
              call to a ESMF method, and disabled afterwards. Default is to
              not trace these calls.
             \end{sloppypar}
        \item [{[highResTimestampFlag]}]
              Sets the extended elapsed timestamp flag.  If set to {\tt .true.}, a timestamp
              from {\tt ESMF\_VMWtime} will be included in each log message.  Default is
              to not add the additional timestamps.
        \item [{[indentCount]}]
              Number of leading white spaces.
        \item [{[noPrefix]}]
              If set to {\tt .false.}, log messages are prefixed with time stamps,
              message type and PET number.  If set to {\tt .true.} the messages will be
              written without the prefixes.
        \item [{[rc]}]
              Return code; equals {\tt ESMF\_SUCCESS} if there are no errors.
        \end{description}
   
%/////////////////////////////////////////////////////////////
 
\mbox{}\hrulefill\ 
 
\subsubsection [ESMF\_LogSetError] {ESMF\_LogSetError - Set ESMF return code for error and write msg}


 
\bigskip{\sf INTERFACE:}
\begin{verbatim}       subroutine ESMF_LogSetError(rcToCheck,  &
                                   msg, line, file, method, &
                                   rcToReturn, log)
 \end{verbatim}{\em ARGUMENTS:}
\begin{verbatim}       integer,          intent(in)              :: rcToCheck
 -- The following arguments require argument keyword syntax (e.g. rc=rc). --
       character(len=*), intent(in),    optional :: msg
       integer,          intent(in),    optional :: line
       character(len=*), intent(in),    optional :: file
       character(len=*), intent(in),    optional :: method
       integer,          intent(out),   optional :: rcToReturn
       type(ESMF_Log),   intent(inout), optional :: log
 \end{verbatim}
{\sf STATUS:}
   \begin{itemize}
   \item\apiStatusCompatibleVersion{5.2.0r}
   \end{itemize}
  
{\sf DESCRIPTION:\\ }


        This subroutine sets the {\tt rcToReturn} value to {\tt rcToCheck} if
        {\tt rcToReturn} is present and writes this error code to the {\tt ESMF\_Log}
        if an error is generated.  A predefined error message will added to the
        {\tt ESMF\_Log} along with a user added {\tt msg}, {\tt line}, {\tt file}
        and {\tt method}.
  
        The arguments are:
        \begin{description}
  
        \item [rcToCheck]
              rc value for set
        \item [{[msg]}]
              User-provided message string.
        \item [{[line]}]
              Integer source line number.  Expected to be set by
              using the preprocessor macro {\tt \_\_LINE\_\_} macro.
        \item [{[file]}]
              User-provided source file name.
        \item [{[method]}]
              User-provided method string.
        \item [{[rcToReturn]}]
              If specified, copy the {\tt rcToCheck} value to {\tt rcToreturn}.
              This is not the return code for this function; it allows
              the calling code to do an assignment of the error code
              at the same time it is testing the value.
        \item [{[log]}]
              An optional {\tt ESMF\_Log} object that can be used instead
              of the default Log.
  
        \end{description}
   
%/////////////////////////////////////////////////////////////
 
\mbox{}\hrulefill\ 
 
\subsubsection [ESMF\_LogWrite] {ESMF\_LogWrite - Write to Log file(s)}


 
\bigskip{\sf INTERFACE:}
\begin{verbatim}       recursive subroutine ESMF_LogWrite(msg, logmsgFlag, &
                         logmsgList,      & ! DEPRECATED ARGUMENT
                         line, file, method, log, rc)\end{verbatim}{\em ARGUMENTS:}
\begin{verbatim}       character(len=*),      intent(in)             :: msg
       type(ESMF_LogMsg_Flag),intent(in),optional    :: logmsgFlag
       type(ESMF_LogMsg_Flag),intent(in),optional::logmsgList ! DEPRECATED ARG
 -- The following arguments require argument keyword syntax (e.g. rc=rc). --
       integer,               intent(in),   optional :: line
       character(len=*),      intent(in),   optional :: file
       character(len=*),      intent(in),   optional :: method
       type(ESMF_Log),        intent(inout),optional :: log
       integer,               intent(out),  optional :: rc
 \end{verbatim}
{\sf STATUS:}
   \begin{itemize}
   \item\apiStatusCompatibleVersion{5.2.0r}
   \item\apiStatusModifiedSinceVersion{5.2.0r}
   \begin{description}
   \item[5.2.0rp1] Added argument {\tt logmsgFlag}.
                   Started to deprecate argument {\tt logmsgList}.
                   This corrects inconsistent use of the {\tt List} suffix on
                   the argument name. In ESMF this suffix indicates
                   one--dimensional array arguments.
   \end{description}
   \end{itemize}
  
{\sf DESCRIPTION:\\ }


        This subroutine writes to the file associated with an {\tt ESMF\_Log}.
        A message is passed in along with the {\tt logmsgFlag}, {\tt line},
        {\tt file} and {\tt method}.  If the write to the {\tt ESMF\_Log}
        is successful, the function will return a logical {\tt true}.  This
        function is the base function used by all the other {\tt ESMF\_Log}
        writing methods.
  
        The arguments are:
        \begin{description}
  
        \item [msg]
              User-provided message string.
        \item [{[logmsgFlag]}]
              The type of message.  See Section~\ref{const:logmsgflag} for
              possible values.  If not specified, the default is {\tt ESMF\_LOGMSG\_INFO}.
        \item [{[logmsgList]}]
              \apiDeprecatedArgWithReplacement{logmsgFlag}
        \item [{[line]}]
              Integer source line number.  Expected to be set by
              using the preprocessor macro {\tt \_\_LINE\_\_} macro.
        \item [{[file]}]
              User-provided source file name.
        \item [{[method]}]
              User-provided method string.
        \item [{[log]}]
              An optional {\tt ESMF\_Log} object that can be used instead
              of the default Log.
        \item [{[rc]}]
              Return code; equals {\tt ESMF\_SUCCESS} if there are no errors.
        \end{description}
  
%...............................................................
\setlength{\parskip}{\oldparskip}
\setlength{\parindent}{\oldparindent}
\setlength{\baselineskip}{\oldbaselineskip}
