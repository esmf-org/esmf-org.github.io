%                **** IMPORTANT NOTICE *****
% This LaTeX file has been automatically produced by ProTeX v. 1.1
% Any changes made to this file will likely be lost next time
% this file is regenerated from its source. Send questions 
% to Arlindo da Silva, dasilva@gsfc.nasa.gov
 
\setlength{\oldparskip}{\parskip}
\setlength{\parskip}{1.5ex}
\setlength{\oldparindent}{\parindent}
\setlength{\parindent}{0pt}
\setlength{\oldbaselineskip}{\baselineskip}
\setlength{\baselineskip}{11pt}
 
%--------------------- SHORT-HAND MACROS ----------------------
\def\bv{\begin{verbatim}}
\def\ev{\end{verbatim}}
\def\be{\begin{equation}}
\def\ee{\end{equation}}
\def\bea{\begin{eqnarray}}
\def\eea{\end{eqnarray}}
\def\bi{\begin{itemize}}
\def\ei{\end{itemize}}
\def\bn{\begin{enumerate}}
\def\en{\end{enumerate}}
\def\bd{\begin{description}}
\def\ed{\end{description}}
\def\({\left (}
\def\){\right )}
\def\[{\left [}
\def\]{\right ]}
\def\<{\left  \langle}
\def\>{\right \rangle}
\def\cI{{\cal I}}
\def\diag{\mathop{\rm diag}}
\def\tr{\mathop{\rm tr}}
%-------------------------------------------------------------

\markboth{Left}{Source File: ESMF\_Util.F90,  Date: Tue May  5 20:59:27 MDT 2020
}

 
%/////////////////////////////////////////////////////////////

  
%/////////////////////////////////////////////////////////////
 
\mbox{}\hrulefill\ 
 
\subsubsection [ESMF\_UtilGetArg] {ESMF\_UtilGetArg - Return a command line argument}


  
\bigskip{\sf INTERFACE:}
\begin{verbatim}   subroutine ESMF_UtilGetArg(argindex, argvalue, arglength, rc)\end{verbatim}{\em ARGUMENTS:}
\begin{verbatim}     integer,      intent(in)            :: argindex
 -- The following arguments require argument keyword syntax (e.g. rc=rc). --
     character(*), intent(out), optional :: argvalue
     integer,      intent(out), optional :: arglength
     integer,      intent(out), optional :: rc\end{verbatim}
{\sf STATUS:}
   \begin{itemize}
   \item\apiStatusCompatibleVersion{5.2.0r}
   \end{itemize}
  
{\sf DESCRIPTION:\\ }


   This method returns a copy of a command line argument specified
   when the process was started.  This argument is the same as an
   equivalent C++ program would find in the argv array.
  
   Some MPI implementations do not consistently provide command line
   arguments on PETs other than PET 0.  It is therefore recommended
   that PET 0 call this method and broadcast the results to the other
   PETs by using the {\tt ESMF\_VMBroadcast()} method.
  
   The arguments are:
  
   \begin{description}
   \item [{argindex}]
   A non-negative index into the command line argument {\tt argv} array.
   If argindex is negative or greater than the number of user-specified
   arguments, {\tt ESMF\_RC\_ARG\_VALUE} is returned in the {\tt rc} argument.
   \item [{[argvalue]}]
   Returns a copy of the desired command line argument.  If the provided
   character string is longer than the command line argument, the string
   will be blank padded.  If the string is too short, truncation will
   occur and {\tt ESMF\_RC\_ARG\_SIZE} is returned in the {\tt rc} argument.
   \item [{[arglength]}]
   Returns the length of the desired command line argument in characters.
   The length result does not depend on the length of the {\tt value}
   string.  It may be used to query the length of the argument.
   \item [{[rc]}]
   Return code; equals {\tt ESMF\_SUCCESS} if there are no errors.
   \end{description} 
%/////////////////////////////////////////////////////////////
 
\mbox{}\hrulefill\ 
 
\subsubsection [ESMF\_UtilGetArgC] {ESMF\_UtilGetArgC - Return number of command line arguments}


  
\bigskip{\sf INTERFACE:}
\begin{verbatim}   subroutine ESMF_UtilGetArgC(count, rc)\end{verbatim}{\em ARGUMENTS:}
\begin{verbatim}     integer, intent(out)           :: count
 -- The following arguments require argument keyword syntax (e.g. rc=rc). --
     integer, intent(out), optional :: rc\end{verbatim}
{\sf STATUS:}
   \begin{itemize}
   \item\apiStatusCompatibleVersion{5.2.0r}
   \end{itemize}
  
{\sf DESCRIPTION:\\ }


   This method returns the number of command line arguments specified
   when the process was started.
  
   The number of arguments returned does not include the name of the
   command itself - which is typically returned as argument zero.
  
   Some MPI implementations do not consistently provide command line
   arguments on PETs other than PET 0.  It is therefore recommended
   that PET 0 call this method and broadcast the results to the other
   PETs by using the {\tt ESMF\_VMBroadcast()} method.
  
   The arguments are:
  
   \begin{description}
   \item [count]
   Count of command line arguments.
   \item [{[rc]}]
   Return code; equals {\tt ESMF\_SUCCESS} if there are no errors.
   \end{description}
   
%/////////////////////////////////////////////////////////////
 
\mbox{}\hrulefill\ 
 
\subsubsection [ESMF\_UtilGetArgIndex] {ESMF\_UtilGetArgIndex - Return the index of a command line argument}


  
\bigskip{\sf INTERFACE:}
\begin{verbatim}   subroutine ESMF_UtilGetArgIndex(argvalue, argindex, rc)\end{verbatim}{\em ARGUMENTS:}
\begin{verbatim}     character(*), intent(in)            :: argvalue
 -- The following arguments require argument keyword syntax (e.g. rc=rc). --
     integer,      intent(out), optional :: argindex
     integer,      intent(out), optional :: rc\end{verbatim}
{\sf STATUS:}
   \begin{itemize}
   \item\apiStatusCompatibleVersion{5.2.0r}
   \end{itemize}
  
{\sf DESCRIPTION:\\ }


   This method searches for, and returns the index of a desired command
   line argument.  An example might be to find a specific keyword
   (e.g., -esmf\_path) so that its associated value argument could be
   obtained by adding 1 to the argindex and calling {\tt ESMF\_UtilGetArg()}.
  
   Some MPI implementations do not consistently provide command line
   arguments on PETs other than PET 0.  It is therefore recommended
   that PET 0 call this method and broadcast the results to the other
   PETs by using the {\tt ESMF\_VMBroadcast()} method.
  
   The arguments are:
  
   \begin{description}
   \item [argvalue]
   A character string which will be searched for in the command line
   argument list.
   \item [{[argindex]}]
   If the {\tt value} string is found, the position will be returned
   as a non-negative integer.  If the string is not found, a negative
   value will be returned.
   \item [{[rc]}]
   Return code; equals {\tt ESMF\_SUCCESS} if there are no errors.
   \end{description} 
%/////////////////////////////////////////////////////////////
 
\mbox{}\hrulefill\ 
 
\subsubsection [ESMF\_UtilIOGetCWD] {ESMF\_UtilIOGetCWD - Get the current directory}


  
\bigskip{\sf INTERFACE:}
\begin{verbatim}   subroutine ESMF_UtilIOGetCWD (pathName, rc)\end{verbatim}{\em PARAMETERS:}
\begin{verbatim}     character(*), intent(out)           :: pathName
 -- The following arguments require argument keyword syntax (e.g. rc=rc). --
     integer,      intent(out), optional :: rc\end{verbatim}
{\sf DESCRIPTION:\\ }


     Call the system-dependent routine to get the current directory from the file
     system.
  
       The arguments are:
       \begin{description}
       \item[pathName]
         Name of the current working directory.
       \item[{[rc]}]
         Return code; equals {\tt ESMF\_SUCCESS} if there are no errors.
       \end{description} 
%/////////////////////////////////////////////////////////////
 
\mbox{}\hrulefill\ 
 
\subsubsection [ESMF\_UtilIOMkDir] {ESMF\_UtilIOMkDir - Create a directory in the file system}


  
\bigskip{\sf INTERFACE:}
\begin{verbatim}    subroutine ESMF_UtilIOMkDir (pathName,  &
        mode, relaxedFlag,  &
        rc)\end{verbatim}{\em PARAMETERS:}
\begin{verbatim}      character(*), intent(in)            :: pathName
 -- The following arguments require argument keyword syntax (e.g. rc=rc). --
      integer,      intent(in),  optional :: mode
      logical,      intent(in),  optional :: relaxedFlag
      integer,      intent(out), optional :: rc\end{verbatim}
{\sf DESCRIPTION:\\ }


     Call the system-dependent routine to create a directory in the file system.
  
       The arguments are:
       \begin{description}
       \item[pathName]
         Name of the directory to be created.
       \item[{[mode]}]
         File permission mode.  Typically an octal constant is used as a value, for example:
         {\tt mode=o'755'}.  If not specified on POSIX-compliant systems, the default
         is {\tt o'755'} - corresponding to owner read/write/execute,
         group read/execute, and world read/execute.  On native Windows, this argument is
         ignored and default security settings are used.
       \item[{[relaxedFlag]}]
         When set to {\tt .true.}, if the directory already exists, {\tt rc}
         will be set to {\tt ESMF\_SUCCESS} instead of an error.
         If not specified, the default is {\tt .false.}.
       \item[{[rc]}]
         Return code; equals {\tt ESMF\_SUCCESS} if there are no errors.
       \end{description} 
%/////////////////////////////////////////////////////////////
 
\mbox{}\hrulefill\ 
 
\subsubsection [ESMF\_UtilIORmDir] {ESMF\_UtilIORmDir - Remove a directory from the file system}


  
\bigskip{\sf INTERFACE:}
\begin{verbatim}    subroutine ESMF_UtilIORmDir (pathName,  &
        relaxedFlag, rc)\end{verbatim}{\em PARAMETERS:}
\begin{verbatim}      character(*), intent(in)            :: pathName
 -- The following arguments require argument keyword syntax (e.g. rc=rc). --
      logical,      intent(in),  optional :: relaxedFlag
      integer,      intent(out), optional :: rc\end{verbatim}
{\sf DESCRIPTION:\\ }


     Call the system-dependent routine to remove a directory from the file
     system.  Note that the directory must be empty in order to be successfully
     removed.
  
       The arguments are:
       \begin{description}
       \item[pathName]
         Name of the directory to be removed.
       \item[{[relaxedFlag]}]
         If set to {\tt .true.}, and if the specified directory does not exist,
         the error is ignored and {\tt rc} will be set to {\tt ESMF\_SUCCESS}.
         If not specified, the default is {\tt .false.}.
       \item[{[rc]}]
         Return code; equals {\tt ESMF\_SUCCESS} if there are no errors.
       \end{description} 
%/////////////////////////////////////////////////////////////
 
\mbox{}\hrulefill\ 
 
\subsubsection [ESMF\_UtilString2Double] {ESMF\_UtilString2Double - Convert a string to floating point real}


\bigskip{\sf INTERFACE:}
\begin{verbatim}   function ESMF_UtilString2Double(string, rc)\end{verbatim}{\em RETURN VALUE:}
\begin{verbatim}     real(ESMF_KIND_R8) :: ESMF_UtilString2Double\end{verbatim}{\em ARGUMENTS:}
\begin{verbatim}     character(len=*), intent(in)            :: string
 -- The following arguments require argument keyword syntax (e.g. rc=rc). --
     integer,          intent(out), optional :: rc\end{verbatim}
{\sf DESCRIPTION:\\ }


     Return the numerical real value represented by the {\tt string}.
  
     Leading and trailing blanks in {\tt string} are ignored when directly
     converting into integers.
  
     This procedure may fail when used in an expression in a {\tt write} statement
     with some older, pre-Fortran 2003, compiler environments that do not support
     re-entrant I/O calls.
  
     The arguments are:
     \begin{description}
     \item[string]
       The string to be converted
     \item[{[rc]}]
       Return code; equals {\tt ESMF\_SUCCESS} if there are no errors.
     \end{description}
   
%/////////////////////////////////////////////////////////////
 
\mbox{}\hrulefill\ 
 
\subsubsection [ESMF\_UtilString2Int] {ESMF\_UtilString2Int - Convert a string to an integer}


\bigskip{\sf INTERFACE:}
\begin{verbatim}   function ESMF_UtilString2Int(string,  &
       specialStringList, specialValueList, rc)\end{verbatim}{\em RETURN VALUE:}
\begin{verbatim}     integer :: ESMF_UtilString2Int\end{verbatim}{\em ARGUMENTS:}
\begin{verbatim}     character(len=*), intent(in)            :: string
 -- The following arguments require argument keyword syntax (e.g. rc=rc). --
     character(len=*), intent(in),  optional :: specialStringList(:)
     integer,          intent(in),  optional :: specialValueList(:)
     integer,          intent(out), optional :: rc\end{verbatim}
{\sf DESCRIPTION:\\ }


     Return the numerical integer value represented by the {\tt string}.
     If {\tt string} matches a string in the optional {\tt specialStringList}, the
     corresponding special value will be returned instead.
  
     If special strings are to be taken into account, both 
     {\tt specialStringList} and {\tt specialValueList} arguments must be
     present and of same size.
     
     An error is returned, and return value set to 0, if {\tt string} is not
     found in {\tt specialStringList}, and does not convert into an integer
     value.
  
     Leading and trailing blanks in {\tt string} are ignored when directly
     converting into integers.
  
     This procedure may fail when used in an expression in a {\tt write} statement
     with some older, pre-Fortran 2003, compiler environments that do not support
     re-entrant I/O calls.
  
     The arguments are:
     \begin{description}
     \item[string]
       The string to be converted
     \item[{[specialStringList]}]
       List of special strings.
     \item[{[specialValueList]}]
       List of values associated with special strings.
     \item[{[rc]}]
       Return code; equals {\tt ESMF\_SUCCESS} if there are no errors.
     \end{description}
   
%/////////////////////////////////////////////////////////////
 
\mbox{}\hrulefill\ 
 
\subsubsection [ESMF\_UtilString2Real] {ESMF\_UtilString2Real - Convert a string to floating point real}


\bigskip{\sf INTERFACE:}
\begin{verbatim}   function ESMF_UtilString2Real(string, rc)\end{verbatim}{\em RETURN VALUE:}
\begin{verbatim}     real :: ESMF_UtilString2Real\end{verbatim}{\em ARGUMENTS:}
\begin{verbatim}     character(len=*), intent(in)            :: string
 -- The following arguments require argument keyword syntax (e.g. rc=rc). --
     integer,          intent(out), optional :: rc\end{verbatim}
{\sf DESCRIPTION:\\ }


     Return the numerical real value represented by the {\tt string}.
  
     Leading and trailing blanks in {\tt string} are ignored when directly
     converting into integers.
  
     This procedure may fail when used in an expression in a {\tt write} statement
     with some older, pre-Fortran 2003, compiler environments that do not support
     re-entrant I/O calls.
  
     The arguments are:
     \begin{description}
     \item[string]
       The string to be converted
     \item[{[rc]}]
       Return code; equals {\tt ESMF\_SUCCESS} if there are no errors.
     \end{description}
   
%/////////////////////////////////////////////////////////////
 
\mbox{}\hrulefill\ 
 
\subsubsection [ESMF\_UtilStringInt2String] {ESMF\_UtilStringInt2String - convert integer to character string}


  
\bigskip{\sf INTERFACE:}
\begin{verbatim}     function ESMF_UtilStringInt2String (i, rc)\end{verbatim}{\em ARGUMENTS:}
\begin{verbatim}       integer, intent(in) :: i
 -- The following arguments require argument keyword syntax (e.g. rc=rc). --
       integer, intent(out), optional  :: rc\end{verbatim}{\em RETURN VALUE:}
\begin{verbatim}       character(int2str_len (i)) :: ESMF_UtilStringInt2String
 \end{verbatim}
{\sf DESCRIPTION:\\ }


     Converts given an integer to string representation.  The returned string is
     sized such that it does not contain leading or trailing blanks.
  
     This procedure may fail when used in an expression in a {\tt write} statement
     with some older, pre-Fortran 2003, compiler environments that do not support
     re-entrant I/O calls.
  
       The arguments are:
       \begin{description}
       \item[i]
         An integer.
       \item[{[rc]}]
         Return code; equals {\tt ESMF\_SUCCESS} if there are no errors.
       \end{description}
  
   
%/////////////////////////////////////////////////////////////
 
\mbox{}\hrulefill\ 
 
\subsubsection [ESMF\_UtilStringLowerCase] {ESMF\_UtilStringLowerCase - convert string to lowercase}


    
\bigskip{\sf INTERFACE:}
\begin{verbatim}     function ESMF_UtilStringLowerCase(string, rc) \end{verbatim}{\em ARGUMENTS:}
\begin{verbatim}       character(len=*), intent(in) :: string
 -- The following arguments require argument keyword syntax (e.g. rc=rc). --
       integer, intent(out), optional  :: rc  \end{verbatim}{\em RETURN VALUE:}
\begin{verbatim}       character(len (string)) :: ESMF_UtilStringLowerCase
 \end{verbatim}
{\sf DESCRIPTION:\\ }


     Converts given string to lowercase.
  
       The arguments are:
       \begin{description}
       \item[string]
         A character string.
       \item[{[rc]}]
         Return code; equals {\tt ESMF\_SUCCESS} if there are no errors.
       \end{description}
  
   
%/////////////////////////////////////////////////////////////
 
\mbox{}\hrulefill\ 
 
\subsubsection [ESMF\_UtilStringUpperCase] {ESMF\_UtilStringUpperCase - convert string to uppercase}


    
\bigskip{\sf INTERFACE:}
\begin{verbatim}       function ESMF_UtilStringUpperCase(string, rc) \end{verbatim}{\em ARGUMENTS:}
\begin{verbatim}       character(len=*), intent(in) :: string
 -- The following arguments require argument keyword syntax (e.g. rc=rc). --
       integer, intent(out), optional  :: rc  \end{verbatim}{\em RETURN VALUE:}
\begin{verbatim}       character(len (string)) :: ESMF_UtilStringUpperCase
 \end{verbatim}
{\sf DESCRIPTION:\\ }


     Converts given string to uppercase.
  
       The arguments are:
       \begin{description}
       \item[string]
         A character string.
       \item[{[rc]}]
         Return code; equals {\tt ESMF\_SUCCESS} if there are no errors.
       \end{description}
  
  
%...............................................................
\setlength{\parskip}{\oldparskip}
\setlength{\parindent}{\oldparindent}
\setlength{\baselineskip}{\oldbaselineskip}
