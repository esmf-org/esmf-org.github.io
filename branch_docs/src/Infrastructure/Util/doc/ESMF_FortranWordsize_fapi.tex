%                **** IMPORTANT NOTICE *****
% This LaTeX file has been automatically produced by ProTeX v. 1.1
% Any changes made to this file will likely be lost next time
% this file is regenerated from its source. Send questions 
% to Arlindo da Silva, dasilva@gsfc.nasa.gov
 
\setlength{\oldparskip}{\parskip}
\setlength{\parskip}{1.5ex}
\setlength{\oldparindent}{\parindent}
\setlength{\parindent}{0pt}
\setlength{\oldbaselineskip}{\baselineskip}
\setlength{\baselineskip}{11pt}
 
%--------------------- SHORT-HAND MACROS ----------------------
\def\bv{\begin{verbatim}}
\def\ev{\end{verbatim}}
\def\be{\begin{equation}}
\def\ee{\end{equation}}
\def\bea{\begin{eqnarray}}
\def\eea{\end{eqnarray}}
\def\bi{\begin{itemize}}
\def\ei{\end{itemize}}
\def\bn{\begin{enumerate}}
\def\en{\end{enumerate}}
\def\bd{\begin{description}}
\def\ed{\end{description}}
\def\({\left (}
\def\){\right )}
\def\[{\left [}
\def\]{\right ]}
\def\<{\left  \langle}
\def\>{\right \rangle}
\def\cI{{\cal I}}
\def\diag{\mathop{\rm diag}}
\def\tr{\mathop{\rm tr}}
%-------------------------------------------------------------

\markboth{Left}{Source File: ESMF\_FortranWordsize.F90,  Date: Tue May  5 20:59:26 MDT 2020
}

 
%/////////////////////////////////////////////////////////////
\subsubsection [ESMF\_FortranWordsize] {ESMF\_FortranWordsize - Return the size in byte units of a scalar }


   
\bigskip{\sf INTERFACE:}
\begin{verbatim}   ! Private name; call using ESMF_FortranWordsize() 
   function ESMF_FortranWordsize<typekind>(var, rc) 
   \end{verbatim}{\em RETURN VALUE:}
\begin{verbatim}   integer :: ESMF_FortranWordsize<typekind> 
   \end{verbatim}{\em ARGUMENTS:}
\begin{verbatim}   <type>(ESMF_KIND_<typekind>), intent(in) :: var 
   integer, intent(out), optional :: rc 
   \end{verbatim}
{\sf DESCRIPTION:\\ }

 
   Return the size in units of bytes of a scalar (var) argument. 
   Valid types and kinds supported by the framework are: 
   integers of 1-byte, 2-byte, 4-byte, and 8-byte size, and 
   reals of 4-byte and 8-bytes size. 
   
   The arguments are: 
   \begin{description} 
   \item [var] 
   Scalar of any supported type and kind 
   \item [rc] 
   Return code; equals {\tt ESMF\_SUCCESS} if there are no errors. 
   \end{description} 
   
%...............................................................
\setlength{\parskip}{\oldparskip}
\setlength{\parindent}{\oldparindent}
\setlength{\baselineskip}{\oldbaselineskip}
