%                **** IMPORTANT NOTICE *****
% This LaTeX file has been automatically produced by ProTeX v. 1.1
% Any changes made to this file will likely be lost next time
% this file is regenerated from its source. Send questions 
% to Arlindo da Silva, dasilva@gsfc.nasa.gov
 
\setlength{\oldparskip}{\parskip}
\setlength{\parskip}{1.5ex}
\setlength{\oldparindent}{\parindent}
\setlength{\parindent}{0pt}
\setlength{\oldbaselineskip}{\baselineskip}
\setlength{\baselineskip}{11pt}
 
%--------------------- SHORT-HAND MACROS ----------------------
\def\bv{\begin{verbatim}}
\def\ev{\end{verbatim}}
\def\be{\begin{equation}}
\def\ee{\end{equation}}
\def\bea{\begin{eqnarray}}
\def\eea{\end{eqnarray}}
\def\bi{\begin{itemize}}
\def\ei{\end{itemize}}
\def\bn{\begin{enumerate}}
\def\en{\end{enumerate}}
\def\bd{\begin{description}}
\def\ed{\end{description}}
\def\({\left (}
\def\){\right )}
\def\[{\left [}
\def\]{\right ]}
\def\<{\left  \langle}
\def\>{\right \rangle}
\def\cI{{\cal I}}
\def\diag{\mathop{\rm diag}}
\def\tr{\mathop{\rm tr}}
%-------------------------------------------------------------

\markboth{Left}{Source File: ESMCI\_Fraction.C,  Date: Tue May  5 20:59:25 MDT 2020
}

 
%/////////////////////////////////////////////////////////////
\subsubsection [Fraction::setw] {Fraction::setw - Set fraction's whole number}


  
\bigskip{\sf INTERFACE:}
\begin{verbatim}       int Fraction::setw(\end{verbatim}{\em RETURN VALUE:}
\begin{verbatim}      int error return code\end{verbatim}{\em ARGUMENTS:}
\begin{verbatim}       ESMC_I8 w) {   // input - the whole number value to set\end{verbatim}
{\sf DESCRIPTION:\\ }


       Sets the fraction's whole number value.
   
%/////////////////////////////////////////////////////////////
 
\mbox{}\hrulefill\ 
 
\subsubsection [Fraction::setn] {Fraction::setn - Set fraction's numerator}


  
\bigskip{\sf INTERFACE:}
\begin{verbatim}       int Fraction::setn(\end{verbatim}{\em RETURN VALUE:}
\begin{verbatim}      int error return code\end{verbatim}{\em ARGUMENTS:}
\begin{verbatim}       ESMC_I8 n) {   // input - the numerator value to set\end{verbatim}
{\sf DESCRIPTION:\\ }


       Sets the fraction's numerator value.
   
%/////////////////////////////////////////////////////////////
 
\mbox{}\hrulefill\ 
 
\subsubsection [Fraction::setd] {Fraction::setd - Set fraction's denominator}


  
\bigskip{\sf INTERFACE:}
\begin{verbatim}       int Fraction::setd(\end{verbatim}{\em RETURN VALUE:}
\begin{verbatim}      int error return code\end{verbatim}{\em ARGUMENTS:}
\begin{verbatim}       ESMC_I8 d) {   // input - the denominator value to set\end{verbatim}
{\sf DESCRIPTION:\\ }


       Sets the fraction's denominator value.
   
%/////////////////////////////////////////////////////////////
 
\mbox{}\hrulefill\ 
 
\subsubsection [Fraction::setr] {Fraction::setr - Set from given real number}


  
\bigskip{\sf INTERFACE:}
\begin{verbatim}       int Fraction::setr(\end{verbatim}{\em RETURN VALUE:}
\begin{verbatim}      none.\end{verbatim}{\em ARGUMENTS:}
\begin{verbatim}       ESMC_R8 rin) {  // input double precision real number\end{verbatim}
{\sf DESCRIPTION:\\ }


       Convert any real number (within limits) to a rational fraction via
       the method of continued fractions (CF), using a standard variation of
       Euclid's GCD algorithm.
   
%/////////////////////////////////////////////////////////////
 
\mbox{}\hrulefill\ 
 
\subsubsection [Fraction::getw] {Fraction::getw - Get fraction's whole number}


  
\bigskip{\sf INTERFACE:}
\begin{verbatim}       ESMC_I8 Fraction::getw(void) const {\end{verbatim}{\em RETURN VALUE:}
\begin{verbatim}      The fraction's whole number value\end{verbatim}{\em ARGUMENTS:}
\begin{verbatim}      none.\end{verbatim}
{\sf DESCRIPTION:\\ }


       Gets the fraction's whole number value.
   
%/////////////////////////////////////////////////////////////
 
\mbox{}\hrulefill\ 
 
\subsubsection [Fraction::getn] {Fraction::getn - Get fraction's numerator}


  
\bigskip{\sf INTERFACE:}
\begin{verbatim}       ESMC_I8 Fraction::getn(void) const {\end{verbatim}{\em RETURN VALUE:}
\begin{verbatim}      The fraction's numerator value.\end{verbatim}{\em ARGUMENTS:}
\begin{verbatim}      none.\end{verbatim}
{\sf DESCRIPTION:\\ }


       Gets the fraction's numerator value.
   
%/////////////////////////////////////////////////////////////
 
\mbox{}\hrulefill\ 
 
\subsubsection [Fraction::getd] {Fraction::getd - Get fraction's denominator}


  
\bigskip{\sf INTERFACE:}
\begin{verbatim}       ESMC_I8 Fraction::getd(void) const {\end{verbatim}{\em RETURN VALUE:}
\begin{verbatim}      The fraction's denominator value.\end{verbatim}{\em ARGUMENTS:}
\begin{verbatim}      none.\end{verbatim}
{\sf DESCRIPTION:\\ }


       Gets the fraction's denominator value.
   
%/////////////////////////////////////////////////////////////
 
\mbox{}\hrulefill\ 
 
\subsubsection [Fraction::getr] {Fraction::getr - Get fraction's value as a real number}


  
\bigskip{\sf INTERFACE:}
\begin{verbatim}       ESMC_R8 Fraction::getr(void) const {\end{verbatim}{\em RETURN VALUE:}
\begin{verbatim}      The fraction's value as a real number.\end{verbatim}{\em ARGUMENTS:}
\begin{verbatim}      none.\end{verbatim}
{\sf DESCRIPTION:\\ }


       Gets the fraction's value as a real number.
   
%/////////////////////////////////////////////////////////////
 
\mbox{}\hrulefill\ 
 
\subsubsection [Fraction::set] {Fraction::set - Set fraction value}


  
\bigskip{\sf INTERFACE:}
\begin{verbatim}       int Fraction::set(\end{verbatim}{\em RETURN VALUE:}
\begin{verbatim}      none.\end{verbatim}{\em ARGUMENTS:}
\begin{verbatim}       ESMC_I8 *w,
       ESMC_I8 *n,
       ESMC_I8 *d) {\end{verbatim}
{\sf DESCRIPTION:\\ }


       Sets the fraction's value.  Supports F90 optional args interface
   
%/////////////////////////////////////////////////////////////
 
\mbox{}\hrulefill\ 
 
\subsubsection [Fraction::set] {Fraction::set - Set fraction value}


  
\bigskip{\sf INTERFACE:}
\begin{verbatim}       int Fraction::set(\end{verbatim}{\em RETURN VALUE:}
\begin{verbatim}      none.\end{verbatim}{\em ARGUMENTS:}
\begin{verbatim}       ESMC_I8 w,
       ESMC_I8 n,
       ESMC_I8 d) {\end{verbatim}
{\sf DESCRIPTION:\\ }


       Sets the fraction's value.
   
%/////////////////////////////////////////////////////////////
 
\mbox{}\hrulefill\ 
 
\subsubsection [Fraction::get] {Fraction::get - Get fraction value}


  
\bigskip{\sf INTERFACE:}
\begin{verbatim}       int Fraction::get(\end{verbatim}{\em RETURN VALUE:}
\begin{verbatim}      none.\end{verbatim}{\em ARGUMENTS:}
\begin{verbatim}       ESMC_I8 *w,
       ESMC_I8 *n,
       ESMC_I8 *d) const {\end{verbatim}
{\sf DESCRIPTION:\\ }


       Gets the fraction's value.
   
%/////////////////////////////////////////////////////////////
 
\mbox{}\hrulefill\ 
 
\subsubsection [Fraction::simplify] {Fraction::simplify - Ensure proper fraction (< 1) and sign; reduce to lowest denominator}


  
\bigskip{\sf INTERFACE:}
\begin{verbatim}       int Fraction::simplify(void) {\end{verbatim}{\em RETURN VALUE:}
\begin{verbatim}      none.\end{verbatim}{\em ARGUMENTS:}
\begin{verbatim}      none.\end{verbatim}
{\sf DESCRIPTION:\\ }


       If fraction >= 1, add to whole part, and adjust fraction to remainder.
       Then reduce to lowest denominator.
   
%/////////////////////////////////////////////////////////////
 
\mbox{}\hrulefill\ 
 
\subsubsection [Fraction::convert] {Fraction::convert - Convert to given denominator}


  
\bigskip{\sf INTERFACE:}
\begin{verbatim}       int Fraction::convert(\end{verbatim}{\em RETURN VALUE:}
\begin{verbatim}      none.\end{verbatim}{\em ARGUMENTS:}
\begin{verbatim}       ESMC_I8 denominator) {  // input\end{verbatim}
{\sf DESCRIPTION:\\ }


       Convert fraction in terms of given denominator
   
%/////////////////////////////////////////////////////////////
 
\mbox{}\hrulefill\ 
 
\subsubsection [ESMCI\_FractionGCD] {ESMCI\_FractionGCD - determine the Greatest Common Divisor}


  
\bigskip{\sf INTERFACE:}
\begin{verbatim}       ESMC_I8 ESMCI_FractionGCD(\end{verbatim}{\em RETURN VALUE:}
\begin{verbatim}      The GCD of a and b.\end{verbatim}{\em ARGUMENTS:}
\begin{verbatim}       ESMC_I8 a,    // in - the first number 
       ESMC_I8 b) {  // in - the second number\end{verbatim}
{\sf DESCRIPTION:\\ }


       Uses Euclid's algorithm to determine the Greatest Common Divisor of 
       a and b.
   
%/////////////////////////////////////////////////////////////
 
\mbox{}\hrulefill\ 
 
\subsubsection [ESMCI\_FractionLCM] {ESMCI\_FractionLCM - determine the Least Common Multiple}


  
\bigskip{\sf INTERFACE:}
\begin{verbatim}       ESMC_I8 ESMCI_FractionLCM(\end{verbatim}{\em RETURN VALUE:}
\begin{verbatim}      the LCM of a and b\end{verbatim}{\em ARGUMENTS:}
\begin{verbatim}       ESMC_I8 a,    // in - the first number 
       ESMC_I8 b) {  // in - the second number\end{verbatim}
{\sf DESCRIPTION:\\ }


        Uses GCD to determine the Least Common Multiple of a and b.
        LCM = (a * b) / GCD(a,b)
   
%/////////////////////////////////////////////////////////////
 
\mbox{}\hrulefill\ 
 
\subsubsection [Fraction(==)] {Fraction(==) - Fraction equality comparison}


  
\bigskip{\sf INTERFACE:}
\begin{verbatim}       bool Fraction::operator==(\end{verbatim}{\em RETURN VALUE:}
\begin{verbatim}      bool result\end{verbatim}{\em ARGUMENTS:}
\begin{verbatim}       const Fraction &fraction) const {   // in - Fraction to compare\end{verbatim}
{\sf DESCRIPTION:\\ }


        Compare for equality the current object's (this) {\tt Fraction}
        with given {\tt Fraction}, return result
   
%/////////////////////////////////////////////////////////////
 
\mbox{}\hrulefill\ 
 
\subsubsection [Fraction(!=)] {Fraction(!=) - Fraction inequality comparison}


  
\bigskip{\sf INTERFACE:}
\begin{verbatim}       bool Fraction::operator!=(\end{verbatim}{\em RETURN VALUE:}
\begin{verbatim}      bool result\end{verbatim}{\em ARGUMENTS:}
\begin{verbatim}       const Fraction &fraction) const {   // in - Fraction to compare\end{verbatim}
{\sf DESCRIPTION:\\ }


        Compare for inequality the current object's (this)
        {\tt Fraction} with given {\tt Fraction}, return result
   
%/////////////////////////////////////////////////////////////
 
\mbox{}\hrulefill\ 
 
\subsubsection [Fraction(<)] {Fraction(<) - Fraction less than comparison}


  
\bigskip{\sf INTERFACE:}
\begin{verbatim}       bool Fraction::operator<(\end{verbatim}{\em RETURN VALUE:}
\begin{verbatim}      bool result\end{verbatim}{\em ARGUMENTS:}
\begin{verbatim}       const Fraction &fraction) const {   // in - Fraction to compare\end{verbatim}
{\sf DESCRIPTION:\\ }


        Compare for less than the current object's (this)
        {\tt Fraction} with given {\tt Fraction}, return result
   
%/////////////////////////////////////////////////////////////
 
\mbox{}\hrulefill\ 
 
\subsubsection [Fraction(>)] {Fraction(>) - Fraction greater than comparison}


  
\bigskip{\sf INTERFACE:}
\begin{verbatim}       bool Fraction::operator>(\end{verbatim}{\em RETURN VALUE:}
\begin{verbatim}      bool result\end{verbatim}{\em ARGUMENTS:}
\begin{verbatim}       const Fraction &fraction) const {   // in - Fraction to compare\end{verbatim}
{\sf DESCRIPTION:\\ }


        Compare for greater than the current object's (this)
        {\tt Fraction} with given {\tt Fraction}, return result
   
%/////////////////////////////////////////////////////////////
 
\mbox{}\hrulefill\ 
 
\subsubsection [Fraction(<=)] {Fraction(<=) - Fraction less or equal than comparison}


  
\bigskip{\sf INTERFACE:}
\begin{verbatim}       bool Fraction::operator<=(\end{verbatim}{\em RETURN VALUE:}
\begin{verbatim}      bool result\end{verbatim}{\em ARGUMENTS:}
\begin{verbatim}       const Fraction &fraction) const {   // in - Fraction to compare\end{verbatim}
{\sf DESCRIPTION:\\ }


        Compare for less than or equal the current object's (this)
        {\tt Fraction} with given {\tt Fraction}, return result
   
%/////////////////////////////////////////////////////////////
 
\mbox{}\hrulefill\ 
 
\subsubsection [Fraction(>=)] {Fraction(>=) - Fraction greater than or equal comparison}


  
\bigskip{\sf INTERFACE:}
\begin{verbatim}       bool Fraction::operator>=(\end{verbatim}{\em RETURN VALUE:}
\begin{verbatim}      bool result\end{verbatim}{\em ARGUMENTS:}
\begin{verbatim}       const Fraction &fraction) const {   // in - Fraction to compare\end{verbatim}
{\sf DESCRIPTION:\\ }


        Compare for greater than or equal the current object's (this)
        {\tt Fraction} with given {\tt Fraction}, return result
   
%/////////////////////////////////////////////////////////////
 
\mbox{}\hrulefill\ 
 
\subsubsection [Fraction(+)] {Fraction(+) - increment Fraction}


  
\bigskip{\sf INTERFACE:}
\begin{verbatim}       Fraction Fraction::operator+(\end{verbatim}{\em RETURN VALUE:}
\begin{verbatim}      Fraction result\end{verbatim}{\em ARGUMENTS:}
\begin{verbatim}       const Fraction &fraction) const {   // in - Fraction increment\end{verbatim}
{\sf DESCRIPTION:\\ }


        Increment current object's (this) {\tt Fraction} with given
        {\tt Fraction}, return result
   
%/////////////////////////////////////////////////////////////
 
\mbox{}\hrulefill\ 
 
\subsubsection [Fraction(-)] {Fraction(-) - decrement Fraction}


  
\bigskip{\sf INTERFACE:}
\begin{verbatim}       Fraction Fraction::operator-(\end{verbatim}{\em RETURN VALUE:}
\begin{verbatim}      Fraction result\end{verbatim}{\em ARGUMENTS:}
\begin{verbatim}       const Fraction &fraction) const {   // in - Fraction decrement\end{verbatim}
{\sf DESCRIPTION:\\ }


        Decrement current object's (this) {\tt Fraction} with given
        {\tt Fraction}, return result
   
%/////////////////////////////////////////////////////////////
 
\mbox{}\hrulefill\ 
 
\subsubsection [Fraction(+=)] {Fraction(+=) - increment Fraction}


  
\bigskip{\sf INTERFACE:}
\begin{verbatim}       Fraction& Fraction::operator+=(\end{verbatim}{\em RETURN VALUE:}
\begin{verbatim}      Fraction& result\end{verbatim}{\em ARGUMENTS:}
\begin{verbatim}       const Fraction &fraction) {   // in - Fraction increment\end{verbatim}
{\sf DESCRIPTION:\\ }


        Increment current object's (this) {\tt Fraction} with given
        {\tt Fraction} 
%/////////////////////////////////////////////////////////////
 
\mbox{}\hrulefill\ 
 
\subsubsection [Fraction(-=)] {Fraction(-=) - decrement Fraction}


  
\bigskip{\sf INTERFACE:}
\begin{verbatim}       Fraction& Fraction::operator-=(\end{verbatim}{\em RETURN VALUE:}
\begin{verbatim}      Fraction& result\end{verbatim}{\em ARGUMENTS:}
\begin{verbatim}       const Fraction &fraction) {   // in - Fraction decrement\end{verbatim}
{\sf DESCRIPTION:\\ }


        Decrement current object's (this) {\tt Fraction} with given
        {\tt Fraction}
   
%/////////////////////////////////////////////////////////////
 
\mbox{}\hrulefill\ 
 
\subsubsection [Fraction(*)] {Fraction(*) - multiply Fraction by integer}


  
\bigskip{\sf INTERFACE:}
\begin{verbatim}       Fraction Fraction::operator*(\end{verbatim}{\em RETURN VALUE:}
\begin{verbatim}      Fraction result\end{verbatim}{\em ARGUMENTS:}
\begin{verbatim}       ESMC_I4 multiplier) const {   // in - integer multiplier\end{verbatim}
{\sf DESCRIPTION:\\ }


        Multiply current object's (this) {\tt Fraction} with given
        integer, return result
   
%/////////////////////////////////////////////////////////////
 
\mbox{}\hrulefill\ 
 
\subsubsection [Fraction(*=)] {Fraction(*=) - multiply Fraction by integer}


  
\bigskip{\sf INTERFACE:}
\begin{verbatim}       Fraction& Fraction::operator*=(\end{verbatim}{\em RETURN VALUE:}
\begin{verbatim}      Fraction& result\end{verbatim}{\em ARGUMENTS:}
\begin{verbatim}       ESMC_I4 multiplier) {   // in - integer multiplier\end{verbatim}
{\sf DESCRIPTION:\\ }


        Multiply current object's (this) {\tt Fraction} with given
        integer. 
%/////////////////////////////////////////////////////////////
 
\mbox{}\hrulefill\ 
 
\subsubsection [Fraction(/)] {Fraction(/) - divide Fraction by integer}


  
\bigskip{\sf INTERFACE:}
\begin{verbatim}       Fraction Fraction::operator/(\end{verbatim}{\em RETURN VALUE:}
\begin{verbatim}      Fraction result\end{verbatim}{\em ARGUMENTS:}
\begin{verbatim}       ESMC_I4 divisor) const {   // in - integer divisor\end{verbatim}
{\sf DESCRIPTION:\\ }


        Divide current object's (this) {\tt Fraction} with given
        integer, return result
   
%/////////////////////////////////////////////////////////////
 
\mbox{}\hrulefill\ 
 
\subsubsection [Fraction(/=)] {Fraction(/=) - divide Fraction by integer}


  
\bigskip{\sf INTERFACE:}
\begin{verbatim}       Fraction& Fraction::operator/=(\end{verbatim}{\em RETURN VALUE:}
\begin{verbatim}      Fraction& result\end{verbatim}{\em ARGUMENTS:}
\begin{verbatim}       ESMC_I4 divisor) {   // in - integer divisor\end{verbatim}
{\sf DESCRIPTION:\\ }


        Divide current object's (this) {\tt Fraction} with given
        integer. 
%/////////////////////////////////////////////////////////////
 
\mbox{}\hrulefill\ 
 
\subsubsection [Fraction(/)] {Fraction(/) - Divide two fractions, return double precision result}


  
\bigskip{\sf INTERFACE:}
\begin{verbatim}       ESMC_R8 Fraction::operator/(\end{verbatim}{\em RETURN VALUE:}
\begin{verbatim}      ESMC_R8 result\end{verbatim}{\em ARGUMENTS:}
\begin{verbatim}       const Fraction &fraction) const {  // in - Fraction to divide by\end{verbatim}
{\sf DESCRIPTION:\\ }


      Returns this fraction divided by given fraction as a ESMC_R8
      precision quotient.
   
%/////////////////////////////////////////////////////////////
 
\mbox{}\hrulefill\ 
 
\subsubsection [Fraction(\%)] {Fraction(\%) - Computes the modulus of two fractions}


  
\bigskip{\sf INTERFACE:}
\begin{verbatim}       Fraction Fraction::operator%(\end{verbatim}{\em RETURN VALUE:}
\begin{verbatim} \end{verbatim}{\em ARGUMENTS:}
\begin{verbatim}       const Fraction &fraction) const {  // in - Fraction to modulo by\end{verbatim}
{\sf DESCRIPTION:\\ }


      Returns this fraction modulo by given fraction
   
%/////////////////////////////////////////////////////////////
 
\mbox{}\hrulefill\ 
 
\subsubsection [Fraction(\%=)] {Fraction(\%=) - Computes the modulus of two fractions}


  
\bigskip{\sf INTERFACE:}
\begin{verbatim}       Fraction& Fraction::operator%=(\end{verbatim}{\em RETURN VALUE:}
\begin{verbatim} \end{verbatim}{\em ARGUMENTS:}
\begin{verbatim}       const Fraction &fraction) {  // in - Fraction to modulo by\end{verbatim}
{\sf DESCRIPTION:\\ }


      Returns this fraction modulo by given fraction
   
%/////////////////////////////////////////////////////////////
 
\mbox{}\hrulefill\ 
 
\subsubsection [Fraction(=)] {Fraction(=) - assignment operator}


  
\bigskip{\sf INTERFACE:}
\begin{verbatim}       Fraction& Fraction::operator=(\end{verbatim}{\em RETURN VALUE:}
\begin{verbatim}      Fraction& result\end{verbatim}{\em ARGUMENTS:}
\begin{verbatim}       const Fraction &fraction) {   // in - Fraction\end{verbatim}
{\sf DESCRIPTION:\\ }


        Assign current object's (this) {\tt Fraction} with given
        {\tt Fraction}.   
%/////////////////////////////////////////////////////////////
 
\mbox{}\hrulefill\ 
 
\subsubsection [Fraction::validate] {Fraction::validate - validate Fraction state}


  
\bigskip{\sf INTERFACE:}
\begin{verbatim}       int Fraction::validate(\end{verbatim}{\em RETURN VALUE:}
\begin{verbatim}      int error return code\end{verbatim}{\em ARGUMENTS:}
\begin{verbatim}       const char *options) const {     // in - options\end{verbatim}
{\sf DESCRIPTION:\\ }


        validate {\tt Fraction} state
   
%/////////////////////////////////////////////////////////////
 
\mbox{}\hrulefill\ 
 
\subsubsection [Fraction::print] {Fraction::print - print Fraction state}


  
\bigskip{\sf INTERFACE:}
\begin{verbatim}       int Fraction::print(\end{verbatim}{\em RETURN VALUE:}
\begin{verbatim}      int error return code\end{verbatim}{\em ARGUMENTS:}
\begin{verbatim}       const char *options) const {    // in - print options\end{verbatim}
{\sf DESCRIPTION:\\ }


        print {\tt Fraction} state for testing/debugging
   
%/////////////////////////////////////////////////////////////
 
\mbox{}\hrulefill\ 
 
\subsubsection [Fraction] {Fraction - default C++ constructor}


  
\bigskip{\sf INTERFACE:}
\begin{verbatim}       Fraction::Fraction(void) :\end{verbatim}{\em RETURN VALUE:}
\begin{verbatim}      none\end{verbatim}{\em ARGUMENTS:}
\begin{verbatim}      none\end{verbatim}
{\sf DESCRIPTION:\\ }


        Initializes a {\tt Fraction} with defaults
   
%/////////////////////////////////////////////////////////////
 
\mbox{}\hrulefill\ 
 
\subsubsection [Fraction] {Fraction - copy C++ constructor}


  
\bigskip{\sf INTERFACE:}
\begin{verbatim}       Fraction::Fraction(\end{verbatim}{\em RETURN VALUE:}
\begin{verbatim}      none\end{verbatim}{\em ARGUMENTS:}
\begin{verbatim}       const Fraction &fraction) :   // in - Fraction\end{verbatim}
{\sf DESCRIPTION:\\ }


        Initializes a {\tt Fraction} with defaults
   
%/////////////////////////////////////////////////////////////
 
\mbox{}\hrulefill\ 
 
\subsubsection [Fraction] {Fraction - C++ constructor}


  
\bigskip{\sf INTERFACE:}
\begin{verbatim}       Fraction::Fraction(\end{verbatim}{\em RETURN VALUE:}
\begin{verbatim}      none\end{verbatim}{\em ARGUMENTS:}
\begin{verbatim}       ESMC_I8 w_in,   // Integer (whole) seconds (signed)
       ESMC_I8 n_in,   // Integer fraction (exact) n/d; numerator (signed)
       ESMC_I8 d_in) : // Integer fraction (exact) n/d; denominator
 \end{verbatim}
{\sf DESCRIPTION:\\ }


        Initializes a {\tt Fraction} with given values
   
%/////////////////////////////////////////////////////////////
 
\mbox{}\hrulefill\ 
 
\subsubsection [Fraction] {Fraction - C++ constructor}


  
\bigskip{\sf INTERFACE:}
\begin{verbatim}       Fraction::Fraction(\end{verbatim}{\em RETURN VALUE:}
\begin{verbatim}      none\end{verbatim}{\em ARGUMENTS:}
\begin{verbatim}       ESMC_R8 r) {  // fraction as a real (signed)
 \end{verbatim}
{\sf DESCRIPTION:\\ }


        Initializes a {\tt Fraction} with given real number value.
   
%/////////////////////////////////////////////////////////////
 
\mbox{}\hrulefill\ 
 
\subsubsection [~Fraction] {~Fraction - default C++ destructor}


  
\bigskip{\sf INTERFACE:}
\begin{verbatim}       Fraction::~Fraction(void) {\end{verbatim}{\em RETURN VALUE:}
\begin{verbatim}      none\end{verbatim}{\em ARGUMENTS:}
\begin{verbatim}      none\end{verbatim}
{\sf DESCRIPTION:\\ }


        Default {\tt Fraction} destructor
  
%...............................................................
\setlength{\parskip}{\oldparskip}
\setlength{\parindent}{\oldparindent}
\setlength{\baselineskip}{\oldbaselineskip}
