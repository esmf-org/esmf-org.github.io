%                **** IMPORTANT NOTICE *****
% This LaTeX file has been automatically produced by ProTeX v. 1.1
% Any changes made to this file will likely be lost next time
% this file is regenerated from its source. Send questions 
% to Arlindo da Silva, dasilva@gsfc.nasa.gov
 
\setlength{\oldparskip}{\parskip}
\setlength{\parskip}{1.5ex}
\setlength{\oldparindent}{\parindent}
\setlength{\parindent}{0pt}
\setlength{\oldbaselineskip}{\baselineskip}
\setlength{\baselineskip}{11pt}
 
%--------------------- SHORT-HAND MACROS ----------------------
\def\bv{\begin{verbatim}}
\def\ev{\end{verbatim}}
\def\be{\begin{equation}}
\def\ee{\end{equation}}
\def\bea{\begin{eqnarray}}
\def\eea{\end{eqnarray}}
\def\bi{\begin{itemize}}
\def\ei{\end{itemize}}
\def\bn{\begin{enumerate}}
\def\en{\end{enumerate}}
\def\bd{\begin{description}}
\def\ed{\end{description}}
\def\({\left (}
\def\){\right )}
\def\[{\left [}
\def\]{\right ]}
\def\<{\left  \langle}
\def\>{\right \rangle}
\def\cI{{\cal I}}
\def\diag{\mathop{\rm diag}}
\def\tr{\mathop{\rm tr}}
%-------------------------------------------------------------

\markboth{Left}{Source File: ESMF\_Calendar.F90,  Date: Tue May  5 20:59:34 MDT 2020
}

 
%/////////////////////////////////////////////////////////////
\subsubsection [ESMF\_CalendarAssignment(=)] {ESMF\_CalendarAssignment(=) - Assign a Calendar to another Calendar}


  
\bigskip{\sf INTERFACE:}
\begin{verbatim}       interface assignment(=)
       calendar1 = calendar2\end{verbatim}{\em ARGUMENTS:}
\begin{verbatim}       type(ESMF_Calendar) :: calendar1
       type(ESMF_Calendar) :: calendar2
   \end{verbatim}
{\sf STATUS:}
   \begin{itemize}
   \item\apiStatusCompatibleVersion{5.2.0r}
   \end{itemize}
  
{\sf DESCRIPTION:\\ }


       Assign {\tt calendar1} as an alias to the same {\tt ESMF\_Calendar} 
       object in memory as {\tt calendar2}. If {\tt calendar2} is invalid, then 
       {\tt calendar1} will be equally invalid after the assignment.
  
       The arguments are:
       \begin{description} 
       \item[calendar1] 
            The {\tt ESMF\_Calendar} object on the left hand side of the 
            assignment.
       \item[calendar2] 
            The {\tt ESMF\_Calendar} object on the right hand side of the 
            assignment.
       \end{description}
   
%/////////////////////////////////////////////////////////////
 
\mbox{}\hrulefill\ 
 
\subsubsection [ESMF\_CalendarOperator(==)] {ESMF\_CalendarOperator(==) - Test if Calendar argument 1 is equal to Calendar argument 2}


  
\bigskip{\sf INTERFACE:}
\begin{verbatim}       interface operator(==)
       if (<calendar argument 1> == <calendar argument 2>) then ... endif
                                   OR
       result = (<calendar argument 1> == <calendar argument 2>)\end{verbatim}{\em RETURN VALUE:}
\begin{verbatim}       logical :: result\end{verbatim}{\em ARGUMENTS:}
\begin{verbatim}       <calendar argument 1>, see below for supported values
       <calendar argument 2>, see below for supported values\end{verbatim}
{\sf DESCRIPTION:\\ }


       \begin{sloppypar}
       Overloads the (==) operator for the {\tt ESMF\_Calendar} class.
       Compare an {\tt ESMF\_Calendar} object or {\tt ESMF\_CalKind\_Flag} with
       another calendar object or calendar kind for equality.  Return
       {\tt .true.} if equal, {\tt .false.} otherwise.  Comparison is based on
       calendar kind, which is a property of a calendar object.
       \end{sloppypar}
  
       If both arguments are {\tt ESMF\_Calendar} objects, and both are of  
       type {\tt ESMF\_CALKIND\_CUSTOM}, then all the calendar's properties, 
       except name, are compared.
  
       If both arguments are {\tt ESMF\_Calendar} objects, and either of them
       is not in the {\tt ESMF\_INIT\_CREATED} status, an error will be logged.
       However, this does not affect the return value, which is {\tt .true.} 
       when both arguments are in the {\em same} status, and {\tt .false.}
       otherwise.
  
       If one argument is an {\tt ESMF\_Calendar} object, and the other is an
       {\tt ESMF\_CalKind\_Flag}, and the calendar object is not in the
       {\tt ESMF\_INIT\_CREATED} status, an error will be logged and
       {\tt .false.} will be returned.
  
       Supported values for <calendar argument 1> are:
       \begin{description}
       \item type(ESMF\_Calendar),     intent(in) :: calendar1
       \item type(ESMF\_CalKind\_Flag), intent(in) :: calkindflag1
       \end{description}
       Supported values for <calendar argument 2> are:
       \begin{description}
       \item type(ESMF\_Calendar),     intent(in) :: calendar2
       \item type(ESMF\_CalKind\_Flag), intent(in) :: calkindflag2
       \end{description}
  
       The arguments are:
       \begin{description}   
       \item[<calendar argument 1>]
            The {\tt ESMF\_Calendar} object or {\tt ESMF\_CalKind\_Flag} on the
            left hand side of the equality operation.
       \item[<calendar argument 2>]
            The {\tt ESMF\_Calendar} object or {\tt ESMF\_CalKind\_Flag} on the
            right hand side of the equality operation.
       \end{description}
   
%/////////////////////////////////////////////////////////////
 
\mbox{}\hrulefill\ 
 
\subsubsection [ESMF\_CalendarOperator(/=)] {ESMF\_CalendarOperator(/=) - Test if Calendar argument 1 is not equal to Calendar argument 2}


  
\bigskip{\sf INTERFACE:}
\begin{verbatim}       interface operator(/=)
       if (<calendar argument 1> /= <calendar argument 2>) then ... endif
                                   OR
       result = (<calendar argument 1> /= <calendar argument 2>)\end{verbatim}{\em RETURN VALUE:}
\begin{verbatim}       logical :: result\end{verbatim}{\em ARGUMENTS:}
\begin{verbatim}       <calendar argument 1>, see below for supported values
       <calendar argument 2>, see below for supported values\end{verbatim}
{\sf DESCRIPTION:\\ }


       \begin{sloppypar}
       Overloads the (/=) operator for the {\tt ESMF\_Calendar} class.
       Compare a {\tt ESMF\_Calendar} object or {\tt ESMF\_CalKind\_Flag} with
       another calendar object or calendar kind for inequality.  Return
       {\tt .true.} if not equal, {\tt .false.} otherwise.  Comparison is based
       on calendar kind, which is a property of a calendar object.
       \end{sloppypar}
  
       If both arguments are {\tt ESMF\_Calendar} objects, and both are of  
       type {\tt ESMF\_CALKIND\_CUSTOM}, then all the calendar's properties,
       except name, are compared.
  
       If both arguments are {\tt ESMF\_Calendar} objects, and either of them
       is not in the {\tt ESMF\_INIT\_CREATED} status, an error will be logged.
       However, this does not affect the return value, which is {\tt .true.} 
       when both arguments are {\em not} in the {\em same} status, and
       {\tt .false.} otherwise.
  
       If one argument is an {\tt ESMF\_Calendar} object, and the other is an
       {\tt ESMF\_CalKind\_Flag}, and the calendar object is not in the
       {\tt ESMF\_INIT\_CREATED} status, an error will be logged and
       {\tt .true.} will be returned.
  
       Supported values for <calendar argument 1> are:
       \begin{description}
       \item type(ESMF\_Calendar),     intent(in) :: calendar1
       \item type(ESMF\_CalKind\_Flag), intent(in) :: calkindflag1
       \end{description}
       Supported values for <calendar argument 2> are:
       \begin{description}
       \item type(ESMF\_Calendar),     intent(in) :: calendar2
       \item type(ESMF\_CalKind\_Flag), intent(in) :: calkindflag2
       \end{description}
  
       The arguments are:
       \begin{description}   
       \item[<calendar argument 1>]
            The {\tt ESMF\_Calendar} object or {\tt ESMF\_CalKind\_Flag} on the
            left hand side of the non-equality operation.
       \item[<calendar argument 2>]
            The {\tt ESMF\_Calendar} object or {\tt ESMF\_CalKind\_Flag} on the
            right hand side of the non-equality operation.
       \end{description}
   
%/////////////////////////////////////////////////////////////
 
\mbox{}\hrulefill\ 
 
\subsubsection [ESMF\_CalendarCreate] {ESMF\_CalendarCreate - Create a new ESMF Calendar of built-in type}


 
\bigskip{\sf INTERFACE:}
\begin{verbatim}       ! Private name; call using ESMF_CalendarCreate()
       function ESMF_CalendarCreateBuiltIn(calkindflag, &
         name, rc)
 \end{verbatim}{\em RETURN VALUE:}
\begin{verbatim}       type(ESMF_Calendar) :: ESMF_CalendarCreateBuiltIn
 \end{verbatim}{\em ARGUMENTS:}
\begin{verbatim}       type(ESMF_CalKind_Flag), intent(in)            :: calkindflag
 -- The following arguments require argument keyword syntax (e.g. rc=rc). --
       character (len=*),       intent(in),  optional :: name
       integer,                 intent(out), optional :: rc
 \end{verbatim}
{\sf STATUS:}
   \begin{itemize}
   \item\apiStatusCompatibleVersion{5.2.0r}
   \end{itemize}
  
{\sf DESCRIPTION:\\ }


       Creates and sets a {\tt calendar} to the given built-in
       {\tt ESMF\_CalKind\_Flag}. 
  
       The arguments are:
       \begin{description}
       \item[calkindflag]
            The built-in {\tt ESMF\_CalKind\_Flag}.  Valid values are:
              \newline
              {\tt ESMF\_CALKIND\_360DAY}, 
              \newline
              {\tt ESMF\_CALKIND\_GREGORIAN},
              \newline
              {\tt ESMF\_CALKIND\_JULIAN}, 
              \newline
              {\tt ESMF\_CALKIND\_JULIANDAY},
              \newline
              {\tt ESMF\_CALKIND\_MODJULIANDAY}, 
              \newline
              {\tt ESMF\_CALKIND\_NOCALENDAR},
              \newline
              and {\tt ESMF\_CALKIND\_NOLEAP}.
              \newline
            See Section ~\ref{subsec:Calendar_options} for a description of each
            calendar kind.
       \item[{[name]}]
            The name for the newly created calendar.  If not specified, a
            default unique name will be generated: "CalendarNNN" where NNN
            is a unique sequence number from 001 to 999.
       \item[{[rc]}]
            Return code; equals {\tt ESMF\_SUCCESS} if there are no errors.
       \end{description}
       
%/////////////////////////////////////////////////////////////
 
\mbox{}\hrulefill\ 
 
\subsubsection [ESMF\_CalendarCreate] {ESMF\_CalendarCreate - Create a copy of an ESMF Calendar}


 
\bigskip{\sf INTERFACE:}
\begin{verbatim}       ! Private name; call using ESMF_CalendarCreate()
       function ESMF_CalendarCreateCopy(calendar, rc)
 \end{verbatim}{\em RETURN VALUE:}
\begin{verbatim}       type(ESMF_Calendar) :: ESMF_CalendarCreateCopy
 \end{verbatim}{\em ARGUMENTS:}
\begin{verbatim}       type(ESMF_Calendar), intent(in)            :: calendar
 -- The following arguments require argument keyword syntax (e.g. rc=rc). --
       integer,             intent(out), optional :: rc
 \end{verbatim}
{\sf STATUS:}
   \begin{itemize}
   \item\apiStatusCompatibleVersion{5.2.0r}
   \end{itemize}
  
{\sf DESCRIPTION:\\ }


       Creates a complete (deep) copy of a given {\tt ESMF\_Calendar}.
  
       The arguments are:
       \begin{description}
       \item[calendar]
          The {\tt ESMF\_Calendar} to copy.
       \item[{[rc]}]
          Return code; equals {\tt ESMF\_SUCCESS} if there are no errors.
       \end{description}
   
%/////////////////////////////////////////////////////////////
 
\mbox{}\hrulefill\ 
 
\subsubsection [ESMF\_CalendarCreate] {ESMF\_CalendarCreate - Create a new custom ESMF Calendar}


 
\bigskip{\sf INTERFACE:}
\begin{verbatim}       ! Private name; call using ESMF_CalendarCreate()
       function ESMF_CalendarCreateCustom(&
         daysPerMonth, secondsPerDay, &
         daysPerYear, daysPerYearDn, daysPerYearDd, name, rc)
 \end{verbatim}{\em RETURN VALUE:}
\begin{verbatim}       type(ESMF_Calendar) :: ESMF_CalendarCreateCustom
 \end{verbatim}{\em ARGUMENTS:}
\begin{verbatim} -- The following arguments require argument keyword syntax (e.g. rc=rc). --
       integer,               intent(in),  optional :: daysPerMonth(:)
       integer(ESMF_KIND_I4), intent(in),  optional :: secondsPerDay
       integer(ESMF_KIND_I4), intent(in),  optional :: daysPerYear
       integer(ESMF_KIND_I4), intent(in),  optional :: daysPerYearDn
       integer(ESMF_KIND_I4), intent(in),  optional :: daysPerYearDd
       character (len=*),     intent(in),  optional :: name
       integer,               intent(out), optional :: rc
 \end{verbatim}
{\sf DESCRIPTION:\\ }


       Creates a custom {\tt ESMF\_Calendar} and sets its properties.
  
       The arguments are:
       \begin{description}
       \item[{[daysPerMonth]}]
            Integer array of days per month, for each month of the year.
            The number of months per year is variable and taken from the
            size of the array.  If unspecified, months per year = 0,
            with the days array undefined.
       \item[{[secondsPerDay]}]
            Integer number of seconds per day.  Defaults to 0 if not 
            specified.
       \item[{[daysPerYear]}]
            Integer number of days per year.  Use with daysPerYearDn and
            daysPerYearDd (see below) to specify a days-per-year calendar
            for any planetary body.  Default = 0.
       \item[{[daysPerYearDn]}]
            \begin{sloppypar}
            Integer numerator portion of fractional number of days per year
            (daysPerYearDn/daysPerYearDd).
            Use with daysPerYear (see above) and daysPerYearDd (see below) to
            specify a days-per-year calendar for any planetary body.
            Default = 0.
            \end{sloppypar}
       \item[{[daysPerYearDd]}]
            Integer denominator portion of fractional number of days per year
            (daysPerYearDn/daysPerYearDd).
            Use with daysPerYear and daysPerYearDn (see above) to
            specify a days-per-year calendar for any planetary body.
            Default = 1.
       \item[{[name]}]
            The name for the newly created calendar.  If not specified, a
            default unique name will be generated: "CalendarNNN" where NNN
            is a unique sequence number from 001 to 999.
       \item[{[rc]}]
            Return code; equals {\tt ESMF\_SUCCESS} if there are no errors.
       \end{description}
        
%/////////////////////////////////////////////////////////////
 
\mbox{}\hrulefill\ 
 
\subsubsection [ESMF\_CalendarDestroy] {ESMF\_CalendarDestroy - Release resources associated with a Calendar}


  
\bigskip{\sf INTERFACE:}
\begin{verbatim}       subroutine ESMF_CalendarDestroy(calendar, rc)\end{verbatim}{\em ARGUMENTS:}
\begin{verbatim}       type(ESMF_Calendar), intent(inout)          :: calendar
 -- The following arguments require argument keyword syntax (e.g. rc=rc). --
       integer,             intent(out),  optional :: rc\end{verbatim}
{\sf STATUS:}
   \begin{itemize}
   \item\apiStatusCompatibleVersion{5.2.0r}
   \end{itemize}
  
{\sf DESCRIPTION:\\ }


       Releases resources associated with this {\tt ESMF\_Calendar}.
  
       The arguments are:
       \begin{description}
       \item[calendar]
         Release resources associated with this {\tt ESMF\_Calendar} and mark the
         object as invalid.  It is an error to pass this object into any other
         routines after being destroyed.
       \item[[rc]]
         Return code; equals {\tt ESMF\_SUCCESS} if there are no errors.
       \end{description}
   
%/////////////////////////////////////////////////////////////
 
\mbox{}\hrulefill\ 
 
\subsubsection [ESMF\_CalendarGet] {ESMF\_CalendarGet - Get Calendar properties}


 
\bigskip{\sf INTERFACE:}
\begin{verbatim}       subroutine ESMF_CalendarGet(calendar, &
         name, calkindflag, daysPerMonth, monthsPerYear, &
         secondsPerDay, secondsPerYear, &
         daysPerYear, daysPerYearDn, daysPerYearDd, rc)
 \end{verbatim}{\em ARGUMENTS:}
\begin{verbatim}       type(ESMF_Calendar),    intent(in)            :: calendar
 -- The following arguments require argument keyword syntax (e.g. rc=rc). --
       type(ESMF_CalKind_Flag),intent(out), optional :: calkindflag
       integer,                intent(out), optional :: daysPerMonth(:)
       integer,                intent(out), optional :: monthsPerYear
       integer(ESMF_KIND_I4),  intent(out), optional :: secondsPerDay
       integer(ESMF_KIND_I4),  intent(out), optional :: secondsPerYear
       integer(ESMF_KIND_I4),  intent(out), optional :: daysPerYear
       integer(ESMF_KIND_I4),  intent(out), optional :: daysPerYearDn
       integer(ESMF_KIND_I4),  intent(out), optional :: daysPerYearDd
       character (len=*),      intent(out), optional :: name
       integer,                intent(out), optional :: rc
 \end{verbatim}
{\sf STATUS:}
   \begin{itemize}
   \item\apiStatusCompatibleVersion{5.2.0r}
   \end{itemize}
  
{\sf DESCRIPTION:\\ }


       Gets one or more of an {\tt ESMF\_Calendar}'s properties.
  
       The arguments are:
       \begin{description}
       \item[calendar]
            The object instance to query.
       \item[{[calkindflag]}]
            The {\tt CalKind\_Flag} ESMF\_CALKIND\_GREGORIAN, 
            ESMF\_CALKIND\_JULIAN, etc.
       \item[{[daysPerMonth]}]
            Integer array of days per month, for each month of the year.
       \item[{[monthsPerYear]}]
            Integer number of months per year; the size of the
            daysPerMonth array.
       \item[{[secondsPerDay]}]
            Integer number of seconds per day.
       \item[{[secondsPerYear]}]
            Integer number of seconds per year.
       \item[{[daysPerYear]}]
            Integer number of days per year.  For calendars with
            intercalations, daysPerYear is the number of days for years without
            an intercalation.  For other calendars, it is the number of days in
            every year.
       \item[{[daysPerYearDn]}]
            \begin{sloppypar}
            Integer fractional number of days per year (numerator).
            For calendars with intercalations, daysPerYearDn/daysPerYearDd is
            the average fractional number of days per year (e.g. 25/100 for
            Julian 4-year intercalation).  For other calendars, it is zero.
            \end{sloppypar}
       \item[{[daysPerYearDd]}]
            Integer fractional number of days per year (denominator).  See
            daysPerYearDn above.
       \item[{[name]}]
            The name of this calendar.
       \item[{[rc]}]
            Return code; equals {\tt ESMF\_SUCCESS} if there are no errors.
       \end{description}
        
%/////////////////////////////////////////////////////////////
 
\mbox{}\hrulefill\ 
 
\subsubsection [ESMF\_CalendarIsCreated] {ESMF\_CalendarIsCreated - Check whether a Calendar object has been created}


 
\bigskip{\sf INTERFACE:}
\begin{verbatim}   function ESMF_CalendarIsCreated(calendar, rc)\end{verbatim}{\em RETURN VALUE:}
\begin{verbatim}     logical :: ESMF_CalendarIsCreated\end{verbatim}{\em ARGUMENTS:}
\begin{verbatim}     type(ESMF_Calendar), intent(in)            :: calendar
 -- The following arguments require argument keyword syntax (e.g. rc=rc). --
     integer,             intent(out), optional :: rc
 \end{verbatim}
{\sf DESCRIPTION:\\ }


     Return {\tt .true.} if the {\tt calendar} has been created. Otherwise return 
     {\tt .false.}. If an error occurs, i.e. {\tt rc /= ESMF\_SUCCESS} is 
     returned, the return value of the function will also be {\tt .false.}.
  
   The arguments are:
     \begin{description}
     \item[calendar]
       {\tt ESMF\_Calendar} queried.
     \item[{[rc]}]
       Return code; equals {\tt ESMF\_SUCCESS} if there are no errors.
     \end{description}
   
%/////////////////////////////////////////////////////////////
 
\mbox{}\hrulefill\ 
 
\subsubsection [ESMF\_CalendarIsLeapYear] {ESMF\_CalendarIsLeapYear - Determine if given year is a leap year}


  
\bigskip{\sf INTERFACE:}
\begin{verbatim}       ! Private name; call using ESMF_CalendarIsLeapYear()
       function ESMF_CalendarIsLeapYear<kind>(calendar, yy, rc)\end{verbatim}{\em RETURN VALUE:}
\begin{verbatim}       logical :: ESMF_CalendarIsLeapYear<kind>\end{verbatim}{\em ARGUMENTS:}
\begin{verbatim}       type(ESMF_Calendar),       intent(in)            :: calendar
       integer(ESMF_KIND_<kind>), intent(in)            :: yy
 -- The following arguments require argument keyword syntax (e.g. rc=rc). --
       integer,                   intent(out), optional :: rc\end{verbatim}
{\sf STATUS:}
   \begin{itemize}
   \item\apiStatusCompatibleVersion{5.2.0r}
   \end{itemize}
  
{\sf DESCRIPTION:\\ }


       \begin{sloppypar}
       Returns {\tt .true.} if the given year is a leap year within the given
       calendar, and {\tt .false.} otherwise.  Custom calendars do not define
       leap years, so {\tt .false.} will always be returned in this case;
       see Section ~\ref{subsec:Calendar_rest}.
       See also {\tt ESMF\_TimeIsLeapYear()}.
       \end{sloppypar}
  
       The arguments are:
       \begin{description}
       \item[calendar]
            {\tt ESMF\_Calendar} to determine leap year within.
       \item[yy]
            Year to check for leap year.  The type is integer and the <kind> can
            be either I4 or I8:  {\tt ESMF\_KIND\_I4} or {\tt ESMF\_KIND\_I8}.
       \item[{[rc]}]
            Return code; equals {\tt ESMF\_SUCCESS} if there are no errors.
       \end{description}
       
%/////////////////////////////////////////////////////////////
 
\mbox{}\hrulefill\ 
 
\subsubsection [ESMF\_CalendarPrint] {ESMF\_CalendarPrint - Print Calendar information}


 
\bigskip{\sf INTERFACE:}
\begin{verbatim}       subroutine ESMF_CalendarPrint(calendar, options, rc)
 \end{verbatim}{\em ARGUMENTS:}
\begin{verbatim}       type(ESMF_Calendar), intent(in)            :: calendar
       character (len=*),   intent(in),  optional :: options
       integer,             intent(out), optional :: rc
 \end{verbatim}
{\sf DESCRIPTION:\\ }


       Prints out an {\tt ESMF\_Calendar}'s properties to {\tt stdio}, 
       in support of testing and debugging.  The options control the 
       type of information and level of detail. \\
  
       The arguments are:
       \begin{description}
       \item[calendar]
            {\tt ESMF\_Calendar} to be printed out.
       \item[{[options]}]
            Print options. If none specified, prints all calendar property
                               values. \\
            "calkindflag"    - print the calendar's type 
                                 (e.g. ESMF\_CALKIND\_GREGORIAN). \\
            "daysPerMonth"   - print the array of number of days for
                                 each month. \\
            "daysPerYear"    - print the number of days per year
                               (integer and fractional parts). \\
            "monthsPerYear"  - print the number of months per year. \\
            "name"           - print the calendar's name. \\
            "secondsPerDay"  - print the number of seconds in a day. \\
            "secondsPerYear" - print the number of seconds in a year. \\
       \item[{[rc]}]
            Return code; equals {\tt ESMF\_SUCCESS} if there are no errors.
       \end{description}
   
%/////////////////////////////////////////////////////////////
 
\mbox{}\hrulefill\ 
 
\subsubsection [ESMF\_CalendarSet] {ESMF\_CalendarSet - Set a Calendar to a built-in type}


 
\bigskip{\sf INTERFACE:}
\begin{verbatim}       ! Private name; call using ESMF_CalendarSet()
       subroutine ESMF_CalendarSetBuiltIn(calendar, calkindflag, &
         name, rc)
 \end{verbatim}{\em ARGUMENTS:}
\begin{verbatim}       type(ESMF_Calendar),     intent(inout)         :: calendar
       type(ESMF_CalKind_Flag), intent(in)            :: calkindflag
 -- The following arguments require argument keyword syntax (e.g. rc=rc). --
       character (len=*),       intent(in),  optional :: name
       integer,                 intent(out), optional :: rc
 \end{verbatim}
{\sf STATUS:}
   \begin{itemize}
   \item\apiStatusCompatibleVersion{5.2.0r}
   \end{itemize}
  
{\sf DESCRIPTION:\\ }


       Sets {\tt calendar} to the given built-in {\tt ESMF\_CalKind\_Flag}. 
  
       The arguments are:
       \begin{description}
       \item[calendar]
            The object instance to initialize.
       \item[calkindflag]
            The built-in {\tt CalKind\_Flag}.  Valid values are:
              \newline
              {\tt ESMF\_CALKIND\_360DAY}, 
              \newline
              {\tt ESMF\_CALKIND\_GREGORIAN},
              \newline
              {\tt ESMF\_CALKIND\_JULIAN}, 
              \newline
              {\tt ESMF\_CALKIND\_JULIANDAY},
              \newline
              {\tt ESMF\_CALKIND\_MODJULIANDAY}, 
              \newline
              {\tt ESMF\_CALKIND\_NOCALENDAR},
              \newline
              and {\tt ESMF\_CALKIND\_NOLEAP}.
              \newline
            See Section ~\ref{subsec:Calendar_options} for a description of each
            calendar kind.
       \item[{[name]}]
            The new name for this calendar.
       \item[{[rc]}]
            Return code; equals {\tt ESMF\_SUCCESS} if there are no errors.
       \end{description}
       
%/////////////////////////////////////////////////////////////
 
\mbox{}\hrulefill\ 
 
\subsubsection [ESMF\_CalendarSet] {ESMF\_CalendarSet - Set properties of a custom Calendar}


 
\bigskip{\sf INTERFACE:}
\begin{verbatim}       ! Private name; call using ESMF_CalendarSet()
       subroutine ESMF_CalendarSetCustom(calendar, &
         daysPerMonth, secondsPerDay, &
         daysPerYear, daysPerYearDn, daysPerYearDd, name, rc)
 \end{verbatim}{\em ARGUMENTS:}
\begin{verbatim}       type(ESMF_Calendar),  intent(inout)         :: calendar
 -- The following arguments require argument keyword syntax (e.g. rc=rc). --
       integer,              intent(in),  optional :: daysPerMonth(:)
       integer(ESMF_KIND_I4),intent(in),  optional :: secondsPerDay
       integer(ESMF_KIND_I4),intent(in),  optional :: daysPerYear
       integer(ESMF_KIND_I4),intent(in),  optional :: daysPerYearDn
       integer(ESMF_KIND_I4),intent(in),  optional :: daysPerYearDd
       character (len=*),    intent(in),  optional :: name
       integer,              intent(out), optional :: rc
 \end{verbatim}
{\sf STATUS:}
   \begin{itemize}
   \item\apiStatusCompatibleVersion{5.2.0r}
   \end{itemize}
  
{\sf DESCRIPTION:\\ }


       Sets properties in a custom {\tt ESMF\_Calendar}.
  
       The arguments are:
       \begin{description}
       \item[calendar]
            The object instance to initialize.
       \item[{[daysPerMonth]}]
            Integer array of days per month, for each month of the year.
            The number of months per year is variable and taken from the
            size of the array.  If unspecified, months per year = 0,
            with the days array undefined.
       \item[{[secondsPerDay]}]
            Integer number of seconds per day.  Defaults to 0 if not 
            specified.
       \item[{[daysPerYear]}]
            Integer number of days per year.  Use with daysPerYearDn and
            daysPerYearDd (see below) to specify a days-per-year calendar
            for any planetary body.  Default = 0.
       \item[{[daysPerYearDn]}]
            Integer numerator portion of fractional number of days per year
            (daysPerYearDn/daysPerYearDd).
            Use with daysPerYear (see above) and daysPerYearDd (see below) to
            specify a days-per-year calendar for any planetary body.
            Default = 0.
       \item[{[daysPerYearDd]}]
            \begin{sloppypar}
            Integer denominator portion of fractional number of days per year
            (daysPerYearDn/daysPerYearDd).
            Use with daysPerYear and daysPerYearDn (see above) to
            specify a days-per-year calendar for any planetary body.
            Default = 1.
            \end{sloppypar}
       \item[{[name]}]
            The new name for this calendar.
       \item[{[rc]}]
            Return code; equals {\tt ESMF\_SUCCESS} if there are no errors.
       \end{description}
        
%/////////////////////////////////////////////////////////////
 
\mbox{}\hrulefill\ 
 
\subsubsection [ESMF\_CalendarSetDefault] {ESMF\_CalendarSetDefault - Set the default Calendar kind}


 
\bigskip{\sf INTERFACE:}
\begin{verbatim}       ! Private name; call using ESMF_CalendarSetDefault()
       subroutine ESMF_CalendarSetDefaultKind(calkindflag, rc)
 \end{verbatim}{\em ARGUMENTS:}
\begin{verbatim}       type(ESMF_CalKind_Flag), intent(in)            :: calkindflag
       integer,                 intent(out), optional :: rc
 \end{verbatim}
{\sf DESCRIPTION:\\ }


       Sets the default {\tt calendar} to the given type.  Subsequent Time
       Manager operations requiring a calendar where one isn't specified will
       use the internal calendar of this type.
  
       The arguments are:
       \begin{description}
       \item[calkindflag]
            The calendar kind to be the default.
       \item[{[rc]}]
            Return code; equals {\tt ESMF\_SUCCESS} if there are no errors.
       \end{description}
       
%/////////////////////////////////////////////////////////////
 
\mbox{}\hrulefill\ 
 
\subsubsection [ESMF\_CalendarSetDefault] {ESMF\_CalendarSetDefault - Set the default Calendar}


 
\bigskip{\sf INTERFACE:}
\begin{verbatim}       ! Private name; call using ESMF_CalendarSetDefault()
       subroutine ESMF_CalendarSetDefaultCal(calendar, rc)
 \end{verbatim}{\em ARGUMENTS:}
\begin{verbatim}       type(ESMF_Calendar),     intent(in)            :: calendar
       integer,                 intent(out), optional :: rc
 \end{verbatim}
{\sf DESCRIPTION:\\ }


       Sets the default {\tt calendar} to the one given.  Subsequent Time
       Manager operations requiring a calendar where one isn't specified will
       use this calendar.
  
       The arguments are:
       \begin{description}
       \item[calendar]
            The object instance to be the default.
       \item[{[rc]}]
            Return code; equals {\tt ESMF\_SUCCESS} if there are no errors.
       \end{description}
       
%/////////////////////////////////////////////////////////////
 
\mbox{}\hrulefill\ 
 
\subsubsection [ESMF\_CalendarValidate] {ESMF\_CalendarValidate - Validate a Calendar's properties}


 
\bigskip{\sf INTERFACE:}
\begin{verbatim}       subroutine ESMF_CalendarValidate(calendar, rc)
  \end{verbatim}{\em ARGUMENTS:}
\begin{verbatim}       type(ESMF_Calendar), intent(in)            :: calendar
 -- The following arguments require argument keyword syntax (e.g. rc=rc). --
       integer,             intent(out), optional :: rc
 \end{verbatim}
{\sf STATUS:}
   \begin{itemize}
   \item\apiStatusCompatibleVersion{5.2.0r}
   \end{itemize}
  
{\sf DESCRIPTION:\\ }


       Checks whether a {\tt calendar} is valid.  
       Must be one of the defined calendar kinds.  daysPerMonth, daysPerYear,
       secondsPerDay must all be greater than or equal to zero.
   
       The arguments are:
       \begin{description}
       \item[calendar]
            {\tt ESMF\_Calendar} to be validated.
       \item[{[rc]}]
            Return code; equals {\tt ESMF\_SUCCESS} if there are no errors.
       \end{description}
  
%...............................................................
\setlength{\parskip}{\oldparskip}
\setlength{\parindent}{\oldparindent}
\setlength{\baselineskip}{\oldbaselineskip}
