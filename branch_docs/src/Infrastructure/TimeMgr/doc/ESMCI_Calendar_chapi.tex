%                **** IMPORTANT NOTICE *****
% This LaTeX file has been automatically produced by ProTeX v. 1.1
% Any changes made to this file will likely be lost next time
% this file is regenerated from its source. Send questions 
% to Arlindo da Silva, dasilva@gsfc.nasa.gov
 
\setlength{\oldparskip}{\parskip}
\setlength{\parskip}{1.5ex}
\setlength{\oldparindent}{\parindent}
\setlength{\parindent}{0pt}
\setlength{\oldbaselineskip}{\baselineskip}
\setlength{\baselineskip}{11pt}
 
%--------------------- SHORT-HAND MACROS ----------------------
\def\bv{\begin{verbatim}}
\def\ev{\end{verbatim}}
\def\be{\begin{equation}}
\def\ee{\end{equation}}
\def\bea{\begin{eqnarray}}
\def\eea{\end{eqnarray}}
\def\bi{\begin{itemize}}
\def\ei{\end{itemize}}
\def\bn{\begin{enumerate}}
\def\en{\end{enumerate}}
\def\bd{\begin{description}}
\def\ed{\end{description}}
\def\({\left (}
\def\){\right )}
\def\[{\left [}
\def\]{\right ]}
\def\<{\left  \langle}
\def\>{\right \rangle}
\def\cI{{\cal I}}
\def\diag{\mathop{\rm diag}}
\def\tr{\mathop{\rm tr}}
%-------------------------------------------------------------

\markboth{Left}{Source File: ESMCI\_Calendar.h,  Date: Tue May  5 20:59:34 MDT 2020
}

 
%/////////////////////////////////////////////////////////////
\subsection{C++:  Class Interface ESMCI::Calendar - encapsulates calendar kinds and behavior (Source File: ESMCI\_Calendar.h)}


  
{\sf DESCRIPTION:\\ }


  
   The code in this file defines the C++ {\tt Calendar} members and method
   signatures (prototypes).  The companion file {\tt ESMC\_Calendar.C} contains
   the full code (bodies) for the {\tt Calendar} methods.
  
   The {\tt Calendar} class encapsulates the knowledge (attributes and
   behavior) of all required calendar kinds:  Gregorian, Julian, Julian Day,
   Modified Julian Day, no-leap, 360-day, custom, and no-calendar.
  
   The {\tt Calendar} class encapsulates the definition of all required
   calendar kinds. For each calendar kind, it contains the number of months
   per year, the number of days in each month, the number of seconds in a day,
   the number of days per year, and the number of fractional days per year.
   This flexible definition allows future calendars to be defined for any
   planetary body, not just Earth.
  
   The {\tt Calendar} class defines two methods for converting in both
   directions between the core {\tt BaseTime} class representation and a
   calendar date.  Calculations of time intervals (deltas) between
   time instants is done by the base class {\tt BaseTime} in the core units
   of seconds and fractional seconds.  Thus,  a calendar is only needed for
   converting core time to calendar time and vice versa.
  
   Notes:
      - Instantiate as few as possible; ideally no more than one calendar
        kind per application (for reference only, like a wall calendar)
        But may have multiples for convenience such as one per component.
      - Generic enough to define for any planetary body
      - if secondsPerDay != 86400, then how are minutes and hours defined ?
        Assume always minute=60 seconds; hour=3600 seconds
  
  -------------------------------------------------------------------------
    
\bigskip{\em USES:}
\begin{verbatim} #include "ESMCI_Fraction.h"
 #include "ESMCI_BaseTime.h"       // inherited BaseTime class
 #include "ESMC_Calendar.h"        // for enum ESMC_CalKind_Flag
 
   TODO: replace with monthsPerYear property
 #define MONTHS_PER_YEAR 12
 
 
 namespace ESMCI{
 
   forward reference to prevent #include recursion
 class Time;
 class TimeInterval;
 \end{verbatim}{\sf PUBLIC TYPES:}
\begin{verbatim}  class Calendar;
 \end{verbatim}{\sf PRIVATE TYPES:}
\begin{verbatim}  // class configuration type:  not needed for Calendar
 
  // class definition type
 class Calendar {
   class Calendar : public ESMC_Base { // TODO: inherit from ESMC_Base
                                             // class when fully aligned with
                                             //  F90 equiv
 
   private:   // corresponds to F90 module 'type ESMF_Calendar' members
 
     char              name[ESMF_MAXSTR];  // name of calendar
     ESMC_CalKind_Flag calkindflag;        // Calendar kind
 
     int daysPerMonth[MONTHS_PER_YEAR];
     int monthsPerYear;
   TODO: make dynamically allocatable with monthsPerYear
     ESMC_I4 secondsPerDay;        // TODO:  fractional secondsPerDay  sN/sD
     ESMC_I4 secondsPerYear;       // TODO:  fractional secondsPerYear yN/yD
     Fraction daysPerYear; // w: integer number of days per year
                           // n: fractional number of days per year (numerator)
                           // d:                                    (denominator)
                           // e.g. for Venus, w=0, n=926, d=1000
 
     // array of calendar kind name strings
     static const char *const calkindflagName[CALENDAR_KIND_COUNT];
 
     // one-of-each calendar kind held automatically, as needed
     static Calendar *internalCalendar[CALENDAR_KIND_COUNT];
     static Calendar *defaultCalendar;  // set-up upon ESMF_Initialize();
                                        // defaults to ESMC_CALKIND_NOCALENDAR
 
     int               id;         // unique identifier. used for equality
                                   //    checks and to generate unique default
                                   //    names.
                                   //    TODO: inherit from ESMC_Base class
     static int        count;      // number of calendars created. Thread-safe
                                   //   because int is atomic.
                                   //    TODO: inherit from ESMC_Base class
 \end{verbatim}{\sf PUBLIC MEMBER FUNCTIONS:}
\begin{verbatim} 
   public:
 
     // set built-in calendar kind
     int set(int               nameLen,
             const char       *name,    // TODO: default (=0)
             ESMC_CalKind_Flag calkindflag);
 
     // set custom calendar kind
     int set(int           nameLen,      
             const char   *name=0,
             int          *daysPerMonth=0,
             int           monthsPerYear=0,
             ESMC_I4      *secondsPerDay=0,
             ESMC_I4      *daysPerYear=0,
             ESMC_I4      *daysPerYearDn=0,
             ESMC_I4      *daysPerYearDd=0);
 
     // get properties of any calendar kind
     int get(int                nameLen,
             int               *tempNameLen,
             char              *tempName,
             ESMC_CalKind_Flag *calkindflag=0,
             int               *daysPerMonth=0,
             int                sizeofDaysPerMonth=0,
             int               *monthsPerYear=0,
             ESMC_I4           *secondsPerDay=0,
             ESMC_I4           *secondsPerYear=0,
             ESMC_I4           *daysPerYear=0,
             ESMC_I4           *daysPerYeardN=0,
             ESMC_I4           *daysPerYeardD=0);
 
     // Calendar doesn't need configuration, hence GetConfig/SetConfig
     // methods are not required
 
     // conversions based on UTC: time zone offset done by client
     //  (TMG 2.4.5, 2.5.6)
     int convertToTime(ESMC_I8 yy, int mm, int dd,
                       ESMC_I8 d, ESMC_R8 d_r8, BaseTime *t) const;
     int convertToDate(BaseTime *t, ESMC_I4 *yy=0, ESMC_I8 *yy_i8=0,
                       int *mm=0, int *dd=0,
                       ESMC_I4 *d=0, ESMC_I8 *d_i8=0,
                       ESMC_R8 *d_r8=0) const;
 
     Time increment(const Time *time, const TimeInterval &timeinterval) const;
 
     Time decrement(const Time *time, const TimeInterval &timeinterval) const;
 
     bool isLeapYear(ESMC_I8 yy, int *rc=0) const;
 
     bool operator==(const Calendar &) const;
     bool operator==(const ESMC_CalKind_Flag &) const;
     bool operator!=(const Calendar &) const;
     bool operator!=(const ESMC_CalKind_Flag &) const;
 
     // TODO:  add method to convert calendar interval to core time ?
 
     // required methods inherited and overridden from the ESMC_Base class
 
     // for persistence/checkpointing
 
     // friend to restore state
     friend Calendar *ESMCI_CalendarReadRestart(int, const char*, int*);
     // save state
     int writeRestart(void) const;
 
     // internal validation
     int validate(const char *options=0) const;
 
     // for testing/debugging
     int print(const char *options=0, const Time *time=0) const;
 
     // native C++ constructors/destructors
     Calendar(void);
     Calendar(const Calendar &calendar);  // copy constructor
     Calendar(const char *name, ESMC_CalKind_Flag calkindflag);
     Calendar(const char *name, int *daysPerMonth, int monthsPerYear,
              ESMC_I4 *secondsPerDay, ESMC_I4 *daysPerYear,
              ESMC_I4 *daysPerYeardN, ESMC_I4 *daysPerYearDd);
     ~Calendar(void);
 
  // < declare the rest of the public interface methods here >
 
     // friend function to allocate and initialize calendar from heap
     friend Calendar *ESMCI_CalendarCreate(int, const char*,
                                           ESMC_CalKind_Flag, int*);
 
     // friend function to allocate and initialize internal calendar from heap
     friend int ESMCI_CalendarCreate(ESMC_CalKind_Flag);
 
     // friend function to allocate and initialize custom calendar from heap
     friend Calendar *ESMCI_CalendarCreate(int, const char*, int*, int,
                                           ESMC_I4*, ESMC_I4*,
                                           ESMC_I4*, ESMC_I4*, int*);
 
     // friend function to copy a calendar
     friend Calendar *ESMCI_CalendarCreate(Calendar*, int*);
 
     // friend function to de-allocate calendar
     friend int ESMCI_CalendarDestroy(Calendar **);
 
     // friend function to de-allocate all internal calendars
     friend int ESMCI_CalendarFinalize(void);
     
     // friend functions to initialize and set the default calendar
     friend int ESMCI_CalendarInitialize(ESMC_CalKind_Flag *);
     friend int ESMCI_CalendarSetDefault(Calendar **);
     friend int ESMCI_CalendarSetDefault(ESMC_CalKind_Flag *);
 \end{verbatim}{\sf PRIVATE MEMBER FUNCTIONS:}
\begin{verbatim}   private:
 
     friend class Time;
     friend class TimeInterval;
 
  // < declare private interface methods here >\end{verbatim}

%...............................................................
\setlength{\parskip}{\oldparskip}
\setlength{\parindent}{\oldparindent}
\setlength{\baselineskip}{\oldbaselineskip}
