%                **** IMPORTANT NOTICE *****
% This LaTeX file has been automatically produced by ProTeX v. 1.1
% Any changes made to this file will likely be lost next time
% this file is regenerated from its source. Send questions 
% to Arlindo da Silva, dasilva@gsfc.nasa.gov
 
\setlength{\oldparskip}{\parskip}
\setlength{\parskip}{1.5ex}
\setlength{\oldparindent}{\parindent}
\setlength{\parindent}{0pt}
\setlength{\oldbaselineskip}{\baselineskip}
\setlength{\baselineskip}{11pt}
 
%--------------------- SHORT-HAND MACROS ----------------------
\def\bv{\begin{verbatim}}
\def\ev{\end{verbatim}}
\def\be{\begin{equation}}
\def\ee{\end{equation}}
\def\bea{\begin{eqnarray}}
\def\eea{\end{eqnarray}}
\def\bi{\begin{itemize}}
\def\ei{\end{itemize}}
\def\bn{\begin{enumerate}}
\def\en{\end{enumerate}}
\def\bd{\begin{description}}
\def\ed{\end{description}}
\def\({\left (}
\def\){\right )}
\def\[{\left [}
\def\]{\right ]}
\def\<{\left  \langle}
\def\>{\right \rangle}
\def\cI{{\cal I}}
\def\diag{\mathop{\rm diag}}
\def\tr{\mathop{\rm tr}}
%-------------------------------------------------------------

\markboth{Left}{Source File: ESMCI\_Alarm.h,  Date: Tue May  5 20:59:34 MDT 2020
}

 
%/////////////////////////////////////////////////////////////

   \subsection{C++:  Class Interface ESMCI::Alarm - maintains ringing times and ringing state (Source File: ESMCI\_Alarm.h)}


   
{\sf DESCRIPTION:\\ }


  
   The code in this file defines the C++ {\tt Alarm} members and method
   signatures (prototypes).  The companion file {\tt ESMC\_Alarm.C} contains
   the full code (bodies) for the {\tt Alarm} methods.
  
   The {\tt Alarm} class encapsulates the required alarm behavior, triggering
   its ringing state on either a one-shot or repeating interval basis.
  
   The {\tt Alarm} class contains {\tt Time} instants and a {\tt TimeInterval}
   to perform one-shot and interval alarming.  A single {\tt TimeInterval}
   holds the alarm interval if used.  A {\tt Time} instant is defined for the
   ring time, used for either the one-shot alarm time or for the next interval
   alarm time.  A {\tt Time} instant is also defined for the previous ring
   time to keep track of alarm intervals.  A {\tt Time} instant for stop time
   defines when alarm intervals end.  If a one-shot alarm is defined, only
   the ring time attribute is used, the others are not.  To keep track of
   alarm state, two logical attributes are defined, one for ringing on/off,
   and the other for alarm enabled/disabled.  An alarm is enabled by default;
   if disabled by the user, it does not function at all.
  
   The primary method is to check whether it is time to set the ringer, which
   is called by the associated clock after performing a time step.  The clock
   will pass a parameter telling the alarm check method whether the ringer is
   to be set upon crossing the ring time in the positive or negative direction.
   This is to handle both positive and negative clock timesteps.  After the
   ringer is set for interval alarms, the check method will recalculate the
   next ring time.  This can be in the positive or negative direction, again
   depending on the parameter passed in by the clock.
  
   Other methods are defined for getting the ringing state, turning the
   ringer on/off, enabling/disabling the alarm, and getting/setting the
   time attributes defined above.
  
   Notes:
      TMG 4.1, 4.2:  Multiple alarms may be instantiated and associated
                     with a clock via clock methods
  
  -------------------------------------------------------------------------
  
\bigskip{\em USES:}
\begin{verbatim} #include "ESMCI_TimeInterval.h"
 #include "ESMCI_Time.h"
 
  // alarm list flags to query from clock
  enum ESMC_AlarmList_Flag {ESMF_ALARMLIST_ALL = 1,
                            ESMF_ALARMLIST_RINGING,   
                            ESMF_ALARMLIST_NEXTRINGING,
                            ESMF_ALARMLIST_PREVRINGING};
   type of Alarm list
 namespace ESMCI {
 
  class Clock;
 \end{verbatim}{\sf PUBLIC TYPES:}
\begin{verbatim}  class Alarm;
  typedef Alarm* ESMCI_AlarmPtr;
 \end{verbatim}{\sf PRIVATE TYPES:}
\begin{verbatim}  // class configuration type:  not needed for Alarm
 
  // class definition type
 class Alarm {
  class Alarm : public ESMC_Base { // TODO: inherit from ESMC_Base class
                                         // when fully aligned with F90 equiv
 
 
   private:   // corresponds to F90 module 'type ESMF_Alarm' members
     char              name[ESMF_MAXSTR];  // name of alarm
                                           // TODO: inherit from ESMC_Base class
     Clock       *clock;        // associated clock
     TimeInterval ringInterval; // (TMG 4.5.2) for periodic alarming
     TimeInterval ringDuration; // how long alarm stays on
     Time         ringTime;     // (TMG 4.5.1) next time to ring
     Time         firstRingTime;    // the first ring time
                                         //   (save for reverse mode)
     Time         prevRingTime; // previous alarm time 
     Time         stopTime;     // when alarm intervals end.
     Time         ringBegin;    // note time when alarm turns on.
     Time         ringEnd;      // save time when alarm is turned off via
                                     //   ESMC_RingerOff().  For reverse mode.
                                     //   TODO: make array for variable
                                     //   turn off durations.
     Time         refTime;      // reference time.
     int               ringTimeStepCount;      // how long alarm rings;
                                               //  mutually exclusive with
                                               //  ringDuration
     int               timeStepRingingCount;   // how long alarm has been
                                               //   ringing in terms of a 
                                               //   number of time steps.
 
     bool              ringing;    // (TMG 4.4) currently ringing
     bool              ringingOnCurrTimeStep; // was ringing immediately after
                                              // current clock timestep.
                                              // (could have been turned off
                                              //  later due to RingerOff or
                                              //  Disable commands or
                                              //  non-sticky alarm expiration).
     bool              ringingOnPrevTimeStep; // was ringing immediately after
                                              // previous clock timestep.
     bool              userChangedRingTime;       // true if changed via Set(),
     bool              userChangedRingInterval;   // used to determine whether
                                                  // to adjust alarm on timeStep
                                                  // direction (sign) change
     bool              enabled;    // able to ring (TMG 4.5.3)
     bool              sticky;     // must be turned off via
                                   //   Alarm::ringerOff(),
                                   //  otherwise will turn self off after
                                   //  ringDuration or ringTimeStepCount.
     int               id;         // unique identifier. used for equality
                                   //    checks and to generate unique default
                                   //    names.
                                   //    TODO: inherit from ESMC_Base class
     static int        count;      // number of alarms created. Thread-safe
                                   //   because int is atomic.
                                   //    TODO: inherit from ESMC_Base class
 
      bool              pad1;       //  TODO:  align on byte boundary
 
      pthread_mutex_t   alarmMutex; // TODO: (TMG 7.5)
 \end{verbatim}{\sf PUBLIC MEMBER FUNCTIONS:}
\begin{verbatim} 
   public:
 
     // Alarm doesn't need configuration, hence GetConfig/SetConfig
     // methods are not required
 
     // accessor methods
 
                int    set(int                nameLen,
                       const char        *name=0,
                       Clock       **clock=0,
                       Time         *ringTime=0,
                       TimeInterval *ringInterval=0,
                       Time         *stopTime=0,
                       TimeInterval *ringDuration=0,
                       int               *ringTimeStepCount=0,
                       Time         *refTime=0,
                       bool              *ringing=0,
                       bool              *enabled=0,  // (TMG 4.1, 4.7)
                       bool              *sticky=0);
 
               int     get(int                nameLen,
                       int               *tempNameLen,
                       char              *tempName=0,
                       Clock       **clock=0,
                       Time         *ringTime=0,
                       Time         *prevRingTime=0,
                       TimeInterval *ringInterval=0,
                       Time         *stopTime=0,
                       TimeInterval *ringDuration=0,
                       int               *ringTimeStepCount=0,
                       int               *timeStepRingingCount=0,
                       Time         *ringBegin=0,
                       Time         *ringEnd=0,
                       Time         *refTime=0,
                       bool              *ringing=0,
                       bool              *ringingOnPrevTimeStep=0,
                       bool              *enabled=0,  // (TMG 4.1, 4.7)
                       bool              *sticky=0);
 
               int      enable(void);    // TMG4.5.3
               int      disable(void);
               bool     isEnabled(int *rc=0) const;
 
               int      ringerOn(void);    // TMG4.6: manually turn on/off
               int      ringerOff(void);
               bool     isRinging(int *rc=0) const;
                                          // TMG 4.4: synchronous query for apps
               bool     willRingNext(TimeInterval *timeStep, int *rc=0) const;
               bool     wasPrevRinging(int *rc=0) const;
 
               int      setToSticky(void);
               int      notSticky(TimeInterval *ringDuration=0,
                              int *ringTimeStepCount=0);
               bool     isSticky(int *rc=0) const;
 
               bool     checkRingTime(int *rc=0);
                          // associated clock should invoke after advance:
                          // TMG4.4, 4.6
                          // Check for crossing ringTime in either positive or
                          //   negative direction
                          // Can be basis for asynchronous alarm reporting
 
     bool operator==(const Alarm &) const; 
     bool operator!=(const Alarm &) const; 
 
     // required methods inherited and overridden from the ESMC_Base class
 
     // for persistence/checkpointing
 
     // friend to restore state
     friend Alarm *ESMCI_alarmReadRestart(int, const char*, int*);
     // save state
     int writeRestart(void) const;
 
     // internal validation
     int validate(const char *options=0) const;
 
     // for testing/debugging
     int print(const char *options=0) const;
 
     // native C++ constructors/destructors
     Alarm(void);
     Alarm(const Alarm &alarm);
     ~Alarm(void);
 
  // < declare the rest of the public interface methods here >
 
     // friend to allocate and initialize alarm from heap
     friend Alarm *ESMCI_alarmCreate(int, const char*, Clock*, 
                                  Time*, TimeInterval*, Time*, 
                                  TimeInterval*, int*, Time*, bool*,
                                  bool*, int*);
 
     // friend function to copy an alarm
     friend Alarm *ESMCI_alarmCreate(Alarm*, int*);
 
     // friend to de-allocate alarm
     friend int ESMCI_alarmDestroy(Alarm **);
 
     // friend function to de-allocate clock, allowing a clock's alarm's
     // clock pointers to be nullified
     friend int ESMCI_ClockDestroy(Clock **);
 \end{verbatim}{\sf PRIVATE MEMBER FUNCTIONS:}
\begin{verbatim}   private:
  // < declare private interface methods here >
 
     // check if time to turn on alarm
     bool checkTurnOn(bool timeStepPositive);
 
     // reconstruct ringBegin during ESMF_DIRECTION_REVERSE
     int resetRingBegin(bool timeStepPositive);
 
     // friend class alarm
     friend class Clock;
 \end{verbatim}

%...............................................................
\setlength{\parskip}{\oldparskip}
\setlength{\parindent}{\oldparindent}
\setlength{\baselineskip}{\oldbaselineskip}
