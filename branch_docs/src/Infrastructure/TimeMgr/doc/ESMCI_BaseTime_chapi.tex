%                **** IMPORTANT NOTICE *****
% This LaTeX file has been automatically produced by ProTeX v. 1.1
% Any changes made to this file will likely be lost next time
% this file is regenerated from its source. Send questions 
% to Arlindo da Silva, dasilva@gsfc.nasa.gov
 
\setlength{\oldparskip}{\parskip}
\setlength{\parskip}{1.5ex}
\setlength{\oldparindent}{\parindent}
\setlength{\parindent}{0pt}
\setlength{\oldbaselineskip}{\baselineskip}
\setlength{\baselineskip}{11pt}
 
%--------------------- SHORT-HAND MACROS ----------------------
\def\bv{\begin{verbatim}}
\def\ev{\end{verbatim}}
\def\be{\begin{equation}}
\def\ee{\end{equation}}
\def\bea{\begin{eqnarray}}
\def\eea{\end{eqnarray}}
\def\bi{\begin{itemize}}
\def\ei{\end{itemize}}
\def\bn{\begin{enumerate}}
\def\en{\end{enumerate}}
\def\bd{\begin{description}}
\def\ed{\end{description}}
\def\({\left (}
\def\){\right )}
\def\[{\left [}
\def\]{\right ]}
\def\<{\left  \langle}
\def\>{\right \rangle}
\def\cI{{\cal I}}
\def\diag{\mathop{\rm diag}}
\def\tr{\mathop{\rm tr}}
%-------------------------------------------------------------

\markboth{Left}{Source File: ESMCI\_BaseTime.h,  Date: Tue May  5 20:59:33 MDT 2020
}

 
%/////////////////////////////////////////////////////////////
\subsection{C++:  Class Interface ESMCI::BaseTime - Base time class (Source File: ESMCI\_BaseTime.h)}


  
{\sf DESCRIPTION:\\ }


  
   The code in this file defines the C++ {\tt ESMC\_BaseTime} members and
   declares method signatures (prototypes).  The companion file
   {\tt ESMC\_BaseTime.C} contains the definitions (full code bodies) for
   the {\tt ESMC\_BaseTime} methods.
  
   The {\tt BaseTime} class is a base class which encapsulates the core
   representation and functionality of time for time intervals and time
   instances.
  
   The {\tt BaseTime} class is designed with a minimum number of elements
   to represent any required time.  The design is based on the idea used
   in the real-time POSIX 1003.1b-1993 standard.  That is, to represent
   time simply as a pair of integers: one for seconds (whole) and one for
   nanoseconds (fractional).  These can then be converted at the interface
   level to any desired format.
  
   For ESMF, this idea is modified and extended, in order to handle the
   requirements for a large time range (> 200,000 years) and to exactly
   represent any rational fraction, not just nanoseconds.  To handle the
   large time range, a 64-bit or greater integer is used for whole seconds.
   Any rational fractional second is expressed using two additional integers:
   a numerator and a denominator.  Both the whole seconds and fractional
   numerator are signed to handle negative time intervals and instants.
   For arithmetic consistency both must carry the same sign (both positive
   or both negative), except, of course, for zero values.  The fractional
   seconds element (numerator) is \htmlref{normalized}{glos:Normalized}
   (bounded) with respect to whole seconds. If the absolute value of the
   numerator becomes greater than or equal to the denominator, the whole
   seconds is incremented or decremented accordingly and the numerator is
   reset to the remainder.  Conversions are performed upon demand by
   interface methods within the derived classes {\tt TimeInterval} and
   {\tt Time}.  This is done because different applications require different
   representations of time intervals and time instances.
  
   The {\tt BaseTime} class defines increment and decrement methods for basic
   time interval calculations between time instants.  It is done here rather
   than in the calendar class because it can be done with simple arithmetic
   that is calendar-independent.  Upon demand by a user, the results are
   converted to user-units via methods in the derived classes {\tt TimeInterval}
   and {\tt Time} and the associated {\tt Calendar} class.
  
   Comparison methods are also defined in the {\tt BaseTime} class.  These
   perform equality/inequality, less than, and greater than comparisons
   between any two {\tt TimeIntervals} or {\tt Time}.  These methods capture
   the common comparison logic between {\tt TimeIntervals} and {\tt Time} and
   hence are defined here for sharing.
  
   The separate class ESMCI::Fraction is inherited to handle fractional
   arithmetic.  ESMC_BaseTime encapsulates common time-specific knowledge,
   whereas ESMCI::Fraction is time-knowledge independent; it simply performs
   generic fractional arithmetic, manipulations and comparisons.
  
   For ease in calendar conversions, a time value of zero (both whole and
   numerator) will correspond to the Julian date of zero.
  
   The {\tt BaseTime} class is only a base class not to be instantiated by any
   application. It is only used by the derived classes {\tt TimeInterval} and
   {\tt Time}.
  
   Core representation meets TMG 1.3, 1.4, 2.2, 5.4
  
   TMG 1.5.2, 2.4.2: Copy from one object to another is handled
   via the class default memberwise assignment method, which uses
   the overloaded equals (=) operator.  E.g.  *obj1 = *obj2 which
   can be called from a C wrapper function mapping F90 to C++
  
   TMG 1.5.1, 2.4.1, 3.1, 4.1: Each class has an Set() function,
   which is used in lieu of a constructor since it can
   return an error code.
  
  -------------------------------------------------------------------------
  
\bigskip{\em USES:}
\begin{verbatim} #include "ESMCI_Fraction.h"
 
 namespace ESMCI {
 
  class Calendar;
 \end{verbatim}{\sf PUBLIC TYPES:}
\begin{verbatim}  class BaseTime;
 \end{verbatim}{\sf PRIVATE TYPES:}
\begin{verbatim}  // class configuration type:  not needed for ESMC_BaseTime
 
  // class definition type
 class BaseTime : public Fraction { // it is a fraction !
  class ESMC_BaseTime : public ESMC_Base { // TODO: inherit from ESMC_Base class
                                            // when fully aligned with F90 equiv
 
   protected:
 
     // TODO:  move ESMCI_Calendar* here to define seconds per day ? then could
     //        add D (Julian Days) to Get()/Set() below, and remove secondsPerDay
     //        from Get()
 
      pthread_mutex_t BaseTimeMutex; // for thread safety (TMG 7.5)
 \end{verbatim}{\sf PUBLIC MEMBER FUNCTIONS:}
\begin{verbatim} 
   public:
 
     // Get/Set methods (primarily to support F90 interface)
     int set(ESMC_I4 *h=0, ESMC_I4 *m=0,
                          ESMC_I4 *s=0, ESMC_I8 *s_i8=0,
                          ESMC_I4 *ms=0, ESMC_I4 *us=0,
                          ESMC_I4 *ns=0,
                          ESMC_R8 *h_r8=0, ESMC_R8 *m_r8=0,
                          ESMC_R8 *s_r8=0,
                          ESMC_R8 *ms_r8=0, ESMC_R8 *us_r8=0,
                          ESMC_R8 *ns_r8=0,
                          ESMC_I4 *sN=0, ESMC_I8 *sN_i8=0,
                          ESMC_I4 *sD=0, ESMC_I8 *sD_i8=0);
 
     int set(ESMC_I8 S, ESMC_I8 sN, ESMC_I8 sD);
 
     int get(const BaseTime *timeToConvert,
                          ESMC_I4 *h=0, ESMC_I4 *m=0,
                          ESMC_I4 *s=0, ESMC_I8 *s_i8=0,
                          ESMC_I4 *ms=0, ESMC_I4 *us=0,
                          ESMC_I4 *ns=0,
                          ESMC_R8 *h_r8=0, ESMC_R8 *m_r8=0,
                          ESMC_R8 *s_r8=0,
                          ESMC_R8 *ms_r8=0, ESMC_R8 *us_r8=0,
                          ESMC_R8 *ns_r8=0,
                          ESMC_I4 *sN=0, ESMC_I8 *sN_i8=0,
                          ESMC_I4 *sD=0, ESMC_I8 *sD_i8=0) const;
 
     // BaseTime doesn't need configuration, hence GetConfig/SetConfig
     // methods are not required
 
     // TODO: should be implicit, but then won't support
     //   F90 ESMF_Time & ESMF_TimeInterval via ESMC_BaseTime_F.C interface
     //   for increment/decrement
     BaseTime& operator=(const Fraction &);
 
     // required methods inherited and overridden from the ESMC_Base class
 
     // for persistence/checkpointing
     int readRestart(int nameLen, const char *name=0);
     int writeRestart(void) const;
 
     // internal validation
     int validate(const char *options=0) const;
 
     // for testing/debugging
     int print(const char *options=0) const;
 
     // native C++ constructors/destructors
     BaseTime(void);
     BaseTime(ESMC_I8 S, ESMC_I8 sN=0, ESMC_I8 sD=1);
     ~BaseTime(void);
 
  // < declare the rest of the public interface methods here >
 \end{verbatim}{\sf PRIVATE MEMBER FUNCTIONS:}
\begin{verbatim}   private:
 
     friend class Calendar;
 
  // < declare private interface methods here >\end{verbatim}

%...............................................................
\setlength{\parskip}{\oldparskip}
\setlength{\parindent}{\oldparindent}
\setlength{\baselineskip}{\oldbaselineskip}
