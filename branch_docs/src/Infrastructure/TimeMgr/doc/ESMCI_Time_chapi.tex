%                **** IMPORTANT NOTICE *****
% This LaTeX file has been automatically produced by ProTeX v. 1.1
% Any changes made to this file will likely be lost next time
% this file is regenerated from its source. Send questions 
% to Arlindo da Silva, dasilva@gsfc.nasa.gov
 
\setlength{\oldparskip}{\parskip}
\setlength{\parskip}{1.5ex}
\setlength{\oldparindent}{\parindent}
\setlength{\parindent}{0pt}
\setlength{\oldbaselineskip}{\baselineskip}
\setlength{\baselineskip}{11pt}
 
%--------------------- SHORT-HAND MACROS ----------------------
\def\bv{\begin{verbatim}}
\def\ev{\end{verbatim}}
\def\be{\begin{equation}}
\def\ee{\end{equation}}
\def\bea{\begin{eqnarray}}
\def\eea{\end{eqnarray}}
\def\bi{\begin{itemize}}
\def\ei{\end{itemize}}
\def\bn{\begin{enumerate}}
\def\en{\end{enumerate}}
\def\bd{\begin{description}}
\def\ed{\end{description}}
\def\({\left (}
\def\){\right )}
\def\[{\left [}
\def\]{\right ]}
\def\<{\left  \langle}
\def\>{\right \rangle}
\def\cI{{\cal I}}
\def\diag{\mathop{\rm diag}}
\def\tr{\mathop{\rm tr}}
%-------------------------------------------------------------

\markboth{Left}{Source File: ESMCI\_Time.h,  Date: Tue May  5 20:59:33 MDT 2020
}

 
%/////////////////////////////////////////////////////////////
\subsection{C++:  Class Interface ESMCI::Time - represents a specific point in time (Source File: ESMCI\_Time.h)}


  
{\sf DESCRIPTION:\\ }


  
   The code in this file defines the C++ {\tt Time} members and method
   signatures (prototypes).  The companion file {\tt ESMC\_Time.C} contains
   the full code (bodies) for the {\tt Time] methods.
  
   A {\tt Time} inherits from the {\tt BaseTime} base class and is designed
   to represent a specific point in time which is dependent upon a calendar
   kind.
  
   {\tt Time} inherits from the base class {\tt BaseTime}.  As such, it gains
   the core representation of time as well as its associated methods.
   {\tt Time} further specializes {\tt BaseTime} by adding shortcut methods
   to set and get a {\tt Time} in a natural way with appropriate unit
   combinations, as per the requirements.  A {\tt Time} is calendar-dependent,
   since its largest units of time are months and years.  {\tt Time} also
   defines special methods for getting the day of the year, day of the week,
   middle of the month, and synchronizing with a real-time clock.  {\tt Time}
   adds its associated {\tt Calendar} and local Timezone attributes to
   {\tt BaseTime}.
  
  -------------------------------------------------------------------------
  
\bigskip{\em USES:}
\begin{verbatim} #include "ESMCI_BaseTime.h"       // inherited BaseTime class
 #include "ESMCI_Calendar.h"       // associated Calendar class
 
  namespace ESMCI{
 
  class TimeInterval;
 \end{verbatim}{\sf PUBLIC TYPES:}
\begin{verbatim}  class Time;
 \end{verbatim}{\sf PRIVATE TYPES:}
\begin{verbatim}  // class configuration type:  not needed for Time
 
  // class definition type
  class Time : public BaseTime { // inherits ESMC_BaseTime 
                                           // TODO: (& ESMC_Base class when
                                           // fully aligned with F90 equiv)
   private:   // corresponds to F90 module 'type ESMF_Time' members 
     Calendar *calendar;    // associated calendar
     int timeZone;               // Offset from UTC
 \end{verbatim}{\sf PUBLIC MEMBER FUNCTIONS:}
\begin{verbatim} 
   public:
     // Time doesn't need configuration, hence GetConfig/SetConfig
     // methods are not required
 
     // accessor methods
     // all get/set routines perform signed conversions, where applicable
 
     // Get/Set methods to support the F90 optional arguments interface
     int set(ESMC_I4 *yy=0, ESMC_I8 *yy_i8=0,
                      int *mm=0, int *dd=0,
                      ESMC_I4 *d=0,  ESMC_I8 *d_i8=0,
                      ESMC_I4 *h=0,  ESMC_I4 *m=0,
                      ESMC_I4 *s=0,  ESMC_I8 *s_i8=0,
                      ESMC_I4 *ms=0, ESMC_I4 *us=0,
                      ESMC_I4 *ns=0,
                      ESMC_R8 *d_r8=0,  ESMC_R8 *h_r8=0,
                      ESMC_R8 *m_r8=0,  ESMC_R8 *s_r8=0,
                      ESMC_R8 *ms_r8=0, ESMC_R8 *us_r8=0,
                      ESMC_R8 *ns_r8=0,
                      ESMC_I4 *sN=0, ESMC_I8 *sN_i8=0,
                      ESMC_I4 *sD=0, ESMC_I8 *sD_i8=0,
                      Calendar **calendar=0, 
                      ESMC_CalKind_Flag *calkindflag=0, 
                      int *timeZone=0);
 
     int get(ESMC_I4 *yy=0, ESMC_I8 *yy_i8=0,
                      int *mm=0, int *dd=0,
                      ESMC_I4 *d=0,  ESMC_I8 *d_i8=0,
                      ESMC_I4 *h=0,  ESMC_I4 *m=0,
                      ESMC_I4 *s=0,  ESMC_I8 *s_i8=0,
                      ESMC_I4 *ms=0, ESMC_I4 *us=0,
                      ESMC_I4 *ns=0,
                      ESMC_R8 *d_r8=0,  ESMC_R8 *h_r8=0,
                      ESMC_R8 *m_r8=0,  ESMC_R8 *s_r8=0,
                      ESMC_R8 *ms_r8=0, ESMC_R8 *us_r8=0,
                      ESMC_R8 *ns_r8=0,
                      ESMC_I4 *sN=0, ESMC_I8 *sN_i8=0,
                      ESMC_I4 *sD=0, ESMC_I8 *sD_i8=0,
                      Calendar **calendar=0, 
                      ESMC_CalKind_Flag *calkindflag=0, 
                      int *timeZone=0,
                      int timeStringLen=0, int *tempTimeStringLen=0,
                      char *tempTimeString=0,
                      int timeStringLenISOFrac=0,
                      int *tempTimeStringLenISOFrac=0,
                      char *tempTimeStringISOFrac=0,
                      int *dayOfWeek=0,
                      Time *midMonth=0,
                      ESMC_I4 *dayOfYear=0,
                      ESMC_R8 *dayOfYear_r8=0,
                      TimeInterval *dayOfYear_intvl=0) const;
 
     // native C++ interface -- via variable argument lists
     //   corresponds to F90 named-optional-arguments interface
 
     // (TMG 2.1, 2.5.1, 2.5.6)
     //int get(const char *timeList, ...) const;
     // e.g. Time::get("YY:MM:DD", (int *)YY,(int *)MM, (int *)DD);
     //int set(Calendar *calendar, int timeZone,
                      //const char *timeList, ...);
     //int set(const char *timeList, ...);
     // e.g. Time::set("s" , (ESMC_R8) s);
 
     bool isLeapYear(int *rc=0) const;
     bool isSameCalendar(const Time *time, int *rc=0) const;
 
     // to support Clock::syncToWallClock() and TMG 2.5.7
     int syncToRealTime(void);
 
     // override BaseTime +/- operators in order to copy Time-only
     // properties (calendar & timeZone) to the result
     Time operator+(const TimeInterval &) const; // time + timeinterval
     Time operator-(const TimeInterval &) const; // time - timeinterval
     Time& operator+=(const TimeInterval &); // time += timeinterval
     Time& operator-=(const TimeInterval &); // time -= timeinterval
 
     // override 2nd BaseTime (-) operator because 1st (-) operator is overridden
     // (compiler can't find 2nd (-) operator at ESMCI::Fraction!)
     TimeInterval operator-(const Time&) const;  // time1 - time2
 
     // TODO: ? override BaseTime arithmetic operators with same operators
     //         which use the BaseTime operators and then specialize
     //         with logic to validate (range check) the new value against
     //         the associated calendar.
 
     // required methods inherited and overridden from the ESMC_Base class
 
     // for persistence/checkpointing
     int readRestart(int nameLen, const char *name=0);
     int writeRestart(void) const;
 
     // internal validation
     int validate(const char *options=0) const;  // (TMG 7.1.1)
 
     // for testing/debugging
     int print(const char *options=0) const;
 
     // native C++ constructors/destructors
     Time(void);
     Time(ESMC_I8 s, ESMC_I8 sN=0, ESMC_I8 sD=1, Calendar *calendar=0,
               ESMC_CalKind_Flag calkindflag=(ESMC_CalKind_Flag)0,
               int timeZone=0);
     int set(ESMC_I8 s, ESMC_I8 sN=0, ESMC_I8 sD=1,
                      Calendar *calendar=0,
                      ESMC_CalKind_Flag calkindflag=(ESMC_CalKind_Flag)0,
                      int timeZone=0);
                                    // used internally instead of constructor
                                    // to cover case of initial entry from F90,
                                    // to avoid automatic destructor invocation
                                    // when leaving scope to return to F90.
     ~Time(void);
 
  // < declare the rest of the public interface methods here >
 
     // return in string format (TMG 2.4.7)
     int getString(char *timeString, const char *options=0) const;
 
     int getDayOfWeek(int *dayOfWeek) const;    // (TMG 2.5.3)
     int getMidMonth(Time *midMonth) const;
 
     int getDayOfYear(ESMC_I4 *dayOfYear) const;
     int getDayOfYear(ESMC_R8 *dayOfYear) const; // (TMG 2.5.2)
     int getDayOfYear(TimeInterval *dayOfYear) const;
 \end{verbatim}{\sf PRIVATE MEMBER FUNCTIONS:}
\begin{verbatim}   private:
 
     friend class TimeInterval;
     friend class Calendar;
                                                         // (TMG 2.5.5)
  // < declare private interface methods here >\end{verbatim}

%...............................................................
\setlength{\parskip}{\oldparskip}
\setlength{\parindent}{\oldparindent}
\setlength{\baselineskip}{\oldbaselineskip}
