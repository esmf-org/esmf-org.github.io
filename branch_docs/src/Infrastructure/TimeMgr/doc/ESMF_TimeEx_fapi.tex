%                **** IMPORTANT NOTICE *****
% This LaTeX file has been automatically produced by ProTeX v. 1.1
% Any changes made to this file will likely be lost next time
% this file is regenerated from its source. Send questions 
% to Arlindo da Silva, dasilva@gsfc.nasa.gov
 
\setlength{\oldparskip}{\parskip}
\setlength{\parskip}{1.5ex}
\setlength{\oldparindent}{\parindent}
\setlength{\parindent}{0pt}
\setlength{\oldbaselineskip}{\baselineskip}
\setlength{\baselineskip}{11pt}
 
%--------------------- SHORT-HAND MACROS ----------------------
\def\bv{\begin{verbatim}}
\def\ev{\end{verbatim}}
\def\be{\begin{equation}}
\def\ee{\end{equation}}
\def\bea{\begin{eqnarray}}
\def\eea{\end{eqnarray}}
\def\bi{\begin{itemize}}
\def\ei{\end{itemize}}
\def\bn{\begin{enumerate}}
\def\en{\end{enumerate}}
\def\bd{\begin{description}}
\def\ed{\end{description}}
\def\({\left (}
\def\){\right )}
\def\[{\left [}
\def\]{\right ]}
\def\<{\left  \langle}
\def\>{\right \rangle}
\def\cI{{\cal I}}
\def\diag{\mathop{\rm diag}}
\def\tr{\mathop{\rm tr}}
%-------------------------------------------------------------

\markboth{Left}{Source File: ESMF\_TimeEx.F90,  Date: Tue May  5 20:59:34 MDT 2020
}

 
%/////////////////////////////////////////////////////////////

 \begin{verbatim}
! !PROGRAM: ESMF_TimeEx - Time initialization and manipulation examples
!
! !DESCRIPTION:
!
! This program shows examples of Time initialization and manipulation
!-----------------------------------------------------------------------------
#include "ESMF.h"

      ! ESMF Framework module
      use ESMF
      use ESMF_TestMod
      implicit none

      ! instantiate two times
      type(ESMF_Time) :: time1, time2

      type(ESMF_VM) :: vm

      ! instantiate a time interval
      type(ESMF_TimeInterval) :: timeinterval1

      ! local variables for Get methods
      integer :: YY, MM, DD, H, M, S

      ! return code
      integer:: rc
 
\end{verbatim}
 
%/////////////////////////////////////////////////////////////

 \begin{verbatim}
      ! initialize ESMF framework
      call ESMF_Initialize(vm=vm, defaultCalKind=ESMF_CALKIND_GREGORIAN, &
        defaultlogfilename="TimeEx.Log", &
        logkindflag=ESMF_LOGKIND_MULTI, rc=rc)
 
\end{verbatim}
 
%/////////////////////////////////////////////////////////////

  \subsubsection{Time initialization}
 
   This example shows how to initialize an {\tt ESMF\_Time}. 
%/////////////////////////////////////////////////////////////

 \begin{verbatim}
      ! initialize time1 to 2/28/2000 2:24:45
      call ESMF_TimeSet(time1, yy=2000, mm=2, dd=28, h=2, m=24, s=45, rc=rc)
 
\end{verbatim}
 
%/////////////////////////////////////////////////////////////

 \begin{verbatim}
      print *, "Time1 = "
      call ESMF_TimePrint(time1, options="string", rc=rc)
 
\end{verbatim}
 
%/////////////////////////////////////////////////////////////

  \subsubsection{Time increment}
 
   This example shows how to increment an {\tt ESMF\_Time} by
   an {\tt ESMF\_TimeInterval}. 
%/////////////////////////////////////////////////////////////

 \begin{verbatim}
      ! initialize a time interval to 2 days, 8 hours, 36 minutes, 15 seconds
      call ESMF_TimeIntervalSet(timeinterval1, d=2, h=8, m=36, s=15, rc=rc)
 
\end{verbatim}
 
%/////////////////////////////////////////////////////////////

 \begin{verbatim}
      print *, "Timeinterval1 = "
      call ESMF_TimeIntervalPrint(timeinterval1, options="string", rc=rc)
 
\end{verbatim}
 
%/////////////////////////////////////////////////////////////

 \begin{verbatim}
      ! increment time1 with timeinterval1
      time2 = time1 + timeinterval1

      call ESMF_TimeGet(time2, yy=YY, mm=MM, dd=DD, h=H, m=M, s=S, rc=rc)
      print *, "time2 = time1 + timeinterval1 = ", YY, "/", MM, "/", DD, &
               " ",  H, ":", M, ":", S
 
\end{verbatim}
 
%/////////////////////////////////////////////////////////////

  \subsubsection{Time comparison}
 
   This example shows how to compare two {\tt ESMF\_Times}. 
%/////////////////////////////////////////////////////////////

 \begin{verbatim}
      if (time2 > time1) then
        print *, "time2 is larger than time1"
      else
        print *, "time1 is smaller than or equal to time2"
      endif

 
\end{verbatim}
 
%/////////////////////////////////////////////////////////////

 \begin{verbatim}
      ! finalize ESMF framework
      call ESMF_Finalize(rc=rc)
 
\end{verbatim}
 
%/////////////////////////////////////////////////////////////

 \begin{verbatim}
      end program ESMF_TimeEx
 
\end{verbatim}

%...............................................................
\setlength{\parskip}{\oldparskip}
\setlength{\parindent}{\oldparindent}
\setlength{\baselineskip}{\oldbaselineskip}
