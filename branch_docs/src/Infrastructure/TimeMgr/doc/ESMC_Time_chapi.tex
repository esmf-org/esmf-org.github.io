%                **** IMPORTANT NOTICE *****
% This LaTeX file has been automatically produced by ProTeX v. 1.1
% Any changes made to this file will likely be lost next time
% this file is regenerated from its source. Send questions 
% to Arlindo da Silva, dasilva@gsfc.nasa.gov
 
\setlength{\oldparskip}{\parskip}
\setlength{\parskip}{1.5ex}
\setlength{\oldparindent}{\parindent}
\setlength{\parindent}{0pt}
\setlength{\oldbaselineskip}{\baselineskip}
\setlength{\baselineskip}{11pt}
 
%--------------------- SHORT-HAND MACROS ----------------------
\def\bv{\begin{verbatim}}
\def\ev{\end{verbatim}}
\def\be{\begin{equation}}
\def\ee{\end{equation}}
\def\bea{\begin{eqnarray}}
\def\eea{\end{eqnarray}}
\def\bi{\begin{itemize}}
\def\ei{\end{itemize}}
\def\bn{\begin{enumerate}}
\def\en{\end{enumerate}}
\def\bd{\begin{description}}
\def\ed{\end{description}}
\def\({\left (}
\def\){\right )}
\def\[{\left [}
\def\]{\right ]}
\def\<{\left  \langle}
\def\>{\right \rangle}
\def\cI{{\cal I}}
\def\diag{\mathop{\rm diag}}
\def\tr{\mathop{\rm tr}}
%-------------------------------------------------------------

\markboth{Left}{Source File: ESMC\_Time.h,  Date: Tue May  5 20:59:34 MDT 2020
}

 
%/////////////////////////////////////////////////////////////
\subsubsection [ESMC\_TimeGet] {ESMC\_TimeGet - Get a Time value}


  
\bigskip{\sf INTERFACE:}
\begin{verbatim} int ESMC_TimeGet(
   ESMC_Time time,                         // in
   ESMC_I4 *yy,                            // out
   ESMC_I4 *h,                             // out
   ESMC_Calendar *calendar,                // out
   enum ESMC_CalKind_Flag *calkindflag,    // out
   int *timeZone                           // out
 );
 \end{verbatim}{\em RETURN VALUE:}
\begin{verbatim}    Return code; equals ESMF_SUCCESS if there are no errors.\end{verbatim}
{\sf DESCRIPTION:\\ }


  
    Gets the value of an {\tt ESMC\_Time} in units specified by the user.
  
    The arguments are:
    \begin{description}
    \item[time]
      {\tt ESMC\_Time} object to be queried.
    \item[{[yy]}]
      Integer year (>= 32-bit).
    \item[{[h]}]
      Integer hours.
    \item[{[calendar]}]
      Associated {\tt ESMC\_Calendar}.
    \item[{[calkindflag]}]
      Associated {\tt ESMC\_CalKind\_Flag}.
    \end{description}
   
%/////////////////////////////////////////////////////////////
 
\mbox{}\hrulefill\ 
 
\subsubsection [ESMC\_TimePrint] {ESMC\_TimePrint - Print a Time}


  
\bigskip{\sf INTERFACE:}
\begin{verbatim} int ESMC_TimePrint(
   ESMC_Time time   // in
 );
 \end{verbatim}{\em RETURN VALUE:}
\begin{verbatim}    Return code; equals ESMF_SUCCESS if there are no errors.\end{verbatim}
{\sf DESCRIPTION:\\ }


    Prints out an {\tt ESMC\_Time}'s properties to {\tt stdio}, 
    in support of testing and debugging.
  
    The arguments are:
    \begin{description}
    \item[time]
      {\tt ESMC\_Time} object to be printed.
    \end{description}
   
%/////////////////////////////////////////////////////////////
 
\mbox{}\hrulefill\ 
 
\subsubsection [ESMC\_TimeSet] {ESMC\_TimeSet - Initialize or set a Time}


  
\bigskip{\sf INTERFACE:}
\begin{verbatim} int ESMC_TimeSet(
   ESMC_Time *time,                       // inout
   ESMC_I4 yy,                            // in
   ESMC_I4 h,                             // in
   ESMC_Calendar calendar,                // in
   enum ESMC_CalKind_Flag calkindflag,    // in
   int timeZone                           // in
 );
 \end{verbatim}{\em RETURN VALUE:}
\begin{verbatim}    Return code; equals ESMF_SUCCESS if there are no errors.\end{verbatim}
{\sf DESCRIPTION:\\ }


  
    Initializes an {\tt ESMC\_Time} with a set of user-specified units.
  
    The arguments are:
    \begin{description}
    \item[time]
      {\tt ESMC\_Time} object to initialize or set.
    \item[yy]
      Integer year (>= 32-bit).
    \item[h]
      Integer hours.
    \item[calendar]
      Associated {\tt ESMC\_Calendar}.  If not created, defaults to calendar
      {\tt ESMC\_CALKIND\_NOCALENDAR} or default specified in
      {\tt ESMC\_Initialize()}.  If created, has precedence over
      calkindflag below.
    \item[calkindflag]
      Specifies associated {\tt ESMC\_Calendar} if calendar argument above
      not created.  More convenient way of specifying a built-in calendar kind.
    \end{description}
  
%...............................................................
\setlength{\parskip}{\oldparskip}
\setlength{\parindent}{\oldparindent}
\setlength{\baselineskip}{\oldbaselineskip}
