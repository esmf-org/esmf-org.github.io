%                **** IMPORTANT NOTICE *****
% This LaTeX file has been automatically produced by ProTeX v. 1.1
% Any changes made to this file will likely be lost next time
% this file is regenerated from its source. Send questions 
% to Arlindo da Silva, dasilva@gsfc.nasa.gov
 
\setlength{\oldparskip}{\parskip}
\setlength{\parskip}{1.5ex}
\setlength{\oldparindent}{\parindent}
\setlength{\parindent}{0pt}
\setlength{\oldbaselineskip}{\baselineskip}
\setlength{\baselineskip}{11pt}
 
%--------------------- SHORT-HAND MACROS ----------------------
\def\bv{\begin{verbatim}}
\def\ev{\end{verbatim}}
\def\be{\begin{equation}}
\def\ee{\end{equation}}
\def\bea{\begin{eqnarray}}
\def\eea{\end{eqnarray}}
\def\bi{\begin{itemize}}
\def\ei{\end{itemize}}
\def\bn{\begin{enumerate}}
\def\en{\end{enumerate}}
\def\bd{\begin{description}}
\def\ed{\end{description}}
\def\({\left (}
\def\){\right )}
\def\[{\left [}
\def\]{\right ]}
\def\<{\left  \langle}
\def\>{\right \rangle}
\def\cI{{\cal I}}
\def\diag{\mathop{\rm diag}}
\def\tr{\mathop{\rm tr}}
%-------------------------------------------------------------

\markboth{Left}{Source File: ESMC\_Calendar.h,  Date: Tue May  5 20:59:33 MDT 2020
}

 
%/////////////////////////////////////////////////////////////
\subsubsection [ESMC\_CalendarCreate] {ESMC\_CalendarCreate - Create a Calendar}


  
\bigskip{\sf INTERFACE:}
\begin{verbatim} ESMC_Calendar ESMC_CalendarCreate(
   const char *name,                      // in
   enum ESMC_CalKind_Flag calkindflag,    // in
   int *rc                                // out
 );
 \end{verbatim}{\em RETURN VALUE:}
\begin{verbatim}    Newly created ESMC_Calendar object.\end{verbatim}
{\sf DESCRIPTION:\\ }


  
    Creates and sets a {\tt ESMC\_Calendar} object to the given built-in
    {\tt ESMC\_CalKind\_Flag}. 
  
    The arguments are:
    \begin{description}
    \item[{[name]}]
      The name for the newly created Calendar.  If not specified, i.e. NULL,
      a default unique name will be generated: "CalendarNNN" where NNN
      is a unique sequence number from 001 to 999.
    \item[calkindflag]
      The built-in {\tt ESMC\_CalKind\_Flag}.  Valid values are:
      \newline
      {\tt ESMC\_CALKIND\_360DAY}, 
      \newline
      {\tt ESMC\_CALKIND\_GREGORIAN},
      \newline
      {\tt ESMC\_CALKIND\_JULIAN}, 
      \newline
      {\tt ESMC\_CALKIND\_JULIANDAY},
      \newline
      {\tt ESMC\_CALKIND\_MODJULIANDAY}, 
      \newline
      {\tt ESMC\_CALKIND\_NOCALENDAR},
      \newline
      and {\tt ESMC\_CALKIND\_NOLEAP}.
      \newline
      See Section ~\ref{subsec:Calendar_options} for a description of each
      calendar kind.
    \item[{[rc]}]
      Return code; equals {\tt ESMF\_SUCCESS} if there are no errors.
    \end{description}
   
%/////////////////////////////////////////////////////////////
 
\mbox{}\hrulefill\ 
 
\subsubsection [ESMC\_CalendarDestroy] {ESMC\_CalendarDestroy - Destroy a Calendar}


  
\bigskip{\sf INTERFACE:}
\begin{verbatim} int ESMC_CalendarDestroy(
   ESMC_Calendar *calendar   // inout
 );
 \end{verbatim}{\em RETURN VALUE:}
\begin{verbatim}    Return code; equals ESMF_SUCCESS if there are no errors.\end{verbatim}
{\sf DESCRIPTION:\\ }


  
    Releases all resources associated with this {\tt ESMC\_Calendar}.
  
    The arguments are:
    \begin{description}
    \item[calendar]
      Destroy contents of this {\tt ESMC\_Calendar}.
    \end{description}
   
%/////////////////////////////////////////////////////////////
 
\mbox{}\hrulefill\ 
 
\subsubsection [ESMC\_CalendarPrint] {ESMC\_CalendarPrint - Print a Calendar}


  
\bigskip{\sf INTERFACE:}
\begin{verbatim} int ESMC_CalendarPrint(
   ESMC_Calendar calendar   // in
 );
 \end{verbatim}{\em RETURN VALUE:}
\begin{verbatim}    Return code; equals ESMF_SUCCESS if there are no errors.\end{verbatim}
{\sf DESCRIPTION:\\ }


    Prints out an {\tt ESMC\_Calendar}'s properties to {\tt stdio}, 
    in support of testing and debugging.
  
    The arguments are:
    \begin{description}
    \item[calendar]
      {\tt ESMC\_Calendar} object to be printed.
    \end{description}
  
%...............................................................
\setlength{\parskip}{\oldparskip}
\setlength{\parindent}{\oldparindent}
\setlength{\baselineskip}{\oldbaselineskip}
