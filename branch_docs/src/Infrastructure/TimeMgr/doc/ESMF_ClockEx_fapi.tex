%                **** IMPORTANT NOTICE *****
% This LaTeX file has been automatically produced by ProTeX v. 1.1
% Any changes made to this file will likely be lost next time
% this file is regenerated from its source. Send questions 
% to Arlindo da Silva, dasilva@gsfc.nasa.gov
 
\setlength{\oldparskip}{\parskip}
\setlength{\parskip}{1.5ex}
\setlength{\oldparindent}{\parindent}
\setlength{\parindent}{0pt}
\setlength{\oldbaselineskip}{\baselineskip}
\setlength{\baselineskip}{11pt}
 
%--------------------- SHORT-HAND MACROS ----------------------
\def\bv{\begin{verbatim}}
\def\ev{\end{verbatim}}
\def\be{\begin{equation}}
\def\ee{\end{equation}}
\def\bea{\begin{eqnarray}}
\def\eea{\end{eqnarray}}
\def\bi{\begin{itemize}}
\def\ei{\end{itemize}}
\def\bn{\begin{enumerate}}
\def\en{\end{enumerate}}
\def\bd{\begin{description}}
\def\ed{\end{description}}
\def\({\left (}
\def\){\right )}
\def\[{\left [}
\def\]{\right ]}
\def\<{\left  \langle}
\def\>{\right \rangle}
\def\cI{{\cal I}}
\def\diag{\mathop{\rm diag}}
\def\tr{\mathop{\rm tr}}
%-------------------------------------------------------------

\markboth{Left}{Source File: ESMF\_ClockEx.F90,  Date: Tue May  5 20:59:34 MDT 2020
}

 
%/////////////////////////////////////////////////////////////

 \begin{verbatim}
! !PROGRAM: ESMF_ClockEx - Clock initialization and time-stepping
!
! !DESCRIPTION:
!
! This program shows an example of how to create, initialize, advance, and
! examine a basic clock
!-----------------------------------------------------------------------------
#include "ESMF.h"

      ! ESMF Framework module
      use ESMF
      use ESMF_TestMod
      implicit none

      ! instantiate a clock 
      type(ESMF_Clock) :: clock

      ! instantiate time_step, start and stop times
      type(ESMF_TimeInterval) :: timeStep
      type(ESMF_Time) :: startTime
      type(ESMF_Time) :: stopTime

      ! local variables for Get methods
      type(ESMF_Time) :: currTime
      integer(ESMF_KIND_I8) :: advanceCount
      integer :: YY, MM, DD, H, M, S

      ! return code
      integer :: rc
 
\end{verbatim}
 
%/////////////////////////////////////////////////////////////

 \begin{verbatim}
      ! initialize ESMF framework
      call ESMF_Initialize(defaultCalKind=ESMF_CALKIND_GREGORIAN, &
        defaultlogfilename="ClockEx.Log", &
                    logkindflag=ESMF_LOGKIND_MULTI, rc=rc)
 
\end{verbatim}
 
%/////////////////////////////////////////////////////////////

  \subsubsection{Clock creation}
 
   This example shows how to create and initialize an {\tt ESMF\_Clock}. 
%/////////////////////////////////////////////////////////////

 \begin{verbatim}
      ! initialize time interval to 2 days, 4 hours (6 timesteps in 13 days)
      call ESMF_TimeIntervalSet(timeStep, d=2, h=4, rc=rc)
 
\end{verbatim}
 
%/////////////////////////////////////////////////////////////

 \begin{verbatim}
      ! initialize start time to 4/1/2003 2:24:00 ( 1/10 of a day )
      call ESMF_TimeSet(startTime, yy=2003, mm=4, dd=1, h=2, m=24, rc=rc)
 
\end{verbatim}
 
%/////////////////////////////////////////////////////////////

 \begin{verbatim}
      ! initialize stop time to 4/14/2003 2:24:00 ( 1/10 of a day )
      call ESMF_TimeSet(stopTime, yy=2003, mm=4, dd=14, h=2, m=24, rc=rc)
 
\end{verbatim}
 
%/////////////////////////////////////////////////////////////

 \begin{verbatim}
      ! initialize the clock with the above values
      clock = ESMF_ClockCreate(timeStep, startTime, stopTime=stopTime, &
                               name="Clock 1", rc=rc)
 
\end{verbatim}
 
%/////////////////////////////////////////////////////////////

  \subsubsection{Clock advance}
 
   This example shows how to time-step an {\tt ESMF\_Clock}. 
%/////////////////////////////////////////////////////////////

 \begin{verbatim}
      ! time step clock from start time to stop time
      do while (.not.ESMF_ClockIsStopTime(clock, rc=rc))
 
\end{verbatim}
 
%/////////////////////////////////////////////////////////////

 \begin{verbatim}
        call ESMF_ClockPrint(clock, options="currTime string", rc=rc)
 
\end{verbatim}
 
%/////////////////////////////////////////////////////////////

 \begin{verbatim}
        call ESMF_ClockAdvance(clock, rc=rc)
 
\end{verbatim}
 
%/////////////////////////////////////////////////////////////

 \begin{verbatim}
      end do
 
\end{verbatim}
 
%/////////////////////////////////////////////////////////////

  \subsubsection{Clock examination}
 
   This example shows how to examine an {\tt ESMF\_Clock}. 
%/////////////////////////////////////////////////////////////

 \begin{verbatim}
      ! get the clock's final current time
      call ESMF_ClockGet(clock, currTime=currTime, rc=rc)
 
\end{verbatim}
 
%/////////////////////////////////////////////////////////////

 \begin{verbatim}
      call ESMF_TimeGet(currTime, yy=YY, mm=MM, dd=DD, h=H, m=M, s=S, rc=rc) 
      print *, "The clock's final current time is ", YY, "/", MM, "/", DD, &
               " ", H, ":", M, ":", S
 
\end{verbatim}
 
%/////////////////////////////////////////////////////////////

 \begin{verbatim}
      ! get the number of times the clock was advanced
      call ESMF_ClockGet(clock, advanceCount=advanceCount, rc=rc)
      print *, "The clock was advanced ", advanceCount, " times."
 
\end{verbatim}
 
%/////////////////////////////////////////////////////////////

  \subsubsection{Clock reversal}
 
   This example shows how to time-step an {\tt ESMF\_Clock} in reverse mode. 
%/////////////////////////////////////////////////////////////

 \begin{verbatim}
      call ESMF_ClockSet(clock, direction=ESMF_DIRECTION_REVERSE, rc=rc)
 
\end{verbatim}
 
%/////////////////////////////////////////////////////////////

 \begin{verbatim}
      ! time step clock in reverse from stop time back to start time;
      !  note use of ESMF_ClockIsDone() rather than ESMF_ClockIsStopTime()
      do while (.not.ESMF_ClockIsDone(clock, rc=rc))
 
\end{verbatim}
 
%/////////////////////////////////////////////////////////////

 \begin{verbatim}
        call ESMF_ClockPrint(clock, options="currTime string", rc=rc)
 
\end{verbatim}
 
%/////////////////////////////////////////////////////////////

 \begin{verbatim}
        call ESMF_ClockAdvance(clock, rc=rc)
 
\end{verbatim}
 
%/////////////////////////////////////////////////////////////

 \begin{verbatim}
      end do
 
\end{verbatim}
 
%/////////////////////////////////////////////////////////////

  \subsubsection{Clock destruction}
 
   This example shows how to destroy an {\tt ESMF\_Clock}. 
%/////////////////////////////////////////////////////////////

 \begin{verbatim}
      ! destroy clock
      call ESMF_ClockDestroy(clock, rc=rc)
 
\end{verbatim}
 
%/////////////////////////////////////////////////////////////

 \begin{verbatim}
      ! finalize ESMF framework
      call ESMF_Finalize(rc=rc)
 
\end{verbatim}
 
%/////////////////////////////////////////////////////////////

 \begin{verbatim}
      end program ESMF_ClockEx
 
\end{verbatim}

%...............................................................
\setlength{\parskip}{\oldparskip}
\setlength{\parindent}{\oldparindent}
\setlength{\baselineskip}{\oldbaselineskip}
