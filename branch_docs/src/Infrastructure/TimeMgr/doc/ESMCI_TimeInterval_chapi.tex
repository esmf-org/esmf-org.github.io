%                **** IMPORTANT NOTICE *****
% This LaTeX file has been automatically produced by ProTeX v. 1.1
% Any changes made to this file will likely be lost next time
% this file is regenerated from its source. Send questions 
% to Arlindo da Silva, dasilva@gsfc.nasa.gov
 
\setlength{\oldparskip}{\parskip}
\setlength{\parskip}{1.5ex}
\setlength{\oldparindent}{\parindent}
\setlength{\parindent}{0pt}
\setlength{\oldbaselineskip}{\baselineskip}
\setlength{\baselineskip}{11pt}
 
%--------------------- SHORT-HAND MACROS ----------------------
\def\bv{\begin{verbatim}}
\def\ev{\end{verbatim}}
\def\be{\begin{equation}}
\def\ee{\end{equation}}
\def\bea{\begin{eqnarray}}
\def\eea{\end{eqnarray}}
\def\bi{\begin{itemize}}
\def\ei{\end{itemize}}
\def\bn{\begin{enumerate}}
\def\en{\end{enumerate}}
\def\bd{\begin{description}}
\def\ed{\end{description}}
\def\({\left (}
\def\){\right )}
\def\[{\left [}
\def\]{\right ]}
\def\<{\left  \langle}
\def\>{\right \rangle}
\def\cI{{\cal I}}
\def\diag{\mathop{\rm diag}}
\def\tr{\mathop{\rm tr}}
%-------------------------------------------------------------

\markboth{Left}{Source File: ESMCI\_TimeInterval.h,  Date: Tue May  5 20:59:33 MDT 2020
}

 
%/////////////////////////////////////////////////////////////

  \subsection{C++:  Class Interface ESMCI::TimeInterval - represents a time interval (Source File: ESMCI\_TimeInterval.h)}


  
{\sf DESCRIPTION:\\ }


     A {\tt TimeInterval} inherits from the {\tt BaseTime} base class and is
     designed to represent time deltas. These can either be independent of
     any calendar or dependent on a calendar and thought of as a calendar
     interval. 
  
     {\tt TimeInterval} inherits from the base class {\tt BaseTime}.  As such,
     it gains the core representation of time as well as its associated methods.
     {\tt TimeInterval} further specializes {\tt BaseTime} by adding shortcut
     methods to set and get a {\tt TimeInterval} in a natural way with 
     appropriate unit combinations, as per the requirements.  The largest
     calendar-independent resolution of time for a {\tt TimeInterval} is in
     hours.  A {\tt TimeInterval} can also be used as a {\tt Calendar} interval.
     Then it becomes calendar-dependent, since its largest resolution of time
     will be in days, months and years.  Days are calendar-specific to allow
     for non-Earth specific calendars.  {\tt TimeInterval} also defines methods
     for multiplication and division of {\tt TimeIntervals} by integers, reals,
     fractions and other {\tt TimeIntervals}.  {\tt TimeInterval} defines
     methods for absolute value and negative absolute value for use with both
     positive or negative time intervals. 
  
     Notes:
         - For arithmetic consistency both whole seconds and the numerator of
           fractional seconds must carry the same sign (both positive or both 
           negative), except, of course, for zero values.
  
  -------------------------------------------------------------------------
 
\bigskip{\em USES:}
\begin{verbatim} #include "ESMCI_BaseTime.h"       // inherited BaseTime class
 #include "ESMCI_Time.h"
 #include "ESMCI_Calendar.h"
 #include "ESMC_TimeInterval.h"    // for use of enumerated types
 
 namespace ESMCI {
 
   PUBLIC TYPES:
 class TimeInterval;
 \end{verbatim}{\sf PRIVATE TYPES:}
\begin{verbatim} 
  // class definition type
 class TimeInterval : public BaseTime { 
                                              // inherits BaseTime
                                              // TODO: (& ESMC_Base class when
                                              // fully aligned with F90 equiv)
   private:
     Time startTime;  // for absolute calendar intervals
     Time endTime;    // for absolute calendar intervals
     Calendar *calendar;  // for calendar intervals on a 
                               //   specific calendar
     ESMC_I8 yy;      // for relative Calendar intervals:  number of years
     ESMC_I8 mm;      // for relative Calendar intervals:  number of months
     ESMC_I8 d;       // for relative Calendar intervals:  integer number of days
     ESMC_R8 d_r8;    // for relative Calendar intervals:  real number of days
 \end{verbatim}{\sf PUBLIC MEMBER FUNCTIONS:}
\begin{verbatim} 
   public:
 
     // accessor methods
 
     // all get/set routines perform signed conversions, where applicable;
     //   direct, one-to-one access to core time elements is provided by the
     //   BaseTime base class
 
     // Get/Set methods to support the F90 optional arguments interface
     int set(ESMC_I4 *yy=0, ESMC_I8 *yy_i8=0,
                              ESMC_I4 *mm=0, ESMC_I8 *mm_i8=0,
                              ESMC_I4 *d=0,  ESMC_I8 *d_i8=0,
                              ESMC_I4 *h=0,  ESMC_I4 *m=0,
                              ESMC_I4 *s=0,  ESMC_I8 *s_i8=0,
                              ESMC_I4 *ms=0, ESMC_I4 *us=0,
                              ESMC_I4 *ns=0,
                              ESMC_R8 *d_r8=0,  ESMC_R8 *h_r8=0,
                              ESMC_R8 *m_r8=0,  ESMC_R8 *s_r8=0,
                              ESMC_R8 *ms_r8=0, ESMC_R8 *us_r8=0,
                              ESMC_R8 *ns_r8=0,
                              ESMC_I4 *sN=0, ESMC_I8 *sN_i8=0,
                              ESMC_I4 *sD=0, ESMC_I8 *sD_i8=0,
                              Time *startTime=0, Time *endTime=0,
                              Calendar **calendar=0,
                              ESMC_CalKind_Flag *calkindflag=0);
 
     int get(ESMC_I4 *yy=0, ESMC_I8 *yy_i8=0,
                              ESMC_I4 *mm=0, ESMC_I8 *mm_i8=0,
                              ESMC_I4 *d=0,  ESMC_I8 *d_i8=0,
                              ESMC_I4 *h=0,  ESMC_I4 *m=0,
                              ESMC_I4 *s=0,  ESMC_I8 *s_i8=0,
                              ESMC_I4 *ms=0, ESMC_I4 *us=0,
                              ESMC_I4 *ns=0,
                              ESMC_R8 *d_r8=0,  ESMC_R8 *h_r8=0,
                              ESMC_R8 *m_r8=0,  ESMC_R8 *s_r8=0,
                              ESMC_R8 *ms_r8=0, ESMC_R8 *us_r8=0,
                              ESMC_R8 *ns_r8=0,
                              ESMC_I4 *sN=0, ESMC_I8 *sN_i8=0,
                              ESMC_I4 *sD=0, ESMC_I8 *sD_i8=0,
                              Time *startTime=0, Time *endTime=0,
                              Calendar **calendar=0,
                              ESMC_CalKind_Flag *calkindflag=0,
                              Time *startTimeIn=0, Time *endTimeIn=0,
                              Calendar **calendarIn=0,
                              ESMC_CalKind_Flag *calkindflagIn=0,
                              int timeStringLen=0, int *tempTimeStringLen=0,
                              char *tempTimeString=0,
                              int timeStringLenISOFrac=0,
                              int *tempTimeStringLenISOFrac=0,
                              char *tempTimeStringISOFrac=0) const;
 
     // native C++ interface -- via variable argument lists
     //   corresponds to F90 named-optional-arguments interface
 
     //int set(const char *timeList, ...);
     // e.g. Set("s" , (ESMC_R8) s);
 
     // (TMG 1.1)
     //int get(const char *timeList, ...) const;
     // e.g. Get("D:S",(int *)D, (int *)S);
 
     // return positive value (TMG 1.5.8)
     TimeInterval absValue(void) const;
 
     // return negative value (TMG 1.5.8)
     TimeInterval negAbsValue(void) const;
 
     // subdivision (TMG 1.5.6, 5.3, 7.2)
     TimeInterval  operator/ (const ESMC_I4 &) const;
     TimeInterval& operator/=(const ESMC_I4 &);
     TimeInterval  operator/ (const ESMC_R8 &) const;
     TimeInterval& operator/=(const ESMC_R8 &);
 
     // division (TMG 1.5.5)
     Fraction div(const TimeInterval &) const;
     ESMC_R8 operator/(const TimeInterval &) const;
 
     // modulus
     TimeInterval  operator% (const TimeInterval &) const;
     TimeInterval& operator%=(const TimeInterval &);
 
     // multiplication (TMG 1.5.7, 7.2)
     TimeInterval  operator* (const ESMC_I4 &) const;
     TimeInterval& operator*=(const ESMC_I4 &);
     TimeInterval  operator* (const Fraction &) const;
     TimeInterval& operator*=(const Fraction &);
     TimeInterval  operator* (const ESMC_R8 &) const;
     TimeInterval& operator*=(const ESMC_R8 &);
 
     // addition, subtraction
     TimeInterval operator+(const TimeInterval &) const;
     TimeInterval operator-(const TimeInterval &) const;
 
     // unary negation
     TimeInterval operator-(void) const;
 
     // comparison methods (TMG 1.5.3, 7.2)
     bool operator==(const TimeInterval &) const;
     bool operator!=(const TimeInterval &) const;
     bool operator< (const TimeInterval &) const;
     bool operator> (const TimeInterval &) const;
     bool operator<=(const TimeInterval &) const;
     bool operator>=(const TimeInterval &) const;
 
     // copy or assign from ESMCI::Fraction expressions, supports Time1-Time2
     // operator in Time.
     // TODO:  should be implicit ?
     TimeInterval& operator=(const Fraction &);
 
     // required methods inherited and overridden from the ESMC_Base class
 
     // for persistence/checkpointing
     int readRestart(int nameLen, const char *name=0);
     int writeRestart(void) const;
 
     // internal validation (TMG 7.1.1)
     int validate(const char *options=0) const;
 
     // for testing/debugging
     int print(const char *options=0) const;
 
     // native C++ constructors/destructors
     TimeInterval(void);
     TimeInterval(ESMC_I8 s, ESMC_I8 sN=0, ESMC_I8 sD=1,
                  ESMC_I8 yy=0, ESMC_I8 mm=0, ESMC_I8 d=0, ESMC_R8 d_r8=0.0,
                  Time *startTime=0, Time *endTime=0,
                  Calendar *calendar=0,
                  ESMC_CalKind_Flag calkindflag=(ESMC_CalKind_Flag)0);
     int set(ESMC_I8 s, ESMC_I8 sN=0, ESMC_I8 sD=1,
             ESMC_I8 yy=0, ESMC_I8 mm=0, ESMC_I8 d=0, ESMC_R8 d_r8=0.0,
             Time *startTime=0, Time *endTime=0,
             Calendar *calendar=0,
             ESMC_CalKind_Flag calkindflag=(ESMC_CalKind_Flag)0);
                                    // used internally instead of constructor
                                    // to cover case of initial entry from F90,
                                    // to avoid automatic destructor invocation
                                    // when leaving scope to return to F90.
 
     ~TimeInterval(void);
 
  // < declare the rest of the public interface methods here >
 
     // commutative complements to TimeInterval class member overloaded
     //   "*" operators
     friend TimeInterval operator* (const ESMC_I4 &,  const TimeInterval &);
     friend TimeInterval operator* (const Fraction &, const TimeInterval &);
     friend TimeInterval operator* (const ESMC_R8 &,  const TimeInterval &);
 \end{verbatim}{\sf PRIVATE MEMBER FUNCTIONS:}
\begin{verbatim}   private:
     // return in string format (TMG 1.5.9)
     int getString(char *timeString, const char *options=0) const;
 
     // common method for overloaded comparison operators
     bool compare(const TimeInterval &, ESMC_ComparisonType) const;
 
     // common method for positive and negative absolute value
     TimeInterval absValue(ESMC_AbsValueType) const;
 
     // reduce time interval to smallest and least number of units
     int reduce(void);
 
     friend class Time;
     friend class Calendar;
                                                         // (TMG 2.5.5)
  // < declare private interface methods here >\end{verbatim}

%...............................................................
\setlength{\parskip}{\oldparskip}
\setlength{\parindent}{\oldparindent}
\setlength{\baselineskip}{\oldbaselineskip}
