%                **** IMPORTANT NOTICE *****
% This LaTeX file has been automatically produced by ProTeX v. 1.1
% Any changes made to this file will likely be lost next time
% this file is regenerated from its source. Send questions 
% to Arlindo da Silva, dasilva@gsfc.nasa.gov
 
\setlength{\oldparskip}{\parskip}
\setlength{\parskip}{1.5ex}
\setlength{\oldparindent}{\parindent}
\setlength{\parindent}{0pt}
\setlength{\oldbaselineskip}{\baselineskip}
\setlength{\baselineskip}{11pt}
 
%--------------------- SHORT-HAND MACROS ----------------------
\def\bv{\begin{verbatim}}
\def\ev{\end{verbatim}}
\def\be{\begin{equation}}
\def\ee{\end{equation}}
\def\bea{\begin{eqnarray}}
\def\eea{\end{eqnarray}}
\def\bi{\begin{itemize}}
\def\ei{\end{itemize}}
\def\bn{\begin{enumerate}}
\def\en{\end{enumerate}}
\def\bd{\begin{description}}
\def\ed{\end{description}}
\def\({\left (}
\def\){\right )}
\def\[{\left [}
\def\]{\right ]}
\def\<{\left  \langle}
\def\>{\right \rangle}
\def\cI{{\cal I}}
\def\diag{\mathop{\rm diag}}
\def\tr{\mathop{\rm tr}}
%-------------------------------------------------------------

\markboth{Left}{Source File: ESMCI\_Clock.h,  Date: Tue May  5 20:59:33 MDT 2020
}

 
%/////////////////////////////////////////////////////////////

  \subsection{C++:  Class Interface ESMCI::Clock - keeps track of model time (Source File: ESMCI\_Clock.h)}


  
{\sf DESCRIPTION:\\ }


  
   The code in this file defines the C++ {\tt Clock} members and declares
   method signatures (prototypes).  The companion file {\tt ESMC\_Clock.C}
   contains the definitions (full code bodies) for the {\tt Clock} methods.
  
   The {\tt Clock} class encapsulates the essential ESM component requirement
   of tracking and time-stepping model time.  It also checks associated alarms
   to trigger their ringing state.
  
   The {\tt Clock} class contains {\tt Time} instants and a {\tt TimeInterval}
   to track and time step model time.  For tracking, {\tt Time} instants are
   instantiated for the current time, stop time, start time, reference time,
   and previous time.  For time stepping, a single {\tt TimeInterval} is
   instantiated.  There is also an integer counter for keeping track of the
   number of timesteps, and an array of associated alarms.  Methods are
   defined for advancing the clock (perform a time step), checking if the
   stop time is reached, synchronizing with a real-time clock, and getting
   values of the class attributes defined above. After performing the time
   step, the advance method will iterate over the alarm list and return a
   list of any active alarms.
  
   Notes:
      TMG 3.2:  Create multiple clocks by simply instantiating this class
                multiple times
  
      TMG 3.3:  Component's responsibility
  
  -------------------------------------------------------------------------
  
\bigskip{\em USES:}
\begin{verbatim} #include "ESMCI_TimeInterval.h"
 #include "ESMCI_Time.h"
 #include "ESMCI_Alarm.h"
 
 namespace ESMCI{
 \end{verbatim}{\sf PUBLIC TYPES:}
\begin{verbatim}  class Clock;
 \end{verbatim}{\sf PRIVATE TYPES:}
\begin{verbatim}  // class configuration type:  not needed for Clock
 
  // class definition type
  class Clock {
   class Clock : public ESMC_Base { // TODO: inherit from ESMC_Base class
                                          // when fully aligned with F90 equiv
 
   private:   // corresponds to F90 module 'type ESMF_Clock' members
     char         name[ESMF_MAXSTR];  // name of clock
     TimeInterval timeStep;
     TimeInterval currAdvanceTimeStep; // timeStep used in current
                                       // ClockAdvance()
     TimeInterval prevAdvanceTimeStep; // timeStep used in previous
                                       // ClockAdvance()
     Time         startTime;
     Time         stopTime;
     Time         refTime;   // reference time
     Time         currTime;  // current time
     Time         prevTime;  // previous time
 
     ESMC_I8      advanceCount;             // number of times
                                                 //   ESMCI_ClockAdvance has
                                                 //   been called (number of
                                                 //   time steps taken so far)
 
     ESMC_Direction    direction;                // forward (default) or reverse
     bool              userChangedDirection;     // used to determine whether
                                                 // to adjust alarm on 
                                                 // direction change
 
     int               alarmCount;               // number of defined alarms
     int               alarmListCapacity;        // max number of defined alarms
                                                 //  before a reallocation is
                                                 //  necessary
     Alarm           **alarmList;                // associated alarm array
 
     bool              stopTimeEnabled;  // true if optional property set
 
     int               id;         // unique identifier. used for equality
                                   //    checks and to generate unique default
                                   //    names.
                                   //    TODO: inherit from ESMC_Base class
     static int        count;      // number of clocks created. Thread-safe
                                   //   because int is atomic.
                                   //    TODO: inherit from ESMC_Base class
 
      pthread_mutex_t clockMutex; // TODO: (TMG 7.5)
 \end{verbatim}{\sf PUBLIC MEMBER FUNCTIONS:}
\begin{verbatim} 
   public:
 
     // Clock doesn't need configuration, hence GetConfig/SetConfig
     // methods are not required
 
     // accessor methods
 
     int set(int                nameLen,
                       const char        *name=0,
                       TimeInterval *timeStep=0,
                       Time         *startTime=0,
                       Time         *stopTime=0,
                       TimeInterval *runDuration=0,
                       int               *runTimeStepCount=0,
   TODO: add overload for ESMC_R8  *runTimeStepCount=0,
                       Time         *refTime=0,    // (TMG 3.1, 3.4.4)
                       Time         *currTime=0,
                       ESMC_I8      *advanceCount=0,
                       ESMC_Direction    *direction=0);
 
     int get(int                nameLen,
                       int               *tempNameLen,
                       char              *tempName=0,
                       TimeInterval *timeStep=0,
                       Time         *startTime=0,
                       Time         *stopTime=0,
                       TimeInterval *runDuration=0,
                       ESMC_R8      *runTimeStepCount=0,
                       Time         *refTime=0,    // (TMG 3.1, 3.4.4)
                       Time         *currTime=0, 
                       Time         *prevTime=0, 
                       TimeInterval *currSimTime=0, 
                       TimeInterval *prevSimTime=0, 
                       Calendar    **calendar=0,
                       ESMC_CalKind_Flag *calkindflag=0,
                       int               *timeZone=0,
                       ESMC_I8      *advanceCount=0, 
                       int               *alarmCount=0,
                       ESMC_Direction    *direction=0);
 
     int advance(TimeInterval *timeStep=0,
                           char *ringingAlarmList1stElementPtr=0, 
                           char *ringingAlarmList2ndElementPtr=0, 
                           int  sizeofRingingAlarmList=0, 
                           int *ringingAlarmCount=0);
 
     // TMG3.4.1  after increment, for each alarm,
     //           calls Alarm::CheckActive()
 
     bool isStopTime(int *rc=0) const;           // TMG3.5.6
     int  stopTimeEnable(Time *stopTime=0); // WRF
     int  stopTimeDisable(void);                 // WRF
     bool isStopTimeEnabled(int *rc=0) const;    // WRF
 
     bool isDone(int *rc=0) const;           // TMG3.5.7
     bool isReverse(int *rc=0) const;        // TMG3.4.6
 
     int getNextTime(Time         *nextTime,
                               TimeInterval *timeStep=0);
 
     int getAlarm(int alarmnameLen, char *alarmname, Alarm **alarm);
 
     int getAlarmList(ESMC_AlarmList_Flag alarmlistflag,
                                char *AlarmList1stElementPtr, 
                                char *AlarmList2ndElementPtr,
                                int  sizeofAlarmList, 
                                int *alarmCount,
                                TimeInterval *timeStep=0);
 
     int syncToRealTime(void); // TMG3.4.5
     // (see Time::SyncToRealTime()
 
     // to suuport copying of the alarmList
     Clock& operator=(const Clock &);
 
     bool operator==(const Clock &) const;
     bool operator!=(const Clock &) const;
 
     // required methods inherited and overridden from the ESMC_Base class
 
     // for persistence/checkpointing
 
     // friend to restore state
     friend Clock *ESMCI_ClockReadRestart(int, const char*, int*);
     // save state
     int writeRestart(void) const;
 
     // internal validation
     int validate(const char *options=0) const;
 
     // for testing/debugging
     int print(const char *options=0) const;
 
     // native C++ constructors/destructors
     Clock(void);
     Clock(const Clock &clock);
     ~Clock(void);
 
  // < declare the rest of the public interface methods here >
 
     // friend function to allocate and initialize clock from heap
     friend Clock *ESMCI_ClockCreate(int, const char*, TimeInterval*,
                                  Time*, Time*, TimeInterval*,
                                  int*, Time*, int*);
   TODO: add overload for ESMC_R8  *runTimeStepCount
 
     // friend function to copy a clock
     friend Clock *ESMCI_ClockCreate(Clock*, int*);
 
     // friend function to de-allocate clock
     friend int ESMCI_ClockDestroy(Clock **);
 
     // friend to allocate and initialize alarm from heap
     //   (needs access to clock current time to initialize alarm ring time)
     friend Alarm *ESMCI_alarmCreate(int, const char*, Clock*, 
                                  Time*, TimeInterval*, Time*, 
                                  TimeInterval*, int*, Time*, bool*,
                                  bool*, int*);
 
     // friend function to copy an alarm
     friend Alarm *ESMCI_alarmCreate(Alarm*, int*);
 
     // friend to de-allocate alarm
     friend int ESMCI_alarmDestroy(Alarm **);
 \end{verbatim}{\sf PRIVATE MEMBER FUNCTIONS:}
\begin{verbatim}   private:
  // < declare private interface methods here >
 
     // called only by friend class Alarm
     int addAlarm(Alarm *alarm);    // alarmCreate(), alarmSet() (TMG 4.1, 4.2)
     int removeAlarm(Alarm *alarm); // alarmDestroy(), alarmSet()
 
     friend class Alarm;
 \end{verbatim}

%...............................................................
\setlength{\parskip}{\oldparskip}
\setlength{\parindent}{\oldparindent}
\setlength{\baselineskip}{\oldbaselineskip}
