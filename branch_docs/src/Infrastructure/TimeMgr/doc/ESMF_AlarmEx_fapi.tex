%                **** IMPORTANT NOTICE *****
% This LaTeX file has been automatically produced by ProTeX v. 1.1
% Any changes made to this file will likely be lost next time
% this file is regenerated from its source. Send questions 
% to Arlindo da Silva, dasilva@gsfc.nasa.gov
 
\setlength{\oldparskip}{\parskip}
\setlength{\parskip}{1.5ex}
\setlength{\oldparindent}{\parindent}
\setlength{\parindent}{0pt}
\setlength{\oldbaselineskip}{\baselineskip}
\setlength{\baselineskip}{11pt}
 
%--------------------- SHORT-HAND MACROS ----------------------
\def\bv{\begin{verbatim}}
\def\ev{\end{verbatim}}
\def\be{\begin{equation}}
\def\ee{\end{equation}}
\def\bea{\begin{eqnarray}}
\def\eea{\end{eqnarray}}
\def\bi{\begin{itemize}}
\def\ei{\end{itemize}}
\def\bn{\begin{enumerate}}
\def\en{\end{enumerate}}
\def\bd{\begin{description}}
\def\ed{\end{description}}
\def\({\left (}
\def\){\right )}
\def\[{\left [}
\def\]{\right ]}
\def\<{\left  \langle}
\def\>{\right \rangle}
\def\cI{{\cal I}}
\def\diag{\mathop{\rm diag}}
\def\tr{\mathop{\rm tr}}
%-------------------------------------------------------------

\markboth{Left}{Source File: ESMF\_AlarmEx.F90,  Date: Tue May  5 20:59:34 MDT 2020
}

 
%/////////////////////////////////////////////////////////////

 \begin{verbatim}
! !PROGRAM: ESMF_AlarmEx - Alarm examples
!
! !DESCRIPTION:
!
! This program shows an example of how to create, initialize, and process
! alarms associated with a clock.
!-----------------------------------------------------------------------------
#include "ESMF.h"

      ! ESMF Framework module
      use ESMF
      use ESMF_TestMod
      implicit none

      ! instantiate time_step, start, stop, and alarm times
      type(ESMF_TimeInterval) :: timeStep, alarmInterval
      type(ESMF_Time) :: alarmTime, startTime, stopTime

      ! instantiate a clock 
      type(ESMF_Clock) :: clock

      ! instantiate Alarm lists
      integer, parameter :: NUMALARMS = 2
      type(ESMF_Alarm) :: alarm(NUMALARMS)

      ! local variables for Get methods
      integer :: ringingAlarmCount  ! at any time step (0 to NUMALARMS)

      ! name, loop counter, result code
      character (len=ESMF_MAXSTR) :: name
      integer :: i, rc, result
 
\end{verbatim}
 
%/////////////////////////////////////////////////////////////

 \begin{verbatim}
      ! initialize ESMF framework
      call ESMF_Initialize(defaultCalKind=ESMF_CALKIND_GREGORIAN, &
        defaultlogfilename="AlarmEx.Log", &
        logkindflag=ESMF_LOGKIND_MULTI, rc=rc)
 
\end{verbatim}
 
%/////////////////////////////////////////////////////////////

  \subsubsection{Clock initialization}
 
   This example shows how to create and initialize an {\tt ESMF\_Clock}. 
%/////////////////////////////////////////////////////////////

 \begin{verbatim}
      ! initialize time interval to 1 day
      call ESMF_TimeIntervalSet(timeStep, d=1, rc=rc)
 
\end{verbatim}
 
%/////////////////////////////////////////////////////////////

 \begin{verbatim}
      ! initialize start time to 9/1/2003
      call ESMF_TimeSet(startTime, yy=2003, mm=9, dd=1, rc=rc)
 
\end{verbatim}
 
%/////////////////////////////////////////////////////////////

 \begin{verbatim}
      ! initialize stop time to 9/30/2003
      call ESMF_TimeSet(stopTime, yy=2003, mm=9, dd=30, rc=rc)
 
\end{verbatim}
 
%/////////////////////////////////////////////////////////////

 \begin{verbatim}
      ! create & initialize the clock with the above values
      clock = ESMF_ClockCreate(timeStep, startTime, stopTime=stopTime, &
                               name="The Clock", rc=rc)
 
\end{verbatim}
 
%/////////////////////////////////////////////////////////////

  \subsubsection{Alarm initialization}
 
   This example shows how to create and initialize two {\tt ESMF\_Alarms} and
   associate them with the clock. 
%/////////////////////////////////////////////////////////////

 \begin{verbatim}
      ! Initialize first alarm to be a one-shot on 9/15/2003 and associate
      ! it with the clock
      call ESMF_TimeSet(alarmTime, yy=2003, mm=9, dd=15, rc=rc)
 
\end{verbatim}
 
%/////////////////////////////////////////////////////////////

 \begin{verbatim}
      alarm(1) = ESMF_AlarmCreate(clock, &
         ringTime=alarmTime, name="Example alarm 1", rc=rc)
 
\end{verbatim}
 
%/////////////////////////////////////////////////////////////

 \begin{verbatim}
      ! Initialize second alarm to ring on a 1 week interval starting 9/1/2003
      ! and associate it with the clock
      call ESMF_TimeSet(alarmTime, yy=2003, mm=9, dd=1, rc=rc)
 
\end{verbatim}
 
%/////////////////////////////////////////////////////////////

 \begin{verbatim}
      call ESMF_TimeIntervalSet(alarmInterval, d=7, rc=rc)
 
\end{verbatim}
 
%/////////////////////////////////////////////////////////////

 \begin{verbatim}
      ! Alarm gets default name "Alarm002"
      alarm(2) = ESMF_AlarmCreate(clock=clock, ringTime=alarmTime, &
                                  ringInterval=alarmInterval, rc=rc)
 
\end{verbatim}
 
%/////////////////////////////////////////////////////////////

  \subsubsection{Clock advance and Alarm processing}
 
   This example shows how to advance an {\tt ESMF\_Clock} and process any 
   resulting ringing alarms. 
%/////////////////////////////////////////////////////////////

 \begin{verbatim}
      ! time step clock from start time to stop time
      do while (.not.ESMF_ClockIsStopTime(clock, rc=rc))
 
\end{verbatim}
 
%/////////////////////////////////////////////////////////////

 \begin{verbatim}
        ! perform time step and get the number of any ringing alarms
        call ESMF_ClockAdvance(clock, ringingAlarmCount=ringingAlarmCount, &
                               rc=rc)
 
\end{verbatim}
 
%/////////////////////////////////////////////////////////////

 \begin{verbatim}
        call ESMF_ClockPrint(clock, options="currTime string", rc=rc)
 
\end{verbatim}
 
%/////////////////////////////////////////////////////////////

 \begin{verbatim}
        ! check if alarms are ringing
        if (ringingAlarmCount > 0) then
          print *, "number of ringing alarms = ", ringingAlarmCount

          do i = 1, NUMALARMS
            if (ESMF_AlarmIsRinging(alarm(i), rc=rc)) then
 
\end{verbatim}
 
%/////////////////////////////////////////////////////////////

 \begin{verbatim}
              call ESMF_AlarmGet(alarm(i), name=name, rc=rc)
              print *, trim(name), " is ringing!"
 
\end{verbatim}
 
%/////////////////////////////////////////////////////////////

 \begin{verbatim}
              ! after processing alarm, turn it off
              call ESMF_AlarmRingerOff(alarm(i), rc=rc)
 
\end{verbatim}
 
%/////////////////////////////////////////////////////////////

 \begin{verbatim}
            end if ! this alarm is ringing
          end do ! each ringing alarm
        endif ! ringing alarms
      end do ! timestep clock
 
\end{verbatim}
 
%/////////////////////////////////////////////////////////////

  \subsubsection{Alarm and Clock destruction}
 
   This example shows how to destroy {\tt ESMF\_Alarms} and {\tt ESMF\_Clocks}. 
%/////////////////////////////////////////////////////////////

 \begin{verbatim}
      call ESMF_AlarmDestroy(alarm(1), rc=rc)
 
\end{verbatim}
 
%/////////////////////////////////////////////////////////////

 \begin{verbatim}
      call ESMF_AlarmDestroy(alarm(2), rc=rc)
 
\end{verbatim}
 
%/////////////////////////////////////////////////////////////

 \begin{verbatim}
      call ESMF_ClockDestroy(clock, rc=rc)
 
\end{verbatim}
 
%/////////////////////////////////////////////////////////////

 \begin{verbatim}
      ! finalize ESMF framework
      call ESMF_Finalize(rc=rc)
 
\end{verbatim}
 
%/////////////////////////////////////////////////////////////

 \begin{verbatim}
      end program ESMF_AlarmEx
 
\end{verbatim}

%...............................................................
\setlength{\parskip}{\oldparskip}
\setlength{\parindent}{\oldparindent}
\setlength{\baselineskip}{\oldbaselineskip}
