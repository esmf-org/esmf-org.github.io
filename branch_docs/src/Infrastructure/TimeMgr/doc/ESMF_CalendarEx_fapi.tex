%                **** IMPORTANT NOTICE *****
% This LaTeX file has been automatically produced by ProTeX v. 1.1
% Any changes made to this file will likely be lost next time
% this file is regenerated from its source. Send questions 
% to Arlindo da Silva, dasilva@gsfc.nasa.gov
 
\setlength{\oldparskip}{\parskip}
\setlength{\parskip}{1.5ex}
\setlength{\oldparindent}{\parindent}
\setlength{\parindent}{0pt}
\setlength{\oldbaselineskip}{\baselineskip}
\setlength{\baselineskip}{11pt}
 
%--------------------- SHORT-HAND MACROS ----------------------
\def\bv{\begin{verbatim}}
\def\ev{\end{verbatim}}
\def\be{\begin{equation}}
\def\ee{\end{equation}}
\def\bea{\begin{eqnarray}}
\def\eea{\end{eqnarray}}
\def\bi{\begin{itemize}}
\def\ei{\end{itemize}}
\def\bn{\begin{enumerate}}
\def\en{\end{enumerate}}
\def\bd{\begin{description}}
\def\ed{\end{description}}
\def\({\left (}
\def\){\right )}
\def\[{\left [}
\def\]{\right ]}
\def\<{\left  \langle}
\def\>{\right \rangle}
\def\cI{{\cal I}}
\def\diag{\mathop{\rm diag}}
\def\tr{\mathop{\rm tr}}
%-------------------------------------------------------------

\markboth{Left}{Source File: ESMF\_CalendarEx.F90,  Date: Tue May  5 20:59:34 MDT 2020
}

 
%/////////////////////////////////////////////////////////////

 \begin{verbatim}
! !PROGRAM: ESMF_CalendarEx - Calendar creation examples
!
! !DESCRIPTION:
!
! This program shows examples of how to create different calendar kinds
!-----------------------------------------------------------------------------
#include "ESMF.h"

      ! ESMF Framework module
      use ESMF
      use ESMF_TestMod
      implicit none

      ! instantiate calendars
      type(ESMF_Calendar) :: gregorianCalendar
      type(ESMF_Calendar) :: julianDayCalendar
      type(ESMF_Calendar) :: marsCalendar

      ! local variables for Get methods
      integer :: sols
      integer(ESMF_KIND_I8) :: dl
      type(ESMF_Time) :: time, marsTime
      type(ESMF_TimeInterval) :: marsTimeStep

      ! return code
      integer:: rc
 
\end{verbatim}
 
%/////////////////////////////////////////////////////////////

 \begin{verbatim}
      ! initialize ESMF framework
      call ESMF_Initialize(defaultlogfilename="CalendarEx.Log", &
                    logkindflag=ESMF_LOGKIND_MULTI, rc=rc)
 
\end{verbatim}
 
%/////////////////////////////////////////////////////////////

  \subsubsection{Calendar creation}
 
   This example shows how to create three {\tt ESMF\_Calendars}. 
%/////////////////////////////////////////////////////////////

 \begin{verbatim}
      ! create a Gregorian calendar
      gregorianCalendar = ESMF_CalendarCreate(ESMF_CALKIND_GREGORIAN, &
                                              name="Gregorian", rc=rc)
 
\end{verbatim}
 
%/////////////////////////////////////////////////////////////

 \begin{verbatim}
      ! create a Julian Day calendar
      julianDayCalendar = ESMF_CalendarCreate(ESMF_CALKIND_JULIANDAY, &
                                              name="JulianDay", rc=rc)
 
\end{verbatim}
 
%/////////////////////////////////////////////////////////////

 \begin{verbatim}
      ! create a Custom calendar for the planet Mars
      ! 1 Mars solar day = 24 hours, 39 minutes, 35 seconds = 88775 seconds
      ! 1 Mars solar year = 668.5921 Mars solar days = 668 5921/10000 sols/year
      ! http://www.giss.nasa.gov/research/briefs/allison_02
      ! http://www.giss.nasa.gov/tools/mars24/help/notes.html
      marsCalendar = ESMF_CalendarCreate(secondsPerDay=88775, &
                                         daysPerYear=668, &
                                         daysPerYearDn=5921, &
                                         daysPerYearDd=10000, &
                                         name="MarsCalendar", rc=rc)
 
\end{verbatim}
 
%/////////////////////////////////////////////////////////////

  \subsubsection{Calendar comparison}
 
   This example shows how to compare an {\tt ESMF\_Calendar} with a known
   calendar kind. 
%/////////////////////////////////////////////////////////////

 \begin{verbatim}
      ! compare calendar kind against a known type
      if (gregorianCalendar == ESMF_CALKIND_GREGORIAN) then
        print *, "gregorianCalendar is of type ESMF_CALKIND_GREGORIAN."
      else
        print *, "gregorianCalendar is not of type ESMF_CALKIND_GREGORIAN."
      end if
 
\end{verbatim}
 
%/////////////////////////////////////////////////////////////

  \subsubsection{Time conversion between Calendars}
 
   This example shows how to convert a time from one {\tt ESMF\_Calendar}
   to another. 
%/////////////////////////////////////////////////////////////

 \begin{verbatim}
      call ESMF_TimeSet(time, yy=2004, mm=4, dd=17, &
                        calendar=gregorianCalendar, rc=rc)
 
\end{verbatim}
 
%/////////////////////////////////////////////////////////////

 \begin{verbatim}
      ! switch time's calendar to perform conversion
      call ESMF_TimeSet(time, calendar=julianDayCalendar, rc=rc)
 
\end{verbatim}
 
%/////////////////////////////////////////////////////////////

 \begin{verbatim}
      call ESMF_TimeGet(time, d_i8=dl, rc=rc)
      print *, "Gregorian date 2004/4/17 is ", dl, &
               " days in the Julian Day calendar."
 
\end{verbatim}
 
%/////////////////////////////////////////////////////////////

  \subsubsection{Add a time interval to a time on a Calendar}
 
   This example shows how to increment a time using a custom {\tt ESMF\_Calendar}. 
%/////////////////////////////////////////////////////////////

 \begin{verbatim}
      ! Set a time to Mars solar year 3, sol 100
      call ESMF_TimeSet(marsTime, yy=3, d=100, &
                        calendar=marsCalendar, rc=rc)
 
\end{verbatim}
 
%/////////////////////////////////////////////////////////////

 \begin{verbatim}
      ! Set a 1 solar year time step
      call ESMF_TimeIntervalSet(marsTimeStep, yy=1, rc=rc)
 
\end{verbatim}
 
%/////////////////////////////////////////////////////////////

 \begin{verbatim}
      ! Perform the increment
      marsTime = marsTime + marsTimeStep
 
\end{verbatim}
 
%/////////////////////////////////////////////////////////////

 \begin{verbatim}
      ! Get the result in sols (2774 = (3+1)*668.5921 + 100)
      call ESMF_TimeGet(marsTime, d=sols, rc=rc)
      print *, "For Mars, 3 solar years, 100 sols + 1 solar year = ", &
                sols, "sols."
 
\end{verbatim}
 
%/////////////////////////////////////////////////////////////

  \subsubsection{Calendar destruction}
 
   This example shows how to destroy three {\tt ESMF\_Calendars}. 
%/////////////////////////////////////////////////////////////

 \begin{verbatim}
      call ESMF_CalendarDestroy(julianDayCalendar, rc=rc)
 
\end{verbatim}
 
%/////////////////////////////////////////////////////////////

 \begin{verbatim}
      call ESMF_CalendarDestroy(gregorianCalendar, rc=rc)
 
\end{verbatim}
 
%/////////////////////////////////////////////////////////////

 \begin{verbatim}
      call ESMF_CalendarDestroy(marsCalendar, rc=rc)
 
\end{verbatim}
 
%/////////////////////////////////////////////////////////////

 \begin{verbatim}
      ! finalize ESMF framework
      call ESMF_Finalize(rc=rc)
 
\end{verbatim}
 
%/////////////////////////////////////////////////////////////

 \begin{verbatim}
      end program ESMF_CalendarEx
 
\end{verbatim}

%...............................................................
\setlength{\parskip}{\oldparskip}
\setlength{\parindent}{\oldparindent}
\setlength{\baselineskip}{\oldbaselineskip}
