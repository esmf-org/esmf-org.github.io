%                **** IMPORTANT NOTICE *****
% This LaTeX file has been automatically produced by ProTeX v. 1.1
% Any changes made to this file will likely be lost next time
% this file is regenerated from its source. Send questions 
% to Arlindo da Silva, dasilva@gsfc.nasa.gov
 
\setlength{\oldparskip}{\parskip}
\setlength{\parskip}{1.5ex}
\setlength{\oldparindent}{\parindent}
\setlength{\parindent}{0pt}
\setlength{\oldbaselineskip}{\baselineskip}
\setlength{\baselineskip}{11pt}
 
%--------------------- SHORT-HAND MACROS ----------------------
\def\bv{\begin{verbatim}}
\def\ev{\end{verbatim}}
\def\be{\begin{equation}}
\def\ee{\end{equation}}
\def\bea{\begin{eqnarray}}
\def\eea{\end{eqnarray}}
\def\bi{\begin{itemize}}
\def\ei{\end{itemize}}
\def\bn{\begin{enumerate}}
\def\en{\end{enumerate}}
\def\bd{\begin{description}}
\def\ed{\end{description}}
\def\({\left (}
\def\){\right )}
\def\[{\left [}
\def\]{\right ]}
\def\<{\left  \langle}
\def\>{\right \rangle}
\def\cI{{\cal I}}
\def\diag{\mathop{\rm diag}}
\def\tr{\mathop{\rm tr}}
%-------------------------------------------------------------

\markboth{Left}{Source File: ESMF\_Clock.F90,  Date: Tue May  5 20:59:34 MDT 2020
}

 
%/////////////////////////////////////////////////////////////
\subsubsection [ESMF\_ClockAssignment(=)] {ESMF\_ClockAssignment(=) - Assign a Clock to another Clock}


  
\bigskip{\sf INTERFACE:}
\begin{verbatim}       interface assignment(=)
       clock1 = clock2\end{verbatim}{\em ARGUMENTS:}
\begin{verbatim}       type(ESMF_Clock) :: clock1
       type(ESMF_Clock) :: clock2\end{verbatim}
{\sf STATUS:}
   \begin{itemize}
   \item\apiStatusCompatibleVersion{5.2.0r}
   \end{itemize}
  
{\sf DESCRIPTION:\\ }


       Assign {\tt clock1} as an alias to the same {\tt ESMF\_Clock} object in
       memory as {\tt clock2}. If {\tt clock2} is invalid, then {\tt clock1}
       will be equally invalid after the assignment.
  
       The arguments are:
       \begin{description}
       \item[clock1]
            The {\tt ESMF\_Clock} object on the left hand side of the
            assignment.
       \item[clock2]
            The {\tt ESMF\_Clock} object on the right hand side of the
            assignment.
       \end{description}
   
%/////////////////////////////////////////////////////////////
 
\mbox{}\hrulefill\ 
 
\subsubsection [ESMF\_ClockOperator(==)] {ESMF\_ClockOperator(==) - Test if Clock 1 is equal to Clock 2}


  
\bigskip{\sf INTERFACE:}
\begin{verbatim}       interface operator(==)
       if (clock1 == clock2) then ... endif
                    OR
       result = (clock1 == clock2)\end{verbatim}{\em RETURN VALUE:}
\begin{verbatim}       logical :: result\end{verbatim}{\em ARGUMENTS:}
\begin{verbatim}       type(ESMF_Clock), intent(in) :: clock1
       type(ESMF_Clock), intent(in) :: clock2\end{verbatim}
{\sf DESCRIPTION:\\ }


       Overloads the (==) operator for the {\tt ESMF\_Clock} class.
       Compare two clocks for equality; return {\tt .true.} if equal,
       {\tt .false.} otherwise. Comparison is based on IDs, which are distinct
       for newly created clocks and identical for clocks created as copies.
  
       If either side of the equality test is not in the
       {\tt ESMF\_INIT\_CREATED} status an error will be logged. However, this
       does not affect the return value, which is {\tt .true.} when both
       sides are in the {\em same} status, and {\tt .false.} otherwise.
  
       The arguments are:
       \begin{description}
       \item[clock1]
            The {\tt ESMF\_Clock} object on the left hand side of the equality
            operation.
       \item[clock2]
            The {\tt ESMF\_Clock} object on the right hand side of the equality
            operation.
       \end{description}
   
%/////////////////////////////////////////////////////////////
 
\mbox{}\hrulefill\ 
 
\subsubsection [ESMF\_ClockOperator(/=)] {ESMF\_ClockOperator(/=) - Test if Clock 1 is not equal to Clock 2}


  
\bigskip{\sf INTERFACE:}
\begin{verbatim}       interface operator(/=)
       if (clock1 /= clock2) then ... endif
                    OR
       result = (clock1 /= clock2)\end{verbatim}{\em RETURN VALUE:}
\begin{verbatim}       logical :: result\end{verbatim}{\em ARGUMENTS:}
\begin{verbatim}       type(ESMF_Clock), intent(in) :: clock1
       type(ESMF_Clock), intent(in) :: clock2\end{verbatim}
{\sf DESCRIPTION:\\ }


       Overloads the (/=) operator for the {\tt ESMF\_Clock} class.
       Compare two clocks for inequality; return {\tt .true.} if not equal,
       {\tt .false.} otherwise. Comparison is based on IDs, which are distinct
       for newly created clocks and identical for clocks created as copies.
  
       If either side of the equality test is not in the
       {\tt ESMF\_INIT\_CREATED} status an error will be logged. However, this
       does not affect the return value, which is {\tt .true.} when both sides
       are {\em not} in the {\em same} status, and {\tt .false.} otherwise.
  
       The arguments are:
       \begin{description}
       \item[clock1]
            The {\tt ESMF\_Clock} object on the left hand side of the
            non-equality operation.
       \item[clock2]
            The {\tt ESMF\_Clock} object on the right hand side of the
            non-equality operation.
       \end{description}
   
%/////////////////////////////////////////////////////////////
 
\mbox{}\hrulefill\ 
 
\subsubsection [ESMF\_ClockAdvance] {ESMF\_ClockAdvance - Advance a Clock's current time by one time step}


 
\bigskip{\sf INTERFACE:}
\begin{verbatim}       subroutine ESMF_ClockAdvance(clock, &
         timeStep, ringingAlarmList, ringingAlarmCount, rc)
 \end{verbatim}{\em ARGUMENTS:}
\begin{verbatim}       type(ESMF_Clock),        intent(inout)         :: clock
 -- The following arguments require argument keyword syntax (e.g. rc=rc). --
       type(ESMF_TimeInterval), intent(in),  optional :: timeStep
       type(ESMF_Alarm),        intent(out), optional :: ringingAlarmList(:)
       integer,                 intent(out), optional :: ringingAlarmCount
       integer,                 intent(out), optional :: rc\end{verbatim}
{\sf STATUS:}
   \begin{itemize}
   \item\apiStatusCompatibleVersion{5.2.0r}
   \end{itemize}
  
{\sf DESCRIPTION:\\ }


       \begin{sloppypar}
       Advances the {\tt clock}'s current time by one time step:  either the
       {\tt clock}'s, or the passed-in {\tt timeStep} (see below).  When the
       {\tt clock} is in {\tt ESMF\_DIRECTION\_FORWARD} (default), this method
       adds the {\tt timeStep} to the {\tt clock}'s current time.
       In {\tt ESMF\_DIRECTION\_REVERSE}, {\tt timeStep} is subtracted from the
       current time.  In either case, {\tt timeStep} can be positive or negative.
       See the "direction" argument in method {\tt ESMF\_ClockSet()}.
       {\tt ESMF\_ClockAdvance()} optionally returns a list and number of ringing
       {\tt ESMF\_Alarm}s.  See also method {\tt ESMF\_ClockGetRingingAlarms()}.
       \end{sloppypar}
  
       The arguments are:
       \begin{description}
       \item[clock]
            The object instance to advance.
       \item[{[timeStep]}]
            Time step is performed with given timeStep, instead of
            the {\tt ESMF\_Clock}'s.  Does not replace the {\tt ESMF\_Clock}'s
            timeStep; use {\tt ESMF\_ClockSet(clock, timeStep, ...)} for
            this purpose.  Supports applications with variable time steps.
            timeStep can be positive or negative.
       \item[{[ringingAlarmList]}]
            Returns the array of alarms that are ringing after the
            time step.
       \item[{[ringingAlarmCount]}]
            The number of alarms ringing after the time step.
       \item[{[rc]}]
            Return code; equals {\tt ESMF\_SUCCESS} if there are no errors.
       \end{description}
   
%/////////////////////////////////////////////////////////////
 
\mbox{}\hrulefill\ 
 
\subsubsection [ESMF\_ClockCreate] {ESMF\_ClockCreate - Create a new ESMF Clock}


 
\bigskip{\sf INTERFACE:}
\begin{verbatim}       ! Private name; call using ESMF_ClockCreate()
       function ESMF_ClockCreateNew(timeStep, startTime, &
         stopTime, runDuration, runTimeStepCount, refTime, name, rc)
 \end{verbatim}{\em RETURN VALUE:}
\begin{verbatim}       type(ESMF_Clock) :: ESMF_ClockCreateNew
 \end{verbatim}{\em ARGUMENTS:}
\begin{verbatim}       type(ESMF_TimeInterval), intent(in)            :: timeStep
       type(ESMF_Time),         intent(in)            :: startTime
 -- The following arguments require argument keyword syntax (e.g. rc=rc). --
       type(ESMF_Time),         intent(in),  optional :: stopTime
       type(ESMF_TimeInterval), intent(in),  optional :: runDuration
       integer,                 intent(in),  optional :: runTimeStepCount
       type(ESMF_Time),         intent(in),  optional :: refTime
       character (len=*),       intent(in),  optional :: name
       integer,                 intent(out), optional :: rc
 \end{verbatim}
{\sf STATUS:}
   \begin{itemize}
   \item\apiStatusCompatibleVersion{5.2.0r}
   \end{itemize}
  
{\sf DESCRIPTION:\\ }


       Creates and sets the initial values in a new {\tt ESMF\_Clock}.
  
       The arguments are:
       \begin{description}
       \item[timeStep]
            The {\tt ESMF\_Clock}'s time step interval, which can be
            positive or negative.
       \item[startTime]
            The {\tt ESMF\_Clock}'s starting time.  Can be less than or
            or greater than stopTime, depending on a positive or negative
            timeStep, respectively, and whether a stopTime is specified;
            see below.
       \item[{[stopTime]}]
            The {\tt ESMF\_Clock}'s stopping time.  Can be greater than or
            less than the startTime, depending on a positive or negative
            timeStep, respectively.  If neither stopTime, runDuration, nor
            runTimeStepCount is specified, clock runs "forever"; user must
            use other means to know when to stop (e.g. ESMF\_Alarm or
            ESMF\_ClockGet(clock, currTime)).  Mutually exclusive with
            runDuration and runTimeStepCount.
       \item[{[runDuration]}]
            Alternative way to specify {\tt ESMF\_Clock}'s stopping time;
               stopTime = startTime + runDuration.
            Can be positive or negative, consistent with the timeStep's sign.
            Mutually exclusive with stopTime and runTimeStepCount.
       \item[{[runTimeStepCount]}]
            Alternative way to specify {\tt ESMF\_Clock}'s stopping time;
               stopTime = startTime + (runTimeStepCount * timeStep).
            stopTime can be before startTime if timeStep is negative.
            Mutually exclusive with stopTime and runDuration.
       \item[{[refTime]}]
            The {\tt ESMF\_Clock}'s reference time.  Provides reference point
            for simulation time (see currSimTime in ESMF\_ClockGet() below).
       \item[{[name]}]
            The name for the newly created clock.  If not specified, a
            default unique name will be generated: "ClockNNN" where NNN
            is a unique sequence number from 001 to 999.
       \item[{[rc]}]
            Return code; equals {\tt ESMF\_SUCCESS} if there are no errors.
       \end{description}
   
%/////////////////////////////////////////////////////////////
 
\mbox{}\hrulefill\ 
 
\subsubsection [ESMF\_ClockCreate] {ESMF\_ClockCreate - Create a copy of an existing ESMF Clock}


 
\bigskip{\sf INTERFACE:}
\begin{verbatim}       ! Private name; call using ESMF_ClockCreate()
       function ESMF_ClockCreateCopy(clock, rc)
 \end{verbatim}{\em RETURN VALUE:}
\begin{verbatim}       type(ESMF_Clock) :: ESMF_ClockCreateCopy
 \end{verbatim}{\em ARGUMENTS:}
\begin{verbatim}       type(ESMF_Clock), intent(in)            :: clock
 -- The following arguments require argument keyword syntax (e.g. rc=rc). --
       integer,          intent(out), optional :: rc
 \end{verbatim}
{\sf STATUS:}
   \begin{itemize}
   \item\apiStatusCompatibleVersion{5.2.0r}
   \end{itemize}
  
{\sf DESCRIPTION:\\ }


       Creates a deep copy of a given {\tt ESMF\_Clock}, but does not copy its
       list of {\tt ESMF\_Alarm}s (pointers), since an {\tt ESMF\_Alarm} can only
       be associated with one {\tt ESMF\_Clock}.  Hence, the returned
       {\tt ESMF\_Clock} copy has no associated {\tt ESMF\_Alarm}s, the same as
       with a newly created {\tt ESMF\_Clock}.  If desired, new
       {\tt ESMF\_Alarm}s must be created and associated with this copied
       {\tt ESMF\_Clock} via {\tt ESMF\_AlarmCreate()}, or existing
       {\tt ESMF\_Alarm}s must be re-associated with this copied
       {\tt ESMF\_Clock} via {\tt ESMF\_AlarmSet(...clock=...)}.
  
       The arguments are:
       \begin{description}
       \item[clock]
          The {\tt ESMF\_Clock} to copy.
       \item[{[rc]}]
          Return code; equals {\tt ESMF\_SUCCESS} if there are no errors.
       \end{description}
   
%/////////////////////////////////////////////////////////////
 
\mbox{}\hrulefill\ 
 
\subsubsection [ESMF\_ClockDestroy] {ESMF\_ClockDestroy - Release resources associated with a Clock}


  
\bigskip{\sf INTERFACE:}
\begin{verbatim}       subroutine ESMF_ClockDestroy(clock, rc)\end{verbatim}{\em ARGUMENTS:}
\begin{verbatim}       type(ESMF_Clock), intent(inout)          :: clock
 -- The following arguments require argument keyword syntax (e.g. rc=rc). --
       integer,          intent(out),  optional :: rc\end{verbatim}
{\sf STATUS:}
   \begin{itemize}
   \item\apiStatusCompatibleVersion{5.2.0r}
   \end{itemize}
  
{\sf DESCRIPTION:\\ }


       \begin{sloppypar}
       Releases resources associated with this {\tt ESMF\_Clock}.  This releases
       the list of associated {\tt ESMF\_Alarm}s (pointers), but not the
       {\tt ESMF\_Alarm}s themselves; the user must explicitly call
       {\tt ESMF\_AlarmDestroy()} on each {\tt ESMF\_Alarm} to release its
       resources.  {\tt ESMF\_ClockDestroy()} and corresponding
       {\tt ESMF\_AlarmDestroy()}s can be called in either order.
       \end{sloppypar}
  
       \begin{sloppypar}
       If {\tt ESMF\_ClockDestroy()} is called before {\tt ESMF\_AlarmDestroy()},
       any {\tt ESMF\_Alarm}s that were in the {\tt ESMF\_Clock}'s list will
       no longer be associated with any {\tt ESMF\_Clock}.  If desired,
       these "orphaned" {\tt ESMF\_Alarm}s can be associated with a different
       {\tt ESMF\_Clock} via a call to {\tt ESMF\_AlarmSet(...clock=...)}.
       \end{sloppypar}
  
       The arguments are:
       \begin{description}
       \item[clock]
         Release resources associated with this {\tt ESMF\_Clock} and mark the
         object as invalid.  It is an error to pass this object into any other
         routines after being destroyed.
       \item[[rc]]
         Return code; equals {\tt ESMF\_SUCCESS} if there are no errors.
       \end{description}
   
%/////////////////////////////////////////////////////////////
 
\mbox{}\hrulefill\ 
 
\subsubsection [ESMF\_ClockGet] {ESMF\_ClockGet - Get a Clock's properties}


 
\bigskip{\sf INTERFACE:}
\begin{verbatim}       subroutine ESMF_ClockGet(clock, &
         timeStep, startTime, stopTime, &
         runDuration, runTimeStepCount, refTime, currTime, prevTime, &
         currSimTime, prevSimTime, calendar, calkindflag, timeZone, &
         advanceCount, alarmCount, direction, name, rc)
 \end{verbatim}{\em ARGUMENTS:}
\begin{verbatim}       type(ESMF_Clock),        intent(in)            :: clock
 -- The following arguments require argument keyword syntax (e.g. rc=rc). --
       type(ESMF_TimeInterval), intent(out), optional :: timeStep
       type(ESMF_Time),         intent(out), optional :: startTime
       type(ESMF_Time),         intent(out), optional :: stopTime
       type(ESMF_TimeInterval), intent(out), optional :: runDuration
       real(ESMF_KIND_R8),      intent(out), optional :: runTimeStepCount
       type(ESMF_Time),         intent(out), optional :: refTime
       type(ESMF_Time),         intent(out), optional :: currTime
       type(ESMF_Time),         intent(out), optional :: prevTime
       type(ESMF_TimeInterval), intent(out), optional :: currSimTime
       type(ESMF_TimeInterval), intent(out), optional :: prevSimTime
       type(ESMF_Calendar),     intent(out), optional :: calendar
       type(ESMF_CalKind_Flag), intent(out), optional :: calkindflag
       integer,                 intent(out), optional :: timeZone
       integer(ESMF_KIND_I8),   intent(out), optional :: advanceCount
       integer,                 intent(out), optional :: alarmCount
       type(ESMF_Direction_Flag),    intent(out), optional :: direction
       character (len=*),       intent(out), optional :: name
       integer,                 intent(out), optional :: rc
 \end{verbatim}
{\sf STATUS:}
   \begin{itemize}
   \item\apiStatusCompatibleVersion{5.2.0r}
   \end{itemize}
  
{\sf DESCRIPTION:\\ }


       Gets one or more of the properties of an {\tt ESMF\_Clock}.
  
       The arguments are:
       \begin{description}
       \item[clock]
            The object instance to query.
       \item[{[timeStep]}]
            The {\tt ESMF\_Clock}'s time step interval.
       \item[{[startTime]}]
            The {\tt ESMF\_Clock}'s starting time.
       \item[{[stopTime]}]
            The {\tt ESMF\_Clock}'s stopping time.
       \item[{[runDuration]}]
            Alternative way to get {\tt ESMF\_Clock}'s stopping time;
               runDuration = stopTime - startTime.
       \item[{[runTimeStepCount]}]
            Alternative way to get {\tt ESMF\_Clock}'s stopping time;
               runTimeStepCount = (stopTime - startTime) / timeStep.
       \item[{[refTime]}]
            The {\tt ESMF\_Clock}'s reference time.
       \item[{[currTime]}]
            The {\tt ESMF\_Clock}'s current time.
       \item[{[prevTime]}]
            The {\tt ESMF\_Clock}'s previous time.  Equals currTime at
            the previous time step.
       \item[{[currSimTime]}]
            The current simulation time (currTime - refTime).
       \item[{[prevSimTime]}]
            The previous simulation time.  Equals currSimTime at
            the previous time step.
       \item[{[calendar]}]
            The {\tt Calendar} on which all the {\tt Clock}'s times are defined.
       \item[{[calkindflag]}]
            The {\tt CalKind\_Flag} on which all the {\tt Clock}'s times are
            defined.
       \item[{[timeZone]}]
            The timezone within which all the {\tt Clock}'s times are defined.
       \item[{[advanceCount]}]
            \begin{sloppypar}
            The number of times the {\tt ESMF\_Clock} has been advanced.
            Increments in {\tt ESMF\_DIRECTION\_FORWARD} and decrements in
            {\tt ESMF\_DIRECTION\_REVERSE}; see "direction" argument below and
            in {\tt ESMF\_ClockSet()}.
            \end{sloppypar}
       \item[{[alarmCount]}]
            The number of {\tt ESMF\_Alarm}s in the {\tt ESMF\_Clock}'s
            {\tt ESMF\_Alarm} list.
       \item[{[direction]}]
            The {\tt ESMF\_Clock}'s time stepping direction.  See also
            {\tt ESMF\_ClockIsReverse()}, an alternative for convenient use in
            "if" and "do while" constructs.
       \item[{[name]}]
            The name of this clock.
       \item[{[rc]}]
            Return code; equals {\tt ESMF\_SUCCESS} if there are no errors.
       \end{description}
   
%/////////////////////////////////////////////////////////////
 
\mbox{}\hrulefill\ 
 
\subsubsection [ESMF\_ClockGetAlarm] {ESMF\_ClockGetAlarm - Get an Alarm in a Clock's Alarm list}


 
\bigskip{\sf INTERFACE:}
\begin{verbatim}       subroutine ESMF_ClockGetAlarm(clock, alarmname, alarm, &
         rc)
 \end{verbatim}{\em ARGUMENTS:}
\begin{verbatim}       type(ESMF_Clock),  intent(in)            :: clock
       character (len=*), intent(in)            :: alarmname
       type(ESMF_Alarm),  intent(out)           :: alarm
 -- The following arguments require argument keyword syntax (e.g. rc=rc). --
       integer,           intent(out), optional :: rc
 \end{verbatim}
{\sf STATUS:}
   \begin{itemize}
   \item\apiStatusCompatibleVersion{5.2.0r}
   \end{itemize}
  
{\sf DESCRIPTION:\\ }


       Gets the {\tt alarm} whose name is the value of alarmname in the
       {\tt clock}'s {\tt ESMF\_Alarm} list.
  
       The arguments are:
       \begin{description}
       \item[clock]
            The object instance to get the {\tt ESMF\_Alarm} from.
       \item[alarmname]
            The name of the desired {\tt ESMF\_Alarm}.
       \item[alarm]
            The desired alarm.
       \item[{[rc]}]
            Return code; equals {\tt ESMF\_SUCCESS} if there are no errors.
       \end{description}
   
%/////////////////////////////////////////////////////////////
 
\mbox{}\hrulefill\ 
 
\subsubsection [ESMF\_ClockGetAlarmList] {ESMF\_ClockGetAlarmList - Get a list of Alarms from a Clock}


 
\bigskip{\sf INTERFACE:}
\begin{verbatim}       subroutine ESMF_ClockGetAlarmList(clock, alarmlistflag, &
         timeStep, alarmList, alarmCount, rc)
 \end{verbatim}{\em ARGUMENTS:}
\begin{verbatim}       type(ESMF_Clock),          intent(in)            :: clock
       type(ESMF_AlarmList_Flag), intent(in)            :: alarmlistflag
 -- The following arguments require argument keyword syntax (e.g. rc=rc). --
       type(ESMF_TimeInterval),   intent(in),  optional :: timeStep
       type(ESMF_Alarm),          intent(out), optional :: alarmList(:)
       integer,                   intent(out), optional :: alarmCount
       integer,                   intent(out), optional :: rc\end{verbatim}
{\sf STATUS:}
   \begin{itemize}
   \item\apiStatusCompatibleVersion{5.2.0r}
   \end{itemize}
  
{\sf DESCRIPTION:\\ }


       Gets the {\tt clock}'s list of alarms and/or number of alarms.
  
       The arguments are:
       \begin{description}
       \item[clock]
            The object instance from which to get an {\tt ESMF\_Alarm} list
            and/or count of {\tt ESMF\_Alarm}s.
       \item[alarmlistflag]
            The kind of list to get:
  
              {\tt ESMF\_ALARMLIST\_ALL} :
                  Returns the {\tt ESMF\_Clock}'s entire list of alarms.
  
              {\tt ESMF\_ALARMLIST\_NEXTRINGING} :
                  Return only those alarms that will ring upon the next
                  {\tt clock} time step.  Can optionally specify argument
                  {\tt timeStep} (see below) to use instead of the {\tt clock}'s.
                  See also method {\tt ESMF\_AlarmWillRingNext()} for checking a
                  single alarm.
  
              {\tt ESMF\_ALARMLIST\_PREVRINGING} :
                  \begin{sloppypar}
                  Return only those alarms that were ringing on the previous
                  {\tt ESMF\_Clock} time step.  See also method
                  {\tt ESMF\_AlarmWasPrevRinging()} for checking a single alarm.
                  \end{sloppypar}
  
              {\tt ESMF\_ALARMLIST\_RINGING} :
                  Returns only those {\tt clock} alarms that are currently
                  ringing.  See also method {\tt ESMF\_ClockAdvance()} for
                  getting the list of ringing alarms subsequent to a time step.
                  See also method {\tt ESMF\_AlarmIsRinging()} for checking a
                  single alarm.
       \item[{[timeStep]}]
            \begin{sloppypar}
            Optional time step to be used instead of the {\tt clock}'s.
            Only used with {\tt ESMF\_ALARMLIST\_NEXTRINGING alarmlistflag}
            (see above); ignored if specified with other {\tt alarmlistflags}.
            \end{sloppypar}
       \item[{[alarmList]}]
            The array of returned alarms.  If given, the array must be large
            enough to hold the number of alarms of the specified
            {\tt alarmlistflag} in the specified {\tt clock}.
       \item[{[alarmCount]}]
            If specified, returns the number of {\tt ESMF\_Alarm}s of the
            specified {\tt alarmlistflag} in the specified {\tt clock}.
       \item[{[rc]}]
            Return code; equals {\tt ESMF\_SUCCESS} if there are no errors.
       \end{description}
   
%/////////////////////////////////////////////////////////////
 
\mbox{}\hrulefill\ 
 
\subsubsection [ESMF\_ClockGetNextTime] {ESMF\_ClockGetNextTime - Calculate a Clock's next time}


 
\bigskip{\sf INTERFACE:}
\begin{verbatim}       subroutine ESMF_ClockGetNextTime(clock, nextTime, &
         timeStep, rc)
 \end{verbatim}{\em ARGUMENTS:}
\begin{verbatim}       type(ESMF_Clock),        intent(in)            :: clock
       type(ESMF_Time),         intent(out)           :: nextTime
 -- The following arguments require argument keyword syntax (e.g. rc=rc). --
       type(ESMF_TimeInterval), intent(in),  optional :: timeStep
       integer,                 intent(out), optional :: rc
 \end{verbatim}
{\sf STATUS:}
   \begin{itemize}
   \item\apiStatusCompatibleVersion{5.2.0r}
   \end{itemize}
  
{\sf DESCRIPTION:\\ }


       Calculates what the next time of the {\tt clock} will be, based on
       the {\tt clock}'s current time step or an optionally passed-in
       {\tt timeStep}.
  
       The arguments are:
       \begin{description}
       \item[clock]
            The object instance for which to get the next time.
       \item[nextTime]
            The resulting {\tt ESMF\_Clock}'s next time.
       \item[{[timeStep]}]
            The time step interval to use instead of the clock's.
       \item[{[rc]}]
            Return code; equals {\tt ESMF\_SUCCESS} if there are no errors.
       \end{description}
   
%/////////////////////////////////////////////////////////////
 
\mbox{}\hrulefill\ 
 
\subsubsection [ESMF\_ClockIsCreated] {ESMF\_ClockIsCreated - Check whether a Clock object has been created}


 
\bigskip{\sf INTERFACE:}
\begin{verbatim}   function ESMF_ClockIsCreated(clock, rc)\end{verbatim}{\em RETURN VALUE:}
\begin{verbatim}     logical :: ESMF_ClockIsCreated\end{verbatim}{\em ARGUMENTS:}
\begin{verbatim}     type(ESMF_Clock), intent(in)            :: clock
 -- The following arguments require argument keyword syntax (e.g. rc=rc). --
     integer,             intent(out), optional :: rc
 \end{verbatim}
{\sf DESCRIPTION:\\ }


     Return {\tt .true.} if the {\tt clock} has been created. Otherwise return
     {\tt .false.}. If an error occurs, i.e. {\tt rc /= ESMF\_SUCCESS} is
     returned, the return value of the function will also be {\tt .false.}.
  
   The arguments are:
     \begin{description}
     \item[clock]
       {\tt ESMF\_Clock} queried.
     \item[{[rc]}]
       Return code; equals {\tt ESMF\_SUCCESS} if there are no errors.
     \end{description}
   
%/////////////////////////////////////////////////////////////
 
\mbox{}\hrulefill\ 
 
\subsubsection [ESMF\_ClockIsDone] {ESMF\_ClockIsDone - Based on its direction, test if the Clock has reached or exceeded its stop time or start time}


 
\bigskip{\sf INTERFACE:}
\begin{verbatim}       function ESMF_ClockIsDone(clock, rc)\end{verbatim}{\em RETURN VALUE:}
\begin{verbatim}       logical :: ESMF_ClockIsDone
 \end{verbatim}{\em ARGUMENTS:}
\begin{verbatim}       type(ESMF_Clock), intent(in)            :: clock
 -- The following arguments require argument keyword syntax (e.g. rc=rc). --
       integer,          intent(out), optional :: rc
 \end{verbatim}
{\sf STATUS:}
   \begin{itemize}
   \item\apiStatusCompatibleVersion{5.2.0r}
   \end{itemize}
  
{\sf DESCRIPTION:\\ }


       Returns true if currentTime is greater than or equal to stopTime
       in {\tt ESMF\_DIRECTION\_FORWARD}, or if currentTime is less than or
       equal to startTime in {\tt ESMF\_DIRECTION\_REVERSE}.  It returns false
       otherwise.
  
       The arguments are:
       \begin{description}
       \item[clock]
            The object instance to check.
       \item[{[rc]}]
            Return code; equals {\tt ESMF\_SUCCESS} if there are no errors.
       \end{description} 
%/////////////////////////////////////////////////////////////
 
\mbox{}\hrulefill\ 
 
\subsubsection [ESMF\_ClockIsReverse] {ESMF\_ClockIsReverse - Test if the Clock is in reverse mode}


 
\bigskip{\sf INTERFACE:}
\begin{verbatim}       function ESMF_ClockIsReverse(clock, rc)\end{verbatim}{\em RETURN VALUE:}
\begin{verbatim}       logical :: ESMF_ClockIsReverse
 \end{verbatim}{\em ARGUMENTS:}
\begin{verbatim}       type(ESMF_Clock), intent(in)            :: clock
 -- The following arguments require argument keyword syntax (e.g. rc=rc). --
       integer,          intent(out), optional :: rc
 \end{verbatim}
{\sf STATUS:}
   \begin{itemize}
   \item\apiStatusCompatibleVersion{5.2.0r}
   \end{itemize}
  
{\sf DESCRIPTION:\\ }


       Returns true if clock is in {\tt ESMF\_DIRECTION\_REVERSE}, and false if
       in {\tt ESMF\_DIRECTION\_FORWARD}.  Allows convenient use in "if" and
       "do while" constructs.  Alternative to
       {\tt ESMF\_ClockGet(...direction=...)}.
  
       The arguments are:
       \begin{description}
       \item[clock]
            The object instance to check.
       \item[{[rc]}]
            Return code; equals {\tt ESMF\_SUCCESS} if there are no errors.
       \end{description} 
%/////////////////////////////////////////////////////////////
 
\mbox{}\hrulefill\ 
 
\subsubsection [ESMF\_ClockIsStopTime] {ESMF\_ClockIsStopTime - Test if the Clock has reached or exceeded its stop time}


 
\bigskip{\sf INTERFACE:}
\begin{verbatim}       function ESMF_ClockIsStopTime(clock, rc)\end{verbatim}{\em RETURN VALUE:}
\begin{verbatim}       logical :: ESMF_ClockIsStopTime
 \end{verbatim}{\em ARGUMENTS:}
\begin{verbatim}       type(ESMF_Clock), intent(in)            :: clock
 -- The following arguments require argument keyword syntax (e.g. rc=rc). --
       integer,          intent(out), optional :: rc
 \end{verbatim}
{\sf STATUS:}
   \begin{itemize}
   \item\apiStatusCompatibleVersion{5.2.0r}
   \end{itemize}
  
{\sf DESCRIPTION:\\ }


       Returns true if the {\tt clock} has reached or exceeded its stop time,
       and false otherwise.
  
       The arguments are:
       \begin{description}
       \item[clock]
            The object instance to check.
       \item[{[rc]}]
            Return code; equals {\tt ESMF\_SUCCESS} if there are no errors.
       \end{description} 
%/////////////////////////////////////////////////////////////
 
\mbox{}\hrulefill\ 
 
\subsubsection [ESMF\_ClockIsStopTimeEnabled] {ESMF\_ClockIsStopTimeEnabled - Test if the Clock's stop time is enabled}


 
\bigskip{\sf INTERFACE:}
\begin{verbatim}       function ESMF_ClockIsStopTimeEnabled(clock, rc)\end{verbatim}{\em RETURN VALUE:}
\begin{verbatim}       logical :: ESMF_ClockIsStopTimeEnabled
 \end{verbatim}{\em ARGUMENTS:}
\begin{verbatim}       type(ESMF_Clock), intent(in)            :: clock
 -- The following arguments require argument keyword syntax (e.g. rc=rc). --
       integer,          intent(out), optional :: rc
 \end{verbatim}
{\sf STATUS:}
   \begin{itemize}
   \item\apiStatusCompatibleVersion{5.2.0r}
   \end{itemize}
  
{\sf DESCRIPTION:\\ }


       Returns true if the {\tt clock}'s stop time is set and enabled,
       and false otherwise.
  
       The arguments are:
       \begin{description}
       \item[clock]
            The object instance to check.
       \item[{[rc]}]
            Return code; equals {\tt ESMF\_SUCCESS} if there are no errors.
       \end{description} 
%/////////////////////////////////////////////////////////////
 
\mbox{}\hrulefill\ 
 
\subsubsection [ESMF\_ClockPrint] {ESMF\_ClockPrint - Print Clock information}


 
\bigskip{\sf INTERFACE:}
\begin{verbatim}       subroutine ESMF_ClockPrint(clock, options, preString, unit, rc)
 \end{verbatim}{\em ARGUMENTS:}
\begin{verbatim}       type(ESMF_Clock),  intent(in)            :: clock
 -- The following arguments require argument keyword syntax (e.g. rc=rc). --
       character (len=*), intent(in),  optional :: options
       character(*),      intent(in),  optional :: preString
       character(*),      intent(out), optional :: unit
       integer,           intent(out), optional :: rc
 \end{verbatim}
{\sf DESCRIPTION:\\ }


       Prints out an {\tt ESMF\_Clock}'s properties to {\tt stdout}, in
       support of testing and debugging.  The options control the type of
       information and level of detail. \\
  
       The arguments are:
       \begin{description}
       \item[clock]
            {\tt ESMF\_Clock} to be printed out.
       \item[{[options]}]
            Print options. If none specified, prints all {\tt clock} property
            values.\\
            "advanceCount" - print the number of times the clock has been
                             advanced. \\
            "alarmCount"   - print the number of alarms in the clock's list. \\
            "alarmList"    - print the clock's alarm list. \\
            "currTime"     - print the current clock time. \\
            "direction"    - print the clock's timestep direction. \\
            "name"         - print the clock's name. \\
            "prevTime"     - print the previous clock time. \\
            "refTime"      - print the clock's reference time. \\
            "startTime"    - print the clock's start time. \\
            "stopTime"     - print the clock's stop time. \\
            "timeStep"     - print the clock's time step. \\
       \item[{[preString]}]
            Optionally prepended string. Default to empty string.
       \item[{[unit]}]
            Internal unit, i.e. a string. Default to printing to stdout.
       \item[{[rc]}]
            Return code; equals {\tt ESMF\_SUCCESS} if there are no errors.
       \end{description}
   
%/////////////////////////////////////////////////////////////
 
\mbox{}\hrulefill\ 
 
\subsubsection [ESMF\_ClockSet] {ESMF\_ClockSet - Set one or more properties of a Clock}


 
\bigskip{\sf INTERFACE:}
\begin{verbatim}       subroutine ESMF_ClockSet(clock, &
         timeStep, startTime, stopTime, &
         runDuration, runTimeStepCount, refTime, currTime, advanceCount, &
         direction, name, rc)
 \end{verbatim}{\em ARGUMENTS:}
\begin{verbatim}       type(ESMF_Clock),        intent(inout)         :: clock
 -- The following arguments require argument keyword syntax (e.g. rc=rc). --
       type(ESMF_TimeInterval), intent(in),  optional :: timeStep
       type(ESMF_Time),         intent(in),  optional :: startTime
       type(ESMF_Time),         intent(in),  optional :: stopTime
       type(ESMF_TimeInterval), intent(in),  optional :: runDuration
       integer,                 intent(in),  optional :: runTimeStepCount
       type(ESMF_Time),         intent(in),  optional :: refTime
       type(ESMF_Time),         intent(in),  optional :: currTime
       integer(ESMF_KIND_I8),   intent(in),  optional :: advanceCount
       type(ESMF_Direction_Flag),    intent(in),  optional :: direction
       character (len=*),       intent(in),  optional :: name
       integer,                 intent(out), optional :: rc
 \end{verbatim}
{\sf STATUS:}
   \begin{itemize}
   \item\apiStatusCompatibleVersion{5.2.0r}
   \end{itemize}
  
{\sf DESCRIPTION:\\ }


       \begin{sloppypar}
       Sets/resets one or more of the properties of an {\tt ESMF\_Clock} that
       was previously initialized via {\tt ESMF\_ClockCreate()}.
       \end{sloppypar}
  
       The arguments are:
       \begin{description}
       \item[clock]
            The object instance to set.
       \item[{[timeStep]}]
            The {\tt ESMF\_Clock}'s time step interval, which can be positive or
            negative.  This is used to change a clock's timestep property for
            those applications that need variable timesteps.  See
            {\tt ESMF\_ClockAdvance()} below for specifying variable timesteps
            that are NOT saved as the clock's internal time step property.
            See "direction" argument below for behavior with
            {\\t ESMF\_DIRECTION\_REVERSE} direction.
       \item[{[startTime]}]
            The {\tt ESMF\_Clock}'s starting time.  Can be less than or
            or greater than stopTime, depending on a positive or negative
            timeStep, respectively, and whether a stopTime is specified;
            see below.
       \item[{[stopTime]}]
            The {\tt ESMF\_Clock}'s stopping time.  Can be greater than or
            less than the startTime, depending on a positive or negative
            timeStep, respectively.  If neither stopTime, runDuration, nor
            runTimeStepCount is specified, clock runs "forever"; user must
            use other means to know when to stop (e.g. ESMF\_Alarm or
            ESMF\_ClockGet(clock, currTime)).
            Mutually exclusive with runDuration and runTimeStepCount.
       \item[{[runDuration]}]
            Alternative way to specify {\tt ESMF\_Clock}'s stopping time;
               stopTime = startTime + runDuration.
            Can be positive or negative, consistent with the timeStep's sign.
            Mutually exclusive with stopTime and runTimeStepCount.
       \item[{[runTimeStepCount]}]
            Alternative way to specify {\tt ESMF\_Clock}'s stopping time;
               stopTime = startTime + (runTimeStepCount * timeStep).
            stopTime can be before startTime if timeStep is negative.
            Mutually exclusive with stopTime and runDuration.
       \item[{[refTime]}]
            The {\tt ESMF\_Clock}'s reference time.
            See description in {\tt ESMF\_ClockCreate()} above.
       \item[{[currTime]}]
            The current time.
       \item[{[advanceCount]}]
            The number of times the clock has been timestepped.
       \item[{[direction]}]
            Sets the clock's time-stepping direction.  If called with
            {\tt ESMF\_DIRECTION\_REVERSE}, sets the clock in "reverse" mode,
            causing it to timestep back towards its startTime.  If called
            with {\tt ESMF\_DIRECTION\_FORWARD}, sets the clock in normal,
            "forward" mode, causing it to timestep in the direction of its
            startTime to stopTime.  This holds true for negative timestep
            clocks as well, which are initialized (created) with
            stopTime < startTime.  The default mode is
            {\tt ESMF\_DIRECTION\_FORWARD}, established at
            {\tt ESMF\_ClockCreate()}.  timeStep can also be specified as an
            argument at the same time, which allows for a change in magnitude
            and/or sign of the clock's timeStep.  If not specified with
            {\tt ESMF\_DIRECTION\_REVERSE}, the clock's current timeStep is
            effectively negated.  If timeStep is specified, its sign is used as
            specified; it is not negated internally.  E.g., if the specified
            timeStep is negative and the clock is placed in
            {\tt ESMF\_DIRECTION\_REVERSE}, subsequent calls to
            {\tt ESMF\_ClockAdvance()} will cause the clock's current time to
            be decremented by the new timeStep's magnitude.
       \item[{[name]}]
            The new name for this clock.
       \item[{[rc]}]
            Return code; equals {\tt ESMF\_SUCCESS} if there are no errors.
       \end{description}
   
%/////////////////////////////////////////////////////////////
 
\mbox{}\hrulefill\ 
 
\subsubsection [ESMF\_ClockStopTimeDisable] {ESMF\_ClockStopTimeDisable - Disable a Clock's stop time}


 
\bigskip{\sf INTERFACE:}
\begin{verbatim}       subroutine ESMF_ClockStopTimeDisable(clock, rc)
 \end{verbatim}{\em ARGUMENTS:}
\begin{verbatim}       type(ESMF_Clock), intent(inout)         :: clock
 -- The following arguments require argument keyword syntax (e.g. rc=rc). --
       integer,          intent(out), optional :: rc
 \end{verbatim}
{\sf STATUS:}
   \begin{itemize}
   \item\apiStatusCompatibleVersion{5.2.0r}
   \end{itemize}
  
{\sf DESCRIPTION:\\ }


       Disables a {\tt ESMF\_Clock}'s stop time; {\tt ESMF\_ClockIsStopTime()}
       will always return false, allowing a clock to run past its stopTime.
  
       The arguments are:
       \begin{description}
       \item[clock]
            The object instance whose stop time to disable.
       \item[{[rc]}]
            Return code; equals {\tt ESMF\_SUCCESS} if there are no errors.
       \end{description} 
%/////////////////////////////////////////////////////////////
 
\mbox{}\hrulefill\ 
 
\subsubsection [ESMF\_ClockStopTimeEnable] {ESMF\_ClockStopTimeEnable - Enable an Clock's stop time}


 
\bigskip{\sf INTERFACE:}
\begin{verbatim}       subroutine ESMF_ClockStopTimeEnable(clock, stopTime, rc)
 \end{verbatim}{\em ARGUMENTS:}
\begin{verbatim}       type(ESMF_Clock), intent(inout)         :: clock
 -- The following arguments require argument keyword syntax (e.g. rc=rc). --
       type(ESMF_Time),  intent(in),  optional :: stopTime
       integer,          intent(out), optional :: rc
 \end{verbatim}
{\sf STATUS:}
   \begin{itemize}
   \item\apiStatusCompatibleVersion{5.2.0r}
   \end{itemize}
  
{\sf DESCRIPTION:\\ }


       Enables a {\tt ESMF\_Clock}'s stop time, allowing
       {\tt ESMF\_ClockIsStopTime()} to respect the stopTime.
  
       The arguments are:
       \begin{description}
       \item[clock]
            The object instance whose stop time to enable.
       \item[{[stopTime]}]
            The stop time to set or reset.
       \item[{[rc]}]
            Return code; equals {\tt ESMF\_SUCCESS} if there are no errors.
       \end{description} 
%/////////////////////////////////////////////////////////////
 
\mbox{}\hrulefill\ 
 
\subsubsection [ESMF\_ClockSyncToRealTime] {ESMF\_ClockSyncToRealTime - Set Clock's current time to wall clock time}


 
\bigskip{\sf INTERFACE:}
\begin{verbatim}       subroutine ESMF_ClockSyncToRealTime(clock, rc)
 \end{verbatim}{\em ARGUMENTS:}
\begin{verbatim}       type(ESMF_Clock), intent(inout)         :: clock
 -- The following arguments require argument keyword syntax (e.g. rc=rc). --
       integer,          intent(out), optional :: rc
 \end{verbatim}
{\sf STATUS:}
   \begin{itemize}
   \item\apiStatusCompatibleVersion{5.2.0r}
   \end{itemize}
  
{\sf DESCRIPTION:\\ }


       Sets a {\tt clock}'s current time to the wall clock time.  It is
       accurate to the nearest second.
  
       The arguments are:
       \begin{description}
       \item[clock]
            The object instance to be synchronized with wall clock time.
       \item[{[rc]}]
            Return code; equals {\tt ESMF\_SUCCESS} if there are no errors.
       \end{description}
   
%/////////////////////////////////////////////////////////////
 
\mbox{}\hrulefill\ 
 
\subsubsection [ESMF\_ClockValidate] {ESMF\_ClockValidate - Validate a Clock's properties}


 
\bigskip{\sf INTERFACE:}
\begin{verbatim}       subroutine ESMF_ClockValidate(clock, rc)
 \end{verbatim}{\em ARGUMENTS:}
\begin{verbatim}       type(ESMF_Clock),  intent(in)            :: clock
 -- The following arguments require argument keyword syntax (e.g. rc=rc). --
       integer,           intent(out), optional :: rc
 \end{verbatim}
{\sf STATUS:}
   \begin{itemize}
   \item\apiStatusCompatibleVersion{5.2.0r}
   \end{itemize}
  
{\sf DESCRIPTION:\\ }


       Checks whether a {\tt clock} is valid.
       Must have a valid startTime and timeStep.  If {\tt clock} has a
       stopTime, its currTime must be within startTime to stopTime, inclusive;
       also startTime's and stopTime's calendars must be the same.
  
       The arguments are:
       \begin{description}
       \item[clock]
            {\tt ESMF\_Clock} to be validated.
       \item[{[rc]}]
            Return code; equals {\tt ESMF\_SUCCESS} if there are no errors.
       \end{description}
  
%...............................................................
\setlength{\parskip}{\oldparskip}
\setlength{\parindent}{\oldparindent}
\setlength{\baselineskip}{\oldbaselineskip}
