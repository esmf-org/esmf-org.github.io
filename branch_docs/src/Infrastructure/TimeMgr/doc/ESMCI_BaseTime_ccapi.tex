%                **** IMPORTANT NOTICE *****
% This LaTeX file has been automatically produced by ProTeX v. 1.1
% Any changes made to this file will likely be lost next time
% this file is regenerated from its source. Send questions 
% to Arlindo da Silva, dasilva@gsfc.nasa.gov
 
\setlength{\oldparskip}{\parskip}
\setlength{\parskip}{1.5ex}
\setlength{\oldparindent}{\parindent}
\setlength{\parindent}{0pt}
\setlength{\oldbaselineskip}{\baselineskip}
\setlength{\baselineskip}{11pt}
 
%--------------------- SHORT-HAND MACROS ----------------------
\def\bv{\begin{verbatim}}
\def\ev{\end{verbatim}}
\def\be{\begin{equation}}
\def\ee{\end{equation}}
\def\bea{\begin{eqnarray}}
\def\eea{\end{eqnarray}}
\def\bi{\begin{itemize}}
\def\ei{\end{itemize}}
\def\bn{\begin{enumerate}}
\def\en{\end{enumerate}}
\def\bd{\begin{description}}
\def\ed{\end{description}}
\def\({\left (}
\def\){\right )}
\def\[{\left [}
\def\]{\right ]}
\def\<{\left  \langle}
\def\>{\right \rangle}
\def\cI{{\cal I}}
\def\diag{\mathop{\rm diag}}
\def\tr{\mathop{\rm tr}}
%-------------------------------------------------------------

\markboth{Left}{Source File: ESMCI\_BaseTime.C,  Date: Tue May  5 20:59:33 MDT 2020
}

 
%/////////////////////////////////////////////////////////////
\subsubsection [ESMCI::BaseTime::set] {ESMCI::BaseTime::set - set sub-day values of a basetime}


  
\bigskip{\sf INTERFACE:}
\begin{verbatim}       int BaseTime::set(\end{verbatim}{\em RETURN VALUE:}
\begin{verbatim}      int error return code\end{verbatim}{\em ARGUMENTS:}
\begin{verbatim}       ESMC_I4 *h,       // in - integer hours
       ESMC_I4 *m,       // in - integer minutes
       ESMC_I4 *s,       // in - integer seconds (>= 32 bit)
       ESMC_I8 *s_i8,    // in - integer seconds (large, >= 64 bit)
       ESMC_I4 *ms,      // in - integer milliseconds
       ESMC_I4 *us,      // in - integer microseconds
       ESMC_I4 *ns,      // in - integer nanoseconds
       ESMC_R8 *h_r8,    // in - floating point hours
       ESMC_R8 *m_r8,    // in - floating point minutes
       ESMC_R8 *s_r8,    // in - floating point seconds
       ESMC_R8 *ms_r8,   // in - floating point milliseconds
       ESMC_R8 *us_r8,   // in - floating point microseconds
       ESMC_R8 *ns_r8,   // in - floating point nanoseconds
       ESMC_I4 *sN,      // in - fractional seconds numerator
       ESMC_I8 *sN_i8,   // in - fractional seconds numerator  (large, >= 64 bit)
       ESMC_I4 *sD,      // in - fractional seconds denominator
       ESMC_I8 *sD_i8) { // in - fractional seconds denominator(large, >= 64 bit)\end{verbatim}
{\sf DESCRIPTION:\\ }


        Sets sub-day (non-calendar dependent) values of a {\tt ESMCI\_BaseTime}.
        Primarily to support F90 interface.
   
%/////////////////////////////////////////////////////////////
 
\mbox{}\hrulefill\ 
 
\subsubsection [ESMCI::BaseTime::set] {ESMCI::BaseTime::set - direct core value initializer}


  
\bigskip{\sf INTERFACE:}
\begin{verbatim}       int BaseTime::set(\end{verbatim}{\em RETURN VALUE:}
\begin{verbatim}      int error return code\end{verbatim}{\em ARGUMENTS:}
\begin{verbatim}       ESMC_I8 s,      // in - integer seconds
       ESMC_I8 sN,     // in - fractional seconds, numerator
       ESMC_I8 sD ) {  // in - fractional seconds, denominator\end{verbatim}
{\sf DESCRIPTION:\\ }


        Initialzes a {\tt ESMCI::BaseTime} with given values
   
%/////////////////////////////////////////////////////////////
 
\mbox{}\hrulefill\ 
 
\subsubsection [ESMCI::BaseTime::get] {ESMCI::BaseTime::get - get units of a basetime}


  
\bigskip{\sf INTERFACE:}
\begin{verbatim}       int BaseTime::get(\end{verbatim}{\em RETURN VALUE:}
\begin{verbatim}      int error return code\end{verbatim}{\em ARGUMENTS:}
\begin{verbatim}       const BaseTime *timeToConvert, // in  - the time to convert
                                           //     (divide) into requested units
       ESMC_I4 *h,              // out - integer hours
       ESMC_I4 *m,              // out - integer minutes
       ESMC_I4 *s,              // out - integer seconds (>= 32-bit)
       ESMC_I8 *s_i8,           // out - integer seconds (large, >= 64-bit)
       ESMC_I4 *ms,             // out - integer milliseconds
       ESMC_I4 *us,             // out - integer microseconds
       ESMC_I4 *ns,             // out - integer nanoseconds
       ESMC_R8 *h_r8,           // out - floating point hours
       ESMC_R8 *m_r8,           // out - floating point minutes
       ESMC_R8 *s_r8,           // out - floating point seconds
       ESMC_R8 *ms_r8,          // out - floating point milliseconds
       ESMC_R8 *us_r8,          // out - floating point microseconds
       ESMC_R8 *ns_r8,          // out - floating point nanoseconds
       ESMC_I4 *sN,             // out - fractional seconds numerator
       ESMC_I8 *sN_i8,          // out - fractional seconds numerator
                                //                            (large, >= 64-bit)
       ESMC_I4 *sD,             // out - fractional seconds denominator
       ESMC_I8 *sD_i8) const {  // out - fractional seconds denominator
                                //                            (large, >= 64-bit)
 \end{verbatim}
{\sf DESCRIPTION:\\ }


        Get non-calendar dependent values of a {\tt ESMC\_BaseTime}
        converted to user units.  Primarily to support F90 interface
   
%/////////////////////////////////////////////////////////////
 
\mbox{}\hrulefill\ 
 
\subsubsection [ESMCI::BaseTime(=)] {ESMCI::BaseTime(=) - assignment operator}


  
\bigskip{\sf INTERFACE:}
\begin{verbatim}       BaseTime& BaseTime::operator=(\end{verbatim}{\em RETURN VALUE:}
\begin{verbatim}      ESMCI::BaseTime& result\end{verbatim}{\em ARGUMENTS:}
\begin{verbatim}       const Fraction &fraction) {   // in - Fraction\end{verbatim}
{\sf DESCRIPTION:\\ }


        Assign current object's (this) {\tt ESMC\_Fraction} with given
        {\tt ESMC\_Fraction}.   
%/////////////////////////////////////////////////////////////
 
\mbox{}\hrulefill\ 
 
\subsubsection [ESMCI::BaseTime::readRestart] {ESMCI::BaseTime::readRestart - restore BaseTime state}


  
\bigskip{\sf INTERFACE:}
\begin{verbatim}       int BaseTime::readRestart(\end{verbatim}{\em RETURN VALUE:}
\begin{verbatim}      int error return code\end{verbatim}{\em ARGUMENTS:}
\begin{verbatim}       int          nameLen,   // in
       const char  *name) {    // in\end{verbatim}
{\sf DESCRIPTION:\\ }


        restore {\tt BaseTime} state for persistence/checkpointing.
   
%/////////////////////////////////////////////////////////////
 
\mbox{}\hrulefill\ 
 
\subsubsection [ESMCI::BaseTime::writeRestart] {ESMCI::BaseTime::writeRestart - save BaseTime state}


  
\bigskip{\sf INTERFACE:}
\begin{verbatim}       int BaseTime::writeRestart(void) const {\end{verbatim}{\em RETURN VALUE:}
\begin{verbatim}      int error return code\end{verbatim}{\em ARGUMENTS:}
\begin{verbatim}      none\end{verbatim}
{\sf DESCRIPTION:\\ }


        Save {\tt BaseTime} state for persistence/checkpointing
   
%/////////////////////////////////////////////////////////////
 
\mbox{}\hrulefill\ 
 
\subsubsection [ESMCI::BaseTime::validate] {ESMCI::BaseTime::validate - validate BaseTime state}


  
\bigskip{\sf INTERFACE:}
\begin{verbatim}       int BaseTime::validate(\end{verbatim}{\em RETURN VALUE:}
\begin{verbatim}      int error return code\end{verbatim}{\em ARGUMENTS:}
\begin{verbatim}       const char *options) const {     // in - options\end{verbatim}
{\sf DESCRIPTION:\\ }


        validate {\tt ESMCI::BaseTime} state
   
%/////////////////////////////////////////////////////////////
 
\mbox{}\hrulefill\ 
 
\subsubsection [ESMCI::BaseTime::print] {ESMCI::BaseTime::print - print BaseTime state}


  
\bigskip{\sf INTERFACE:}
\begin{verbatim}       int BaseTime::print(\end{verbatim}{\em RETURN VALUE:}
\begin{verbatim}      int error return code\end{verbatim}{\em ARGUMENTS:}
\begin{verbatim}       const char *options) const {    // in - print options\end{verbatim}
{\sf DESCRIPTION:\\ }


        print {\tt ESMCI::BaseTime} state for testing/debugging
   
%/////////////////////////////////////////////////////////////
 
\mbox{}\hrulefill\ 
 
\subsubsection [ESMCI::BaseTime] {ESMCI::BaseTime - native default C++ constructor}


  
\bigskip{\sf INTERFACE:}
\begin{verbatim}       BaseTime::BaseTime(void) {\end{verbatim}{\em RETURN VALUE:}
\begin{verbatim}      none\end{verbatim}{\em ARGUMENTS:}
\begin{verbatim}      none\end{verbatim}
{\sf DESCRIPTION:\\ }


        Initializes a {\tt ESMCI::BaseTime} with defaults
   
%/////////////////////////////////////////////////////////////
 
\mbox{}\hrulefill\ 
 
\subsubsection [ESMCI::BaseTime] {ESMCI::BaseTime - native C++ constructor}


  
\bigskip{\sf INTERFACE:}
\begin{verbatim}       BaseTime::BaseTime(\end{verbatim}{\em RETURN VALUE:}
\begin{verbatim}      none\end{verbatim}{\em ARGUMENTS:}
\begin{verbatim}       ESMC_I8 s,              // in - integer seconds
       ESMC_I8 sN,             // in - fractional seconds, numerator
       ESMC_I8 sD) :           // in - fractional seconds, denominator\end{verbatim}
{\sf DESCRIPTION:\\ }


        Initializes a {\tt ESMC\_BaseTime}
   
%/////////////////////////////////////////////////////////////
 
\mbox{}\hrulefill\ 
 
\subsubsection [ESMCI::~BaseTime] {ESMCI::~BaseTime - native default C++ destructor}


  
\bigskip{\sf INTERFACE:}
\begin{verbatim}       BaseTime::~BaseTime(void) {\end{verbatim}{\em RETURN VALUE:}
\begin{verbatim}      none\end{verbatim}{\em ARGUMENTS:}
\begin{verbatim}      none\end{verbatim}
{\sf DESCRIPTION:\\ }


        Default {\tt ESMCI::BaseTime} destructor
  
%...............................................................
\setlength{\parskip}{\oldparskip}
\setlength{\parindent}{\oldparindent}
\setlength{\baselineskip}{\oldbaselineskip}
