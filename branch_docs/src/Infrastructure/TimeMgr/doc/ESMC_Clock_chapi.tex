%                **** IMPORTANT NOTICE *****
% This LaTeX file has been automatically produced by ProTeX v. 1.1
% Any changes made to this file will likely be lost next time
% this file is regenerated from its source. Send questions 
% to Arlindo da Silva, dasilva@gsfc.nasa.gov
 
\setlength{\oldparskip}{\parskip}
\setlength{\parskip}{1.5ex}
\setlength{\oldparindent}{\parindent}
\setlength{\parindent}{0pt}
\setlength{\oldbaselineskip}{\baselineskip}
\setlength{\baselineskip}{11pt}
 
%--------------------- SHORT-HAND MACROS ----------------------
\def\bv{\begin{verbatim}}
\def\ev{\end{verbatim}}
\def\be{\begin{equation}}
\def\ee{\end{equation}}
\def\bea{\begin{eqnarray}}
\def\eea{\end{eqnarray}}
\def\bi{\begin{itemize}}
\def\ei{\end{itemize}}
\def\bn{\begin{enumerate}}
\def\en{\end{enumerate}}
\def\bd{\begin{description}}
\def\ed{\end{description}}
\def\({\left (}
\def\){\right )}
\def\[{\left [}
\def\]{\right ]}
\def\<{\left  \langle}
\def\>{\right \rangle}
\def\cI{{\cal I}}
\def\diag{\mathop{\rm diag}}
\def\tr{\mathop{\rm tr}}
%-------------------------------------------------------------

\markboth{Left}{Source File: ESMC\_Clock.h,  Date: Tue May  5 20:59:33 MDT 2020
}

 
%/////////////////////////////////////////////////////////////
\subsubsection [ESMC\_ClockAdvance] {ESMC\_ClockAdvance - Advance a Clock's current time by one time step}


 
  
\bigskip{\sf INTERFACE:}
\begin{verbatim} int ESMC_ClockAdvance(
   ESMC_Clock clock   // in
 );
 \end{verbatim}{\em RETURN VALUE:}
\begin{verbatim}    Return code; equals ESMF_SUCCESS if there are no errors.\end{verbatim}
{\sf DESCRIPTION:\\ }


  
    Advances the {\tt ESMC\_Clock}'s current time by one time step.
  
    The arguments are:
    \begin{description}
    \item[clock]
      {\tt ESMC\_Clock} object to be advanced.
    \end{description}
   
%/////////////////////////////////////////////////////////////
 
\mbox{}\hrulefill\ 
 
\subsubsection [ESMC\_ClockCreate] {ESMC\_ClockCreate - Create a Clock}


  
\bigskip{\sf INTERFACE:}
\begin{verbatim} ESMC_Clock ESMC_ClockCreate(
   const char *name,             // in
   ESMC_TimeInterval timeStep,   // in
   ESMC_Time startTime,          // in
   ESMC_Time stopTime,           // in
   int *rc                       // out
 );
 \end{verbatim}{\em RETURN VALUE:}
\begin{verbatim}    Newly created ESMC_Clock object.\end{verbatim}
{\sf DESCRIPTION:\\ }


  
   Creates and sets the initial values in a new {\tt ESMC\_Clock} object. 
  
    The arguments are:
    \begin{description}
    \item[{[name]}]
      The name for the newly created Clock.  If not specified, i.e. NULL,
      a default unique name will be generated: "ClockNNN" where NNN
      is a unique sequence number from 001 to 999.
    \item[timeStep]
      The {\tt ESMC\_Clock}'s time step interval, which can be
      positive or negative.
    \item[startTime]
      The {\tt ESMC\_Clock}'s starting time.  Can be less than or
      or greater than stopTime, depending on a positive or negative
      timeStep, respectively, and whether a stopTime is specified;
      see below.
    \item[stopTime]
      The {\tt ESMC\_Clock}'s stopping time.  Can be greater than or
      less than the startTime, depending on a positive or negative
      timeStep, respectively.
    \item[{[rc]}]
      Return code; equals {\tt ESMF\_SUCCESS} if there are no errors.
    \end{description}
   
%/////////////////////////////////////////////////////////////
 
\mbox{}\hrulefill\ 
 
\subsubsection [ESMC\_ClockDestroy] {ESMC\_ClockDestroy - Destroy a Clock}


  
\bigskip{\sf INTERFACE:}
\begin{verbatim} int ESMC_ClockDestroy(
   ESMC_Clock *clock   // inout
 );
 \end{verbatim}{\em RETURN VALUE:}
\begin{verbatim}    Return code; equals ESMF_SUCCESS if there are no errors.\end{verbatim}
{\sf DESCRIPTION:\\ }


  
    Releases all resources associated with this {\tt ESMC\_Clock}.
  
    The arguments are:
    \begin{description}
    \item[clock]
      Destroy contents of this {\tt ESMC\_Clock}.
    \end{description}
   
%/////////////////////////////////////////////////////////////
 
\mbox{}\hrulefill\ 
 
\subsubsection [ESMC\_ClockGet] {ESMC\_ClockGet - Get a Clock's properties}


  
\bigskip{\sf INTERFACE:}
\begin{verbatim} int ESMC_ClockGet(
   ESMC_Clock clock,                 // in
   ESMC_TimeInterval *currSimTime,   // out
   ESMC_I8 *advanceCount             // out
 );
 \end{verbatim}{\em RETURN VALUE:}
\begin{verbatim}    Return code; equals ESMF_SUCCESS if there are no errors.\end{verbatim}
{\sf DESCRIPTION:\\ }


  
   Gets one or more of the properties of an {\tt ESMC\_Clock}.
  
    The arguments are:
    \begin{description}
    \item[clock]
      {\tt ESMC\_Clock} object to be queried.
    \item[{[currSimTime]}]
      The current simulation time.
    \item[{[advanceCount]}]
      The number of times the {\tt ESMC\_Clock} has been advanced.
    \end{description}
   
%/////////////////////////////////////////////////////////////
 
\mbox{}\hrulefill\ 
 
\subsubsection [ESMC\_ClockPrint] {ESMC\_ClockPrint - Print the contents of a Clock}


  
\bigskip{\sf INTERFACE:}
\begin{verbatim} int ESMC_ClockPrint(
   ESMC_Clock clock   // in
 );
 \end{verbatim}{\em RETURN VALUE:}
\begin{verbatim}    Return code; equals ESMF_SUCCESS if there are no errors.\end{verbatim}
{\sf DESCRIPTION:\\ }


  
    Prints out an {\tt ESMC\_Clock}'s properties to {\tt stdio}, 
    in support of testing and debugging.
  
    The arguments are:
    \begin{description}
    \item[clock]
      {\tt ESMC\_Clock} object to be printed.
    \end{description}
  
%...............................................................
\setlength{\parskip}{\oldparskip}
\setlength{\parindent}{\oldparindent}
\setlength{\baselineskip}{\oldbaselineskip}
