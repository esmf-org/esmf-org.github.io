%                **** IMPORTANT NOTICE *****
% This LaTeX file has been automatically produced by ProTeX v. 1.1
% Any changes made to this file will likely be lost next time
% this file is regenerated from its source. Send questions 
% to Arlindo da Silva, dasilva@gsfc.nasa.gov
 
\setlength{\oldparskip}{\parskip}
\setlength{\parskip}{1.5ex}
\setlength{\oldparindent}{\parindent}
\setlength{\parindent}{0pt}
\setlength{\oldbaselineskip}{\baselineskip}
\setlength{\baselineskip}{11pt}
 
%--------------------- SHORT-HAND MACROS ----------------------
\def\bv{\begin{verbatim}}
\def\ev{\end{verbatim}}
\def\be{\begin{equation}}
\def\ee{\end{equation}}
\def\bea{\begin{eqnarray}}
\def\eea{\end{eqnarray}}
\def\bi{\begin{itemize}}
\def\ei{\end{itemize}}
\def\bn{\begin{enumerate}}
\def\en{\end{enumerate}}
\def\bd{\begin{description}}
\def\ed{\end{description}}
\def\({\left (}
\def\){\right )}
\def\[{\left [}
\def\]{\right ]}
\def\<{\left  \langle}
\def\>{\right \rangle}
\def\cI{{\cal I}}
\def\diag{\mathop{\rm diag}}
\def\tr{\mathop{\rm tr}}
%-------------------------------------------------------------

\markboth{Left}{Source File: ESMF\_TimeIntervalEx.F90,  Date: Tue May  5 20:59:34 MDT 2020
}

 
%/////////////////////////////////////////////////////////////

 \begin{verbatim}
! !PROGRAM: ESMF_TimeIntervalEx - Time Interval initialization and 
!                                 manipulation examples
!
! !DESCRIPTION:
!
! This program shows examples of Time Interval initialization and manipulation
!-----------------------------------------------------------------------------
#include "ESMF.h"

      ! ESMF Framework module
      use ESMF
      use ESMF_TestMod
      implicit none

      ! instantiate some time intervals
      type(ESMF_TimeInterval) :: timeinterval1, timeinterval2, timeinterval3

      ! local variables
      integer :: d, h, m, s

      ! return code
      integer:: rc
 
\end{verbatim}
 
%/////////////////////////////////////////////////////////////

 \begin{verbatim}
      ! initialize ESMF framework
      call ESMF_Initialize(defaultCalKind=ESMF_CALKIND_GREGORIAN, &
        defaultlogfilename="TimeIntervalEx.Log", &
                    logkindflag=ESMF_LOGKIND_MULTI, rc=rc)
 
\end{verbatim}
 
%/////////////////////////////////////////////////////////////

  \subsubsection{TimeInterval initialization}
 
   This example shows how to initialize two {\tt ESMF\_TimeIntervals}. 
%/////////////////////////////////////////////////////////////

 \begin{verbatim}
      ! initialize time interval1 to 1 day
      call ESMF_TimeIntervalSet(timeinterval1, d=1, rc=rc)
 
\end{verbatim}
 
%/////////////////////////////////////////////////////////////

 \begin{verbatim}
      call ESMF_TimeIntervalPrint(timeinterval1, options="string", rc=rc)
 
\end{verbatim}
 
%/////////////////////////////////////////////////////////////

 \begin{verbatim}
      ! initialize time interval2 to 4 days, 1 hour, 30 minutes, 10 seconds
      call ESMF_TimeIntervalSet(timeinterval2, d=4, h=1, m=30, s=10, rc=rc)
 
\end{verbatim}
 
%/////////////////////////////////////////////////////////////

 \begin{verbatim}
      call ESMF_TimeIntervalPrint(timeinterval2, options="string", rc=rc)
 
\end{verbatim}
 
%/////////////////////////////////////////////////////////////

  \subsubsection{TimeInterval conversion}
 
   This example shows how to convert {\tt ESMF\_TimeIntervals} into 
   different units. 
%/////////////////////////////////////////////////////////////

 \begin{verbatim}
      call ESMF_TimeIntervalGet(timeinterval1, s=s, rc=rc)
      print *, "Time Interval1 = ", s, " seconds."
 
\end{verbatim}
 
%/////////////////////////////////////////////////////////////

 \begin{verbatim}
      call ESMF_TimeIntervalGet(timeinterval2, h=h, m=m, s=s, rc=rc)
      print *, "Time Interval2 = ", h, " hours, ", m, " minutes, ", &
                                    s, " seconds."
 
\end{verbatim}
 
%/////////////////////////////////////////////////////////////

  \subsubsection{TimeInterval difference}
 
   This example shows how to calculate the difference between two 
   {\tt ESMF\_TimeIntervals}.  
%/////////////////////////////////////////////////////////////

 \begin{verbatim}
      ! difference between two time intervals
      timeinterval3 = timeinterval2 - timeinterval1
     call ESMF_TimeIntervalGet(timeinterval3, d=d, h=h, m=m, s=s, rc=rc)
     print *, "Difference between TimeInterval2 and TimeInterval1 = ", &
           d, " days, ", h, " hours, ", m, " minutes, ", s, " seconds."
 
\end{verbatim}
 
%/////////////////////////////////////////////////////////////

  \subsubsection{TimeInterval multiplication}
 
   This example shows how to multiply an {\tt ESMF\_TimeInterval}.  
%/////////////////////////////////////////////////////////////

 \begin{verbatim}
      ! multiply time interval by an integer
      timeinterval3 = timeinterval2 * 3
      call ESMF_TimeIntervalGet(timeinterval3, d=d, h=h, m=m, s=s, rc=rc)
      print *, "TimeInterval2 multiplied by 3 = ", d, " days, ", h, &
               " hours, ", m, " minutes, ", s, " seconds."
 
\end{verbatim}
 
%/////////////////////////////////////////////////////////////

  \subsubsection{TimeInterval comparison}
 
   This example shows how to compare two {\tt ESMF\_TimeIntervals}.  
%/////////////////////////////////////////////////////////////

 \begin{verbatim}
      ! comparison
      if (timeinterval1 < timeinterval2) then
        print *, "TimeInterval1 is smaller than TimeInterval2"
      else 
        print *, "TimeInterval1 is larger than or equal to TimeInterval2"
      end if

 
\end{verbatim}
 
%/////////////////////////////////////////////////////////////

 \begin{verbatim}
      end program ESMF_TimeIntervalEx
 
\end{verbatim}

%...............................................................
\setlength{\parskip}{\oldparskip}
\setlength{\parindent}{\oldparindent}
\setlength{\baselineskip}{\oldbaselineskip}
