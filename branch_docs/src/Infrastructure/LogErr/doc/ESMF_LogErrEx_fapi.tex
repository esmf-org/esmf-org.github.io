%                **** IMPORTANT NOTICE *****
% This LaTeX file has been automatically produced by ProTeX v. 1.1
% Any changes made to this file will likely be lost next time
% this file is regenerated from its source. Send questions 
% to Arlindo da Silva, dasilva@gsfc.nasa.gov
 
\setlength{\oldparskip}{\parskip}
\setlength{\parskip}{1.5ex}
\setlength{\oldparindent}{\parindent}
\setlength{\parindent}{0pt}
\setlength{\oldbaselineskip}{\baselineskip}
\setlength{\baselineskip}{11pt}
 
%--------------------- SHORT-HAND MACROS ----------------------
\def\bv{\begin{verbatim}}
\def\ev{\end{verbatim}}
\def\be{\begin{equation}}
\def\ee{\end{equation}}
\def\bea{\begin{eqnarray}}
\def\eea{\end{eqnarray}}
\def\bi{\begin{itemize}}
\def\ei{\end{itemize}}
\def\bn{\begin{enumerate}}
\def\en{\end{enumerate}}
\def\bd{\begin{description}}
\def\ed{\end{description}}
\def\({\left (}
\def\){\right )}
\def\[{\left [}
\def\]{\right ]}
\def\<{\left  \langle}
\def\>{\right \rangle}
\def\cI{{\cal I}}
\def\diag{\mathop{\rm diag}}
\def\tr{\mathop{\rm tr}}
%-------------------------------------------------------------

\markboth{Left}{Source File: ESMF\_LogErrEx.F90,  Date: Tue May  5 20:59:28 MDT 2020
}

 
%/////////////////////////////////////////////////////////////

 \begin{verbatim}
! !PROGRAM: ESMF_LogErrEx - Log Error examples
!
! !DESCRIPTION:
!
! This program shows examples of Log Error writing
!-----------------------------------------------------------------------------
 
\end{verbatim}
 
%/////////////////////////////////////////////////////////////

 \begin{verbatim}
! Macros for cpp usage
! File define
#define ESMF_FILENAME "ESMF_LogErrEx.F90"
! Method define
#define ESMF_METHOD "program ESMF_LogErrEx"
#include "ESMF_LogMacros.inc"

    ! ESMF Framework module
    use ESMF
    use ESMF_TestMod
    implicit none

    ! return variables
    integer :: rc1, rc2, rc3, rcToTest, allocRcToTest, result
    type(ESMF_LOG) :: alog  ! a log object that is not the default log
    type(ESMF_LogKind_Flag) :: logkindflag
    type(ESMF_Time) :: time
    type(ESMF_VM) :: vm
    integer, pointer :: intptr(:)
 
\end{verbatim}
 
%/////////////////////////////////////////////////////////////

  \subsubsection{Default Log}
 
   This example shows how to use the default Log.  This example does not use cpp
   macros but does use multi Logs.  A separate Log will be created for each PET. 
%/////////////////////////////////////////////////////////////

 \begin{verbatim}
    ! Initialize ESMF to initialize the default Log
    call ESMF_Initialize(vm=vm, defaultlogfilename="LogErrEx.Log", &
                     logkindflag=ESMF_LOGKIND_MULTI, rc=rc1)

 
\end{verbatim}
 
%/////////////////////////////////////////////////////////////

 \begin{verbatim}
    ! LogWrite
    call ESMF_LogWrite("Log Write 2", ESMF_LOGMSG_INFO, rc=rc2)
 
\end{verbatim}
 
%/////////////////////////////////////////////////////////////

 \begin{verbatim}
    ! LogMsgSetError
    call ESMF_LogSetError(ESMF_RC_OBJ_BAD, msg="Convergence failure", &
                             rcToReturn=rc2)
 
\end{verbatim}
 
%/////////////////////////////////////////////////////////////

 \begin{verbatim}
    ! LogMsgFoundError
    call ESMF_TimeSet(time, calkindflag=ESMF_CALKIND_NOCALENDAR)
    call ESMF_TimeSyncToRealTime(time, rc=rcToTest)
    if (ESMF_LogFoundError(rcToTest, msg="getting wall clock time", &
                              rcToReturn=rc2)) then
        ! Error getting time. The previous call will have printed the error
        ! already into the log file.  Add any additional error handling here.
        ! (This call is expected to provoke an error from the Time Manager.)
    endif

    ! LogMsgFoundAllocError
    allocate(intptr(10), stat=allocRcToTest)
    if (ESMF_LogFoundAllocError(allocRcToTest, msg="integer array", &
                                   rcToReturn=rc2)) then
        ! Error during allocation.  The previous call will have logged already
        ! an error message into the log.
    endif
    deallocate(intptr)
 
\end{verbatim}
 
%/////////////////////////////////////////////////////////////

  \subsubsection{User created Log}
   This example shows how to use a user created Log.  This example uses
   cpp macros. 
%/////////////////////////////////////////////////////////////

 \begin{verbatim}
    ! Open a Log named "Testlog.txt" associated with alog.
    call ESMF_LogOpen(alog, "TestLog.txt", rc=rc1)
 
\end{verbatim}
 
 
%/////////////////////////////////////////////////////////////

 \begin{verbatim}

 
%/////////////////////////////////////////////////////////////

 \begin{verbatim}
    ! LogWrite
    call ESMF_LogWrite("Log Write 2", ESMF_LOGMSG_INFO, &
                       line=__LINE__, file=ESMF_FILENAME, &
                       method=ESMF_METHOD, log=alog, rc=rc2)
 
\end{verbatim}
 
%/////////////////////////////////////////////////////////////

 \begin{verbatim}
    ! LogMsgSetError
    call ESMF_LogSetError(ESMF_RC_OBJ_BAD,  msg="Interpolation Failure", &
                          line=__LINE__, file=ESMF_FILENAME, &
                           method=ESMF_METHOD, rcToReturn=rc2, log=alog)
 
\end{verbatim}
 
%/////////////////////////////////////////////////////////////

  \subsubsection{Get and Set}
   This example shows how to use Get and Set routines, on both the default Log
   and the user created Log from the previous examples. 
%/////////////////////////////////////////////////////////////

 \begin{verbatim}
    ! This is an example showing a query of the default Log.  Please note that
    ! no Log is passed in the argument list, so the default Log will be used.
    call ESMF_LogGet(logkindflag=logkindflag, rc=rc3)
 
\end{verbatim}
 
%/////////////////////////////////////////////////////////////

 \begin{verbatim}
    ! This is an example setting a property of a Log that is not the default.
    ! It was opened in a previous example, and the handle for it must be
    ! passed in the argument list.
    call ESMF_LogSet(log=alog, logmsgAbort=(/ESMF_LOGMSG_ERROR/), rc=rc2)
 
\end{verbatim}
 
%/////////////////////////////////////////////////////////////

 \begin{verbatim}
    ! Close the user log.
    call ESMF_LogClose(alog, rc=rc3)
 
\end{verbatim}
 
%/////////////////////////////////////////////////////////////

 \begin{verbatim}
    ! Finalize ESMF to close the default log
    call ESMF_Finalize(rc=rc1)
 
\end{verbatim}

%...............................................................
\setlength{\parskip}{\oldparskip}
\setlength{\parindent}{\oldparindent}
\setlength{\baselineskip}{\oldbaselineskip}
