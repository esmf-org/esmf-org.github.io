%                **** IMPORTANT NOTICE *****
% This LaTeX file has been automatically produced by ProTeX v. 1.1
% Any changes made to this file will likely be lost next time
% this file is regenerated from its source. Send questions 
% to Arlindo da Silva, dasilva@gsfc.nasa.gov
 
\setlength{\oldparskip}{\parskip}
\setlength{\parskip}{1.5ex}
\setlength{\oldparindent}{\parindent}
\setlength{\parindent}{0pt}
\setlength{\oldbaselineskip}{\baselineskip}
\setlength{\baselineskip}{11pt}
 
%--------------------- SHORT-HAND MACROS ----------------------
\def\bv{\begin{verbatim}}
\def\ev{\end{verbatim}}
\def\be{\begin{equation}}
\def\ee{\end{equation}}
\def\bea{\begin{eqnarray}}
\def\eea{\end{eqnarray}}
\def\bi{\begin{itemize}}
\def\ei{\end{itemize}}
\def\bn{\begin{enumerate}}
\def\en{\end{enumerate}}
\def\bd{\begin{description}}
\def\ed{\end{description}}
\def\({\left (}
\def\){\right )}
\def\[{\left [}
\def\]{\right ]}
\def\<{\left  \langle}
\def\>{\right \rangle}
\def\cI{{\cal I}}
\def\diag{\mathop{\rm diag}}
\def\tr{\mathop{\rm tr}}
%-------------------------------------------------------------

\markboth{Left}{Source File: ESMF\_VMSendVMRecvEx.F90,  Date: Tue May  5 20:59:29 MDT 2020
}

 
%/////////////////////////////////////////////////////////////

  
   \subsubsection{Communication - Send and Recv}
  
   The VM layer provides MPI-like point-to-point communication. Use 
   {\tt ESMF\_VMSend()} and {\tt ESMF\_VMRecv()} to pass data between two PETs.
   The following code sends data from PET 'src' and receives it on PET 'dst'.
   Both PETs must be part of the same VM.
   
%/////////////////////////////////////////////////////////////

 \begin{verbatim}
  integer, allocatable:: localData(:)
 
\end{verbatim}
 
%/////////////////////////////////////////////////////////////

 \begin{verbatim}
  count = 10
  allocate(localData(count))
  do i=1, count
    localData(i) = localPet*100 + i
  enddo
 
\end{verbatim}
 
%/////////////////////////////////////////////////////////////

 \begin{verbatim}
  if (localPet==src) then
    call ESMF_VMSend(vm, sendData=localData, count=count, dstPet=dst, rc=rc)
  endif
 
\end{verbatim}
 
%/////////////////////////////////////////////////////////////

 \begin{verbatim}
  if (localPet==dst) then
    call ESMF_VMRecv(vm, recvData=localData, count=count, srcPet=src, rc=rc)
  endif
 
\end{verbatim}

%...............................................................
\setlength{\parskip}{\oldparskip}
\setlength{\parindent}{\oldparindent}
\setlength{\baselineskip}{\oldbaselineskip}
