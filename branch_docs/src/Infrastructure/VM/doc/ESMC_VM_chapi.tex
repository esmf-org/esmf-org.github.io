%                **** IMPORTANT NOTICE *****
% This LaTeX file has been automatically produced by ProTeX v. 1.1
% Any changes made to this file will likely be lost next time
% this file is regenerated from its source. Send questions 
% to Arlindo da Silva, dasilva@gsfc.nasa.gov
 
\setlength{\oldparskip}{\parskip}
\setlength{\parskip}{1.5ex}
\setlength{\oldparindent}{\parindent}
\setlength{\parindent}{0pt}
\setlength{\oldbaselineskip}{\baselineskip}
\setlength{\baselineskip}{11pt}
 
%--------------------- SHORT-HAND MACROS ----------------------
\def\bv{\begin{verbatim}}
\def\ev{\end{verbatim}}
\def\be{\begin{equation}}
\def\ee{\end{equation}}
\def\bea{\begin{eqnarray}}
\def\eea{\end{eqnarray}}
\def\bi{\begin{itemize}}
\def\ei{\end{itemize}}
\def\bn{\begin{enumerate}}
\def\en{\end{enumerate}}
\def\bd{\begin{description}}
\def\ed{\end{description}}
\def\({\left (}
\def\){\right )}
\def\[{\left [}
\def\]{\right ]}
\def\<{\left  \langle}
\def\>{\right \rangle}
\def\cI{{\cal I}}
\def\diag{\mathop{\rm diag}}
\def\tr{\mathop{\rm tr}}
%-------------------------------------------------------------

\markboth{Left}{Source File: ESMC\_VM.h,  Date: Tue May  5 20:59:29 MDT 2020
}

 
%/////////////////////////////////////////////////////////////
\subsubsection [ESMC\_VMBarrier] {ESMC\_VMBarrier - block calling PETs until all PETS called}


  
\bigskip{\sf INTERFACE:}
\begin{verbatim} int ESMC_VMBarrier(
   ESMC_VM vm                   // in
 );\end{verbatim}{\em RETURN VALUE:}
\begin{verbatim}    Return code; equals ESMF_SUCCESS if there are no errors.\end{verbatim}
{\sf DESCRIPTION:\\ }


  
    Collective {\tt ESMC\_VM} communication call that blocks calling PET until 
    all PETs of the VM context have issued the call.
  
    The arguments are:
    \begin{description}
    \item[vm] 
      Queried {\tt ESMC\_VM} object.
    \item[{[rc]}]
      Return code; equals {\tt ESMF\_SUCCESS} if there are no errors.
    \end{description}
   
%/////////////////////////////////////////////////////////////
 
\mbox{}\hrulefill\ 
 
\subsubsection [ESMC\_VMBroadcast] {ESMC\_VMBroadcast - Broadcast data across the VM}


  
\bigskip{\sf INTERFACE:}
\begin{verbatim} int ESMC_VMBroadcast(ESMC_VM vm,
                   void *bcstData,
                   int count,
                   enum ESMC_TypeKind_Flag *typekind,
                   int rootPet);\end{verbatim}{\em RETURN VALUE:}
\begin{verbatim}    Return code; equals ESMF_SUCCESS if there are no errors.\end{verbatim}
{\sf DESCRIPTION:\\ }


    Collective {\tt ESMC\_VM} communication call that broadcasts a contiguous 
    data array from {\tt rootPet} to all other PETs of the {\tt ESMC\_VM}
    object.
  
    This method is overloaded for:
    {\tt ESMC\_TYPEKIND\_I4}, {\tt ESMC\_TYPEKIND\_I8},
    {\tt ESMC\_TYPEKIND\_R4}, {\tt ESMC\_TYPEKIND\_R8}, 
    {\tt ESMC\_TYPEKIND\_LOGICAL}.
  
    The arguments are:
    \begin{description}
    \item[vm] 
      {\tt ESMC\_VM} object.
    \item[bcstData]
      Contiguous data array. On {\tt rootPet} {\tt bcstData} holds data that
      is to be broadcasted to all other PETs. On all other PETs 
      {\tt bcstData} is used to receive the broadcasted data.
    \item[count] 
      Number of elements in {\tt bcstData}. Must be the same on all PETs.
    \item[typekind]
      The typekind of the data to be reduced. See section 
      \ref{const:ctypekind} for a list of valid typekind options.
    \item[rootPet] 
      PET that holds data that is being broadcast.
    \end{description}
   
%/////////////////////////////////////////////////////////////
 
\mbox{}\hrulefill\ 
 
\subsubsection [ESMC\_VMGet] {ESMC\_VMGet - Get VM internals}


  
\bigskip{\sf INTERFACE:}
\begin{verbatim} int ESMC_VMGet(
   ESMC_VM vm,                   // in
   int *localPet,                // out
   int *petCount,                // out
   int *peCount,                 // out
   MPI_Comm *mpiCommunicator,    // out
   int *pthreadsEnabledFlag,     // out
   int *openMPEnabledFlag        // out
 );\end{verbatim}{\em RETURN VALUE:}
\begin{verbatim}    Return code; equals ESMF_SUCCESS if there are no errors.\end{verbatim}
{\sf DESCRIPTION:\\ }


  
    Get internal information about the specified {\tt ESMC\_VM} object.
  
    The arguments are:
    \begin{description}
    \item[vm] 
      Queried {\tt ESMC\_VM} object.
    \item[{[localPet]}] 
      Upon return this holds the id of the PET that issued this call.
    \item[{[petCount]}]
      Upon return this holds the number of PETs in the specified {\tt ESMC\_VM}
      object.
    \item[{[peCount]}]
      Upon return this holds the number of PEs referenced by the specified
      {\tt ESMC\_VM} object.
    \item[{[mpiCommunicator]}]
      Upon return this holds the MPI intra-communicator used by the 
      specified {\tt ESMC\_VM} object. This communicator may be used for
      user-level MPI communications. It is recommended that the user
      duplicates the communicator via {\tt MPI\_Comm\_Dup()} in order to
      prevent any interference with ESMC communications.
    \item[{[pthreadsEnabledFlag]}]
      A return value of '1' indicates that the ESMF library was compiled with
      Pthreads enabled. A return value of '0' indicates that Pthreads are
      disabled in the ESMF library.
    \item[{[openMPEnabledFlag]}]
      A return value of '1' indicates that the ESMF library was compiled with
      OpenMP enabled. A return value of '0' indicates that OpenMP is
      disabled in the ESMF library.
    \item[{[rc]}]
      Return code; equals {\tt ESMF\_SUCCESS} if there are no errors.
    \end{description}
   
%/////////////////////////////////////////////////////////////
 
\mbox{}\hrulefill\ 
 
\subsubsection [ESMC\_VMGetCurrent] {ESMC\_VMGetCurrent - Get current VM}


  
\bigskip{\sf INTERFACE:}
\begin{verbatim} ESMC_VM ESMC_VMGetCurrent(
   int *rc                     // out
 );\end{verbatim}{\em RETURN VALUE:}
\begin{verbatim}    VM object of the current execution context.\end{verbatim}
{\sf DESCRIPTION:\\ }


  
   \begin{sloppypar}
    Get the {\tt ESMC\_VM} object of the current execution context. Calling
    {\tt ESMC\_VMGetCurrent()} within an ESMC Component, will return the
    same VM object as {\tt ESMC\_GridCompGet(..., vm=vm, ...)} or
    {\tt ESMC\_CplCompGet(..., vm=vm, ...)}. 
   \end{sloppypar}
  
    The main purpose of providing {\tt ESMC\_VMGetCurrent()} is to simplify ESMF
    adoption in legacy code. Specifically, code that uses {\tt MPI\_COMM\_WORLD}
    deep within its calling tree can easily be modified to use the correct MPI
    communicator of the current ESMF execution context. The advantage is that
    these modifications are very local, and do not require wide reaching
    interface changes in the legacy code to pass down the ESMF component object,
    or the MPI communicator.
  
    The use of {\tt ESMC\_VMGetCurrent()} is strongly discouraged in newly
    written Component code. Instead, the ESMF Component object should be used as
    the appropriate container of ESMF context information. This object should be
    passed between the subroutines of a Component, and be queried for any
    Component specific information.
  
    Outside of a Component context, i.e. within the driver context, the call
    to {\tt ESMC\_VMGetCurrent()} is identical to {\tt ESMC\_VMGetGlobal()}.
    \newline
  
    The arguments are:
    \begin{description}
    \item[{[rc]}]
      Return code; equals {\tt ESMF\_SUCCESS} if there are no errors.
    \end{description}
   
%/////////////////////////////////////////////////////////////
 
\mbox{}\hrulefill\ 
 
\subsubsection [ESMC\_VMGetGlobal] {ESMC\_VMGetGlobal - Get global VM}


  
\bigskip{\sf INTERFACE:}
\begin{verbatim} ESMC_VM ESMC_VMGetGlobal(
   int *rc                     // out
 );\end{verbatim}{\em RETURN VALUE:}
\begin{verbatim}    VM object of the global execution context.\end{verbatim}
{\sf DESCRIPTION:\\ }


  
    Get the global {\tt ESMC\_VM} object. This is the VM object
    that is created during {\tt ESMC\_Initialize()} and is the ultimate
    parent of all VM objects in an ESMF application. It is identical to the VM
    object returned by {\tt ESMC\_Initialize(..., vm=vm, ...)}.
  
    The {\tt ESMC\_VMGetGlobal()} call provides access to information about the
    global execution context via the global VM. This call is necessary because
    ESMF does not create a global ESMF Component during
    {\tt ESMC\_Initialize()} that could be queried for information about
    the global execution context of an ESMF application.
    
    Usage of {\tt ESMC\_VMGetGlobal()} from within Component code is
    strongly discouraged. ESMF Components should only access their own VM
    objects through Component methods. Global information, if required by
    the Component user code, should be passed down to the Component from the 
    driver through the Component calling interface.
    \newline
  
    The arguments are:
    \begin{description}
    \item[{[rc]}]
      Return code; equals {\tt ESMF\_SUCCESS} if there are no errors.
    \end{description}
   
%/////////////////////////////////////////////////////////////
 
\mbox{}\hrulefill\ 
 
\subsubsection [ESMC\_VMPrint] {ESMC\_VMPrint - Print a VM}


  
\bigskip{\sf INTERFACE:}
\begin{verbatim} int ESMC_VMPrint(
   ESMC_VM vm                // in
 );\end{verbatim}{\em RETURN VALUE:}
\begin{verbatim}    Return code; equals ESMF_SUCCESS if there are no errors.\end{verbatim}
{\sf DESCRIPTION:\\ }


  
    Print internal information of the specified {\tt ESMC\_VM} object.
  
    The arguments are:
    \begin{description}
    \item[vm] 
      {\tt ESMC\_VM} object to be printed.
    \end{description}
   
%/////////////////////////////////////////////////////////////
 
\mbox{}\hrulefill\ 
 
\subsubsection [ESMC\_VMReduce] {ESMC\_VMReduce - Reduce data from across the VM}


  
\bigskip{\sf INTERFACE:}
\begin{verbatim} int ESMC_VMReduce(ESMC_VM vm,
                   void *sendData,
                   void *recvData,
                   int count,
                   enum ESMC_TypeKind_Flag *typekind,
                   enum ESMC_Reduce_Flag *reduceflag,
                   int rootPet);\end{verbatim}{\em RETURN VALUE:}
\begin{verbatim}    Return code; equals ESMF_SUCCESS if there are no errors.\end{verbatim}
{\sf DESCRIPTION:\\ }


  
    Collective {\tt ESMC\_VM} communication call that reduces a contiguous data 
    array across the {\tt ESMC\_VM} object into a contiguous data array of 
    the same <type><kind>. The result array is returned on {\tt rootPet}. 
    Different reduction operations can be specified.
  
    This method is overloaded for:
    {\tt ESMC\_TYPEKIND\_I4}, {\tt ESMC\_TYPEKIND\_I8},
    {\tt ESMC\_TYPEKIND\_R4}, {\tt ESMC\_TYPEKIND\_R8}.
  
    The arguments are:
    \begin{description}
    \item[vm] 
      {\tt ESMC\_VM} object.
    \item[sendData]
      Contiguous data array holding data to be sent. All PETs must specify a
      valid source array.
    \item[recvData] 
      Contiguous data array for data to be received. Only the {\tt recvData}
      array specified by the {\tt rootPet} will be used by this method.
    \item[count] 
      Number of elements in sendData and recvData. Must be the same on all PETs.
    \item[typekind]
      The typekind of the data to be reduced. See section 
      \ref{const:ctypekind} for a list of valid typekind options.
    \item[reduceflag] 
      Reduction operation. See section \ref{const:creduce} for a list of 
      valid reduce operations.
    \item[rootPet] 
      PET on which reduced data is returned.
    \end{description}
  
%...............................................................
\setlength{\parskip}{\oldparskip}
\setlength{\parindent}{\oldparindent}
\setlength{\baselineskip}{\oldbaselineskip}
