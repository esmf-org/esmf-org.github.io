%                **** IMPORTANT NOTICE *****
% This LaTeX file has been automatically produced by ProTeX v. 1.1
% Any changes made to this file will likely be lost next time
% this file is regenerated from its source. Send questions 
% to Arlindo da Silva, dasilva@gsfc.nasa.gov
 
\setlength{\oldparskip}{\parskip}
\setlength{\parskip}{1.5ex}
\setlength{\oldparindent}{\parindent}
\setlength{\parindent}{0pt}
\setlength{\oldbaselineskip}{\baselineskip}
\setlength{\baselineskip}{11pt}
 
%--------------------- SHORT-HAND MACROS ----------------------
\def\bv{\begin{verbatim}}
\def\ev{\end{verbatim}}
\def\be{\begin{equation}}
\def\ee{\end{equation}}
\def\bea{\begin{eqnarray}}
\def\eea{\end{eqnarray}}
\def\bi{\begin{itemize}}
\def\ei{\end{itemize}}
\def\bn{\begin{enumerate}}
\def\en{\end{enumerate}}
\def\bd{\begin{description}}
\def\ed{\end{description}}
\def\({\left (}
\def\){\right )}
\def\[{\left [}
\def\]{\right ]}
\def\<{\left  \langle}
\def\>{\right \rangle}
\def\cI{{\cal I}}
\def\diag{\mathop{\rm diag}}
\def\tr{\mathop{\rm tr}}
%-------------------------------------------------------------

\markboth{Left}{Source File: ESMF\_VMHigherRankDataEx.F90,  Date: Tue May  5 20:59:29 MDT 2020
}

 
%/////////////////////////////////////////////////////////////

  
   \subsubsection{Using VM communication methods with data of rank greater than one}
   \label{vm_higherrank}
  
   In the current implementation of the VM communication methods all the data
   array arguments are declared as {\em assumed shape} dummy arrays of rank one.
   The assumed shape flavor was chosen in order to minimize the chance of 
   copy in/out problems, associated with the other options for declaring the 
   dummy data arguments.
   However, currently the interfaces are not overloaded for higher ranks. This
   restriction requires that users that need to communicate data arrays with
   rank greater than one, must only pass the first dimension of the data array
   into the VM communication calls. Specifying the full size of the data arrays
   (considering {\em all} dimensions) ensure that the complete data is
   transferred in or out of the contiguous array memory. 
%/////////////////////////////////////////////////////////////

 \begin{verbatim}
  integer, allocatable:: sendData(:,:)
  integer, allocatable:: recvData(:,:,:,:)
 
\end{verbatim}
 
%/////////////////////////////////////////////////////////////

 \begin{verbatim}
  count1 = 5
  count2 = 8
  allocate(sendData(count1,count2)) ! 5 x 8 = 40 elements
  do j=1, count2
    do i=1, count1
      sendData(i,j) = localPet*100 + i + (j-1)*count1
    enddo
  enddo
  
  count1 = 2
  count2 = 5
  count3 = 1
  count4 = 4
  allocate(recvData(count1,count2,count3,count4)) ! 2 x 5 x 1 x 4 = 40 elements
  do l=1, count4
    do k=1, count3
      do j=1, count2
        do i=1, count1
          recvData(i,j,k,l) = 0
        enddo
      enddo
    enddo
  enddo
  
 
\end{verbatim}
 
%/////////////////////////////////////////////////////////////

 \begin{verbatim}
  if (localPet==src) then
    call ESMF_VMSend(vm, &
      sendData=sendData(:,1), & ! 1st dimension as contiguous array section
      count=count1*count2, &    ! total count of elements
      dstPet=dst, rc=rc)
  endif
 
\end{verbatim}
 
%/////////////////////////////////////////////////////////////

 \begin{verbatim}
  if (localPet==dst) then
    call ESMF_VMRecv(vm, &
      recvData=recvData(:,1,1,1), & ! 1st dimension as contiguous array section
      count=count1*count2*count3*count4, &  ! total count of elements
      srcPet=src, rc=rc)
  endif
 
\end{verbatim}

%...............................................................
\setlength{\parskip}{\oldparskip}
\setlength{\parindent}{\oldparindent}
\setlength{\baselineskip}{\oldbaselineskip}
