%                **** IMPORTANT NOTICE *****
% This LaTeX file has been automatically produced by ProTeX v. 1.1
% Any changes made to this file will likely be lost next time
% this file is regenerated from its source. Send questions 
% to Arlindo da Silva, dasilva@gsfc.nasa.gov
 
\setlength{\oldparskip}{\parskip}
\setlength{\parskip}{1.5ex}
\setlength{\oldparindent}{\parindent}
\setlength{\parindent}{0pt}
\setlength{\oldbaselineskip}{\baselineskip}
\setlength{\baselineskip}{11pt}
 
%--------------------- SHORT-HAND MACROS ----------------------
\def\bv{\begin{verbatim}}
\def\ev{\end{verbatim}}
\def\be{\begin{equation}}
\def\ee{\end{equation}}
\def\bea{\begin{eqnarray}}
\def\eea{\end{eqnarray}}
\def\bi{\begin{itemize}}
\def\ei{\end{itemize}}
\def\bn{\begin{enumerate}}
\def\en{\end{enumerate}}
\def\bd{\begin{description}}
\def\ed{\end{description}}
\def\({\left (}
\def\){\right )}
\def\[{\left [}
\def\]{\right ]}
\def\<{\left  \langle}
\def\>{\right \rangle}
\def\cI{{\cal I}}
\def\diag{\mathop{\rm diag}}
\def\tr{\mathop{\rm tr}}
%-------------------------------------------------------------

\markboth{Left}{Source File: ESMF\_VMComponentEx.F90,  Date: Tue May  5 20:59:29 MDT 2020
}

 
%/////////////////////////////////////////////////////////////

  
   \subsubsection{VM and Components}
  
   The following example shows the role that the VM plays in connection with ESMF 
   Components. A single Component is created in the main program. Through the
   optional {\tt petList} argument the driver code specifies that only resources
   associated with PET 0 are given to the {\tt gcomp} object. 
  
   When the Component code is invoked through the standard ESMF Component methods
   Initialize, Run, or Finalize the Component's VM is automatically entered.
   Inside of the user-written Component code the Component VM can be obtained
   by querying the Component object. The VM object will indicate that only a
   single PET is executing the Component code.
   
%/////////////////////////////////////////////////////////////

 \begin{verbatim}
module ESMF_VMComponentEx_gcomp_mod
 
\end{verbatim}
 
%/////////////////////////////////////////////////////////////

 \begin{verbatim}
  recursive subroutine mygcomp_init(gcomp, istate, estate, clock, rc)
    type(ESMF_GridComp)   :: gcomp
    type(ESMF_State)      :: istate, estate
    type(ESMF_Clock)      :: clock
    integer, intent(out)  :: rc

    ! local variables
    type(ESMF_VM):: vm
    
    ! get this Component's vm    
    call ESMF_GridCompGet(gcomp, vm=vm)
    
    ! the VM object contains information about the execution environment of
    ! the Component

    call ESMF_VMPrint(vm, rc=rc)
    
    rc = 0
  end subroutine !--------------------------------------------------------------

  
  recursive subroutine mygcomp_run(gcomp, istate, estate, clock, rc)
    type(ESMF_GridComp)   :: gcomp
    type(ESMF_State)      :: istate, estate
    type(ESMF_Clock)      :: clock
    integer, intent(out)  :: rc
    
    ! local variables
    type(ESMF_VM):: vm
    
    ! get this Component's vm    
    call ESMF_GridCompGet(gcomp, vm=vm)
    
    ! the VM object contains information about the execution environment of
    ! the Component

    call ESMF_VMPrint(vm, rc=rc)
    
    rc = 0
  end subroutine !--------------------------------------------------------------

  recursive subroutine mygcomp_final(gcomp, istate, estate, clock, rc)
    type(ESMF_GridComp)   :: gcomp
    type(ESMF_State)      :: istate, estate
    type(ESMF_Clock)      :: clock
    integer, intent(out)  :: rc
    
    ! local variables
    type(ESMF_VM):: vm
    
    ! get this Component's vm    
    call ESMF_GridCompGet(gcomp, vm=vm)
    
    ! the VM object contains information about the execution environment of
    ! the Component

    call ESMF_VMPrint(vm, rc=rc)
    
    rc = 0
  end subroutine !--------------------------------------------------------------

end module
 
\end{verbatim}
 
%/////////////////////////////////////////////////////////////

 \begin{verbatim}
program ESMF_VMComponentEx
#include "ESMF.h"
  use ESMF
  use ESMF_TestMod
  use ESMF_VMComponentEx_gcomp_mod
  implicit none
  
  ! local variables
 
\end{verbatim}
 
%/////////////////////////////////////////////////////////////

 \begin{verbatim}
  gcomp = ESMF_GridCompCreate(petList=(/0/), rc=rc)
 
\end{verbatim}
 
%/////////////////////////////////////////////////////////////

 \begin{verbatim}
  call ESMF_GridCompSetServices(gcomp, userRoutine=mygcomp_register, rc=rc)
 
\end{verbatim}
 
%/////////////////////////////////////////////////////////////

 \begin{verbatim}
  call ESMF_GridCompInitialize(gcomp, rc=rc)
 
\end{verbatim}
 
%/////////////////////////////////////////////////////////////

 \begin{verbatim}
  call ESMF_GridCompRun(gcomp, rc=rc)
 
\end{verbatim}
 
%/////////////////////////////////////////////////////////////

 \begin{verbatim}
  call ESMF_GridCompFinalize(gcomp, rc=rc)
 
\end{verbatim}
 
%/////////////////////////////////////////////////////////////

 \begin{verbatim}
  call ESMF_GridCompDestroy(gcomp, rc=rc)
 
\end{verbatim}
 
%/////////////////////////////////////////////////////////////

 \begin{verbatim}
  call ESMF_Finalize(rc=rc)
 
\end{verbatim}
 
%/////////////////////////////////////////////////////////////

 \begin{verbatim}
end program
 
\end{verbatim}

%...............................................................
\setlength{\parskip}{\oldparskip}
\setlength{\parindent}{\oldparindent}
\setlength{\baselineskip}{\oldbaselineskip}
