%                **** IMPORTANT NOTICE *****
% This LaTeX file has been automatically produced by ProTeX v. 1.1
% Any changes made to this file will likely be lost next time
% this file is regenerated from its source. Send questions 
% to Arlindo da Silva, dasilva@gsfc.nasa.gov
 
\setlength{\oldparskip}{\parskip}
\setlength{\parskip}{1.5ex}
\setlength{\oldparindent}{\parindent}
\setlength{\parindent}{0pt}
\setlength{\oldbaselineskip}{\baselineskip}
\setlength{\baselineskip}{11pt}
 
%--------------------- SHORT-HAND MACROS ----------------------
\def\bv{\begin{verbatim}}
\def\ev{\end{verbatim}}
\def\be{\begin{equation}}
\def\ee{\end{equation}}
\def\bea{\begin{eqnarray}}
\def\eea{\end{eqnarray}}
\def\bi{\begin{itemize}}
\def\ei{\end{itemize}}
\def\bn{\begin{enumerate}}
\def\en{\end{enumerate}}
\def\bd{\begin{description}}
\def\ed{\end{description}}
\def\({\left (}
\def\){\right )}
\def\[{\left [}
\def\]{\right ]}
\def\<{\left  \langle}
\def\>{\right \rangle}
\def\cI{{\cal I}}
\def\diag{\mathop{\rm diag}}
\def\tr{\mathop{\rm tr}}
%-------------------------------------------------------------

\markboth{Left}{Source File: ESMF\_VMDefaultBasicsEx.F90,  Date: Tue May  5 20:59:29 MDT 2020
}

 
%/////////////////////////////////////////////////////////////

  
   \subsubsection{Global VM}
  
   This complete example program demonstrates the simplest ESMF application, 
   consisting of only a main program without any Components. The global
   VM, which is automatically created during the {\tt ESMF\_Initialize()} call,
   is obtained using two different methods. First the global VM will be returned
   by {\tt ESMF\_Initialize()} if the optional {\tt vm} argument is specified.
   The example uses the VM object obtained this way to call the VM print method.
   Second, the global VM can be obtained anywhere in the user application using
   the {\tt ESMF\_VMGetGlobal()} call. The identical VM is returned and several
   VM query methods are called to inquire about the associated resources.
   
%/////////////////////////////////////////////////////////////

 \begin{verbatim}
program ESMF_VMDefaultBasicsEx
#include "ESMF.h"

  use ESMF
  use ESMF_TestMod
  
  implicit none
  
  ! local variables
  integer:: rc
  type(ESMF_VM):: vm
  integer:: localPet, petCount, peCount, ssiId, vas
 
\end{verbatim}
 
%/////////////////////////////////////////////////////////////

 \begin{verbatim}
  call ESMF_Initialize(vm=vm, defaultlogfilename="VMDefaultBasicsEx.Log", &
                    logkindflag=ESMF_LOGKIND_MULTI, rc=rc)
  ! Providing the optional vm argument to ESMF_Initialize() is one way of
  ! obtaining the global VM.
 
\end{verbatim}
 
%/////////////////////////////////////////////////////////////

 \begin{verbatim}
  call ESMF_VMPrint(vm, rc=rc)
 
\end{verbatim}
 
%/////////////////////////////////////////////////////////////

 \begin{verbatim}
  call ESMF_VMGetGlobal(vm=vm, rc=rc)
  ! Calling ESMF_VMGetGlobal() anywhere in the user application is the other
  ! way to obtain the global VM object.
 
\end{verbatim}
 
%/////////////////////////////////////////////////////////////

 \begin{verbatim}
  call ESMF_VMGet(vm, localPet=localPet, petCount=petCount, peCount=peCount, &
    rc=rc)
  ! The VM object contains information about the associated resources. If the
  ! user code requires this information it must query the VM object.
 
\end{verbatim}
 
%/////////////////////////////////////////////////////////////

 \begin{verbatim}
  print *, "This PET is localPet: ", localPet
  print *, "of a total of ",petCount," PETs in this VM."
  print *, "There are ", peCount," PEs referenced by this VM"

  call ESMF_VMGet(vm, localPet, peCount=peCount, ssiId=ssiId, vas=vas, rc=rc)
 
\end{verbatim}
 
%/////////////////////////////////////////////////////////////

 \begin{verbatim}
  print *, "This PET is executing in virtual address space (VAS) ", vas
  print *, "located on single system image (SSI) ", ssiId
  print *, "and is associated with ", peCount, " PEs."

 
\end{verbatim}
 
%/////////////////////////////////////////////////////////////

 \begin{verbatim}
end program
 
\end{verbatim}

%...............................................................
\setlength{\parskip}{\oldparskip}
\setlength{\parindent}{\oldparindent}
\setlength{\baselineskip}{\oldbaselineskip}
