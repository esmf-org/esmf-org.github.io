%                **** IMPORTANT NOTICE *****
% This LaTeX file has been automatically produced by ProTeX v. 1.1
% Any changes made to this file will likely be lost next time
% this file is regenerated from its source. Send questions 
% to Arlindo da Silva, dasilva@gsfc.nasa.gov
 
\setlength{\oldparskip}{\parskip}
\setlength{\parskip}{1.5ex}
\setlength{\oldparindent}{\parindent}
\setlength{\parindent}{0pt}
\setlength{\oldbaselineskip}{\baselineskip}
\setlength{\baselineskip}{11pt}
 
%--------------------- SHORT-HAND MACROS ----------------------
\def\bv{\begin{verbatim}}
\def\ev{\end{verbatim}}
\def\be{\begin{equation}}
\def\ee{\end{equation}}
\def\bea{\begin{eqnarray}}
\def\eea{\end{eqnarray}}
\def\bi{\begin{itemize}}
\def\ei{\end{itemize}}
\def\bn{\begin{enumerate}}
\def\en{\end{enumerate}}
\def\bd{\begin{description}}
\def\ed{\end{description}}
\def\({\left (}
\def\){\right )}
\def\[{\left [}
\def\]{\right ]}
\def\<{\left  \langle}
\def\>{\right \rangle}
\def\cI{{\cal I}}
\def\diag{\mathop{\rm diag}}
\def\tr{\mathop{\rm tr}}
%-------------------------------------------------------------

\markboth{Left}{Source File: ESMF\_VMUserMpiCommEx.F90,  Date: Tue May  5 20:59:29 MDT 2020
}

 
%/////////////////////////////////////////////////////////////

  
   \subsubsection{Nesting ESMF inside a user MPI application on a subset of MPI ranks}
   \label{vm_nesting_esmf}
  
   \begin{sloppypar}
   The previous example demonstrated that it is possible to nest an ESMF 
   application, i.e. {\tt ESMF\_Initialize()}...{\tt ESMF\_Finalize()} inside
   {\tt MPI\_Init()}...{\tt MPI\_Finalize()}. It is not necessary that all
   MPI ranks enter the ESMF application. The following example shows how the
   user code can pass an MPI communicator to {\tt ESMF\_Initialize()}, and
   enter the ESMF application on a subset of MPI ranks.
   \end{sloppypar}
   
%/////////////////////////////////////////////////////////////

 \begin{verbatim}
  ! User code initializes MPI.
  call MPI_Init(ierr)
 
\end{verbatim}
 
%/////////////////////////////////////////////////////////////

 \begin{verbatim}
  ! User code determines the local rank.
  call MPI_Comm_rank(MPI_COMM_WORLD, rank, ierr)
 
\end{verbatim}
 
%/////////////////////////////////////////////////////////////

 \begin{verbatim}
  ! User code prepares MPI communicator "esmfComm", that allows rank 0 and 1
  ! to be grouped together.
  if (rank < 2) then
    ! first communicator split with color=0
    call MPI_Comm_split(MPI_COMM_WORLD, 0, 0, esmfComm, ierr)
  else
    ! second communicator split with color=1
    call MPI_Comm_split(MPI_COMM_WORLD, 1, 0, esmfComm, ierr)
  endif
 
\end{verbatim}
 
%/////////////////////////////////////////////////////////////

 \begin{verbatim}
  if (rank < 2) then
    ! Only call ESMF_Initialize() on rank 0 and 1, passing the prepared MPI
    ! communicator that spans these ranks.
    call ESMF_Initialize(mpiCommunicator=esmfComm, &
      defaultlogfilename="VMUserMpiCommEx.Log", &
      logkindflag=ESMF_LOGKIND_MULTI, rc=rc)
 
\end{verbatim}
 
%/////////////////////////////////////////////////////////////

 \begin{verbatim}
    ! Use ESMF here...
 
\end{verbatim}
 
%/////////////////////////////////////////////////////////////

 \begin{verbatim}
    ! Calling ESMF_Finalize() with endflag=ESMF_END_KEEPMPI instructs ESMF
    ! to keep MPI active.
    call ESMF_Finalize(endflag=ESMF_END_KEEPMPI, rc=rc)
 
\end{verbatim}
 
%/////////////////////////////////////////////////////////////

 \begin{verbatim}
  else
    ! Ranks 2 and above do non-ESMF work...
 
\end{verbatim}
 
%/////////////////////////////////////////////////////////////

 \begin{verbatim}
  endif
 
\end{verbatim}
 
%/////////////////////////////////////////////////////////////

 \begin{verbatim}
  ! Free the MPI communicator before finalizing MPI.
  call MPI_Comm_free(esmfComm, ierr)
  
  ! It is the responsibility of the outer user code to finalize MPI.
  call MPI_Finalize(ierr)
 
\end{verbatim}

%...............................................................
\setlength{\parskip}{\oldparskip}
\setlength{\parindent}{\oldparindent}
\setlength{\baselineskip}{\oldbaselineskip}
