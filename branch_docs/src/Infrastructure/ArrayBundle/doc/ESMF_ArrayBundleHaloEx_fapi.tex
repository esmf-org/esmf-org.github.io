%                **** IMPORTANT NOTICE *****
% This LaTeX file has been automatically produced by ProTeX v. 1.1
% Any changes made to this file will likely be lost next time
% this file is regenerated from its source. Send questions 
% to Arlindo da Silva, dasilva@gsfc.nasa.gov
 
\setlength{\oldparskip}{\parskip}
\setlength{\parskip}{1.5ex}
\setlength{\oldparindent}{\parindent}
\setlength{\parindent}{0pt}
\setlength{\oldbaselineskip}{\baselineskip}
\setlength{\baselineskip}{11pt}
 
%--------------------- SHORT-HAND MACROS ----------------------
\def\bv{\begin{verbatim}}
\def\ev{\end{verbatim}}
\def\be{\begin{equation}}
\def\ee{\end{equation}}
\def\bea{\begin{eqnarray}}
\def\eea{\end{eqnarray}}
\def\bi{\begin{itemize}}
\def\ei{\end{itemize}}
\def\bn{\begin{enumerate}}
\def\en{\end{enumerate}}
\def\bd{\begin{description}}
\def\ed{\end{description}}
\def\({\left (}
\def\){\right )}
\def\[{\left [}
\def\]{\right ]}
\def\<{\left  \langle}
\def\>{\right \rangle}
\def\cI{{\cal I}}
\def\diag{\mathop{\rm diag}}
\def\tr{\mathop{\rm tr}}
%-------------------------------------------------------------

\markboth{Left}{Source File: ESMF\_ArrayBundleHaloEx.F90,  Date: Tue May  5 20:59:45 MDT 2020
}

 
%/////////////////////////////////////////////////////////////

   \subsubsection{Halo communication}
  
   One of the most fundamental communication pattern in domain decomposition
   codes is the {\em halo} operation. The ESMF Array class supports halos
   by allowing memory for extra elements to be allocated on each DE. See
   section \ref{Array:Halo} for a discussion of the Array level halo operation.
   The ArrayBundle level extents the Array halo operation to bundles of Arrays.
  
   First create an {\tt ESMF\_ArrayBundle} object containing a set of ESMF
   Arrays. 
%/////////////////////////////////////////////////////////////

 \begin{verbatim}
  arraybundle = ESMF_ArrayBundleCreate(arrayList=arrayList, &
    name="MyArrayBundle", rc=rc)
 
\end{verbatim}
 
%/////////////////////////////////////////////////////////////

   The ArrayBundle object can be treated as a single entity. The
   {\tt ESMF\_ArrayBundleHaloStore()} call determines the most efficient
   halo exchange pattern for {\em all} Arrays that are part of
   {\tt arraybundle}. 
%/////////////////////////////////////////////////////////////

 \begin{verbatim}
  call ESMF_ArrayBundleHaloStore(arraybundle=arraybundle, &
    routehandle=haloHandle, rc=rc)
 
\end{verbatim}
 
%/////////////////////////////////////////////////////////////

   The halo exchange pattern stored in {\tt haloHandle} can now be applied to
   the {\tt arraybundle} object, or any other ArrayBundle that is compatible
   to the one used during the {\tt ESMF\_ArrayBundleHaloStore()} call. 
%/////////////////////////////////////////////////////////////

 \begin{verbatim}
  call ESMF_ArrayBundleHalo(arraybundle=arraybundle, routehandle=haloHandle, &
    rc=rc)
 
\end{verbatim}
 
%/////////////////////////////////////////////////////////////

   Finally, when no longer needed, the resources held by {\tt haloHandle} need
   to be returned to the system by calling {\tt ESMF\_ArrayBundleHaloRelease()}. 
%/////////////////////////////////////////////////////////////

 \begin{verbatim}
  call ESMF_ArrayBundleHaloRelease(routehandle=haloHandle, rc=rc)
 
\end{verbatim}
 
%/////////////////////////////////////////////////////////////

   Finally the ArrayBundle object can be destroyed. 
%/////////////////////////////////////////////////////////////

 \begin{verbatim}
  call ESMF_ArrayBundleDestroy(arraybundle, rc=rc)
 
\end{verbatim}

%...............................................................
\setlength{\parskip}{\oldparskip}
\setlength{\parindent}{\oldparindent}
\setlength{\baselineskip}{\oldbaselineskip}
