%                **** IMPORTANT NOTICE *****
% This LaTeX file has been automatically produced by ProTeX v. 1.1
% Any changes made to this file will likely be lost next time
% this file is regenerated from its source. Send questions 
% to Arlindo da Silva, dasilva@gsfc.nasa.gov
 
\setlength{\oldparskip}{\parskip}
\setlength{\parskip}{1.5ex}
\setlength{\oldparindent}{\parindent}
\setlength{\parindent}{0pt}
\setlength{\oldbaselineskip}{\baselineskip}
\setlength{\baselineskip}{11pt}
 
%--------------------- SHORT-HAND MACROS ----------------------
\def\bv{\begin{verbatim}}
\def\ev{\end{verbatim}}
\def\be{\begin{equation}}
\def\ee{\end{equation}}
\def\bea{\begin{eqnarray}}
\def\eea{\end{eqnarray}}
\def\bi{\begin{itemize}}
\def\ei{\end{itemize}}
\def\bn{\begin{enumerate}}
\def\en{\end{enumerate}}
\def\bd{\begin{description}}
\def\ed{\end{description}}
\def\({\left (}
\def\){\right )}
\def\[{\left [}
\def\]{\right ]}
\def\<{\left  \langle}
\def\>{\right \rangle}
\def\cI{{\cal I}}
\def\diag{\mathop{\rm diag}}
\def\tr{\mathop{\rm tr}}
%-------------------------------------------------------------

\markboth{Left}{Source File: ESMF\_LocStreamEx.F90,  Date: Tue May  5 20:59:56 MDT 2020
}

 
%/////////////////////////////////////////////////////////////

  \subsubsection{Create a LocStream with user allocated memory}
  
   The following is an example of creating a LocStream object.
   After creation, key data is added, and a Field is created to hold data
   (temperature) at each location. 
   
%/////////////////////////////////////////////////////////////

 \begin{verbatim}

   !-------------------------------------------------------------------
   ! Get parallel information. Here petCount is the total number of 
   ! running PETs, and localPet is the number of this particular PET.
   !-------------------------------------------------------------------
   call ESMF_VMGet(vm, localPet=localPet, petCount=petCount, rc=rc)

 
\end{verbatim}
 
%/////////////////////////////////////////////////////////////

 \begin{verbatim}

   !-------------------------------------------------------------------
   ! Allocate and set example location information. Locations on a PET
   ! are wrapped around sphere. Each PET occupies a different latitude
   ! ranging from +50.0 to -50.0.
   !-------------------------------------------------------------------
   numLocations = 20
   allocate(lon(numLocations))
   allocate(lat(numLocations))

   do i=1,numLocations
      lon(i)=360.0*i/numLocations
      lat(i)=100*REAL(localPet,ESMF_KIND_R8)/REAL(petCount,ESMF_KIND_R8)-50.0
   enddo

   !-------------------------------------------------------------------
   ! Allocate and set example Field data
   !-------------------------------------------------------------------
   allocate(temperature(numLocations))

   do i=1,numLocations
      temperature(i)= 300 - abs(lat(i))
   enddo

   !-------------------------------------------------------------------
   ! Create the LocStream:  Allocate space for the LocStream object, 
   ! define the number and distribution of the locations. 
   !-------------------------------------------------------------------
   locstream=ESMF_LocStreamCreate(name="Temperature Measurements",   &
                                  localCount=numLocations, &
                                  coordSys=ESMF_COORDSYS_SPH_DEG,   &
                                  rc=rc)
 
\end{verbatim}
 
%/////////////////////////////////////////////////////////////

 \begin{verbatim}

   !-------------------------------------------------------------------
   ! Add key data, referencing a user data pointer. By changing the 
   ! datacopyflag to ESMF_DATACOPY_VALUE an internally allocated copy of the 
   ! user data may also be set.  
   !-------------------------------------------------------------------
   call ESMF_LocStreamAddKey(locstream,              &
                             keyName="ESMF:Lat",     &
                             farray=lat,             &
                             datacopyflag=ESMF_DATACOPY_REFERENCE, &
                             keyUnits="Degrees",     &
                             keyLongName="Latitude", rc=rc)
 
\end{verbatim}
 
%/////////////////////////////////////////////////////////////

 \begin{verbatim}

   call ESMF_LocStreamAddKey(locstream,              &
                             keyName="ESMF:Lon",     &
                             farray=lon,             &
                             datacopyflag=ESMF_DATACOPY_REFERENCE, &
                             keyUnits="Degrees",     &
                             keyLongName="Longitude", rc=rc)

 
\end{verbatim}
 
%/////////////////////////////////////////////////////////////

 \begin{verbatim}

   !-------------------------------------------------------------------
   ! Create a Field on the Location Stream. In this case the 
   ! Field is created from a user array, but any of the other
   ! Field create methods (e.g. from ArraySpec) would also apply.
   !-------------------------------------------------------------------       
   field_temperature=ESMF_FieldCreate(locstream,   &
                                   temperature, &
                                   name="temperature", &
                                   rc=rc)


 
\end{verbatim}
 
%/////////////////////////////////////////////////////////////

  \subsubsection{Create a LocStream with internally allocated memory}
  
   The following is an example of creating a LocStream object.
   After creation, key data is internally allocated,
   the pointer is retrieved, and the data is set.
   A Field is also created on the LocStream to hold data
   (temperature) at each location. 
   
%/////////////////////////////////////////////////////////////

 \begin{verbatim}

   !-------------------------------------------------------------------
   ! Get parallel information. Here petCount is the total number of 
   ! running PETs, and localPet is the number of this particular PET.
   !-------------------------------------------------------------------
   call ESMF_VMGet(vm, localPet=localPet, petCount=petCount, rc=rc)

 
\end{verbatim}
 
%/////////////////////////////////////////////////////////////

 \begin{verbatim}
   numLocations = 20

   !-------------------------------------------------------------------
   ! Create the LocStream:  Allocate space for the LocStream object, 
   ! define the number and distribution of the locations. 
   !-------------------------------------------------------------------
   locstream=ESMF_LocStreamCreate(name="Temperature Measurements",   &
                                  localCount=numLocations, &
                                  coordSys=ESMF_COORDSYS_SPH_DEG,   &
                                  rc=rc)
 
\end{verbatim}
 
%/////////////////////////////////////////////////////////////

 \begin{verbatim}


   !-------------------------------------------------------------------
   ! Add key data (internally allocating memory).
   !-------------------------------------------------------------------
   call ESMF_LocStreamAddKey(locstream,                    &
                             keyName="ESMF:Lat",           &
                             KeyTypeKind=ESMF_TYPEKIND_R8, &
                             keyUnits="Degrees",           &
                             keyLongName="Latitude", rc=rc)
 
\end{verbatim}
 
%/////////////////////////////////////////////////////////////

 \begin{verbatim}

   call ESMF_LocStreamAddKey(locstream,                    &
                             keyName="ESMF:Lon",           &
                             KeyTypeKind=ESMF_TYPEKIND_R8, &
                             keyUnits="Degrees",           &
                             keyLongName="Longitude", rc=rc)

 
\end{verbatim}
 
%/////////////////////////////////////////////////////////////

 \begin{verbatim}

   !-------------------------------------------------------------------
   ! Get key data. 
   !-------------------------------------------------------------------
   call ESMF_LocStreamGetKey(locstream,                    &
                             keyName="ESMF:Lat",           &
                             farray=lat,                   &
                             rc=rc)
 
\end{verbatim}
 
%/////////////////////////////////////////////////////////////

 \begin{verbatim}

   call ESMF_LocStreamGetKey(locstream,                    &
                             keyName="ESMF:Lon",           &
                             farray=lon,                   &
                             rc=rc)
 
\end{verbatim}
 
%/////////////////////////////////////////////////////////////

 \begin{verbatim}

   !-------------------------------------------------------------------
   ! Set example location information. Locations on a PET are wrapped 
   ! around sphere. Each PET occupies a different latitude ranging 
   ! from +50.0 to -50.0.
   !-------------------------------------------------------------------
   do i=1,numLocations
      lon(i)=360.0*i/numLocations
      lat(i)=100*REAL(localPet,ESMF_KIND_R8)/REAL(petCount,ESMF_KIND_R8)-50.0
   enddo


   !-------------------------------------------------------------------
   ! Allocate and set example Field data
   !-------------------------------------------------------------------
   allocate(temperature(numLocations))
   do i=1,numLocations
      temperature(i)= 300 - abs(lat(i))
   enddo

   !-------------------------------------------------------------------
   ! Create a Field on the Location Stream. In this case the 
   ! Field is created from a user array, but any of the other
   ! Field create methods (e.g. from ArraySpec) would also apply.
   !-------------------------------------------------------------------    
   field_temperature=ESMF_FieldCreate(locstream,   &
                                 temperature, &
                                 name="temperature", &
                                 rc=rc)


 
\end{verbatim}
 
%/////////////////////////////////////////////////////////////

  \subsubsection{Create a LocStream with a distribution based on a Grid}
  
   The following is an example of using the LocStream create from background
   Grid capability. Using this capability, the newly created LocStream 
   is a copy of the old LocStream, but with a new distribution. The new LocStream 
   is distributed such that if the coordinates of a location in the LocStream lie 
   within a Grid cell, then that location is put on the same PET as the Grid cell. 
   
%/////////////////////////////////////////////////////////////

 \begin{verbatim}

   !-------------------------------------------------------------------
   ! Get parallel information. Here petCount is the total number of 
   ! running PETs, and localPet is the number of this particular PET.
   !-------------------------------------------------------------------
   call ESMF_VMGet(vm, localPet=localPet, petCount=petCount, rc=rc)

 
\end{verbatim}
 
%/////////////////////////////////////////////////////////////

 \begin{verbatim}
   !-------------------------------------------------------------------
   ! Create the LocStream:  Allocate space for the LocStream object, 
   ! define the number and distribution of the locations. 
   !-------------------------------------------------------------------
   numLocations = 20
   locstream=ESMF_LocStreamCreate(name="Temperature Measurements",   &
                                  localCount=numLocations, &
                                  coordSys=ESMF_COORDSYS_SPH_DEG,   &
                                  rc=rc)
 
\end{verbatim}
 
%/////////////////////////////////////////////////////////////

 \begin{verbatim}

   !-------------------------------------------------------------------
   ! Add key data (internally allocating memory).
   !-------------------------------------------------------------------
   call ESMF_LocStreamAddKey(locstream,                    &
                             keyName="ESMF:Lon",           &
                             KeyTypeKind=ESMF_TYPEKIND_R8, &
                             keyUnits="Degrees",           &
                             keyLongName="Longitude", rc=rc)

 
\end{verbatim}
 
%/////////////////////////////////////////////////////////////

 \begin{verbatim}

   call ESMF_LocStreamAddKey(locstream,                    &
                             keyName="ESMF:Lat",           &
                             KeyTypeKind=ESMF_TYPEKIND_R8, &
                             keyUnits="Degrees",           &
                             keyLongName="Latitude", rc=rc)

 
\end{verbatim}
 
%/////////////////////////////////////////////////////////////

 \begin{verbatim}


   !-------------------------------------------------------------------
   ! Get Fortran arrays which hold the key data, so that it can be set. 
   !-------------------------------------------------------------------
   call ESMF_LocStreamGetKey(locstream,                    &
                             keyName="ESMF:Lon",           &
                             farray=lon,                   &
                             rc=rc)
 
\end{verbatim}
 
%/////////////////////////////////////////////////////////////

 \begin{verbatim}

   call ESMF_LocStreamGetKey(locstream,                    &
                             keyName="ESMF:Lat",           &
                             farray=lat,                   &
                             rc=rc)

 
\end{verbatim}
 
%/////////////////////////////////////////////////////////////

 \begin{verbatim}

   !-------------------------------------------------------------------
   ! Set example location information. Locations on a PET are wrapped 
   ! around sphere. Each PET occupies a different latitude ranging 
   ! from +50.0 to -50.0.
   !-------------------------------------------------------------------
   do i=1,numLocations
      lon(i)=360.0*i/numLocations
      lat(i)=100*REAL(localPet,ESMF_KIND_R8)/REAL(petCount,ESMF_KIND_R8)-50.0
   enddo

   !-------------------------------------------------------------------
   ! Create a Grid to use as the background. The Grid is 
   ! GridLonSize by GridLatSize with the default distribution 
   ! (The first dimension split across the PETs). The coordinate range
   ! is  0 to 360 in longitude and -90 to 90 in latitude. Note that we 
   ! use indexflag=ESMF_INDEX_GLOBAL for the Grid creation. At this time 
   ! this is required for a Grid to be usable as a background Grid.
   ! Note that here the points are treated as cartesian.
   !-------------------------------------------------------------------
   grid=ESMF_GridCreateNoPeriDim(maxIndex=(/GridLonSize,GridLatSize/), &
                                 coordSys=ESMF_COORDSYS_SPH_DEG,       &
                                 indexflag=ESMF_INDEX_GLOBAL,          &
                                 rc=rc)
 
\end{verbatim}
 
%/////////////////////////////////////////////////////////////

 \begin{verbatim}

   !-------------------------------------------------------------------
   ! Allocate the corner stagger location in which to put the coordinates. 
   ! (The corner stagger must be used for the Grid to be usable as a
   !  background Grid.)
   !-------------------------------------------------------------------
   call ESMF_GridAddCoord(grid, staggerloc=ESMF_STAGGERLOC_CORNER, rc=rc)

 
\end{verbatim}
 
%/////////////////////////////////////////////////////////////

 \begin{verbatim}

   !-------------------------------------------------------------------
   ! Get access to the Fortran array pointers that hold the Grid 
   ! coordinate information and then set the coordinates to be uniformly 
   ! distributed around the globe. 
   !-------------------------------------------------------------------
   call ESMF_GridGetCoord(grid,                                  &
                          staggerLoc=ESMF_STAGGERLOC_CORNER,     &
                          coordDim=1, computationalLBound=clbnd, &
                          computationalUBound=cubnd,             & 
                          farrayPtr=farrayPtrLonC, rc=rc)
 
\end{verbatim}
 
%/////////////////////////////////////////////////////////////

 \begin{verbatim}


   call ESMF_GridGetCoord(grid,                                  &
                         staggerLoc=ESMF_STAGGERLOC_CORNER,      &
                          coordDim=2, farrayPtr=farrayPtrLatC, rc=rc)

 
\end{verbatim}
 
%/////////////////////////////////////////////////////////////

 \begin{verbatim}

   do i1=clbnd(1),cubnd(1)
   do i2=clbnd(2),cubnd(2)
      ! Set Grid longitude coordinates as 0 to 360
      farrayPtrLonC(i1,i2) = REAL(i1-1)*360.0/REAL(GridLonSize)

      ! Set Grid latitude coordinates as -90 to 90
      farrayPtrLatC(i1,i2) = -90. + REAL(i2-1)*180.0/REAL(GridLatSize) + &
                                      0.5*180.0/REAL(GridLatSize)
   enddo
   enddo


   !-------------------------------------------------------------------
   ! Create newLocstream on the background Grid using the 
   ! "Lon" and "Lat" keys as the coordinates for the entries in 
   ! locstream. The entries in newLocstream with coordinates (lon,lat)
   ! are on the same PET as the piece of grid which contains (lon,lat). 
   !-------------------------------------------------------------------
   newLocstream=ESMF_LocStreamCreate(locstream, &
                  background=grid, rc=rc)


   !-------------------------------------------------------------------
   ! A Field can now be created on newLocstream and 
   ! ESMF_FieldRedist() can be used to move data between Fields built 
   ! on locstream and Fields built on newLocstream.
   !-------------------------------------------------------------------
 
\end{verbatim}
 
%/////////////////////////////////////////////////////////////

  \subsubsection{Regridding from a Grid to a LocStream}
  
   The following is an example of how a LocStream object can be used in regridding.
   
%/////////////////////////////////////////////////////////////

 \begin{verbatim}
   !-------------------------------------------------------------------
   ! Create a global Grid to use as the regridding source. The Grid is 
   ! GridLonSize by GridLatSize with the default distribution 
   ! (The first dimension split across the PETs). The coordinate range
   ! is  0 to 360 in longitude and -90 to 90 in latitude. Note that we 
   ! use indexflag=ESMF_INDEX_GLOBAL for the Grid creation to calculate
   ! coordinates across PETs.
   !-------------------------------------------------------------------
   grid=ESMF_GridCreate1PeriDim(maxIndex=(/GridLonSize,GridLatSize/), &
                                coordSys=ESMF_COORDSYS_SPH_DEG,       &
                                indexflag=ESMF_INDEX_GLOBAL,          &
                                rc=rc)
 
\end{verbatim}
 
%/////////////////////////////////////////////////////////////

 \begin{verbatim}
   !-------------------------------------------------------------------
   ! Allocate the center stagger location in which to put the coordinates. 
   !-------------------------------------------------------------------
   call ESMF_GridAddCoord(grid, staggerloc=ESMF_STAGGERLOC_CENTER, rc=rc)
 
\end{verbatim}
 
%/////////////////////////////////////////////////////////////

 \begin{verbatim}
   !-------------------------------------------------------------------
   ! Get access to the Fortran array pointers that hold the Grid 
   ! coordinate information.
   !------------------------------------------------------------------- 
   ! Longitudes 
   call ESMF_GridGetCoord(grid,                                  &
                          staggerLoc=ESMF_STAGGERLOC_CENTER,     &
                          coordDim=1, computationalLBound=clbnd, &
                          computationalUBound=cubnd,             &
                          farrayPtr=farrayPtrLonC, rc=rc)
 
\end{verbatim}
 
%/////////////////////////////////////////////////////////////

 \begin{verbatim}
   ! Latitudes
   call ESMF_GridGetCoord(grid,                                  &
                          staggerLoc=ESMF_STAGGERLOC_CENTER,     &
                          coordDim=2, computationalLBound=clbnd, &
                          computationalUBound=cubnd,             &
                          farrayPtr=farrayPtrLatC, rc=rc)
 
\end{verbatim}
 
%/////////////////////////////////////////////////////////////

 \begin{verbatim}

   !-------------------------------------------------------------------
   ! Create a source Field to hold the data to be regridded to the 
   ! destination
   !-------------------------------------------------------------------
   srcField = ESMF_FieldCreate(grid, typekind=ESMF_TYPEKIND_R8,   &
                               staggerloc=ESMF_STAGGERLOC_CENTER, &
                               name="source", rc=rc)
 
\end{verbatim}
 
%/////////////////////////////////////////////////////////////

 \begin{verbatim}
   !-------------------------------------------------------------------
   ! Set the Grid coordinates to be uniformly distributed around the globe. 
   !-------------------------------------------------------------------
   do i1=clbnd(1),cubnd(1)
   do i2=clbnd(2),cubnd(2)
      ! Set Grid longitude coordinates as 0 to 360
      farrayPtrLonC(i1,i2) = REAL(i1-1)*360.0/REAL(GridLonSize)

      ! Set Grid latitude coordinates as -90 to 90
      farrayPtrLatC(i1,i2) = -90. + REAL(i2-1)*180.0/REAL(GridLatSize) + &
                                       0.5*180.0/REAL(GridLatSize)
 
\end{verbatim}
 
%/////////////////////////////////////////////////////////////

 \begin{verbatim}
   enddo
   enddo

   !-------------------------------------------------------------------
   ! Set the number of points the destination LocStream will have
   ! depending on the PET. 
   !-------------------------------------------------------------------
   if (petCount .eq. 1) then
     numLocationsOnThisPet=7
   else
     if (localpet .eq. 0) then
       numLocationsOnThisPet=2
     else if (localpet .eq. 1) then
       numLocationsOnThisPet=2
     else if (localpet .eq. 2) then
       numLocationsOnThisPet=2
     else if (localpet .eq. 3) then
       numLocationsOnThisPet=1
     endif
   endif

   !-------------------------------------------------------------------
   ! Create the LocStream:  Allocate space for the LocStream object,
   ! define the number of locations on this PET. 
   !-------------------------------------------------------------------
   locstream=ESMF_LocStreamCreate(name="Test Data",                 &
                                  localCount=numLocationsOnThisPet, &
                                  coordSys=ESMF_COORDSYS_SPH_DEG,   &
                                  rc=rc)
 
\end{verbatim}
 
%/////////////////////////////////////////////////////////////

 \begin{verbatim}
   !-------------------------------------------------------------------
   ! Add key data to LocStream(internally allocating memory).
   !-------------------------------------------------------------------
   call ESMF_LocStreamAddKey(locstream,                    &
                             keyName="ESMF:Lat",           &
                             KeyTypeKind=ESMF_TYPEKIND_R8, &
                             keyUnits="degrees",           &
                             keyLongName="Latitude", rc=rc)
 
\end{verbatim}
 
%/////////////////////////////////////////////////////////////

 \begin{verbatim}
   call ESMF_LocStreamAddKey(locstream,                    &
                             keyName="ESMF:Lon",           &
                             KeyTypeKind=ESMF_TYPEKIND_R8, &
                             keyUnits="degrees",           &
                             keyLongName="Longitude", rc=rc)
 
\end{verbatim}
 
%/////////////////////////////////////////////////////////////

 \begin{verbatim}
   !-------------------------------------------------------------------
   ! Get access to the Fortran array pointers that hold the key data.
   !-------------------------------------------------------------------
   ! Longitudes
   call ESMF_LocStreamGetKey(locstream,           &
                             keyName="ESMF:Lon",  &
                             farray=lonArray,     &
                             rc=rc)
 
\end{verbatim}
 
%/////////////////////////////////////////////////////////////

 \begin{verbatim}
   ! Latitudes
   call ESMF_LocStreamGetKey(locstream,           &
                             keyName="ESMF:Lat",  &
                             farray=latArray,     &
                             rc=rc)
 
\end{verbatim}
 
%/////////////////////////////////////////////////////////////

 \begin{verbatim}

   !-------------------------------------------------------------------
   ! Set coordinates in key arrays depending on the PET.
   ! For this example use an arbitrary set of points around globe.  
   !-------------------------------------------------------------------
   if (petCount .eq. 1) then
     latArray = (/-87.75, -56.25, -26.5, 0.0, 26.5, 56.25, 87.75 /)
     lonArray = (/51.4, 102.8, 154.2, 205.6, 257.0, 308.4, 359.8 /)
   else
     if (localpet .eq. 0) then
       latArray = (/ -87.75, -56.25 /)
       lonArray = (/ 51.4, 102.8 /)
     else if (localpet .eq.1) then
       latArray = (/ -26.5, 0.0 /)
       lonArray = (/ 154.2, 205.6 /)
     else if (localpet .eq.2) then
       latArray = (/ 26.5, 56.25 /)
       lonArray = (/ 257.0, 308.4 /)
     else if (localpet .eq.3) then
       latArray = (/ 87.75 /)
       lonArray = (/ 359.8 /)
     endif
   endif

   !-------------------------------------------------------------------
   ! Create the destination Field on the LocStream to hold the 
   ! result of the regridding. 
   !-------------------------------------------------------------------
   dstField = ESMF_FieldCreate(locstream, typekind=ESMF_TYPEKIND_R8, &
                               name="dest", rc=rc)
 
\end{verbatim}
 
%/////////////////////////////////////////////////////////////

 \begin{verbatim}

   !-------------------------------------------------------------------
   ! Calculate the RouteHandle that represents the regridding from 
   ! the source to destination Field using the Bilinear regridding method.
   !-------------------------------------------------------------------
   call ESMF_FieldRegridStore( srcField=srcField,                       &
                               dstField=dstField,                       &
                               routeHandle=routeHandle,                 &
                               regridmethod=ESMF_REGRIDMETHOD_BILINEAR, &
                               rc=rc)
 
\end{verbatim}
 
%/////////////////////////////////////////////////////////////

 \begin{verbatim}


   !-------------------------------------------------------------------
   ! Regrid from srcField to dstField
   !-------------------------------------------------------------------
   ! Can loop here regridding from srcField to dstField as src data changes
   ! do i=1,...

        ! (Put data into srcField)

        !-------------------------------------------------------------------
        ! Use the RouteHandle to regrid data from srcField to dstField.
        !-------------------------------------------------------------------
        call ESMF_FieldRegrid(srcField, dstField, routeHandle, rc=rc)

        ! (Can now use the data in dstField)

   ! enddo

 
\end{verbatim}
 
%/////////////////////////////////////////////////////////////

 \begin{verbatim}
   !-------------------------------------------------------------------
   ! Now that we are done, release the RouteHandle freeing its memory. 
   !-------------------------------------------------------------------
   call ESMF_FieldRegridRelease(routeHandle, rc=rc)
 
\end{verbatim}

%...............................................................
\setlength{\parskip}{\oldparskip}
\setlength{\parindent}{\oldparindent}
\setlength{\baselineskip}{\oldbaselineskip}
