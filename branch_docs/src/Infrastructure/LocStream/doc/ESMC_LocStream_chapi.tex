%                **** IMPORTANT NOTICE *****
% This LaTeX file has been automatically produced by ProTeX v. 1.1
% Any changes made to this file will likely be lost next time
% this file is regenerated from its source. Send questions 
% to Arlindo da Silva, dasilva@gsfc.nasa.gov
 
\setlength{\oldparskip}{\parskip}
\setlength{\parskip}{1.5ex}
\setlength{\oldparindent}{\parindent}
\setlength{\parindent}{0pt}
\setlength{\oldbaselineskip}{\baselineskip}
\setlength{\baselineskip}{11pt}
 
%--------------------- SHORT-HAND MACROS ----------------------
\def\bv{\begin{verbatim}}
\def\ev{\end{verbatim}}
\def\be{\begin{equation}}
\def\ee{\end{equation}}
\def\bea{\begin{eqnarray}}
\def\eea{\end{eqnarray}}
\def\bi{\begin{itemize}}
\def\ei{\end{itemize}}
\def\bn{\begin{enumerate}}
\def\en{\end{enumerate}}
\def\bd{\begin{description}}
\def\ed{\end{description}}
\def\({\left (}
\def\){\right )}
\def\[{\left [}
\def\]{\right ]}
\def\<{\left  \langle}
\def\>{\right \rangle}
\def\cI{{\cal I}}
\def\diag{\mathop{\rm diag}}
\def\tr{\mathop{\rm tr}}
%-------------------------------------------------------------

\markboth{Left}{Source File: ESMC\_LocStream.h,  Date: Tue May  5 20:59:56 MDT 2020
}

 
%/////////////////////////////////////////////////////////////
\subsubsection [ESMC\_LocStreamCreateLocal] {ESMC\_LocStreamCreateLocal - Create a LocStream}


  
\bigskip{\sf INTERFACE:}
\begin{verbatim} ESMC_LocStream ESMC_LocStreamCreateLocal(
                                     int ls_size,
                                     enum ESMC_IndexFlag *indexflag,
                                     enum ESMC_CoordSys_Flag *coordSys,
                                     int *rc
 );
 \end{verbatim}{\em RETURN VALUE:}
\begin{verbatim}    Newly created ESMC_LocStream object.\end{verbatim}
{\sf DESCRIPTION:\\ }


  
    Creates a {\tt ESMC\_LocStream} object.
  
    The arguments are:
    \begin{description}
    \item[{[ls_size]}]
      number of points in the location stream.
    \item[indexflag]
      Indicates the indexing scheme to be used in the new LocStream. If not present,
      defaults to ESMC\_INDEX\_DELOCAL.
    \item[coordSys]
      The coordinated system of the LocStream coordinate data. If not specified then
      defaults to ESMF\_COORDSYS\_SPH\_DEG.
    \item[{[rc]}]
      Return code; equals {\tt ESMF\_SUCCESS} if there are no errors.
    \end{description}
   
%/////////////////////////////////////////////////////////////
 
\mbox{}\hrulefill\ 
 
\subsubsection [ESMC\_LocStreamGetBounds] {ESMC\_LocStreamGetBounds - Get the LocStream bounds}


  
\bigskip{\sf INTERFACE:}
\begin{verbatim} int ESMC_LocStreamGetBounds(
   ESMC_LocStream locstream,      // in
   int localDe,
   int *cLBound,
   int *cUBound
 );
 \end{verbatim}{\em RETURN VALUE:}
\begin{verbatim}    Return code; equals ESMF_SUCCESS if there are no errors.\end{verbatim}
{\sf DESCRIPTION:\\ }


  
    Get the LocStream bounds from the {\tt ESMC\_LocStream}.
  
    The arguments are:
    \begin{description}
    \item[LocStream]
      {\tt ESMC\_LocStream} whose bounds will be returned
    \item[localDe]
      The local DE of the {\tt ESMC\_LocStream} (not implemented)
    \item[exclusiveLBound]
      The exclusive lower bounds of the {\tt ESMC\_LocStream}
    \item[exclusiveUBound]
      The exclusive upper bounds of the {\tt ESMC\_LocStream}
    \end{description}
   
%/////////////////////////////////////////////////////////////
 
\mbox{}\hrulefill\ 
 
\subsubsection [ESMC\_LocStreamAddKeyAlloc] {ESMC\_LocStreamAddKeyAlloc - allocate memory for adding a key}


  
\bigskip{\sf INTERFACE:}
\begin{verbatim} int ESMC_LocStreamAddKeyAlloc(
   ESMC_LocStream locstream,      // in
   const char *keyName,
   enum ESMC_TypeKind_Flag *keyTypeKind
 );
 \end{verbatim}{\em RETURN VALUE:}
\begin{verbatim}    Return code; equals ESMF_SUCCESS if there are no errors.\end{verbatim}
{\sf DESCRIPTION:\\ }


  
    allocate memory for a new key in {\tt ESMC\_LocStream}.
  
    The arguments are:
    \begin{description}
    \item[LocStream]
      {\tt ESMC\_LocStream} who will have the new key
    \item[keyName]
      name of the new key in {\tt ESMC\_LocStream}
    \end{description}
   
%/////////////////////////////////////////////////////////////
 
\mbox{}\hrulefill\ 
 
\subsubsection [ESMC\_LocStreamGetKeyPtr] {ESMC\_LocStreamGetKeyPtr - Get a pointer to a LocStream key}


  
\bigskip{\sf INTERFACE:}
\begin{verbatim} void *ESMC_LocStreamGetKeyPtr(
   ESMC_LocStream locstream,      // in
   const char *keyName,
   int localDe,
   int *rc
 );
 \end{verbatim}{\em RETURN VALUE:}
\begin{verbatim}    Return code; equals ESMF_SUCCESS if there are no errors.\end{verbatim}
{\sf DESCRIPTION:\\ }


  
    Get a pointer to a specified key in {\tt ESMC\_LocStream}.
  
    The arguments are:
    \begin{description}
    \item[LocStream]
      {\tt ESMC\_LocStream} containing the key.
    \item[keyName]
      name of the new key in {\tt ESMC\_LocStream}
    \item[localDe]
      The local DE of the {\tt ESMC\_LocStream} (not implemented)
    \item[{[rc]}]
      Return code; equals {\tt ESMF\_SUCCESS} if there are no errors.
    \end{description}
   
%/////////////////////////////////////////////////////////////
 
\mbox{}\hrulefill\ 
 
\subsubsection [ESMC\_LocStreamGetKeyArray] {ESMC\_LocStreamGetKeyArray - Get the internal Array stored in the LocStream}


  
\bigskip{\sf INTERFACE:}
\begin{verbatim}   ESMC_Array ESMC_LocStreamGetKeyArray(
                                 ESMC_LocStream locstream,     // in
                                 const char *keyName,          // in
                                 int *rc                       // out
                                 );
 \end{verbatim}{\em RETURN VALUE:}
\begin{verbatim}    The ESMC_Array object stored in the ESMC_LocStream.\end{verbatim}
{\sf DESCRIPTION:\\ }


  
    Get the internal Array stored in the {\tt ESMC\_LocStream}.
  
    The arguments are:
    \begin{description}
    \item[LocStream]
      {\tt ESMC\_LocStream} containing the array.
    \item[keyName]
      name of the new key in {\tt ESMC\_LocStream}
    \item[{[rc]}]
      Return code; equals {\tt ESMF\_SUCCESS} if there are no errors.
    \end{description}
   
%/////////////////////////////////////////////////////////////
 
\mbox{}\hrulefill\ 
 
\subsubsection [ESMC\_LocStreamDestroy] {ESMC\_LocStreamDestroy - Destroy a LocStream}


  
\bigskip{\sf INTERFACE:}
\begin{verbatim} int ESMC_LocStreamDestroy(
   ESMC_LocStream *locstream     // inout
 );
 \end{verbatim}{\em RETURN VALUE:}
\begin{verbatim}    Return code; equals ESMF_SUCCESS if there are no errors.\end{verbatim}
{\sf DESCRIPTION:\\ }


  
    Releases all resources associated with this {\tt ESMC\_LocStream}.
      Return code; equals {\tt ESMF\_SUCCESS} if there are no errors.
  
    The arguments are:
    \begin{description}
    \item[LocStream]
      Destroy contents of this {\tt ESMC\_LocStream}.
    \end{description}
  
%...............................................................
\setlength{\parskip}{\oldparskip}
\setlength{\parindent}{\oldparindent}
\setlength{\baselineskip}{\oldbaselineskip}
