%                **** IMPORTANT NOTICE *****
% This LaTeX file has been automatically produced by ProTeX v. 1.1
% Any changes made to this file will likely be lost next time
% this file is regenerated from its source. Send questions 
% to Arlindo da Silva, dasilva@gsfc.nasa.gov
 
\setlength{\oldparskip}{\parskip}
\setlength{\parskip}{1.5ex}
\setlength{\oldparindent}{\parindent}
\setlength{\parindent}{0pt}
\setlength{\oldbaselineskip}{\baselineskip}
\setlength{\baselineskip}{11pt}
 
%--------------------- SHORT-HAND MACROS ----------------------
\def\bv{\begin{verbatim}}
\def\ev{\end{verbatim}}
\def\be{\begin{equation}}
\def\ee{\end{equation}}
\def\bea{\begin{eqnarray}}
\def\eea{\end{eqnarray}}
\def\bi{\begin{itemize}}
\def\ei{\end{itemize}}
\def\bn{\begin{enumerate}}
\def\en{\end{enumerate}}
\def\bd{\begin{description}}
\def\ed{\end{description}}
\def\({\left (}
\def\){\right )}
\def\[{\left [}
\def\]{\right ]}
\def\<{\left  \langle}
\def\>{\right \rangle}
\def\cI{{\cal I}}
\def\diag{\mathop{\rm diag}}
\def\tr{\mathop{\rm tr}}
%-------------------------------------------------------------

\markboth{Left}{Source File: ESMF\_DistGridRegDecomp.F90,  Date: Tue May  5 20:59:41 MDT 2020
}

 
%/////////////////////////////////////////////////////////////
\subsubsection [ESMF\_DistGridRegDecompSetCubic] {ESMF\_DistGridRegDecompSetCubic - Construct a DistGrid regDecomp}


\bigskip{\sf INTERFACE:}
\begin{verbatim}   subroutine ESMF_DistGridRegDecompSetCubic(regDecomp, deCount, rc)\end{verbatim}{\em ARGUMENTS:}
\begin{verbatim}     integer,        target, intent(out)           :: regDecomp(:)
 -- The following arguments require argument keyword syntax (e.g. rc=rc). --
     integer,                intent(in),  optional :: deCount
     integer,                intent(out), optional :: rc\end{verbatim}
{\sf DESCRIPTION:\\ }


     Construct a regular decomposition argument for DistGrid that is most cubic
     in {\tt dimCount} dimensions, and multiplies out to {\tt deCount} DEs. The
     factorization is stable monotonic decreasing, ensuring that earlier entries
     in {\tt regDecomp} are larger or equal to the later entires.
  
     The arguments are:
     \begin{description}
     \item[regDecomp]
       The regular decomposition description being constructed.
     \item[{[deCount]}]
       The number of DEs. Defaults to {\tt petCount}.
     \item[{[rc]}]
       Return code; equals {\tt ESMF\_SUCCESS} if there are no errors.
     \end{description}
  
%...............................................................
\setlength{\parskip}{\oldparskip}
\setlength{\parindent}{\oldparindent}
\setlength{\baselineskip}{\oldbaselineskip}
