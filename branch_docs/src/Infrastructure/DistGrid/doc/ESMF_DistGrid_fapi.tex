%                **** IMPORTANT NOTICE *****
% This LaTeX file has been automatically produced by ProTeX v. 1.1
% Any changes made to this file will likely be lost next time
% this file is regenerated from its source. Send questions 
% to Arlindo da Silva, dasilva@gsfc.nasa.gov
 
\setlength{\oldparskip}{\parskip}
\setlength{\parskip}{1.5ex}
\setlength{\oldparindent}{\parindent}
\setlength{\parindent}{0pt}
\setlength{\oldbaselineskip}{\baselineskip}
\setlength{\baselineskip}{11pt}
 
%--------------------- SHORT-HAND MACROS ----------------------
\def\bv{\begin{verbatim}}
\def\ev{\end{verbatim}}
\def\be{\begin{equation}}
\def\ee{\end{equation}}
\def\bea{\begin{eqnarray}}
\def\eea{\end{eqnarray}}
\def\bi{\begin{itemize}}
\def\ei{\end{itemize}}
\def\bn{\begin{enumerate}}
\def\en{\end{enumerate}}
\def\bd{\begin{description}}
\def\ed{\end{description}}
\def\({\left (}
\def\){\right )}
\def\[{\left [}
\def\]{\right ]}
\def\<{\left  \langle}
\def\>{\right \rangle}
\def\cI{{\cal I}}
\def\diag{\mathop{\rm diag}}
\def\tr{\mathop{\rm tr}}
%-------------------------------------------------------------

\markboth{Left}{Source File: ESMF\_DistGrid.F90,  Date: Tue May  5 20:59:41 MDT 2020
}

 
%/////////////////////////////////////////////////////////////
\subsubsection [ESMF\_DistGridAssignment(=)] {ESMF\_DistGridAssignment(=) - DistGrid assignment}


  
\bigskip{\sf INTERFACE:}
\begin{verbatim}     interface assignment(=)
     distgrid1 = distgrid2\end{verbatim}{\em ARGUMENTS:}
\begin{verbatim}     type(ESMF_DistGrid) :: distgrid1
     type(ESMF_DistGrid) :: distgrid2\end{verbatim}
{\sf STATUS:}
   \begin{itemize}
   \item\apiStatusCompatibleVersion{5.2.0r}
   \end{itemize}
  
{\sf DESCRIPTION:\\ }


     Assign distgrid1 as an alias to the same ESMF DistGrid object in memory
     as distgrid2. If distgrid2 is invalid, then distgrid1 will be equally
     invalid after the assignment.
  
     The arguments are:
     \begin{description}
     \item[distgrid1]
       The {\tt ESMF\_DistGrid} object on the left hand side of the assignment.
     \item[distgrid2]
       The {\tt ESMF\_DistGrid} object on the right hand side of the assignment.
     \end{description}
   
%/////////////////////////////////////////////////////////////
 
\mbox{}\hrulefill\ 
 
\subsubsection [ESMF\_DistGridOperator(==)] {ESMF\_DistGridOperator(==) - DistGrid equality operator}


  
\bigskip{\sf INTERFACE:}
\begin{verbatim}   interface operator(==)
     if (distgrid1 == distgrid2) then ... endif
               OR
     result = (distgrid1 == distgrid2)\end{verbatim}{\em RETURN VALUE:}
\begin{verbatim}     logical :: result\end{verbatim}{\em ARGUMENTS:}
\begin{verbatim}     type(ESMF_DistGrid), intent(in) :: distgrid1
     type(ESMF_DistGrid), intent(in) :: distgrid2\end{verbatim}
{\sf STATUS:}
   \begin{itemize}
   \item\apiStatusCompatibleVersion{5.2.0r}
   \end{itemize}
  
{\sf DESCRIPTION:\\ }


     Test whether distgrid1 and distgrid2 are valid aliases to the same ESMF
     DistGrid object in memory. For a more general comparison of two 
     ESMF DistGrids, going beyond the simple alias test, the 
     {\tt ESMF\_DistGridMatch()} function (not yet fully implemented) must 
     be used.
  
     The arguments are:
     \begin{description}
     \item[distgrid1]
       The {\tt ESMF\_DistGrid} object on the left hand side of the equality
       operation.
     \item[distgrid2]
       The {\tt ESMF\_DistGrid} object on the right hand side of the equality
       operation.
     \end{description}
   
%/////////////////////////////////////////////////////////////
 
\mbox{}\hrulefill\ 
 
\subsubsection [ESMF\_DistGridOperator(/=)] {ESMF\_DistGridOperator(/=) - DistGrid not equal operator}


  
\bigskip{\sf INTERFACE:}
\begin{verbatim}   interface operator(/=)
     if (distgrid1 /= distgrid2) then ... endif
               OR
     result = (distgrid1 /= distgrid2)\end{verbatim}{\em RETURN VALUE:}
\begin{verbatim}     logical :: result\end{verbatim}{\em ARGUMENTS:}
\begin{verbatim}     type(ESMF_DistGrid), intent(in) :: distgrid1
     type(ESMF_DistGrid), intent(in) :: distgrid2\end{verbatim}
{\sf STATUS:}
   \begin{itemize}
   \item\apiStatusCompatibleVersion{5.2.0r}
   \end{itemize}
  
{\sf DESCRIPTION:\\ }


     Test whether distgrid1 and distgrid2 are {\it not} valid aliases to the
     same ESMF DistGrid object in memory. For a more general comparison of two
     ESMF DistGrids, going beyond the simple alias test, the
     {\tt ESMF\_DistGridMatch()} function (not yet fully implemented) must 
     be used.
  
     The arguments are:
     \begin{description}
     \item[distgrid1]
       The {\tt ESMF\_DistGrid} object on the left hand side of the non-equality
       operation.
     \item[distgrid2]
       The {\tt ESMF\_DistGrid} object on the right hand side of the non-equality
       operation.
     \end{description}
   
%/////////////////////////////////////////////////////////////
 
\mbox{}\hrulefill\ 
 
\subsubsection [ESMF\_DistGridCreate] {ESMF\_DistGridCreate - Create DistGrid object from DistGrid}


 
\bigskip{\sf INTERFACE:}
\begin{verbatim}   ! Private name; call using ESMF_DistGridCreate()
   function ESMF_DistGridCreateDG(distgrid, &
     firstExtra, lastExtra, indexflag, connectionList, balanceflag, &
     delayout, vm, rc)
           \end{verbatim}{\em RETURN VALUE:}
\begin{verbatim}     type(ESMF_DistGrid) :: ESMF_DistGridCreateDG\end{verbatim}{\em ARGUMENTS:}
\begin{verbatim}     type(ESMF_DistGrid),           intent(in)            :: distgrid
 -- The following arguments require argument keyword syntax (e.g. rc=rc). --
     integer, target,               intent(in),  optional :: firstExtra(:)
     integer, target,               intent(in),  optional :: lastExtra(:)
     type(ESMF_Index_Flag),         intent(in),  optional :: indexflag
     type(ESMF_DistGridConnection), intent(in),  optional :: connectionList(:)
     logical,                       intent(in),  optional :: balanceflag
     type(ESMF_DELayout),           intent(in),  optional :: delayout
     type(ESMF_VM),                 intent(in),  optional :: vm
     integer,                       intent(out), optional :: rc\end{verbatim}
{\sf STATUS:}
   \begin{itemize}
   \item\apiStatusCompatibleVersion{5.2.0r}
   \item\apiStatusModifiedSinceVersion{5.2.0r}
   \begin{description}
   \item[6.3.0r] Added argument {\tt vm} to support object creation on a
                 different VM than that of the current context.
   \item[8.0.0] Added argument {\tt delayout} to support changing the layout of
                DEs across PETs.\newline
                Added argument {\tt balanceflag} to support rebalancing of the
                incoming DistGrids decomposition.
   \end{description}
   \end{itemize}
  
{\sf DESCRIPTION:\\ }


       Create a new DistGrid from an existing DistGrid, keeping the decomposition
       unchanged. The {\tt firstExtra} and {\tt lastExtra} arguments allow extra
       elements to be added at the first/last edge DE in each dimension. The 
       method also allows the {\tt indexflag} to be set. Further, if the 
       {\tt connectionList} argument is provided it will be used to set 
       connections in the newly created DistGrid, otherwise the connections of
       the incoming DistGrid will be used.
       If neither {\tt firstExtra}, {\tt lastExtra}, {\tt indexflag}, nor 
       {\tt connectionList} arguments are specified, the method reduces to a 
       deep copy of the incoming DistGrid object.
  
       The arguments are:
       \begin{description}
       \item[distgrid]
            Incoming DistGrid object.
       \item[{[firstExtra]}]
            Extra elements added to the first DE along each 
            dimension. This increases the size of the index space compared to 
            that of the incoming {\tt distgrid}. The decomposition of the
            enlarged index space is constructed to align with the original index
            space provided by {\tt distgrid}.
            The default is a zero vector.
       \item[{[lastExtra]}]
            Extra elements added to the last DE along each 
            dimension. This increases the size of the index space compared to 
            that of the incoming {\tt distgrid}. The decomposition of the
            enlarged index space is constructed to align with the original index
            space provided by {\tt distgrid}.
            The default is a zero vector.
       \item[{[indexflag]}]
            If present, override the indexflag setting of the incoming
            {\tt distgrid}. See section \ref{const:indexflag} for a 
            complete list of options. By default use the indexflag setting of 
            {\tt distgrid}. 
       \item[{[connectionList]}]
            If present, override the connections of the incoming {\tt distgrid}.
            See section \ref{api:DistGridConnectionSet} for the associated Set()
            method. By default use the connections defined in {\tt distgrid}.
       \item[{[balanceflag]}]
            If set to {\tt .true}, rebalance the incoming {\tt distgrid}
            decompositon. The default is {\tt .false.}.
       \item[{[delayout]}]
            If present, override the DELayout of the incoming {\tt distgrid}.
            By default use the DELayout defined in {\tt distgrid}.
       \item[{[vm]}]
            If present, the DistGrid object and the DELayout object
            are created on the specified {\tt ESMF\_VM} object. The 
            default is to use the VM of the current context. 
       \item[{[rc]}]
            Return code; equals {\tt ESMF\_SUCCESS} if there are no errors.
       \end{description}
   
%/////////////////////////////////////////////////////////////
 
\mbox{}\hrulefill\ 
 
\subsubsection [ESMF\_DistGridCreate] {ESMF\_DistGridCreate - Create DistGrid object from DistGrid (multi-tile version)}


 
\bigskip{\sf INTERFACE:}
\begin{verbatim}   ! Private name; call using ESMF_DistGridCreate()
   function ESMF_DistGridCreateDGT(distgrid, firstExtraPTile, &
     lastExtraPTile, indexflag, connectionList, balanceflag, &
     delayout, vm, rc)
           \end{verbatim}{\em RETURN VALUE:}
\begin{verbatim}     type(ESMF_DistGrid) :: ESMF_DistGridCreateDGT\end{verbatim}{\em ARGUMENTS:}
\begin{verbatim}     type(ESMF_DistGrid),           intent(in)            :: distgrid
     integer, target,               intent(in)            :: firstExtraPTile(:,:)
     integer, target,               intent(in)            :: lastExtraPTile(:,:)
 -- The following arguments require argument keyword syntax (e.g. rc=rc). --
     type(ESMF_Index_Flag),         intent(in),  optional :: indexflag
     type(ESMF_DistGridConnection), intent(in),  optional :: connectionList(:)
     logical,                       intent(in),  optional :: balanceflag
     type(ESMF_DELayout),           intent(in),  optional :: delayout
     type(ESMF_VM),                 intent(in),  optional :: vm
     integer,                       intent(out), optional :: rc\end{verbatim}
{\sf STATUS:}
   \begin{itemize}
   \item\apiStatusCompatibleVersion{5.2.0r}
   \item\apiStatusModifiedSinceVersion{5.2.0r}
   \begin{description}
   \item[6.3.0r] Added argument {\tt vm} to support object creation on a
                 different VM than that of the current context.
   \item[8.0.0] Added argument {\tt delayout} to support changing the layout of
                DEs across PETs.\newline
                Added argument {\tt balanceflag} to support rebalancing of the
                incoming DistGrids decomposition.
   \end{description}
   \end{itemize}
  
{\sf DESCRIPTION:\\ }


       Create a new DistGrid from an existing DistGrid, keeping the decomposition
       unchanged. The {\tt firstExtraPTile} and {\tt lastExtraPTile} arguments allow extra
       elements to be added at the first/last edge DE in each dimension. The 
       method also allows the {\tt indexflag} to be set. Further, if the 
       {\tt connectionList} argument provided in it will be used to set 
       connections in the newly created DistGrid, otherwise the connections of
       the incoming DistGrid will be used.
       If neither {\tt firstExtraPTile}, {\tt lastExtraPTile}, {\tt indexflag}, nor 
       {\tt connectionList} arguments are specified, the method reduces to a 
       deep copy of the incoming DistGrid object.
  
       The arguments are:
       \begin{description}
       \item[distgrid]
            Incoming DistGrid object.
       \item[firstExtraPTile]
            Extra elements added to the first DE along each dimension for each 
            tile. This increases the size of the index space compared to 
            that of the incoming {\tt distgrid}. The decomposition of the
            enlarged index space is constructed to align with the original index
            space provided by {\tt distgrid}.
            The default is a zero vector.
       \item[lastExtraPTile]
            Extra elements added to the last DE along each dimension for each
            tile. This increases the size of the index space compared to 
            that of the incoming {\tt distgrid}. The decomposition of the
            enlarged index space is constructed to align with the original index
            space provided by {\tt distgrid}.
            The default is a zero vector.
       \item[{[indexflag]}]
            If present, override the indexflag setting of the incoming
            {\tt distgrid}. See section \ref{const:indexflag} for a 
            complete list of options. By default use the indexflag setting of 
            {\tt distgrid}. 
       \item[{[connectionList]}]
            If present, override the connections of the incoming {\tt distgrid}.
            See section \ref{api:DistGridConnectionSet} for the associated Set()
            method. By default use the connections defined in {\tt distgrid}.
       \item[{[balanceflag]}]
            If set to {\tt .true}, rebalance the incoming {\tt distgrid}
            decompositon. The default is {\tt .false.}.
       \item[{[delayout]}]
            If present, override the DELayout of the incoming {\tt distgrid}.
            By default use the DELayout defined in {\tt distgrid}.
       \item[{[vm]}]
            If present, the DistGrid object and the DELayout object
            are created on the specified {\tt ESMF\_VM} object. The 
            default is to use the VM of the current context. 
       \item[{[rc]}]
            Return code; equals {\tt ESMF\_SUCCESS} if there are no errors.
       \end{description}
   
%/////////////////////////////////////////////////////////////
 
\mbox{}\hrulefill\ 
 
\subsubsection [ESMF\_DistGridCreate] {ESMF\_DistGridCreate - Create DistGrid object with regular decomposition}


 
\bigskip{\sf INTERFACE:}
\begin{verbatim}   ! Private name; call using ESMF_DistGridCreate()
   function ESMF_DistGridCreateRD(minIndex, maxIndex, regDecomp, &
     decompflag, regDecompFirstExtra, regDecompLastExtra, deLabelList, &
     indexflag, connectionList, delayout, vm, indexTK, rc)
           \end{verbatim}{\em RETURN VALUE:}
\begin{verbatim}     type(ESMF_DistGrid) :: ESMF_DistGridCreateRD\end{verbatim}{\em ARGUMENTS:}
\begin{verbatim}     integer,                        intent(in)            :: minIndex(:)
     integer,                        intent(in)            :: maxIndex(:)
 -- The following arguments require argument keyword syntax (e.g. rc=rc). --
     integer,                target, intent(in),  optional :: regDecomp(:)
     type(ESMF_Decomp_Flag), target, intent(in),  optional :: decompflag(:)
     integer,                target, intent(in),  optional :: regDecompFirstExtra(:)
     integer,                target, intent(in),  optional :: regDecompLastExtra(:)
     integer,                target, intent(in),  optional :: deLabelList(:)
     type(ESMF_Index_Flag),          intent(in),  optional :: indexflag
     type(ESMF_DistGridConnection),  intent(in),  optional :: connectionList(:)
     type(ESMF_DELayout),            intent(in),  optional :: delayout
     type(ESMF_VM),                  intent(in),  optional :: vm
     type(ESMF_TypeKind_Flag),       intent(in),  optional :: indexTK
     integer,                        intent(out), optional :: rc\end{verbatim}
{\sf STATUS:}
   \begin{itemize}
   \item\apiStatusCompatibleVersion{5.2.0r}
   \item\apiStatusModifiedSinceVersion{5.2.0r}
   \begin{description}
   \item[8.1.0] Added argument {\tt indexTK} to support explicit selection
                between 32-bit and 64-bit sequence indices.
   \end{description}
   \end{itemize}
  
{\sf DESCRIPTION:\\ }


       Create an {\tt ESMF\_DistGrid} from a single logically rectangular tile.
       The tile has a regular decomposition, where the tile is decomposed
       into a fixed number of DEs along each dimension. A regular decomposition
       of a single tile is expressed by a single {\tt regDecomp} list of DE 
       counts in each dimension.
  
       The arguments are:
       \begin{description}
       \item[minIndex]
            Index space tuple of the lower corner of the single tile.
       \item[maxIndex]
            Index space tuple of the upper corner of the single tile.
       \item[{[regDecomp]}]
            List of DE counts for each dimension. The total {\tt deCount} is
            determined as the product of {\tt regDecomp} elements.
            By default {\tt regDecomp} = (/{\tt deCount},1,...,1/), 
            where {\tt deCount}
            is the number of DEs in the {\tt delayout}. If the default
            {\tt delayout} is used, the {\tt deCount} is equal to {\tt petCount}.
            This leads to a simple 1 DE per PET distribution, where the
            decompsition is only along the first dimension.
       \item[{[decompflag]}]
            List of decomposition flags indicating how each dimension of the
            tile is to be divided between the DEs. The default setting
            is {\tt ESMF\_DECOMP\_BALANCED} in all dimensions. See section
            \ref{const:decompflag} for a list of valid decomposition options.
       \item[{[regDecompFirstExtra]}]
            Specify how many extra elements on the first DEs along each 
            dimension to consider when applying the regular decomposition 
            algorithm. This does {\em not} add extra elements to the 
            index space defined by {\tt minIndex} and {\tt maxIndex}. Instead
            {\tt regDecompFirstExtra} is used to correctly interpret the 
            specified index space: The {\tt regDecomp} is first applied to the
            index space {\em without} the extra elements. The extra elements are
            then added back in to arrive at the final decomposition. This is 
            useful when aligning the decomposition of index spaces that only
            differ in extra elements along the edges, e.g. when dealing with
            different stagger locations.
            The default is a zero vector, assuming no extra elements.
       \item[{[regDecompLastExtra]}]
            Specify how many extra elements on the last DEs along each 
            dimension to consider when applying the regular decomposition 
            algorithm. This does {\em not} add extra elements to the 
            index space defined by {\tt minIndex} and {\tt maxIndex}. Instead
            {\tt regDecompLastExtra} is used to correctly interpret the 
            specified index space: The {\tt regDecomp} is first applied to the
            index space {\em without} the extra elements. The extra elements are
            then added back in to arrive at the final decomposition. This is 
            useful when aligning the decomposition of index spaces that only
            differ in extra elements along the edges, e.g. when dealing with
            different stagger locations.
            The default is a zero vector, assuming no extra elements.
       \item[{[deLabelList]}]
            List assigning DE labels to the default sequence of DEs. The default
            sequence is given by the column major order of the {\tt regDecomp}
            argument.
       \item[{[indexflag]}]
            Indicates whether the indices provided by the {\tt minIndex} and
            {\tt maxIndex} arguments are forming a global
            index space or not. This does {\em not} affect the indices held
            by the DistGrid object, which are always identical to what was
            specified by {\tt minIndex} and {\tt maxIndex}, regardless of the
            {\tt indexflag} setting. However, it does affect whether an
            {\tt ESMF\_Array} object created on the DistGrid can choose global
            indexing or not. The default is {\tt ESMF\_INDEX\_DELOCAL}.
            See section \ref{const:indexflag} for a complete list of options.
       \item[{[connectionList]}]
            List of {\tt ESMF\_DistGridConnection} objects, defining connections
            between DistGrid tiles in index space.
            See section \ref{api:DistGridConnectionSet} for the associated Set()
            method.
       \item[{[delayout]}]
            {\tt ESMF\_DELayout} object to be used. If a DELayout object is
            specified its {\tt deCount} must match the number indicated by 
            {\tt regDecomp}. By default a new DELayout object will be created 
            with the correct number of DEs.
       \item[{[vm]}]
            If present, the DistGrid object (and the DELayout object if not 
            provided) are created on the specified {\tt ESMF\_VM} object. The 
            default is to use the VM of the current context. 
       \item[{[indexTK]}]
            Typekind used for global sequence indexing. See section 
            \ref{const:typekind} for a list of typekind options. Only integer
            types are supported. The default is to have ESMF automatically choose
            between {\tt ESMF\_TYPEKIND\_I4} and {\tt ESMF\_TYPEKIND\_I8},
            depending on whether the global number of elements held by the
            DistGrid is below or above the 32-bit limit, respectively.
            Because of the use of signed integers for sequence indices, 
            element counts of $ > 2^{31}-1 = 2,147,483,647$ will switch to 64-bit 
            indexing.
       \item[{[rc]}]
            Return code; equals {\tt ESMF\_SUCCESS} if there are no errors.
       \end{description}
   
%/////////////////////////////////////////////////////////////
 
\mbox{}\hrulefill\ 
 
\subsubsection [ESMF\_DistGridCreate] {ESMF\_DistGridCreate - Create DistGrid object with regular decomposition (multi-tile version)}


 
\bigskip{\sf INTERFACE:}
\begin{verbatim}   ! Private name; call using ESMF_DistGridCreate()
   function ESMF_DistGridCreateRDT(minIndexPTile, maxIndexPTile, &
     regDecompPTile, decompflagPTile, regDecompFirstExtraPTile,&
     regDecompLastExtraPTile, deLabelList, indexflag, connectionList, &
     delayout, vm, indexTK, rc)
           \end{verbatim}{\em RETURN VALUE:}
\begin{verbatim}     type(ESMF_DistGrid) :: ESMF_DistGridCreateRDT\end{verbatim}{\em ARGUMENTS:}
\begin{verbatim}     integer,                        intent(in)            :: minIndexPTile(:,:)
     integer,                        intent(in)            :: maxIndexPTile(:,:)
 -- The following arguments require argument keyword syntax (e.g. rc=rc). --
     integer,                        intent(in),  optional :: regDecompPTile(:,:)
     type(ESMF_Decomp_Flag), target, intent(in),  optional :: decompflagPTile(:,:)
     integer,                target, intent(in),  optional :: regDecompFirstExtraPTile(:,:)
     integer,                target, intent(in),  optional :: regDecompLastExtraPTile(:,:)
     integer,                        intent(in),  optional :: deLabelList(:)
     type(ESMF_Index_Flag),          intent(in),  optional :: indexflag
     type(ESMF_DistGridConnection),  intent(in),  optional :: connectionList(:)
     type(ESMF_DELayout),            intent(in),  optional :: delayout
     type(ESMF_VM),                  intent(in),  optional :: vm
     type(ESMF_TypeKind_Flag),       intent(in),  optional :: indexTK
     integer,                        intent(out), optional :: rc\end{verbatim}
{\sf STATUS:}
   \begin{itemize}
   \item\apiStatusCompatibleVersion{5.2.0r}
   \item\apiStatusModifiedSinceVersion{5.2.0r}
   \begin{description}
   \item[8.1.0] Added argument {\tt indexTK} to support explicit selection
                between 32-bit and 64-bit sequence indices.
   \end{description}
   \end{itemize}
  
{\sf DESCRIPTION:\\ }


       Create an {\tt ESMF\_DistGrid} from multiple logically rectangular tiles. 
       Each tile has a regular decomposition, where the tile is decomposed
       into a fixed number of DEs along each dimension. A regular decomposition
       of a multi-tile DistGrid is expressed by a list of DE count vectors, one
       vector for each tile. If a DELayout is specified, it must contain at least
       as many DEs as there are tiles.
  
       The arguments are:
       \begin{description}
       \item[minIndexPTile]
            The first index provides the index space tuple of the lower 
            corner of a tile. The second index indicates the tile number.
       \item[maxIndexPTile]
            The first index provides the index space tuple of the upper
            corner of a tile. The second index indicates the tile number.
       \item[{[regDecompPTile]}]
            List of DE counts for each dimension. The second index steps through
            the tiles. The total {\tt deCount} is determined as ths sum over
            the products of {\tt regDecomp} elements for each tile.
            By default each tile is decomposed only along the first dimension.
            The default number of DEs per tile is at least 1, but may be greater
            for the leading tiles if the {\tt deCount} is greater than the 
            {\tt tileCount}. If no DELayout is specified, the {\tt deCount} is 
            by default set equal to the number of PETs ({\tt petCount}), or the 
            number of tiles ({\tt tileCount}), which ever is greater. This means
            that as long as {\tt petCount} > {\tt tileCount}, the resulting
            default distribution will be 1 DE per PET. Notice that some tiles
            may be decomposed into more DEs than other tiles.
       \item[{[decompflagPTile]}]
            List of decomposition flags indicating how each dimension of each
            tile is to be divided between the DEs. The default setting
            is {\tt ESMF\_DECOMP\_BALANCED} in all dimensions for all tiles. 
            See section \ref{const:decompflag} for a list of valid decomposition
            flag options. The second index indicates the tile number.
       \item[{[regDecompFirstExtraPTile]}]
            Specify how many extra elements on the first DEs along each 
            dimension to consider when applying the regular decomposition 
            algorithm. This does {\em not} add extra elements to the 
            index space defined by {\tt minIndex} and {\tt maxIndex}. Instead
            {\tt regDecompFirstExtraPTile} is used to correctly interpret the 
            specified index space: The {\tt regDecomp} is first applied to the
            index space {\em without} the extra elements. The extra elements are
            then added back in to arrive at the final decomposition. This is 
            useful when aligning the decomposition of index spaces that only
            differ in extra elements along the edges, e.g. when dealing with
            different stagger locations.
            The default is a zero vector, assuming no extra elements.
       \item[{[regDecompLastExtraPTile]}]
            Specify how many extra elements on the last DEs along each 
            dimension to consider when applying the regular decomposition 
            algorithm. This does {\em not} add extra elements to the 
            index space defined by {\tt minIndex} and {\tt maxIndex}. Instead
            {\tt regDecompLastExtraPTile} is used to correctly interpret the 
            specified index space: The {\tt regDecomp} is first applied to the
            index space {\em without} the extra elements. The extra elements are
            then added back in to arrive at the final decomposition. This is 
            useful when aligning the decomposition of index spaces that only
            differ in extra elements along the edges, e.g. when dealing with
            different stagger locations.
            The default is a zero vector, assuming no extra elements.
       \item[{[deLabelList]}]
            List assigning DE labels to the default sequence of DEs. The default
            sequence is given by the column major order of the {\tt regDecompPTile}
            elements in the sequence as they appear following the tile index.
       \item[{[indexflag]}]
            Indicates whether the indices provided by the {\tt minIndexPTile} and
            {\tt maxIndexPTile} arguments are forming a global index space or 
            not. This does {\em not} affect the indices held by the DistGrid 
            object, which are always identical to what was specified by 
            {\tt minIndexPTile} and {\tt maxIndexPTile}, regardless of the
            {\tt indexflag} setting. However, it does affect whether an
            {\tt ESMF\_Array} object created on the DistGrid can choose global
            indexing or not. The default is {\tt ESMF\_INDEX\_DELOCAL}.
            See section \ref{const:indexflag} for a complete list of options.
       \item[{[connectionList]}]
            List of {\tt ESMF\_DistGridConnection} objects, defining connections
            between DistGrid tiles in index space.
            See section \ref{api:DistGridConnectionSet} for the associated Set()
            method.
       \item[{[delayout]}]
            Optional {\tt ESMF\_DELayout} object to be used. By default a new
            DELayout object will be created with as many DEs as there are PETs,
            or tiles, which ever is greater. If a DELayout object is specified,
            the number of DEs must match {\tt regDecompPTile}, if present. In the
            case that {\tt regDecompPTile} was not specified, the {\tt deCount}
            must be at least that of the default DELayout. The 
            {\tt regDecompPTile} will be constructed accordingly.
       \item[{[vm]}]
            Optional {\tt ESMF\_VM} object of the current context. Providing the
            VM of the current context will lower the method's overhead.
       \item[{[indexTK]}]
            Typekind used for global sequence indexing. See section 
            \ref{const:typekind} for a list of typekind options. Only integer
            types are supported. The default is to have ESMF automatically choose
            between {\tt ESMF\_TYPEKIND\_I4} and {\tt ESMF\_TYPEKIND\_I8},
            depending on whether the global number of elements held by the
            DistGrid is below or above the 32-bit limit, respectively.
            Because of the use of signed integers for sequence indices, 
            element counts of $ > 2^{31}-1 = 2,147,483,647$ will switch to 64-bit 
            indexing.
       \item[{[rc]}]
            Return code; equals {\tt ESMF\_SUCCESS} if there are no errors.
       \end{description}
   
%/////////////////////////////////////////////////////////////
 
\mbox{}\hrulefill\ 
 
\subsubsection [ESMF\_DistGridCreate] {ESMF\_DistGridCreate - Create DistGrid object with DE blocks}


 
\bigskip{\sf INTERFACE:}
\begin{verbatim}   ! Private name; call using ESMF_DistGridCreate()
   function ESMF_DistGridCreateDB(minIndex, maxIndex, deBlockList, &
     deLabelList, indexflag, connectionList, delayout, vm, &
     indexTK, rc)
           \end{verbatim}{\em RETURN VALUE:}
\begin{verbatim}     type(ESMF_DistGrid) :: ESMF_DistGridCreateDB\end{verbatim}{\em ARGUMENTS:}
\begin{verbatim}     integer,                       intent(in)            :: minIndex(:)
     integer,                       intent(in)            :: maxIndex(:)
     integer,                       intent(in)            :: deBlockList(:,:,:)
 -- The following arguments require argument keyword syntax (e.g. rc=rc). --
     integer,                       intent(in),  optional :: deLabelList(:)
     type(ESMF_Index_Flag),         intent(in),  optional :: indexflag
     type(ESMF_DistGridConnection), intent(in),  optional :: connectionList(:)
     type(ESMF_DELayout),           intent(in),  optional :: delayout
     type(ESMF_VM),                 intent(in),  optional :: vm
     type(ESMF_TypeKind_Flag),      intent(in),  optional :: indexTK
     integer,                       intent(out), optional :: rc\end{verbatim}
{\sf STATUS:}
   \begin{itemize}
   \item\apiStatusCompatibleVersion{5.2.0r}
   \item\apiStatusModifiedSinceVersion{5.2.0r}
   \begin{description}
   \item[7.1.0r] Added argument {\tt indexTK} to support selecting between
                 32-bit and 64-bit sequence indices.
   \end{description}
   \end{itemize}
  
{\sf DESCRIPTION:\\ }


       \begin{sloppypar}
       Create an {\tt ESMF\_DistGrid} from a single logically rectangular 
       tile with decomposition specified by {\tt deBlockList}.
       \end{sloppypar}
  
       The arguments are:
       \begin{description}
       \item[minIndex]
            Index space tuple of the lower corner of the single tile.
       \item[maxIndex]
            Index space tuple of the upper corner of the single tile.
       \item[deBlockList]
            List of DE-local blocks. The third index of {\tt deBlockList}
            steps through the deBlock elements (i.e. deCount), which are defined
            by the first two indices. 
            The first index must be of size {\tt dimCount} and the 
            second index must be of size 2. Each element of {\tt deBlockList}
            defined by the first two indices hold the following information.
            \begin{verbatim}
                     +---------------------------------------> 2nd index
                     |    1               2           
                     | 1  minIndex(1)    maxIndex(1)
                     | 2  minIndex(2)    maxIndex(2)
                     | .  minIndex(.)    maxIndex(.)
                     | .
                     v
                    1st index
            \end{verbatim}
            It is required that there be no overlap between the DE blocks.
       \item[{[deLabelList]}]
            List assigning DE labels to the default sequence of DEs. The default
            sequence is given by the order of DEs in the {\tt deBlockList} 
            argument.
       \item[{[indexflag]}]
            Indicates whether the indices provided by the {\tt minIndex} and
            {\tt maxIndex} arguments are forming a global
            index space or not. This does {\em not} affect the indices held
            by the DistGrid object, which are always identical to what was
            specified by {\tt minIndex} and {\tt maxIndex}, regardless of the
            {\tt indexflag} setting. However, it does affect whether an
            {\tt ESMF\_Array} object created on the DistGrid can choose global
            indexing or not. The default is {\tt ESMF\_INDEX\_DELOCAL}.
            See section \ref{const:indexflag} for a complete list of options.
       \item[{[connectionList]}]
            List of {\tt ESMF\_DistGridConnection} objects, defining connections
            between DistGrid tiles in index space.
            See section \ref{api:DistGridConnectionSet} for the associated Set()
            method.
       \item[{[delayout]}]
            Optional {\tt ESMF\_DELayout} object to be used. By default a new
            DELayout object will be created with the correct number of DEs. If
            a DELayout object is specified its number of DEs must match the 
            number indicated by {\tt regDecomp}.
       \item[{[vm]}]
            Optional {\tt ESMF\_VM} object of the current context. Providing the
            VM of the current context will lower the method's overhead.
       \item[{[indexTK]}]
            Typekind used for global sequence indexing. See section 
            \ref{const:typekind} for a list of typekind options. Only integer
            types are supported. The default is to have ESMF automatically choose
            between {\tt ESMF\_TYPEKIND\_I4} and {\tt ESMF\_TYPEKIND\_I8},
            depending on whether the global number of elements held by the
            DistGrid is below or above the 32-bit limit, respectively.
            Because of the use of signed integers for sequence indices, 
            element counts of $ > 2^{31}-1 = 2,147,483,647$ will switch to 64-bit 
            indexing.
       \item[{[rc]}]
            Return code; equals {\tt ESMF\_SUCCESS} if there are no errors.
       \end{description}
   
%/////////////////////////////////////////////////////////////
 
\mbox{}\hrulefill\ 
 
\subsubsection [ESMF\_DistGridCreate] {ESMF\_DistGridCreate - Create DistGrid object with DE blocks (multi-tile version)}


 
\bigskip{\sf INTERFACE:}
\begin{verbatim}   ! Private name; call using ESMF_DistGridCreate()
   function ESMF_DistGridCreateDBT(minIndexPTile, maxIndexPTile, deBlockList, &
     deToTileMap, deLabelList, indexflag, connectionList, &
     delayout, vm, indexTK, rc)
           \end{verbatim}{\em RETURN VALUE:}
\begin{verbatim}     type(ESMF_DistGrid) :: ESMF_DistGridCreateDBT\end{verbatim}{\em ARGUMENTS:}
\begin{verbatim}     integer,                       intent(in)            :: minIndexPTile(:,:)
     integer,                       intent(in)            :: maxIndexPTile(:,:)
     integer,                       intent(in)            :: deBlockList(:,:,:)
     integer,                       intent(in)            :: deToTileMap(:)
 -- The following arguments require argument keyword syntax (e.g. rc=rc). --
     integer,                       intent(in),  optional :: deLabelList(:)
     type(ESMF_Index_Flag),         intent(in),  optional :: indexflag
     type(ESMF_DistGridConnection), intent(in),  optional :: connectionList(:)
     type(ESMF_DELayout),           intent(in),  optional :: delayout
     type(ESMF_VM),                 intent(in),  optional :: vm
     type(ESMF_TypeKind_Flag),      intent(in),  optional :: indexTK
     integer,                       intent(out), optional :: rc\end{verbatim}
{\sf DESCRIPTION:\\ }


       Create an {\tt ESMF\_DistGrid} on multiple logically 
       rectangular tiles with decomposition specified by {\tt deBlockList}.
  
       The arguments are:
       \begin{description}
       \item[minIndexPTile]
            The first index provides the index space tuple of the lower 
            corner of a tile. The second index indicates the tile number.
       \item[maxIndexPTile]
            The first index provides the index space tuple of the upper
            corner of a tile. The second index indicates the tile number.
       \item[deBlockList]
            List of DE-local blocks. The third index of {\tt deBlockList}
            steps through the deBlock elements (i.e. deCount), which are defined
            by the first two indices. 
            The first index must be of size {\tt dimCount} and the 
            second index must be of size 2. Each element of {\tt deBlockList}
            defined by the first two indices hold the following information.
            \begin{verbatim}
                     +---------------------------------------> 2nd index
                     |    1               2           
                     | 1  minIndex(1)    maxIndex(1)
                     | 2  minIndex(2)    maxIndex(2)
                     | .  minIndex(.)    maxIndex(.)
                     | .
                     v
                    1st index
            \end{verbatim}
            It is required that there be no overlap between the DE blocks.
       \item[deToTileMap]
            List assigning each DE to a specific tile. The size of 
            {\tt deToTileMap} must be equal to {\tt deCount}.
            The order of DEs is the same as in {\tt deBlockList}.
       \item[{[deLabelList]}]
            List assigning DE labels to the default sequence of DEs. The default
            sequence is given by the order of DEs in the {\tt deBlockList} 
            argument.
       \item[{[indexflag]}]
            Indicates whether the indices provided by the {\tt minIndexPTile} and
            {\tt maxIndexPTile} arguments are forming a global index space or 
            not. This does {\em not} affect the indices held by the DistGrid 
            object, which are always identical to what was specified by 
            {\tt minIndexPTile} and {\tt maxIndexPTile}, regardless of the
            {\tt indexflag} setting. However, it does affect whether an
            {\tt ESMF\_Array} object created on the DistGrid can choose global
            indexing or not. The default is {\tt ESMF\_INDEX\_DELOCAL}.
            See section \ref{const:indexflag} for a complete list of options.
       \item[{[connectionList]}]
            List of {\tt ESMF\_DistGridConnection} objects, defining connections
            between DistGrid tiles in index space.
            See section \ref{api:DistGridConnectionSet} for the associated Set()
            method.
       \item[{[delayout]}]
            Optional {\tt ESMF\_DELayout} object to be used. By default a new
            DELayout object will be created with the correct number of DEs. If
            a DELayout object is specified its number of DEs must match the 
            number indicated by {\tt regDecomp}.
       \item[{[vm]}]
            Optional {\tt ESMF\_VM} object of the current context. Providing the
            VM of the current context will lower the method's overhead.
       \item[{[indexTK]}]
            Typekind used for global sequence indexing. See section 
            \ref{const:typekind} for a list of typekind options. Only integer
            types are supported. The default is to have ESMF automatically choose
            between {\tt ESMF\_TYPEKIND\_I4} and {\tt ESMF\_TYPEKIND\_I8},
            depending on whether the global number of elements held by the
            DistGrid is below or above the 32-bit limit, respectively.
            Because of the use of signed integers for sequence indices, 
            element counts of $ > 2^{31}-1 = 2,147,483,647$ will switch to 64-bit 
            indexing.
       \item[{[rc]}]
            Return code; equals {\tt ESMF\_SUCCESS} if there are no errors.
       \end{description}
   
%/////////////////////////////////////////////////////////////
 
\mbox{}\hrulefill\ 
 
\subsubsection [ESMF\_DistGridCreate] {ESMF\_DistGridCreate - Create 1D DistGrid object from user's arbitrary sequence index list 1 DE per PET}


 
\bigskip{\sf INTERFACE:}
\begin{verbatim}   ! Private name; call using ESMF_DistGridCreate()
   function ESMF_DistGridCreateDBAI1D1DE(arbSeqIndexList, rc)
           \end{verbatim}{\em RETURN VALUE:}
\begin{verbatim}     type(ESMF_DistGrid) :: ESMF_DistGridCreateDBAI1D1DE\end{verbatim}{\em ARGUMENTS:}
\begin{verbatim}     integer, intent(in)            :: arbSeqIndexList(:)
 -- The following arguments require argument keyword syntax (e.g. rc=rc). --
     integer, intent(out), optional :: rc\end{verbatim}
{\sf STATUS:}
   \begin{itemize}
   \item\apiStatusCompatibleVersion{5.2.0r}
   \end{itemize}
  
{\sf DESCRIPTION:\\ }


       Create an {\tt ESMF\_DistGrid} of {\tt dimCount} 1 from a PET-local list
       of sequence indices. The PET-local size of the {\tt arbSeqIndexList}
       argument determines the number of local elements in the created DistGrid.
       The sequence indices must be unique across all PETs. A default
       DELayout with 1 DE per PET across all PETs of the current VM is 
       automatically created.
  
       This is a {\em collective} method across the current VM.
  
       The arguments are:
       \begin{description}
       \item[arbSeqIndexList]
            List of arbitrary sequence indices that reside on the local PET.
       \item[{[rc]}]
            Return code; equals {\tt ESMF\_SUCCESS} if there are no errors.
       \end{description}
   
%/////////////////////////////////////////////////////////////
 
\mbox{}\hrulefill\ 
 
\subsubsection [ESMF\_DistGridCreate] {ESMF\_DistGridCreate - Create 1D DistGrid object from user's arbitrary 64-bit sequence index list 1 DE per PET}


 
\bigskip{\sf INTERFACE:}
\begin{verbatim}   ! Private name; call using ESMF_DistGridCreate()
   function ESMF_DistGridCreateDBAI1D1DEI8(arbSeqIndexList, rc)
           \end{verbatim}{\em RETURN VALUE:}
\begin{verbatim}     type(ESMF_DistGrid) :: ESMF_DistGridCreateDBAI1D1DEI8\end{verbatim}{\em ARGUMENTS:}
\begin{verbatim}     integer(ESMF_KIND_I8),  intent(in)            :: arbSeqIndexList(:)
 -- The following arguments require argument keyword syntax (e.g. rc=rc). --
     integer,                intent(out), optional :: rc\end{verbatim}
{\sf DESCRIPTION:\\ }


       Create an {\tt ESMF\_DistGrid} of {\tt dimCount} 1 from a PET-local list
       of sequence indices. The PET-local size of the {\tt arbSeqIndexList}
       argument determines the number of local elements in the created DistGrid.
       The sequence indices must be unique across all PETs. A default
       DELayout with 1 DE per PET across all PETs of the current VM is 
       automatically created.
  
       This is a {\em collective} method across the current VM.
  
       The arguments are:
       \begin{description}
       \item[arbSeqIndexList]
            List of arbitrary sequence indices that reside on the local PET.
       \item[{[rc]}]
            Return code; equals {\tt ESMF\_SUCCESS} if there are no errors.
       \end{description}
   
%/////////////////////////////////////////////////////////////
 
\mbox{}\hrulefill\ 
 
\subsubsection [ESMF\_DistGridCreate] {ESMF\_DistGridCreate - Create 1D DistGrid object from user's arbitrary sequence index list multiple DE/PET}


 
\bigskip{\sf INTERFACE:}
\begin{verbatim}   ! Private name; call using ESMF_DistGridCreate()
   function ESMF_DistGridCreateDBAI1D(arbSeqIndexList, rc)
           \end{verbatim}{\em RETURN VALUE:}
\begin{verbatim}     type(ESMF_DistGrid) :: ESMF_DistGridCreateDBAI1D\end{verbatim}{\em ARGUMENTS:}
\begin{verbatim}     type(ESMF_PtrInt1D), intent(in) :: arbSeqIndexList(:)
 -- The following arguments require argument keyword syntax (e.g. rc=rc). --
     integer, intent(out), optional  :: rc\end{verbatim}
{\sf DESCRIPTION:\\ }


       Create an {\tt ESMF\_DistGrid} of {\tt dimCount} 1 from a PET-local list
       of sequence index lists. The PET-local size of the {\tt arbSeqIndexList}
       argument determines the number of local DEs in the created DistGrid.
       Each of the local DEs is associated with as many index space elements as
       there are arbitrary sequence indices in the associated list.
       The sequence indices must be unique across all DEs. A default
       DELayout with the correct number of DEs per PET is automatically created.
  
       This is a {\em collective} method across the current VM.
  
       The arguments are:
       \begin{description}
       \item[arbSeqIndexList]
            List of arbitrary sequence index lists that reside on the local PET.
       \item[{[rc]}]
            Return code; equals {\tt ESMF\_SUCCESS} if there are no errors.
       \end{description}
   
%/////////////////////////////////////////////////////////////
 
\mbox{}\hrulefill\ 
 
\subsubsection [ESMF\_DistGridCreate] {ESMF\_DistGridCreate - Create (1+n)D DistGrid object from user's arbitrary sequence index list and minIndexPTile/maxIndexPTile}


 
\bigskip{\sf INTERFACE:}
\begin{verbatim}   ! Private name; call using ESMF_DistGridCreate()
   function ESMF_DistGridCreateDBAI(arbSeqIndexList, arbDim, &
     minIndexPTile, maxIndexPTile, rc)
           \end{verbatim}{\em RETURN VALUE:}
\begin{verbatim}     type(ESMF_DistGrid) :: ESMF_DistGridCreateDBAI\end{verbatim}{\em ARGUMENTS:}
\begin{verbatim}     integer, intent(in)            :: arbSeqIndexList(:)
     integer, intent(in)            :: arbDim
     integer, intent(in)            :: minIndexPTile(:)
     integer, intent(in)            :: maxIndexPTile(:)
 -- The following arguments require argument keyword syntax (e.g. rc=rc). --
     integer, intent(out), optional :: rc\end{verbatim}
{\sf STATUS:}
   \begin{itemize}
   \item\apiStatusCompatibleVersion{5.2.0r}
   \end{itemize}
  
{\sf DESCRIPTION:\\ }


       Create an {\tt ESMF\_DistGrid} of {\tt dimCount} $1+n$, where 
       $n=$ {\tt size(minIndexPTile)} = {\tt size(maxIndexPTile)}.
  
       The resulting DistGrid will have a 1D distribution determined by the
       PET-local {\tt arbSeqIndexList}. The PET-local size of the
       {\tt arbSeqIndexList} argument determines the number of local elements 
       along the arbitrarily distributed dimension in the created DistGrid. The
       sequence indices must be unique across all PETs. The associated,
       automatically created DELayout will have 1 DE per PET across all PETs of
       the current VM.
  
       In addition to the arbitrarily distributed dimension, regular DistGrid
       dimensions can be specified in {\tt minIndexPTile} and {\tt maxIndexPTile}. The
       $n$ dimensional subspace spanned by the regular dimensions is "multiplied"
       with the arbitrary dimension on each DE, to form a $1+n$ dimensional
       total index space described by the DistGrid object. The {\tt arbDim}
       argument allows to specify which dimension in the resulting DistGrid
       corresponds to the arbitrarily distributed one.
  
       This is a {\em collective} method across the current VM.
  
       The arguments are:
       \begin{description}
       \item[arbSeqIndexList]
            List of arbitrary sequence indices that reside on the local PET.
       \item[arbDim]
            Dimension of the arbitrary distribution.
       \item[minIndexPTile]
            Index space tuple of the lower corner of the tile. The 
            arbitrary dimension is {\em not} included in this tile
       \item[maxIndexPTile]
            Index space tuple of the upper corner of the tile. The
            arbitrary dimension is {\em not} included in this tile
       \item[{[rc]}]
            Return code; equals {\tt ESMF\_SUCCESS} if there are no errors.
       \end{description}
   
%/////////////////////////////////////////////////////////////
 
\mbox{}\hrulefill\ 
 
\subsubsection [ESMF\_DistGridDestroy] {ESMF\_DistGridDestroy - Release resources associated with a DistGrid }


 
\bigskip{\sf INTERFACE:}
\begin{verbatim}   subroutine ESMF_DistGridDestroy(distgrid, noGarbage, rc)\end{verbatim}{\em ARGUMENTS:}
\begin{verbatim}     type(ESMF_DistGrid), intent(inout)          :: distgrid
 -- The following arguments require argument keyword syntax (e.g. rc=rc). --
     logical,             intent(in),   optional :: noGarbage
     integer,             intent(out),  optional :: rc  
           \end{verbatim}
{\sf STATUS:}
   \begin{itemize}
   \item\apiStatusCompatibleVersion{5.2.0r}
   \item\apiStatusModifiedSinceVersion{5.2.0r}
   \begin{description}
   \item[7.0.0] Added argument {\tt noGarbage}.
     The argument provides a mechanism to override the default garbage collection
     mechanism when destroying an ESMF object.
   \end{description}
   \end{itemize}
  
{\sf DESCRIPTION:\\ }


     Destroys an {\tt ESMF\_DistGrid}, releasing the resources associated
     with the object.
  
     By default a small remnant of the object is kept in memory in order to 
     prevent problems with dangling aliases. The default garbage collection
     mechanism can be overridden with the {\tt noGarbage} argument.
  
   The arguments are:
   \begin{description}
   \item[distgrid] 
        {\tt ESMF\_DistGrid} object to be destroyed.
   \item[{[noGarbage]}]
        If set to {\tt .TRUE.} the object will be fully destroyed and removed
        from the ESMF garbage collection system. Note however that under this 
        condition ESMF cannot protect against accessing the destroyed object 
        through dangling aliases -- a situation which may lead to hard to debug 
        application crashes.
   
        It is generally recommended to leave the {\tt noGarbage} argument
        set to {\tt .FALSE.} (the default), and to take advantage of the ESMF 
        garbage collection system which will prevent problems with dangling
        aliases or incorrect sequences of destroy calls. However this level of
        support requires that a small remnant of the object is kept in memory
        past the destroy call. This can lead to an unexpected increase in memory
        consumption over the course of execution in applications that use 
        temporary ESMF objects. For situations where the repeated creation and 
        destruction of temporary objects leads to memory issues, it is 
        recommended to call with {\tt noGarbage} set to {\tt .TRUE.}, fully 
        removing the entire temporary object from memory.
   \item[{[rc]}] 
        Return code; equals {\tt ESMF\_SUCCESS} if there are no errors.
   \end{description}
   
%/////////////////////////////////////////////////////////////
 
\mbox{}\hrulefill\ 
 
\subsubsection [ESMF\_DistGridGet] {ESMF\_DistGridGet - Get object-wide DistGrid information}


 
\bigskip{\sf INTERFACE:}
\begin{verbatim}   ! Private name; call using ESMF_DistGridGet()
   subroutine ESMF_DistGridGetDefault(distgrid, delayout, &
     dimCount, tileCount, deCount, localDeCount, minIndexPTile, maxIndexPTile, &
     elementCountPTile, elementCountPTileI8, minIndexPDe, maxIndexPDe, &
     elementCountPDe, elementCountPDeI8, localDeToDeMap, deToTileMap, &
     indexCountPDe, collocation, regDecompFlag, indexTK, indexflag, &
     connectionCount, connectionList, rc)\end{verbatim}{\em ARGUMENTS:}
\begin{verbatim}     type(ESMF_DistGrid),      intent(in)            :: distgrid
 -- The following arguments require argument keyword syntax (e.g. rc=rc). --
     type(ESMF_DELayout),      intent(out), optional :: delayout
     integer,                  intent(out), optional :: dimCount
     integer,                  intent(out), optional :: tileCount
     integer,                  intent(out), optional :: deCount
     integer,                  intent(out), optional :: localDeCount
     integer,          target, intent(out), optional :: minIndexPTile(:,:)
     integer,          target, intent(out), optional :: maxIndexPTile(:,:)
     integer,          target, intent(out), optional :: elementCountPTile(:)
 integer(ESMF_KIND_I8),target, intent(out), optional :: elementCountPTileI8(:)
     integer,          target, intent(out), optional :: minIndexPDe(:,:)
     integer,          target, intent(out), optional :: maxIndexPDe(:,:)
     integer,          target, intent(out), optional :: elementCountPDe(:)
 integer(ESMF_KIND_I8),target, intent(out), optional :: elementCountPDeI8(:)
     integer,          target, intent(out), optional :: localDeToDeMap(:)
     integer,          target, intent(out), optional :: deToTileMap(:)
     integer,          target, intent(out), optional :: indexCountPDe(:,:)
     integer,          target, intent(out), optional :: collocation(:)
     logical,                  intent(out), optional :: regDecompFlag
     type(ESMF_TypeKind_Flag), intent(out), optional :: indexTK
     type(ESMF_Index_Flag),    intent(out), optional :: indexflag
     integer,                  intent(out), optional :: connectionCount
     type(ESMF_DistGridConnection), &
                       target, intent(out), optional :: connectionList(:)
     integer,                  intent(out), optional :: rc
           \end{verbatim}
{\sf STATUS:}
   \begin{itemize}
   \item\apiStatusCompatibleVersion{5.2.0r}
   \item\apiStatusModifiedSinceVersion{5.2.0r}
   \begin{description}
   \item[7.0.0] Added argument {\tt deCount} to simplify access to this 
                variable. \newline
                Added arguments {\tt connectionCount} and {\tt connectionList}
                to provide user access to the explicitly defined connections in
                a DistGrid.
   \item[8.0.0] Added arguments {\tt localDeCount} and {\tt localDeToDeMap}
                to simplify access to these variables.
   \item[8.1.0] Added argument {\tt indexTK} to allow query of the sequence index
                typekind.\newline
                Added arguments {\tt elementCountPTileI8} and
                {\tt elementCountPDeI8} to provide 64-bit access.\newline
                Added argument {\tt indexflag} to allow user to query this
                setting.
   \end{description}
   \end{itemize}
           
{\sf DESCRIPTION:\\ }


     Access internal DistGrid information.
  
     The arguments are:
     \begin{description}
     \item[distgrid] 
       Queried {\tt ESMF\_DistGrid} object.
     \item[{[delayout]}]
       {\tt ESMF\_DELayout} object associated with {\tt distgrid}.
     \item[{[dimCount]}]
       Number of dimensions (rank) of {\tt distgrid}.
     \item[{[tileCount]}]
       Number of tiles in {\tt distgrid}.
     \item[{[deCount]}]
       Number of DEs in the DELayout in {\tt distgrid}.
     \item[{[localDeCount]}]
       Number of local DEs in the DELayout in {\tt distgrid} on this PET.
     \item[{[minIndexPTile]}]
       \begin{sloppypar}
       Lower index space corner per tile. Must enter
       allocated with {\tt shape(minIndexPTile) == (/dimCount, tileCount/)}.
     \item[{[maxIndexPTile]}]
       Upper index space corner per tile. Must enter
       allocated with {\tt shape(maxIndexPTile) == (/dimCount, tileCount/)}.
     \item[{[elementCountPTile]}]
       Number of elements in the exclusive region per tile. Must enter
       allocated with {\tt shape(elementCountPTile) == (/tileCount/)}.
       An error will be returned if any of the counts goes above the 32-bit
       limit.
     \item[{[elementCountPTileI8]}]
       Same as {\tt elementCountPTile}, but of 64-bit integer kind.
     \item[{[minIndexPDe]}]
       Lower index space corner per DE. Must enter
       allocated with {\tt shape(minIndexPDe) == (/dimCount, deCount/)}.
     \item[{[maxIndexPDe]}]
       Upper index space corner per DE. Must enter
       allocated with {\tt shape(maxIndexPDe) == (/dimCount, deCount/)}.
     \item[{[elementCountPDe]}]
       Number of elements in the exclusive region per DE. Must enter
       allocated with {\tt shape(elementCountPDe) == (/deCount/)}.
       An error will be returned if any of the counts goes above the 32-bit
       limit.
     \item[{[elementCountPDeI8]}]
       Same as {\tt elementCountPDe}, but of 64-bit integer kind.
     \item[{[localDeToDeMap]}]
       Global DE index for each local DE. Must enter allocated with
       {\tt shape(localDeToDeMap) == (/localDeCount/)}.
     \item[{[deToTileMap]}]
       Map each DE uniquely to a tile. Must enter allocated with
       {\tt shape(deToTileMap) == (/deCount/)}.
     \item[{[indexCountPDe]}]
       Number of indices for each dimension per DE. Must enter
       allocated with {\tt shape(indexCountPDe) == (/dimCount, deCount/)}.
     \item[{[collocation]}]
       Collocation identifier for each dimension. Must enter
       allocated with {\tt shape(collocation) == (/dimCount/)}.
     \item[{[regDecompFlag]}]
       Decomposition scheme. A return value of {\tt .true.} indicates
       a regular decomposition, i.e. the decomposition is based on a 
       logically rectangular scheme with specific number of DEs along
       each dimension. A return value of {\tt .false.} indicates that the
       decomposition was {\em not} generated from a regular decomposition 
       description, e.g. a {\tt deBlockList} was used instead.
     \item[{[indexTK]}]
       Typekind used by the global sequence indexing. See section 
       \ref{const:typekind} for a list of typekind options. Only the integer
       types are supported for sequence indices.
     \item[{[indexflag]}]
       Return the indexing option used by the {\tt distgrid} object. See section
       \ref{const:indexflag} for a complete list of options.
     \item[{[connectionCount]}]
       Number of explicitly defined connections in {\tt distgrid}.
     \item[{[connectionList]}]
       List of explicitly defined connections in {\tt distgrid}. Must enter
       allocated with {\tt shape(connectionList) == (/connectionCount/)}.
     \item[{[rc]}] 
       Return code; equals {\tt ESMF\_SUCCESS} if there are no errors.
       \end{sloppypar}
     \end{description}
   
%/////////////////////////////////////////////////////////////
 
\mbox{}\hrulefill\ 
 
\subsubsection [ESMF\_DistGridGet] {ESMF\_DistGridGet - Get DE-local DistGrid information}


 
\bigskip{\sf INTERFACE:}
\begin{verbatim}   ! Private name; call using ESMF_DistGridGet()
   subroutine ESMF_DistGridGetPLocalDe(distgrid, localDe, &
     de, tile, collocation, arbSeqIndexFlag, seqIndexList, seqIndexListI8, &
     elementCount, elementCountI8, rc)\end{verbatim}{\em ARGUMENTS:}
\begin{verbatim}     type(ESMF_DistGrid),      intent(in)            :: distgrid
     integer,                  intent(in)            :: localDe
 -- The following arguments require argument keyword syntax (e.g. rc=rc). --
     integer,                  intent(out), optional :: de
     integer,                  intent(out), optional :: tile
     integer,                  intent(in),  optional :: collocation
     logical,                  intent(out), optional :: arbSeqIndexFlag
     integer,          target, intent(out), optional :: seqIndexList(:)
 integer(ESMF_KIND_I8),target, intent(out), optional :: seqIndexListI8(:)
     integer,                  intent(out), optional :: elementCount
     integer,                  intent(out), optional :: elementCountI8
     integer,                  intent(out), optional :: rc
           \end{verbatim}
{\sf STATUS:}
   \begin{itemize}
   \item\apiStatusCompatibleVersion{5.2.0r}
   \item\apiStatusModifiedSinceVersion{5.2.0r}
   \begin{description}
   \item[8.0.0] Added arguments {\tt de} and {\tt tile} to simplify usage.
   \item[8.1.0] Added arguments {\tt seqIndexListI8} and {\tt elementCountI8}
                to provide 64-bit access.
   \end{description}
   \end{itemize}
  
{\sf DESCRIPTION:\\ }


     Access internal DistGrid information.
  
     The arguments are:
     \begin{description}
     \item[distgrid]
       Queried {\tt ESMF\_DistGrid} object.
     \item[localDe]
       Local DE for which information is requested. {\tt [0,..,localDeCount-1]}
     \item[{[de]}]
       The global DE associated with the {\tt localDe}. DE indexing starts at 0.
     \item[{[tile]}]
       The tile on which the {\tt localDe} is located. Tile indexing starts at 1.
     \item[{[collocation]}]
       Collocation for which information is requested. Default to first
       collocation in {\tt collocation} list.
     \item[{[arbSeqIndexFlag]}]
       A returned value of {\tt .true.} indicates that the {\tt collocation}
       is associated with arbitrary sequence indices. For {\tt .false.},
       canoncial sequence indices are used.
     \item[{[seqIndexList]}]
       The sequence indices associated with the {\tt localDe}. This argument must
       enter allocated with a size equal to 
       {\tt elementCountPDe(localDeToDeMap(localDe))}.
       An error will be returned if any of the sequence indices are above the
       32-bit limit.
     \item[{[seqIndexListI8]}]
       Same as {\tt seqIndexList}, but of 64-bit integer kind.
     \item[{[elementCount]}]
       Number of elements in the localDe, i.e. identical to
       elementCountPDe(localDeToDeMap(localDe)).
       An error will be returned if the count is above the 32-bit limit.
     \item[{[elementCountI8]}]
       Same as {\tt elementCount}, but of 64-bit integer kind.
     \item[{[rc]}]
       Return code; equals {\tt ESMF\_SUCCESS} if there are no errors.
     \end{description}
   
%/////////////////////////////////////////////////////////////
 
\mbox{}\hrulefill\ 
 
\subsubsection [ESMF\_DistGridGet] {ESMF\_DistGridGet - Get DE-local DistGrid information for a specific dimension}


 
\bigskip{\sf INTERFACE:}
\begin{verbatim}   ! Private name; call using ESMF_DistGridGet()
   subroutine ESMF_DistGridGetPLocalDePDim(distgrid, localDe, dim, &
            indexList, rc)\end{verbatim}{\em ARGUMENTS:}
\begin{verbatim}     type(ESMF_DistGrid),    intent(in)            :: distgrid
     integer,                intent(in)            :: localDe
     integer,                intent(in)            :: dim
     integer,        target, intent(out)           :: indexList(:)
 -- The following arguments require argument keyword syntax (e.g. rc=rc). --
     integer,                intent(out), optional :: rc
           \end{verbatim}
{\sf STATUS:}
   \begin{itemize}
   \item\apiStatusCompatibleVersion{5.2.0r}
   \end{itemize}
  
{\sf DESCRIPTION:\\ }


     Access internal DistGrid information.
  
     The arguments are:
     \begin{description}
     \item[distgrid] 
       Queried {\tt ESMF\_DistGrid} object.
     \item[localDe] 
       Local DE for which information is requested. {\tt [0,..,localDeCount-1]}
     \item[dim] 
       Dimension for which information is requested. {\tt [1,..,dimCount]}
     \item[indexList]
       Upon return this holds the list of DistGrid tile-local indices
       for {\tt localDe} along dimension {\tt dim}. The supplied variable 
       must be at least of size {\tt indexCountPDe(dim, localDeToDeMap(localDe))}.
     \item[{[rc]}] 
       Return code; equals {\tt ESMF\_SUCCESS} if there are no errors.
     \end{description}
   
%/////////////////////////////////////////////////////////////
 
\mbox{}\hrulefill\ 
 
\subsubsection [ESMF\_DistGridIsCreated] {ESMF\_DistGridIsCreated - Check whether a DistGrid object has been created}


 
\bigskip{\sf INTERFACE:}
\begin{verbatim}   function ESMF_DistGridIsCreated(distgrid, rc)\end{verbatim}{\em RETURN VALUE:}
\begin{verbatim}     logical :: ESMF_DistGridIsCreated\end{verbatim}{\em ARGUMENTS:}
\begin{verbatim}     type(ESMF_DistGrid), intent(in)            :: distgrid
 -- The following arguments require argument keyword syntax (e.g. rc=rc). --
     integer,             intent(out), optional :: rc
 \end{verbatim}
{\sf DESCRIPTION:\\ }


     Return {\tt .true.} if the {\tt distgrid} has been created. Otherwise return
     {\tt .false.}. If an error occurs, i.e. {\tt rc /= ESMF\_SUCCESS} is 
     returned, the return value of the function will also be {\tt .false.}.
  
   The arguments are:
     \begin{description}
     \item[distgrid]
       {\tt ESMF\_DistGrid} queried.
     \item[{[rc]}]
       Return code; equals {\tt ESMF\_SUCCESS} if there are no errors.
     \end{description}
   
%/////////////////////////////////////////////////////////////
 
\mbox{}\hrulefill\ 
 
\subsubsection [ESMF\_DistGridMatch] {ESMF\_DistGridMatch - Check if two DistGrid objects match}


 
\bigskip{\sf INTERFACE:}
\begin{verbatim}   function ESMF_DistGridMatch(distgrid1, distgrid2, rc)\end{verbatim}{\em RETURN VALUE:}
\begin{verbatim}     type(ESMF_DistGridMatch_Flag) :: ESMF_DistGridMatch
       \end{verbatim}{\em ARGUMENTS:}
\begin{verbatim}     type(ESMF_DistGrid),  intent(in)            :: distgrid1
     type(ESMF_DistGrid),  intent(in)            :: distgrid2
 -- The following arguments require argument keyword syntax (e.g. rc=rc). --
     integer,              intent(out), optional :: rc  \end{verbatim}
{\sf DESCRIPTION:\\ }


     Determine to which level {\tt distgrid1} and {\tt distgrid2} match. 
  
     Returns a range of values of type {\tt ESMF\_DistGridMatch\_Flag},
     indicating how closely the DistGrids match. For a description of the
     possible return values, see~\ref{const:distgridmatch}. 
     Note that this call only performs PET local matching. Different return values
     may be returned on different PETs for the same DistGrid pair.
  
     The arguments are:
     \begin{description}
     \item[distgrid1] 
       {\tt ESMF\_DistGrid} object.
     \item[distgrid2] 
       {\tt ESMF\_DistGrid} object.
     \item[{[rc]}] 
       Return code; equals {\tt ESMF\_SUCCESS} if there are no errors.
     \end{description}
   
%/////////////////////////////////////////////////////////////
 
\mbox{}\hrulefill\ 
 
\subsubsection [ESMF\_DistGridPrint] {ESMF\_DistGridPrint - Print DistGrid information}


 
\bigskip{\sf INTERFACE:}
\begin{verbatim}   subroutine ESMF_DistGridPrint(distgrid, rc)\end{verbatim}{\em ARGUMENTS:}
\begin{verbatim}     type(ESMF_DistGrid),  intent(in)            :: distgrid
 -- The following arguments require argument keyword syntax (e.g. rc=rc). --
     integer,              intent(out), optional :: rc  
           \end{verbatim}
{\sf STATUS:}
   \begin{itemize}
   \item\apiStatusCompatibleVersion{5.2.0r}
   \end{itemize}
  
{\sf DESCRIPTION:\\ }


       Prints internal information about the specified {\tt ESMF\_DistGrid} 
       object to {\tt stdout}. \\
  
       The arguments are:
       \begin{description}
       \item[distgrid] 
            Specified {\tt ESMF\_DistGrid} object.
       \item[{[rc]}] 
            Return code; equals {\tt ESMF\_SUCCESS} if there are no errors.
       \end{description}
   
%/////////////////////////////////////////////////////////////
 
\mbox{}\hrulefill\ 
 
\subsubsection [ESMF\_DistGridSet] {ESMF\_DistGridSet - Set arbitrary sequence for a specific local DE}


 
\bigskip{\sf INTERFACE:}
\begin{verbatim}   ! Private name; call using ESMF_DistGridSet()
   subroutine ESMF_DistGridSetPLocalDe(distgrid, localDe, collocation, &
     seqIndexList, seqIndexListI8, rc)\end{verbatim}{\em ARGUMENTS:}
\begin{verbatim}     type(ESMF_DistGrid),      intent(inout)         :: distgrid
     integer,                  intent(in)            :: localDe
 -- The following arguments require argument keyword syntax (e.g. rc=rc). --
     integer,                  intent(in),  optional :: collocation
     integer,          target, intent(in),  optional :: seqIndexList(:)
 integer(ESMF_KIND_I8),target, intent(in),  optional :: seqIndexListI8(:)
     integer,                  intent(out), optional :: rc
           \end{verbatim}
{\sf DESCRIPTION:\\ }


     Set DistGrid information for a specific local DE.
  
     The arguments are:
     \begin{description}
     \item[distgrid]
       Queried {\tt ESMF\_DistGrid} object.
     \item[localDe]
       Local DE for which information is set. {\tt [0,..,localDeCount-1]}
     \item[{[collocation]}]
       Collocation for which information is set. Default to first
       collocation in {\tt collocation} list.
     \item[{[seqIndexList]}]
       Sequence indices for the elements on {\tt localDe}. The {\tt seqIndexList}
       must provide exactly {\tt elementCountPDe(localDeToDeMap(localDe)} many
       entries. When this argument is present, the {\tt indexTK} of
       {\tt distgrid} will be set to {\tt ESMF\_TYPEKIND\_I4}.
       This argument is mutually exclusive with {\tt seqIndexListI8}. Only one
       or the other must be provided. An error will be returned otherwise.
     \item[{[seqIndexListI8]}]
       Same as {\tt seqIndexList}, but of 64-bit integer kind.
       When this argument is present, the {\tt indexTK} of
       {\tt distgrid} will be set to {\tt ESMF\_TYPEKIND\_I8}.
       This argument is mutually exclusive with {\tt seqIndexList}. Only one
       or the other must be provided. An error will be returned otherwise.
     \item[{[rc]}]
       Return code; equals {\tt ESMF\_SUCCESS} if there are no errors.
     \end{description}
   
%/////////////////////////////////////////////////////////////
 
\mbox{}\hrulefill\ 
 
\subsubsection [ESMF\_DistGridValidate] {ESMF\_DistGridValidate - Validate DistGrid internals}


 
\bigskip{\sf INTERFACE:}
\begin{verbatim}   subroutine ESMF_DistGridValidate(distgrid, rc)\end{verbatim}{\em ARGUMENTS:}
\begin{verbatim}     type(ESMF_DistGrid),  intent(in)            :: distgrid
 -- The following arguments require argument keyword syntax (e.g. rc=rc). --
     integer,              intent(out), optional :: rc  
           \end{verbatim}
{\sf STATUS:}
   \begin{itemize}
   \item\apiStatusCompatibleVersion{5.2.0r}
   \end{itemize}
  
{\sf DESCRIPTION:\\ }


        Validates that the {\tt distgrid} is internally consistent.
        The method returns an error code if problems are found.  
  
       The arguments are:
       \begin{description}
       \item[distgrid] 
            Specified {\tt ESMF\_DistGrid} object.
       \item[{[rc]}] 
            Return code; equals {\tt ESMF\_SUCCESS} if there are no errors.
       \end{description}
  
%...............................................................
\setlength{\parskip}{\oldparskip}
\setlength{\parindent}{\oldparindent}
\setlength{\baselineskip}{\oldbaselineskip}
