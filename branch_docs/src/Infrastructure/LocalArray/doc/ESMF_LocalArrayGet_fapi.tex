%                **** IMPORTANT NOTICE *****
% This LaTeX file has been automatically produced by ProTeX v. 1.1
% Any changes made to this file will likely be lost next time
% this file is regenerated from its source. Send questions 
% to Arlindo da Silva, dasilva@gsfc.nasa.gov
 
\setlength{\oldparskip}{\parskip}
\setlength{\parskip}{1.5ex}
\setlength{\oldparindent}{\parindent}
\setlength{\parindent}{0pt}
\setlength{\oldbaselineskip}{\baselineskip}
\setlength{\baselineskip}{11pt}
 
%--------------------- SHORT-HAND MACROS ----------------------
\def\bv{\begin{verbatim}}
\def\ev{\end{verbatim}}
\def\be{\begin{equation}}
\def\ee{\end{equation}}
\def\bea{\begin{eqnarray}}
\def\eea{\end{eqnarray}}
\def\bi{\begin{itemize}}
\def\ei{\end{itemize}}
\def\bn{\begin{enumerate}}
\def\en{\end{enumerate}}
\def\bd{\begin{description}}
\def\ed{\end{description}}
\def\({\left (}
\def\){\right )}
\def\[{\left [}
\def\]{\right ]}
\def\<{\left  \langle}
\def\>{\right \rangle}
\def\cI{{\cal I}}
\def\diag{\mathop{\rm diag}}
\def\tr{\mathop{\rm tr}}
%-------------------------------------------------------------

\markboth{Left}{Source File: ESMF\_LocalArrayGet.F90,  Date: Tue May  5 20:59:39 MDT 2020
}

 
%/////////////////////////////////////////////////////////////
\subsubsection [ESMF\_LocalArrayGet] {ESMF\_LocalArrayGet - Get object-wide LocalArray information}


  
\bigskip{\sf INTERFACE:}
\begin{verbatim}   ! Private name; call using ESMF_LocalArrayGet()
   subroutine ESMF_LocalArrayGetDefault(localarray, &
     typekind, rank, totalCount, totalLBound, totalUBound, rc)\end{verbatim}{\em ARGUMENTS:}
\begin{verbatim}     type(ESMF_LocalArray), intent(in) :: localarray
 -- The following arguments require argument keyword syntax (e.g. rc=rc). --
     type(ESMF_TypeKind_Flag), intent(out), optional :: typekind
     integer, intent(out), optional :: rank
     integer, intent(out), optional :: totalCount(:)
     integer, intent(out), optional :: totalLBound(:)
     integer, intent(out), optional :: totalUBound(:)
     integer, intent(out), optional :: rc\end{verbatim}
{\sf STATUS:}
   \begin{itemize}
   \item\apiStatusCompatibleVersion{5.2.0r}
   \end{itemize}
  
{\sf DESCRIPTION:\\ }


   Returns information about the {\tt ESMF\_LocalArray}.
  
   The arguments are:
   \begin{description}
   \item[localarray]
   Queried {\tt ESMF\_LocalArray} object.
   \item[{[typekind]}]
   TypeKind of the LocalArray object.
   \item[{[rank]}]
   Rank of the LocalArray object.
   \item[{[totalCount]}]
   Count per dimension.
   \item[{[totalLBound]}]
   Lower bound per dimension.
   \item[{[totalUBound]}]
   Upper bound per dimension.
   \item[{[rc]}]
   Return code; equals {\tt ESMF\_SUCCESS} if there are no errors.
   \end{description}
   
%/////////////////////////////////////////////////////////////
 
\mbox{}\hrulefill\ 
 
\subsubsection [ESMF\_LocalArrayGet] {ESMF\_LocalArrayGet - Get a Fortran array pointer from a LocalArray }


   
\bigskip{\sf INTERFACE:}
\begin{verbatim}   ! Private name; call using ESMF_LocalArrayGet() 
   subroutine ESMF_LocalArrayGetData<rank><type><kind>(localarray, farrayPtr, & 
   datacopyflag, rc) 
   \end{verbatim}{\em ARGUMENTS:}
\begin{verbatim}   type(ESMF_LocalArray) :: localarray 
   <type> (ESMF_KIND_<kind>), pointer :: farrayPtr 
 -- The following arguments require argument keyword syntax (e.g. rc=rc). --
   type(ESMF_DataCopy_Flag), intent(in), optional :: datacopyflag 
   integer, intent(out), optional :: rc 
   \end{verbatim}
{\sf STATUS:}
   \begin{itemize} 
   \item\apiStatusCompatibleVersion{5.2.0r} 
   \end{itemize} 
   
{\sf DESCRIPTION:\\ }

 
   Return a Fortran pointer to the data buffer, or return a Fortran pointer 
   to a new copy of the data. 
   
   The arguments are: 
   \begin{description} 
   \item[localarray] 
   The {\tt ESMF\_LocalArray} to get the value from. 
   \item[farrayPtr] 
   An unassociated or associated Fortran pointer correctly allocated.
   \item[{[datacopyflag]}] 
   An optional copy flag which can be specified. 
   Can either make a new copy of the data or reference existing data. 
   See section \ref{const:datacopyflag} for a list of possible values. 
   \item[{[rc]}] 
   Return code; equals {\tt ESMF\_SUCCESS} if there are no errors. 
   \end{description} 
    
%/////////////////////////////////////////////////////////////
 
\mbox{}\hrulefill\ 
 
\subsubsection [ESMF\_LocalArrayIsCreated] {ESMF\_LocalArrayIsCreated - Check whether a LocalArray object has been created}


\bigskip{\sf INTERFACE:}
\begin{verbatim}   function ESMF_LocalArrayIsCreated(localarray, rc)\end{verbatim}{\em RETURN VALUE:}
\begin{verbatim}     logical :: ESMF_LocalArrayIsCreated\end{verbatim}{\em ARGUMENTS:}
\begin{verbatim}     type(ESMF_LocalArray), intent(in) :: localarray
 -- The following arguments require argument keyword syntax (e.g. rc=rc). --
     integer, intent(out), optional :: rc\end{verbatim}
{\sf DESCRIPTION:\\ }


   Return {\tt .true.} if the {\tt localarray} has been created. Otherwise return
   {\tt .false.}. If an error occurs, i.e. {\tt rc /= ESMF\_SUCCESS} is
   returned, the return value of the function will also be {\tt .false.}.
  
   The arguments are:
   \begin{description}
   \item[localarray]
   {\tt ESMF\_LocalArray} queried.
   \item[{[rc]}]
   Return code; equals {\tt ESMF\_SUCCESS} if there are no errors.
   \end{description}
  
%...............................................................
\setlength{\parskip}{\oldparskip}
\setlength{\parindent}{\oldparindent}
\setlength{\baselineskip}{\oldbaselineskip}
