%                **** IMPORTANT NOTICE *****
% This LaTeX file has been automatically produced by ProTeX v. 1.1
% Any changes made to this file will likely be lost next time
% this file is regenerated from its source. Send questions 
% to Arlindo da Silva, dasilva@gsfc.nasa.gov
 
\setlength{\oldparskip}{\parskip}
\setlength{\parskip}{1.5ex}
\setlength{\oldparindent}{\parindent}
\setlength{\parindent}{0pt}
\setlength{\oldbaselineskip}{\baselineskip}
\setlength{\baselineskip}{11pt}
 
%--------------------- SHORT-HAND MACROS ----------------------
\def\bv{\begin{verbatim}}
\def\ev{\end{verbatim}}
\def\be{\begin{equation}}
\def\ee{\end{equation}}
\def\bea{\begin{eqnarray}}
\def\eea{\end{eqnarray}}
\def\bi{\begin{itemize}}
\def\ei{\end{itemize}}
\def\bn{\begin{enumerate}}
\def\en{\end{enumerate}}
\def\bd{\begin{description}}
\def\ed{\end{description}}
\def\({\left (}
\def\){\right )}
\def\[{\left [}
\def\]{\right ]}
\def\<{\left  \langle}
\def\>{\right \rangle}
\def\cI{{\cal I}}
\def\diag{\mathop{\rm diag}}
\def\tr{\mathop{\rm tr}}
%-------------------------------------------------------------

\markboth{Left}{Source File: ESMF\_GridCreateSph2DPlus1Ex.F90,  Date: Tue May  5 20:59:48 MDT 2020
}

 
%/////////////////////////////////////////////////////////////

   \subsubsection{Create a 2D thick sphere grid}
  
   This example  illustrates the creation of a 2D spherical Grid 
   with a 3rd undistributed dimension.  The Grid contains both the center stagger location and a corner
   (i.e. Arakawa B-Grid). The size of the 2D horizontal Grid is gridSize(1) by gridSize(2). The number of
   vertical levels is gridSizeVert. 
%/////////////////////////////////////////////////////////////

 \begin{verbatim}
      ! Use ESMF framework module
      use ESMF
      implicit none

      ! Local variables  
      integer:: rc, finalrc
      integer:: myPet, npets, rootPet
      type(ESMF_VM):: vm
      type(ESMF_Config) :: config
      type(ESMF_DistGrid) :: distgrid
      type(ESMF_Array) :: gridCoordArrays(3,2)
      type(ESMF_Array) :: gridCntrCoordArrayLat,gridCntrCoordArrayLon
      type(ESMF_Array) :: gridNECnrCoordArrayLat,gridNECnrCoordArrayLon
      type(ESMF_Array) :: cntArrayData1
      type(ESMF_StaggerLoc) :: staggerLocs(2), neCenterStaggerLoc
      type(ESMF_ArraySpec) ::  arrayspec1D,arrayspec2D,arrayspec3D
      integer :: gridSize(2), gridVert_size
      integer, allocatable ::  connectionList(:,:)
 
\end{verbatim}
 
%/////////////////////////////////////////////////////////////

   Construct a single tile spherical domain without connection across
   the poles. 
%/////////////////////////////////////////////////////////////

 \begin{verbatim}

      allocate( connectionList(2*2,2) )
      call ESMF_ConnectionElementConstruct(                          &
                          connectionElement=connectionList(:,1),     &
                          tileIndexA=1, tileIndexB=1,              &
                          positionVector = (/gridSize(1),0/),        &
                          repetitionVector= (/1,0/), rc=rc)

      distgrid = ESMF_DistGridCreate( minCorner=(/1,1/),             &
                          maxCorner=(/gridSize(1),gridSize(2)/),     &
                          connectionList=connectionList, rc=rc)  

      deallocate( connectionList )

 
\end{verbatim}
 
%/////////////////////////////////////////////////////////////

    Create arrays into which to put the 2D spherical coordinates. Create one array for each stagger location.  
%/////////////////////////////////////////////////////////////

 \begin{verbatim}
      call ESMF_ArraySpecSet(arrayspec2D, type=ESMF_DATA_REAL,         &
                 kind=ESMF_R8, rank=2)

      gridCntrCoordArrayLon = ESMF_ArrayCreate(arrayspec=arrayspec2D, &
                                                   distgrid=distgrid, rc=rc)
      gridCntrCoordArrayLat = ESMF_ArrayCreate(arrayspec=arrayspec2D, &
                                                  distgrid=distgrid, rc=rc)
      gridNECnrCoordArrayLon = ESMF_ArrayCreate(arrayspec=arrayspec2D,&
                                                      distgrid=distgrid, rc=rc)
      gridNECnrCoordArrayLat = ESMF_ArrayCreate(arrayspec=arrayspec2D, &
                                                     distgrid=distgrid, rc=rc)
 
\end{verbatim}
 
%/////////////////////////////////////////////////////////////

    Create array into which to put the vertical coordinate.
   Note that this array creation subroutine is currently not implemented. What it would do is
   create an undistributed array, but make a copy on each DE.  
%/////////////////////////////////////////////////////////////

 \begin{verbatim}
      call ESMF_ArraySpecSet(arrayspec1D, type=ESMF_DATA_REAL,         &
                 kind=ESMF_R8, rank=1)

      gridCoordArrayVert = ESMF_ArrayCreate(arrayspec=arrayspec1D, &
                                             distgrid=distgrid, dimmap=(/0,0/), &
                                             lbound=(/1/), ubound=(/gridSizeVert/), &
                                             rc=rc) 
 
\end{verbatim}
 
%/////////////////////////////////////////////////////////////

   Now that the arrays are created we could set the coordinates in them.
   (The following are user subroutines.) 
%/////////////////////////////////////////////////////////////

 \begin{verbatim}
!  call SetHorzCoords(gridCntrCoordArrayLon,gridCntrCoordArrayLat, &
!           gridNECnrCoordArrayLon,gridNECnrCoordArrayLat)
!  call SetVertCoords(gridrCoordArrayVert)

 
\end{verbatim}
 
%/////////////////////////////////////////////////////////////

    Create stagger location corresponding to the center of the north east corner
    edge. (i.e. the same vertical level as the 3D center, but in the northeast corner). 
%/////////////////////////////////////////////////////////////

 \begin{verbatim}

      call ESMF_StaggerLocSet(neCenterStaggerLoc,where=(/1,1,0/),rc=rc)

 
\end{verbatim}
 
%/////////////////////////////////////////////////////////////

   Load Stagger location and corresponding coordinate arrays into array of ESMF Arrays.
   
%/////////////////////////////////////////////////////////////

 \begin{verbatim}
      ! Setup center stagger.
     staggerlocs(1)=ESMF_STAGGERLOC_CENTER
     gridCoordArrays(1,1)=gridCntrCoordArrayLon     
     gridCoordArrays(2,1)=gridCntrCoordArrayLat     
     gridCoordArrays(3,1)=gridCoordArrayVert     

      ! Setup corner stagger.
     staggerlocs(2)=neCenterStaggerLoc
     gridCoordArrays(1,2)=gridNECnrCoordArrayLon     
     gridCoordArrays(2,2)=gridNECnrCoordArrayLat     
     gridCoordArrays(3,2)=gridCoordArrayVert     

 
\end{verbatim}
 
%/////////////////////////////////////////////////////////////

      Create a Grid from the coordinate arrays.  
%/////////////////////////////////////////////////////////////

 \begin{verbatim}
     sphericalGrid = ESMF_GridCreate(arrays=gridCoordArrays, &
                                  staggerLocs=staggerlocs,rc=rc)
 
\end{verbatim}
 
%/////////////////////////////////////////////////////////////

    Create a 3D array corresponding to the center stagger into which to put data. 
%/////////////////////////////////////////////////////////////

 \begin{verbatim}
      call ESMF_ArraySpecSet(arrayspec3D, type=ESMF_DATA_REAL,         &
                 kind=ESMF_R8, rank=3)

      cntrArrayData1 = ESMF_ArrayCreate(arrayspec=arrayspec3D,  &
                                        distgrid=distgrid, &
                                        lbounds=(/1/),ubounds=(/gridSizeVert/), &
                                        rc=rc)
 
\end{verbatim}

%...............................................................
\setlength{\parskip}{\oldparskip}
\setlength{\parindent}{\oldparindent}
\setlength{\baselineskip}{\oldbaselineskip}
