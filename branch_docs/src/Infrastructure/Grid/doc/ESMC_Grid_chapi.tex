%                **** IMPORTANT NOTICE *****
% This LaTeX file has been automatically produced by ProTeX v. 1.1
% Any changes made to this file will likely be lost next time
% this file is regenerated from its source. Send questions 
% to Arlindo da Silva, dasilva@gsfc.nasa.gov
 
\setlength{\oldparskip}{\parskip}
\setlength{\parskip}{1.5ex}
\setlength{\oldparindent}{\parindent}
\setlength{\parindent}{0pt}
\setlength{\oldbaselineskip}{\baselineskip}
\setlength{\baselineskip}{11pt}
 
%--------------------- SHORT-HAND MACROS ----------------------
\def\bv{\begin{verbatim}}
\def\ev{\end{verbatim}}
\def\be{\begin{equation}}
\def\ee{\end{equation}}
\def\bea{\begin{eqnarray}}
\def\eea{\end{eqnarray}}
\def\bi{\begin{itemize}}
\def\ei{\end{itemize}}
\def\bn{\begin{enumerate}}
\def\en{\end{enumerate}}
\def\bd{\begin{description}}
\def\ed{\end{description}}
\def\({\left (}
\def\){\right )}
\def\[{\left [}
\def\]{\right ]}
\def\<{\left  \langle}
\def\>{\right \rangle}
\def\cI{{\cal I}}
\def\diag{\mathop{\rm diag}}
\def\tr{\mathop{\rm tr}}
%-------------------------------------------------------------

\markboth{Left}{Source File: ESMC\_Grid.h,  Date: Tue May  5 20:59:48 MDT 2020
}

 
%/////////////////////////////////////////////////////////////
\subsubsection [ESMC\_GridCreateNoPeriDim] {ESMC\_GridCreateNoPeriDim - Create a Grid with no periodic dimensions}


  
\bigskip{\sf INTERFACE:}
\begin{verbatim} ESMC_Grid ESMC_GridCreateNoPeriDim(
   ESMC_InterArrayInt *maxIndex,           // in
   enum ESMC_CoordSys_Flag *coordSys,      // in
   enum ESMC_TypeKind_Flag *coordTypeKind, // in
   enum ESMC_IndexFlag *indexflag,         // in
   int *rc                                 // out
 );\end{verbatim}{\em RETURN VALUE:}
\begin{verbatim}    type(ESMC_Grid)\end{verbatim}
{\sf DESCRIPTION:\\ }


  
    This call creates an ESMC\_Grid with no periodic dimensions.
  
    The arguments are:
    \begin{description}
    \item[maxIndex]
        The upper extent of the grid array.
    \item[coordSys]
        The coordinated system of the grid coordinate data. If not specified then
        defaults to ESMF\_COORDSYS\_SPH\_DEG.
    \item[coordTypeKind]
        The type/kind of the grid coordinate data.  If not specified then the
        type/kind will be 8 byte reals.
    \item[indexflag]
        Indicates the indexing scheme to be used in the new Grid. If not present,
        defaults to ESMC\_INDEX\_DELOCAL.
    \item[rc]
        Return code; equals {\tt ESMF\_SUCCESS} if there are no errors.
    \end{description}
   
%/////////////////////////////////////////////////////////////
 
\mbox{}\hrulefill\ 
 
\subsubsection [ESMC\_GridCreate1PeriDim] {ESMC\_GridCreate1PeriDim - Create a Grid with 1 periodic dimension}


  
\bigskip{\sf INTERFACE:}
\begin{verbatim} ESMC_Grid ESMC_GridCreate1PeriDim(
   ESMC_InterArrayInt *maxIndex,           // in
   ESMC_InterArrayInt *polekindflag,       // in
   int *periodicDim,                       // in
   int *poleDim,                           // in
   enum ESMC_CoordSys_Flag *coordSys,      // in
   enum ESMC_TypeKind_Flag *coordTypeKind, // in
   enum ESMC_IndexFlag *indexflag,         // in
   int *rc                                 // out
 );\end{verbatim}{\em RETURN VALUE:}
\begin{verbatim}    type(ESMC_Grid)\end{verbatim}
{\sf DESCRIPTION:\\ }


  
    This call creates an ESMC\_Grid with 1 periodic dimension.
  
    The arguments are:
    \begin{description}
    \item[maxIndex]
        The upper extent of the grid array.
    \item[polekindflag]
        Two item array which specifies the type of connection which occurs at the
        pole. polekindflag(1) the connection that occurs at the minimum end of the
        index dimension. polekindflag(2) the connection that occurs at the maximum
        end of the index dimension. If not specified, the default is
        ESMF\_POLETYPE\_MONOPOLE for both.
    \item[periodicDim]
        The periodic dimension.  If not specified, defaults to 1.
    \item[poleDim]
        The dimension at which the poles are located at the ends.  If not
        specified, defaults to 2.
    \item[coordSys]
        The coordinated system of the grid coordinate data. If not specified then
        defaults to ESMF\_COORDSYS\_SPH\_DEG.
    \item[coordTypeKind]
        The type/kind of the grid coordinate data.  If not specified then the
        type/kind will be 8 byte reals.
    \item[indexflag]
        Indicates the indexing scheme to be used in the new Grid. If not present,
        defaults to ESMC\_INDEX\_DELOCAL.
    \item[rc]
        Return code; equals {\tt ESMF\_SUCCESS} if there are no errors.
    \end{description}
   
%/////////////////////////////////////////////////////////////
 
\mbox{}\hrulefill\ 
 
\subsubsection [ESMC\_GridCreateCubedSphere] {ESMC\_GridCreateCubedSphere - Create a cubed sphere Grid}


  
\bigskip{\sf INTERFACE:}
\begin{verbatim} ESMC_Grid ESMC_GridCreateCubedSphere(
   int *tilesize,                      // in
   ESMC_InterArrayInt *regDecompPTile, // in
   //ESMC_InterArrayInt *decompFlagPTile,  // in
   //ESMC_InterArrayInt *deLabelList,    // in
   //ESMC_DELayout *delayout,            // in
   ESMC_InterArrayInt *staggerLocList,   // in
   const char *name,                   // in
   int *rc);                           // out\end{verbatim}{\em RETURN VALUE:}
\begin{verbatim}    type(ESMC_Grid)\end{verbatim}
{\sf DESCRIPTION:\\ }


  
    Create a six-tile {\tt ESMC\_Grid} for a cubed sphere grid using regular
    decomposition.  Each tile can have different decomposition. The grid
    coordinates are generated based on the algorithm used by GEOS-5. The tile
    resolution is defined by tileSize.
  
    The arguments are:
    \begin{description}
    \item[tilesize]
        The number of elements on each side of the tile of the cubed sphere grid.
    \item[regDecompPTile]
        List of DE counts for each dimension. The second index steps through
        the tiles. The total {\tt deCount} is determined as the sum over
        the products of {\tt regDecompPTile} elements for each tile.
        By default every tile is decomposed in the same way.  If the total
        PET count is less than 6, one tile will be assigned to one DE and the DEs
        will be assigned to PETs sequentially, therefore, some PETs may have
        more than one DE. If the total PET count is greater than 6, the total
        number of DEs will be a multiple of 6 and less than or equal to the total
        PET count. For instance, if the total PET count is 16, the total DE count
        will be 12 with each tile decomposed into 1x2 blocks. The 12 DEs are mapped
        to the first 12 PETs and the remaining 4 PETs have no DEs locally, unless
        an optional {\tt delayout} is provided.
    \item[staggerLocList]
        The list of stagger locations to fill with coordinates. Only {\tt ESMF\_STAGGERLOC\_CENTER} and
        {\tt ESMF\_STAGGERLOC\_CORNER} are supported. If not present, no coordinates will be added or filled.
    \item[name]
        The name of the {\tt ESMC\_Grid}.
    \item[rc]
        Return code; equals {\tt ESMF\_SUCCESS} if there are no errors.
    \end{description}
   
%/////////////////////////////////////////////////////////////
 
\mbox{}\hrulefill\ 
 
\subsubsection [ESMC\_GridCreateFromFile] {ESMC\_GridCreateFromFile - Create a Grid from a NetCDF file specification.}


  
\bigskip{\sf INTERFACE:}
\begin{verbatim} ESMC_Grid ESMC_GridCreateFromFile(const char *filename, int fileTypeFlag, 
                   int *regDecomp, int *decompflag,
                   int *isSphere, ESMC_InterArrayInt *polekindflag,
                   int *addCornerStagger,
                   int *addUserArea, enum ESMC_IndexFlag *indexflag,
                   int *addMask, const char *varname,
                   const char **coordNames, int *rc);\end{verbatim}{\em RETURN VALUE:}
\begin{verbatim}    type(ESMC_Grid)\end{verbatim}
{\sf DESCRIPTION:\\ }


   This function creates a {\tt ESMC\_Grid} object from the specification in
   a NetCDF file.
  
    The arguments are:
    \begin{description}
   \item[filename]
       The NetCDF Grid filename.
   \item[fileTypeFlag]
       The Grid file format, please see Section~\ref{const:cfileformat}
           for a list of valid options. 
   \item[regDecomp] 
        A 2 element array specifying how the grid is decomposed.
        Each entry is the number of decounts for that dimension.
        The total decounts cannot exceed the total number of PETs.  In other
        word, at most one DE is allowed per processor.
        If not specified, the default decomposition will be petCountx1.
   \item[{[decompflag]}]
        List of decomposition flags indicating how each dimension of the
        tile is to be divided between the DEs. The default setting
        is {\tt ESMF\_DECOMP\_BALANCED} in all dimensions. Please see
        Section~\ref{const:cdecompflag} for a full description of the 
        possible options. 
   \item[{[isSphere]}]
        Set to 1 for a spherical grid, or 0 for regional. Defaults to 1.
    \item[polekindflag]
        Two item array which specifies the type of connection which occurs at the
        pole. polekindflag(1) the connection that occurs at the minimum end of the
        index dimension. polekindflag(2) the connection that occurs at the maximum
        end of the index dimension. If not specified, the default is
        ESMF\_POLETYPE\_MONOPOLE for both.
   \item[{[addCornerStagger]}]
        Set to 1 to use the information in the grid file to add the Corner stagger to 
        the Grid. The coordinates for the corner stagger are required for conservative
        regridding. If not specified, defaults to 0. 
   \item[{[addUserArea]}]
        Set to 1 to read in the cell area from the Grid file; otherwise, ESMF will 
        calculate it.  This feature is only supported when the grid file is in the SCRIP
        format.  
    \item[indexflag]
        Indicates the indexing scheme to be used in the new Grid. If not present,
        defaults to ESMC\_INDEX\_DELOCAL.
   \item[{[addMask]}]
        Set to 1 to generate the mask using the missing\_value attribute defined in 'varname'.
        This flag is only needed when the grid file is in the GRIDSPEC format.
   \item[{[varname]}]
        If addMask is non-zero, provide a variable name stored in the grid file and
        the mask will be generated using the missing value of the data value of
        this variable.  The first two dimensions of the variable has to be the
        longitude and the latitude dimension and the mask is derived from the
        first 2D values of this variable even if this data is 3D, or 4D array.
  \item[{[coordNames]}]
        A two-element array containing the longitude and latitude variable names in a
        GRIDSPEC file if there are multiple coordinates defined in the file.
   \item[{[rc]}]
        Return code; equals {\tt ESMF\_SUCCESS} if there are no errors.
    \end{description}
   
%/////////////////////////////////////////////////////////////
 
\mbox{}\hrulefill\ 
 
\subsubsection [ESMC\_GridDestroy] {ESMC\_GridDestroy - Destroy a Grid}


  
\bigskip{\sf INTERFACE:}
\begin{verbatim} int ESMC_GridDestroy(
   ESMC_Grid *grid             // in
 );
 \end{verbatim}{\em RETURN VALUE:}
\begin{verbatim}    Return code; equals ESMF_SUCCESS if there are no errors.\end{verbatim}
{\sf DESCRIPTION:\\ }


    Destroy the Grid.
  
    The arguments are:
    \begin{description}
    \item[grid]
      Grid object whose memory is to be freed. 
    \end{description}
   
%/////////////////////////////////////////////////////////////
 
\mbox{}\hrulefill\ 
 
\subsubsection [ESMC\_GridAddItem] {ESMC\_GridAddItem - Add items to a Grid}


  
\bigskip{\sf INTERFACE:}
\begin{verbatim} int ESMC_GridAddItem(
   ESMC_Grid grid,                   // in
   enum ESMC_GridItem_Flag itemflag, // in
   enum ESMC_StaggerLoc staggerloc   // in
 );
 \end{verbatim}{\em RETURN VALUE:}
\begin{verbatim}    Return code; equals ESMF_SUCCESS if there are no errors.\end{verbatim}
{\sf DESCRIPTION:\\ }


    Add an item (e.g. a mask) to the Grid.
  
    The arguments are:
    \begin{description}
    \item[grid]
      Grid object to which the coordinates will be added
    \item[itemflag]
      The grid item to add.
    \item[staggerloc]
      The stagger location to add.
    \end{description}
   
%/////////////////////////////////////////////////////////////
 
\mbox{}\hrulefill\ 
 
\subsubsection [ESMC\_GridGetItem] {ESMC\_GridGetItem - Get item from a Grid}


  
\bigskip{\sf INTERFACE:}
\begin{verbatim} void * ESMC_GridGetItem(
   ESMC_Grid grid,                         // in
   enum ESMC_GridItem_Flag itemflag,       // in
   enum ESMC_StaggerLoc staggerloc,        // in
   int *localDE,                           // in
   int *rc                                 // out
 );
 \end{verbatim}{\em RETURN VALUE:}
\begin{verbatim}    A pointer to the item data. \end{verbatim}
{\sf DESCRIPTION:\\ }


    Get a pointer to item data (e.g. mask data) in the Grid.
  
    The arguments are:
    \begin{description}
    \item[grid]
      Grid object from which to obtain the coordinates.
    \item[itemflag]
      The grid item to add.
    \item[staggerloc]
      The stagger location to add.
    \item[localDE]
      The local decompositional element. If not present, defaults to 0.
    \item[rc]
    Return code; equals {\tt ESMF\_SUCCESS} if there are no errors. 
    \end{description}
   
%/////////////////////////////////////////////////////////////
 
\mbox{}\hrulefill\ 
 
\subsubsection [ESMC\_GridAddCoord] {ESMC\_GridAddCoord - Add coordinates to a Grid}


  
\bigskip{\sf INTERFACE:}
\begin{verbatim} int ESMC_GridAddCoord(
   ESMC_Grid grid,                   // in
   enum ESMC_StaggerLoc staggerloc   // in
 );
 \end{verbatim}{\em RETURN VALUE:}
\begin{verbatim}    Return code; equals ESMF_SUCCESS if there are no errors.\end{verbatim}
{\sf DESCRIPTION:\\ }


    Add coordinates to the Grid.
  
    The arguments are:
    \begin{description}
    \item[grid]
      Grid object to which the coordinates will be added
    \item[staggerloc]
      The stagger location to add.
    \end{description}
   
%/////////////////////////////////////////////////////////////
 
\mbox{}\hrulefill\ 
 
\subsubsection [ESMC\_GridGetCoord] {ESMC\_GridGetCoord - Get coordinates from a Grid}


  
\bigskip{\sf INTERFACE:}
\begin{verbatim} void * ESMC_GridGetCoord(
   ESMC_Grid grid,                         // in
   int coordDim,                           // in
   enum ESMC_StaggerLoc staggerloc,        // in
   int *localDE,
   int *exclusiveLBound,                   // out
   int *exclusiveUBound,                   // out
   int *rc                                 // out
 );
 \end{verbatim}{\em RETURN VALUE:}
\begin{verbatim}    A pointer to coordinate data in the Grid. \end{verbatim}
{\sf DESCRIPTION:\\ }


    Get a pointer to coordinate data in the Grid.
  
    The arguments are:
    \begin{description}
    \item[grid]
      Grid object from which to obtain the coordinates.
    \item[coordDim]
      The coordinate dimension from which to get the data.
    \item[staggerloc]
      The stagger location to add.
    \item[localDE]
      The local decompositional element. If not present, defaults to 0.
    \item[exclusiveLBound]
      Upon return this holds the lower bounds of the exclusive region. This bound
      must be allocated to be of size equal to the coord dimCount.  
    \item[exclusiveUBound]
      Upon return this holds the upper bounds of the exclusive region. This bound
      must be allocated to be of size equal to the coord dimCount.  
    \item[rc]
    Return code; equals {\tt ESMF\_SUCCESS} if there are no errors. 
    \end{description}
   
%/////////////////////////////////////////////////////////////
 
\mbox{}\hrulefill\ 
 
\subsubsection [ESMC\_GridGetCoordBounds] {ESMC\_GridGetCoordBounds - Get coordinate bounds from a Grid}


  
\bigskip{\sf INTERFACE:}
\begin{verbatim} int ESMC_GridGetCoordBounds(
   ESMC_Grid grid,                         // in
   enum ESMC_StaggerLoc staggerloc,        // in
   int *localDE,                           // in
   int *exclusiveLBound,                   // out
   int *exclusiveUBound,                   // out
   int *rc                                 // out
 );
 \end{verbatim}{\em RETURN VALUE:}
\begin{verbatim}    Return code; equals ESMF_SUCCESS if there are no errors.\end{verbatim}
{\sf DESCRIPTION:\\ }


    Get coordinates bounds from the Grid.
  
    The arguments are:
    \begin{description}
    \item[grid]
      Grid object from which to obtain the coordinates.
    \item[staggerloc]
      The stagger location to add.
    \item[localDE]
      The local decompositional element. If not present, defaults to 0.
    \item[exclusiveLBound]
      Upon return this holds the lower bounds of the exclusive region. This bound
      must be allocated to be of size equal to the coord dimCount.  
    \item[exclusiveUBound]
      Upon return this holds the upper bounds of the exclusive region. This bound
      must be allocated to be of size equal to the coord dimCount.  
    \item[rc]
    Return code; equals {\tt ESMF\_SUCCESS} if there are no errors. 
    \end{description}
  
%...............................................................
\setlength{\parskip}{\oldparskip}
\setlength{\parindent}{\oldparindent}
\setlength{\baselineskip}{\oldbaselineskip}
