%                **** IMPORTANT NOTICE *****
% This LaTeX file has been automatically produced by ProTeX v. 1.1
% Any changes made to this file will likely be lost next time
% this file is regenerated from its source. Send questions 
% to Arlindo da Silva, dasilva@gsfc.nasa.gov
 
\setlength{\oldparskip}{\parskip}
\setlength{\parskip}{1.5ex}
\setlength{\oldparindent}{\parindent}
\setlength{\parindent}{0pt}
\setlength{\oldbaselineskip}{\baselineskip}
\setlength{\baselineskip}{11pt}
 
%--------------------- SHORT-HAND MACROS ----------------------
\def\bv{\begin{verbatim}}
\def\ev{\end{verbatim}}
\def\be{\begin{equation}}
\def\ee{\end{equation}}
\def\bea{\begin{eqnarray}}
\def\eea{\end{eqnarray}}
\def\bi{\begin{itemize}}
\def\ei{\end{itemize}}
\def\bn{\begin{enumerate}}
\def\en{\end{enumerate}}
\def\bd{\begin{description}}
\def\ed{\end{description}}
\def\({\left (}
\def\){\right )}
\def\[{\left [}
\def\]{\right ]}
\def\<{\left  \langle}
\def\>{\right \rangle}
\def\cI{{\cal I}}
\def\diag{\mathop{\rm diag}}
\def\tr{\mathop{\rm tr}}
%-------------------------------------------------------------

\markboth{Left}{Source File: ESMF\_StaggerLoc.F90,  Date: Tue May  5 20:59:49 MDT 2020
}

 
%/////////////////////////////////////////////////////////////
\subsubsection [ESMF\_StaggerLocGet] {ESMF\_StaggerLocGet - Get the value of one dimension of a StaggerLoc}


 
\bigskip{\sf INTERFACE:}
\begin{verbatim}   ! Private name; call using ESMF_StaggerLocGet() 
       subroutine ESMF_StaggerLocGetDim(staggerloc, dim, loc, &
            rc)\end{verbatim}{\em ARGUMENTS:}
\begin{verbatim}       type (ESMF_StaggerLoc), intent(in)  :: staggerloc
       integer,                intent(in)  :: dim
 -- The following arguments require argument keyword syntax (e.g. rc=rc). --
       integer, optional,      intent(out) :: loc
       integer, optional                   :: rc 
 \end{verbatim}
{\sf DESCRIPTION:\\ }


     Gets the position of a particular dimension of a cell {\tt staggerloc}
     The argument {\tt loc} will be only be 0,1. 
      If {\tt loc} is 0 it means the position 
      should be in the center in that dimension. If {\tt loc} is +1 then
      for the dimension, the position should be on the positive side of the cell. 
      Please see Section~\ref{sec:usage:staggerloc:adv} for diagrams.
  
       The arguments are:
       \begin{description}
       \item[staggerloc]
            Stagger location for which to get information. 
       \item[dim]
            Dimension for which to get information (1-7).
       \item[{[loc]}]
            Output position data (should be either 0,1).
       \item[{[rc]}]
            Return code; equals {\tt ESMF\_SUCCESS} if there are no errors.
     \end{description}
   
%/////////////////////////////////////////////////////////////
 
\mbox{}\hrulefill\ 
 
\subsubsection [ESMF\_StaggerLocSet] {ESMF\_StaggerLocSet - Set a StaggerLoc to a particular position in the cell}


 
\bigskip{\sf INTERFACE:}
\begin{verbatim}   ! Private name; call using ESMF_StaggerLocSet() 
      subroutine ESMF_StaggerLocSetAllDim(staggerloc, loc, rc)\end{verbatim}{\em ARGUMENTS:}
\begin{verbatim}       type (ESMF_StaggerLoc), intent(inout) :: staggerloc
       integer,                intent(in)    :: loc(:)
 -- The following arguments require argument keyword syntax (e.g. rc=rc). --
       integer, optional                     :: rc 
 \end{verbatim}
{\sf STATUS:}
   \begin{itemize}
   \item\apiStatusCompatibleVersion{5.2.0r}
   \end{itemize}
  
{\sf DESCRIPTION:\\ }


      Sets a custom {\tt staggerloc} to a position in a cell by using the array
      {\tt loc}. The values in the array should only be 0,1. If loc(i) is 0 it 
  !    means the position should be in the center in that dimension. If loc(i) is 1 then
      for dimension i, the position should be on the side of the cell. 
      Please see Section~\ref{sec:usage:staggerloc:adv}
      for diagrams and further discussion of custom stagger locations. 
  
       The arguments are:
       \begin{description}
       \item[staggerloc]
            Grid location to be initialized
       \item[loc]
            Array holding position data. Each entry in {\tt loc} should only
            be  0 or 1. note that dimensions beyond those specified are set to 0. 
       \item[{[rc]}]
            Return code; equals {\tt ESMF\_SUCCESS} if there are no errors.
     \end{description}
   
%/////////////////////////////////////////////////////////////
 
\mbox{}\hrulefill\ 
 
\subsubsection [ESMF\_StaggerLocSet] {ESMF\_StaggerLocSet - Set one dimension of a StaggerLoc to a particular position}


 
\bigskip{\sf INTERFACE:}
\begin{verbatim}   ! Private name; call using ESMF_StaggerLocSet() 
        subroutine ESMF_StaggerLocSetDim(staggerloc, dim, loc, &
             rc)\end{verbatim}{\em ARGUMENTS:}
\begin{verbatim}       type (ESMF_StaggerLoc), intent(inout) :: staggerloc
       integer,                intent(in)    :: dim
       integer,                intent(in)    :: loc
 -- The following arguments require argument keyword syntax (e.g. rc=rc). --
       integer, optional                     :: rc 
 \end{verbatim}
{\sf STATUS:}
   \begin{itemize}
   \item\apiStatusCompatibleVersion{5.2.0r}
   \end{itemize}
  
{\sf DESCRIPTION:\\ }


     Sets a particular dimension of a custom {\tt staggerloc} to a position in a cell 
      by using the variable {\tt loc}. The variable {\tt loc} should only be 0,1. 
      If {\tt loc} is 0 it means the position 
      should be in the center in that dimension. If {\tt loc} is +1 then
      for the dimension, the position should be on the positive side of the cell. 
      Please see Section~\ref{sec:usage:staggerloc:adv}
      for diagrams and further discussion of custom stagger locations. 
  
       The arguments are:
       \begin{description}
       \item[staggerloc]
            Stagger location to be initialized
       \item[dim]
            Dimension to be changed (1-7).
       \item[loc]
            Position data should be either 0,1.
       \item[{[rc]}]
            Return code; equals {\tt ESMF\_SUCCESS} if there are no errors.
     \end{description}
   
%/////////////////////////////////////////////////////////////
 
\mbox{}\hrulefill\ 
 
\subsubsection [ESMF\_StaggerLocString] {ESMF\_StaggerLocString - Return a StaggerLoc as a string}


  
\bigskip{\sf INTERFACE:}
\begin{verbatim}       subroutine ESMF_StaggerLocString(staggerloc, string, &
            rc)\end{verbatim}{\em ARGUMENTS:}
\begin{verbatim}       type(ESMF_StaggerLoc), intent(in)  :: staggerloc
       character (len = *),   intent(out) :: string
 -- The following arguments require argument keyword syntax (e.g. rc=rc). --
       integer, optional,     intent(out) :: rc\end{verbatim}
{\sf STATUS:}
   \begin{itemize}
   \item\apiStatusCompatibleVersion{5.2.0r}
   \end{itemize}
  
{\sf DESCRIPTION:\\ }


       Return an {\tt ESMF\_StaggerLoc} as a printable string.
  
       The arguments are:
       \begin{description}
       \item [staggerloc]
             The {\tt ESMF\_StaggerLoc} to be turned into a string.
       \item [string]
            Return string.
       \item [{[rc]}]
             Return code; equals {\tt ESMF\_SUCCESS} if there are no errors.
       \end{description}
  
   
%/////////////////////////////////////////////////////////////
 
\mbox{}\hrulefill\ 
 
\subsubsection [ESMF\_StaggerLocPrint] {ESMF\_StaggerLocPrint - Print StaggerLoc information}


 
\bigskip{\sf INTERFACE:}
\begin{verbatim}       subroutine ESMF_StaggerLocPrint(staggerloc, rc)\end{verbatim}{\em ARGUMENTS:}
\begin{verbatim}       type (ESMF_StaggerLoc), intent(in)  :: staggerloc
 -- The following arguments require argument keyword syntax (e.g. rc=rc). --
       integer, optional,      intent(out) :: rc 
 \end{verbatim}
{\sf STATUS:}
   \begin{itemize}
   \item\apiStatusCompatibleVersion{5.2.0r}
   \end{itemize}
  
{\sf DESCRIPTION:\\ }


       Print the internal data members of an {\tt ESMF\_StaggerLoc} object. \\
  
       The arguments are:
       \begin{description}
       \item[staggerloc]
            ESMF\_StaggerLoc object as the method input
       \item[{[rc]}]
            Return code; equals {\tt ESMF\_SUCCESS} if there are no errors.
     \end{description}
  
%...............................................................
\setlength{\parskip}{\oldparskip}
\setlength{\parindent}{\oldparindent}
\setlength{\baselineskip}{\oldbaselineskip}
