%                **** IMPORTANT NOTICE *****
% This LaTeX file has been automatically produced by ProTeX v. 1.1
% Any changes made to this file will likely be lost next time
% this file is regenerated from its source. Send questions 
% to Arlindo da Silva, dasilva@gsfc.nasa.gov
 
\setlength{\oldparskip}{\parskip}
\setlength{\parskip}{1.5ex}
\setlength{\oldparindent}{\parindent}
\setlength{\parindent}{0pt}
\setlength{\oldbaselineskip}{\baselineskip}
\setlength{\baselineskip}{11pt}
 
%--------------------- SHORT-HAND MACROS ----------------------
\def\bv{\begin{verbatim}}
\def\ev{\end{verbatim}}
\def\be{\begin{equation}}
\def\ee{\end{equation}}
\def\bea{\begin{eqnarray}}
\def\eea{\end{eqnarray}}
\def\bi{\begin{itemize}}
\def\ei{\end{itemize}}
\def\bn{\begin{enumerate}}
\def\en{\end{enumerate}}
\def\bd{\begin{description}}
\def\ed{\end{description}}
\def\({\left (}
\def\){\right )}
\def\[{\left [}
\def\]{\right ]}
\def\<{\left  \langle}
\def\>{\right \rangle}
\def\cI{{\cal I}}
\def\diag{\mathop{\rm diag}}
\def\tr{\mathop{\rm tr}}
%-------------------------------------------------------------

\markboth{Left}{Source File: ESMCI\_IO\_XML.h,  Date: Tue May  5 20:59:47 MDT 2020
}

 
%/////////////////////////////////////////////////////////////
\subsection{C++:  Class Interface ESMCI::IO\_XML - Handles low-level XML IO for ESMF internals and (Source File: ESMCI\_IO\_XML.h)}


    ESMF user API.  Translates between ESMF internals/API and specific XML
    protocols/API such as SAX2 and DOM.  Xerces is used for SAX2 currently,
    (via ESMCI::SAX2[Write,Read]Handler), although Xerces/DOM could 
    be added via a separate class such as ESMCI::DOM.
  
{\sf DESCRIPTION:\\ }


    TODO
  -------------------------------------------------------------------------
  
\bigskip{\em USES:}
\begin{verbatim} #include <fstream>
 #include <string>
 #include <stdarg.h>
 #include "ESMCI_Base.h"           // inherited Base class
 #include "ESMCI_SAX2WriteHandler.h"
   #include "ESMCI_SAX2ReadHandler.h"  // TODO: ?
 
 namespace ESMCI{
 \end{verbatim}{\sf PUBLIC TYPES:}
\begin{verbatim}  class IO_XML;
  class Attribute;
 \end{verbatim}{\sf PRIVATE TYPES:}
\begin{verbatim} 
  // class definition type
  class IO_XML : public ESMC_Base { // inherit from ESMC_Base class
   private:   // corresponds to F90 module 'type ESMF_IO_XML' members
     Attribute *attr;    // root node of associated object's attributes
     std::string  fileName;
     std::string  schemaFileName;
 #ifdef ESMF_XERCES
     SAX2WriteHandler* writeHandler;   // to file; a future use could be to 
                                       // write to a network protocol rather
                                       // than a file.
     // SAX2ReadHandler* readHandler;  // TODO:  multiple reads per
     // SAX2XMLReader* parser;         //        IO_XML object lifetime ?
 #else
     std::ofstream writeFile;
 #endif
 \end{verbatim}{\sf PUBLIC MEMBER FUNCTIONS:}
\begin{verbatim} 
   public:
     // accessor methods
 
     // Read/Write (via SAX2 API)
     int read(const std::string& fileName,
              const std::string& schemaFileName);
 
     // maps to SAX2 startElement() & characters(), but not endElement();
     //   use to open a nested tag section
     int writeStartElement(const std::string& name,
                           const std::string& value,
                           const int     indentLevel,
                           const int     nPairs, ...); // nPairs of
                  // (char *attrName, char *attrValue)
 
     // maps to SAX2 startElement, characters() & endElement();
     //   use to write an entire tag, with xml attrs, and with no nested tags
     int writeElement(const std::string& name,
                      const std::string& value,
                      const int     indentLevel,
                      const int     nPairs, ...); // nPairs of
                  // (char *attrName, char *attrValue)
 
     // maps to SAX2 endElement(); use to close a nested tag section
     int writeEndElement(const std::string& name,
                         const int     indentLevel);
 
     // write an XML comment
     int writeComment(const std::string& comment, const int indentLevel=0);
 
     int write(const std::string& fileName,
               const char* outChars, int flag);
 
     // internal validation
     int validate(const char *options=0) const;
 
     // for testing/debugging
     int print(const char *options=0) const;
 
     // native C++ constructors/destructors
     IO_XML(void);
     IO_XML(Attribute*);
     // IO_XML(const IO_XML &io_xml);  TODO
     ~IO_XML(){destruct();}
 \end{verbatim}{\sf PRIVATE MEMBER FUNCTIONS:}
\begin{verbatim}   private:
  // < declare private interface methods here >
 
     // used internally by public methods writeStartElement() & writeElement()
     //   to share the common logic of writing the bulk of the tag
     //   (the difference is in the handling of the end-of-line/end-of-tag)
     int writeElementCore(const std::string& name,
                          const std::string& value,
                          const int     indentLevel,
                          const int     nPairs,
                          va_list       args); // nPairs of
                      // (char *attrName, char *attrValue)
     void destruct();
 
     // replace special characters to XML entities to prevent malformed XML
     int replaceXMLEntities(std::string& str);
 
     // friend function to allocate and initialize IO_XML object from heap
     friend IO_XML *ESMCI_IO_XMLCreate(const std::string& name,
                                       const std::string& fileName,
                                       Attribute*, int*);
 
     // friend function to copy an io_xml  TODO ?
     //friend IO_XML *ESMCI_IO_XML(IO_XML*, int*);
 
     // friend function to de-allocate IO_XML
     friend int ESMCI_IO_XMLDestroy(IO_XML**);
 \end{verbatim}

%...............................................................
\setlength{\parskip}{\oldparskip}
\setlength{\parindent}{\oldparindent}
\setlength{\baselineskip}{\oldbaselineskip}
