%                **** IMPORTANT NOTICE *****
% This LaTeX file has been automatically produced by ProTeX v. 1.1
% Any changes made to this file will likely be lost next time
% this file is regenerated from its source. Send questions 
% to Arlindo da Silva, dasilva@gsfc.nasa.gov
 
\setlength{\oldparskip}{\parskip}
\setlength{\parskip}{1.5ex}
\setlength{\oldparindent}{\parindent}
\setlength{\parindent}{0pt}
\setlength{\oldbaselineskip}{\baselineskip}
\setlength{\baselineskip}{11pt}
 
%--------------------- SHORT-HAND MACROS ----------------------
\def\bv{\begin{verbatim}}
\def\ev{\end{verbatim}}
\def\be{\begin{equation}}
\def\ee{\end{equation}}
\def\bea{\begin{eqnarray}}
\def\eea{\end{eqnarray}}
\def\bi{\begin{itemize}}
\def\ei{\end{itemize}}
\def\bn{\begin{enumerate}}
\def\en{\end{enumerate}}
\def\bd{\begin{description}}
\def\ed{\end{description}}
\def\({\left (}
\def\){\right )}
\def\[{\left [}
\def\]{\right ]}
\def\<{\left  \langle}
\def\>{\right \rangle}
\def\cI{{\cal I}}
\def\diag{\mathop{\rm diag}}
\def\tr{\mathop{\rm tr}}
%-------------------------------------------------------------

\markboth{Left}{Source File: ESMF\_DELayout.F90,  Date: Tue May  5 20:59:36 MDT 2020
}

 
%/////////////////////////////////////////////////////////////
\subsubsection [ESMF\_DELayoutAssignment(=)] {ESMF\_DELayoutAssignment(=) - DELayout assignment}


  
\bigskip{\sf INTERFACE:}
\begin{verbatim}     interface assignment(=)
     delayout1 = delayout2\end{verbatim}{\em ARGUMENTS:}
\begin{verbatim}     type(ESMF_DELayout) :: delayout1
     type(ESMF_DELayout) :: delayout2\end{verbatim}
{\sf STATUS:}
   \begin{itemize}
   \item\apiStatusCompatibleVersion{5.2.0r}
   \end{itemize}
  
{\sf DESCRIPTION:\\ }


     Assign delayout1 as an alias to the same ESMF DELayout object in memory
     as delayout2. If delayout2 is invalid, then delayout1 will be equally
     invalid after the assignment.
  
     The arguments are:
     \begin{description}
     \item[delayout1]
       The {\tt ESMF\_DELayout} object on the left hand side of the assignment.
     \item[delayout2]
       The {\tt ESMF\_DELayout} object on the right hand side of the assignment.
     \end{description}
   
%/////////////////////////////////////////////////////////////
 
\mbox{}\hrulefill\ 
 
\subsubsection [ESMF\_DELayoutOperator(==)] {ESMF\_DELayoutOperator(==) - DELayout equality operator}


  
\bigskip{\sf INTERFACE:}
\begin{verbatim}   interface operator(==)
     if (delayout1 == delayout2) then ... endif
               OR
     result = (delayout1 == delayout2)\end{verbatim}{\em RETURN VALUE:}
\begin{verbatim}     logical :: result\end{verbatim}{\em ARGUMENTS:}
\begin{verbatim}     type(ESMF_DELayout), intent(in) :: delayout1
     type(ESMF_DELayout), intent(in) :: delayout2\end{verbatim}
{\sf STATUS:}
   \begin{itemize}
   \item\apiStatusCompatibleVersion{5.2.0r}
   \end{itemize}
  
{\sf DESCRIPTION:\\ }


     Test whether delayout1 and delayout2 are valid aliases to the same ESMF
     DELayout object in memory. For a more general comparison of two
     ESMF DELayouts, going beyond the simple alias test, the 
     ESMF\_DELayoutMatch() function (not yet implemented) must
     be used.
  
     The arguments are:
     \begin{description}
     \item[delayout1]
       The {\tt ESMF\_DELayout} object on the left hand side of the equality
       operation.
     \item[delayout2]
       The {\tt ESMF\_DELayout} object on the right hand side of the equality
       operation.
     \end{description}
   
%/////////////////////////////////////////////////////////////
 
\mbox{}\hrulefill\ 
 
\subsubsection [ESMF\_DELayoutOperator(/=)] {ESMF\_DELayoutOperator(/=) - DELayout not equal operator}


  
\bigskip{\sf INTERFACE:}
\begin{verbatim}   interface operator(/=)
     if (delayout1 /= delayout2) then ... endif
               OR
     result = (delayout1 /= delayout2)\end{verbatim}{\em RETURN VALUE:}
\begin{verbatim}     logical :: result\end{verbatim}{\em ARGUMENTS:}
\begin{verbatim}     type(ESMF_DELayout), intent(in) :: delayout1
     type(ESMF_DELayout), intent(in) :: delayout2\end{verbatim}
{\sf STATUS:}
   \begin{itemize}
   \item\apiStatusCompatibleVersion{5.2.0r}
   \end{itemize}
  
{\sf DESCRIPTION:\\ }


     Test whether delayout1 and delayout2 are {\it not} valid aliases to the
     same ESMF DELayout object in memory. For a more general comparison of two
     ESMF DELayouts, going beyond the simple alias test, the 
     ESMF\_DELayoutMatch() function (not yet implemented) must
     be used.
  
     The arguments are:
     \begin{description}
     \item[delayout1]
       The {\tt ESMF\_DELayout} object on the left hand side of the non-equality
       operation.
     \item[delayout2]
       The {\tt ESMF\_DELayout} object on the right hand side of the non-equality
       operation.
     \end{description}
   
%/////////////////////////////////////////////////////////////
 
\mbox{}\hrulefill\ 
 
\subsubsection [ESMF\_DELayoutCreate] {ESMF\_DELayoutCreate - Create DELayout object}


 
\bigskip{\sf INTERFACE:}
\begin{verbatim}   ! Private name; call using ESMF_DELayoutCreate()
   recursive function ESMF_DELayoutCreateDefault(deCount, &
     deGrouping, pinflag, petList, vm, rc)
           \end{verbatim}{\em RETURN VALUE:}
\begin{verbatim}     type(ESMF_DELayout) :: ESMF_DELayoutCreateDefault\end{verbatim}{\em ARGUMENTS:}
\begin{verbatim} -- The following arguments require argument keyword syntax (e.g. rc=rc). --
     integer,                      intent(in),  optional :: deCount
     integer, target,              intent(in),  optional :: deGrouping(:)
     type(ESMF_Pin_Flag),          intent(in),  optional :: pinflag
     integer, target,              intent(in),  optional :: petList(:)
     type(ESMF_VM),                intent(in),  optional :: vm
     integer,                      intent(out), optional :: rc\end{verbatim}
{\sf STATUS:}
   \begin{itemize}
   \item\apiStatusCompatibleVersion{5.2.0r}
   \end{itemize}
  
{\sf DESCRIPTION:\\ }


       Create an {\tt ESMF\_DELayout} object on the basis of optionally provided
       restrictions. By default a DELayout with deCount equal to petCount will
       be created, each DE mapped to a single PET. However, the number of DEs
       as well grouping of DEs and PETs can be specified via the optional
       arguments.
  
       The arguments are:
       \begin{description}
       \item[{[deCount]}]
            Number of DEs to be provided by the created DELayout. By default
            the number of DEs equals the number of PETs in the associated VM
            context. Specifying a {\tt deCount} smaller than the number
            of PETs will result in unassociated PETs.
            This may be used to share VM resources between DELayouts within the
            same ESMF component. Specifying a {\tt deCount} greater than the 
            number of PETs will result in multiple DE to PET mapping.
       \item[{[deGrouping]}]
            This optional argument must be of size deCount. Its content assigns
            a DE group index to each DE of the DELayout. A group index of -1 
            indicates that the associated DE isn't member of any particular 
            group. The significance of DE groups is that all the DEs belonging
            to a certain group will be mapped against the {\em same} PET. This
            does not, however, mean that DEs belonging to different DE groups 
            must be mapped to different PETs.
       \item[{[pinflag]}]
            This flag specifies which type of resource DEs are pinned to. 
            The default is to pin DEs to PETs. Alternatively it is
            also possible to pin DEs to VASs. See section 
            \ref{const:pin_flag} for a list of valid pinning options.
       \item[{[petList]}]
            List specifying PETs to be used by this DELayout. This can be used
            to control the PET overlap between DELayouts within the same
            ESMF component. It is erroneous to specify PETs that are not within 
            the provided VM context. The default is to include all the PETs of
            the VM.
       \item[{[vm]}]
            If present, the DELayout object is created on the specified 
            {\tt ESMF\_VM} object. The default is to create on the VM of the 
            current context.
       \item[{[rc]}]
            Return code; equals {\tt ESMF\_SUCCESS} if there are no errors.
       \end{description}
   
%/////////////////////////////////////////////////////////////
 
\mbox{}\hrulefill\ 
 
\subsubsection [ESMF\_DELayoutCreate] {ESMF\_DELayoutCreate - Create DELayout from petMap}


 
\bigskip{\sf INTERFACE:}
\begin{verbatim}   ! Private name; call using ESMF_DELayoutCreate()
   recursive function ESMF_DELayoutCreateFromPetMap(petMap, &
     pinflag, vm, rc)
           \end{verbatim}{\em RETURN VALUE:}
\begin{verbatim}     type(ESMF_DELayout) :: ESMF_DELayoutCreateFromPetMap\end{verbatim}{\em ARGUMENTS:}
\begin{verbatim}     integer,                      intent(in)            :: petMap(:)
 -- The following arguments require argument keyword syntax (e.g. rc=rc). --
     type(ESMF_Pin_Flag),          intent(in),  optional :: pinflag
     type(ESMF_VM),                intent(in),  optional :: vm
     integer,                      intent(out), optional :: rc\end{verbatim}
{\sf STATUS:}
   \begin{itemize}
   \item\apiStatusCompatibleVersion{5.2.0r}
   \end{itemize}
  
{\sf DESCRIPTION:\\ }


       Create an {\tt ESMF\_DELayout} with exactly specified DE to PET mapping.
  
       This ESMF method must be called in unison by all PETs of the VM. Calling
       this method from a PET not part of the VM or not calling it from a PET
       that is part of the VM will result in undefined behavior. ESMF does not
       guard against violation of the unison requirement. The call is not
       collective, there is no communication between PETs.
  
       The arguments are:
       \begin{description}
       \item[petMap]
            List specifying the DE-to-PET mapping. The list elements correspond 
            to DE 0, 1, 2, ... and map against the specified PET of the VM
            context. The size of the {\tt petMap} 
            argument determines the number of DEs in the created DELayout. It is
            erroneous to specify a PET identifier that lies outside the VM 
            context.
       \item[{[pinflag]}]
            This flag specifies which type of resource DEs are pinned to. 
            The default is to pin DEs to PETs. Alternatively it is
            also possible to pin DEs to VASs. See section 
            \ref{const:pin_flag} for a list of valid pinning options.
       \item[{[vm]}]
            If present, the DELayout object is created on the specified 
            {\tt ESMF\_VM} object. The default is to create on the VM of the 
            current context.
       \item[{[rc]}]
            Return code; equals {\tt ESMF\_SUCCESS} if there are no errors.
       \end{description}
   
%/////////////////////////////////////////////////////////////
 
\mbox{}\hrulefill\ 
 
\subsubsection [ESMF\_DELayoutDestroy] {ESMF\_DELayoutDestroy - Release resources associated with DELayout object}


 
\bigskip{\sf INTERFACE:}
\begin{verbatim}   recursive subroutine ESMF_DELayoutDestroy(delayout, noGarbage, rc)\end{verbatim}{\em ARGUMENTS:}
\begin{verbatim}     type(ESMF_DELayout),  intent(inout)          :: delayout
 -- The following arguments require argument keyword syntax (e.g. rc=rc). --
     logical,              intent(in),   optional :: noGarbage
     integer,              intent(out),  optional :: rc  \end{verbatim}
{\sf STATUS:}
   \begin{itemize}
   \item\apiStatusCompatibleVersion{5.2.0r}
   \item\apiStatusModifiedSinceVersion{5.2.0r}
   \begin{description}
   \item[7.0.0] Added argument {\tt noGarbage}.
     The argument provides a mechanism to override the default garbage collection
     mechanism when destroying an ESMF object.
   \end{description}
   \end{itemize}
  
{\sf DESCRIPTION:\\ }


     Destroy an {\tt ESMF\_DELayout} object, releasing the resources associated
     with the object.
  
     By default a small remnant of the object is kept in memory in order to 
     prevent problems with dangling aliases. The default garbage collection
     mechanism can be overridden with the {\tt noGarbage} argument.
  
   The arguments are:
   \begin{description}
   \item[delayout] 
        {\tt ESMF\_DELayout} object to be destroyed.
   \item[{[noGarbage]}]
        If set to {\tt .TRUE.} the object will be fully destroyed and removed
        from the ESMF garbage collection system. Note however that under this 
        condition ESMF cannot protect against accessing the destroyed object 
        through dangling aliases -- a situation which may lead to hard to debug 
        application crashes.
   
        It is generally recommended to leave the {\tt noGarbage} argument
        set to {\tt .FALSE.} (the default), and to take advantage of the ESMF 
        garbage collection system which will prevent problems with dangling
        aliases or incorrect sequences of destroy calls. However this level of
        support requires that a small remnant of the object is kept in memory
        past the destroy call. This can lead to an unexpected increase in memory
        consumption over the course of execution in applications that use 
        temporary ESMF objects. For situations where the repeated creation and 
        destruction of temporary objects leads to memory issues, it is 
        recommended to call with {\tt noGarbage} set to {\tt .TRUE.}, fully 
        removing the entire temporary object from memory.
   \item[{[rc]}] 
        Return code; equals {\tt ESMF\_SUCCESS} if there are no errors.
   \end{description}
   
%/////////////////////////////////////////////////////////////
 
\mbox{}\hrulefill\ 
 
\subsubsection [ESMF\_DELayoutGet] {ESMF\_DELayoutGet - Get object-wide DELayout information}


 
\bigskip{\sf INTERFACE:}
\begin{verbatim}   recursive subroutine ESMF_DELayoutGet(delayout, vm, deCount,&
     petMap, vasMap, oneToOneFlag, pinflag, localDeCount, localDeToDeMap, &
     localDeList, &      ! DEPRECATED ARGUMENT
     vasLocalDeCount, vasLocalDeToDeMap, &
     vasLocalDeList, &   ! DEPRECATED ARGUMENT
     rc)\end{verbatim}{\em ARGUMENTS:}
\begin{verbatim}     type(ESMF_DELayout),      intent(in)            :: delayout
 -- The following arguments require argument keyword syntax (e.g. rc=rc). --
     type(ESMF_VM),            intent(out), optional :: vm
     integer,                  intent(out), optional :: deCount
     integer, target,          intent(out), optional :: petMap(:)
     integer, target,          intent(out), optional :: vasMap(:)
     logical,                  intent(out), optional :: oneToOneFlag
     type(ESMF_Pin_Flag),      intent(out), optional :: pinflag
     integer,                  intent(out), optional :: localDeCount
     integer, target,          intent(out), optional :: localDeToDeMap(:)
     integer, target, intent(out), optional :: localDeList(:)  !DEPRECATED ARG
     integer,                  intent(out), optional :: vasLocalDeCount
     integer, target,          intent(out), optional :: vasLocalDeToDeMap(:)
     integer, target, intent(out), optional :: vasLocalDeList(:) !DEPRECATED ARG
     integer,                  intent(out), optional :: rc  \end{verbatim}
{\sf STATUS:}
   \begin{itemize}
   \item\apiStatusCompatibleVersion{5.2.0r}
   \item\apiStatusModifiedSinceVersion{5.2.0r}
   \begin{description}
   \item[5.2.0rp1] Added arguments {\tt localDeToDeMap} and {\tt vasLocalDeToDeMap}.
                   Started to deprecate arguments {\tt localDeList} and 
                   {\tt vasLocalDeList}. 
                   The new argument names correctly use the {\tt Map} suffix and
                   better describe the returned information.
                   This was pointed out by user request.
   \end{description}
   \end{itemize}
  
{\sf DESCRIPTION:\\ }


     Access to DELayout information.
  
     The arguments are:
     \begin{description}
     \item[delayout] 
       Queried {\tt ESMF\_DELayout} object.
     \item[{[vm]}]
       The {\tt ESMF\_VM} object on which {\tt delayout} is defined.
     \item[{[deCount]}]
       The total number of DEs in the DELayout.
     \item[{[petMap]}]
       List of PETs against which the DEs are mapped. The {\tt petMap} 
       argument must at least be of size {\tt deCount}.
     \item[{[vasMap]}]
       List of VASs against which the DEs are mapped. The {\tt vasMap}
       argument must at least be of size {\tt deCount}.
     \item[{[oneToOneFlag]}]
       A value of {\tt .TRUE.} indicates that {\tt delayout} maps each DE to a
       single PET, and each PET maps to a single DE. All other layouts return
       a value of {\tt .FALSE.}.
     \item[{[pinflag]}]
       The type of DE pinning. See section \ref{const:pin_flag} for a list
       of valid pinning options.
     \item[{[localDeCount]}]
       The number of DEs in the DELayout associated with the local PET.
     \item[{[localDeToDeMap]}]
       Mapping between localDe indices and the (global) DEs associated with
       the local PET. The localDe index variables are discussed in sections
       \ref{DELayout_general_mapping} and \ref{Array_native_language_localde}.
       The provided actual argument must be of size {\tt localDeCount}.
     \item[{[localDeList]}]
       \apiDeprecatedArgWithReplacement{localDeToDeMap}
     \item[{[vasLocalDeCount]}]
       The number of DEs in the DELayout associated with the local VAS.
     \item[{[vasLocalDeToDeMap]}]
       Mapping between localDe indices and the (global) DEs associated with
       the local VAS. The localDe index variables are discussed in sections
       \ref{DELayout_general_mapping} and \ref{Array_native_language_localde}.
       The provided actual argument must be of size {\tt localDeCount}.
     \item[{[vasLocalDeList]}]
       \apiDeprecatedArgWithReplacement{vasLocalDeToDeMap}
     \item[{[rc]}] 
       Return code; equals {\tt ESMF\_SUCCESS} if there are no errors.
     \end{description}
   
%/////////////////////////////////////////////////////////////
 
\mbox{}\hrulefill\ 
 
\subsubsection [ESMF\_DELayoutIsCreated] {ESMF\_DELayoutIsCreated - Check whether a DELayout object has been created}


 
\bigskip{\sf INTERFACE:}
\begin{verbatim}   function ESMF_DELayoutIsCreated(delayout, rc)\end{verbatim}{\em RETURN VALUE:}
\begin{verbatim}     logical :: ESMF_DELayoutIsCreated\end{verbatim}{\em ARGUMENTS:}
\begin{verbatim}     type(ESMF_DELayout), intent(in)            :: delayout
 -- The following arguments require argument keyword syntax (e.g. rc=rc). --
     integer,             intent(out), optional :: rc
 \end{verbatim}
{\sf DESCRIPTION:\\ }


     Return {\tt .true.} if the {\tt delayout} has been created. Otherwise return 
     {\tt .false.}. If an error occurs, i.e. {\tt rc /= ESMF\_SUCCESS} is 
     returned, the return value of the function will also be {\tt .false.}.
  
   The arguments are:
     \begin{description}
     \item[delayout]
       {\tt ESMF\_DELayout} queried.
     \item[{[rc]}]
       Return code; equals {\tt ESMF\_SUCCESS} if there are no errors.
     \end{description}
   
%/////////////////////////////////////////////////////////////
 
\mbox{}\hrulefill\ 
 
\subsubsection [ESMF\_DELayoutPrint] {ESMF\_DELayoutPrint - Print DELayout information}


 
\bigskip{\sf INTERFACE:}
\begin{verbatim}   subroutine ESMF_DELayoutPrint(delayout, rc)\end{verbatim}{\em ARGUMENTS:}
\begin{verbatim}     type(ESMF_DELayout),  intent(in)            :: delayout
 -- The following arguments require argument keyword syntax (e.g. rc=rc). --
     integer,              intent(out), optional :: rc  \end{verbatim}
{\sf STATUS:}
   \begin{itemize}
   \item\apiStatusCompatibleVersion{5.2.0r}
   \end{itemize}
  
{\sf DESCRIPTION:\\ }


       Prints internal information about the specified {\tt ESMF\_DELayout} 
       object to {\tt stdout}. \\
  
       The arguments are:
       \begin{description}
       \item[delayout] 
            Specified {\tt ESMF\_DELayout} object.
       \item[{[rc]}] 
            Return code; equals {\tt ESMF\_SUCCESS} if there are no errors.
       \end{description}
   
%/////////////////////////////////////////////////////////////
 
\mbox{}\hrulefill\ 
 
\subsubsection [ESMF\_DELayoutServiceComplete] {ESMF\_DELayoutServiceComplete - Close service window}


 
\bigskip{\sf INTERFACE:}
\begin{verbatim}   recursive subroutine ESMF_DELayoutServiceComplete(delayout, de, rc)\end{verbatim}{\em ARGUMENTS:}
\begin{verbatim}     type(ESMF_DELayout),  intent(in)            :: delayout
     integer,              intent(in)            :: de
 -- The following arguments require argument keyword syntax (e.g. rc=rc). --
     integer,              intent(out), optional :: rc  \end{verbatim}
{\sf STATUS:}
   \begin{itemize}
   \item\apiStatusCompatibleVersion{5.2.0r}
   \end{itemize}
  
{\sf DESCRIPTION:\\ }


     The PET who's service offer was accepted for {\tt de} must use 
     {\tt ESMF\_DELayoutServiceComplete} to close the service window.
  
       The arguments are:
       \begin{description}
       \item[delayout] 
            Specified {\tt ESMF\_DELayout} object.
       \item[de]
            DE for which to close service window.
       \item[{[rc]}] 
            Return code; equals {\tt ESMF\_SUCCESS} if there are no errors.
       \end{description}
   
%/////////////////////////////////////////////////////////////
 
\mbox{}\hrulefill\ 
 
\subsubsection [ESMF\_DELayoutServiceOffer] {ESMF\_DELayoutServiceOffer - Offer service for a DE in DELayout}


 
\bigskip{\sf INTERFACE:}
\begin{verbatim}   recursive function ESMF_DELayoutServiceOffer(delayout, de, rc)
           \end{verbatim}{\em RETURN VALUE:}
\begin{verbatim}     type(ESMF_ServiceReply_Flag) :: ESMF_DELayoutServiceOffer\end{verbatim}{\em ARGUMENTS:}
\begin{verbatim}     type(ESMF_DELayout),  intent(in)            :: delayout
     integer,              intent(in)            :: de
 -- The following arguments require argument keyword syntax (e.g. rc=rc). --
     integer,              intent(out), optional :: rc\end{verbatim}
{\sf STATUS:}
   \begin{itemize}
   \item\apiStatusCompatibleVersion{5.2.0r}
   \end{itemize}
  
{\sf DESCRIPTION:\\ }


       \begin{sloppypar}
       Offer service for a DE in the {\tt ESMF\_DELayout} object. This call
       together with {\tt ESMF\_DELayoutServiceComplete()} provides the
       synchronization primitives between the PETs of an ESMF multi-threaded VM
       necessary for dynamic load balancing via a work queue approach.
  
       The calling PET will either receive {\tt ESMF\_SERVICEREPLY\_ACCEPT} if
       the service offer has been accepted by DELayout or 
       {\tt ESMF\_SERVICEREPLY\_DENY} if the service offer was denied. The 
       service offer paradigm is different from a simple mutex approach in that
       the DELayout keeps track of the number of service offers issued for each
       DE by each PET and accepts only one PET's offer for each offer increment.
       This requires that all PETs use {\tt ESMF\_DELayoutServiceOffer()} in 
       unison. See section \ref{const:servicereply_flag} for the potential return
       values.
       \end{sloppypar}
  
       The arguments are:
       \begin{description}
       \item[delayout] 
            Specified {\tt ESMF\_DELayout} object.
       \item[de]
            DE for which service is offered by the calling PET.
       \item[{[rc]}]
            Return code; equals {\tt ESMF\_SUCCESS} if there are no errors.
       \end{description}
   
%/////////////////////////////////////////////////////////////
 
\mbox{}\hrulefill\ 
 
\subsubsection [ESMF\_DELayoutValidate] {ESMF\_DELayoutValidate - Validate DELayout internals}


 
\bigskip{\sf INTERFACE:}
\begin{verbatim}   subroutine ESMF_DELayoutValidate(delayout, rc)\end{verbatim}{\em ARGUMENTS:}
\begin{verbatim}     type(ESMF_DELayout),  intent(in)            :: delayout
 -- The following arguments require argument keyword syntax (e.g. rc=rc). --
     integer,              intent(out), optional :: rc  \end{verbatim}
{\sf STATUS:}
   \begin{itemize}
   \item\apiStatusCompatibleVersion{5.2.0r}
   \end{itemize}
  
{\sf DESCRIPTION:\\ }


        Validates that the {\tt delayout} is internally consistent.
        The method returns an error code if problems are found.  
  
       The arguments are:
       \begin{description}
       \item[delayout] 
            Specified {\tt ESMF\_DELayout} object.
       \item[{[rc]}] 
            Return code; equals {\tt ESMF\_SUCCESS} if there are no errors.
       \end{description}
  
%...............................................................
\setlength{\parskip}{\oldparskip}
\setlength{\parindent}{\oldparindent}
\setlength{\baselineskip}{\oldbaselineskip}
