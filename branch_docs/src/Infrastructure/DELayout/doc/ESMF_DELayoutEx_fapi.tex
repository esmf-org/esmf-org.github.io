%                **** IMPORTANT NOTICE *****
% This LaTeX file has been automatically produced by ProTeX v. 1.1
% Any changes made to this file will likely be lost next time
% this file is regenerated from its source. Send questions 
% to Arlindo da Silva, dasilva@gsfc.nasa.gov
 
\setlength{\oldparskip}{\parskip}
\setlength{\parskip}{1.5ex}
\setlength{\oldparindent}{\parindent}
\setlength{\parindent}{0pt}
\setlength{\oldbaselineskip}{\baselineskip}
\setlength{\baselineskip}{11pt}
 
%--------------------- SHORT-HAND MACROS ----------------------
\def\bv{\begin{verbatim}}
\def\ev{\end{verbatim}}
\def\be{\begin{equation}}
\def\ee{\end{equation}}
\def\bea{\begin{eqnarray}}
\def\eea{\end{eqnarray}}
\def\bi{\begin{itemize}}
\def\ei{\end{itemize}}
\def\bn{\begin{enumerate}}
\def\en{\end{enumerate}}
\def\bd{\begin{description}}
\def\ed{\end{description}}
\def\({\left (}
\def\){\right )}
\def\[{\left [}
\def\]{\right ]}
\def\<{\left  \langle}
\def\>{\right \rangle}
\def\cI{{\cal I}}
\def\diag{\mathop{\rm diag}}
\def\tr{\mathop{\rm tr}}
%-------------------------------------------------------------

\markboth{Left}{Source File: ESMF\_DELayoutEx.F90,  Date: Tue May  5 20:59:36 MDT 2020
}

 
%/////////////////////////////////////////////////////////////

   \subsubsection{Default DELayout}
   
   Without specifying any of the optional parameters the created 
   {\tt ESMF\_DELayout}
   defaults into having as many DEs as there are PETs in the associated VM 
   object. Consequently the resulting DELayout describes a simple 1-to-1 DE to
   PET mapping. 
%/////////////////////////////////////////////////////////////

 \begin{verbatim}
  delayout = ESMF_DELayoutCreate(rc=rc)
 
\end{verbatim}
 
%/////////////////////////////////////////////////////////////

   The default DE to PET mapping is simply:
   \begin{verbatim}
   DE 0  -> PET 0
   DE 1  -> PET 1
   ...
   \end{verbatim}
  
   DELayout objects that are not used any longer should be destroyed. 
%/////////////////////////////////////////////////////////////

 \begin{verbatim}
  call ESMF_DELayoutDestroy(delayout, rc=rc)
 
\end{verbatim}
 
%/////////////////////////////////////////////////////////////

   The optional {\tt vm} argument can be provided to DELayoutCreate() to lower 
   the method's overhead by the amount it takes to determine the current VM. 
%/////////////////////////////////////////////////////////////

 \begin{verbatim}
  delayout = ESMF_DELayoutCreate(vm=vm, rc=rc)
 
\end{verbatim}
 
%/////////////////////////////////////////////////////////////

   By default all PETs of the associated VM will be considered. However, if the 
   optional argument {\tt petList} is present DEs will only be mapped against
   the PETs contained in the list. When the following example is executed on
   four PETs it creates a DELayout with four DEs by default that are mapped 
   to the provided PETs in their given order. It is erroneous to specify PETs 
   that are not part of the VM context on which the DELayout is defined.  
%/////////////////////////////////////////////////////////////

 \begin{verbatim}
  delayout = ESMF_DELayoutCreate(petList=(/(i,i=petCount-1,1,-1)/), rc=rc)
 
\end{verbatim}
 
%/////////////////////////////////////////////////////////////

   Once the end of the petList has been reached the DE to PET mapping 
   continues from the beginning of the list. For a 4 PET VM the above created
   DELayout will end up with the following DE to PET mapping:
  
   \begin{verbatim}
   DE 0  -> PET 3
   DE 1  -> PET 2
   DE 2  -> PET 1
   DE 2  -> PET 3
   \end{verbatim} 
%/////////////////////////////////////////////////////////////

   \subsubsection{DELayout with specified number of DEs}
   
   The {\tt deCount} argument can be used to specify the number of DEs. In this
   example a DELayout is created that contains four times as many DEs as there 
   are PETs in the VM. 
%/////////////////////////////////////////////////////////////

 \begin{verbatim}
  delayout = ESMF_DELayoutCreate(deCount=4*petCount, rc=rc)
 
\end{verbatim}
 
%/////////////////////////////////////////////////////////////

   Cyclic DE to PET mapping is the default. For 4 PETs this means:
   \begin{verbatim}
   DE 0, 4,  8, 12  -> PET 0
   DE 1, 5,  9, 13  -> PET 1
   DE 2, 6, 10, 14  -> PET 2
   DE 3, 7, 11, 15  -> PET 3
   \end{verbatim}
   The default DE to PET mapping can be overridden by providing the
   {\tt deGrouping} argument. This argument provides a positive integer group 
   number for each DE in the DELayout. All of the DEs of a group will be mapped 
   against the same PET. The actual group index is arbitrary (but must be 
   positive) and its value is of no consequence. 
%/////////////////////////////////////////////////////////////

 \begin{verbatim}
  delayout = ESMF_DELayoutCreate(deCount=4*petCount, &
    deGrouping=(/(i/4,i=0,4*petCount-1)/), rc=rc)
 
\end{verbatim}
 
%/////////////////////////////////////////////////////////////

   This will achieve blocked DE to PET mapping. For 4 PETs this means:
   \begin{verbatim}
   DE  0,  1,  2,  3  -> PET 0
   DE  4,  5,  6,  7  -> PET 1
   DE  8,  9, 10, 11  -> PET 2
   DE 12, 13, 14, 15  -> PET 3
   \end{verbatim} 
%/////////////////////////////////////////////////////////////

   \subsubsection{DELayout with computational and communication weights}
   
   The quality of the partitioning expressed by the DE to PET mapping depends
   on the amount and quality of information provided during DELayout creation.
   In the following example the {\tt compWeights} argument is used to specify
   relative computational weights for all DEs and communication weights for
   DE pairs are provided by the {\tt commWeights} argument. The example assumes
   four DEs. 
%/////////////////////////////////////////////////////////////

 \begin{verbatim}
  allocate(compWeights(4))
  allocate(commWeights(4, 4))
  ! setup compWeights and commWeights according to computational problem
  delayout = ESMF_DELayoutCreate(deCount=4, compWeights=compWeights, &
    commWeights=commWeights, rc=rc)
  deallocate(compWeights, commWeights)
 
\end{verbatim}
 
%/////////////////////////////////////////////////////////////

   The resulting DE to PET mapping depends on the specifics of the VM object and
   the provided compWeights and commWeights arrays. 
%/////////////////////////////////////////////////////////////

   \subsubsection{DELayout from petMap}
   
   Full control over the DE to PET mapping is provided via the {\tt petMap}
   argument. This example maps the DEs to PETs in reverse order. In the 4-PET
   case this will result in the following mapping:
   \begin{verbatim}
   DE 0 -> PET 3
   DE 1 -> PET 2
   DE 2 -> PET 1
   DE 3 -> PET 0
   \end{verbatim} 
%/////////////////////////////////////////////////////////////

 \begin{verbatim}
  delayout = ESMF_DELayoutCreate(petMap=(/(i,i=petCount-1,0,-1)/), rc=rc)
 
\end{verbatim}
 
%/////////////////////////////////////////////////////////////
 
%/////////////////////////////////////////////////////////////

   \subsubsection{DELayout from petMap with multiple DEs per PET}
   
   The {\tt petMap} argument gives full control over DE to PET mapping. The 
   following example run on 4 or more PETs maps DEs to PETs according to the 
   following table:
   \begin{verbatim}
   DE 0 -> PET 3
   DE 1 -> PET 3
   DE 2 -> PET 1
   DE 3 -> PET 0
   DE 4 -> PET 2
   DE 5 -> PET 1
   DE 6 -> PET 3
   DE 7 -> PET 1
   \end{verbatim} 
%/////////////////////////////////////////////////////////////

 \begin{verbatim}
  delayout = ESMF_DELayoutCreate(petMap=(/3, 3, 1, 0, 2, 1, 3, 1/), rc=rc)
 
\end{verbatim}
 
%/////////////////////////////////////////////////////////////

   \subsubsection{Working with a DELayout - simple 1-to-1 DE-to-PET mapping}
   
   The simplest case is a DELayout where there is exactly one DE for every PET.
   Of course this implies that the number of DEs equals the number of PETs. 
   This special 1-to-1 DE-to-PET mapping is very common and many applications
   assume it. The following example shows how a DELayout can be queried about
   its mapping.
  
   First a default DELayout is created where the number of DEs equals the number
   of PETs, and are associated 1-to-1. 
%/////////////////////////////////////////////////////////////

 \begin{verbatim}
  delayout = ESMF_DELayoutCreate(rc=rc)
 
\end{verbatim}
 
%/////////////////////////////////////////////////////////////

   Next the DELayout is queried for the {\tt oneToOneFlag}, and the user code
   makes a decision based on its value. 
%/////////////////////////////////////////////////////////////

 \begin{verbatim}
  call ESMF_DELayoutGet(delayout, oneToOneFlag=oneToOneFlag, rc=rc)
  if (rc /= ESMF_SUCCESS) call ESMF_Finalize(endflag=ESMF_END_ABORT)
  if (.not. oneToOneFlag) then
    ! handle the unexpected case of not dealing with a 1-to-1 mapping
  else
 
\end{verbatim}
 
%/////////////////////////////////////////////////////////////

   1-to-1 mapping is guaranteed in this branch and the following code can
   work under the simplifying assumption that every PET holds exactly one DE: 
%/////////////////////////////////////////////////////////////

 \begin{verbatim}
    allocate(localDeToDeMap(1))
    call ESMF_DELayoutGet(delayout, localDeToDeMap=localDeToDeMap, rc=rc)
    if (rc /= ESMF_SUCCESS) finalrc=rc
    myDe = localDeToDeMap(1)
    deallocate(localDeToDeMap)
    if (finalrc /= ESMF_SUCCESS) call ESMF_Finalize(endflag=ESMF_END_ABORT)
  endif
 
\end{verbatim}
 
%/////////////////////////////////////////////////////////////

   \subsubsection{Working with a DELayout - general DE-to-PET mapping}
   \label{DELayout_general_mapping}
   
   In general a DELayout may map any number (including zero) of DEs against
   a single PET. The exact situation can be detected by querying the DELayout
   for the {\tt oneToOneFlag}. If this flag comes back as {\tt .true.} then the 
   DELayout maps exactly one DE against each PET, but if it comes back as
   {\tt .false.} the DELayout describes a more general DE-to-PET layout. The 
   following example shows how code can be be written to work for a general
   DELayout.
  
   First a DELayout is created with two more DEs than there are PETs. The 
   DELayout will consequently map some DEs to the same PET. 
%/////////////////////////////////////////////////////////////

 \begin{verbatim}
  delayout = ESMF_DELayoutCreate(deCount=petCount+2, rc=rc)
 
\end{verbatim}
 
%/////////////////////////////////////////////////////////////

   The first piece of information needed on each PET is the {\tt localDeCount}.
   This number may be different on each PET and indicates how many DEs are 
   mapped against the local PET. 
%/////////////////////////////////////////////////////////////

 \begin{verbatim}
  call ESMF_DELayoutGet(delayout, localDeCount=localDeCount, rc=rc)
 
\end{verbatim}
 
%/////////////////////////////////////////////////////////////

   The DELayout can further be queried for a list of DEs that are held by
   the local PET. This information is provided by the {\tt localDeToDeMap}
   argument. In ESMF a {\tt localDe} is an index that enumerates the DEs that
   are associated with the local PET. In many cases the exact bounds of the
   {\tt localDe} index range, e.g. $[0...localDeCount-1]$, or $[1...localDeCount]$ 
   does not matter, since it only affects how user code indexes into variables
   the user allocated, and therefore set the specific bounds. However, there are 
   a few Array and Field level calls that take {\tt localDe} input arguments. In 
   all those cases where the {\tt localDe} index variable is passed into an ESMF
   call as an input argument, it {\em must} be defined with a range starting at
   zero, i.e. $[0...localDeCount-1]$.
  
   For consistency with Array and Field, the following code uses a 
   $[0...localDeCount-1]$ range for the {\tt localDe} index variable, 
   although it is not strictly necessary here: 
%/////////////////////////////////////////////////////////////

 \begin{verbatim}
  allocate(localDeToDeMap(0:localDeCount-1))
  call ESMF_DELayoutGet(delayout, localDeToDeMap=localDeToDeMap, rc=rc)
  if (rc /= ESMF_SUCCESS) finalrc=rc
  do localDe=0, localDeCount-1
    workDe = localDeToDeMap(localDe)
!    print *, "I am PET", localPET, " and I am working on DE ", workDe
  enddo
  deallocate(localDeToDeMap)
  if (finalrc /= ESMF_SUCCESS) call ESMF_Finalize(endflag=ESMF_END_ABORT)
 
\end{verbatim}
 
%/////////////////////////////////////////////////////////////

   \subsubsection{Work queue dynamic load balancing}
   
   The DELayout API includes two calls that can be used to easily implement
   work queue dynamic load balancing. The workload is broken up into DEs
   (more than there are PETs) and processed by the PETs. Load balancing is
   only possible for ESMF multi-threaded VMs and requires that DEs are pinned
   to VASs instead of the PETs (default). The following example will
   run for any VM and DELayout, however, load balancing will only occur under the
   mentioned conditions. 
%/////////////////////////////////////////////////////////////

 \begin{verbatim}
  delayout = ESMF_DELayoutCreate(deCount=petCount+2, &
    pinflag=ESMF_PIN_DE_TO_VAS, rc=rc)
 
\end{verbatim}
 
%/////////////////////////////////////////////////////////////

 \begin{verbatim}
  call ESMF_DELayoutGet(delayout, vasLocalDeCount=localDeCount, rc=rc)
  if (rc /= ESMF_SUCCESS) finalrc=rc
  allocate(localDeToDeMap(localDeCount))
  call ESMF_DELayoutGet(delayout, vasLocalDeToDeMap=localDeToDeMap, rc=rc)
  if (rc /= ESMF_SUCCESS) finalrc=rc
  do i=1, localDeCount
    workDe = localDeToDeMap(i)
    print *, "I am PET", localPET, &
             " and I am offering service for DE ", workDe
    reply = ESMF_DELayoutServiceOffer(delayout, de=workDe, rc=rc)
    if (rc /= ESMF_SUCCESS) finalrc=rc
    if (reply == ESMF_SERVICEREPLY_ACCEPT) then
      ! process work associated with workDe
      print *, "I am PET", localPET, ", service offer for DE ", workDe, &
        " was accepted."
      call ESMF_DELayoutServiceComplete(delayout, de=workDe, rc=rc)
      if (rc /= ESMF_SUCCESS) finalrc=rc
    endif
  enddo
  deallocate(localDeToDeMap)
  if (finalrc /= ESMF_SUCCESS) call ESMF_Finalize(endflag=ESMF_END_ABORT)
 
\end{verbatim}

%...............................................................
\setlength{\parskip}{\oldparskip}
\setlength{\parindent}{\oldparindent}
\setlength{\baselineskip}{\oldbaselineskip}
