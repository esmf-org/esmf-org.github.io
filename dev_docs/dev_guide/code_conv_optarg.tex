% CVS $Id$

\subsection{Code: Optional Argument Conventions for the C/C++ API}


\subsubsection{Overview}

Optional arguments in the public ESMC interface are implemented using the
variable-length argument list functionality provided by \texttt{<stdarg.h>}.
The \texttt{<stdarg.h>} header contains a set of macros which allows portable
functions that accept variable-length argument lists (also known as variadic
functions) to be written. Variadic functions are declared with an ellipsis in
place of the last parameter and must have at least one named (fixed) parameter.
A typical example of such a function is printf which is declared as

\begin{verbatim}
        int printf(char *fmt, ...).
\end{verbatim}

\noindent
The following type and macros are provided by {\tt<stdarg.h>} for processing
variable-length argument lists.
\begin{itemize}
\item \texttt{va\_list}
\newline
Type suitable for use in accessing the variable-length argument list of a function
with the stdarg macros;
\item \texttt{void va\_start(va\_list ap, last\_arg)}
\newline
Initializes the variable argument list pointer \textit{ap} to the beginning of the
variable argument list, before any calls to \texttt{va\_arg}. \textit{last\_arg} is
the last fixed argument being passed to the function (the argument before the
ellipsis).
\item \textit{type} \texttt{va\_arg(va\_list ap, }\textit{type}\texttt{)}
\newline
Returns the next argument in the variable-length argument list pointed to by
\texttt{ap}. Each invocation of \texttt{va\_arg} modifies \texttt{ap} such that
the values of successive arguments are returned in turn. The \textit{type}
parameter is the type the argument is expected to be. The parameter \textit{type}
is a type name specified so that the type of a pointer to an object that has the
specified type can be obtained simply by adding a \texttt{*} to \textit{type}.
Different types can be mixed, but it is up to the function to know what type of
argument is expected. Note that \texttt{ap} must be initialized with
\texttt{va\_start}. If there is no next argument, then the result is undefined.
\item \texttt{void va\_end(va\_list ap)}
\newline
Destroys the variable argument list pointer \texttt{ap}, rendering it invalid
unless \texttt{va\_start} is called again. Each invocation of \texttt{va\_start}
must be matched by a corresponding invocation of \texttt{va\_end} in the same
function. After the call \texttt{va\_end(ap)} the variable \texttt{ap} is
undefined. Multiple traversals of the list, each bracketed by \texttt{va\_start}
and \texttt{va\_end} are possible.
\end{itemize}
\noindent
Arguments corresponding to the variable-length argument list specified by
\texttt{",..."} in a function prototype always undergo the following argument
conversions: \texttt{bool} or \texttt{char} or \texttt{short} are converted to
\texttt{int}, and \texttt{float} is converted to \texttt{double}.
\newline

In order to correctly retrieve arguments from the variable argument list it is
necessary to know the type of each argument. In order to correctly terminate
parsing the variable argument list it is also necessary to know the number of
arguments in the list. Since the number of arguments and their types
are not known at compile time the best one can do is establish a convention
such that the caller will provide type and length information about the variable
argument list. The convention, however, is not a guarantee that the caller does
not mistakenly specify the types or number of arguments.
The \texttt{scanf} and \texttt{printf} functions do this by having
the last fixed argument be a character (format) string in which the caller
specifies the types (for example, \texttt{"\%f\%d\%s"} for each argument in the
variable argument list. Although, the format string specifies the number of
expected arguments there is no actual "termination-flag" inserted at the end of
the variable argument list. If the caller incorrectly specifies the in the format
string more arguments than are actually provided then the function will likely
access invalid sections of memory while parsing the variable argument list.
Unless this error is trapped by the system the function will unwittingly process
erroneous data.


\subsubsection{Approach}

The following conventions are established for the public ESMC optional argument
API. The "class" associated with this API is called \texttt{ESMC\_Arg} and its
header files are located in \texttt{Infrastructure/Util/include}. The public
optional argument API for a class is provided by the public header for that class
together with \texttt{ESMC\_Arg.h}. Macros for internal processing of optional
arguments are provided by the \texttt{ESMCI\_Arg.h} header.
Each ESMC class has a set of named optional arguments associated with its
methods. Associated with each optional argument of a class is a unique (within
the class) optional argument identifier (\textit{id}) of type
\texttt{ESMCI\_ArgID}. The optional argument list provided by
the caller must consist of a sequence of argument identifier and argument
(\textit{id}, \textit{arg}) pairs. The optional argument list must be terminated
by the global identifier \texttt{ESMCI\_ArgLastID}.
The \texttt{ESMC\_ArgLast} macro is provided by \texttt{ESMC\_Arg.h} for the
user to indicate the end of an optional argument list.
\newline

The global identifier \texttt{ESMCI\_ArgBaseID} is the starting identifier for
local (class) optional argument identifiers. \texttt{ESMCI\_ArgBaseID}
establishes a set of global optional argument identifiers, such that, the global
set does not intersect with any class identifier set.
The optional argument identifier list for a class will be declared in the public
header for that class. The naming convention for optional argument identifiers
is
\begin{verbatim}
                ESMCI_<ClassName>Arg<ArgName>ID
\end{verbatim}
where \texttt{ClassName} is the
name of the class and \texttt{ArgName} is the name of the optional argument with
the first letter capitalized. The optional argument identifier list is declared
as enumeration constants with the first identifier set equal to
\texttt{ESMCI\_ArgBaseID}. Here is an example for a class called "Xclass".
\begin{verbatim}
   // Optional argument identifier list for the ESMC_Xclass API.
   enum {
     ESMCI_XclassArgAoptID = ESMCI_ArgBaseID,  // ESMC_I1
     ESMCI_XclassArgBoptID,                    // ESMC_I4
     ESMCI_XclassArgCoptID,                    // ESMC_R4
     ESMCI_XclassArgDoptID,                    // ESMC_R8
     ESMCI_XclassArgEoptID,                    // ESMC_Fred
   };
\end{verbatim}
\noindent
It is helpful for the class developer to list the data type of
each optional argument (as a comment) with its associated \textit{id}.
\newline

The \texttt{ESMCI\_Arg(ID,AR)} internal macro (declared in \texttt{ESMC\_Arg.h})
is provided for casting an optional argument and its associated \textit{id} into
the required sequence for passing to functions.
\begin{verbatim}
   #define ESMCI_Arg(ID,ARG)  ((ESMCI_ArgID)ID),(ARG)
\end{verbatim}
\noindent
Each class will use this internal macro in its public header to declare
user macros for the class specific optional arguments. 
The naming convention for class specific optional argument expansion macros 
is
\begin{verbatim}
                ESMC_<ClassName>Arg<ArgName>(ARG)
\end{verbatim}
where \texttt{ClassName} is the
name of the class and \texttt{ArgName} is the name of the optional argument with
the first letter capitalized. The argument of the macro is the actual caller
provided optional argument.  Here is an example of how the macros are used for
the class "Xclass".
\begin{verbatim}
   // Argument expansion macros for the ESMC_Xclass API.
   #define ESMC_XclassArgAopt(ARG)  ESMCI_Arg(ESMCI_XclassArgAoptID,ARG)
   #define ESMC_XclassArgBopt(ARG)  ESMCI_Arg(ESMCI_XclassArgBoptID,ARG)
   #define ESMC_XclassArgCopt(ARG)  ESMCI_Arg(ESMCI_XclassArgCoptID,ARG)
   #define ESMC_XclassArgDopt(ARG)  ESMCI_Arg(ESMCI_XclassArgDoptID,ARG)
   #define ESMC_XclassArgEopt(ARG)  ESMCI_Arg(ESMCI_XclassArgEoptID,ARG)
\end{verbatim}
\noindent
Here is an example call, with properly specified optional arguments, to a class
function. The function has three named (fixed) arguments.
\begin{verbatim}
   rc = ESMC_XclassFunc(fixedArg1,fixedArg2,fixedArg3,
                        ESMC_XclassArgAopt(aOptArg),
                        ESMC_XclassArgCopt(cOptArg),
                        ESMC_XclassArgDopt(dOptArg),
                        ESMC_ArgLast)
\end{verbatim}


\subsubsection{Internal Macros for Processing the Optional Argument List}

Internal macros for processing the optional argument list are declared in
\texttt{ESMCI\_Arg.h}. These macros provide a consistent internal interface to
the macros defined in \texttt{stdarg.h}. What follows is a description of each
macro available to a class developer.

\begin{itemize}
\item \texttt{ESMCI\_ArgList}
\newline
Optional argument list data type (\texttt{va\_list}). The variable used for
accessing the optional argument list must be declared as this type.
\item \texttt{ESMCI\_ArgStart(AP,LAST)}
\newline
Internal macro to initialize optional argument list processing. \texttt{AP} is
the optional argument list pointer. \texttt{LAST} is the last fixed argument
(before the optional argument list).
\item \texttt{ESMCI\_ArgEnd(AP)}
\newline
Internal macro to finalize optional argument list processing. \texttt{AP} is
the optional argument list pointer.
\item \texttt{ESMCI\_ArgGetID(AP)}
\newline
Internal macro to return an optional argument identifier. \texttt{AP} is
the optional argument list pointer.
\end{itemize}

\noindent
The following type specific \texttt{ESMCI\_ArgGet} macros are provided for
returning optional argument values. These macros correctly account for the
forced type conversions described above. In each macro, \texttt{AP} is the
optional argument list pointer.
\newline\newline
\noindent
Internal macros for returning standard C (non-pointer) types.
\begin{verbatim}
   #define ESMCI_ArgGetChar(AP)           (char)va_arg(AP,int)
   #define ESMCI_ArgGetShort(AP)         (short)va_arg(AP,int)
   #define ESMCI_ArgGetInt(AP)             (int)va_arg(AP,int)
   #define ESMCI_ArgGetLong(AP)           (long)va_arg(AP,long)
   #define ESMCI_ArgGetLongLong(AP)  (long long)va_arg(AP,long long)
   #define ESMCI_ArgGetFloat(AP)         (float)va_arg(AP,double)
   #define ESMCI_ArgGetDouble(AP)       (double)va_arg(AP,double)
\end{verbatim}
\noindent
Internal macros for returning defined ESMC types.
\begin{verbatim}
   #define ESMCI_ArgGetI1(AP)          (ESMC_I1)va_arg(AP,int)
   #define ESMCI_ArgGetI2(AP)          (ESMC_I2)va_arg(AP,int)
   #define ESMCI_ArgGetI4(AP)          (ESMC_I4)va_arg(AP,int)
   #define ESMCI_ArgGetI8(AP)          (ESMC_I8)va_arg(AP,ESMC_I8)
   #define ESMCI_ArgGetR8(AP)          (ESMC_R8)va_arg(AP,double)
   #define ESMCI_ArgGetR4(AP)          (ESMC_R4)va_arg(AP,double)
\end{verbatim}
\noindent
Special internal macro for returning strings.
\begin{verbatim}
   #define ESMCI_ArgGetString(AP)               va_arg(AP,char*)
\end{verbatim}
\noindent
General internal macro for returning all non-converted types.
\begin{verbatim}
   #define ESMCI_ArgGet(AP,TYPE)                va_arg(AP,TYPE)
\end{verbatim}
\noindent
\texttt{TYPE} is the expected type of returned argument.


\subsubsection{Parsing the Optional Argument List}

Each class function with optional arguments will declare an optional argument
list pointer and optional argument list \textit{id}.
\begin{verbatim}
  ESMCI_ArgList argPtr;       // optional argument list pointer
  ESMCI_ArgID argID;          // optional argument list id
\end{verbatim}

\noindent
Parsing the optional argument list consists of three distinct phases: initializing
the argument list pointer, retrieving each argument in succession based on its
\textit{id}, and finalizing the argument list pointer. The block for retrieving
arguments must first retrieve an argument \textit{id}. If that \textit{id} is
not \texttt{ESMCI\_ArgLastID} and is in the set of valid ids for that function,
then the argument is retrieved using a type-appropriate \texttt{ESMCI\_ArgGet}
macro. The following example code should be considered as "template" code for
parsing an optional argument list.
\begin{verbatim}
  // parse the optional argument list:
  ESMCI_ArgStart(argPtr,ifix);
  while ( (argID = ESMCI_ArgGetID(argPtr)) != ESMCI_ArgLastID ) {
    switch ( argID ) {
      case ESMCI_XclassArgAoptID:
        aOpt = ESMCI_ArgGetI1(argPtr);
        break;
      case ESMCI_XclassArgBoptID:
        bOpt = ESMCI_ArgGetI4(argPtr);
        break;
      case ESMCI_XclassArgCoptID:
        cOpt = ESMCI_ArgGetR4(argPtr);
        break;
      case ESMCI_XclassArgDoptID:
        dOpt = ESMCI_ArgGetR8(argPtr);
        break;
      case ESMCI_XclassArgEoptID:
        eOpt = ESMCI_ArgGet(argPtr,ESMC_Fred);
        break;
      default:
        ESMC_LogDefault.MsgFoundError(ESMC_RC_OPTARG_BAD, "", &rc);
        return rc;
    } // end switch (argID)
  } // end while (argID)
  ESMCI_ArgEnd(argPtr);
\end{verbatim}

\noindent
Prior to actually parsing the optional argument list a function must check the
list for proper specification. This is done by using a code block similar to the
parsing block shown above. The difference is that in this case the
\texttt{ESMCI\_ArgGet} macros are only used to increment the argument list pointer.
\begin{verbatim}
  // check the optional argument list:
  ESMCI_ArgStart(argPtr,lastFixed);
  while ( (argID = ESMCI_ArgGetID(argPtr)) != ESMCI_ArgLastID ) {
    switch ( argID ) {
      case ESMCI_XclassArgAoptID: ESMCI_ArgGetI1(argPtr); break;
      case ESMCI_XclassArgBoptID: ESMCI_ArgGetI4(argPtr); break;
      case ESMCI_XclassArgCoptID: ESMCI_ArgGetR4(argPtr); break;
      case ESMCI_XclassArgDoptID: ESMCI_ArgGetR8(argPtr); break;
      case ESMCI_XclassArgEoptID: ESMCI_ArgGet(argPtr,ESMC_Fred); break;
      default:
        ESMC_LogDefault.MsgFoundError(ESMC_RC_OPTARG_BAD, "", &rc);
        return rc;
    } // end switch (argID)
  } // end while (argID)
  ESMCI_ArgEnd(argPtr);
\end{verbatim}

\noindent
The following example shows how an optional argument list check block can be
structured to handle an optional argument whose type is call dependent. In this
case the type of the \texttt{dValue} optional argument depends on the value of the
\texttt{ESMC\_TypeKind\_Flag} argument \texttt{tk} (which is the last fixed argument).
\begin{verbatim}
  // check the optional argument list:
  ESMCI_ArgStart(argPtr,tk);
  while ( (argID = ESMCI_ArgGetID(argPtr)) != ESMCI_ArgLastID ) {
    switch ( argID ) {
      case ESMCI_ConfigArgCountID: ESMCI_ArgGetInt(argPtr); break;
      case ESMCI_ConfigArgLabelID: ESMCI_ArgGetString(argPtr); break;
      case ESMCI_ConfigArgDvalueID:
        switch ( tk ) {
          case ESMC_TYPEKIND_I4: ESMCI_ArgGetI4(argPtr); break;
          case ESMC_TYPEKIND_I8: ESMCI_ArgGetI8(argPtr); break;
          case ESMC_TYPEKIND_R4: ESMCI_ArgGetR4(argPtr); break;
          case ESMC_TYPEKIND_R8: ESMCI_ArgGetR8(argPtr); break;
          case ESMC_TYPEKIND_LOGICAL: ESMCI_ArgGetInt(argPtr); break;
          case ESMC_TYPEKIND_CHARACTER: ESMCI_ArgGetString(argPtr); break;
        } // end switch (tk)
        break;
      default:
        ESMC_LogDefault.MsgFoundError(ESMC_RC_OPTARG_BAD, "", &rc);
        return rc;
    } // end switch (argID)
  } // end while (argID)
  ESMCI_ArgEnd(argPtr);
\end{verbatim}



