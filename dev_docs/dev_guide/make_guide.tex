% CVS $Id$

\subsection{Code: Makefile Conventions} 	 

The makefiles use GNU standard target names when possible.
The default rule is to remake the ESMF library.  Targets exist
to build the unit and system tests, the examples, the demos,
and to run them together or individually.  Targets also exist
to build the documentation and create pdf and html versions.
For more details on targets, refer to the README file in the top
level ESMF directory, and also the {\it ESMF User's Guide}.


\subsubsection{Code Building Rules} 
During software development it is advantageous to recompile a portion of
ESMF and not build the entire framework when testing localized software
modifications.  The makefiles are configured to compile only files in the
current directory and rebuild the shared libraries only when necessary.
Therefore when "gmake" is entered in any directory, only files in that
directory are compiled.  However, the entire ESMF framework may be built
from any directory by entering "gmake lib".

The unit tests, the system tests, the examples, and the demos all share
a common set of targets.  The following target list illustrates the
options for the examples.  The names {\tt unit\_tests}, {\tt system\_tests},
{\tt demos}, and {\tt all\_tests} can be substituted wherever {\tt examples}
occurs.
\begin{description} 
\item [examples] build and run all examples
\item [examples\_uni] build and run all examples single process
\item [build\_examples] build the examples
\item [run\_examples] run the examples and report the success/fail status
\item [run\_examples\_uni] run the examples single process
\item [check\_examples] report the success/fail status without reexecution
\item [clean\_examples] remove the example executables
\end{description} 

\subsubsection{Document Building Rules} 
The makefile rules for building documents are now located in:
\begin{verbatim} build/common.mk \end{verbatim} 
The rules exist
as pattern rules and are controlled by the variables set in a doc
directory's makefile.  The pattern rules will build .tex  from source code
and .pdf, .dvi and html files from .ctex files.

Variables that can be set in the individual doc/makefiles are called
"makefile variables".  These are: 
\begin{description} 
\item [DVIFILES]
\item [PDFFILES] 
\item [HTMLFILES] 
\item [TEXTFILES\_TO\_MAKE]
\end{description}

Document file dependencies are set with: 
\begin{description} 
\item [REFDOC\_DEP\_FILES]
\end{description}

The makefiles targets work from both local directory and from ESMF\_DIR:
\begin{description} 
\item [gmake alldoc] 
\item [gmake pdf] 
\item [gmake dvi] 
\item [gmake html] 
\end{description}


\subsubsection{Include Files} 
At the highest directory level, the "include"
directory will contain public include files intended to be used in
user-written code.  They will be broken up into separate files, with a
single "ESMF.h" include file which includes all the others.

At the 'src' level, parallel to Infrastructure, Superstructure, and doc,
is another include directory.  This is for private include files which are
ESMF-wide.  These might include system-wide architecture dependent
\#defines constants, etc.

Below the Superstructure \& Infrastructure level are the individual
component levels (TimeMgr, Field, Component, etc).  Under each of these
directories is an include directory for the component specific include
files.

