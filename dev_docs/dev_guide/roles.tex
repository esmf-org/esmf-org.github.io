%===============================================================================
% CVS $Id$
%===============================================================================

\section{Groups and Roles in ESMF Development}

For more detail on the groups below, including their Terms of
Reference, see the {\it ESMF Project Plan}\cite{bib:ESMFprojectplan}.
This document is available via the {\bf Management} link on 
the ESMF website navigation bar.

\subsection{Core Team}
ESMF software implementation is led by a distributed {\bf Core Team} which is based in the 
Technology Development Division (TDD) of the National Center for
Atmospheric Research (NCAR).  The Core Team relies on close interaction
with customers, and the work of many contributors.  Core Team
activities are open to all active ESMF developers.

All Core Team members are responsible for helping to 
develop effective project processes and for following
processes that are in place.

\subsubsection{Core Team Roles}
\label{core}
{\bf Developers} are the Core Team members that design, implement, document, and test
ESMF software.  They are expected to interact with customers throughout the entire
development process in order to understand customer requirements.  Developer priorities are set by the Change Review Board (CRB) but developers are expected to understand customer requirements and
needs and to manage their own time.

The {\bf Integrator} is the lead tester. He or she is responsible for running regression
tests, managing project computing accounts, preparing for releases,
generating project metrics, and overseeing request and bug tracking.

The {\bf Core Team Manager} is responsible for overseeing the overall development, and 
for coordinating the activities of multiple developers
so that project schedules and priorities are achieved.
Responsibilities include:

\begin{itemize}
\item Project administration and the assignment of  tasks
\item Acquisition of funding
\item Deciding on short-term priorities based on longer-term objectives
\item Monitoring conformance to processes and conventions
\item Representation of the Core Team to NCAR executive management, the CRB, and sponsors.
\end{itemize}

Code {\bf Advocates} are Core Team members who have been assigned to interface with owners of
a particular code e.g. CCSM, GEOS-5. Advocates are expected to contact the code owners and keep track of what and how they are doing. They should know the following things about their assigned code:

\begin{itemize}
\item contact info
\item models or components being ESMF-ized
\item what the customer is trying to do
\item what platforms they are working on
\item what pieces of ESMF they are using
\item their current status of ESMF-ization
\item what is holding them up if anything
\item open tickets (see section ~\ref{tracking_tools} for types) and their status
\end{itemize}

Code {\bf Advocates} should do the following for their code:
\begin{itemize}
\item Touch base periodically and find out how things are going
\item Coordinate the resolution of all tickets (see section ~\ref{tracking_tools} for types)
\item Be prepared to brief the status of all requests and issues
\item Update or ensure the update of related tickets (see Section ~\ref{sec:usr_support} for details on
the proper handling of tickets).
\end{itemize}

{\bf Advocates} are NOT responsible for:
\begin{itemize}
\item Representing the code to the CRB
\item Fielding support related phone calls.

All customers should send support requests to \htmladdnormallink{esmf\_support@ucar.edu}{mailto:esmf\_support@ucar.edu}.
They should not call their Advocate or other developers.  Dire emergencies are exempt from the no-phone rule. 
There are many good reason for doing this, including developer absences, group visibility and communication,
and enabling developers to set priorities.  If a customer really feels the need to talk with an
Advocate or developer, they need to arrange for a date and time through the support list.
\end {itemize}

{\bf Handlers} are developers assigned to fix a particular problem reported on one of the trackers (see 
~\ref{tracking_tools} for list). They are expected to be sufficiently familiar with the issue and the
code involved so as to not unnecessarily pester the customer with requests for information (e.g. the
information is contained in previous tickets from the same customer). See section ~\ref{sec:usr_support} for
a complete list of {\bf Handlers} activities in the support process.  

{\bf Support Lead} is a Core Team member who has been assigned the responsibility for monitoring customer relations and the quality of the customer support process. Specific duties include:
\begin{itemize}
\item Ensure IMAP folders are properly named and utilized
\item Assist Advocates in understanding the status of support requests related to their codes
\item Clear out mail list spam filters and clear all pending moderation requests
\item Conduct monthly support request review meetings
\item With the assistance of the Advocates, update the Core Team on the status of codes
\item Monitor the teams customer relations management (CRM) software and database
\end{itemize}

\subsection{Joint Specification Team (JST)}
The Core Team, along with contributors, customers, technical
managers, and other stakeholders in development, are collectively
referred to as the {\bf ESMF Joint Specification Team}, or {\bf JST}.  
JST membership is open to members of the science, computing, and
related communities.


\subsection{Change Review Board (CRB)}
\label{crb}
ESMF development priorities and schedules are set by a 
{\bf Change Review Board} (CRB) that consists of individuals
associated with major ESMF initiatives and applications.  
Members of the CRB are chosen by the ESMF Executive Management.

\subsection{Quality Assurance Responsibilities}

Collective public reviews and the intelligence and attentiveness of
project staff are the two primary mechanisms for software quality assurance.  

Requirements, design, code, and project documents such as plans
are subject to public review.  The purpose of the reviews is
to look for models of good documentation, as well as inconsistencies,
errors, inefficiencies, and areas for improvement.  Reviews also
increase coordination and awareness within the ESMF project.

The Integrator ensures that support requests
and bugs are reported, tracked, and resolved, and that automated
test and backup scripts are operating correctly.

The Core Team Manager works with team to ensure that documentation
is sufficient, tracks consistency between documentation and source
code before releases, and monitors conformance to coding and
documentation standards.  The Core Team Manager is also responsible
for ensuring the quality and accuracy of the ESMF source code
distribution overall, by leading the development,
implementation and documentation of development, test, and release
procedures.










